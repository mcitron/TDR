%_______________________________________________________________________________
%_______________________________________________________________________________
%_______________________________________________________________________________

\ifthenelse{\boolean{cms@external}}{\providecommand{\suppMaterial}{the supplemental material [URL will be inserted by publisher]}}{\providecommand{\suppMaterial}{Appendix~\ref{app:suppMat}}}

\newcommand{\jet}[1]{\ensuremath{\mathrm{j}_\text{#1}}\xspace}
\newcommand{\njet}{\ensuremath{n_{\text{jet}}}\xspace}
\newcommand{\nb}{\ensuremath{n_{\text{b}}}\xspace}
\newcommand{\scalht}{\ensuremath{H_{\text{T}}}\xspace}
\newcommand{\ctau}{\ensuremath{c\tau_{0}}\xspace}
\newcommand{\um}{\ensuremath{\,{\mu}\text{m}}\xspace}
\newcommand{\alphat}{\ensuremath{\alpha_{\mathrm{T}}}\xspace}
\newcommand{\bdphi}{\ensuremath{\Delta\phi^{*}_\text{min}}\xspace}

\title{Search for natural and split supersymmetry in proton-proton
  collisions at $\sqrt{s} = 13\TeV$ in final states with jets and
  missing transverse momentum
  \texorpdfstring{ \\[1cm] ---Supplemental Material---}{: Supplemental Material}}

\author[cern]{The CMS Collaboration}
\address[cern]{CERN} 

\date{\today}

\abstract{}

\hypersetup{ 
  pdfauthor={Robert Bainbridge, Eshwen Bhal, Shane Breeze, Oliver
    Buchmueller, Stefano Casasso, Matthew Citron, Adam Elwood, Henning
    Flaecher, Aran Garcia-Bellido, Ben Krikler, Christian Laner, Kin
    Ho Lo, Sarah Alam Malik, Bjoern Penning, Tai Sakuma, Dominic
    Smith, Alex Tapper},
  pdftitle={Search for natural and split supersymmetry in
    proton-proton collisions at 13 TeV in final states with jets and
    missing transverse momentum}, 
  pdfsubject={CMS, supersymmetry, AlphaT},
  pdfkeywords={Supersymmetry, split, natural, long-lived gluinos, dark matter}
}


%\cmsNoteHeader{SUS-16-038}
\maketitle

\clearpage
\begin{figure} \begin{center}
    \includegraphics[width=0.3\textwidth]{Supplementary/CMS-SUS-16-038_Figure-aux_001}
        \caption{
            Graphical representation of the production and decay of
            supersymmetric particles in the T1qqqq model.
        }
        \label{fig:simplified-models-feyn-T1qqqq}
\end{center} \end{figure}

\begin{figure} \begin{center}
    \includegraphics[width=0.3\textwidth]{Supplementary/CMS-SUS-16-038_Figure-aux_002}
        \caption{
            Graphical representation of the production and decay of
            supersymmetric particles in the T2qq model.
        }
        \label{fig:simplified-models-feyn-T2qq}
\end{center} \end{figure}

\begin{figure}[h!] \begin{center}
    \includegraphics[width=0.3\textwidth]{Supplementary/CMS-SUS-16-038_Figure-aux_003}
        \caption{
            Graphical representation of the production and decay of
            supersymmetric particles in the T1bbbb model.
        }
        \label{fig:simplified-models-feyn-T1bbbb}
\end{center} \end{figure}

\begin{figure}[h!] \begin{center}
    \includegraphics[width=0.3\textwidth]{Supplementary/CMS-SUS-16-038_Figure-aux_004}
        \caption{
            Graphical representation of the production and decay of
            supersymmetric particles in the T1tttt model.
        }
        \label{fig:simplified-models-feyn-T1tttt}
\end{center} \end{figure}

\begin{figure}[h!] \begin{center}
    \includegraphics[width=0.3\textwidth]{Supplementary/CMS-SUS-16-038_Figure-aux_005}
        \caption{
            Graphical representation of the production and decay of
            supersymmetric particles in the T2bb model.
        }
        \label{fig:simplified-models-feyn-T2bb}
\end{center} \end{figure}

\begin{figure}[h!] \begin{center}
    \includegraphics[width=0.3\textwidth]{Supplementary/CMS-SUS-16-038_Figure-aux_006}
        \caption{
            Graphical representation of the production and decay of
            supersymmetric particles in the T2tt model.
        }
        \label{fig:simplified-models-feyn-T2tt}
\end{center} \end{figure}

\begin{figure}[h!] \begin{center}
    \includegraphics[width=0.3\textwidth]{Supplementary/CMS-SUS-16-038_Figure-aux_007}
        \caption{
            Graphical representation of the production and decay of
            supersymmetric particles in the T2cc model.
        }
        \label{fig:simplified-models-feyn-T2cc}
\end{center} \end{figure}

\clearpage
\begin{figure}[p]
  \caption{ The \alphat distribution in data and simulation for events
    satisfying the baseline selection criteria plus the
    additional requirements $\njet \geq 2$, $\pt^{\jet{2}} > 100\GeV$,
    and $\scalht > 900\GeV$. The statistical uncertainties for the
    multijet and SM expectations are represented by the hatched areas
    (visible only for statistically limited bins). The final bin of
    this distribution contains the overflow events.
    \label{fig:alphaT} }
  \begin{center}
  \includegraphics[width=0.7\textwidth]{Supplementary/CMS-SUS-16-038_Figure-aux_008}
  \end{center}
\end{figure}


\begin{figure}[p]
    \caption{ The \bdphi distribution in data and simulation for
      events satisfying the baseline selection criteria plus
      the additional requirements $\njet \geq 2$, $\pt^{\jet{2}} >
      100\GeV$, and $\scalht > 900\GeV$. The statistical uncertainties
      for the multijet and SM expectations are represented by the
      hatched areas (visible only for statistically limited bins). The
      final bin of this distribution contains the overflow events. 
      \label{fig:bDPhi} }
  \begin{center}
  \includegraphics[width=0.7\textwidth]{Supplementary/CMS-SUS-16-038_Figure-aux_009}
  \end{center}
\end{figure}


\clearpage
\begin{figure}
  \centering
  \includegraphics[width=\textwidth]{Supplementary/CMS-SUS-16-038_Figure-aux_010}
  \caption{Covariance matrix for the SM background estimates
    determined from the CR-only fit using the simplified binning
	schema defined in the paper.
	An electronic version of this figure is available as CMS-SUS-16-038\_Figure-aux\_010.root.
	} %Table~\ref{tab:simplified}.}
  \label{fig:covariance_aux}
\end{figure} 
\clearpage

\begin{figure}[h!]
  \centering
  \includegraphics[width=0.95\linewidth]{Supplementary/CMS-SUS-16-038_Figure-aux_011} 
  \caption{Counts of signal events (solid markers) and SM expectations
    with associated uncertainties (statistical and systematic, black
    histograms and shaded bands) 
    %%
    %before fitting (i.e.\ simulation with scale factors applied)
    as determined from the CR-only fit
    %%
    %as a function of \nb, \scalht, and \mht for the event categories
    %$\njet = 4$ (upper), $=5$ (middle), and ${\geq}6$ (lower).
    for the simplified binning scheme.
    %%
    The centre panel shows the ratios of
    observed counts and the SM expectations, while the lower panel
    shows the significance of deviations observed in data with respect
    to the SM expectations expressed in terms of the total uncertainty
    in the SM expectations.
    An electronic version of this figure is available as CMS-SUS-16-038\_Figure-aux\_011.csv.
    }
  \label{fig:aggregated_results_cr-only}
\end{figure}

\begin{figure}[h!]
  \centering
  \includegraphics[width=0.95\linewidth]{Supplementary/CMS-SUS-16-038_Figure-aux_012} 
  \caption{Counts of signal events (solid markers) and SM expectations
    with associated uncertainties (statistical and systematic, black
    histograms and shaded bands) 
    %%
    %before fitting (i.e.\ simulation with scale factors applied)
    as determined by the full fit to signal and control regions.
    %%
    %as a function of \nb, \scalht, and \mht for the event categories
    %$\njet = 4$ (upper), $=5$ (middle), and ${\geq}6$ (lower).
    for the simplified binning scheme.
    %%
    The centre panel shows the ratios of
    observed counts and the SM expectations, while the lower panel
    shows the significance of deviations observed in data with respect
    to the SM expectations expressed in terms of the total uncertainty
    in the SM expectations.
    An electronic version of this figure is available as CMS-SUS-16-038\_Figure-aux\_012.csv.
    }
  \label{fig:aggregated_results_full-fit}
\end{figure}

\begin{figure}
    \begin{center}
            \includegraphics[width=0.98\textwidth]{Supplementary/CMS-SUS-16-038_Figure-aux_013-a}
            \includegraphics[width=0.98\textwidth]{Supplementary/CMS-SUS-16-038_Figure-aux_013-b}\\
            \includegraphics[width=0.98\textwidth]{Supplementary/CMS-SUS-16-038_Figure-aux_013-c}\\
  \caption{Counts of signal events (solid markers) and SM expectations
    with associated uncertainties (statistical and systematic, black
    histograms and shaded bands) 
    %%
    %before fitting (i.e.\ simulation with scale factors applied)
    %as determined from the CR-only fit
    as determined by the full fit to signal and control regions.
    %%
    as a function of \nb, \scalht, and \mht for the event categories
	    $\njet = 1$ and ${\geq}2a$ (a), $=2$ (b), and $=3$ (c).
    %$\njet = 4$ (upper), $=5$ (middle), and ${\geq}6$ (lower).
    %for the simplified binning scheme.
    %%
    The centre panel shows the ratios of
    observed counts and the SM expectations, while the lower panel
    shows the significance of deviations observed in data with respect
    to the SM expectations expressed in terms of the total uncertainty
    in the SM expectations.
    An electronic version of these figures is available as CMS-SUS-16-038\_Figure-aux\_013.csv.
    }
        \label{fig:T1qqqqLL_full-fit_123}
    \end{center}
\end{figure}

\begin{figure}
    \begin{center}
            \includegraphics[width=0.98\textwidth]{Supplementary/CMS-SUS-16-038_Figure-aux_014-a}
            \includegraphics[width=0.98\textwidth]{Supplementary/CMS-SUS-16-038_Figure-aux_014-b}\\
            \includegraphics[width=0.98\textwidth]{Supplementary/CMS-SUS-16-038_Figure-aux_014-c}\\
  \caption{Counts of signal events (solid markers) and SM expectations
    with associated uncertainties (statistical and systematic, black
    histograms and shaded bands) 
    %%
    %before fitting (i.e.\ simulation with scale factors applied)
    %as determined from the CR-only fit
    as determined by the full fit to signal and control regions.
    %%
    as a function of \nb, \scalht, and \mht for the event categories
    %$\njet = 1$ and ${\geq}2a$ (upper), $=2$ (middle), and $=3$
    $\njet = 4$ (a), $=5$ (b), and ${\geq}6$ (c).
    %for the simplified binning scheme.
    %%
    The centre panel shows the ratios of
    observed counts and the SM expectations, while the lower panel
    shows the significance of deviations observed in data with respect
    to the SM expectations expressed in terms of the total uncertainty
    in the SM expectations.
    An electronic version of these figures is available as CMS-SUS-16-038\_Figure-aux\_013.csv.
    }
        \label{fig:T1qqqqLL_full-fit_456}
    \end{center}
\end{figure}

\clearpage
\begin{table}
  \topcaption{A list of benchmark simplified models organized
    according to production and decay modes (family), 
    and the expected and observed upper limits
    on the production cross section $\sigma_\text{UL}$ relative to the
    theoretical value $\sigma_\text{th}$ calculated at NLO+NLL
    accuracy. 
    See the paper for a discussion of the uncertainties and signal acceptance times efficiency
    }
  \label{tab:benchmarks_aux}
  \centering
    \begin{tabular}{ lrrcc }
      \hline
      Family
      & $(m_{\text{SUSY}}, m_{\mathrm{LSP}})$
      & \multicolumn{2}{c}{$\mu$ (95\% CL)}                                                                            \\ [0.3ex]
      ($c\tau$)
      & [\GeVns{}]
      & Exp.
      & Obs.                                                                                                           \\ [0.3ex]
      \hline
      \multirow{2}{*}{T2bb} & (1000, 100) & 0.89 & 0.56 \\ & (550, 450)  & 1.46 & 1.05 \\ [0.5ex]
      \multirow{3}{*}{T2tt} & (1000, 50)  & 0.96 & 0.88 \\ & (450, 200)  & 1.25 & 1.07 \\ & (250, 150)  & 3.14 & 2.86 \\ [0.5ex]
      \multirow{1}{*}{T2cc} & (500, 480)  & 1.52 & 2.43 \\ [0.5ex]
      \multirow{2}{*}{T2qq\_8fold} & (1250, 100) & 0.91 & 0.72 \\ & (700, 600)  & 1.81 & 2.48 \\ [0.5ex]
      \multirow{2}{*}{T2qq\_1fold} & (700, 100)  & 1.07 & 2.20 \\ & (400, 300)  & 1.60 & 1.60 \\ [0.5ex]
      \multirow{2}{*}{T1bbbb} & (1900, 100) & 1.16 & 1.15 \\ & (1300, 1100) & 0.78 & 1.03 \\ [0.5ex]
      \multirow{2}{*}{T1tttt} & (1700, 100) & 0.67 & 0.70 \\ & (950, 600)  & 3.39 & 2.95 \\ [0.5ex]
	    T1qqqqLL & (1800, 200) & 1.36 & 2.00 \\ ($1\um$) & (1000, 900) & 1.46 & 2.40 \\ [0.5ex]
      T1qqqqLL & (1800, 200) & 0.52 & 0.60 \\ ($1\unit{mm}$) & (1000, 900) & 0.41 & 0.47 \\ [0.5ex]
      T1qqqqLL & (1000, 200) & 1.23 & 1.40 \\ ($1\unit{m}$) & (1000, 900) & 1.22 & 1.90 \\
      \hline
    \end{tabular}
\end{table}

\clearpage
\begin{figure}
    \begin{center}
            \includegraphics[width=0.50\textwidth]{Supplementary/CMS-SUS-16-038_Figure-aux_015-a}
            \includegraphics[width=0.50\textwidth]{Supplementary/CMS-SUS-16-038_Figure-aux_015-b}
        \caption{ (a) Observed upper limit in cross section at 95\% CL (indicated
        by the colour scale) as a function of 
        the $\PSg$ and \PSGczDo %%%
        masses for the 
        T1qqqq %%%
        simplified  model.  The  black  solid thick  (thin)  line indicates  the
        observed  excluded  region  assuming   the  nominal  (${\pm}1$  standard
        deviation in theoretical uncertainty)  production cross section. The red
        dashed  thick  (thin)  line  indicates  the  median  (${\pm}1$  standard
        deviation in experimental uncertainty) expected excluded region.
    An electronic version of this figure is available as CMS-SUS-16-038\_Figure-aux\_015-a.root.
        (b) The signal acceptance times efficiency as a function of 
        the $\PSg$ and \PSGczDo %%%
        masses following the application of the event selection criteria for the signal region.
    An electronic version of this figure is available as CMS-SUS-16-038\_Figure-aux\_015-b.root.
        }
        \label{fig:T1qqqq}
    \end{center}
\end{figure}

\begin{figure}
    \begin{center}
            \includegraphics[width=0.50\textwidth]{Supplementary/CMS-SUS-16-038_Figure-aux_016-a}
            \includegraphics[width=0.50\textwidth]{Supplementary/CMS-SUS-16-038_Figure-aux_016-b}
        \caption{ (a) Observed upper limit in cross section at 95\% CL (indicated
        by the colour scale) as a function of 
        the $\PSg$ and \PSGczDo %%%
        masses for the 
        T1bbbb %%%
        simplified  model.  The  black  solid thick  (thin)  line indicates  the
        observed  excluded  region  assuming   the  nominal  (${\pm}1$  standard
        deviation in theoretical uncertainty)  production cross section. The red
        dashed  thick  (thin)  line  indicates  the  median  (${\pm}1$  standard
        deviation in experimental uncertainty) expected excluded region.
    An electronic version of this figure is available as CMS-SUS-16-038\_Figure-aux\_016-a.root.
        (b) The signal acceptance times efficiency as a function of 
        the $\PSg$ and \PSGczDo %%%
        masses following the application of the event selection criteria for the signal region.
    An electronic version of this figure is available as CMS-SUS-16-038\_Figure-aux\_016-b.root.
        }
        \label{fig:T1bbbb}
    \end{center}
\end{figure}

\begin{figure}
    \begin{center}
            \includegraphics[width=0.50\textwidth]{Supplementary/CMS-SUS-16-038_Figure-aux_017-a}
            \includegraphics[width=0.50\textwidth]{Supplementary/CMS-SUS-16-038_Figure-aux_017-b}
        \caption{ (a) Observed upper limit in cross section at 95\% CL (indicated
        by the colour scale) as a function of 
        the $\PSg$ and \PSGczDo %%%
        masses for the 
        T1tttt %%%
        simplified  model.  The  black  solid thick  (thin)  line indicates  the
        observed  excluded  region  assuming   the  nominal  (${\pm}1$  standard
        deviation in theoretical uncertainty)  production cross section. The red
        dashed  thick  (thin)  line  indicates  the  median  (${\pm}1$  standard
        deviation in experimental uncertainty) expected excluded region.
    An electronic version of this figure is available as CMS-SUS-16-038\_Figure-aux\_017-a.root.
        (b) The signal acceptance times efficiency as a function of 
        the $\PSg$ and \PSGczDo %%%
        masses following the application of the event selection criteria for the signal region.
    An electronic version of this figure is available as CMS-SUS-16-038\_Figure-aux\_017-b.root.
        }
        \label{fig:T1tttt}
    \end{center}
\end{figure}

\begin{figure}
    \begin{center}
            \includegraphics[width=0.50\textwidth]{Supplementary/CMS-SUS-16-038_Figure-aux_018-a}
            \includegraphics[width=0.50\textwidth]{Supplementary/CMS-SUS-16-038_Figure-aux_018-b}
        \caption{ (a) Observed upper limit in cross section at 95\% CL (indicated
        by the colour scale) as a function of 
        the $\PSQb$ and \PSGczDo %%%
        masses for the 
        T2bb %%%
        simplified  model.  The  black  solid thick  (thin)  line indicates  the
        observed  excluded  region  assuming   the  nominal  (${\pm}1$  standard
        deviation in theoretical uncertainty)  production cross section. The red
        dashed  thick  (thin)  line  indicates  the  median  (${\pm}1$  standard
        deviation in experimental uncertainty) expected excluded region.
    An electronic version of this figure is available as CMS-SUS-16-038\_Figure-aux\_018-a.root.
        (b) The signal acceptance times efficiency as a function of 
        the $\PSQb$ and \PSGczDo %%%
        masses following the application of the event selection criteria for the signal region.
    An electronic version of this figure is available as CMS-SUS-16-038\_Figure-aux\_018-b.root.
        }
        \label{fig:T2bb}
    \end{center}
\end{figure}

\begin{figure}
    \begin{center}
            \includegraphics[width=0.50\textwidth]{Supplementary/CMS-SUS-16-038_Figure-aux_019-a}
            \includegraphics[width=0.50\textwidth]{Supplementary/CMS-SUS-16-038_Figure-aux_019-b}
        \caption{ (a) Observed upper limit in cross section at 95\% CL (indicated
        by the colour scale) as a function of 
        the $\PSQt$ and \PSGczDo %%%
        masses for the 
        T2tt %%%
        simplified  model.  The  black  solid thick  (thin)  line indicates  the
        observed  excluded  region  assuming   the  nominal  (${\pm}1$  standard
        deviation in theoretical uncertainty)  production cross section. The red
        dashed  thick  (thin)  line  indicates  the  median  (${\pm}1$  standard
        deviation in experimental uncertainty) expected excluded region.
    An electronic version of this figure is available as CMS-SUS-16-038\_Figure-aux\_019-a.root.
        (b) The signal acceptance times efficiency as a function of 
        the $\PSQt$ and \PSGczDo %%%
        masses following the application of the event selection criteria for the signal region.
    An electronic version of this figure is available as CMS-SUS-16-038\_Figure-aux\_019-b.root.
        }
        \label{fig:T2tt}
    \end{center}
\end{figure}

\begin{figure}
    \begin{center}
            \includegraphics[width=0.50\textwidth]{Supplementary/CMS-SUS-16-038_Figure-aux_020-a}
            \includegraphics[width=0.50\textwidth]{Supplementary/CMS-SUS-16-038_Figure-aux_020-b}
        \caption{ (a) Observed upper limit in cross section at 95\% CL (indicated
        by the colour scale) as a function of 
        the $\PSQt$ and \PSGczDo %%%
        masses for the 
        T2cc %%%
        simplified  model.  The  black  solid thick  (thin)  line indicates  the
        observed  excluded  region  assuming   the  nominal  (${\pm}1$  standard
        deviation in theoretical uncertainty)  production cross section. The red
        dashed  thick  (thin)  line  indicates  the  median  (${\pm}1$  standard
        deviation in experimental uncertainty) expected excluded region.
    An electronic version of this figure is available as CMS-SUS-16-038\_Figure-aux\_020-a.root.
        (b) The signal acceptance times efficiency as a function of 
        the $\PSQt$ and \PSGczDo %%%
        masses following the application of the event selection criteria for the signal region.
    An electronic version of this figure is available as CMS-SUS-16-038\_Figure-aux\_020-b.root.
        }
        \label{fig:T2cc}
    \end{center}
\end{figure}

\begin{figure}
    \begin{center}
            \includegraphics[width=0.50\textwidth]{Supplementary/CMS-SUS-16-038_Figure-aux_021-a}
            \includegraphics[width=0.50\textwidth]{Supplementary/CMS-SUS-16-038_Figure-aux_021-b}
        \caption{ (a) Observed upper limit in cross section at 95\% CL (indicated
        by the colour scale) as a function of 
        the $\PSQ$ and \PSGczDo %%%
        masses for the 
        T2qq %%%
        simplified  model.  The  black  solid thick  (thin)  line indicates  the
        observed  excluded  region  assuming   the  nominal  (${\pm}1$  standard
        deviation in theoretical uncertainty)  production cross section. The red
        dashed  thick  (thin)  line  indicates  the  median  (${\pm}1$  standard
        deviation in experimental uncertainty) expected excluded region.
    An electronic version of this figure is available as CMS-SUS-16-038\_Figure-aux\_021-a.root.
        (b) The signal acceptance times efficiency as a function of 
        the $\PSQ$ and \PSGczDo %%%
        masses following the application of the event selection criteria for the signal region.
    An electronic version of this figure is available as CMS-SUS-16-038\_Figure-aux\_021-b.root.
        }
        \label{fig:T2qq}
    \end{center}
\end{figure}

\clearpage
\begin{figure}
    \begin{center}
            \includegraphics[width=0.50\textwidth]{Supplementary/CMS-SUS-16-038_Figure-aux_022-a}
	    \includegraphics[width=0.50\textwidth]{Supplementary/CMS-SUS-16-038_Figure-aux_022-b}
        \caption{ (a) Observed upper limit in cross section at 95\% CL (indicated
        by the colour scale) as a function of 
        the $\PSg$ and \PSGczDo %%%
        masses for the 
        T1qqqq split-SUSY model with meta-stable gluinos. 
         The  black  solid thick  (thin)  line indicates  the
        observed  excluded  region  assuming   the  nominal  (${\pm}1$  standard
        deviation in theoretical uncertainty)  production cross section. The red
        dashed  thick  (thin)  line  indicates  the  median  (${\pm}1$  standard
        deviation in experimental uncertainty) expected excluded region.
    An electronic version of this figure is available as CMS-SUS-16-038\_Figure-aux\_022-a.root.
        (b) The signal acceptance times efficiency as a function of 
        the $\PSg$ and \PSGczDo %%%
        masses following the application of the event selection criteria for the signal region.
    An electronic version of this figure is available as CMS-SUS-16-038\_Figure-aux\_022-b.root.
        }
        \label{fig:T1qqqqLL}
    \end{center}
\end{figure}

\clearpage
\begin{figure}
    \begin{center}
            \includegraphics[width=0.30\textwidth]{Supplementary/CMS-SUS-16-038_Figure-aux_023-a}
            \includegraphics[width=0.30\textwidth]{Supplementary/CMS-SUS-16-038_Figure-aux_023-b}
            \includegraphics[width=0.30\textwidth]{Supplementary/CMS-SUS-16-038_Figure-aux_023-c}
            \includegraphics[width=0.30\textwidth]{Supplementary/CMS-SUS-16-038_Figure-aux_023-d}
            \includegraphics[width=0.30\textwidth]{Supplementary/CMS-SUS-16-038_Figure-aux_023-e}
            \includegraphics[width=0.30\textwidth]{Supplementary/CMS-SUS-16-038_Figure-aux_023-f}
            \includegraphics[width=0.30\textwidth]{Supplementary/CMS-SUS-16-038_Figure-aux_023-g}
            \includegraphics[width=0.30\textwidth]{Supplementary/CMS-SUS-16-038_Figure-aux_023-h}
            \includegraphics[width=0.30\textwidth]{Supplementary/CMS-SUS-16-038_Figure-aux_023-i}
  \caption{ The signal acceptance times efficiency as a function of 
	    the $\PSg$ and \PSGczDo
    masses for simplified models that assume the production of pairs
    of long-lived gluinos that each decay via highly virtual
    light-flavour squarks to the neutralino and SM particles
    (T1qqqqLL). 
	    Each subfigure represents a different gluino lifetime:
    $\ctau = 1$ (a), 10 (b), and $100\um$ (c);
	     1 (d), 10 (e), and $100\unit{mm}$ (f);
	     and 1 (g), 10 (h), and $100\unit{m}$ (j).
    Electronic versions of these figures are available as
	    CMS-SUS-16-038\_Figure-aux\_023-a.root,
	    CMS-SUS-16-038\_Figure-aux\_023-b.root, etc.}
        \label{fig:T1qqqqLL_eff}
    \end{center}
\end{figure}


\clearpage
\begin{figure}[!t]
  \centering
  \includegraphics[width=0.6\textwidth]{Supplementary/CMS-SUS-16-038_Figure-aux_024}\\
  \caption{Observed and expected mass exclusions at 95\% CL
    (indicated, respectively, by solid and dashed contours) for
    simplified models with an intermediate squark.
    Five model families involve the direct pair
    production of squarks. The first scenario considers the pair
    production and decay of bottom squarks (T2bb). Two
    scenarios involve the production and decay of top squark pairs
    (T2tt and T2cc). The grey shaded region denotes
    T2tt models that are not considered for
    interpretation. Two further scenarios involve, respectively, the 
    production and decay of light-flavour squarks, with different
    assumptions on the mass degeneracy of the squarks as described in
    the text (T2qq\_8fold and T2qq\_1fold).}
  \label{fig:limits-sms_aux_squarks} 
\end{figure}

\clearpage
\begin{figure}[!t]
  \centering
  \includegraphics[width=0.6\textwidth]{Supplementary/CMS-SUS-16-038_Figure-aux_025}\\
  \caption{Observed and expected mass exclusions at 95\% CL
    (indicated, respectively, by solid and dashed contours) for
    simplified models that assume the production of pairs of
    long-lived gluinos that each decay via highly virtual
    light-flavour squarks to the neutralino and SM particles
    (T1qqqqLL). The mass exclusion contours are shown for each 
    value of the gluino proper decay length \ctau considered by this
    search. }
  \label{fig:limits-sms_aux_long_lived} 
\end{figure}

\clearpage
\begin{figure}[!t]
  \centering
  \includegraphics[width=0.6\textwidth]{Supplementary/CMS-SUS-16-038_Figure-aux_026}\\
  \caption{Observed and expected gluino mass exclusions at 95\% CL
    (indicated, respectively, by solid and dashed contours) for
    simplified models that assume the production of pairs of
    long-lived gluinos that each decay via highly virtual
    light-flavour squarks to the neutralino and SM particles
    (T1qqqqLL). The gluino mass exclusions are shown for two different
    assumptions on the neutralino mass and as a function of
    the gluino proper decay length \ctau. The prompt and stable values
    refer to the mass exclusions obtained with the T1qqqq and T1qqqqLL
    metastable ($\ctau = 10^{18}\unit{mm}$) scenarios, respectively.}
  \label{fig:limits_vs_ctau} 
\end{figure}

\clearpage
\begin{figure}[p]
    \begin{center}
        \includegraphics[width=1.00\textwidth]{Supplementary/CMS-SUS-16-038_Figure-aux_027.pdf}
  \caption{Counts of events in the signal region from data (solid markers), SM expectations
    with associated uncertainties (statistical and systematic, black
    histograms and shaded bands) as determined from the CR-only fit,
    and the predicted signal shape and uncertainty assuming a production cross
    section calculated at NLO+NLL accuracy for the
        T2bb benchmark model with ($m_{\PSQb}$, $m_{\PSGczDo}$) = (550, 450) GeV
    (magenta histogram and shaded band),
    as a function of \njet, \nb, \scalht, and \mht using the simplified binning schema.
    The centre panel of each subfigure shows the ratios of
    observed counts and the SM expectations, while the lower panel
    shows the significance of deviations observed in data with respect
    to the SM expectations expressed in terms of the total uncertainty
    in the SM expectations.  
	An electronic version of the signal histogram is available as CMS-SUS-16-038\_Figure-aux\_027.csv.
        }
        \label{fig:T2bb_550_450_MR_sig}
    \end{center}
\end{figure}

\begin{figure}[p]
    \begin{center}
        \includegraphics[width=1.00\textwidth]{Supplementary/CMS-SUS-16-038_Figure-aux_028.pdf}
  \caption{Counts of events in the signal region from data (solid markers), SM expectations
    with associated uncertainties (statistical and systematic, black
    histograms and shaded bands) as determined from the CR-only fit,
    and the predicted signal shape and uncertainty assuming a production cross
    section calculated at NLO+NLL accuracy for the
        T2cc benchmark model with ($m_{\PSQt}$, $m_{\PSGczDo}$) = (500, 480) GeV
    (magenta histogram and shaded band),
    as a function of \njet, \nb, \scalht, and \mht using the simplified binning schema.
    The centre panel of each subfigure shows the ratios of
    observed counts and the SM expectations, while the lower panel
    shows the significance of deviations observed in data with respect
    to the SM expectations expressed in terms of the total uncertainty
    in the SM expectations.  
	An electronic version of the signal histogram is available as CMS-SUS-16-038\_Figure-aux\_028.csv.
        }
        \label{fig:T2cc_500_480_MR_sig}
    \end{center}
\end{figure}

\begin{figure}[p]
    \begin{center}
        \includegraphics[width=1.00\textwidth]{Supplementary/CMS-SUS-16-038_Figure-aux_029.pdf}
  \caption{Counts of events in the signal region from data (solid markers), SM expectations
    with associated uncertainties (statistical and systematic, black
    histograms and shaded bands) as determined from the CR-only fit,
    and the predicted signal shape and uncertainty assuming a production cross
    section calculated at NLO+NLL accuracy for the
        T1bbbb benchmark model with ($m_{\PSg}$, $m_{\PSGczDo}$) = (1900, 100) GeV
    (magenta histogram and shaded band),
    as a function of \njet, \nb, \scalht, and \mht using the simplified binning schema.
    The centre panel of each subfigure shows the ratios of
    observed counts and the SM expectations, while the lower panel
    shows the significance of deviations observed in data with respect
    to the SM expectations expressed in terms of the total uncertainty
    in the SM expectations.  
	An electronic version of the signal histogram is available as CMS-SUS-16-038\_Figure-aux\_029.csv.
        }
        \label{fig:T1bbbb_1900_100_MR_sig}
    \end{center}
\end{figure}

\begin{figure}[p]
    \begin{center}
        \includegraphics[width=1.00\textwidth]{Supplementary/CMS-SUS-16-038_Figure-aux_030.pdf}
  \caption{Counts of events in the signal region from data (solid markers), SM expectations
    with associated uncertainties (statistical and systematic, black
    histograms and shaded bands) as determined from the CR-only fit,
    and the predicted signal shape and uncertainty assuming a production cross
    section calculated at NLO+NLL accuracy for the
        T1qqqqLL benchmark model with $\ctau=1$~mm, ($m_{\PSg}$, $m_{\PSGczDo}$) = (1800, 200) GeV
    (magenta histogram and shaded band),
    as a function of \njet, \nb, \scalht, and \mht using the simplified binning schema.
    The centre panel of each subfigure shows the ratios of
    observed counts and the SM expectations, while the lower panel
    shows the significance of deviations observed in data with respect
    to the SM expectations expressed in terms of the total uncertainty
    in the SM expectations.  
	An electronic version of the signal histogram is available as CMS-SUS-16-038\_Figure-aux\_030.csv.
        }
        \label{fig:T1qqqqLL_1_1800_200_MR_sig}
    \end{center}
\end{figure}

\clearpage
\begin{table}[p!]
    \begin{center}
        \begin{tabular}{lrrrrr}
      \hline
                                 \multirow{5}{*}{Event Selection} &                          &\multicolumn{4}{c}{Benchmark Model} \\\cline{3-6}
                                                                  &                          &  T2bb &  T2cc & T1bbbb & T1qqqqLL  \\
                                                                  & $m_{\textrm{SUSY}}$ (GeV)&   550 &   500 &   1900 &     1800  \\
                                                                  & $m_{\textrm{LSP}}$ (GeV) &   450 &   480 &   100  &     200   \\
                                                                  & $\ctau$ (mm)             &    -- &    -- &    --  &     1     \\
      \hline
\multicolumn{2}{l}{Before selection}                                                         & 100.0 & 100.0 &  100.0 &     100.0 \\
%\multicolumn{2}{l}{Before selection}                                                        &  99.6 & 101.4 &  104.6 &     102.8 \\
%\multicolumn{2}{l}{Event veto for single isolated tracks}                                   &  95.6 &  97.7 &  101.0 &      87.8 \\
%\multicolumn{2}{l}{Event veto for muons and electrons}                                      &  94.9 &  97.5 &  100.3 &      87.2 \\
%\multicolumn{2}{l}{Event veto for photons}                                                  &  94.6 &  97.2 &   99.4 &      86.0 \\
\multicolumn{2}{l}{Single isolated track, muon, electron, \& photon vetos}                   &  94.6 &  97.2 &   99.4 &      86.0 \\
\multicolumn{2}{l}{Event veto for jets failing ID}                                           &  94.3 &  96.8 &   98.7 &      85.7 \\
\multicolumn{2}{l}{$p_{\mathrm{T}}^\mathrm{j_1} > 100$ GeV}                                  &  62.9 &  84.3 &   98.7 &      85.7 \\
\multicolumn{2}{l}{$0.1 <$ $f^{\mathrm{j_1}}_{\mathrm{h}^\pm} < 0.95$}                       &  59.8 &  77.4 &   93.9 &      82.1 \\
\multicolumn{2}{l}{$H_{\mathrm{T}} > 200\,\mathrm{GeV}$}                                     &  49.5 &  64.5 &   93.9 &      82.1 \\
\multicolumn{2}{l}{$H_{\mathrm{T}}^{\mathrm{miss}} > 200\,\mathrm{GeV}$}                     &  18.8 &  48.3 &   88.5 &      77.4 \\
\multicolumn{2}{l}{Event veto for forward jets ($|\eta| > 2.4$)}                             &  13.6 &  35.8 &   69.9 &      63.7 \\
\multicolumn{2}{l}{$H_{\mathrm{T}}^{\mathrm{miss}} / E_{\mathrm{T}}^{\mathrm{miss}} < 1.25$} &  12.9 &  34.1 &   69.3 &      60.3 \\
\multicolumn{2}{l}{\njet- and $H_{\mathrm{T}}$-dependent $\alpha_{\mathrm{T}}$ thresholds}   &   8.3 &  24.9 &   69.2 &      60.1 \\
\multicolumn{2}{l}{$\Delta\phi^{*}_{\mathrm{min}} > 0.5$}                                    &   5.7 &  20.5 &   25.1 &      22.9 \\
      \hline
        \end{tabular}
        \caption{
            A summary of the cumulative signal acceptance times efficiency,
            $\mathcal{A}\epsilon$ [\%], for various benchmark models found in
            Table 4 of the paper, following the successive application of the
            event selection criteria used to define the signal region.  See
            discussion and Table~1 in the paper for detailed descriptions of
            the selection. }
        \label{tab:cut_flow}
    \end{center}
\end{table}
