\RCS$Revision: $
\RCS$HeadURL: $
\RCS$Id: $

% Sentences to reference Supplemental Material
% Update aggregated result tables
% Table of limits from aggregated result

%_______________________________________________________________________________
%_______________________________________________________________________________
%_______________________________________________________________________________

% Typesetting
\newcommand\T{\rule{0pt}{2.6ex}}
\newcommand\B{\rule[-1.2ex]{0pt}{0pt}}
\newcommand{\ph}[1]{\phantom{#1}}

% Symbols
\newcommand{\tf}{\ensuremath{\mathcal{R}}\xspace}
\newcommand{\cls}{\ensuremath{\text{CL}_\text{s}}\xspace}
\newcommand{\dm}{\ensuremath{{\Delta}m}\xspace}
\newcommand{\higgsmass}{\ensuremath{m_{\text{H}}}\xspace}
\newcommand{\jet}[1]{\ensuremath{\text{j}_\text{#1}}\xspace}
%\newcommand{\PASQ}{\ensuremath{\overline{\widetilde{\cmsSymbolFace{q}}}}\xspace} 
\newcommand{\ate}{\ensuremath{\mathcal{A}\varepsilon}\xspace}
\newcommand{\ctau}{\ensuremath{c\tau_{0}}\xspace}

% Variables
\newcommand{\Et}{\ensuremath{{E_{\text T}}}\xspace}
\newcommand{\njet}{\ensuremath{n_{\text{jet}}}\xspace}
\newcommand{\nb}{\ensuremath{n_{\text{b}}}\xspace}
\newcommand{\scalht}{\ensuremath{H_{\text{T}}}\xspace}
\newcommand{\scalst}{\ensuremath{\mathcal{E}_\text{T}}\xspace}
\newcommand{\mht}{\ensuremath{H_{\text{T}}^{\text{miss}}}\xspace}
%\newcommand{\ptmiss}{\ensuremath{p_{\text{T}}^{\kern1pt\text{miss}}}\xspace}
\newcommand{\dst}{\ensuremath{\Delta\scalst}\xspace}
\newcommand{\alphat}{\ensuremath{\alpha_{\text{T}}}\xspace}
\newcommand{\bdphi}{\ensuremath{\Delta\phi^{*}_\text{min}}\xspace}
\newcommand{\bdphimod}{\ensuremath{\Delta\phi^{*_{\, 25}}_\text{min}}\xspace}
\newcommand{\minchi}{\ensuremath{\chi_\text{min}}\xspace}
\newcommand{\mhtmet}{\ensuremath{\mht / \ptmiss}\xspace}

% Selections
\newcommand{\jets}{\ensuremath{\text{jets}}}
\newcommand{\mj}{\ensuremath{\mu + \jets}\xspace}
\newcommand{\mmj}{\ensuremath{\mu\mu + \jets}\xspace}
\newcommand{\mmjpm}{\ensuremath{\mu^\pm\mu^\mp + \jets}\xspace}
\newcommand{\gj}{\ensuremath{\gamma + \jets}\xspace}

% Processes
\providecommand{\PSl}{\ensuremath{\widetilde{\ell}}\xspace}
\newcommand{\zmumu}{\ensuremath{\cPZ \to \mu\mu}\xspace}
\newcommand{\znunu}{\ensuremath{\cPZ \to \cPgn\cPagn}\xspace}
\newcommand{\zj}{\ensuremath{\cPZ + \jets}\xspace}
\newcommand{\zmumuj}{\ensuremath{\cPZ (\to \mu\mu) + \jets}\xspace}
\newcommand{\znunuj}{\ensuremath{\cPZ (\to \cPgn\cPagn) + \jets}\xspace}
\newcommand{\wj}{\ensuremath{\PW + \jets}\xspace}
\newcommand{\wmj}{\ensuremath{\PW (\to \mu\nu) + \text{jets}}\xspace}
\newcommand{\wlj}{\ensuremath{\PW (\to \ell\nu) + \text{jets}}\xspace}
\newcommand{\ttw}{\ensuremath{\ttbar\PW}\xspace}
\newcommand{\ttz}{\ensuremath{\ttbar\cPZ}\xspace}
\newcommand{\lost}{\ensuremath{\ell_\text{lost}}\xspace}
\newcommand{\ra}{\ensuremath{\rightarrow}}

\newlength\cmsFigWidth
\newlength\cmsFigWidthTwo
\ifthenelse{\boolean{cms@external}}{\setlength\cmsFigWidth{0.49\textwidth}}{\setlength\cmsFigWidth{0.7\textwidth}}
\ifthenelse{\boolean{cms@external}}{\setlength\cmsFigWidthTwo{0.95\textwidth}}{\setlength\cmsFigWidthTwo{0.95\textwidth}}
\ifthenelse{\boolean{cms@external}}{\providecommand{\cmsLeft}{top\xspace}}{\providecommand{\cmsLeft}{left\xspace}}
\ifthenelse{\boolean{cms@external}}{\providecommand{\cmsRight}{bottom\xspace}}{\providecommand{\cmsRight}{right\xspace}}
\ifthenelse{\boolean{cms@external}}{\providecommand{\cmsTable}[1]{#1}}{\providecommand{\cmsTable}[1]{\resizebox{\textwidth}{!}{#1}}}
\ifthenelse{\boolean{cms@external}}{\providecommand{\suppMaterial}{the supplemental material [URL will be inserted by publisher]}}{\providecommand{\suppMaterial}{Appendix~\ref{app:suppMat}}}

\newlength\cmsTableLabelSkip
\setlength\cmsTableLabelSkip{-1.2ex}

%_______________________________________________________________________________
%_______________________________________________________________________________
%_______________________________________________________________________________

\cmsNoteHeader{SUS-16-038}

\title{Search for natural and split supersymmetric scenarios in
  proton-proton collisions at $\sqrt{s} = 13\TeV$ in all-jet final states}
%  using the $\alpha_\text{T}$ variable}

\author[cern]{The CMS Collaboration}
\address[cern]{CERN} 

\date{\today}

\abstract{A search for supersymmetry is performed in final states
  comprising one or more jets and missing transverse momentum from proton-proton
  collisions at a centre-of-mass energy of $13~\text{TeV}$. The
  data were recorded with the CMS detector at the CERN LHC in 2016 and
  correspond to an integrated luminosity of $35.9~\text{fb}^{-1}$.
  %
  %Several kinematical variables are used to suppress the multijet
  %background to a subdominant level with respect to all other standard
  %model backgrounds. Further Several kinematical variables are used to
  %provide signal-to-background discrimination. 
  The number of candidate signal events is found to agree with the
  expected event counts from standard model processes.
  %
  The result is interpreted with simplified models of supersymmetry
  that assume the production of gluino or squark pairs and their
  prompt decay to quarks and the neutralino.
  % 
  %Gluinos and their daughter neutralinos are probed up to masses of X
  %and Y TeV, respectively. In the case of direct production of
  %light-flavour, bottom, or top squark pairs, masses up to X, Y, and
  %Z\GeV are probed.
  %
  The masses of bottom, top, and mass-degenerate light-flavour squarks
  are probed up to 1050, 1000, and 1325\GeV, respectively. The gluino
  mass is probed up to 1900, 1650, and 1650\GeV when the gluino decays
  via virtual states of the aforementioned squarks. The strongest mass
  bounds on the daughter neutralinos from gluino and squark decays are
  1150 and 575\GeV, respectively.
  %
  The search also provides sensitivity to simplified models inspired
  by split supersymmetry that involve the production and decay of
  long-lived gluinos. 
  % that form bound colour-singlet states known as R-hadrons. 
  %
  Proper decay lengths from 0.001 to 100\,000\unit{mm} are considered,
  as well as a metastable gluino scenario. Masses up to 1750\GeV and
  900\GeV are probed for a proper decay length of 1\unit{mm} and the
  metastable state, respectively.
  %
  The search provides coverage that is complementary to existing
  techniques at the LHC.  }

\hypersetup{ 
  pdfauthor={Robert Bainbridge, Eshwen Bhal, Shane Breeze, Oliver
    Buchmueller, Stefano Casasso, Matthew Citron, Adam Elwood, Henning
    Flaecher, Aran Garcia-Bellido, Ben Krikler, Christian Laner, Kin
    Ho Lo, Sarah Alam Malik, Bjoern Penning, Tai Sakuma, Dominic
    Smith, Alex Tapper},
  pdftitle={Search for natural and split supersymmetric scenarios in proton-proton
  collisions at 13 TeV in all-jet final states},
  pdfsubject={CMS, supersymmetry, AlphaT},
  pdfkeywords={Supersymmetry, split, natural, long-lived gluinos, dark matter}
}

\maketitle

%_______________________________________________________________________________
%_______________________________________________________________________________
%_______________________________________________________________________________

\section{Introduction}
\label{sec:introduction}

Supersymmetry (SUSY)~\cite{ref:SUSY-1, ref:SUSY0, ref:SUSY3,
  ref:SUSY1} is an extension of the standard model (SM) of particle
physics that introduces at least one bosonic (fermionic) superpartner
for each fermionic (bosonic) SM particle, where the superpartner
differs in its spin from its SM counterpart by one half unit.
Supersymmetry offers a potential solution to the hierarchy
problem~\cite{ref:hierarchy1, ref:hierarchy2}, predicts unification of
the gauge couplings at high energy~\cite{Dimopoulos:1981yj,
  Ibanez:1981yh, Marciano:1981un}, and provides a candidate for dark
matter (DM). Under the assumption of $R$-parity
conservation~\cite{Farrar:1978xj}, SUSY particles are expected to be
produced in pairs at the CERN LHC and to decay to the lightest stable SUSY
particle (LSP). The LSP is assumed to be the neutralino \PSGczDo, a
weakly interacting massive particle and a viable DM
candidate~\cite{Jungman:1995df, 1674-1137-38-9-090001}.  So-called
natural SUSY models, which invoke only a minimal fine tuning of the bare
Higgs boson mass parameter, require only the gluino, third-generation
squarks, and a higgsino-like \PSGczDo to have masses at or near the electroweak
scale~\cite{ref:barbierinsusy}. The interest in natural models is
motivated by the discovery of a low-mass Higgs
boson~\cite{Aad:2012tfa, Chatrchyan:2012ufa, Chatrchyan:2013lba,
  Khachatryan:2014jba, Aad:2014aba, Aad:2015zhl}. The characteristic
signature of natural SUSY production at the LHC is a final state
containing an abundance of jets originating from the hadronization of
heavy-flavour quarks and significant missing transverse momentum
\ptvecmiss.

%%%%%%%%%%

Split supersymmetry~\cite{ArkaniHamed:2004fb, Giudice:2004tc} does not
address the hierarchy problem---in contrast to natural SUSY models---but
preserves the appealing aspects of gauge coupling unification and a
DM candidate. In such a model, only the fermionic superpartners, and a
finely tuned scalar Higgs boson, may be realized at a mass scale that
is kinematically accessible at the LHC. All other SUSY particles are
assumed to be ultraheavy. Hence, within split SUSY models, the gluino
decay is suppressed because of the highly virtual squark states. For
gluino lifetimes beyond a picosecond, the gluino
hadronizes and forms a bound colour-singlet state containing the
gluino and quarks or gluons~\cite{Fairbairn:2006gg}, known as an
R-hadron, before eventually decaying to quark pairs and the
\PSGczDo. The long-lived gluino can lead to final states with
significant \ptvecmiss from the undetected \PSGczDo particles and to jets
with particle vertices located a significant distance from the
luminous region of the proton beams (displaced jets). A metastable
gluino, with a proper decay length significantly beyond the scale of
the CMS detector, can escape undetected.

%%%%%%%%%%%%%%

This paper presents a search for new-physics processes in final states
with one or more energetic jets and significant \ptvecmiss. The search
is performed with a sample of proton--proton (\Pp\Pp) collision data at
a centre-of-mass energy of 13\TeV recorded by the CMS experiment in 2016.
The analysed data sample corresponds to an integrated luminosity of $35.9
\pm 0.9\fbinv$~\cite{CMS:2017sdi}. Earlier searches using the same technique have been
performed in pp collisions at $\sqrt{s} = 7$, 8, and
13\TeV~\cite{Khachatryan:2011tk, Chatrchyan:2011zy, Chatrchyan:2012wa,
  Chatrchyan:2013mys, Khachatryan:2016pxa, Khachatryan:2016dvc} 
by the CMS Collaboration. The data
analysed in this analysis is a factor of 16 larger than that
presented in Ref.~\cite{Khachatryan:2016dvc}. The search strategy aims
to provide sensitivity to a broad range of SUSY-inspired
models that predict the existence of a DM candidate, and the search is
used to constrain the parameter spaces of a number of simplified SUSY
models~\cite{Alwall:2008ag, Alwall:2008va, sms}. The overwhelmingly
dominant background for a new-physics search in all-jet final states
resulting from \Pp\Pp\ collisions is the production of multijet events
through the strong interaction, a manifestation of quantum
chromodynamics. Several dedicated variables are employed to suppress
the multijet background to a negligible level while maintaining low
kinematical thresholds and high experimental acceptance for final
states characterized by the presence of significant \ptvecmiss. Signal
extraction is performed using several kinematical variables, namely
the number of jets, the number of jets identified as originating from
bottom quarks, and the scalar and vector sums of the jet transverse
momenta. The ATLAS and CMS Collaborations have performed similar
searches in all-jet final states at $\sqrt{s} = 13\TeV$, of which
those providing the tightest constraints are described in
Refs.~\cite{Aaboud:2016zdn, Sirunyan:2017cwe, Sirunyan:2017kqq}. This
search does not employ specialized reconstruction
techniques~\cite{Aaboud:2017iio, Aaboud:2016uth, Aaboud:2016dgf,
  Aad:2013gva, Khachatryan:2016sfv, Khachatryan:2015jha} that target
long-lived gluinos.

% involving: the pair production of squarks or any
%generation of gluinos, large or small mass splittings between the
%parent SUSY particle and the \PSGczDo, and meta-stable gluino
%particles.

%This paper presents a search for new-physics processes in final states
%with one or more energetic jets and significant \ptvecmiss. The search
%is performed with a sample of proton-proton (pp) collision data at a
%centre-of-mass energy of 13\TeV. The analysed data sample recorded by
%the CMS experiment corresponds to an integrated luminosity of $35.9
%\pm 0.9\fbinv$. Earlier searches using the same technique have been
%performed in pp collisions at $\sqrt{s} = 7$, 8, and
%13\TeV~\cite{RA1Paper, RA1Paper2011, RA1Paper2011FULL, RA1Paper2012,
%  RA1Parked, Khachatryan:2016dvc} by the CMS Collaboration. The data
%analysed in this analysis is a factor ${\approx}16$ larger than that
%presented in Ref.~\cite{Khachatryan:2016dvc}. The ATLAS and CMS
%Collaborations have performed similar searches in all-jet final states
%at $\sqrt{s} = 13\TeV$, of which the most constraining are described
%in Refs.~\cite{Aaboud:2016zdn, Sirunyan:2017cwe, Sirunyan:2017kqq}.

%The search employs several dedicated variables to discriminate against
%this background while maintaining experimental acceptance to events
%characterized by the presence of significant \ptvecmiss. Signal
%extraction is performed via several discriminating kinematical
%variables: the number of reconstructed jets per event (including the
%``monojet'' final state), the number of these jets identified as
%originating from bottom quarks, and the scalar and vector \pt sums of
%these jets. The strong discrimination against the multijet background
%permits the application of low thresholds on kinematical variables. This
%allows to maintain a large acceptance to a broad range of models, such
%as those that assume the strong production of massive coloured SUSY
%particles including third-generation squark signatures, both large and
%small mass splittings between the parent SUSY particle and the LSP,
%and long-lived gluino particles with a range of lifetimes. It is noted
%that the search employs standard event reconstruction techniques,
%without the use of specialized algorithms to target displaced vertices
%arising from the decays of long-lived states. The result of the search
%is used to constrain the parameter space of a number of simplified
%SUSY models~\cite{Alwall:2008ag, Alwall:2008va, sms}.

This paper is organized as follows. Section~\ref{sec:reconstruction}
describes the CMS apparatus and the event reconstruction
algorithms. Section~\ref{sec:selection} summarizes the selection
criteria used to identify and categorize candidate signal events and
samples of control data. Section~\ref{sec:simulation} outlines the
various software packages used to generate the samples of simulated
events. Sections~\ref{sec:ewk} and \ref{sec:qcd} describe the methods
used to estimate the background contributions from SM processes. The
result and interpretations are described in Sections~\ref{sec:result}
and~\ref{sec:interpretations}, respectively, and summarized in
Section~\ref{sec:summary}.

%_______________________________________________________________________________
%_______________________________________________________________________________
%_______________________________________________________________________________

%\clearpage
\section{The CMS detector and event reconstruction}
\label{sec:reconstruction}

The central feature of the CMS detector is a superconducting solenoid
of 6\unit{m} internal diameter, providing a magnetic field of
3.8\unit{T}. Within the solenoid volume are a silicon pixel and strip
tracker, a lead tungstate crystal electromagnetic calorimeter (ECAL),
and a brass and scintillator hadron calorimeter (HCAL), each composed
of a barrel and two endcap sections. Forward calorimeters extend the
pseudorapidity coverage provided by the barrel and endcap
detectors. Muons are detected in gas-ionization chambers embedded in
the steel flux-return yoke outside the solenoid. A more detailed
description of the CMS detector, together with a definition of the
coordinate system used and the relevant kinematical variables, can be
found in Ref.~\cite{Chatrchyan:2008zzk}.

Events of interest are selected using a two-tiered trigger
system~\cite{Khachatryan:2016bia}. The first level, composed of custom
hardware processors, uses information from the calorimeters and muon
detectors to select events at a rate of around 100\unit{kHz} within a
time interval of less than 4\mus. The second level, known as the
high-level trigger, consists of a farm of processors running a version
of the full event reconstruction software optimized for fast
processing, and reduces the event rate to less than 1\unit{kHz} before
data storage. The trigger logic used by this search is summarized in
Section~\ref{sec:selection}.

The particle-flow (PF) event reconstruction
algorithm~\cite{CMS-PRF-14-001} reconstructs and identifies candidate
physics objects, including photons~\cite{Khachatryan:2015iwa},
electrons~\cite{Khachatryan:2015hwa}, muons~\cite{Chatrchyan:2012xi},
and charged and neutral hadrons, with an optimized combination of
information from the various elements of the CMS detector. The
reconstruction techniques and physics object definitions used by this
search are not specialized to target specific experimental signatures
(such as displaced jets). The physics object requirements are defined
below and are summarized in Table~\ref{tab:selections}. In the case of
photons and leptons, further details can be found in
Ref.~\cite{Khachatryan:2016dvc} and references therein.

\begingroup
\renewcommand*{\arraystretch}{1.2}
\newcommand{\mybox}[3]{\makebox[\widthof{\hspace{#1}}][#2]{#3}}
\begin{table}[!t]
  \topcaption{Summary of physics objects, baseline event selections,
    signal and control regions, and event categorization. The nominal 
    categorization schema is defined in full in \suppMaterial.
  }  
  \label{tab:selections}
  \centering
  \resizebox{\textwidth}{!}{
    \begin{tabular}{ ll }
      \hline
      \multicolumn{2}{l}{\bf Physics objects}                                                                                                   \\
      Jet                               & $\pt > 40\GeV$, $\abs{\eta} < 2.4$                                                                    \\ 
      Photon                            & $\pt > 25\GeV$, $\abs{\eta} < 2.4$, isolated in cone ${\Delta}R < 0.3$                                \\ 
      Electron                          & $\pt > 10\GeV$, $\abs{\eta} < 2.4$, $I^\text{rel} < 0.1$ in cone $0.05 < {\Delta}R(\pt) < 0.2$        \\
      Muon                              & $\pt > 10\GeV$, $\abs{\eta} < 2.4$, $I^\text{rel} < 0.2$ in cone $0.05 < {\Delta}R(\pt) < 0.2$        \\ 
      Single isolated track             & $\pt > 10\GeV$, $\abs{\eta} < 2.4$, $I^\text{track} < 0.1$ in cone ${\Delta}R < 0.3$                  \\ 
      \hline
      \multicolumn{2}{l}{\bf Baseline event selection}                                                                                          \\
      All-jet final state               & Veto events containing photons, electrons, muons, SITs, as defined above                              \\
      \ptmiss cleaning                  & Veto events based on filters related to beam and instrumental effects                                 \\ 
      Jet cleaning                      & Veto events containing jets that fail identification criteria or $0.1 < f_{h^{\pm}}^{\jet{1}} < 0.95$ \\ 
      Jet energy and sums               & $\pt^{\jet{1}} > 100\GeV$, $\scalht > 200\GeV$, $\mht > 200\GeV$                                      \\
      Jets outside acceptance           & $\mhtmet < 1.25$, veto events containing jets with $\pt > 40\GeV$ and $\abs{\eta} > 2.4$              \\
      \hline
      {\bf Signal region}               & Baseline selection +                                                                                  \\
%                                       &                                                                                                       \\
      \alphat threshold (\scalht range) & 0.65 (200--250\GeV), 0.60 (250--300), 0.55 (300--350), 0.53 (350--400), 0.52 (400--900)               \\
      \bdphi threshold                  & $\bdphi > 0.5$ ($\njet \geq 2$), $\bdphimod > 0.5$ ($\njet = 1$)                                      \\
      \hline
      \multicolumn{2}{l}{\bf Nominal categorization schema}                                                                                     \\
      \njet                             & \mybox{5cm}{l}{1} (monojet)                                                                           \\
                                        & \mybox{5cm}{l}{${\geq}2a$} ($a$ denotes asymmetric, $40 < \pt^{\jet{2}} < 100\GeV$)                   \\
                                        & \mybox{5cm}{l}{2, 3, 4, 5, ${\geq}6$} (symmetric, $\pt^{\jet{2}} > 100\GeV$)                          \\
      \nb                               & \mybox{5cm}{l}{0, 1, 2, 3, ${\geq}4$} (can be dropped/merged \vs \njet)                               \\
      \scalht boundaries [\GeVns{}]     & \mybox{5cm}{l}{200, 400, 600, 900, 1200} (can be dropped/merged \vs \njet, \nb)                       \\
      \mht boundaries [\GeVns{}]        & \mybox{5cm}{l}{200, 400, 600, 900} (can be dropped/merged \vs \njet, \nb, \scalht)                    \\
      \hline
      \multicolumn{2}{l}{\bf Simplified categorization schema}                                                                                  \\
      Topology (\njet, \nb)             
                                        & \mybox{2.5cm}{l}{Monojet-like} ($1 \cap {\geq}2a, 0$), ($1 \cap {\geq}2a, {\geq}1$)                   \\
                                        & \mybox{2.5cm}{l}{Low \njet} ($2 \cap 3, 0 \cap 1$), ($2 \cap 3, {\geq}2$)                             \\
                                        & \mybox{2.5cm}{l}{Medium \njet} ($4 \cap 5, 0 \cap 1$), ($4 \cap 5, {\geq}2$)                          \\
                                        & \mybox{2.5cm}{l}{High \njet} (${\geq}6, 0 \cap 1$), (${\geq}6, {\geq}2$)                              \\
      \scalht boundaries [\GeVns{}]     & $\scalht > 200\GeV$ ($\njet \leq 3$), $\scalht > 400\GeV$ ($\njet \geq 4$)                            \\
      \mht boundaries [\GeVns{}]        & 200, 400, 600, 900                                                                                    \\
      \hline
      {\bf Control regions}             & Baseline selection +                                                                                  \\
%                                       &                                                                                                       \\
      \mj (inverted $\mu$ veto)         
                                        & $\pt^{\mu_1} > 30\GeV$, $\abs{\eta^{\mu_1}} < 2.1$, 
                                        ${\Delta}R(\mu,\jet{i}) > 0.5$,
                                        $30 < m_\text{T}(\ptvec^\mu,\ptvecmiss) < 125\GeV$                                                      \\
      \mmj (inverted $\mu$ veto)        
                                        & $\pt^{\mu_{1,2}} > 30\GeV$, $\abs{\eta^{\mu_{1,2}}} < 2.1$, 
                                        ${\Delta}R(\mu_{1,2},\jet{i}) > 0.5$, 
                                        $ \abs{m_{\mu\mu} - m_\text{Z}} < 25\GeV$                                                               \\
      Multijet-enriched                 & Sidebands to signal region: $\mht/\ptmiss > 1.25$ and/or $\bdphi < 0.5$                               \\  
      \hline
    \end{tabular}
  }
\end{table}
\endgroup

The reconstructed vertex with the largest value of summed
physics-object $\pt^2$ is taken to be the primary \Pp\Pp\ interaction
vertex (PV). The physics objects considered are those returned by a
jet finding algorithm~\cite{Cacciari:2008gp, Cacciari:2011ma} applied
to all charged particle tracks associated with the vertex, plus the
corresponding associated \ptmiss. Charged particle tracks associated
with vertices from additional \Pp\Pp\ interactions within the same or
nearby bunch crossings (pileup) are not considered by the PF algorithm
as part of the global event reconstruction.

Samples of candidate signal events and control data are defined,
respectively, by the absence or presence of photons and leptons that
are isolated from other activity in the event. Photons are required to
be isolated~\cite{Khachatryan:2015iwa} within a cone around the photon
trajectory defined by the radius ${\Delta}R =
\sqrt{\smash[b]{(\Delta\phi)^2 + (\Delta\eta)^2}} = 0.3$, where
$\Delta\phi$ and $\Delta\eta$ represent differences in pseudorapidity
and the azimuthal angle. Isolation for an electron or muon is a
relative quantity, $I^\text{rel}$, defined as the scalar \pt sum of
all candidate particles within a cone around its trajectory, divided
by the lepton \pt. The cone radius is dependent on the lepton \pt,
${\Delta}R = \min [ \max( 0.05, 10\GeV / \pt ), 0.2 ]$, to maintain
high efficiency for semileptonic decays of Lorentz-boosted top
quarks~\cite{Rehermann:2010vq}. Isolated electrons and muons are
required to satisfy $I^\text{rel} < 0.1$ and 0.2, respectively.
Electron and muon candidates that fail any of the aforementioned
requirements, as well as charged-hadron candidates from hadronically
decaying tau leptons, are collectively labelled as single isolated
tracks (SIT) if the scalar \pt sum of additional tracks associated to
the PV within a cone ${\Delta}R < 0.3$ around the track trajectory,
relative to the track \pt, satisfies $I^\text{track} < 0.1$. All
isolation variables exclude the contributions from the physics object
itself and pileup events. The experimental acceptances for photons,
electrons, muons, and SITs are defined by the transverse momentum
requirements $\pt > 25$, 10, 10, and 10\GeV, respectively, and the
pseudorapidity requirement $\abs{\eta} < 2.4$.

Jets are reconstructed from the particle candidates, clustered by the
anti-\kt algorithm~\cite{Cacciari:2008gp, Cacciari:2011ma} with a
distance parameter of 0.4. In this process, the raw jet energy is
obtained from the sum of the candidate particle energies and the raw
jet momentum by the vectorial sum of the candidate particle momenta,
which results in a nonzero jet mass. An offset correction is applied
to jet energies to take into account the contributions from neutral
particles produced in pileup events~\cite{Cacciari:2007fd,
  CMS-PAS-JME-14-001}. The raw jet energies are then corrected to
establish a relative uniform response of the calorimeter in $\eta$ and
a calibrated absolute response in \pt. Jet energy corrections are
derived from simulation, and are confirmed with in situ measurements
of the energy balance in events with a dijet topology or containing a
photon and a jet~\cite{Khachatryan:2016kdb}. Jets are required to
satisfy $\pt > 40\GeV$ and $\abs{\eta} < 2.4$. Jets are also subjected
to a standard set of identification criteria~\cite{2011JInst611002C}
that require each jet contains at least two particle candidates and at
least one charged particle track, and has a nonzero energy fraction
attributed to charged-hadron particle candidates, $f_{h^{\pm}} > 0$.

Jets can be identified as originating from b quarks using the combined
secondary vertex (CSV) algorithm~\cite{Chatrchyan:2012jua}. Data
samples~\cite{CMS-PAS-BTV-15-001} are used to measure the b-tagging
efficiency, which is the probability to correctly identify jets
originating from b quarks, as well as the mistag probability to
identify jets originating from light-flavour (LF) partons (u, d, s
quarks or gluon) or a charm quark as a mistagged jet. A working point
is employed that yields a b-tagging efficiency of ${\approx}69\%$, and
charm and LF mistag probabilities of ${\approx}18$ and ${\approx}1\%$,
respectively, for jets with $\pt > 30\GeV$ from \ttbar events.

Finally, the most accurate estimator for \ptvecmiss is defined as the
projection on the plane perpendicular to the beams of the negative
vector sum of the momenta of all PF candidate particles in an
event. Its magnitude is referred to as \ptmiss.

%_______________________________________________________________________________
%_______________________________________________________________________________
%_______________________________________________________________________________

%\clearpage
\section{Event selection and categorization}
\label{sec:selection}

A baseline set of event selection criteria, described in
Section~\ref{sec:baseline}, is used as a basis for all data samples
used in this search. Two additional requirements, described in
Section~\ref{sec:signal}, are employed to define a sample of candidate
signal events, labelled henceforth as the signal region (SR). The
categorization of candidate signal events and the background
composition are described in Sections~\ref{sec:categorization} and
\ref{sec:bkgd}, respectively. Three independent control regions (CRs),
comprising large samples of event data, are defined by the selection
criteria described in Section~\ref{sec:control}. All selection
criteria are summarized in Table~\ref{tab:selections}.

\subsection{Baseline selections}
\label{sec:baseline}

Events containing isolated photons, electrons and muons, or SITs that
satisfy the requirements summarized in Table~\ref{tab:selections} are
vetoed to select all-jet final states, suppress SM processes that
produce final states containing neutrinos, and reduce backgrounds from
misreconstructed or nonisolated leptons as well as single-prong
hadronic decays of $\tau$ leptons.

Beam halo, spurious jet-like features originating from isolated noise
patterns in the calorimeter systems, detector inefficiencies, and
reconstruction failures can all lead to large values of \ptmiss. Such
events are rejected with high efficiency using dedicated
vetoes~\cite{CMS-PAS-JME-16-004, Khachatryan:2014gga}. Events are
vetoed if any jet fails the identification criteria described in
Section~\ref{sec:reconstruction}. Further, the highest-\pt jet of the
event $\jet{1}$ is required to satisfy $0.1 < f_{h^{\pm}}^{\jet{1}} <
0.95$ to further suppress beam halo and rare reconstruction failures.

The highest-\pt jet in the event is required to satisfy $\pt^{\jet{1}}
> 100\GeV$. The mass scale of each event is estimated from the scalar
sum of the \pt of jets, defined as $\scalht = \sum_{\jet{i} =
  1}^{\njet} \pt^{\,\jet{i}}$, where \njet is the number of jets
within the experimental acceptance. The estimator for \ptvecmiss used
by this search is given by the magnitude of the vector sum of the \pt
of jets, $\mht = |\sum_{\jet{i} = 1}^{\njet}
\ptvec^{\,\jet{i}}|$. Significant hadronic activity and \ptvecmiss,
typical of supersymmetric processes, is ensured by requiring $\scalht
> 200\GeV$ and $\mht > 200\GeV$, respectively.

Events are vetoed if any additional jet satisfies $\pt > 40\GeV$ and
$|\eta| > 2.4$ to maintain the resolution of the \mht variable.
An additional veto is employed to deal with the circumstance in which
several jets with transverse momentum below the \pt thresholds and
collinear in $\phi$ can result in significant \mht relative to
\ptmiss, the latter of which is less sensitive to jet thresholds. This
type of event topology, which is typical of multijet events, is
suppressed while maintaining high efficiency for new-physics processes
with significant \ptvecmiss by requiring $\mhtmet < 1.25$.

%_______________________________________________________________________________
%_______________________________________________________________________________
%_______________________________________________________________________________

\subsection{Signal region}
\label{sec:signal}

The multijet background dominates over all other SM backgrounds
following the application of the baseline event selection
criteria. The multijet background is suppressed to a negligible level
through the application of two dedicated variables that provide strong
discrimination between multijet events with \ptvecmiss resulting from
instrumental sources, such as jet energy mismeasurements, and
new-physics processes that involve the production of weakly
interacting particles that escape detection.

The first variable, \alphat~\cite{Randall:2008rw, Khachatryan:2011tk}, is
designed to be intrinsically robust against jet energy
mismeasurements. In its simplest form, the \alphat variable is defined
as $\alphat = \Et^{\jet{2}}/M_\text{T}$, where $M_\text{T} = \sqrt{ 2
  \Et^{\jet{1}} \Et^{\jet{2}} (1 - \cos\phi_{\jet{1},\jet{2}})}$ and
$\phi_{\jet{1},\jet{2}}$ is defined as the azimuthal angle between
jets $\jet{1}$ and $\jet{2}$.
% and $M_\text{T}$ is the transverse mass of a dijet system. 
In the absence of jet energy mismeasurements, and in the limit for
which the \Et of each jet is large compared with its mass, a well
measured dijet event with $\Et^{\jet{1}} = \Et^{\jet{2}}$ and
back-to-back jets ($\phi_{\jet{1},\jet{2}} = \pi$) yields an \alphat
value of 0.5. In the presence of a jet energy mismeasurement,
$\Et^{\jet{1}} > \Et^{\jet{2}}$ and $\alphat < 0.5$. Values
significantly greater than 0.5 can be observed when the two jets are
not back-to-back and recoil against \ptvecmiss from weakly interacting
particles that escape the detector. The definition of the \alphat
variable can be generalized for events with two or more jets, as
described in Ref.~\cite{Khachatryan:2011tk}. Multijet events populate the region
$\alphat \lesssim 0.5$ and the \alphat distribution is characterized
by a sharp edge at 0.5, beyond which the multijet event yield falls by
several orders of magnitude. The SM backgrounds that involve the
prompt production of neutrinos (\eg not semileptonic heavy flavour
decays) result in a long tail in \alphat beyond values of 0.5. A
\scalht-dependent \alphat threshold that decreases from 0.65 at low
\scalht to 0.52 at high \scalht within the range $200 < \scalht <
900\GeV$ is employed to maintain an approximately constant rejection
power against the multijet background.% that evolves because of jet
%acceptances and resolution effects.

The second variable, known as \bdphi, considers the minimum azimuthal
angular separation between each jet in the event and the vector sum of
the \pt of all other jets in the event. Multijet events typically
populate the region $\bdphi \approx 0$ while events with genuine
\ptvecmiss can have values up to $\bdphi = \pi$. The requirement
$\bdphi > 0.5$ is sufficient to reduce significantly the multijet
background, including rare contributions from energetic multijet
events that yield both high jet multiplicities and significant
\ptvecmiss because of high-multiplicity neutrino production in
semileptonic heavy-flavour decays. 
%The neutrinos are typically collinear with respect to the axis of a
%jet and carry a significant fraction of the energy. 
For events that satisfy $\njet = 1$, a small modification to the
\bdphi variable is utilized that considers any additional jets with
$25 < \pt < 40\GeV$ that are outside the nominal experimental
acceptance ($\bdphimod > 0.5$).

The requirements on \alphat and \bdphi, summarized in
Table~\ref{tab:selections}, suppress the expected contribution from
multijet events to the sub-percent level with respect to the total
expected background counts from all other SM processes. 
%, collectively labelled henceforth as nonmultijet backgrounds. 
For the region $\scalht > 900\GeV$, the necessary control of the
multijet background is achieved solely with the \bdphi and \bdphimod
variables. These requirements complete the definition of the SR.

Candidate signal events are recorded with a number of trigger
algorithms. Events with $\njet \geq 2$ must satisfy thresholds on both
\scalht and \alphat that are looser than those used to define the
SR. Additional trigger logic that requires $\scalht > 900\GeV$ is used
to record high-activity events. Finally, a trigger condition that
requires $\mht > 120\GeV$, $\ptmiss > 120\GeV$, and a single jet with
$\pt > 20\GeV$ and $|\eta| < 5.2$ is also used to efficiently record
candidate signal events for all categories of the SR, including those
that satisfy $\njet \geq 1$. The combined performance of these trigger
algorithms yields high efficiencies, as determined from samples of CR
data enriched in vector boson + jets and \ttbar events. The
efficiencies are primarily \scalht-dependent that range from
97.4--97.9\% ($200 < \scalht < 600\GeV$) to 100\% ($\scalht >
600\GeV$) with statistical and systematic uncertainties at the percent
level. Trigger efficiencies for a range of benchmark signal models are
typically comparable or higher (${\approx}100\%$).

%_______________________________________________________________________________
%_______________________________________________________________________________
%_______________________________________________________________________________

\subsection{Event categorization}
\label{sec:categorization}

Candidate signal events are categorized into 27 discrete topologies
according to \njet and the number of b-tagged jets \nb. Events are
further binned according to the energy sums \scalht and \mht. The
binning schema is determined primarily by the statistical power of the
\mj and \mmj CRs. 

Seven bins in \njet are considered, as summarized in
Table~\ref{tab:selections}. Events that contain only a single jet
within the experimental acceptance ($\njet = 1$) are labelled as
``monojet''. Events containing two or more jets are categorized
according to the second highest jet \pt. Events that satisfy $\njet
\geq 2$ with only the highest-\pt jet satisfying $\pt > 100\GeV$ are
labelled as ``asymmetric''. Events for which the second-highest jet
\pt also satisfies $\pt > 100\GeV$ are labelled as ``symmetric'' and
are categorized according to \njet (2, 3, 4, 5, and ${\geq}6$). The
symmetric topology targets the pair production of SUSY particles and
their prompt cascade decays, while the monojet and asymmetric
topologies preferentially target models with a compressed mass
spectrum ($m_\text{SUSY} - m_{\PSGczDo} \lesssim 100\GeV$) and
long-lived SUSY particles.
%, as well as the direct production of weakly interacting massive particles.

Events are also categorized according to \nb (0, 1, 2, 3, ${\geq}4$),
where \nb is bounded from above by \njet and the choice of
categorization is dependent on \njet. Higher \nb multiplicities target
the production of third-generation squarks.
%In total, candidate signal events are categorized into 27 discrete
%(\njet, \nb) topologies, as detailed in Table~\ref{tab:binning}.

The nominal binning schema for \scalht is defined as follows: four
bounded bins that satisfy 200--400, 400--600, 600--900, and
900--1200\GeV, and a final open bin $\scalht > 1200\GeV$. This schema
is adapted per (\njet, \nb) category as follows: only the region
$\scalht > 400\GeV$ is considered for events that satisfy $\njet \geq
4$, and bins at high \scalht are merged with lower-\scalht bins to
satisfy a threshold on the minimum number of events in the
corresponding bins of the CRs. 
%These adjustments ensure all bins in the CRs are sufficiently
%populated to estimate the SM backgrounds in the corresponding bins of
%the SR and validate assumptions in the likelihood model.

The \mht variable is used to further categorize events according to
three bounded bins that satisfy 200--400, 400--600, and 600--900, and
a final open bin $\scalht > 900\GeV$. The \mht binning depends on
\njet, \nb, and \scalht. Given that \mht cannot exceed \scalht by
construction, the lower bound of the final \mht bin is restricted to
be not higher than the lower bound of the \scalht bin in
question. Events that satisfy $\njet = 1$ or $200 < \scalht < 400\GeV$
are counted inclusively with respect to \mht.

In total, there are 254 bins in the SR. An alternate, simplified
binning schema is also provided in which events are categorized
according to eight topologies defined in terms of \njet and \nb. For
each topology, event yields are integrated over the full available
\scalht range and categorized according to the four nominal \mht bins
defined above. This schema has 32 bins that are exclusive, contiguous,
and provide a complete coverage of the SR. The SM background estimates
are obtained from the same likelihood model used to determine the
nominal result. 

%_______________________________________________________________________________
%_______________________________________________________________________________
%_______________________________________________________________________________

\subsection{Background composition}
\label{sec:bkgd}

Following the application of the SR selection criteria, 
%including the \alphat and \bdphi variables, 
the multijet background is reduced to a negligible level. 
%and it is estimated using the multijet-enriched sidebands to the SR
%defined in Section~\ref{sec:control}. The method is described in
%Section~\ref{sec:qcd}.
The remaining background counts are dominated by processes that
involve the production of high-\pt neutrinos in the final state. The
associated production of jets and Z bosons that decay to \znunu
dominate the background counts for events containing low numbers of
jets and b-tagged jets. The \znunuj background is irreducible.
%and estimated using the \mmj CR defined in Section~\ref{sec:control}
%according to the method described in Section~\ref{sec:ewk}.
The associated production of jets and W bosons, decaying to $\PW^\pm
\to \ell\nu$ ($\ell=\Pe$, $\Pgm$, $\Pgt$), is also a significant
background in the same phase space. The production and semileptonic
decay of top quark-antiquark pairs (\ttbar) to W bosons and b quarks
becomes the dominant background process for events containing high
numbers of jets or b-tagged jets. 
%
Events that contain the leptonic decay of a W boson are typically
rejected by the vetoes that identify the presence of leptons or single
isolated tracks. If the lepton is outside the experimental acceptance,
or is not identified or isolated, then the event is not vetoed and the
aforementioned processes lead to what is collectively known as the
``lost lepton'' (\lost) background.
%
Residual contributions from other SM processes are also considered,
such as single top production, WW, WZ, ZZ (diboson) production, and
the associated production of \ttbar and a boson ({\ttbar}W, {\ttbar}Z,
{\ttbar}$\gamma$, and {\ttbar}H).
%, may also lead to W bosons in the final state.  
%, which is estimated using the \mj CR defined in
%Section~\ref{sec:control} and the method described in
%Section~\ref{sec:ewk}.

%_______________________________________________________________________________
%_______________________________________________________________________________
%_______________________________________________________________________________

\subsection{Control regions}
\label{sec:control}

Topological and kinematical requirements, summarized in
Table~\ref{tab:selections}, ensure the samples of CR data are enriched
in the same or similar SM processes that populate the SR, as well as
being depleted in contributions from SUSY processes (signal
contamination).

Three sidebands to the SR comprising multijet-enriched event samples
are defined by: $1.25 < \mhtmet < 3.0$ (region $A$), $0.2 < \bdphi <
0.5$ ($B$), and both $1.25 < \mhtmet < 3.0$ and $0.2 < \bdphi < 0.5$
($C$). Events are categorized according to \njet and \scalht,
identically to the SR. 
%A model is assumed to determine the estimates as a function of \nb
%and \mht. 
Events are recorded with the signal triggers described above.

Two additional CRs comprising \mj and \mmj event samples are defined
by the application of the baseline selections and requirements on
isolated, central, high-\pt muons. Tighter isolation requirements for
the muons are applied with respect to those indicated in
Table~\ref{tab:selections} to ensure high trigger efficiencies. A
trigger condition that requires an isolated muon with $\pt > 24\GeV$
and $\abs{\eta} < 2.1$ is used to record the \mj and \mmj event
samples with efficiencies of ${\approx}90$ and ${\approx}99\%$,
respectively. For both samples, no requirements on \alphat nor \bdphi
are imposed. The kinematical properties of events in the \mj and \mmj
CRs and SR are comparable once the muon or dimuon system is ignored in
the calculation of event-level quantities such as \scalht and \mht.
Events in both samples are categorized according to \njet, \scalht,
and \nb, with counts integrated over \mht. The \njet categorization is
identical to the SR. Background predictions are determined using up to
eleven bins in \scalht that are then aggregated to match the \scalht
binning schema used by the SR. 
%This approach allows more granular corrections to be applied to
%simulated events and a more accurate modelling of the SM
%backgrounds. 
The \nb categorization for \mj events is identical to the SR, whereas
\mmj events are subdivided according to $\nb = 0$ and $\nb \geq
1$. Differences in the binning schemas between the SR and CRs are
accounted for in the background estimation methods through
simulation-based templates, the modelling of which is validated
against control data.

The \mj event sample is enriched in events from \wmj and \ttbar
production, as well as other SM processes (\eg single top and diboson
production), that are manifest in the SR as the \lost
backgrounds. Each event is required to contain a single isolated muon
with $\pt > 30\GeV$ and $\abs{\eta} < 2.1$ to satisfy trigger
conditions and is well separated from each jet $\text{j}_i$ in the
event according to ${\Delta}R(\mu,\text{j}_i) > 0.5$. The transverse
mass formed by the muon \pt and \ptvecmiss system must satisfy $30 <
m_\text{T} < 125\GeV$ to select a sample of events rich in W bosons. 
%, produced promptly or from the decay of top quarks.

The \mmj sample is enriched in $\PZ\! \rightarrow\!  \mu^+\mu^-$
events that have similar acceptance and kinematical properties to
\znunuj events when the muons are ignored. The sample uses selection
criteria similar to the \mj sample, but requires two oppositely
charged, isolated muons that both satisfy $\pt > 30\GeV$, $\abs{\eta}
< 2.1$, and ${\Delta}R(\mu_{1,2},\text{j}_i) > 0.5$. The muons are
also required to have a dilepton invariant mass $m_{\mu\mu}$ within a
${\pm}25\GeV$ window around the mass of the Z
boson~\cite{1674-1137-38-9-090001}.

%_______________________________________________________________________________
%_______________________________________________________________________________
%_______________________________________________________________________________

%\clearpage
\section{Monte Carlo simulation}
\label{sec:simulation}

The search relies on several samples of simulated events, produced
with Monte Carlo (MC) generator programs, to estimate SM backgrounds
and potential signal contributions.

The {\MADGRAPH{}5\_a\MCATNLO} 2.2.2~\cite{Alwall2014} event generator
is used at leading-order (LO) accuracy to produce samples of \wj, \zj,
\ttbar, and multijet events. Up to three or four additional partons
are included in the matrix-element calculation for \ttbar and
vector boson production, respectively. Simulated \wj and \zj events are
weighted according to the true vector boson \pt %obtained from the
%{\MADGRAPH{}5\_a\MCATNLO} generator code (used at LO) 
to account for the effect of missing NLO QCD and electroweak terms in
the matrix-element calculation, according to the method described in
Ref.~\cite{Khachatryan:2016mdm}. Within the range of vector boson
transverse momenta that can be probed by this search, the QCD and EWK
corrections~\cite{Kuhn:2005gv} are largest, ${\approx}40\%$ and
${\approx}15\%$, at low and high values of boson \pt,
respectively. Simulated events for \ttbar are weighted to improve the
description of jets arising from initial-state radiation
(ISR)~\cite{Chatrchyan:2013xna}. The weights vary from 0.92 to 0.51
depending on the number of jets (1--6) from ISR, with an uncertainty
of one half the deviation from unity. The {\MADGRAPH{}5\_a\MCATNLO}
generator is used at next-to-leading-order (NLO) accuracy to generate
samples of s-channel production of single top, as well as {\ttbar}W
and {\ttbar}Z events. The NLO \POWHEG v2~\cite{powheg, powheg_top_Wt}
generator is used to describe the $t$- and tW-channel production of
events containing single top quarks, as well as {\ttbar}H events. The
\PYTHIA 8.205~\cite{pythia} program is used to generate diboson (WW, WZ,
ZZ) production. 

Event samples for signal models involving the production of gluino or
squark pairs, in association with up to two additional partons, are
generated at LO with {\MADGRAPH{}5\_a\MCATNLO}, and the decay of the
SUSY particles is performed with \PYTHIA 8.205~\cite{pythia}. The
\textsc{NNPDF}3.0 LO and \textsc{NNPDF}3.0 NLO~\cite{nnpdf} parton
distribution functions (PDF) are used, respectively, with the LO and
NLO generators described above.

The simulated samples for SM processes are normalized according to
production cross sections that are calculated with NLO and next-to-NLO
precision~\cite{Alwall2014, wphys, fewz, wwxs, top++, nlotop,
  powheg_top_Wt}. The production cross sections for pairs of gluinos
or squarks are determined at NLO plus next-to-leading-logarithm (NLL)
precision~\cite{Beenakker:1996ch, Kulesza:2008jb, Kulesza:2009kq,
  Beenakker:2009ha, Beenakker:2011fu, Borschensky:2014cia}. All SUSY
particles other than the \PSGczDo are assumed to be heavy and
decoupled from the interaction. Uncertainties in the cross sections
are determined from different choices of PDF sets, and factorization
and renormalization scales ($\mu_\text{R}$ and $\mu_\text{F}$),
according to the prescription in Ref.~\cite{Borschensky:2014cia}. The
\PYTHIA 8.205~\cite{pythia} program is used to describe parton showering
and hadronization for all simulated samples.

%Kraan:2004tz, Mackeprang:2006gx, Mackeprang:2009ad
The \textsc{rhadrons} package within the \PYTHIA program is used to
describe the formation of R-hadrons through the hadronization of
gluinos~\cite{Fairbairn:2006gg, Kraan:2004tz, Mackeprang:2006gx}. The
hadronization process, which is steered according to the default
parameter settings of the \textsc{rhadrons} package, predominantly
yields meson-like (\PSg\Pq\Paq) and baryon-like (\PSg\Pq\Pq\Pq)
states, but also R-hadrons comprising a gluino and a valence gluon
with a probability of 10\%. 
%The phenomenological ``cloud model'' is commonly used to describe
%these loosely bound states and their interactions with
%matter~\cite{}. 
The gluino is assumed to undergo a three-body decay, to a \Pq\Paq\
pair and the \PSGczDo, according to its proper decay length \ctau [mm]
that is a parameter of the simplified model. The details of the
R-hadron model are expected to have little influence on the
hadronization of the quarks originating from the gluino decay (and
hence the final state in terms of event topology and kinematical
variables) for the models considered in this paper. The interactions
of R-hadrons with the detector material are not simulated, as these
interactions are not exploited by the generic nature of this search.
%, unlike more dedicated techniques used elsewhere.~\cite{}\fixme{eg
%  reference dE/dx, stopped searches, disappearing/kinked tracks}. 

The description of the detector response is implemented using the
\GEANTfour~\cite{geant} package for all simulated SM processes. Scale
factors are applied to simulated event samples that correct for
differences in the b-tag efficiency 
%($\text{SF}_\text{b}$) 
and mistag probability of LF partons and charm quarks 
%($\text{SF}_\text{LF}$) 
with respect to data. The scale factors have typical values of
${\approx}$0.95--1.00 and ${\approx}$1.00--1.20, respectively, for a
jet-\pt range of 40--600\GeV. All remaining signal models rely on the
CMS fast simulation package~\cite{fastsim} that provides a description
that is consistent with \GEANTfour following the application of
near-unity corrections %of X--Y and X--Y
for differences in the b-tag efficiency and mistag probability,
%respectively, 
as well as corrections %of X--Y
for differences in the modelling of \mht variable. To model the
effects of multiple pp collisions within the same or neighbouring
bunch crossings (pileup), all simulated events are generated with a
nominal distribution of pp interactions per bunch crossing and then
reweighted to match the pileup distribution as measured in data.

%_______________________________________________________________________________
%_______________________________________________________________________________
%_______________________________________________________________________________

%\clearpage
\section{Nonmultijet background evaluation}
\label{sec:ewk}

The \lost and \znunuj backgrounds, collectively labelled henceforth as
the nonmultijet backgrounds, are estimated from data samples in CRs
and transfer factors $\mathcal{R}$ determined from ratios of expected
counts obtained from simulation:

\begin{align}
  \tf^{\lost} \, & = \,
  \frac{N^{\lost}_\text{MC}(\njet, \scalht, \nb, \mht)}
  {N^{\mj}_\text{MC}(\njet, \scalht, \nb)\hfill} \; ,
  & 
  N^{\lost}_\text{pred} \, & = \,
  \tf^{\lost} \; N^{\mj}_\text{data} \; ,
  \\
  \tf^{\znunu} \, & = \,
  \frac{N^{\znunu}_\text{MC}(\njet, \scalht, \nb, \mht)}
  {N^{\mmj}_\text{MC}(\njet, \scalht, \nb)\hfill} \; ,
  & 
  N^\text{\znunu}_\text{pred} \, & = \,
  \tf^{\znunu} \; N^{\mmj}_\text{data} \; ,
\end{align}

where $\tf^{\lost}$ and $\tf^{\znunu}$ are the transfer factors that
act as multiplier terms on the event counts $N^{\mj}_\text{data}$ and
$N^{\mmj}_\text{data}$ observed in each (\njet, \scalht, \nb) bin of,
respectively, the \mj and \mmj CRs to estimate the \lost or \znunuj
background counts $N^{\lost}_\text{pred}$ and $N^{\znunu}_\text{pred}$
in the corresponding (\njet, \scalht, \nb, \mht) bins of the SR. 
%The categorization of the \mj and \mmj event samples is described in
%Section~\ref{sec:control}.
Several sources of uncertainty in the transfer factors are evaluated.
In addition to statistical uncertainties arising from finite-size
simulated event samples, the most relevant systematic effects are
discussed below.

%, and generally fall into one of three categories. The first category
%concerns uncertainties in corrections applied to simulation that are
%obtained from theoretical calculations or measurements determined from
%data samples to account for the mismodelling of experimental
%parameters. The second category concerns ``closure tests'' in data
%that probe specific extrapolations or assumptions made in the
%analysis. Finally, the third category concerns the use of multiple
%samples of CR data to evaluate the degree to which the simulation
%describes the \mht distributions observed in data, and to assign
%appropriate systematic uncertainties in addition to known theoretical
%and experimental uncertainties.

The uncertainties from known theoretical and experimental sources are
propagated through to the transfer factors to ascertain the magnitude
of variations related to: the jet energy scale, the efficiency and
mistag probability of b-tagged jets, the efficiency to trigger on and
identify, or veto, well-reconstructed isolated leptons, 
%the inclusive W boson and \ttbar production cross
%sections~\cite{Sirunyan:2017wgx, Sirunyan:2017uhy},
the parton density functions~\cite{Butterworth:2015oua},
$\mu_\text{R}$ and $\mu_\text{F}$, and the modelling of jets from ISR
produced in association with
\ttbar~\cite{Chatrchyan:2013xna}. Uncertainties of 100\% in both the
NLO QCD and electroweak corrections to the \wj and \zj simulated
samples are also considered. A 5\% uncertainty in the minimum bias
cross section~\cite{Aaboud:2016mmw} is assumed and propagated through
to the reweighting procedure to account for differences between the
simulated and data-derived measurements of the pileup
distributions. Uncertainties in the signal trigger efficiency
measurements are also propagated to the transfer factors. The effect
of the aforementioned systematic uncertainties are summarized in
Table~\ref{tab:bkgd_systs}, in terms of representative ranges.  Each
source of uncertainty is assumed to vary with a fully correlated
behaviour across the full phase space of the SR and CRs.
%Generally, all uncertainties lead to variations in the simulated \mht
%distributions that are subdominant with respect to variations as a
%function of \njet, \nb, or \scalht, except for the uncertainties in
%the NLO QCD and electroweak corrections.

\begingroup
\renewcommand*{\arraystretch}{1.2}
\begin{table}[!t]
  \topcaption{
    Systematic uncertainties in the $\lost$ and $\znunu$ background
    evaluation. The quoted ranges are representative of the minimum
    and maximum variations observed across all bins of the signal
    region. Pairs of ranges are quoted for uncertainties determined
    from closure tests in data, which correspond to variations as a
    function of \njet and \scalht, respectively.
%    The two uncertainty ranges quoted determined from closure
%    tests in data correspond to variations as a function of \njet and
%    \scalht, respectively. 
  } 
  \label{tab:bkgd_systs}
  \centering
  %\small
  \begin{tabular}{ lcc }
    \hline
    Source of uncertainty               & \multicolumn{2}{c}{Magnitude [\%]}      \\
    \cline{2-3}
    & $\lost$            & $\znunu$           \\
    \hline
    Finite-size simulated samples       & 1--50              & 1--50              \\
%    \multicolumn{3}{l}{\bf Uncertainties in corrections applied to simulation:}   \\
    Minimum bias cross section (pileup) & 0.6--3.8           & 2.3--2.8           \\
    $\mu_R$ / $\mu_F$ scales            & 2.3--3.6           & 0.9--4.7           \\
    Parton density functions            & 1.1--2.7           & 0.0--3.3           \\
    \wj cross section                   & 0.2--1.4           & --                 \\
    \ttbar cross section                & 0.0--1.0           & --                 \\
    NLO QCD corrections                 & 1.5--13            & 2.6--17            \\
    NLO electroweak corrections         & 0.1--9.5           & 0.0--7.8           \\
    ISR (\ttbar)                        & 0.8--1.1           & --                 \\
    Signal trigger efficiency           & 0.0--3.1           & 0.0--2.0           \\
    Lepton efficiency (selection)       & 2.0                & 4.0                \\
    Lepton efficiency (veto)            & 5.0                & 5.0                \\
    Jet energy scale                    & 3.4--5.5           & 5.3--8.0           \\
    b quark tag efficiency              & 0.4--0.6           & 0.3--0.6           \\
    b quark mistag probability          & 0.1--1.4           & 0.2--1.8           \\
%    \multicolumn{3}{l}{\bf Uncertainties determined from closure tests in data:}  \\
%    \alphat extrapolation              & 3.3--9.4, 2.1--5.9 & 3.3--9.4, 2.1--5.9 \\
%    \bdphi extrapolation               & 2.7--22, 1.6--18   & 2.7--22, 1.6--18   \\
%    W boson polarization               & 0.9--6.6, 1.5--6.6 & --                 \\
    \alphat extrapolation               & 3--9, 2--6         & 3--9, 2--6         \\
    \bdphi extrapolation                & 3--22, 2--18       & 3--22, 2--18       \\
    W boson polarization                & 1--7, 2--7         & --                 \\
    Single isolated track veto          & 0--10, 0--13       & --                 \\
%    \mht modelling                     & 1--30              & 1--50              \\
    \hline
  \end{tabular}
\end{table}
\endgroup

Sources of additional uncertainties are determined from closure tests
performed using control data that aim to identify \njet- or
\scalht-dependent sources of systematic bias arising from
extrapolations in kinematical variables covered by the transfer
factors.
%Each closure test uses the observed event counts in up to eleven bins
%in \scalht, integrated over \njet, \nb, and \mht in the \mj and \mmj
%control samples to obtain a prediction of the observed yields in
%another control sample. The extrapolation is performed using transfer
%factors. Any nonclosure between the data count and prediction per
%\scalht bin is assigned as a systematic uncertainty, uncorrelated
%between \scalht bins and fully correlated across common bins in \njet,
%\nb, and \mht for the SR. The closure tests are repeated using the
%observed event counts in up to seven bins in \njet, integrated over
%\scalht, \nb, and \mht, to determine \njet-dependent systematic
%uncertainties that are employed using a similar correlation
%schema. 
Several sets of tests are performed. The accuracy of the modelling of
the efficiencies of both the \alphat and \bdphi requirements are
estimated from both the \mj and \mmj samples. The effects of W
polarization are probed by using \mj events with a positively charged
muon to predict those containing a negatively charged muon. Finally,
the efficiency of the single isolated track veto is also probed using
a sample of \mj events. The uncertainties are summarized in
Table~\ref{tab:bkgd_systs}.

The simulation modelling of the \nb distributions for the \znunuj
background in the region $\nb \geq 1$ is evaluated through a binned
maximum-likelihood fit to the observed \nb distributions in data in
each (\njet, \scalht) bin of the \mmj CR. Additional checks are
performed in \mmj samples that are enriched in mistagged jets
originating from gluons or LF quarks and genuine tags of b quarks from
gluon splitting through the use of loose and tight working points of
the b-tagging algorithm. No tests reveal evidence of significant bias
in the simulation modelling of the \nb variable.

Finally, the \mht modelling in simulated events is compared to the
distributions observed in \mj and \mmj control data, and inspected for
trends, by assuming a linear behaviour of the ratio of observed and
simulated counts as a function of \mht. Linear fits are performed
independently for each \njet category while integrating event counts
over \nb and \scalht, and the repeated for each \scalht bin while
integrating event counts over \njet and \nb. 
%, in a procedure analogous to that performed for the closure tests. 
Systematic uncertainties are determined from any nonclosure between
data and simulation as a function of \njet and are assumed to be
correlated in \scalht (and \nb), and vice versa. The uncertainties can
be as large as ${\approx}50\%$ in the most sensitive \mht bins.

%_______________________________________________________________________________
%_______________________________________________________________________________
%_______________________________________________________________________________

%\clearpage
\section{Multijet background evaluation}
\label{sec:qcd}

The multijet background is estimated using the three data sidebands
defined in Section~\ref{sec:control}.
%terms of the variables \mhtmet and \bdphi. The sidebands are defined
%as follows. The ``\mhtmet'' sideband comprises events that satisfy the
%SR selection criteria except for the inverted requirement $1.25 <
%\mhtmet < 3.0$. The ``\bdphi'' sideband is defined similarly, except
%for the inverted requirement $0.2 < \bdphi < 0.5$. Finally, a
%``double'' sideband comprises events that satisfy all SR selection
%criteria and the inverted requirements $1.25 < \mhtmet < 3.0$ and $0.2
%< \bdphi < 0.5$.
Events in each sideband are categorized according to \njet and
\scalht. The event counts in data are corrected to account for
contamination from nonmultijet SM processes, such as vector boson and
\ttbar production, plus residual contributions from other SM
processes. The nonmultijet processes are estimated from the \mj and
\mmj CRs, following a procedure similar to the one described in
Section~\ref{sec:ewk}. The corrected counts are assumed to arise
solely from multijet production.
%
For each sideband, a transfer factor per (\njet, \scalht) bin is
obtained from simulation, defined as the ratio of the number of
multijet events that satisfy the sideband requirement to the number
that fail this requirement. Estimates of the multijet background per
(\njet, \scalht) bin are obtained per sideband from the product of the
transfer factors and the corrected data counts.

The final estimate per (\njet, \scalht) bin is a weighted sum of the
three estimates. The multijet background is found to be small,
typically at the percent level, relative to the sum of all nonmultijet
backgrounds in all (\njet, \nb) bins of the SR.
%
The \mhtmet and \bdphi variables that are used to define the sidebands
are determined to be only weakly correlated for multijet events, and
the estimates from each sideband are assumed to be uncorrelated.
%
Statistical uncertainties associated with the finite event counts in
data and simulated event samples, as large as ${\approx}100\%$, are
propagated to each estimate. Uncertainties as large as ${\approx}20\%$
in the estimates of nonmultijet contamination
%, determined following the prescriptions described in
%Section~\ref{sec:ewk}, 
are also propagated to the corrected events. 
%
Any differences between the three estimates per (\njet, \scalht) bin
are adequately covered by systematic uncertainties of 100\%, which are
assumed to be uncorrelated across (\njet, \scalht) bins.

A model is assumed to determine the estimates as a function of \nb and
\mht. The distribution of multijet events as a function of \nb and
\mht per (\njet, \scalht) bin is assumed to be identical to the
distribution expected for the nonmultijet backgrounds. This assumption
is based on studies in simulation %, which can in principle lead to an
%over- (under-) estimate of multijet events at low (high) values of \nb
%and \mht. However, the approach 
and is a valid simplification given the magnitude of the multijet
background relative to the sum of all other SM backgrounds, as well as
the magnitude of the statistical and systematic uncertainties in the
estimates described above.

%_______________________________________________________________________________
%_______________________________________________________________________________
%_______________________________________________________________________________

%\clearpage
\section{Result}
\label{sec:result}

%A likelihood model of the observed data counts, in the 254 bins of the
%SR and the 320 bins of the \mj and \mmj CRs, is used to obtain the SM
%expectations in the SR and each CR, as well as to test for the
%presence of new-physics signals. 

A likelihood model is used to obtain the SM expectations in the SR and
each CR, as well as to test for the presence of new-physics
signals. The observed event count in each bin, defined in terms of the
\njet, \nb, \scalht, and \mht variables, is modelled as a
Poisson-distributed variable around the SM expectation and a potential
signal contribution (assumed to be zero in the following
discussion). The expected event counts from nonmultijet processes in
the SR are related to those in the \mj and \mmj CRs via
simulation-based transfer factors, as described in
Section~\ref{sec:ewk}. The systematic uncertainties in the nonmultijet
estimates, summarized in Table~\ref{tab:bkgd_systs}, are accommodated
in the likelihood model as nuisance parameters, the measurements of
which are assumed to follow a log-normal distribution. In the case of
\mht modelling, alternative templates are used to describe the
uncertainties in the \mht modelling and a vertical template morphing
schema~\cite{Prosper:2011zz, Khachatryan:2016dvc} is used to
interpolate between the nominal and alternative templates. The
multijet background estimates, determined using the method described
in Section~\ref{sec:qcd}, are also included in the likelihood model.

%\clearpage
\begin{figure}[!p]
  \centering
  \caption{Counts of candidate signal events (solid markers) and SM
    expectations with associated uncertainties (statistical and
    systematic, black histograms and shaded bands) as determined from
    the CR-only fit as a function of \nb, \scalht, and \mht for the
    event categories $\njet = 1$ and ${\geq}2a$ (upper \cmsLeft), $=2$
    (upper \cmsRight), $=3$ (middle \cmsLeft), $=4$ (middle
    \cmsRight), $=5$ (lower \cmsLeft), and ${\geq}6$ (lower
    \cmsRight). The centre and lower panels of each subfigure show the
    ratio and pull of event counts with respect to SM expectations.}
  \includegraphics[width=0.48\textwidth, trim=10 0 60 10, clip=true]{Figures/1jet_cr-only.pdf}~ 
  \includegraphics[width=0.48\textwidth, trim=10 0 60 10, clip=true]{Figures/2jet_cr-only.pdf}\\
  \includegraphics[width=0.48\textwidth, trim=10 0 60 10, clip=true]{Figures/3jet_cr-only.pdf}~
  \includegraphics[width=0.48\textwidth, trim=10 0 60 10, clip=true]{Figures/4jet_cr-only.pdf}\\
  \includegraphics[width=0.48\textwidth, trim=10 0 60 10, clip=true]{Figures/5jet_cr-only.pdf}~
  \includegraphics[width=0.48\textwidth, trim=10 0 60 10, clip=true]{Figures/6jet_cr-only.pdf}\\
  \label{fig:result}
\end{figure}

Figure~\ref{fig:result} summarizes the binned counts of candidate
signal events and the corresponding SM expectations as determined from
a ``CR-only'' fit that uses only the data counts in the \mj and \mmj
control regions to constrain the model parameters related to the
nonmultijet backgrounds. The uncertainties in the SM expectations
reflect both statistical and systematic contributions. The multijet
background estimates are determined independently and included in the
SM expectations. The fit does not consider the event counts in the
signal region. Figure~\ref{fig:result} also shows the ratios of the
event counts and SM expectations, as well as the significance of
deviations observed in data with respect to the SM expectations
expressed in terms of the total uncertainty in the SM expectations
(pull).

Hypothesis testing with regards to a potential signal contribution is
performed by considering a full fit to the event counts in the SR and
CRs. 
%A quantitative statement on the degree of compatibility between the
%observed event counts and the SM expectations under the
%SM-background-only (null) hypothesis is determined with a
%goodness-of-fit test using a one-sided (LHC-style) profile likelihood
%ratio~\cite{CMS-NOTE-2011-005} as the test statistic and a saturated
%model~\cite{sat-llk} as the alternative hypothesis. The distribution of
%the test statistic obtained from 1000 pseudo-experiments is
%characterized by a mean of 259, the observed value is 301, and the
%$p$-value is 4\%. Hence, no 
No significant tension is observed between the predictions and data in
the SR and CRs, and the data counts appear to be adequately modelled
by the SM expectations with no significant kinematical patterns.

Event counts, SM background estimates, and the associated correlation
matrix are also determined using the simplified 32-bin schema, which
can be found in \suppMaterial. 

%_______________________________________________________________________________
%_______________________________________________________________________________
%_______________________________________________________________________________

%\clearpage
\section{Interpretations}
\label{sec:interpretations}

The search result is used to constrain the parameter spaces of
simplified SUSY models~\cite{Alwall:2008ag, Alwall:2008va,
  sms}. Interpretations are provided for nine unique model families,
as summarized in Table~\ref{tab:sms}. Each family of models realizes a
unique production and decay mode. The model parameters are the masses
of the parent gluino ($m_\PSg$) or bottom, top, and LF squark
($m_{\PSQb}$, $m_{\PSQt}$, $m_{\PSQ}$) particles, also collectively
labelled as $m_\text{SUSY}$, and the \PSGczDo ($m_{\PSGczDo}$). Two
scenarios are considered for LF squarks: one with an eightfold mass
degeneracy for $\PSQ_\cmsSymbolFace{L}$ and $\PSQ_\cmsSymbolFace{R}$
with $\PSQ = \{\PSQu, \PSQd, \PSQs, \PSQc\}$ and the other with just a
single light squark ($\PSQu_\cmsSymbolFace{L}$). All other SUSY
particles are assumed to be too heavy to be produced directly. Gluinos
are assumed to undergo prompt three-body decays via highly virtual
squarks. In the case of split SUSY models (\texttt{T1qqqqLL}), the
gluino is assumed to be long-lived with proper decay lengths in the
range $10^{-3} < \ctau < 10^{5}\unit{mm}$. A scenario involving a
metastable gluino with $\ctau = 10^{18}\unit{mm}$ is also considered.

Under the signal+background hypothesis, and in the presence of a
nonzero signal contribution, a modified frequentist approach is used
to determine observed upper limits at 95\% confidence level (CL) on
the cross section $\sigma_\text{UL}$ to produce pairs of SUSY
particles as a function of $m_\text{SUSY}$, $m_{\PSGczDo}$, and \ctau
(if applicable). The approach is based on the %aforementioned
profile likelihood ratio as the test
statistic~\cite{CMS-NOTE-2011-005}, the \cls criterion~\cite{junk,
  read}, and asymptotic formulae~\cite{Cowan:2010js} to approximate
the distributions of the test statistic under the SM-background-only
and signal+background hypotheses.  An Asimov data
set~\cite{Cowan:2010js} is used to determine the expected
$\sigma_\text{UL}$ on the allowed cross section for a given
model. Potential signal contributions to event counts in all bins of
the SR and CRs are considered.%, even though the only
%significant contribution occurs in the SR. 

\begingroup
\renewcommand*{\arraystretch}{1.2}
\begin{table}[!t]
  \topcaption{Summary of the simplified SUSY models used to
    interpret the result of this search.} 
  \label{tab:sms}
  \centering
  \begin{tabular}{ lll }
    \hline
    Model family
    & Production and decay
    & Additional assumptions                                                         \\
    \hline
    \multicolumn{3}{l}{\bf Production and prompt decay of squark pairs}           \\
    \texttt{T2bb}
    & $\Pp\Pp \to \PSQb\PASQb$,
    $\PSQb \to \cPqb\PSGczDo$
    & --                                                                             \\
    \texttt{T2tt}
    & $\Pp\Pp \to \PSQt\PASQt$,
    $\PSQt \to \cPqt\PSGczDo$
    & --                                                                             \\
    \texttt{T2cc}
    & $\Pp\Pp \to \PSQt\PASQt$,
    $\PSQt\to \cPqc\PSGczDo$
    & $10 < m_{\,\PSQt} - m_{\PSGczDo} < 80\GeV$                                     \\
    \texttt{T2qq\_8fold}
    & $\Pp\Pp \to \PSQ\PASQ$,
    $\PSQ \to \cPq\PSGczDo$
    & $m_{\PSQ_\cmsSymbolFace{L}} = m_{\PSQ_\cmsSymbolFace{R}}$,
    $\PSQ = \{ \PSQu, \PSQd, \PSQs, \PSQc \}$                                     \\
    \texttt{T2qq\_1fold}
    & $\Pp\Pp \to \PSQ\PASQ$,
    $\PSQ \to \cPq\PSGczDo$
    & $m_{\PSQ (\PSQ \neq \PSQu_\cmsSymbolFace{L})} \gg m_{\PSQu_\cmsSymbolFace{L}}$ \\
    \multicolumn{3}{l}{\bf Production and prompt decay of gluino pairs}           \\
    \texttt{T1bbbb}
    & $\Pp\Pp \to \PSg\PSg$,
    $\PSg\to \cPaqb\PSQb^* \to \cPaqb\cPqb\PSGczDo$
    & $m_{\PSQb} \gg m_{\PSg}$                                                       \\
    \texttt{T1tttt}
    & $\Pp\Pp \to \PSg\PSg$,
    $\PSg\to \cPaqt\PSQt^* \to \cPaqt\cPqt\PSGczDo$                                                                   
    & $m_{\PSQt} \gg m_{\PSg}$                                                       \\
    \texttt{T1qqqq}
    & $\Pp\Pp \to \PSg\PSg$,
    $\PSg\to \cPaq\PSQ^* \to \cPaq\cPq\PSGczDo$                                                                   
    & $m_{\PSQ} \gg m_{\PSg}$                                                        \\
    \multicolumn{3}{l}{\bf Production and decay of long-lived gluino pairs}       \\
    \texttt{T1qqqqLL}
    & $\Pp\Pp \to \PSg\PSg$,
    $\PSg \to \cPaq\PSQ^* \to \cPaq\cPq\PSGczDo$    
    & $m_{\PSQ} \gg m_{\PSg}$, $10^{-3} < \ctau < 10^{5}\unit{mm}$ or metastable    \\
    \hline
  \end{tabular}
\end{table}
\endgroup 

The experimental acceptance times efficiency (\ate) is evaluated
independently for each model, defined in terms of $m_\text{SUSY}$,
$m_{\PSGczDo}$, and \ctau (if applicable). 
%Several sources of uncertainty in \ate are considered. 
The effects of several sources of uncertainty in \ate, as well as the
potential for migration of events between bins of the SR, are
considered. Correlations are taken into account where appropriate,
including those relevant to signal contamination that may contribute
to counts in the CRs. 
%The magnitude of each source of uncertainty is dependent on the model
%and its parameters.

%One or more of the following sources of uncertainty typically dominate
%in any given region of the model parameter space.
The statistical uncertainty arising from the finite size of simulated
samples can be as large as ${\approx}30\%$. The \ate for models with a
compressed mass spectrum relies on jets arising from ISR, the
modelling of which is evaluated 
%by comparing the simulated and measured \pt
%spectra of the system recoiling against the ISR jets in \ttbar events,
using the technique described in Ref.~\cite{Chatrchyan:2013xna}. The
associated uncertainty can be as large as ${\approx}30\%$.
%for models with a compressed mass spectrum. 
The corrections to the jet energy scale evaluated with simulated
events can lead to variations in event counts as large as
${\approx}25\%$ for models yielding high jet multiplicities. 
%characterized by high jet multiplicities in the final state. 
The uncertainties in the modelling of scale factors applied to
simulated event samples that correct for differences in the b-tag
efficiency
%($\text{SF}_\text{b}$) 
and mistag probability of LF partons and charm quarks
%($\text{SF}_\text{LF}$), evaluated independently, 
can be as large as ${\approx}20\%$.

Table~\ref{tab:benchmarks} defines a number of benchmark models that
are close to the limit of the search sensitivity. %, which are used to
%develop an understanding of the expected parameter-space coverage. 
All model families %(of production and decay modes)
are represented, and the model parameters ($m_\text{SUSY}$,
$m_{\PSGczDo}$, and \ctau if applicable) are chosen to select models
with large and small differences in $m_\text{SUSY}$ and
$m_{\PSGczDo}$, as well as a range of \ctau values. 
%prompt-like, b-quark-like, and stable-like decay lengths. 
Table~\ref{tab:benchmarks} summarizes the aforementioned uncertainties
for each benchmark model, presented in terms of a characteristic range
that is representative of the variations observed across the bins of
the SR. The upper bound for each range may be subject to moderate
statistical fluctuations.

\begingroup
\renewcommand*{\arraystretch}{1.1}
\begin{table}[!t]
  \topcaption{A list of benchmark simplified models organized
    according to production and decay modes (family), %a representative
    %(\njet, \nb) topology that indicates a sensitive region for each
    %model, 
    the \ate, %this topology, 
    representative values for some of the dominant sources of 
    systematic uncertainty, and the expected and observed upper limits
    on the production cross section, expressed in terms of the signal
    strength parameter ($\mu$). Additional uncertainties concerning
    the \texttt{T1qqqqLL} models are not listed here and are discussed
    in the text.   
  }
  \label{tab:benchmarks}
  \centering
  \resizebox{\textwidth}{!}{
    \begin{tabular}{ lrcrrrrrcc }
      \hline
      Family
        & $(m_{\text{SUSY}}, m_{\PSGczDo})$
      % & Sensitive%(\njet, \nb)
        & \ate
        & \multicolumn{4}{c}{Systematic uncertainties [\%]}
        & \multicolumn{2}{c}{$\mu$ (95\% CL)}                \\ [0.3ex]
      \cline{4-7}
      (\ctau)
        & [\GeVns{}]
      % & topology
        & [\%]
        & MC stat.
        & ISR
        & JEC
        & b-tag %$\text{SF}_\text{b}$
%       & $\text{SF}_\text{LF}$
        & Exp.
        & Obs.                                               \\ [0.3ex]
      \hline
      \multirow{2}{*}{\texttt{T2bb}}
        & (800, 50)    % & (${\geq}6, {\geq}2$)  
        & 40.1           & 14--23                & 1--7   & 4--11  & 1--4  % & --    
        & 0.62           & 0.67                              \\
        & (375, 300)   % & (${\geq}6, {\geq}2$)  
        & \phantom{1}5.7 & 9--22                 & 4--15  & 4--15  & 3--7  % & --    
        & 0.76           & 1.21                              \\ [0.5ex]
      \multirow{3}{*}{\texttt{T2tt}}
        & (1000, 50)   % & (${\geq}6, {\geq}2$)  
        & 23.8           & 14--27                & 3--7   & 4--14  & 1--5  % & --    
        & 0.82           & 0.85                              \\
        & (450, 200)   % & (${\geq}6, {\geq}2$)  
        & \phantom{1}4.2 & 6--19                 & 4--12  & 6--15  & 4--9  % & --    
        & 0.56           & 0.73                              \\ [0.5ex]
        & (250, 150)   % & (${\geq}6, {\geq}2$)  
        & \phantom{1}0.3 & 10--23                & 13--27 & 8--22  & 6--16 % & --    
        & 0.71           & 0.66                              \\ [0.5ex]
      \multirow{1}{*}{\texttt{T2cc}}
        & (500, 480)   % & (${\geq}6, {\geq}2$)  
        & 20.5           & 6--19                 & 4--18  & 5--13  & 1--4  % & --    
        & 0.68           & 1.38                              \\ [0.5ex]
      \multirow{2}{*}{\texttt{T2qq\_8fold}}
        & (1250, 100)  % & (${\geq}6, {\geq}2$)  
        & 42.9           & 12--24                & 2--7   & 5--14  & 1--1  % & --    
        & 0.54           & 0.66                              \\
        & (700, 600)   % & (${\geq}6, {\geq}2$)  
        & \ph{1}7.7      & 6--22                 & 4--17  & 4--13  & 2--5  % & --    
        & 0.75           & 1.13                              \\ [0.5ex]
      \multirow{2}{*}{\texttt{T2qq\_1fold}}
        & (700, 100)   % & (${\geq}6, {\geq}2$)  
        & 32.9           & 4--22                 & 2--7   & 3--10  & 0--5  % & --    
        & 0.60           & 0.88                              \\
        & (400, 300)   % & (${\geq}6, {\geq}2$)  
        & \ph{1}4.5      & 6--20                 & 5--22  & 5--18  & 3--5  % & --    
        & 0.61           & 0.46                              \\ [0.5ex]
      \multirow{2}{*}{\texttt{T1bbbb}}
        & (1900, 100)  % & (${\geq}6, {\geq}2$)  
        & 25.1           & 11--19                & 3--9   & 4--6   & 7--11 % & --    
        & 0.56           & 1.25                              \\
        & (1300, 1100) % & (${\geq}6, {\geq}2$)  
        & 14.6           & 11--22                & 2--11  & 3--11  & 2--5  % & --    
        & 0.44           & 1.15                              \\ [0.5ex]
      \multirow{2}{*}{\texttt{T1tttt}}
        & (1700, 100)  % & (${\geq}6, {\geq}2$)  
        & \phantom{1}6.9 & 12--24                & 2--6   & 3--15  & 2--6  % & --    
        & 0.51           & 1.31                              \\
        & (950, 600)   % & (${\geq}6, {\geq}2$)  
        & \phantom{1}0.3 & 15--30                & 5--9   & 12--26 & 2--6  % & --    
        & 0.89           & 1.51                              \\ [0.5ex]
      \texttt{T1qqqqLL}
        & (1800, 200)  % & (${\geq}6, {\leq}1$)  
        & 27.8           & 8--20                 & 3--5   & 3--9   & 0--1  % & 2--9  
        & 1.02           & 1.91                              \\
      ($10^{-3}\unit{mm}$)
        & (1000, 900)  % & (${\leq}2a, {\leq}1$) 
        & \ph{1}6.7      & 15--21                & 2--10  & 4--14  & 0--1  % & 0--1  
        & 0.68           & 1.26                              \\ [0.5ex]
      \texttt{T1qqqqLL}
        & (1800, 200)  % & (${\geq}6, {\geq}2$)  
        & 22.9           & 11--20                & 2--5   & 3--9   & 17--59 % & 3--14 
        & 0.43           & 1.00                              \\
      ($1\unit{mm}$)
        & (1000, 900)  % & (${\leq}2a, {\leq}1$) 
        & \ph{1}5.2      & 17--26                & 2--9   & 4--17  & 10--41  % & 8--13 
        & 0.74           & 1.58                              \\ [0.5ex]
      \texttt{T1qqqqLL}
        & (1000, 200)  % & (4--5, ${\leq}1$)     
        & 11.2           & 16--22                & 2--14  & 4--9  & 0--1  % & 1--1  
        & 0.74           & 1.58                              \\
      ($100\,000\unit{mm}$)
        & (1000, 900)  % & (${\leq}2a, {\leq}1$) 
        & 10.4           & 14--26                & 3--14  & 2--12  & 0--1  % & 1--1  
        & 0.63           & 0.45                              \\ [0.5ex]
%                                                            \\
%      \texttt{T1qqqqLL}
%       & (1800, 200)    & (${\geq}6, {\leq}1$)  & \ph{1}0.1 \\
%       & (1000, 900)    & (${\leq}2a, {\leq}1$) & \ph{1}0.5 \\
%      \texttt{T1qqqqLL}
%       & (1800, 200)    & (${\geq}4, {\geq}2$)  & \ph{1}0.1 \\
%       & (1000, 900)    & (${\leq}2a, {\leq}1$) & \ph{1}0.6 \\
%      \texttt{T1qqqqLL}
%       & (1000, 200)    & (4--5, ${\leq}1$)     & \ph{1}0.5 \\
%       & (1000, 900)    & (${\leq}2a, {\leq}1$) & \ph{1}0.6 \\
      \hline
    \end{tabular}
  }
\end{table}
\endgroup

Additional subdominant contributions to the total uncertainty are also
considered. The uncertainty in the integrated luminosity is determined
to be 2.5\%~\cite{CMS:2017sdi}. Uncertainties in the production cross
section arising from the choice of PDF set, and variations therein, as
well as variations in $\mu_\text{R}$ and $\mu_\text{F}$ at LO are
considered. Uncertainties in event migration between bins from
variations in the PDF sets are assumed to be correlated with, and
adequately covered by, the uncertainties in the modelling of
ISR. Uncertainties from $\mu_\text{R}$ and $\mu_\text{F}$ variations
are determined to be ${\approx}5\%$. The effect of a 5\% uncertainty
in the total inelastic cross section~\cite{Aaboud:2016mmw} is
propagated through the reweighting procedure that corrects for
differences between the simulated and measured pileup, resulting in
event-count variations of ${\approx}10\%$. Uncertainty in the
modelling of the efficiency to identify high-quality, isolated leptons
is ${\approx}5\%$ and is treated as anticorrelated between the SR and
\mj and \mmj CRs. The uncertainty in the trigger efficiency to record
candidate signal events is ${<}10\%$.

The \ate for the \texttt{T1qqqqLL} family of models depends on \ctau
in addition to $m_\PSg$ and $m_{\PSGczDo}$. Scenarios involving a
compressed mass spectrum or gluinos with $\ctau \gtrsim
10\,000\unit{mm}$ increase the probability that the decay of the
gluino-pair system escapes detection and the \ate is reduced for such
models, as indicated in Table~\ref{tab:benchmarks}, because of an
increased reliance on jets from ISR. Scenarios with $m_\PSg -
m_{\PSGczDo} \gtrsim 100\GeV$ and $1000 \lesssim \ctau \lesssim
10\,000\unit{mm}$ often lead to one or both gluinos decaying within
the calorimeter systems to yield energetic jets comprising particle
candidates that have no associated charged-particle track. Hence, the
efficiencies for the event vetoes related to the jet identification
and $f_{h^{\pm}}^{\jet{1}}$ requirements, described in
Sections~\ref{sec:reconstruction} and \ref{sec:baseline}, can be as
low as ${\approx}90\%$ and ${\approx}30\%$, respectively, for this
region of the model parameter space. Uncertainties as large as
${\approx}10\%$ are assumed. %, motivated in part by studies of the
%simulation modelling of the $f_{h^{\pm}}^{\jet{1}}$ variable in data
%control regions. 
The efficiencies %of the aforementioned event vetoes 
for all other scenarios are typically ${\approx}100\%$. Jet
identification requirements in the trigger logic lead to
inefficiencies and uncertainties not larger than 2\%. Finally, models
with $1 \lesssim \ctau \lesssim 10\unit{mm}$ often lead to jets that
are tagged by the CSV algorithm with efficiencies as high as
${\approx}60\%$, which are comparable to the values obtained for jets
originating from b quarks. Uncertainties of 20--50\% in the tagging
efficiency are assumed to cover differences with respect to jets
originating from b quarks, as indicated in
Table~\ref{tab:benchmarks}. 
%, in addition to uncertainties in $\text{SF}_\text{b}$.

Figure~\ref{fig:limits-sms} summarizes the excluded regions of the
mass parameter space for the nine families of simplified models. The
regions are determined by comparing $\sigma_\text{UL}$
%the upper limits on the measured fiducial cross section, corrected
%for the experimental \ate,  
with the theoretical cross sections calculated at NLO+NLL accuracy. 
%as described in Section~\ref{sec:simulation}. 
The former value is determined as a function of $m_{\text{SUSY}}$ and
$m_{\PSGczDo}$, while the latter has a dependence solely on
$m_{\text{SUSY}}$. The exclusion of models is evaluated using observed
data counts in the signal region (solid contours) and also expected
counts based on an Asimov data set (dashed contours). 
%
The observed excluded regions for the \texttt{T1bbbb} and
\texttt{T1tttt} families, as shown in Fig.~\ref{fig:limits-sms}
(\cmsRight), can be up to 2--3 standard deviations weaker than the
expected excluded regions when $m_{\PSg} - m_{\PSGczDo} \approx
350\GeV$. These differences are typically due to fluctuations in data
for events that satisfy $\njet \geq 5$ and $\njet \geq
3$. Figure~\ref{fig:limits-sms} (\cmsRight) also allows a comparison
of the sensitivity to \texttt{T1qqqq} and \texttt{T1qqqqLL} models,
which assume the prompt-decay and metastable gluino scenarios,
respectively. The latter scenario leads to a monojet-like final state as
the gluino escapes detection, resulting in a reach in $m_\PSg$ that is
independent of $m_{\PSGczDo}$.

Figure~\ref{fig:limits-ll} summarizes the evolution of the search
sensitivity to the \texttt{T1qqqqLL} family of models as a function of
\ctau. Each subfigure presents the observed $\sigma_\text{UL}$ as a
function of $m_{\PSg}$ and $m_{\PSGczDo}$ for simplified models that
involve the production of gluino pairs. The excluded mass regions
based on the observed and expected values of $\sigma_\text{UL}$ are
also shown, along with contours determined under variations in
theoretical and experimental uncertainties. The top row of subfigures
cover the range $0.001 < \ctau < 0.1\unit{mm}$ and demonstrate
coverage comparable to the \texttt{T1qqqq} prompt-decay scenario. A
moderate improvement in sensitivity for models with $1 \lesssim \ctau
\lesssim 10\unit{mm}$ is observed because of the additional
signal-to-background discrimination provided by the \nb variable. The
sensitivity reduces for models with lifetimes in the region $\ctau >
100\unit{mm}$ because of a reduced acceptance for the jets from the
gluino decay and an increased reliance on jets from ISR. The coverage
is independent of \ctau beyond values of $10\,000\unit{mm}$ and
comparable to the limiting case of a metastable gluino. 

\begin{figure}[!t]
  \centering
  \includegraphics[width=0.6\textwidth]{Figures/squarkSUMMARY.pdf}\\
  \includegraphics[width=0.6\textwidth]{Figures/gluinoAllSUMMARY.pdf}\\
  \caption{Observed and expected mass exclusions at 95\% CL
    (indicated, respectively, by solid and dashed contours) for
    various families of simplified models. 
    % 
    (\cmsLeft) Five model families involve the direct pair
    production of squarks. The first scenario considers the pair
    production and decay of bottom squarks (\texttt{T2bb}). Two
    scenarios involve the production and decay of top squark pairs
    (\texttt{T2tt} and \texttt{T2cc}). The grey shaded region denotes
    \texttt{T2tt} models that are not considered for
    interpretation. Two further scenarios involve, respectively, the 
    production and decay of light-flavour squarks, with different
    assumptions on the mass degeneracy of the squarks as described in
    the text (\texttt{T2qq\_8fold} and \texttt{T2qq\_1fold}). 
    % 
    (\cmsRight) Three scenarios involve the production and prompt
    decay of gluino pairs via virtual squarks (\texttt{T1bbbb},
    \texttt{T1tttt}, and \texttt{T1qqqq}). A final scenario involves
    the production of gluinos that are assumed to be metastable on the
    detector scale (\texttt{T1qqqqLL}).}
  \label{fig:limits-sms} 
\end{figure}

%\begin{figure}[!h]
%  \centering
%  \includegraphics[width=0.6\textwidth]{Figures/T1qqqqLLPromptXSEC.pdf}\\
%  \includegraphics[width=0.6\textwidth]{Figures/T1qqqqLLStableXSEC.pdf}\\
%  \caption{ Observed upper limit in cross section at 95\% CL
%    (indicated by the colour scale) as a function of the gluino and
%    \PSGczDo masses for simplified models that assume the production
%    of pairs of gluino particles that (\cmsLeft) each decay promptly
%    via virtual light-flavour squarks to the neutralino and SM
%    particles (\texttt{T1qqqq}) or (\cmsRight) are stable
%    (\texttt{T1qqqqLL}). The black solid thick (thin) line indicates
%    the observed excluded region assuming the nominal (${\pm}1$ standard
%    deviation in theory uncertainty) production cross section. The red
%    dashed thick (thin) line indicates the median (${\pm}1$ standard
%    deviation in experimental uncertainty) expected excluded region.
%  }
%  \label{fig:limits-stable} 
%\end{figure}

%\clearpage
\begin{figure}[!t]
  \centering
  \includegraphics[width=0.33\textwidth]{Figures/T1qqqqLL0p001XSEC}~
  \includegraphics[width=0.33\textwidth]{Figures/T1qqqqLL0p01XSEC}~
  \includegraphics[width=0.33\textwidth]{Figures/T1qqqqLL0p1XSEC}\\
  \includegraphics[width=0.33\textwidth]{Figures/T1qqqqLL1XSEC}~
  \includegraphics[width=0.33\textwidth]{Figures/T1qqqqLL10XSEC}~
  \includegraphics[width=0.33\textwidth]{Figures/T1qqqqLL100XSEC}\\
  \includegraphics[width=0.33\textwidth]{Figures/T1qqqqLL1000XSEC}~
  \includegraphics[width=0.33\textwidth]{Figures/T1qqqqLL10000XSEC}~
  \includegraphics[width=0.33\textwidth]{Figures/T1qqqqLL100000XSEC}\\
  \caption{Observed upper limit in cross section at 95\% CL (indicated
    by the colour scale) as a function of the $\PSg$ and \PSGczDo
    masses for simplified models that assume the production of pairs
    of long-lived $\PSg$ particles that each decay via highly virtual
    light-flavour squarks to the neutralino and SM particles
    (\texttt{T1qqqqLL}). Each subfigure represents a different gluino
    lifetime: 
    $\ctau = 10^{-3}\unit{mm}$ (upper left),
    $10^{-2}\unit{mm}$ (upper centre),
    $10^{-1}\unit{mm}$ (upper right),
    $1\unit{mm}$ (middle left),
    $10\unit{mm}$ (middle centre),
    $100\unit{mm}$ (middle right),
    $1000\unit{mm}$ (lower left),
    $10\,000\unit{mm}$ (lower centre), 
    and $100\,000\unit{mm}$ (lower right). 
%    The contours indicate the excluded regions as defined in the
%    caption of Fig.~\ref{fig:limits-stable}. 
    % 
    The black solid thick (thin) line indicates the observed excluded
    region assuming the nominal (${\pm}1$ standard deviation in theory
    uncertainty) production cross section. The red dashed thick (thin)
    line indicates the median (${\pm}1$ standard deviation in
    experimental uncertainty) expected excluded region.  
  }
  \label{fig:limits-ll} 
\end{figure} 

%\clearpage
Table~\ref{tab:limits} summarizes the strongest expected and observed
mass limits for each family of simplified models. The simplified
result based on the 32-bin schema, summarized in \suppMaterial, yields
limits on $\sigma_\text{UL}$ that are typically a factor ${\approx}2$
weaker than those obtained with the nominal result.

\begin{table}[!h]
  \topcaption{Summary of the mass limits obtained for each family of
    simplified models. The limits indicate the strongest observed 
    mass exclusions for the parent SUSY particle (gluino or 
    squark) and \PSGczDo.
    % The quoted values have uncertainties of ${\pm}25$ and
    % ${\pm}10\GeV$ for models involving the pair production of,
    % respectively, gluinos and squarks.
  }
  \label{tab:limits}
  \centering
  \begin{tabular}{ lcc }
    \hline
    Model family                             & \multicolumn{2}{c}{Best mass limit [GeV]} \\ [0.5ex]
    \cline{2-3}
                                             & Gluino or squark & \PSGczDo               \\ [0.5ex]
    \hline
    \texttt{T2bb}                            & 1050             & \ph{1}500              \\
    \texttt{T2tt}                            & 1000             & \ph{1}400              \\
    \texttt{T2cc}                            & \ph{1}500        & \ph{1}475              \\
    \texttt{T2qq\_8fold}                     & 1325             & \ph{1}575              \\
    \texttt{T2qq\_1fold}                     & \ph{1}675        & \ph{1}350              \\
    \texttt{T1bbbb}                          & 1900             & 1150                   \\
    \texttt{T1tttt}                          & 1650             & \ph{1}850              \\
    \texttt{T1qqqq}                          & 1650             & \ph{1}900              \\
    \texttt{T1qqqqLL} (Metastable $\PSg$)    & \ph{1}900        & --                     \\
    \texttt{T1qqqqLL} ($\ctau = 1\unit{mm}$) & 1750             & 1000                   \\
    \hline
  \end{tabular}
\end{table}

%_______________________________________________________________________________
%_______________________________________________________________________________
%_______________________________________________________________________________

%\clearpage
\section{Summary}
\label{sec:summary}

A search for supersymmetry with the CMS experiment is reported, based
on a data sample of pp collisions collected in 2016 at $\sqrt{s} =
13\TeV$ that corresponds to an integrated luminosity of $35.9 \pm 0.9
\fbinv$. Final states with jets and significant \ptvecmiss, as
expected from the production and decay of massive gluinos and squarks,
are considered. 
%Several kinematical variables are used to suppress the multijet
%background to a subdominant level with respect to all other standard
%model backgrounds. 
Candidate signal events are categorized according to the number of
reconstructed jets, the number of jets identified to originate from
bottom quarks, and the scalar and vector sums of the transverse
momenta of jets. %to provide signal-to-background discrimination.
The standard model background is estimated from a binned
likelihood fit to event yields in the signal region and data control
samples. The observed yields in the signal region are found to be in
agreement with the expected contributions from standard model
processes. Supplemental material is provided to aid futher
interpretation of the result in \suppMaterial.

Limits are determined in the parameter spaces of simplified models
that assume the production and prompt decay of gluino or squark
pairs. The strongest exclusion bounds (95\% CL) for squark masses are
1050, 1000, and 1325\GeV for bottom, top, and mass-degenerate
light-flavour squarks, respectively. The corresponding mass bounds on
the neutralino \PSGczDo from squark decays are 500, 400, and
575\GeV. The gluino mass is probed up to 1900, 1650, and 1650\GeV when
the gluino decays via virtual states of the aforementioned
squarks. The strongest mass bound on the \PSGczDo from the gluino
decay is 1150\GeV.

Sensitivity to simplified models inspired by split supersymmetry is
also demonstrated. These models assume the production of long-lived
gluino pairs that decay to final states containing displaced
jets and missing transverse momentum from the undetected \PSGczDo
particles. No specialized techniques of event reconstruction that
target long-lived gluinos are employed. Models that assume a \PSGczDo
mass of 100\GeV and gluino masses up to 1600\GeV are excluded for a proper
decay length \ctau below 0.1\unit{mm}. The bound on the gluino mass
strengthens to 1750\GeV at $\ctau = 1\unit{mm}$, before weakening to
900--1000\GeV for models with $\ctau > 10\,000\unit{mm}$ or metastable
gluinos.
For all values of \ctau considered, the exclusion bounds on the gluino
mass weaken to about 1\TeV when the difference between the gluino and
\PSGczDo mass is greater than 10\GeV.
The
search provides coverage that is complementary to other dedicated
techniques at the LHC, such as for models with long-lived gluinos with
$\ctau \lesssim 1\unit{mm}$.

%_______________________________________________________________________________
%_______________________________________________________________________________
%_______________________________________________________________________________

\clearpage
\bibliography{auto_generated}

%_______________________________________________________________________________
%_______________________________________________________________________________
%_______________________________________________________________________________

\clearpage
\appendix
\section{Supplemental material\label{app:suppMat}}
\begingroup
\renewcommand*{\arraystretch}{1.2}
\newcommand{\tmp}{\phantom{, 200}}
\begin{table}[!h]
  \topcaption{Summary of the nominal (\njet, \nb, \scalht, \mht)
    binning schema. Each entry (and the following entry, if present)
    signifies the lower (upper) bound of an \mht bin within a given
    (\njet, \nb, \scalht) bin. Unique or final entries represent \mht
    bins unbounded from above. A dash (--) signifies that the \scalht 
    bin in a given (\njet, \nb) category is not used in the analysis,
    in which case counts in high-\scalht bins are integrated into the
    adjacent lower-\scalht bin. For monojet events, $\scalht \equiv
    \mht$. 
  }
  \label{tab:binning}
  \centering
  \resizebox{\textwidth}{!}{
    \begin{tabular}{rrlllll}
      \hline
      \njet      & \nb       & \multicolumn{5}{c}{\scalht [GeV]}                                        \\ 
      \cline{3-7}
                 &           & 200 & 400      & 600           & 900                & 1200               \\
      \hline
      1          & 0         & 200 & 400 \tmp & 600 \tmp \tmp & 900 \tmp \tmp \tmp & --                 \\ 
      1          & 1         & 200 & 400 \tmp & 600 \tmp \tmp & --                 & --                 \\ 
      ${\geq}2a$ & 0         & 200 & 200, 400 & 200, 400, 600 & 200, 900 \tmp \tmp & --                 \\ 
      ${\geq}2a$ & 1         & 200 & 200, 400 & 200, 400, 600 & 200, 900 \tmp \tmp & --                 \\ 
      ${\geq}2a$ & 2         & 200 & 200, 400 & 200, 400, 600 & 200, 900 \tmp \tmp & --                 \\ 
      ${\geq}2a$ & ${\geq}3$ & 200 & 200, 400 & 200, 400, 600 & --                 & --                 \\ 
      2          & 0         & 200 & 200, 400 & 200, 400, 600 & 200, 400, 600, 900 & 200, 400, 600, 900 \\ 
      2          & 1         & 200 & 200, 400 & 200, 400, 600 & 200, 400, 600, 900 & 200, 400, 600, 900 \\ 
      2          & 2         & 200 & 200, 400 & 200, 400, 600 & --                 & --                 \\ 
      3          & 0         & 200 & 200, 400 & 200, 400, 600 & 200, 400, 600, 900 & 200, 400, 600, 900 \\ 
      3          & 1         & 200 & 200, 400 & 200, 400, 600 & 200, 400, 600, 900 & 200, 400, 600, 900 \\ 
      3          & 2         & 200 & 200, 400 & 200, 400, 600 & 200, 400, 600, 900 & 200, 400, 600, 900 \\ 
      3          & 3         & 200 & 200, 400 & 200, 400, 600 & --                 & --                 \\ 
      4          & 0         & --  & 200, 400 & 200, 400, 600 & 200, 400, 600, 900 & 200, 400, 600, 900 \\ 
      4          & 1         & --  & 200, 400 & 200, 400, 600 & 200, 400, 600, 900 & 200, 400, 600, 900 \\ 
      4          & 2         & --  & 200, 400 & 200, 400, 600 & 200, 400, 600, 900 & 200, 400, 600, 900 \\ 
      4          & ${\geq}3$ & --  & 200, 400 & 200, 400, 600 & 200, 400, 600, 900 & --                 \\ 
      5          & 0         & --  & 200, 400 & 200, 400, 600 & 200, 400, 600 \tmp & 200, 400, 600, 900 \\ 
      5          & 1         & --  & 200, 400 & 200, 400, 600 & 200, 400, 600 \tmp & 200, 400, 600, 900 \\ 
      5          & 2         & --  & 200, 400 & 200, 400, 600 & 200, 400, 600 \tmp & 200, 400, 600, 900 \\ 
      5          & 3         & --  & 200, 400 & 200, 400, 600 & 200, 400, 600 \tmp & --                 \\ 
      5          & ${\geq}4$ & --  & 200, 400 & --            & --                 & --                 \\ 
      ${\geq}6$  & 0         & --  & 200 \tmp & 200, 400 \tmp & 200, 400, 600 \tmp & 200, 400, 600, 900 \\ 
      ${\geq}6$  & 1         & --  & 200 \tmp & 200, 400 \tmp & 200, 400, 600 \tmp & 200, 400, 600, 900 \\ 
      ${\geq}6$  & 2         & --  & 200 \tmp & 200, 400 \tmp & 200, 400, 600 \tmp & 200, 400, 600, 900 \\ 
      ${\geq}6$  & 3         & --  & 200 \tmp & 200, 400 \tmp & 200, 400, 600 \tmp & 200, 400, 600, 900 \\ 
      ${\geq}6$  & ${\geq}4$ & --  & 200 \tmp & --            & --                 & --                 \\ 
      \hline
    \end{tabular}
  }
\end{table}
\endgroup

\begingroup
\renewcommand*{\arraystretch}{1.1}
\begin{table}[!t]
  \topcaption{Observed counts of candidate signal events and SM
    expectations determined from the CR-only fit using the simplified
    binning schema, as a function of \njet, \nb, and \mht. All counts
    are integrated over \scalht. The uncertainties include both
    statistical and systematic contributions.  
  }
  \label{tab:simplified}
  \centering
  \begin{tabular}{rrlr@{}lr@{}lr@{}lr@{}l}
    \hline
    \njet           & \nb       &      & \multicolumn{8}{c}{\mht [GeV]}                                                                             \\
    \cline{4-11}
                    &           &      & 200        &                   & 400       &                & 600   &               & 900   &              \\
    \hline
%    =1, ${\geq}2a$  & 0         & Data & 411\,184   &                   & 11\,448   &                & 1116  &               & 111                  \\
%                    &           & SM   & $360\,000$ & $\,\pm\, 35\,000$ & $9990$    & $\,\pm\, 890$  & $910$ & $\,\pm\, 300$ & $107$ & $\,\pm\, 64$ \\[0.2ex]
%    =1, ${\geq}2a$  & ${\geq}1$ & Data & 31\,174    &                   & 769       &                & 96    &               & 16                   \\
%                    &           & SM   & $25\,500$  & $\,\pm\, 2300$    & $649$     & $\,\pm\, 61$   & $65$  & $\,\pm\, 20$  & $11$  & $.4 \pm 6.7$ \\[0.2ex]
%    =2, =3          & =0, =1    & Data & 66\,955    &                   & 5946      &                & 903   &               & 100                  \\
%                    &           & SM   & $58\,000$  & $\,\pm\, 10\,000$ & $5410$    & $\,\pm\, 970$  & $860$ & $\,\pm\, 330$ & $113$ & $\,\pm\, 76$ \\[0.2ex]
%    =2, =3          & ${\geq}2$ & Data & 1045       &                   & 70        &                & 6     &               & 0                    \\
%                    &           & SM   & $870$      & $\,\pm\, 130$     & $56$      & $.9 \pm 8.8$   & $7$   & $.1 \pm 2.6$  & $1$   & $.0 \pm 0.7$ \\[0.2ex]
%    =4, =5          & =0, =1    & Data & 9546       &                   & 1734      &                & 315   &               & 44                   \\
%                    &           & SM   & $10490$    & $\,\pm\, 1100$    & $1880$    & $\,\pm\, 250$  & $320$ & $\,\pm\, 110$ & $40$  & $\,\pm\, 27$ \\[0.2ex]
%    =4, =5          & ${\geq}2$ & Data & 1012       &                   & 93        &                & 4     &               & 3                    \\
%                    &           & SM   & $970$      & $\,\pm\, 120$     & $81$      & $.2 \pm 9.4$   & $8$   & $.4 \pm 2.4$  & $1$   & $.2 \pm 0.7$ \\[0.2ex]
%    ${\geq}6$       & =0, =1    & Data & 758        &                   & 141       &                & 33    &               & 5                    \\
%                    &           & SM   & $910$      & $\,\pm\, 180$     & $167$     & $\,\pm\, 71$   & $33$  & $\,\pm\, 25$  & $4$   & $.2 \pm 5.4$ \\[0.2ex]
%    ${\geq}6$       & ${\geq}2$ & Data & 197        &                   & 14        &                & 3     &               & 0                    \\
%                    &           & SM   & $189$      & $\,\pm\, 39$      & $16$      & $.9 \pm 4.8$   & $2$   & $.1 \pm 1.2$  & $0$   & $.2 \pm 0.3$ \\
    =1, ${\geq}2a$ & 0         & Data & 411\,184   &                   & 11\,448   &                & 1116  &               & 111                  \\
                   &           & SM   & $360\,000$ & $\,\pm\, 38\,000$ & $10\,000$ & $\,\pm\, 1400$ & $910$ & $\,\pm\, 170$ & $107$ & $\,\pm\, 28$ \\[0.2ex]
    =1, ${\geq}2a$ & ${\geq}1$ & Data & 31\,174    &                   & 769       &                & 105   &               & 7                    \\
                   &           & SM   & $25\,500$  & $\,\pm\, 2500$    & $649$     & $\,\pm\, 91$   & $69$  & $\,\pm\, 13$  & $6$   & $.4 \pm 1.8$ \\[0.2ex]
    =2, =3         & =0, =1    & Data & 66\,955    &                   & 5946      &                & 903   &               & 100                  \\
                   &           & SM   & $58\,000$  & $\,\pm\, 11\,000$ & $5400$    & $\,\pm\, 1100$ & $860$ & $\,\pm\, 220$ & $113$ & $\,\pm\, 41$ \\[0.2ex]
    =2, =3         & ${\geq}2$ & Data & 1045       &                   & 70        &                & 6     &               & 0                    \\
                   &           & SM   & $870$      & $\,\pm\, 130$     & $56$      & $.9 \pm 9.4$   & $7$   & $.1 \pm 1.7$  & $1$   & $.0 \pm 0.4$ \\[0.2ex]
    =4, =5         & =0, =1    & Data & 9546       &                   & 1734      &                & 315   &               & 44                   \\
                   &           & SM   & $10\,500$    & $\,\pm\, 1100$    & $1880$    & $\,\pm\, 310$  & $319$ & $\,\pm\, 71$  & $40$  & $\,\pm\, 14$ \\[0.2ex]
    =4, =5         & ${\geq}2$ & Data & 1012       &                   & 93        &                & 4     &               & 3                    \\
                   &           & SM   & $970$      & $\,\pm\, 110$     & $81$      & $\, \pm 11$    & $8$   & $.4 \pm 1.7$  & $1$   & $.2 \pm 0.4$ \\[0.2ex]
    ${\geq}6$      & =0, =1    & Data & 758        &                   & 141       &                & 33    &               & 5                    \\
                   &           & SM   & $910$      & $\,\pm\, 180$     & $167$     & $\,\pm\, 76$   & $33$  & $\,\pm\, 25$  & $4$   & $.2 \pm 5.0$ \\[0.2ex]
    ${\geq}6$      & ${\geq}2$ & Data & 197        &                   & 14        &                & 3     &               & 0                    \\
                   &           & SM   & $189$      & $\,\pm\, 40$      & $16$      & $.9 \pm 4.9$   & $2$   & $.1 \pm 1.2$  & $0$   & $.2 \pm 0.2$ \\
    \hline
  \end{tabular}
\end{table}
\endgroup

\begin{figure}[!b]
  \centering
  \includegraphics[width=0.6\textwidth]{Figures/correlation_supplementary.pdf}
  \caption{Correlation matrix for the SM background estimates
    determined from the CR-only fit using the simplified binning
    schema defined in Table~\ref{tab:simplified}.}
  \label{fig:correlation}
\end{figure} 


%_______________________________________________________________________________
%_______________________________________________________________________________
%_______________________________________________________________________________
