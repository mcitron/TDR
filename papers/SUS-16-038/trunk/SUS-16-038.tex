\RCS$Revision: $
\RCS$HeadURL: $
\RCS$Id: $

%_______________________________________________________________________________
%_______________________________________________________________________________
%_______________________________________________________________________________

% Typesetting
\newcommand\T{\rule{0pt}{2.6ex}}
\newcommand\B{\rule[-1.2ex]{0pt}{0pt}}
\newcommand{\NA}{\ensuremath{\text{---}}\xspace}
\newcommand{\dash}{\multicolumn{1}{c}{\NA}}
\newcommand{\ph}[1]{\phantom{#1}}

% Symbols
\newcommand{\tf}{\ensuremath{\mathcal{T}}\xspace}
\newcommand{\cls}{\ensuremath{\mathrm{CL}_\mathrm{s}}\xspace}
\newcommand{\dm}{\ensuremath{\Delta m}\xspace}
\newcommand{\higgsmass}{\ensuremath{m_{\textrm{H}}}\xspace}

% Variables
\newcommand{\njet}{\ensuremath{n_{\mathrm{jet}}}\xspace}
\newcommand{\nb}{\ensuremath{n_{\mathrm{b}}}\xspace}
\newcommand{\scalht}{\ensuremath{H_{\mathrm{T}}}\xspace}
\newcommand{\scalst}{\ensuremath{\mathcal{E}_\mathrm{T}}\xspace}
\newcommand{\HTmiss}{\ensuremath{H_{\mathrm{T}}^{\text{miss}}}\xspace}

\newcommand{\dst}{\ensuremath{\Delta\scalst}\xspace}
\newcommand{\alphat}{\ensuremath{\alpha_{\mathrm{T}}}\xspace}
\newcommand{\dphi}{\ensuremath{\Delta\phi^{*}_\text{min}}\xspace}
\newcommand{\bdphi}{\ensuremath{\Delta\phi^{*}_\text{min}}\xspace}
\newcommand{\bdphimod}{\ensuremath{\Delta\phi^{*_{\, 25}}_\text{min}}\xspace}
\newcommand{\minchi}{\ensuremath{\chi_\text{min}}\xspace}
\newcommand{\mhtmet}{\ensuremath{\HTmiss / \ETmiss}\xspace}

% Selections
\newcommand{\jets}{\ensuremath{\text{jets}}}
\newcommand{\mj}{\ensuremath{\mu + \jets}\xspace}
\newcommand{\mmj}{\ensuremath{\mu\mu + \jets}\xspace}
\newcommand{\mmjpm}{\ensuremath{\mu^\pm\mu^\mp + \jets}\xspace}
\newcommand{\gj}{\ensuremath{\gamma + \jets}\xspace}

% Processes
\providecommand{\PSl}{\ensuremath{\widetilde{\ell}}\xspace}
\newcommand{\zmumu}{\ensuremath{\cPZ \to \mu\mu}\xspace}
\newcommand{\znunu}{\ensuremath{\cPZ \to \cPgn\cPagn}\xspace}
\newcommand{\zj}{\ensuremath{\cPZ + \jets}\xspace}
\newcommand{\zmumuj}{\ensuremath{\cPZ (\to \mu\mu) + \jets}\xspace}
\newcommand{\znunuj}{\ensuremath{\cPZ (\to \cPgn\cPagn) + \jets}\xspace}
\newcommand{\wj}{\ensuremath{\PW + \jets}\xspace}
\newcommand{\wmj}{\ensuremath{\PW (\to \mu\nu) + \text{jets}}\xspace}
\newcommand{\wlj}{\ensuremath{\PW (\to \ell\nu) + \text{jets}}\xspace}
\newcommand{\ttw}{\ensuremath{\ttbar\PW}\xspace}
\newcommand{\ttz}{\ensuremath{\ttbar\cPZ}\xspace}

\newlength\cmsFigWidth
\newlength\cmsFigWidthTwo
\ifthenelse{\boolean{cms@external}}{\setlength\cmsFigWidth{0.49\textwidth}}{\setlength\cmsFigWidth{0.7\textwidth}}
\ifthenelse{\boolean{cms@external}}{\setlength\cmsFigWidthTwo{0.95\textwidth}}{\setlength\cmsFigWidthTwo{0.95\textwidth}}
\ifthenelse{\boolean{cms@external}}{\providecommand{\cmsLeft}{top\xspace}}{\providecommand{\cmsLeft}{left\xspace}}
\ifthenelse{\boolean{cms@external}}{\providecommand{\cmsRight}{bottom\xspace}}{\providecommand{\cmsRight}{right\xspace}}
\ifthenelse{\boolean{cms@external}}{\providecommand{\cmsTable}[1]{#1}}{\providecommand{\cmsTable}[1]{\resizebox{\textwidth}{!}{#1}}}

\newlength\cmsTableLabelSkip
\setlength\cmsTableLabelSkip{-1.2ex}

%_______________________________________________________________________________
%_______________________________________________________________________________
%_______________________________________________________________________________

\cmsNoteHeader{SUS-16-038}

\title{Search for supersymmetry in pp collisions using final states
  with at least one jet and missing transverse momentum}
\titlerunning{Search for new physics in pp collisions using final
  states with at least one jet and missing transverse momentum}

\address[cern]{CERN} 
\author[cern]{The CMS Collaboration}

\date{\today}

\abstract{A search for supersymmetry is performed in final states
  containing one or more jets and an imbalance in transverse
  momentum. A sample of events resulting from pp collisions at a
  centre-of-mass energy of 13\TeV is analysed. The event data were
  recorded with the CMS detector at the CERN LHC in 2016 and
  correspond to an integrated luminosity of 36.3\fbinv. The search
  provides sensitivity to models that predict a stable weakly
  interacting massive particle, such as the neutralino. The search
  employs several kinematic variables to suppress multijet production
  to a subdominant level with respect to all other standard model
  backgrounds. Several further kinematic variables are used for signal
  extraction using a binned maximum-likelihood fit to the data. The
  number of candidate signal events is found to agree with the
  expected event counts from standard model processes. A number of
  simplified supersymmetric models that assume the production of
  gluino or squark pairs that decay to neutralinos and standard model
  particles, are used to interpret the result. For models that assume
  the production of gluino pairs, gluino and neutralino masses up to X
  and Y\GeV, respectively, are excluded. For models that assume In the
  case of the production of squark pairs, light-flavour, bottom, and
  top squarks masses up to X, Y, and Z\GeV are excluded,
  respectively. Models with near-degenerate mass spectra are also
  considered. For models that assume top-squark production and their
  decays to nearly mass-degenerate neutralinos, top squark masses up
  to X\GeV are excluded. }

\hypersetup{ 
  pdfauthor={Robert Bainbridge, Freya Blekman, Shane Breeze, Oliver
    Buchmueller, Stefano Casasso, Matthew Citron, Adam Elwood, Henning
    Flaecher, Aran Garcia-Bellido, Christian Laner, Kin Ho Lo, Sarah
    Alam Malik, Bjoern Penning, Tai Sakuma, Dominic Smith, Alex
    Tapper},
  pdftitle={Search for supersymmetry in pp collisions using final
    states with at least one jet and missing transverse momentum},
  pdfsubject={CMS},
  pdfkeywords={CMS, jets, monojet, missing transverse momentum,
    supersymmetry, dark matter, AlphaT} 
}

\maketitle

%_______________________________________________________________________________
%_______________________________________________________________________________
%_______________________________________________________________________________

\section{Introduction}
\label{sec:introduction}

Supersymmetry (SUSY)~\cite{ref:SUSY-1, ref:SUSY0, ref:SUSY3,
  ref:SUSY1} is an extension of the standard model (SM) of particle
physics that is motivated through several aspects, including the
possibility to unify the gauge coupling constants at high
energy~\cite{Dimopoulos:1981yj, Ibanez:1981yh, Marciano:1981un}, to
provide a solution to the gauge hierarchy
problem~\cite{ref:hierarchy1, ref:hierarchy2}, and to provide a dark
matter (DM). SUSY introduces at least one bosonic (fermionic)
superpartner for each fermionic (bosonic) SM particle, which differ in
spin by one-half unit. The particle content of the minimal
supersymmetric standard model~\cite{ref:SUSY2} can be summarised as
follows. The superpartners to the gluons, quarks, and leptons are the
gluinos $\PSg$, the light-flavour $\PSQ$, bottom $\PSQb$, and top
$\PSQt$ squarks, and the sleptons $\PSl$, respectively. An extended
Higgs sector comprising a quintet of scalar particle states is
predicted. The higgsino and gaugino superpartners to the Higgs and
electroweak gauge bosons are expected to mix to give six observable
states: two charginos $\PSGcpm_{1,2}$ and four neutralinos
$\PSGcz_{1,2,3,4}$. 

The assumption of $R$-parity conservation~\cite{Farrar:1978xj} has
important consequences for cosmology and collider
phenomenology. Supersymmetric particles are expected to be produced in
pairs at the LHC, with heavy states decaying eventually to the stable
lightest SUSY particle (LSP). The LSP is generally assumed to be the
$\PSGczDo$, which is weakly interacting and massive. This SUSY
particle is considered to be a candidate for dark matter
(DM)~\cite{Jungman:1995df}, the existence of which is supported by
astrophysical data~\cite{1674-1137-38-9-090001}. Hence, the
characteristic signature of natural SUSY production at the LHC is a
final state containing an abundance of jets, possibly originating from
top or bottom quarks, accompanied by a significant imbalance in
transverse momentum, \ptvecmiss. 

The gauge hierachy problem of the SM, in which corrections from loop
processes lead to a Higgs boson mass \higgsmass close to the cutoff
scale for the theory, which can only be avoided by an extreme fine
tuning of the bare Higgs boson mass parameter. The presence of
superpartners can alleviate this problem by cancelling the
contributions to \higgsmass from SM loop processes. So-called
``natural'' SUSY models require only a minimal fine-tuning of the bare
Higgs boson mass parameter if the gluino, third-generation squarks,
and a (higgsino-like) $\PSGczDo$ are at or near the electroweak
scale~\cite{ref:barbierinsusy}. This solution has attracted
considerable interest~\cite{Delgado:2012eu, Boehm:1999tr,
  Carena:2008mj, Grober:2014aha, Grober:2015fia} following the
discovery of the Higgs boson~\cite{ref:atlashiggsdiscovery,
  ref:cmshiggsdiscovery, ref:cmshiggsdiscoverylong} at $\higgsmass =
125\GeV$. Models with light and nearly degenerate $\PSQt$ and
$\PSGczDo$ masses are also well motivated ~\cite{Boehm:1999bj,
  Balazs:2004bu, Martin:2007gf, Martin:2007hn}. 

This paper presents a search for new physics processes in final states
with one or more energetic jets and significant \ptvecmiss. The search
is performed with a sample of proton-proton (pp) collision data at a
centre-of-mass energy of 13\TeV. The analysed data sample corresponds
to an integrated luminosity of $36.4 \pm 1.0\fbinv$ recorded by the
CMS experiment. Earlier searches using the same technique have been
performed in pp collisions at both $\sqrt{s} = 7$~\cite{RA1Paper,
  RA1Paper2011, RA1Paper2011FULL}, 8~\cite{RA1Paper2012, RA1Parked},
and 13\TeV~\cite{RA1Paper2015} by the CMS Collaboration. Several
similar searches have already been performed at $\sqrt{s} = 13\TeV$ by
the ATLAS~\cite{} and CMS~\cite{} collaborations. These searches
extend the reach of the most constraining searches performed during
Run~1 at $\sqrt{s} = 7$ and 8\TeV by the ATLAS~\cite{Aad:2015iea,
  Aad:2015pfx} (and references therein) and CMS~\cite{}
collaborations. The data analysed in this analysis is a factor
$\sim$16 larger than that presented in Ref.~\cite{RA1Paper2015}, which
provides a significant gain in sensitivity to the production of
\TeV-scale coloured SUSY particles, particularly for models with
nearly mass-degenerate spectra that are characterised by soft final
state signitures and relatively low experimental acceptance.

The search strategy aims to provide robust sensitivity to a wide range
of SUSY and non-SUSY models that postulate the existence of a stable,
weakly interacting, massive particle. The overwhelmingly dominant
background for a new-physics search in all-jet final states resulting
from proton-proton collisions is multijet production, a manifestation
of quantum chromodynamics. The search employs several dedicated
variables to discriminate against this background while maintaining
experimental acceptance to events characterised by the presence of
significant \ptvecmiss. Signal extraction is performed via several
discriminating kinematic variables: the number of reconstructed jets
per event, the number of these jets identified as originating from
bottom quarks, and the scalar and vector \pt sums of these jets. The
strong suppression of the multijet background permits the application
of low thresholds on kinematic variables. This allows to maintain a
large acceptance to a broad range of models, such as those that assume
the strong production of massive coloured SUSY particles, including
third-generation squark signatures, and both large and small mass
splittings between the parent SUSY particle and the LSP. The search
also considers final states containing a ``monojet'' topology, which
is expected to improve the sensitivity to DM particle production in pp
collisions~\cite{Fox:2012ee, Buchmueller:2015eea}.

The results of the search are used to constrain the mass parameter
space of simplified SUSY models~\cite{Alwall:2008ag, Alwall:2008va,
  sms}. Nine unique combinations of production and decay modes are
considered. Models that assume the direct or gluino-mediated
production of squark pairs, and their subsequent (prompt) decays to SM
particles and the LSP, are examined. Gluinos are assumed to undergo
three-body decays via off-shell squarks. Further, both light and heavy
flavours of squarks are considered, as well as a range of differences
in mass (\dm) between the parent SUSY particle and the LSP. In the
case of top squark decays, various decay modes, dependent on \dm, are
considered.

This paper is organised as follows. Sections~\ref{sec:detector}
and~\ref{sec:simulation} describe the CMS apparatus and the various
software packages used to generate samples of simulated events,
respectively. Sections~\ref{sec:reconstruction}
and~\ref{sec:selection} provide details on, respectively, the
reconstruction algorithms and selection criteria used to identify
candidate signal events and control
samples. Section~\ref{sec:backgrounds} describes the methods used to
estimate the background contributions from SM processes. The search
results and physics interpretations are described in
Sections~\ref{sec:result} and~\ref{sec:interpretations},
respectively. Finally, we summarise in Section~\ref{sec:summary}.

%_______________________________________________________________________________
%_______________________________________________________________________________
%_______________________________________________________________________________

\clearpage

\section{The CMS detector}
\label{sec:detector}

The central feature of the CMS apparatus is a superconducting solenoid
of 6\unit{m} internal diameter, providing a magnetic field of
3.8\unit{T}. Within the solenoid volume are a silicon pixel and strip
tracker, a lead tungstate crystal electromagnetic calorimeter (ECAL),
and a brass and scintillator hadron calorimeter (HCAL), each composed
of a barrel and two endcap sections. Forward calorimeters extend the
pseudorapidity coverage provided by the barrel and endcap
detectors. Muons are measured in gas-ionization detectors embedded in
the steel flux-return yoke outside the solenoid. 

Events of interest are selected using a two-tiered trigger
system~\cite{Khachatryan:2016bia}. The first level (L1), composed of
custom hardware processors, uses information from the calorimeters and
muon detectors to select events at a rate of around 100\unit{kHz}
within a time interval of less than 4\mus. The second level, known as
the high-level trigger (HLT), consists of a farm of processors running
a version of the full event reconstruction software optimized for fast
processing, and reduces the event rate to less than 1\unit{kHz} before
data storage.

A more detailed description of the CMS detector, together with a
definition of the coordinate system used and the relevant kinematic
variables, can be found in Ref.~\cite{Chatrchyan:2008zzk}.

%_______________________________________________________________________________
%_______________________________________________________________________________
%_______________________________________________________________________________

\clearpage
\section{Simulated event samples}
\label{sec:simulation}

%_______________________________________________________________________________
%_______________________________________________________________________________
%_______________________________________________________________________________

\clearpage
\section{Event reconstruction}
\label{sec:reconstruction}

\subsection{Identifying physics objects}

The physics objects used by this search are determined from particle
candidates provided by the particle-flow (PF) event algorithm, which
aims to reconstruct and identify each individual particle in an event
with an optimized combination of information from the various elements
of the CMS detector. 

The energy of photons is directly obtained from
the ECAL measurement, corrected for zero-suppression effects. The
energy of electrons is determined from a combination of the electron
momentum at the primary interaction vertex as determined by the
tracker, the energy of the corresponding ECAL cluster, and the energy
sum of all bremsstrahlung photons spatially compatible with
originating from the electron track. The energy of muons is obtained
from the curvature of the corresponding track. The energy of charged
hadrons is determined from a combination of their momentum measured in
the tracker and the matching ECAL and HCAL energy deposits, corrected
for zero-suppression effects and for the response function of the
calorimeters to hadronic showers. Finally, the energy of neutral
hadrons is obtained from the corresponding corrected ECAL and HCAL
energy.

The most accurate estimator for the missing transverse momentum vector
\ptvecmiss is defined as the projection on the plane perpendicular to
the beams of the negative vector sum of the momenta of all PF
candidate particles in an event. Its magnitude is referred to as
\ETmiss.

Jets are reconstructed from the PF candidate particles, clustered by
the anti-$k_\mathrm{t}$ algorithm~\cite{Cacciari:2008gp,
  Cacciari:2011ma} with a size parameter of 0.4. In this process, raw
jet energy is obtained from the sum of the candidate particle
energies, and the raw jet momentum by the vectorial sum of the
candidate particle momenta, which results in a nonzero jet mass.


Jet momentum is determined from simulation to be within 5 to
10\% of the true momentum over the whole \pt spectrum and detector
acceptance. An offset correction is applied to jet energies to take
into account the contribution from additional proton-proton
interactions within the same or nearby bunch
crossings~\cite{pileup}. 
The raw jet energies are then corrected to establish a relative
uniform response of the calorimeter in $\eta$ and a calibrated
absolute response in transverse momentum \pt. 

Jet energy corrections are derived from simulation, and are confirmed
with in situ measurements of the energy balance in dijet and photon +
jet events~\cite{Chatrchyan:2011ds}. The jet energy resolution amounts
typically to 15\% at 10\GeV, 8\% at 100\GeV, and 4\% at 1\TeV,
respectively. 


Selection criteria are applied to each event to
remove spurious jet-like features originating from isolated noise
patterns in certain HCAL regions.


%_______________________________________________________________________________
%_______________________________________________________________________________
%_______________________________________________________________________________

\clearpage

\begin{table}[!t]
  \topcaption{Summary of \scalht-dependent thresholds on \alphat that
    are used, in addition to the requirements $\HTmiss > 130\GeV$,
    $\bdphi > 0.5$, and $\mhtmet < 1.25$, to suppress the multijet
    background to the percent level relative to all other SM
    backgrounds. }
  \label{tab:multijet_variables}
  \centering
  {\begin{tabular}{ lcccccccc }
    \hline
    $\scalht [\GeVns{}]$\T\B & 200--250 & 250--300 & 300--350 & 350--400 & 400--600 & 600--900 & 900-1200 & $>$1200 \\
    \hline
    $\alphat$\T\B            & $>$0.65  & $>$0.60  & $>$0.55  & $>$0.53  & $>$0.52  & $>$0.52  & \dash    & \dash   \\
    \hline
  \end{tabular}}
\end{table}

\begin{table}[!t]
  \topcaption{Summary of the lower bounds of the first and final bins
    in \scalht [\GeVns{}] as a function of \njet and \nb. The nominal
    binning scheme comprises four bounded bins and one open bin
    covering the \scalht ranges 200--400, 400--600, 600-900,
    900--1200, and $>$1200\GeV.}
  \label{tab:ht_binning}
  \centering
  {\begin{tabular}{ lccccc }
    \hline
    $\njet \backslash\, \nb$\T\B & 0              & 1              & 2              & 3              & $\geq$4        \\
    \hline
    1\T                          & 200, \ph{1}900 & 200, \ph{1}600 & \dash          & \dash          & \dash          \\
    2                            & 200, 1200      & 200, 1200      & 200, \ph{1}600 & \dash          & \dash          \\
    3                            & 200, 1200      & 200, 1200      & 200, 1200      & 200, \ph{1}600 & \dash          \\
    4                            & 200, 1200      & 200, 1200      & 200, 1200      & 200, \ph{1}900 & 200, \ph{1}\NA \\
    5                            & 400, 1200      & 400, 1200      & 400, 1200      & 400, \ph{1}900 & 400, \ph{1}\NA \\
    $\geq$6                      & 400, 1200      & 400, 1200      & 400, 1200      & 400, 1200      & 400, \ph{1}\NA \\
    asymmetric\B                 & 200, \ph{1}900 & 200, \ph{1}900 & 200, \ph{1}900 & 200, \ph{1}600 & 200, 400       \\
    \hline
  \end{tabular}}
\end{table}

\begin{table}[!t]
  \topcaption{Summary of the lower bounds of the final bin in \HTmiss
  [\GeVns{}] as a function of \njet and \scalht region [\GeVns{}]. The
  nominal binning scheme comprises a bin of 130-200\GeV, further bins
  of width 100 and 200\GeV in the ranges $200 < \HTmiss < 800\GeV$ and
  $\HTmiss > 800\GeV$, respectively, until the final open bin, as
  indicated below.  }
  \label{tab:mht_binning}
  \centering
  {\begin{tabular}{ lccccc }
    \hline
    $\njet \backslash\, \scalht [\GeVns{}]$\T\B & 200--400 & 400--600 & 600--900 & 900--1200 & $>$1200 \\
    \hline
    1\T                                         & 300      & 500      & 800      & 1000      & \dash   \\
    2                                           & 300      & 500      & 800      & 1000      & 1000    \\
    3                                           & 300      & 500      & 800      & 1000      & 1000    \\
    4                                           & 300      & 500      & 800      & 1000      & 1000    \\
    5                                           & 300      & 500      & 800      & 1000      & 1000    \\
    $\geq$6                                     & \dash    & 400      & 700      & 900       & 1000    \\
    asymmetric\B                                & \dash    & 500      & 800      & 1000      & \dash   \\
    \hline
  \end{tabular}}
\end{table}



%_______________________________________________________________________________

\clearpage
\bibliography{auto_generated}
