\RCS$Revision: $
\RCS$HeadURL: $
\RCS$Id: $

%_______________________________________________________________________________
%_______________________________________________________________________________
%_______________________________________________________________________________

%\newcommand{\eslash}{{\hbox{$E$\kern-0.6em\lower-.05ex\hbox{/}\kern0.10em}}}
\newcommand{\met}{\mbox{$\eslash_\text{T}$}\xspace}
\newcommand{\cls}{\mbox{CL$_s$}\xspace}
\newcommand{\wtaunu}{\ensuremath{\PW \rightarrow \Pgt\cPgn}}
\newcommand{\Et}{\ensuremath{{E_{\text T}}}\xspace}
\newcommand{\Hslash}{{\hbox{$H$\kern-0.8em\lower-.05ex\hbox{/}\kern0.10em}}}

\newcommand{\scalht}{\mbox{$H_\text{T}$}\xspace}
\newcommand{\mht}{\mbox{$\Hslash_\text{T}$}\xspace}
\newcommand{\HTmiss}{\mbox{$H_\text{T}^\text{miss}$}\xspace}
\newcommand{\dht}{\ensuremath{\Delta\scalht}\xspace}
\newcommand{\alphat}{\ensuremath{\alpha_{\text T}}\xspace}
\newcommand{\njet}{\ensuremath{N_{\text{jet}}}\xspace}
\newcommand{\njetlow}{\ensuremath{2 \leq \njet \leq 3}\xspace}
\newcommand{\njethigh}{\ensuremath{\njet \geq 4}\xspace}
\newcommand{\nb}{\ensuremath{N_{\text{b}}}\xspace}
\newcommand{\mj}{\ensuremath{\mu\! +\! \text{jets}}\xspace}
\newcommand{\mmj}{\ensuremath{\mu\mu\! +\! \text{jets}}\xspace}
\newcommand{\gj}{\ensuremath{\gamma\! +\! \text{jets}}\xspace}
\newcommand{\wjets}{\ensuremath{\PW\! +\! \text{jets}}\xspace}
\newcommand{\zjets}{\ensuremath{\cPZ\! +\! \text{jets}}\xspace}
\newcommand{\znunujets}{\ensuremath{\cPZ\! \rightarrow\! \cPgn\cPagn\! +\! \text{jets}}\xspace}
\newcommand{\znunu}{\ensuremath{\cPZ\! \rightarrow\! \cPgn\cPagn}\xspace}
\newcommand{\zmumu}{\ensuremath{\cPZ\! \rightarrow\! \mu\mu}\xspace}
\newcommand{\zmumujets}{\ensuremath{\cPZ\! \rightarrow\! \mu\mu\! +\! \text{jets}}\xspace}
\newcommand\T{\rule{0pt}{2.6ex}}
\newcommand\B{\rule[-1.2ex]{0pt}{0pt}}
\def\mhtmet{\mbox{$\HTmiss / \ETmiss$}\xspace}
\newcommand{\Pt}{\ensuremath{{p_{\text T}}}\xspace}
\newcommand{\dphi}{\ensuremath{\Delta\phi^{*}_\text{min}}\xspace}
\newcommand{\dm}{\ensuremath{\Delta m}\xspace}
\newcommand{\alphatmin}{\ensuremath{\alphat^\text{min}}\xspace}

%\newcommand\rs{\raisebox{1.0ex}[-1.0ex]}

% PROCESSES

\newcommand{\ra}{\ensuremath{\rightarrow}}
\newcommand{\jets}{\ensuremath{\text{jets}}}

\newcommand{\lj}{\ensuremath{\ell\! +\! \jets}\xspace}
\newcommand{\mj}{\ensuremath{\mu\! +\! \jets}\xspace}
\newcommand{\ej}{\ensuremath{e\! +\! \jets}\xspace}

\newcommand{\llj}{\ensuremath{\ell\ell\! +\! \jets}\xspace}
\newcommand{\mmj}{\ensuremath{\mu\mu\! +\! \jets}\xspace}
\newcommand{\eej}{\ensuremath{ee\! +\! \jets}\xspace}
\newcommand{\mmjpm}{\ensuremath{\mu^\pm\mu^\mp\! +\! \jets}\xspace}

\newcommand{\gj}{\ensuremath{\gamma\! +\! \jets}\xspace}

\newcommand{\wj}{\ensuremath{\PW\! +\! \jets}\xspace}
\newcommand{\wlj}{\ensuremath{\PW\! (\ra\! \ell\nu)\! +\! \textrm{jets}}\xspace}
\newcommand{\wmj}{\ensuremath{\PW\! (\ra\! \mu\nu)\! +\! \textrm{jets}}\xspace}
\newcommand{\wej}{\ensuremath{\PW\! (\ra\! e\nu)\! +\! \textrm{jets}}\xspace}

\newcommand{\wlnu}{\ensuremath{\PW\! \ra\! \ell\nu}\xspace}
\newcommand{\wmunu}{\ensuremath{\PW\! \ra\! \mu\nu}\xspace}
\newcommand{\wenu}{\ensuremath{\PW\! \ra\! e\nu}\xspace}
\newcommand{\wtaunu}{\ensuremath{\PW \rightarrow \Pgt\cPgn}\xspace}

\newcommand{\zj}{\ensuremath{\cPZ\! +\! \jets}\xspace}
\newcommand{\zllj}{\ensuremath{\cPZ\! (\ra\! \ell\ell)\! + \! \jets}\xspace}
\newcommand{\zmumuj}{\ensuremath{\cPZ\! (\ra\! \mu\mu)\! +\! \jets}\xspace}
\newcommand{\zeej}{\ensuremath{\cPZ\! (\ra\! ee)\! +\! \jets}\xspace}
\newcommand{\znunuj}{\ensuremath{\cPZ\! (\ra\! \cPgn\cPagn)\! +\! \jets}\xspace}

\newcommand{\zll}{\ensuremath{\cPZ\! \ra\! \ell\ell}\xspace}
\newcommand{\zmumu}{\ensuremath{\cPZ\! \ra\! \mu\mu}\xspace}
\newcommand{\zee}{\ensuremath{\cPZ\! \ra\! ee}\xspace}
\newcommand{\znunu}{\ensuremath{\cPZ\! \ra\! \cPgn\cPagn}\xspace}

\newcommand{\ttj}{\ensuremath{\ttbar\! +\! \jets}\xspace}
\newcommand{\ttw}{\ensuremath{\ttbar\PW}\xspace}
\newcommand{\ttz}{\ensuremath{\ttbar\cPZ}\xspace}

% SIGNAL 

\newcommand{\Ttwocc}{\ensuremath{\text{pp}\,\ra\,\sTop\sTop^{*}\,\ra\,\text{c}\chiz\,\bar{\text{c}}\chiz}}
\newcommand{\Ttwodegen}{\ensuremath{\text{pp}\,\ra\,\sTop\sTop^{*}\,\ra\,\text{b}ff'\chiz \,\text{b}ff'\chiz}}
\newcommand{\Ttwobw}{\ensuremath{\text{pp}\,\ra\,\sTop\sTop^{*}\,\ra\,\text{b}W\chiz \,\bar{\text{b}}W\chiz}}
\newcommand{\Ttwott}{\ensuremath{\text{pp}\,\ra\,\sTop\sTop^{*}\,\ra\,\text{t}\chiz\,\bar{\text{t}}\chiz}}
\newcommand{\Ttwobb}{\ensuremath{\text{pp}\,\ra\,\sBot\sBot^{*}\,\ra\,\text{b}\chiz\,\bar{\text{b}}\chiz}}
\newcommand{\Ttwoqq}{\ensuremath{\text{pp}\,\ra\,\sQua\sQua^{*}\,\ra\,\text{q}\chiz\,\bar{\text{q}}\chiz}}
\newcommand{\Tonebbbb}{\ensuremath{\text{pp}\,\ra\,\sGlunew\sGlunew^{*}\,\ra\,\bar{\text{b}}\text{b}\chiz\,\bar{\text{b}}\text{b}\chiz}}
\newcommand{\Toneqqqq}{\ensuremath{\text{pp}\,\ra\,\sGlunew\sGlunew^{*}\,\ra\,\bar{\text{q}}\text{q}\chiz\,\bar{\text{q}}\text{q}\chiz}}
\newcommand{\Tonetttt}{\ensuremath{\text{pp}\,\ra\,\sGlunew\sGlunew^{*}\,\ra\,\bar{\text{t}}\text{t}\chiz\,\bar{\text{t}}\text{t}\chiz}}

%\newcommand{\dphi}{\ensuremath{\Delta \phi}}
\newcommand{\dphi}{\ensuremath{\Delta\phi^{*}_{\rm min}}\xspace}
\newcommand{\dphijj}{\ensuremath{\Delta \phi_{ j1,j2}}}
\newcommand{\Pt}{\ensuremath{{p_{\text T}}}\xspace}
\newcommand{\pts}{\ensuremath{p_{\text T}{\text s}}\xspace}
\newcommand{\Et}{\ensuremath{{E_{\text T}}}\xspace}
\newcommand{\ptjf}{\ensuremath{p_{\rm T}^{ {\rm j}_1} }}
\newcommand{\ptjs}{\ensuremath{p_{\rm T}^{ {\rm j}_2} }}
\newcommand{\ptjt}{\ensuremath{p_{\rm T}^{ {\rm j}_3} }}
\newcommand{\etajf}{\ensuremath{\eta^{ {\rm j}_1} }}
\newcommand{\etajs}{\ensuremath{\eta^{ {\rm j}_2} }}
\newcommand{\etajt}{\ensuremath{\eta^{ {\rm j}_3} }}
\newcommand{\al}{\ensuremath{\alpha}}
\newcommand{\alt}{\ensuremath{\alpha_{\text{T}}}\xspace}
\newcommand{\etaabs}{\ensuremath{|\eta|}}
%\newcommand{\gev}{\ensuremath{\mathrm{\,Ge\kern -0.1em V}}}
\newcommand{\pb}{\ensuremath{pb^{-1}}}
\newcommand{\mjj}{\ensuremath{M_{\text{inv}}^{j1,j2}}}
\newcommand{\chiznew}{\ensuremath{\chi^{0}}\xspace}
\newcommand{\chipnew}{\ensuremath{\chi^{+}}\xspace}
\newcommand{\sQuanew}{\ensuremath{\tilde{\rm q}}\xspace}
\newcommand{\sGlunew}{\ensuremath{\tilde{\rm g}}\xspace}
\newcommand{\ttNew}{\ensuremath{\rm{t}\bar{\rm{t}}}\xspace}
\newcommand{\tev}{\TeV}
%<TW date="30/10/2010">
%\newcommand{\Et}{E_{T}}
\newcommand{\combIso}{Iso_{\textrm{comb.}}}
\renewcommand{\arraystretch}{1.2}
\newcommand{\bigNum}[2]{#1 \, \times \, 10 \, ^{#2}}
%</TW>

\newcommand{\raT}{\ensuremath{R_{\alt}}}
\newcommand{\RaT}{\ensuremath{R_{\alt}}\xspace}

\newcommand\T{\rule{0pt}{2.6ex}}
\newcommand\B{\rule[-1.2ex]{0pt}{0pt}}

\def\eslash{{\hbox{$E$\kern-0.6em\lower-.05ex\hbox{/}\kern0.10em}}}
\def\vecmet{\mbox{$\vec{\eslash}_T$}} %missing ET vector
\def\vecet{\mbox{$\vec{E}_\text{T}$}} % ET vector
\def\MET{\mbox{$\eslash_\text{T}$}\xspace}
%\def\met{\mbox{$\eslash_\text{T}$}\xspace}
\def\met{\mbox{$E_\text{T}^{\rm miss}$}\xspace}
\def\pfmet{\mbox{$\eslash_\text{T}^{\rm PF}$}\xspace}
\def\mex{\mbox{$\eslash_\text{x}$}} %missing Ex
\def\mey{\mbox{$\eslash_\text{y}$}} %missing Ey
\def\mepar{\mbox{$\eslash_\parallel$}}
\def\meperp{\mbox{$\eslash_\perp$}}
\def\Zmm{Z \rightarrow \mu\mu}
\def\metvec{\mbox{$\vec{\met}$}\xspace}
\def\metvecrec{\mbox{$\vec{\met}^{\rm rec}$}\xspace}
\def\metvecgen{\mbox{$\vec{\met}^{\rm gen}$}\xspace}
\def\metgen{\mbox{$\met^{\rm gen}$}\xspace}
\def\metparl{\mbox{$\mepar^{\rm rec}$}\xspace}
\def\metperp{\mbox{$\meperp^{\rm rec}$}\xspace}
\def\deltamet{\mbox{$\Delta\met$}\xspace}
\def\pthat{\mbox{$\hat{p}_T$}\xspace}
\def\hslash{{\hbox{$H$\kern-0.8em\lower-.05ex\hbox{/}\kern0.10em}}}
%\def\MHT{\mbox{$\hslash_\text{T}$}\xspace}
\def\mht{\mbox{$\hslash_\text{T}$}\xspace}
%\def\mht{\mbox{$H_{\rm T}^{\rm miss}$}\xspace}
\newcommand{\HTmiss}{\ensuremath{H_{\text{T}}^{\text{miss}}}\xspace}
\newcommand{\HTmissvec}{\ensuremath{\vec{H_{\text{T}}}^{\text{miss}}}\xspace}
\def\mhtvec{\mbox{$\vec{H}_{\rm T}^{\rm miss}$}\xspace}
%\def\mhtmet{\mbox{$\hslash_\text{T} / \eslash_\text{T}$}\xspace}
%\def\mhtmet{\mbox{$\mht / \met$}\xspace}
\newcommand{\mhtmet}{\ensuremath{H_{\mathrm{T}}^{\text{miss}} / E_{\mathrm{T}}^{\text{miss}}}\xspace}
\def\mhtmetmiss{\mbox{$\H_\text{T}^{\rm miss} / \E_\text{T}^{\rm miss}$}\xspace}
%\def\rmhtmet{\mbox{$R_{\hslash_\text{T} / \eslash_\text{T}}$}\xspace}
\def\rmhtmet{\mbox{$R_{\mht / \met}$}\xspace}
\def\sumet{\mbox{$\sum \rm{E}_\text{T}$}\xspace}
\def\etmiss{\mbox{$\eslash_\text{T}$}\xspace}
\def\htmiss{\mbox{$\hslash_\text{T}$}\xspace}
\def\mtt{\mbox{$\rm{M}_\text{T2}$}\xspace}
\def\rmec{\mbox{$R_{\mht/\met}$}\xspace}
\def\bdphi{\mbox{$\Delta\phi^{*}_{\rm min}$}\xspace}
\def\bdphimod{\mbox{$\Delta\phi^{*_{\, 25}}_{\rm min}$}\xspace}
\def\bigeslash{{\hbox{$E$\kern-0.38em\lower-.05ex\hbox{/}\kern0.10em}}}
\def\bigmet{\mbox{$\bigeslash_T$}}
\def\bighslash{{\hbox{$H$\kern-0.6em\lower-.05ex\hbox{/}\kern0.10em}}}
\def\bigmht{\mbox{$\bighslash_T$}}
\def\incl{\includegraphics[width=0.49\linewidth]}
\def\inclrot{\includegraphics[angle=90,width=0.47\linewidth]}
\def\INCL{\includegraphics[angle=90,width=0.45\linewidth]}
\def\Incl{\includegraphics[angle=90,width=0.60\linewidth]}
\def\cls{\mbox{CL$_s$}\xspace}

\newcommand{\zero}{\ensuremath{\phantom{0}}}

\newcommand{\scalht}{\ensuremath{H_{\mathrm{T}}}\xspace}
\newcommand{\dm}{\ensuremath{\Delta m}\xspace}


% Typesetting
\newcommand\T{\rule{0pt}{2.6ex}}
\newcommand\B{\rule[-1.2ex]{0pt}{0pt}}
\newcommand{\NA}{\ensuremath{\text{---}}\xspace}
\newcommand{\dash}{\multicolumn{1}{c}{\NA}}
\newcommand{\ph}[1]{\phantom{#1}}

% Symbols
\newcommand{\tf}{\ensuremath{\mathcal{T}}\xspace}
\newcommand{\cls}{\ensuremath{\mathrm{CL}_\mathrm{s}}\xspace}
\newcommand{\dm}{\ensuremath{\Delta m}\xspace}
\newcommand{\higgsmass}{\ensuremath{m_{\textrm{H}}}\xspace}

% Variables
\newcommand{\Et}{\ensuremath{{E_{\text T}}}\xspace}
\newcommand{\Pt}{\ensuremath{{p_{\text T}}}\xspace}
\newcommand{\njet}{\ensuremath{n_{\mathrm{jet}}}\xspace}
\newcommand{\nb}{\ensuremath{n_{\mathrm{b}}}\xspace}
\newcommand{\scalht}{\ensuremath{H_{\mathrm{T}}}\xspace}
\newcommand{\scalst}{\ensuremath{\mathcal{E}_\mathrm{T}}\xspace}
\newcommand{\met}{\ensuremath{E_{\mathrm{T}}^{\text{miss}}}\xspace}
\newcommand{\mht}{\ensuremath{H_{\mathrm{T}}^{\text{miss}}}\xspace}

\newcommand{\dst}{\ensuremath{\Delta\scalst}\xspace}
\newcommand{\alphat}{\ensuremath{\alpha_{\mathrm{T}}}\xspace}
\newcommand{\bdphi}{\ensuremath{\Delta\phi^{*}_\text{min}}\xspace}
\newcommand{\bdphimod}{\ensuremath{\Delta\phi^{*_{\, 25}}_\text{min}}\xspace}
\newcommand{\minchi}{\ensuremath{\chi_\text{min}}\xspace}
\newcommand{\mhtmet}{\ensuremath{\mht / \met}\xspace}

% Selections
\newcommand{\jets}{\ensuremath{\text{jets}}}
\newcommand{\mj}{\ensuremath{\mu + \jets}\xspace}
\newcommand{\mmj}{\ensuremath{\mu\mu + \jets}\xspace}
\newcommand{\mmjpm}{\ensuremath{\mu^\pm\mu^\mp + \jets}\xspace}
\newcommand{\gj}{\ensuremath{\gamma + \jets}\xspace}

% Processes
\providecommand{\PSl}{\ensuremath{\widetilde{\ell}}\xspace}
\newcommand{\zmumu}{\ensuremath{\cPZ \to \mu\mu}\xspace}
\newcommand{\znunu}{\ensuremath{\cPZ \to \cPgn\cPagn}\xspace}
\newcommand{\zj}{\ensuremath{\cPZ + \jets}\xspace}
\newcommand{\zmumuj}{\ensuremath{\cPZ (\to \mu\mu) + \jets}\xspace}
\newcommand{\znunuj}{\ensuremath{\cPZ (\to \cPgn\cPagn) + \jets}\xspace}
\newcommand{\wj}{\ensuremath{\PW + \jets}\xspace}
\newcommand{\wmj}{\ensuremath{\PW (\to \mu\nu) + \text{jets}}\xspace}
\newcommand{\wlj}{\ensuremath{\PW (\to \ell\nu) + \text{jets}}\xspace}
\newcommand{\ttw}{\ensuremath{\ttbar\PW}\xspace}
\newcommand{\ttz}{\ensuremath{\ttbar\cPZ}\xspace}

\newlength\cmsFigWidth
\newlength\cmsFigWidthTwo
\ifthenelse{\boolean{cms@external}}{\setlength\cmsFigWidth{0.49\textwidth}}{\setlength\cmsFigWidth{0.7\textwidth}}
\ifthenelse{\boolean{cms@external}}{\setlength\cmsFigWidthTwo{0.95\textwidth}}{\setlength\cmsFigWidthTwo{0.95\textwidth}}
\ifthenelse{\boolean{cms@external}}{\providecommand{\cmsLeft}{top\xspace}}{\providecommand{\cmsLeft}{left\xspace}}
\ifthenelse{\boolean{cms@external}}{\providecommand{\cmsRight}{bottom\xspace}}{\providecommand{\cmsRight}{right\xspace}}
\ifthenelse{\boolean{cms@external}}{\providecommand{\cmsTable}[1]{#1}}{\providecommand{\cmsTable}[1]{\resizebox{\textwidth}{!}{#1}}}

\newlength\cmsTableLabelSkip
\setlength\cmsTableLabelSkip{-1.2ex}

%_______________________________________________________________________________
%_______________________________________________________________________________
%_______________________________________________________________________________

\cmsNoteHeader{SUS-16-038}

\title{Search for supersymmetry in pp collisions using final states
  with at least one jet and missing transverse momentum}
\titlerunning{Search for new physics in pp collisions using final
  states with at least one jet and missing transverse momentum}

\address[cern]{CERN} 
\author[cern]{The CMS Collaboration}

\date{\today}

\abstract{A search for supersymmetry is performed in final states
  containing one or more jets and an imbalance in transverse
  momentum. A sample of events resulting from pp collisions at a
  centre-of-mass energy of 13\TeV is analysed. The event data were
  recorded with the CMS detector at the CERN LHC in 2016 and
  correspond to an integrated luminosity of 35.9\fbinv. The search
  provides sensitivity to models that predict a stable weakly
  interacting massive particle, such as the neutralino. The search
  employs several kinematic variables to suppress multijet production
  to a subdominant level with respect to all other standard model
  background processes. Further kinematic variables are used for
  signal extraction using a binned maximum-likelihood fit to the
  data. The number of candidate signal events is found to agree with
  the expected event counts from standard model processes.  The result
  is interpretated using simplified supersymmetric models that assume
  the gluino-mediated or direct production of bottom squark pairs,
  which each decay to a neutralino and a bottom quark. For the
  gluino-mediated scenario, gluinos and neutralinos up to X and Y\GeV,
  respectively, are probed. For the direct production mode, bottom
  squarks and neutralinos up to X and Y\GeV are probed.
%  A number of simplified supersymmetric models that assume the
%  production of gluino or squark pairs that decay to neutralinos and
%  standard model particles, are used to interpret the result. For
%  models that assume the production of gluino pairs, gluino and
%  neutralino masses up to X and Y\GeV, respectively, are excluded. For
%  models that assume In the case of the production of squark pairs,
%  light-flavour, bottom, and top squarks masses up to X, Y, and Z\GeV
%  are excluded, respectively. Models with near-degenerate mass spectra
%  are also considered. For models that assume top-squark production
%  and their decays to nearly mass-degenerate neutralinos, top squark
%  masses up to X\GeV are excluded. 
}

\hypersetup{ 
  pdfauthor={Robert Bainbridge, Freya Blekman, Shane Breeze, Oliver
    Buchmueller, Stefano Casasso, Matthew Citron, Adam Elwood, Henning
    Flaecher, Aran Garcia-Bellido, Christian Laner, Kin Ho Lo, Sarah
    Alam Malik, Bjoern Penning, Tai Sakuma, Dominic Smith, Alex
    Tapper},
  pdftitle={Search for supersymmetry in pp collisions using final
    states with at least one jet and missing transverse momentum},
  pdfsubject={CMS},
  pdfkeywords={CMS, jets, monojet, missing transverse momentum,
    supersymmetry, dark matter, AlphaT} 
}

\maketitle

%_______________________________________________________________________________
%_______________________________________________________________________________
%_______________________________________________________________________________

\section{Introduction}
\label{sec:introduction}

Supersymmetry (SUSY)~\cite{ref:SUSY-1, ref:SUSY0, ref:SUSY3,
  ref:SUSY1} is an extension of the standard model (SM) of particle
physics, which introduces at least one bosonic (fermionic)
superpartner for each fermionic (bosonic) SM particle, which differ in
spin by one-half unit. The particle content of the minimal
supersymmetric standard model~\cite{ref:SUSY2} can be summarised as
follows. The superpartners to the gluons, quarks, and leptons are the
gluinos $\PSg$, the light-flavour $\PSQ$, bottom $\PSQb$, and top
$\PSQt$ squarks, and the sleptons $\PSl$, respectively. An extended
Higgs sector comprising a quintet of scalar particle states is
predicted. The higgsino and gaugino superpartners to the Higgs and
electroweak gauge bosons are expected to mix to give six observable
states: two charginos $\PSGcpm_{1,2}$ and four neutralinos
$\PSGcz_{1,2,3,4}$. 

SUSY has several appealing features, including the possibility to
unify the gauge coupling constants at high
energy~\cite{Dimopoulos:1981yj, Ibanez:1981yh, Marciano:1981un}, to
provide a solution to the gauge hierarchy
problem~\cite{ref:hierarchy1, ref:hierarchy2}, and to provide a dark
matter (DM) candidate. The gauge hierachy problem of the SM, in which
corrections from loop processes lead to a Higgs boson mass \higgsmass
close to the cutoff scale for the theory, which can only be avoided by
an extreme fine tuning of the bare Higgs boson mass parameter. The
presence of superpartners can alleviate this problem by cancelling the
contributions to \higgsmass from SM loop processes. So-called
``natural'' SUSY models require only a minimal fine-tuning of the bare
Higgs boson mass parameter if the gluino, third-generation squarks,
and a (higgsino-like) $\PSGczDo$ are at or near the electroweak
scale~\cite{ref:barbierinsusy}. These models have garnered
considerable interest~\cite{Delgado:2012eu, Boehm:1999tr,
  Carena:2008mj, Grober:2014aha, Grober:2015fia, Boehm:1999bj,
  Balazs:2004bu, Martin:2007gf, Martin:2007hn} since the discovery of
the Higgs boson~\cite{ref:atlashiggsdiscovery, ref:cmshiggsdiscovery,
  ref:cmshiggsdiscoverylong} at $\higgsmass = 125\GeV$. Further, the
assumption of $R$-parity conservation~\cite{Farrar:1978xj} has
important consequences for cosmology and collider phenomenology. SUSY
particles are expected to be produced in pairs at the LHC, with heavy
states decaying eventually to the lightest stable SUSY particle
(LSP). The LSP is generally assumed to be the $\PSGczDo$, which is
weakly interacting and massive. This SUSY particle is considered to be
a candidate for DM~\cite{Jungman:1995df}, the existence of which is
supported by astrophysical data~\cite{1674-1137-38-9-090001}. Hence,
the characteristic signature of natural SUSY production at the LHC is
a final state containing an abundance of jets, originating from top or
bottom quarks, accompanied by a significant imbalance in transverse
momentum, \ptvecmiss.

This paper presents a search for new physics processes in final states
with one or more energetic jets and significant \ptvecmiss. The search
is performed with a sample of proton-proton (pp) collision data at a
centre-of-mass energy of 13\TeV. The analysed data sample corresponds
to an integrated luminosity of $35.9 \pm 1.0\fbinv$ recorded by the
CMS experiment. Earlier searches using the same technique have been
performed in pp collisions at $\sqrt{s} = 7$, 8, and
13\TeV~\cite{RA1Paper, RA1Paper2011, RA1Paper2011FULL, RA1Paper2012,
  RA1Parked, RA1Paper2015} by the CMS Collaboration. The data analysed
in this analysis is a factor $\sim$16 larger than that presented in
Ref.~\cite{RA1Paper2015}, which results in a significant gain in
sensitivity to the production of \TeV-scale coloured SUSY particles,
particularly for models with nearly mass-degenerate spectra that are
characterised by soft final state signatures and relatively low
experimental acceptance. The ATLAS and CMS Collaborations have
performed similar searches in all-jets final states at $\sqrt{s} =
13\TeV$, of which the most constraining are described in
Refs.~\cite{}.

The strategy of this search aims to provide sensitivity to a broad
range of SUSY and non-SUSY models that postulate the existence of a
stable, weakly interacting, massive particle. The overwhelmingly
dominant background for a new-physics search in all-jet final states
resulting from proton-proton collisions is multijet production, a
manifestation of quantum chromodynamics. The search employs several
dedicated variables to discriminate against this background while
maintaining experimental acceptance to events characterised by the
presence of significant \ptvecmiss. Signal extraction is performed via
several discriminating kinematic variables: the number of
reconstructed jets per event, the number of these jets identified as
originating from bottom quarks, and the scalar and vector \pt sums of
these jets. The strong discrimination against the multijet background
permits the application of low thresholds on kinematic variables. This
allows to maintain a large acceptance to a broad range of models, such
as those that assume the strong production of massive coloured SUSY
particles, including third-generation squark signatures, and both
large and small mass splittings between the parent SUSY particle and
the LSP. The search also considers final states containing a
``monojet'' topology, which is expected to improve the sensitivity to
DM particle production in pp collisions~\cite{Fox:2012ee,
  Buchmueller:2015eea}.

The results of the search are used to constrain the mass parameter
space of simplified SUSY models~\cite{Alwall:2008ag, Alwall:2008va,
  sms}. Models that assume the direct or gluino-mediated production of
bottom squark pairs, and their subsequent (prompt) decays to SM
particles and the LSP, are examined. Gluinos are assumed to undergo
three-body decays via off-shell bottom squarks. For each production
and decay mode, a range of differences in mass (\dm) between the
parent SUSY particle and the LSP are considered. 

This paper is organised as follows. Sections~\ref{sec:detector}
and~\ref{sec:simulation} describe the CMS apparatus and the various
software packages used to generate samples of simulated events,
respectively. Sections~\ref{sec:reconstruction}
and~\ref{sec:selection} provide details on, respectively, the
reconstruction algorithms and selection criteria used to identify
candidate signal events and control
samples. Section~\ref{sec:backgrounds} describes the methods used to
estimate the background contributions from SM processes. The search
results and physics interpretations are described in
Sections~\ref{sec:result} and~\ref{sec:interpretations},
respectively. Finally, we summarise in Section~\ref{sec:summary}.

%_______________________________________________________________________________

%The gauge hierachy problem of the SM, in which corrections from loop
%processes lead to a Higgs boson mass \higgsmass close to the cutoff
%scale for the theory, which can only be avoided by an extreme fine
%tuning of the bare Higgs boson mass parameter. The presence of
%superpartners can alleviate this problem by cancelling the
%contributions to \higgsmass from SM loop processes. So-called
%``natural'' SUSY models require only a minimal fine-tuning of the bare
%Higgs boson mass parameter if the gluino, third-generation squarks,
%and a (higgsino-like) $\PSGczDo$ are at or near the electroweak
%scale~\cite{ref:barbierinsusy}. This solution has attracted
%considerable interest~\cite{Delgado:2012eu, Boehm:1999tr,
%  Carena:2008mj, Grober:2014aha, Grober:2015fia} following the
%discovery of the Higgs boson~\cite{ref:atlashiggsdiscovery,
%  ref:cmshiggsdiscovery, ref:cmshiggsdiscoverylong} at $\higgsmass =
%125\GeV$. Models with light and nearly degenerate $\PSQt$ and
%$\PSGczDo$ masses are also well motivated ~\cite{Boehm:1999bj,
%  Balazs:2004bu, Martin:2007gf, Martin:2007hn}. 

%This paper presents a search for new physics processes in final states
%with one or more energetic jets and significant \ptvecmiss. The search
%is performed with a sample of proton-proton (pp) collision data at a
%centre-of-mass energy of 13\TeV. The analysed data sample corresponds
%to an integrated luminosity of $36.4 \pm 1.0\fbinv$ recorded by the
%CMS experiment. Earlier searches using the same technique have been
%performed in pp collisions at both $\sqrt{s} = 7$~\cite{RA1Paper,
%  RA1Paper2011, RA1Paper2011FULL}, 8~\cite{RA1Paper2012, RA1Parked},
%and 13\TeV~\cite{RA1Paper2015} by the CMS Collaboration. Several
%similar searches have already been performed at $\sqrt{s} = 13\TeV$ by
%the ATLAS~\cite{} and CMS~\cite{} collaborations. These searches
%extend the reach of the most constraining searches performed during
%Run~1 at $\sqrt{s} = 7$ and 8\TeV by the ATLAS~\cite{Aad:2015iea,
%  Aad:2015pfx} (and references therein) and CMS~\cite{}
%collaborations. The data analysed in this analysis is a factor
%$\sim$16 larger than that presented in Ref.~\cite{RA1Paper2015}, which
%provides a significant gain in sensitivity to the production of
%\TeV-scale coloured SUSY particles, particularly for models with
%nearly mass-degenerate spectra that are characterised by soft final
%state signitures and relatively low experimental acceptance.

%The search strategy aims to provide robust sensitivity to a wide range
%of SUSY and non-SUSY models that postulate the existence of a stable,
%weakly interacting, massive particle. The overwhelmingly dominant
%background for a new-physics search in all-jet final states resulting
%from proton-proton collisions is multijet production, a manifestation
%of quantum chromodynamics. The search employs several dedicated
%variables to discriminate against this background while maintaining
%experimental acceptance to events characterised by the presence of
%significant \ptvecmiss. Signal extraction is performed via several
%discriminating kinematic variables: the number of reconstructed jets
%per event, the number of these jets identified as originating from
%bottom quarks, and the scalar and vector \pt sums of these jets. The
%strong suppression of the multijet background permits the application
%of low thresholds on kinematic variables. This allows to maintain a
%large acceptance to a broad range of models, such as those that assume
%the strong production of massive coloured SUSY particles, including
%third-generation squark signatures, and both large and small mass
%splittings between the parent SUSY particle and the LSP. The search
%also considers final states containing a ``monojet'' topology, which
%is expected to improve the sensitivity to DM particle production in pp
%collisions~\cite{Fox:2012ee, Buchmueller:2015eea}.

%The results of the search are used to constrain the mass parameter
%space of simplified SUSY models~\cite{Alwall:2008ag, Alwall:2008va,
%  sms}. Nine unique combinations of production and decay modes are
%considered. Models that assume the direct or gluino-mediated
%production of squark pairs, and their subsequent (prompt) decays to SM
%particles and the LSP, are examined. Gluinos are assumed to undergo
%three-body decays via off-shell squarks. Further, both light and heavy
%flavours of squarks are considered, as well as a range of differences
%in mass (\dm) between the parent SUSY particle and the LSP. In the
%case of top squark decays, various decay modes, dependent on \dm, are
%considered.

%This paper is organised as follows. Sections~\ref{sec:detector}
%and~\ref{sec:simulation} describe the CMS apparatus and the various
%software packages used to generate samples of simulated events,
%respectively. Sections~\ref{sec:reconstruction}
%and~\ref{sec:selection} provide details on, respectively, the
%reconstruction algorithms and selection criteria used to identify
%candidate signal events and control
%samples. Section~\ref{sec:backgrounds} describes the methods used to
%estimate the background contributions from SM processes. The search
%results and physics interpretations are described in
%Sections~\ref{sec:result} and~\ref{sec:interpretations},
%respectively. Finally, we summarise in Section~\ref{sec:summary}.

%_______________________________________________________________________________
%_______________________________________________________________________________
%_______________________________________________________________________________

%\clearpage
\section{The CMS detector}
\label{sec:detector}

The central feature of the CMS apparatus is a superconducting solenoid
of 6\unit{m} internal diameter, providing a magnetic field of
3.8\unit{T}. Within the solenoid volume are a silicon pixel and strip
tracker, a lead tungstate crystal electromagnetic calorimeter (ECAL),
and a brass and scintillator hadron calorimeter (HCAL), each composed
of a barrel and two endcap sections. Forward calorimeters extend the
pseudorapidity coverage provided by the barrel and endcap
detectors. Muons are measured in gas-ionization detectors embedded in
the steel flux-return yoke outside the solenoid. Events of interest
are selected using a two-tiered trigger
system~\cite{Khachatryan:2016bia}. The first level (L1), composed of
custom hardware processors, uses information from the calorimeters and
muon detectors to select events at a rate of around 100\unit{kHz}
within a time interval of less than 4\mus. The second level, known as
the high-level trigger (HLT), consists of a farm of processors running
a version of the full event reconstruction software optimized for fast
processing, and reduces the event rate to less than 1\unit{kHz} before
data storage.  A more detailed description of the CMS detector,
together with a definition of the coordinate system used and the
relevant kinematic variables, can be found in
Ref.~\cite{Chatrchyan:2008zzk}.

%_______________________________________________________________________________
%_______________________________________________________________________________
%_______________________________________________________________________________

%\clearpage
\section{Simulated event samples}
\label{sec:simulation}

%_______________________________________________________________________________
%_______________________________________________________________________________
%_______________________________________________________________________________

%\clearpage
\section{Event reconstruction}
\label{sec:reconstruction}

The physics objects used by this search are determined from particle
candidates provided by the particle-flow (PF) event algorithm, which
aims to reconstruct and identify each individual particle in an event
with an optimized combination of information from the various elements
of the CMS detector~\cite{CMS-PAS-PFT-09-001, CMS-PAS-PFT-10-001}.

The energy of photons is directly obtained from the ECAL measurement,
corrected for zero-suppression effects. The energy of electrons is
determined from a combination of the electron momentum at the primary
interaction vertex as determined by the tracker, the energy of the
corresponding ECAL cluster, and the energy sum of all bremsstrahlung
photons spatially compatible with originating from the electron
track. The energy of muons is obtained from the curvature of the
corresponding track. The energy of charged hadrons is determined from
a combination of their momentum measured in the tracker and the
matching ECAL and HCAL energy deposits, corrected for zero-suppression
effects and for the response function of the calorimeters to hadronic
showers. Finally, the energy of neutral hadrons is obtained from the
corresponding corrected ECAL and HCAL energy.

Jets are reconstructed from the PF candidate particles, clustered by
the anti-$k_\mathrm{t}$ algorithm~\cite{Cacciari:2008gp,
  Cacciari:2011ma} with a size parameter of 0.4. In this process, raw
jet energy is obtained from the sum of the candidate particle
energies, and the raw jet momentum by the vectorial sum of the
candidate particle momenta, which results in a nonzero jet
mass. Selection criteria are applied to each event to remove spurious
jet-like features originating from isolated noise patterns from or
inefficiencies within the calorimeter systems. Further, the fraction
of candidate particles within jets that are identified as charged and
hadronic in nature (\eg charged pions), $f_{h^{\pm}}$, is required to
be within the range 0.1--0.95 to suppress the effects of beam halo and
rare reconstruction failures. An offset correction is applied to jet
energies to take into account the contribution from additional
proton-proton interactions within the same or nearby bunch
crossings~\cite{pileup}.  The raw jet energies are then corrected to
establish a relative uniform response of the calorimeter in $\eta$ and
a calibrated absolute response in transverse momentum \pt. Jet energy
corrections are derived from simulation, and are confirmed with in
situ measurements of the energy balance in dijet and photon + jet
events~\cite{Chatrchyan:2011ds}. The jet energy resolution amounts
typically to 15\% at 10\GeV, 8\% at 100\GeV, and 4\% at 1\TeV,
respectively.
%Jet momentum is determined from simulation to be within 5 to 10\% of
%the true momentum over the whole \pt spectrum and detector acceptance.

Jets are identified as originating from b quarks using the combined
secondary vertex algorithm~\cite{CMS-PAS-BTV-12-001}. Data
samples~\cite{bjets} are used to measure the probability of correctly
identifying jets as originating from b quarks (b tagging efficiency),
and the probability of misidentifying jets originating from
light-flavour partons (u, d, s quarks or gluons) or a charm quark as a
b-tagged jet (the light-flavour and charm mistag probabilities). A
working point is employed that yields a b tagging efficiency of 65\%,
and charm and light-flavour mistag probabilities of approximately 12
and 1\%, respectively, for jets with \pt that is typical of \ttbar
events.

The most accurate estimator for the missing transverse momentum vector
\ptvecmiss is defined as the projection on the plane perpendicular to
the beams of the negative vector sum of the momenta of all PF
candidate particles in an event. Its magnitude is referred to as
\met.

%_______________________________________________________________________________
%_______________________________________________________________________________
%_______________________________________________________________________________

%\clearpage
\section{Event selection}
\label{sec:selection}

\subsection{Baseline selection for all-jets events}

In order to suppress SM processes with genuine \ptvecmiss from
neutrinos and select only multijet final states, events containing an
isolated electron~\cite{PAS-EGM-10-004} or muon~\cite{PAS-MUO-10-004}
with $\Pt > 10\GeV$ or isolated photon~\cite{PAS-EGM-10-006} with $\pt
> 25\GeV$ are vetoed. Furthermore, events containing an isolated track
with $\Pt > 10\GeV$ are also vetoed in order to reduce the background
contribution from final states containing hadronically-decaying tau
leptons. 

A number of beam- and detector-related effects can lead to events with
large values of \met, such as beam halo, reconstruction failures, 
spurious detector noise, or event misreconstruction due to detector
inefficiencies. These events, with large, non-physical values of \met,
are rejected with high efficiency by applying a range of dedicated
vetoes~\cite{RA1Paper2012, cms-met}.

Jets considered in the analysis are required to have a transverse
momentum above $40\GeV$ and $|\eta| < 2.4$. The most energetic jet in
the event is required to satisfy $\Pt > 100\GeV$. An estimate of the
mass scale of the physics process being probed is given qby the scalar
sum of the transverse momenta $\Pt$ of these jets, defined as $\scalht
= \sum_{i=1}^{N_{\rm jet}} \Pt^{\,\mathrm{j}_i}$, where $N_{\rm jet}$
is the number of jets within the experimental acceptance. The
estimator for \met used by this search is given by the magnitude of
the vector sum of the transverse momenta of these jets, $\mht =
|\sum_{i=1}^{N_\text{jet}} \ptvec^{\,\mathrm{j}_i}|$. Significant
hadronic activity and \ptvecmiss in the event is ensured by requiring
$\scalht > 200\GeV$ and $\mht > 200\GeV$, respectively.

Events are vetoed if any additional jet satisfies $\Pt > 40\GeV$ and
$|\eta| > 2.4$, in order to maintain the performance of the variable
\mht as an estimator of \met. An additional dedicated veto is employed
to deal with the circumstance in which several jets with transverse
momentum below the \Pt thresholds and collinear in $\phi$ can result
in significant \mht relative to \met, the latter of which is less
sensitive to jet thresholds. This type of background, typical of
multijet events, is suppressed while maintaining high efficiency for
SM or new physics processes with significant \ptvecmiss by requiring
$\mhtmet < 1.25$.

\begin{table}[!tb]
  \topcaption{Summary of the event selection requirements and
    categorisations used to define the signal and control regions.}
  \label{tab:selections}
  \centering
  \footnotesize
  \begin{tabular}{ ll }
    \hline
    \multicolumn{2}{l}{\bf Baseline selection for all-jets final states}\T\B                                                                        \\
    Lepton/photon vetoes            & $\Pt > 10,\, 10,\, 25\GeV$ for isolated tracks, leptons, photons (respectively) and $\abs{\eta} < 2.5$        \\ 
    Jet $j_\text{i}$ acceptance     & Consider each jet $j_\text{i}$ that satisfies $\Pt^{j_\text{i}} > 40\GeV$ and $\abs{\eta^{j_\text{1}}} < 2.4$ \\
    Jet $j_\text{1}$ acceptance     & $\Pt^{j_\text{1}} > 100\GeV$                                                                                  \\
    \met cleaning                   & Filters related to beam and instrumental effects                                                              \\ 
    Beam halo                       & Charged hadron fraction for $j_\text{1}$, $0.1 < f_{h^{\pm}} < 0.95$                                          \\
    Energy sums                     & $\scalht > 200\GeV$ and $\mht > 200\GeV$                                                                      \\
    Forward jet veto                & Veto events containing jet satisfying $\Pt > 40\GeV$ and $\abs{\eta} > 2.4$                                   \\
    Jets below threshold \B         & $\mhtmet < 1.25$                                                                                              \\
    \hline
    \multicolumn{2}{l}{\bf Event categorisation}\T\B                                                                                                \\
    \njet\;(monojet)                & 1\phantom{2, 3, 4, 5, $\geq$} \quad($\Pt^{j_\text{2}} < 40\GeV$, \ie outside acceptance)                      \\
    \phantom{\njet}\;(asymmetric)   & $\geq$2\phantom{2, 3, 4, 5, } \quad($40 < \Pt^{j_\text{2}} < 100\GeV$)                                        \\
    \phantom{\njet}\;(symmetric)    & 2, 3, 4, 5, $\geq$6 \quad($\Pt^{j_\text{2}} > 100\GeV$)                                                       \\
    \nb                             & 0, 1, 2, 3, $\geq$4 \quad($\nb \leq \njet$)                                                                   \\
    \scalht (GeV)                   & 200, 400, 600, 900, $>$1200\GeV (categories can be dropped/merged \vs \njet)                                  \\
    \mht (GeV) \B                   & 200, 400, 600, $>$900\GeV (categories can be dropped/merged \vs \njet, \scalht, \nb)                          \\
    \hline
    {\bf Signal region (SR)}        & Baseline selection + \T\B                                                                                     \\
    Multijet rejection \quad        & $\alphat > 0.65$--0.52 (\scalht-dependent requirements for the range $200 < \scalht < 900\GeV$)               \\
    Multijet rejection              & $\bdphi > 0.5$ ($\njet \geq 2$), $\bdphimod > 0.5$ ($\njet = 1$) \B                                           \\[0.5ex]
    \hline
    {\bf Control regions (CR)} \T\B & Baseline selection +                                                                                          \\
    \mj                             & 
    1$\mu$ with $\Pt > 30\GeV$, $\abs{\eta} < 2.1$, 
%    $I^{\mu}_\text{rel} < 0.1$, 
    $\Delta R(\mu,j_{\text{i}}) > 0.5$,
    $30 < m_\text{T}(\ptvec^\mu,\ptvecmiss) < 125\GeV$                                                                                              \\[0.5ex]
    \mmj                            & 
    2$\mu$ with $\Pt > 30\GeV$, $\abs{\eta} < 2.1$, 
%    $I^{\mu}_\text{rel} < 0.1$, 
    $\Delta R(\mu_{1,2},j_{\text{i}}) > 0.5$, 
    $ \abs{m_{\mu\mu} - m_\text{Z}} < 25\GeV$                                                                                                       \\[0.5ex]
    Multijet-enriched \B            & SR + (inverted) requirements $\mht/\met > 1.25$ and/or $\bdphi < 0.5$                                         \\  
    \hline
  \end{tabular}
\end{table}

The aforementioned selection requirements, summarised in
Table~\ref{tab:selections}, define a baseline set that are common to
all data samples used in the analysis. Additional requirements,
described in Sec.~\ref{sec:signalregion} below, are employed to define
a sample of candidate signal events, labelled henceforth as the signal
region. Three control regions are used to estimate the background
contributions from SM processes in the signal region. The selection
requirements that defined the event samples for these control regions
are described in Sec.~\ref{sec:control} and modify and expand on the
baseline selection requirements.

\subsection{Event categorisation}
\label{sec:categorisation}

Candidate signal events, as well as events in the control regions, are
categorised as a function of \njet, the number of jets identified
(``tagged'') as originated from b quarks \nb, \scalht, and \mht. The
categorisatio is determined primarily by the statistical power of the
\mj and \mmj event samples.

Seven categories in \njet are considered, as summarised in
Table~\ref{tab:selections}. Events that contain only one jet ($\njet =
1$) satisfying the requirement $\Pt > 40\GeV$ are labelled as
``monojet''. Events containing two or more jets are categorised
according to the \Pt of the second-most energetic jet. Events that
satisfy $\njet \geq 2$ with only the most energetic jet satisfying
$\Pt > 100\GeV$ are labelled as ``asymmetric''. Events for which the
second-most energetic jet also satisfies $\Pt > 100\GeV$ are labelled
as ``symmetric'' and are categorised according to \njet (2, 3, 4, 5,
and $\geq$6). The symmetric topology targets the pair production of
SUSY particles and their cascade decays, while the monojet and
asymmetric topologies target nearly mass-degenerate SUSY models, as
well as the direct production of weakly interacting massive particles.

\begin{table}[!tb]
  \topcaption{Summary of the lower bound on \scalht [GeV] for the
    ({\it first, final open}) categories used in the search as a
    function of \njet and \nb. A dash (--) signifies the categories
    that are not used. The label (asym) signifies the asymmetric
    topology.
  }
  \newcommand{\ph}{\phantom{1}}
  \label{tab:categorisation}
  \centering
  \begin{tabular}{ lccccc }
    \hline
    $\njet\, / \,\nb$ & 0             & 1             & 2             & 3             & $\geq$4    \\
    \hline
    1                 & (200, \ph900) & (200, \ph900) & --            & --            & --         \\ 
    $\geq$2 (asym)    & (200, \ph900) & (200, \ph900) & (200, \ph900) & (200, \ph600) & --         \\ 
    2                 & (200, 1200)   & (200, 1200)   & (200, \ph900) & --            & --         \\ 
    3                 & (200, 1200)   & (200, 1200)   & (200, 1200)   & (200, \ph900) & --         \\ 
    4                 & (400, 1200)   & (400, 1200)   & (400, 1200)   & (400, \ph900) & --         \\ 
    5                 & (400, 1200)   & (400, 1200)   & (400, 1200)   & (400, \ph900) & (400, 400) \\ 
    $\geq$6           & (400, 1200)   & (400, 1200)   & (400, 1200)   & (400, 1200)   & (400, 400) \\ 
    \hline
  \end{tabular}
\end{table}

Events are also categorised according to \nb (0, 1, 2, 3, $\geq$4),
where \nb is bounded from above by \njet. The nomimal categorisation
scheme for \scalht is defined as follows: four bounded regions that
satisfy 200--400, 400--600, 600--900, and 900--1200\GeV, and a final
open region $\scalht > 1200\GeV$. Only the region $\scalht > 400\GeV$
is considered for events that satisfy $\njet \geq 4$. Further, the
categorisation by \nb and \scalht is adapted independently per \njet
category by removing or merging categories to satisfy a threshold on
the minimum number of data events in the control regions, which are
used to estimate SM backgrounds, provide checks, and validate
assumptions within the methods. The categorisation scheme is
summarised in Table~\ref{tab:categorisation}.

\begin{table}[!h]
  \topcaption{Summary of the lower bound on \mht [GeV] for the final
    category used in the search as a function of \njet and
    \scalht. The \mht category is bounded if the \scalht category is
    bounded, as indicated in Table~\ref{tab:categorisation}. A 
    dash (--) signifies the categories that are not used. The label
    (asym) signifies the asymmetric topology. 
  }
  \label{tab:mht-binning}
  \centering
  \begin{tabular}{ lccccc }
    \hline
    $\njet\, / \,\scalht$ [GeV] & 400--600 & 600--900 & 900--1200 & $>$1200 \\
    \hline
    2                           & 400      & 600      & --        & --      \\ 
    3                           & 400      & 600      & 900       & --      \\ 
    4                           & 400      & 600      & 900       & --      \\ 
    5                           & 400      & 600      & 900       & 400     \\ 
    $\geq$6                     & 400      & 600      & 900       & 400     \\ 
    \hline
  \end{tabular}
\end{table}

Finally, the \mht variable is also used to categorise events as
follows: three bounded regions that satisfy 200--400, 400--600, and
600--900, and a final open region $\scalht > 900\GeV$. By
construction, the maximum value for \mht is bounded from above by
\scalht and the allowed values are determined by the combination of
\njet and \scalht. As a result, the \mht variable is not employed for
events identified as monojet and asymmetric topologies or that satisfy
$200 < \scalht < 400\GeV$. The \mht categorisation scheme is identical
across all \nb categories and depends only on \njet and \scalht, as
summarised in Table~\ref{tab:mhtcat}.

The control regions employ an identical categorisation scheme except
for a more granular choice versus \scalht than used for the signal
region. Up to eleven categories are used, from which background
estimates are determined and then aggregated to map onto up to five
\scalht categories used for the signal region.  This approach allows
to correct simulated event samples based on granular measurements
determined from the data control regions, which in turn improves the
modelling of the SM backgrounds. The categorisation schemes and
mapping are summarised in Table~\ref{tab:thresholds}.



These categorisations, summarised in
Table~\ref{tab:selections}, are used identically for the signal region
and the four control samples. Finally, the search exploits the use of
the \mht variable as a discriminant between the dominant SM
backgrounds and new-physics signatures. The expected distribution of
events as a function of \mht is determined from simulation, an
approach that is validated in multiple data control samples.

\subsection{Signal region}
\label{sec:signalregion}

The multijet background dominates over all other SM backgrounds
following the application of the baseline event selection
criteria. The multijet background is suppressed to a negligible level
through the application of two dedicated variables that provide strong
discrimation between multijet events with a transverse momentum
imbalance resulting from instrumental sources, such as jet energy
mismeasurement, and new-physics processes with genuine \ptvecmiss
resulting from weakly interacting particles that escape detection.

The first variable, \alphat~\cite{Randall:2008rw, RA1Paper}, is
designed to be intrinsically robust against jet energy
mismeasurements. In its simplest form, the \alphat variable is defined
as $\alphat = \Et^{\rm j_2}/M_\text{T}$, where $M_\textrm{T} = \sqrt{
  2 \Et^{\rm j_1} \Et^{\rm j_2} (1 - \cos\phi_{\rm j_1,j_2})}$ where
$\phi_{\rm j_1,j_2}$ is defined as the azimuthal angle between jets
${\rm j_1}$ and ${\rm j_1}$ and $M_\textrm{T}$ is the transerve mass
of a dijet system. In the absence of jet energy mismeasurements, and
in the limit in which the \Et of each jet is large compared with its
mass, a well measured dijet event with $\Et^{\mathrm{j_1}} =
\Et^{\mathrm{j_2}}$ and back-to-back jets ($\phi_{\rm j_1,j_2} = \pi$)
yields an \alphat value of 0.5. In the presence of a jet energy
mismeasurement, $\Et^{\mathrm{j_1}} > \Et^{\mathrm{j_2}}$ and $\alphat
< 0.5$. Values significantly greater than 0.5 are observed when the
two jets are not back-to-back and recoil against \ptvecmiss from
weakly interacting particles that escape the detector. The definition
of the \alphat variable can be generalised for events with two or more
jets, as described in Ref.~\cite{}. Multijet events populate the
region $\alphat \lesssim 0.5$ and the \alphat distribution is
characterised by a sharp edge at 0.5, beyond which the multijet event
yield falls by several orders of magnitude. SM backgrounds that
involve the prompt production of neutrinos (\eg not semileptonic heavy
flavour decays) result in a long tail in \alphat beyond values of
0.5. A \scalht-dependent requirement on \alphat that decreases from
0.65 at low \scalht to 0.52 at high \scalht within the range $200 <
\scalht < 900\GeV$ is required to maintain a roughly constant
rejection power against the multijet background due to evolving jet
acceptance and resolution effects.

The second variable, known as \bdphi, considers the minimum azimuthal
angular separation between a jet and the vector sum determined from
the transverse momenta of all other jets in the event. The requirement
$\bdphi > 0.5$ is sufficient to effectively suppress the multijet
background. The \bdphi variable is particularly efficient at
suppressing rare contributions from energetic multijet events that
yield both high jet multiplicities and significant \met due to
high-multiplicity neutrino production in semileptonic heavy-flavour
decays. The neutrinos are typically collinear with respect to the axis
of a jet and carry a significant fraction of the energy. For monojet
events, a small modification to the \bdphi variable, which considers
soft jets with $\pt > 25\GeV$ ($\bdphimod$), is utilised. 

\begin{table}[tb!]
  \topcaption{Lower bound of \scalht categories used in the control
    regions, and \alphat and \bdphi thresholds as a function of
    \scalht. For monojet events, \bdphimod replaces \bdphi
    with identical thresholds.} 
  \label{tab:thresholds}
  \centering
  \begin{tabular}{ lccccccccccc }
    \hline
    Variable & \multicolumn{11}{c}{\scalht [GeV]}                                           \\
%    \cline{2-8}
             & 200  & 250  & 300  & 350  & 400  & 500  & 600  & 750  & 900 & 1050 & $>$1200 \\
    \hline                                                                         
    \alphat  & 0.65 & 0.60 & 0.55 & 0.53 & 0.52 & 0.52 & 0.52 & 0.52 & --  & --   & --      \\
    \bdphi   & 0.5  & 0.5  & 0.5  & 0.5  & 0.5  & 0.5  & 0.5  & 0.5  & 0.5 & 0.5  & 0.5 \B    \\
    \hline
  \end{tabular}
\end{table}

For the region $200 < \scalht < 900\GeV$, requirements on both the
\alphat and \bdphi variables are utilised, whereas for the region
$\scalht > 900\GeV$, the necessary control of the QCD multijet
background is achieved solely with the \bdphi requirement. The
requirements on \alphat and \bdphi are summarised in
Table~\ref{tab:thresholds}. The tight requirements on \alphat and
\bdphi suppress the expected contribution from multijet events to the
sub-percent level with respect to the total expected background counts
from other SM processes. The aforementioned requirements complete the
definition of the signal region, and are summarised in
Table~\ref{tab:selections}.

%Figure~\ref{fig:alphat-bdphi} shows the \alphat and \bdphi
%distributions observed in data for events that satisfy all other
%signal region selection criteria plus $\scalht > 300\GeV$ and $\scalht
%> 800\GeV$, respectively. In the case of the \alphat distribution, the
%events that satisfy $\alphat < 0.55$ must only fulfill the baseline
%selection criteria defined in Table~\ref{tab:selections}, no \mht
%requirement is made, and the events are recorded with an unbiased set
%of trigger \scalht conditions.

%\begin{figure*}[tb!]
%  \begin{center}
%    \includegraphics[width=0.49\textwidth]{alphaT_v4} \,
%    \includegraphics[width=0.49\textwidth]{bDPhi_v4} \\
%  \end{center}
%  \caption{(Left) The \alphat distribution observed in data for events
%    that are recorded with unbiased trigger conditions and satisfy the
%    baseline (full signal region) selection criteria for the region
%    $\alphat < 0.55$ ($\alphat > 0.55$). (Right) The \bdphi
%    distribution observed in data for events that satisfy the full
%    signal region selection criteria and $\scalht > 800\GeV$.  The
%    distributions for the QCD multijet backgrounds are determined from
%    simulation while all other SM backgrounds are estimated using a
%    $\mu$ + jets data control sample. %The uncertainties in the SM
%    %expectation are dominated by the statistical uncertainties 
%    %associated with the limited sample of simulated multijet events.
%    \label{fig:alphat-bdphi} 
%  }
%\end{figure*}

\begin{table}[tb!]
  \topcaption{Efficiency $\varepsilon_\textrm{trig}$ [\%] of the signal triggers as
    a function of \scalht after applying the signal region selection
    criteria, as determined from data. The first (asymmetric) and
    second set of are statistical and systematic in nature.
  } 
%  \footnotesize
  \centering
  \begin{tabular}{lcccc} 
    \hline
    \scalht [GeV]                    & 200--400 & 400--600 & 600--900 & $>$900 \\
    $\varepsilon_\textrm{trig}$ [\%] & $97.4^{+0.5}_{-0.6}{}^{+0.6}_{-0.6}$ 
                                     & $97.9^{+0.8}_{-1.2}{}^{+1.0}_{-1.0}$ 
                                     & $100.0^{+0.0}_{-1.8}{}^{+0.0}_{-0.6}$ 
                                     & $100.0^{+0.0}_{-3.6}{}^{+0.0}_{-0.1}$ \B  \\
    \hline
  \end{tabular}
  \label{tab:triggers}
\end{table}

Candidate signal events are recorded with a number of trigger
conditions. Trigger logic requiring the presence of significant \mht
and \met is used to record events containing one or more
jets. Additional logic is used that simultaneously requires
predetermined thresholds on \scalht and \alphat to be
satisfied. Finally, a trigger condition based solely on \scalht is
used to record candidate events for the region $\scalht >
900\GeV$. The combined trigger strategy provides high efficiencies for
all event categories of the the signal region, which are primarily
dependent on \scalht, as summarised in Table~\ref{tab:triggers}.

%_______________________________________________________________________________
%_______________________________________________________________________________
%_______________________________________________________________________________

\clearpage


%_______________________________________________________________________________

\clearpage
\bibliography{auto_generated}



%The first control sample is enriched in multijet events and is used to
%estimate the multijet contribution in the signal region. Three
%additional control samples comprising \gj, \mj, or \mmj events,
%defined by the baseline set of selections and the inversion of one of
%the photon or lepton vetoes, are used to estimate background
%contributions from SM processes, predominantly \wlj, \znunuj, and
%\ttbar production, that lead to final states containing jets and
%significant \ptvecmiss. Additional kinematic requirements are employed
%to ensure the control samples are enriched in the same SM processes
%that contribute to background events in the signal region, and are
%depleted in contributions from multijet production or a wide variety
%of SUSY models (\ie so-called signal contamination).  The control
%samples are defined such that the kinematic properties of events in
%the control regions and the candidate signal events resemble as
%closely as possible one another, once the photon, muon, or dimuon
%system is ignored in the calculation of quantities such as \scalht and
%\mht. The event selection requirements for the four control samples
%are summarised in Table~\ref{tab:selections}.
