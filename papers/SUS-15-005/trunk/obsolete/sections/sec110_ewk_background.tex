%%__________________________________________________________________||
\subsection{Backgrounds with genuine \texorpdfstring{\ETmiss}{MET}}
\label{sec:ewk_background}

Following the suppression of multijet events through the use of the
\alphat and \bdphi variables, the dominant nonmultijet backgrounds
involve SM processes that produce high-\Pt neutrinos in the final
state. In events with few jets or few b quark jets, the associated
production of W or Z bosons and jets, with the decays $\PW^\pm
\!\rightarrow\! \ell\nu$ ($\ell=\Pe$, $\Pgm$, $\Pgt$) or \znunu,
dominate the background counts. For W boson decays that yield an
electron or muon (possibly originating from leptonic $\Pgt$ decays),
the background contributions result from events containing an $\Pe$ or
$\Pgm$ that are not rejected by the lepton vetoes. 
%the background arises when the event containing the $\Pe$ or $\Pgm$ is
%not rejected by the lepton vetoes.
The veto of events containing at least one isolated track further
suppresses these backgrounds, including those from single-prong
$\tau$-lepton decays. At higher jet or b-quark jet multiplicities,
single top quark and \ttbar production, followed by semileptonic top
quark decay, also become an important source of background.

The method to estimate the nonmultijet backgrounds in the signal
region relies on the use of a transfer (\tf) factor determined from
simulation that is constructed per bin (in terms of \njet, \nb, and
\scalht) per control region. Each \tf factor is defined as the ratio
of the expected yields in the same (\njet, \nb, \scalht) bins of the
signal region $\mathcal{N}^\text{SR}_\text{MC}$ and one of the control
regions $\mathcal{N}^\text{CR}_\text{MC}$.  The \tf factors are used to
extrapolate from the event yields observed in each bin of a data
control sample $\mathcal{N}^\text{CR}_\text{data}$ to provide an
estimate for the background, integrated over \HTmiss, from a
particular SM process or processes in the corresponding bin of the
signal region $\mathcal{N}^\text{SR}_\text{data}$.  The superscript SR
or CR refers to, respectively, the process or processes being
estimated and one of the \mj, \mmj, and \gj control regions, described
in Section~\ref{sec:control_regions}. The subscript refers to whether
the counts are obtained from data, simulation (``MC''), or an estimate
(``pred'').

The method aims to minimise the effects of simulation mismodelling, as
many systematic biases in the simulation are expected to largely
cancel in the \tf factors, given that the events in any given (\njet, \nb,
\scalht) bin of the control regions closely mirror those in the
corresponding bin in the signal region in terms of the event scale,
topology, and kinematics, and the relative background compositions. In
short, minimal extrapolations are made. Uncertainties in the \tf factors
are determined from data, as described below.

Three independent estimates of the irreducible background of \znunu +
jets events are determined from the \gj, \mmj, and \mj data control
samples. The \gj and \zmumu + jets processes have similar kinematic
properties when the photon or muons are ignored~\cite{Bern:2011pa},
albeit different acceptances. In addition, the \gj process has a
larger production cross section than \znunu + jets events. The \mj
data sample is used to provide an estimate for both the \znunu\ + jets
background, as well as the other dominant SM processes, \ttbar and W
boson production (labelled collectively as $\PW/\ttbar$). Residual
contributions from all other SM relevant processes, such as single top
quark, diboson, and Drell--Yan production, are also included as part
of the $\PW/\ttbar$ estimate from the \mj sample. The definition of
the various \tf factors used in the search are given below:

\begin{align} 
  \mathcal{N}^\text{\PW/\ttbar}_\text{pred} \, & = \,
  \tf^\text{\PW/\ttbar}_{\mj} \; 
  \mathcal{N}^\text{\mj}_\text{data}, &
  \tf^\text{\PW/\ttbar}_{\mj} \, & = \,
  \bigg( 
  \frac{\,\mathcal{N}^{\PW/\ttbar}_\text{MC}\,}
  {\mathcal{N}^\text{\mj}_\text{MC}}
  \bigg); \\
  \mathcal{N}^{\znunu}_\text{pred} \, & = \,
  \tf^{\text{\znunu}}_{\mj} \; 
  \mathcal{N}^\text{\mj}_\text{data}, &
  \tf^{\text{\znunu}}_{\mj} \, & = \,
  \bigg( 
  \frac{\,\mathcal{N}^\text{\znunu}_\text{MC}\,}
  {\mathcal{N}^\text{\mj}_\text{MC}}
  \bigg); \\
  \mathcal{N}^\text{\znunu}_\text{pred} \, & = \,
  \tf^{\znunu}_{\mmj}  \;
  \mathcal{N}^\text{\mmj}_\text{data}, &
  \tf^{\znunu}_{\mmj}  \, & = \,
  \bigg( 
  \frac{\,\mathcal{N}^\text{\znunu}_\text{MC}\,}
  {\mathcal{N}^\text{\mmj}_\text{MC}}
  \bigg); \\
  \mathcal{N}^\text{\znunu}_\text{pred} \, & = \,
  \tf^{\znunu}_{\gj}  \;
  \mathcal{N}^\text{\gj}_\text{data}, &
  \tf^{\znunu}_{\gj}  \, & = \,
  \bigg( 
  \frac{\,\mathcal{N}^\text{\znunu}_\text{MC}\,}
  {\mathcal{N}^\text{\gj}_\text{MC}}
  \bigg).
\end{align} 

The likelihood function, described in Section~\ref{sec:result},
encodes the estimate via the \tf factors of the $\PW/\ttbar$ background, as
well as the three independent estimates of the \znunu background,
which are considered simultaneously.

Several sources of uncertainty in the \tf factors are evaluated.  The most
relevant effects are discussed below, and generally fall into one of
two categories. The first category concerns uncertainties in the
``scale factor'' corrections applied to simulation, which are
determined using inclusive data samples that are defined by loose
selection criteria, to account for the mismodelling of theoretical and
experimental parameters. The second category concerns ``closure
tests'' in data that probe various aspects of the accuracy of the
simulation to model correctly the \tf factors in the phase space of this
search.

The uncertainties in the \tf factors are studied for variations in
scale factors related to the jet energy scale (that result in
uncertainties in the \tf factors as large as $\sim$15\%), the
efficiency and misidentification probability of b quark jets (up to
5\%), and the efficiency to identify well-reconstructed, isolated
leptons (up to $\sim$5\%). A 5\% uncertainty in the total inelastic
cross section, $\sigma_\text{in} = 69.0 \pm
3.5\unit{mb}$~\cite{Aaboud:2016mmw}, is assumed and propagated through
to the reweighting procedure to account for differences between the
simulated measured pileup, which results in changes of up to
$\sim$10\%. The modelling of the transverse momentum of top quarks
($\Pt^\text{t}$) is evaluated by comparing the simulated and measured
\Pt spectra of reconstructed top quarks in \ttbar
events~\cite{Khachatryan:2015oqa}.  Simulated events are reweighted
according to scale factors that decrease from a value of $\sim$1.2 to
$\sim$0.7, with uncertainties of $\sim$10--20\%, within the range
$\Pt^\text{t} < 400\GeV$.
%The resulting change in the \tf factors is as large as $\sim$20\%.
The systematic uncertainties in $\tf^\text{\PW/\ttbar}_{\mj}$ arising
from variations in the $\Pt^\text{t}$ scale factors are typically
small ($\lesssim 5\%$), due to the comparable phase space probed by the
signal and control regions, while larger uncertainties
($\lesssim 20\%$) in $\tf^{\text{\znunu}}_{\mj}$ are observed due to
the potential for significant contamination from \ttbar when using
\wlj to predict \znunuj.

\newcommand{\phh}{\ensuremath{\phantom{1-}}}
\begin{table*}[h!]
  \caption{
    Systematic uncertainties (in percent) in the transfer (\tf) factors 
    used in the method to estimate the SM backgrounds with genuine
    \ptvecmiss in the signal region. The quoted ranges provide
    representative values of the observed variations as a function of
    \njet and \scalht. 
  } 
  \label{tab:bkgd_systs}
  \centering
  \footnotesize
  \begin{tabular}{ lrrrr }
    \hline
    Systematic source            & \multicolumn{4}{c}{Uncertainty in \tf factor [\%]} \\ 
    \cline{2-5} 
%    source                      & $\mj \Rightarrow \ttbar/\PW$ 
%                                & $\mj \Rightarrow \znunu$ 
%                                & $\mmj \Rightarrow \znunu$ 
%                                & $\gj \Rightarrow \znunu$                           \\
                                 & $\tf^\text{\PW/\ttbar\T}_{\mj\B}$
                                 & $\tf^{\text{\znunu}}_{\mj}$ 
                                 & $\tf^{\znunu}_{\mmj}$       
                                 & $\tf^{\znunu}_{\gj}$                               \\       
    \hline                                                    
    \multicolumn{5}{l}{\it Scale factors (applied to simulation):}                    \\
    Jet energy scale             & $<15$    & $<15$   & $<10$   & $<15$               \\
    b tagging eff \& mistag rate & $<5$     & $<5$    & $<2$    & $<2$                \\
    Lepton identification        & $2-5$    & $2-5$   & $2-5$   & $-$                 \\
    Pileup                       & $<10$    & $<6$    & $<4$    & $<3$                \\
    Top quark \Pt                & $<5$     & $<20$   & $<4$    & $-$                 \\ [0.5ex]
    \multicolumn{5}{l}{\it Closure tests:}                                            \\
    W/Z ratio                    & $-$      & $10-30$ & $-$     & $-$                 \\
    Z/$\gamma$ ratio             & $-$      & $-$     & $-$     & $10-30$             \\
    W/\ttbar composition         & $10-100$ & $-$     & $-$     & $-$                 \\
    W polarisation               & $5-50$   & $5-50$  & $-$     & $-$                 \\
    $\alphat\,/\,\bdphi$\B       & $5-80$   & $5-80$  & $50-80$ & $-$                 \\
    \hline
  \end{tabular}
\end{table*}

The aforementioned systematic uncertainties, resulting from variations
in scale factors, are summarised in Table~\ref{tab:bkgd_systs}, along
with representative magnitudes.  Each source of uncertainty is assumed
to vary with a fully correlated behaviour across the full phase space
of the signal and control regions.

The second category of uncertainty is determined from sets of closure
tests based on data control samples~\cite{RA1Paper2012}. Each set uses
the observed event counts in up to eight bins in \scalht for each of
the nine \njet event categories in one of the three independent data
control regions. These counts are used with the corresponding \tf
factors, determined from simulation, to obtain a prediction
$\mathcal{N}^\text{pred}(\njet, \scalht)$ of the observed yields
$\mathcal{N}^\text{obs}(\njet, \scalht)$ in another control sample
(or, in one case, \nb event category).

Each set of tests targets a specific (potential) source of bias in the
simulation modelling that may introduce an \njet- or \scalht-dependent
source of systematic bias in the \tf factors~\cite{RA1Paper2012}. Several
sets of tests are performed. The $\PZ/\gamma$ ratio determined from
simulation is tested against the same ratio measured using \zmumuj
events and the \gj sample.
%The $\PW/\PZ$ ratio is also probed using the \mj and \mmj samples. 
The $\PW/\PZ$ ratio is also probed using the \mj and \mmj
samples, which directly tests the simulation modelling of vector
boson production, as well as the modelling of \ttbar contamination in
the \mj sample. 
A further set probes the modelling of the relative composition between
\wlj and \ttbar events using \mj events containing exactly zero or one
more b-tagged jets, which represents a larger extrapolation in
relative composition than used in the search.  The effects of W
polarisation are probed by using \mj events with a positively charged
muon to predict those containing a negatively charged muon. Finally,
the accuracy of the modelling of the efficiencies of the \alphat and
\bdphi requirements are estimated using the \mj sample.

For each set of tests, the level of closure, %$\mathcal{C} =
($\mathcal{N}^\text{obs} - \mathcal{N}^\text{pred}) /
\mathcal{N}^\text{obs}$, which considers only statistical
uncertainties, is inspected to ensure no statistically significant
biases are observed as a function of the nine \njet categories or the
eight \scalht bins. In the absence of such a bias, the level of
closure is recomputed by integrating over either all monojet and
asymmetric \njet categories, or the symmetric \njet categories. The
level of closure and its statistical uncertainty are combined in
quadrature to determine additional contributions to the uncertainties
in the \tf factors. These uncertainties are considered to be
fully correlated between the monojet and asymmetric \njet categories
or the symmetric \njet categories, and fully uncorrelated between
these two regions in \njet and \scalht bins. If the closure tests use
the \mmj sample, the level of closure is determined by additionally
integrating over pairs of adjacent \scalht bins. These uncertainties,
derived from the closure tests in data, are summarised in
Table~\ref{tab:bkgd_systs}, along with representative
magnitudes. These uncertainties are the dominant contribution to the
total uncertainty in the \tf factors, due to the limited number
of events in the data control regions.

%%%%%

% and to
%assign appropriate systematic uncertainties that can be in excess of
%$>100\%$ in the most sensitive \HTmiss bins. Independent systematic
%uncertainties in the templates for the \znunu + jets background and
%the W + jets and \ttbar backgrounds are treated as fully uncorrelated
%across \njet and \nb categories and \scalht bins, and with respect to
%the ``normalisation'' systematic uncertainties summarised in
%Table~\ref{tab:bkgd_systs}.

%%%%%

As introduced in Section~\ref{sec:mht_templates}, templates are
derived from simulation to predict the \HTmiss distributions of the
background. The uncertainties in the \tf factors are used to constrain the
normalisation of the \HTmiss templates. The uncertainties in the
\HTmiss shape are discussed below.

%The highly granular binning in \njet and \scalht ensures that events
%categorised by \njet and within the same \scalht bin are produced at a
%similar scale. The effects of missing higher-order calculations in the
%simulated samples are largely mitigated through the transfer factor
%approach and the reliance of measurements in data from multiple
%control regions. Hence, the scale, which is manifest in the
%normalisation of the expected counts integrated over \HTmiss, is
%controlled accurately via the use of transfer factors and measurements
%in data control regions. As a consequence, the simulation is expected
%to accurately model the \HTmiss distribution in data. This assumption is
%based on the expectation that the simulation is able to accurately
%describe the kinematics of events at a similar scale. Conversely, an
%accurate kinematic description is not expected if events of
%different scales are allowed to mix.

The accuracy to which the simulation describes the \HTmiss
distributions is evaluated with respect to data in each (\njet, \nb,
\scalht) bin in each of the \mj, \mmj, and \gj data control regions. %
The level of agreement between data and simulation, defined in terms
of the ratio of observed and expected counts (from simulation) as a
function of \HTmiss, is parameterised using an orthogonal first-order
polynomial, $f(x) = p_0 + p_1(\bar{x}-x)$, and described by two
uncorrelated parameters, $p_0$ and $p_1$. A binned likelihood fit is
performed in each (\njet, \nb, \scalht) bin of each control region,
and the best fit value $p_1$ and its uncertainty is used to determine
the presence of biases dependent on \HTmiss. The pull of $p_1$ from a
value of zero is defined as the best fit value over its standard
deviation, considering only statistical uncertainties associated with
the finite size of the data and simulated samples.

The lower bound of the final (open) bin in \HTmiss is not more than
800\GeV and is bounded from above by the upper bound of the \scalht
bin in question. The lower bound of the final \HTmiss bin is merged
with lower bins if fewer than ten events in the data control regions
are observed. If a bin in (\njet, \nb, \scalht) contains fewer than
ten events, the \HTmiss template is not used and the background
estimates are determined inclusively with respect to \HTmiss. The
merging of bins is typically only relevant for event categories that
satisfy $\nb \geq 2$.

The presence of systematic biases is evaluated at a statistical level
by considering the distribution of pulls obtained from each control
region, which are consistent with statistical fluctuations, with no
indication of trends across the full phase space of each control
region. The $p$-values obtained from the fits are uniformly
distributed.

The uncertainty in the \HTmiss modelling is extracted under the
hypothesis of no bias. This is done using the maximum likelihood (ML)
values of the fit parameters to determine the statistical precision to
which this hypothesis can be confirmed. The quadrature sum of the ML
value and its uncertainty for $p_1$ from each fit is used to define
alternative templates that represent $\pm1\sigma$ variations to the
nominal \HTmiss template. These alternative templates are encoded in
the likelihood function, as described in Section~\ref{sec:result}. The
observed variations are compatible with the expected values obtained
from studies relying only on simulated event samples. The
uncertainties in the final \HTmiss bin of the templates depend on the
event category and \scalht bin, and are typically found to be in the
range $\sim$10--100\%.

The effect on the \HTmiss templates is determined under $\pm1\sigma$
variations in the jet energy scale, the efficiency and
misidentification probability of b-quark jets, the efficiency to
identify well-reconstructed, isolated leptons, the pileup reweighting,
and the modelling of the top quark \Pt. These effects
are easily covered by the uncertainties determined from data, as
described above, across the full phase space of the control regions,
which mirror closely that of the signal region.




% A check has also been performed on the systematic effect on the
% background prediction due to QCD contamination in the control
% samples, which has been found to be at the 5\% level for the gamma +
% jets control region. Applying an arbitrarily large variation of
% +/-100\% on the number of Monte Carlo QCD events leads to a
% systematic variation on the transfer factors of at most 5%
% in the majority of bins. This preliminary study suggests that effect
% from QCD contamination in the gamma + jets control region is small
% compared to the total uncertainty assigned to transfer factors. This
% systematic source is covered anyway in the data-driven study using
% the photon control region, described in Sec. 12.2.

%The choice of PDF set, or variations therein, predominantly affects
%$\mathcal{A}\times\varepsilon$ through changes in the \Pt spectrum of
%the system recoil. Uncertainties in $\mathcal{A}\times\varepsilon$ due
%to variations in the renormalisation and factorisation scales are
%determined to be relatively small. In both cases, contributions to the
%uncertainty in the theoretical production cross section are considered.

%Uncertainties in the acceptance associated with the PDFs,
%including those related to the renormalisation and factorisation
%scales, are evaluated by varying the PDF sets used to produce the
%simulated samples. These uncertainties are defined by the maximum
%deviations observed from 100 variations of the NNPDF3.0LO PDFs for tt
%and W+jets events. 
