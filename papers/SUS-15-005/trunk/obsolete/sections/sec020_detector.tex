\section{The CMS detector}
\label{sec:detector}

% Text copy/paste from https://twiki.cern.ch/twiki/bin/viewauth/CMS/Internal/PubDetector

The central feature of the CMS apparatus is a superconducting solenoid
of 6\unit{m} internal diameter, providing an axial magnetic field of
3.8\unit{T}. The bore of the solenoid is instrumented with several
particle detection systems. A silicon pixel and strip tracker measures
charged particles within the pseudorapidity range $\abs{\eta} < 2.5$.
%, where $\eta \equiv -\ln[\tan(\theta/2)]$ and $\theta$ is the polar
%angle of the trajectory of the particle with respect to the
%counterclockwise beam direction. 
A lead tungstate crystal electromagnetic calorimeter (ECAL), and a
brass and scintillator hadron calorimeter (HCAL), each composed of a
barrel and two endcap sections, extend over a range $\abs{\eta} <
3.0$. %
Outside the bore of the solenoid, forward calorimeters extend the
coverage to $\abs{\eta} < 5.0$, and muons are measured within
$\abs{\eta} < 2.4$ by gas-ionisation detectors embedded in the steel
flux-return yoke outside the solenoid. A two-tier trigger system
selects pp collision events of interest. The first level of the
trigger system, composed of custom hardware processors, uses
information from the calorimeters and muon detectors to select the
most interesting events in a fixed time interval of less than
4\mus. The high-level trigger processor farm further decreases the
event rate from around 100\unit{kHz} to less than 1\unit{kHz}, before
data storage. The CMS detector is nearly hermetic, which allows for
momentum balance measurements in the plane transverse to the beam
axis. A more detailed description of the CMS detector, together with a
definition of the coordinate system used and the relevant kinematic
variables, can be found in Ref.~\cite{Chatrchyan:2008zzk}.
