\section{Event reconstruction}
\label{sec:event_reconstruction}

Global event reconstruction is provided by the particle flow (PF)
algorithm~\cite{CMS-PAS-PFT-09-001,CMS-PAS-PFT-10-001}, designed to
identify each particle using an optimised combination of information
from all detector systems. In this process, the identification of the
particle type (photon, electron, muon, charged hadron, neutral hadron)
plays an important role in the determination of the particle direction
and energy.

Among the vertices reconstructed within 24 (2)\unit{cm} of the
detector centre parallel (perpendicular) to the beam axis, the primary
vertex (PV) is assigned to be the one with the largest sum of charged
particle (track) $\pt^2$ values. 
%The primary vertex (PV) for each recorded event is assumed to be the
%reconstructed vertex with the largest sum of charged-particle track
%$\pt^2$ values that is found within 24\unit{cm} and 2\unit{cm} of the
%centre of the detector in the directions along and perpendicular to
%the beam axis, respectively. 
Charged-particle tracks associated with reconstructed vertices from
pileup events are not considered by the PF algorithm as part of the
global event reconstruction.

Photon candidates~\cite{CMS:EGM-14-001} are identified as ECAL energy
clusters not linked to the extrapolation of any track to the ECAL. The
energy of photons is directly obtained from the ECAL measurement,
corrected for contributions from pileup events.  Various
quality-related criteria must be satisfied in order to identify
photons with high efficiency while minimising the misidentification of
electrons and associated bremstrahlung, jets, or ECAL noise as
photons. The criteria include the following: the shower shape of the
energy deposition in the ECAL must be consistent with that expected
from a photon, the energy detected in the HCAL behind the photon
shower must not exceed 5\% of the photon energy, and no matched hits
in the pixel tracker must be found.

Electron candidates~\cite{Khachatryan:2015hwa} are identified as a
track associated with an ECAL cluster compatible with
the track trajectory, as well as additional ECAL energy clusters from
potential bremsstrahlung photons emitted as the electron traverses
material of the silicon tracker. The energy of electrons is determined
from a combination of the track momentum at the main interaction
vertex, the corresponding ECAL cluster energy, and the energy sum of
all bremsstrahlung photons associated with the track. The quality
criteria required for electrons are similar to those for photons, with
regards to the ECAL shower shape and the relative contributions to the
total energy deposited in the ECAL and HCAL. Additional requirements
are also made on the associated track, which consider the track
quality, energy-momentum matching, and compatibility with the PV in
terms of the transverse and longitudinal impact parameters.

Muon candidates~\cite{Chatrchyan:2012xi} are identified as a track in the
silicon tracker consistent with either a track or several hits in the
muon system. The track and hit parameters must satisfy various
quality-related criteria, described in Ref.~\cite{Chatrchyan:2012xi}.
% and associated with an energy deficit in the calorimeters.
The energy of muons is obtained from the corresponding track momentum.

Charged hadrons are identified as tracks not
classified as either electrons or muons.
%charged particle tracks neither identified as electrons, nor as muons.
The energy of charged hadrons is determined from a combination of the
track momentum and the corresponding ECAL and HCAL energies, corrected
for contributions from pileup events and the response function of the
calorimeters to hadronic showers. Neutral hadrons are identified as
HCAL energy clusters not linked to any charged-hadron trajectory, or
as ECAL and HCAL energy excesses with respect to the expected
charged-hadron energy deposit.  The energy of neutral hadrons is
obtained from the corresponding corrected ECAL and HCAL energy.

Photons are required to be isolated from other activity in the event,
such as charged and neutral hadrons, within a cone $\Delta R =
\sqrt{(\Delta\phi)^2 + (\Delta\eta)^2} = 0.3$ around the photon
trajectory, corrected for contributions from pileup events and the
photon itself. Electrons and muons are also required to be isolated
from other reconstructed particles in the event, primarily to suppress
background contributions from semileptonic heavy-flavour decays in
multijet events. The isolation $I^\text{mini}_\text{rel}$ is defined
as the scalar \Pt sum of all charged and neutral hadrons, and photons,
within a cone around the lepton direction, divided by the lepton
\Pt. The ``mini'' cone radius is dependent on the lepton \Pt,
primarily to identify with high efficiency the collimated daughter
particles of semileptonically decaying Lorentz-boosted top quarks,
according to the following: $R = 0.2$ and 0.05 for, respectively, $\Pt
< 50\GeV$ and $\Pt > 200\GeV$, and $R = 10\GeV / \Pt$ for $50 < \Pt <
200\GeV$. The variable $I^\text{mini}_\text{rel}$ excludes
contributions from the lepton itself and pileup events. The isolation
for electrons and muons is required to satisfy, respectively,
$I^\text{mini}_\text{rel} < 0.1$ and 0.2 for the signal region and
nonleptonic control sample selection criteria.  A tighter definition
of muon isolation $I^{\mu}_\text{rel}$ is used for the definition of
control regions that are required to contain at least one muon. The
variable $I^{\mu}_\text{rel}$ is determined identically to
$I^\text{mini}_\text{rel}$ except that a cone of fixed radius $R =
0.4$ is assumed.

Electron and muon candidates identified by the PF algorithm that do
not satisfy the quality criteria or the $I^\text{mini}_\text{rel}$
isolation requirements described above, as well as charged hadrons,
are collectively labelled as ``single isolated tracks'' if they are
isolated from neighbouring tracks associated to the
PV. The isolation $I^\text{track}_\text{rel}$ is defined as the scalar
\Pt sum of tracks (excluding the track under
consideration) within a cone $\Delta R < 0.3$ around the track
direction, divided by the track \Pt. The requirement
$I^\text{track}_\text{rel} < 0.1$ is imposed.

Jets are clustered from the PF candidate particles with the infrared-
and collinear-safe anti-$k_t$ algorithm~\cite{antikt}, operated with a
distance parameter of 0.4. The jet momentum is determined as the
vectorial sum of all particle momenta in the jet, and is found in the
simulation to be within 5 to 10\% of its true momentum over the whole
\pt spectrum and detector acceptance. Jet energy corrections, to
account for pileup~\cite{pileup} and to establish a uniform relative
response in $\eta$ and a calibrated absolute response in \Pt, are
derived from the simulation, and are confirmed with in situ
measurements using the energy balance in dijet and photon+jet
events~\cite{Chatrchyan:2011ds}. The jet energy resolution is
typically 15\% at 10\GeV, 8\% at 100\GeV, and 4\% at 1\TeV, compared
to about 40\%, 12\%, and 5\% obtained when the calorimeters alone are
used for jet clustering.
% Jet quality criteria as described in Ref. [32] are applied to
% eliminate, for example, spurious events caused by calorimeter noise.
All jets are required to satisfy loose requirements on the relative
composition of their particle constituents to reject noise in the
calorimeter systems or failures in event reconstruction.

Jets are identified as originating from b quarks using the combined
secondary vertex algorithm~\cite{CMS-PAS-BTV-12-001}. Control regions
in data~\cite{bjets} are used to measure the probability of correctly
identifying jets as originating from b quarks (b tagging efficiency),
and the probability of misidentifying jets originating from
light-flavour partons (u, d, s quarks or gluons) or a charm quark as a
b-tagged jet (the light-flavour and charm mistag probabilities). A
working point is employed that yields a b tagging efficiency of 65\%,
and charm and light-flavour mistag probabilities of approximately 12
and 1\%, respectively, for jets with \Pt that is typical of \ttbar
events.

An estimator of \ptvecmiss is given by the projection on the plane
perpendicular to the beams of the negative vector sum of the momenta
of all candidate particles in an event~\cite{cms-met}, as determined
by the PF algorithm. Its magnitude is referred to as \ETmiss.

%The missing transverse momentum vector \ptvecmiss is defined as the
%projection on the plane perpendicular to the beams of the negative
%vector sum of the momenta of all PF candidate particles in an
%event~\cite{cms-met}. Its magnitude is referred to as \ETmiss.
