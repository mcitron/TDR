\section{Simulated event samples}
\label{sec:simulation}

\fixme{NEEDS CHECKING. NEEDS POLISHING.}

The search relies on multiple control regions in data and simulated
event samples to estimate contributions from SM backgrounds. The
definitions of the control regions in data and the background
estimation procedures are described in
Sec.~\ref{needtoaddthisref}. The dominant SM backgrounds for the
search are QCD multijet production, top-antitop (\ttbar), single top,
and vector boson (W, Z, and $\gamma$) production. Residual
contributions from other processes, such as WW, WZ, ZZ (diboson)
production and the associated production of \ttbar and a vector boson
(W and Z), are also considered.

The \MADGRAPH5 aMC@NLO 2.2.2~\cite{Alwall2014} event generator code is
used at leading order (LO) accuracy to produce samples of \wj, \zj,
\gj, \ttbar, and multijet events. The same code is used at
next-to-leading order (NLO) accuracy to generate samples of single top
(both s- and t-channel production), WZ, ZZ, \ttw, and \ttz events. The
NLO \POWHEG v2~\cite{powheg, powheg_top_Wt} generator is used to
describe WW events and the Wt-channel production of single top
events. The simulated samples are normalised according to production
cross sections that are calculated with NLO and next-to-NLO
precision~\cite{nloxs, wphys, fewz, wwxs, top++, nlotop,
  powheg_top_Wt}. The description of the detector response is
implemented using the \GEANTfour~\cite{geant} package.

Event samples for signal models involving gluino or squark pair
production, in association with up to two additional partons, are
generated at leading order with \MADGRAPH5 aMC@NLO, and the sparticles
are decayed with \PYTHIA8~\cite{pythia}. The signal production cross
sections are calculated with NLO plus next-to-leading-logarithm (NLL)
accuracy~\cite{Beenakker:1996ch, PhysRevLett.102.111802,
  PhysRevD.80.095004, 1126-6708-2009-12-041,
  doi:10.1142/S0217751X11053560, susynlo}. The detector response is
provided by the CMS fast simulation package~\cite{fastsim}. 

%which yields consistent results compared with the GEANT4-based
%simulation, except that we apply a correction of 1\% to account for
%differences in the efficiency of the jet quality requirements [32],
%and corrections of 3-10\% to account for differences in the b jet
%tagging efficiency. 

The \textsc{NNPDF}3.0 LO and \textsc{NNPDF}3.0 NLO~\cite{nnpdf} parton
distribution functions (PDF) are used, respectively, with the LO and
NLO generators described above. The LO \PYTHIA 8.2~\cite{pythia}
program is used to describe parton showering and hadronisation for all
simulated samples. To model the effects of multiple pp collisions
within the same or neighboring bunch crossings (pileup), all simulated
events are generated with a nominal distribution of pp interactions
per bunch crossing and then reweighted to match the pileup
distribution as measured in data.

%Corrections to the normalisation of the \gj, W+jets, Z+jets and
%\ttbar+jets processes are derived using sidebands in the \mht variable
%in data control samples.
