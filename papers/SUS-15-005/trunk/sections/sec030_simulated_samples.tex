\section{Simulated event samples}
\label{sec:simulation}

The search relies on multiple event samples, in data or generated from
Monte Carlo (MC) simulations, to estimate the contributions from SM
backgrounds, as described in Section~\ref{sec:event_selection}. 
%The search relies on multiple control regions in data and simulated
%event samples to estimate contributions from SM backgrounds. 
%The definitions of the control regions in data and the background
%estimation procedures are described in
%Section~\ref{needtoaddthisref}. 
The dominant SM backgrounds for the search are QCD multijet
production, and the associated production of jets and top
quark-antiquark (\ttbar), single top, and vector boson (W,
\znunu). Residual contributions from other processes, such as WW, WZ,
ZZ (diboson) production and the associated production of \ttbar and a
vector boson (W and Z), are also considered. Other processes, such as
Drell-Yan ($\cPq\bar{\cPq}\!  \rightarrow\! \PZ/\gamma^*\!
\rightarrow\!  \ell^+\ell^-$) and \gj production, are also relevant
for some control regions, defined in
Section~\ref{sec:control_regions}.

The \MADGRAPH5 aMC@NLO 2.2.2~\cite{Alwall2014} event generator code is
used at leading order (LO) accuracy to produce samples of \wj, \zj,
\gj, \ttbar, and multijet events. The same code is used at
next-to-leading order (NLO) accuracy to generate samples of single top
(both s- and t-channel production), WZ, ZZ, \ttw, and \ttz events. The
NLO \POWHEG v2~\cite{powheg, powheg_top_Wt} generator is used to
describe WW events and the Wt-channel production of single top
events. The simulated samples are normalised according to production
cross sections that are calculated with NLO and next-to-NLO
precision~\cite{Alwall2014, wphys, fewz, wwxs, top++, nlotop,
  powheg_top_Wt}. The \GEANTfour~\cite{geant} package is used to
simulate the detector response. 

Event samples for signal models involving gluino or squark pair
production, in association with up to two additional partons, are
generated at leading order with \MADGRAPH5 aMC@NLO, and the decay of
the sparticles is performed with \PYTHIA
8.205~\cite{pythia}. Inclusive, process-dependent, signal production
cross sections are calculated with NLO plus next-to-leading-logarithm
(NLL) accuracy~\cite{Beenakker:1996ch, PhysRevLett.102.111802,
  PhysRevD.80.095004, 1126-6708-2009-12-041,
  doi:10.1142/S0217751X11053560, susynlo}. The theoretical systematic
uncertainties are typically dominated by the parton density function
(PDF) uncertainties, evaluated using the
CTEQ6.6~\cite{Nadolsky:2008zw} and MSTW2008~\cite{Martin:2009iq} PDFs.
The detector response for signal models is provided by the CMS fast
simulation package~\cite{fastsim}.

%which yields consistent results compared with the GEANT4-based
%simulation, except that we apply a correction of 1\% to account for
%differences in the efficiency of the jet quality requirements [32],
%and corrections of 3-10\% to account for differences in the b jet
%tagging efficiency. 

The \textsc{NNPDF}3.0 LO and \textsc{NNPDF}3.0 NLO~\cite{nnpdf} parton
distribution functions (PDF) are used, respectively, with the LO and
NLO generators described above. The LO \PYTHIA program with the {\sc
  CUETP8M1} underlying event tune~\cite{Khachatryan:2015pea} is used
to describe parton showering and hadronisation for all simulated
samples. To model the effects of multiple pp collisions within the
same or neighboring bunch crossings (pileup), all simulated events are
generated with a nominal distribution of pp interactions per bunch
crossing and then reweighted to match the pileup distribution as
measured in data. On average, approximately fifteen different
collisions, identifiable via their primary interaction vertex, are
reconstructed per event. Finally, (near-unity) corrections to the
normalisation of the simulated samples for the \gj, \wmj, \ttbar, and
\zmumuj, and, equivalently, \znunuj processes are derived using a data
sideband to the control regions.
