\section{Interpretations} 
\label{sec:interpretations}

\subsection{Specification for simplified models} 

The results of the search are used to constrain simplified
supersymmetric models~\cite{Alwall:2008ag, Alwall:2008va, sms}. Each
model assumes the pair production of gluinos or squarks and their
subsequent prompt decays to SM particles and the LSP ($\chiz_1$) with
a 100\% branching ratio (unless indicated otherwise). The gluino
decays contain intermediate on-shell sparticle states (such as the top
squark, $\PSQt$, or the chargino, $\chipm_1$) for a subset of the
models. All other sparticles are assumed to be too heavy ($m_{\sGlu} /
m_{\PSQ} = 10\TeV$) to be produced directly. Three-body decays of
gluinos are assumed to occur via off-shell squarks of light or heavy
flavour. All SM particles with a finite lifetime, such as the W boson,
are assumed to decay naturally.

\begin{table*}[tb]
  \topcaption{A summary of the simplified supersymmetric models used
    to interpret the results of this search. All on-shell sparticles
    in the decay are stated.} 
  \label{tab:simplified-models}
  \centering
  \footnotesize
  \begin{tabular}{ llll }
    \hline
Topology               & Fig.
%                       & Production
                       & Decay
                       & Additional assumptions                                                         \\ [0.5ex]
\hline
\multicolumn{4}{l}{\it Gluino-mediated and direct production of light-flavour squarks}       \\ [0.5ex]
\texttt{T1qqqq}        %& \ref{fig:T1qqqq_feyn}                                                  
                       & $\text{pp}\ra\sGlu\sGlu$
                       & $\sGlu\ra\cPaq\cPq\chiz_1$
                       & --                                                                             \\ [0.5ex]
\texttt{T2qq\_8fold}   %& \ref{fig:T2qq_feyn}                     
                       & $\text{pp}\ra\PSQ\PASQ$        
                       & $\PSQ\ra\cPq\chiz_1$
                       & $m_{\PSQ} = m_{\PSQ_\cmsSymbolFace{L}} = m_{\PSQ_\cmsSymbolFace{R}}$,
                       $\PSQ = \{ \PSQu, \PSQd, \PSQs, \PSQc \}$                                        \\ [0.5ex]
\texttt{T2qq\_1fold}   %& \ref{fig:T2qq_feyn}                                                  
                       & $\text{pp}\ra\PSQ\PASQ$         
                       & $\PSQ\ra\cPq\chiz_1$
                       & $m_{\PSQ (\PSQ \neq \PSQu_\cmsSymbolFace{L})} \gg m_{\PSQu_\cmsSymbolFace{L}}$ \\ [0.5ex]
\multicolumn{4}{l}{\it Gluino-mediated production of off-shell third-generation squarks}               \\ [0.5ex]
\texttt{T1bbbb}        %& \ref{fig:T1bbbb_feyn}                                                   
                       & $\text{pp}\ra\sGlu\sGlu$       
                       & $\sGlu\ra\cPaqb\cPqb\chiz_1$
                       & --                                                                             \\ [0.5ex]
\texttt{T1tttt}        %& \ref{fig:T1tttt_feyn}
                       & $\text{pp}\ra\sGlu\sGlu$       
                       & $\sGlu\ra\cPaqt\PSQt^*\ra\cPaqt\cPqt\chiz_1$
                       & --                                                                             \\ [0.5ex]
\texttt{T1ttbb}        %& \ref{fig:T1ttbb_feyn} 
                       & $\text{pp}\ra\sGlu\sGlu$      
                       & $\sGlu\ra\cPaqt\cPqb\chipm_1\ra\cPaqt\cPqb\PW^*\chiz_1$
                       & $m_{\chipm_1} - m_{\chiz_1} = 5\GeV$                                           \\ [0.5ex]
\multicolumn{4}{l}{\it Natural gluino-mediated production of on-shell top squarks}                     \\ [0.5ex]
\texttt{T5tttt}        %& \ref{fig:T5tttt_feyn}
                       & $\text{pp}\ra\sGlu\sGlu$      
                       & $\sGlu\ra\cPaqt\PSQt\ra\cPaqt\cPqt\chiz_1$ 
                       & $m_{\,\PSQt} - m_{\chiz_1} = 175\GeV$                                          \\ [0.5ex]
\texttt{T5ttcc}        %& \ref{fig:T5ttcc_feyn}            
                       & $\text{pp}\ra\sGlu\sGlu$       
                       & $\sGlu\ra\cPaqt\PSQt\ra\cPaqt\cPqc\chiz_1$ 
                       & $m_{\,\PSQt} - m_{\chiz_1} = 20\GeV$                                           \\ [0.5ex]
\texttt{T5tttt\_degen} %& \ref{fig:T5tttt_degen_feyn}
                       & $\text{pp}\ra\sGlu\sGlu$      
                       & $\sGlu\ra\cPaqt\PSQt\ra\cPaqt\cPqb\PW^*\chiz_1$
                       & $m_{\,\PSQt} - m_{\chiz_1} = 20\GeV$                                           \\ [0.5ex]
\multicolumn{4}{l}{\it Direct production of on-shell third-generation squarks}                         \\ [0.5ex]
\texttt{T2bb}          %& \ref{fig:T2bb_feyn}
                       & $\text{pp}\ra\PSQb\PASQb$     
                       & $\PSQb\ra\cPqb\chiz_1$
                       & --                                                                             \\ [0.5ex]
\texttt{T2tb}          %& \ref{fig:T2tb_feyn}
                       & $\text{pp}\ra\PSQt\PASQt$     
                       & $\PSQt\ra\cPqt\chiz_1 \;\text{or}\; \cPqb\chipm_1\ra\cPqb\PW^*\chiz_1$
                       & $\mathcal{BR} = 50/50\%$, $m_{\chipm_1} - m_{\chiz_1} = 5\GeV$                 \\ [0.5ex]
\texttt{T2tt}          %& \ref{fig:T2tt_feyn}
                       & $\text{pp}\ra\PSQt\PASQt$
                       & $\PSQt\ra\cPqt\chiz_1$
                       & --                                                                             \\ [0.5ex]
\texttt{T2cc}          %& \ref{fig:T2cc_feyn}
                       & $\text{pp}\ra\PSQt\PASQt$      
                       & $\PSQt\ra\cPqc\chiz_1$
                       & $10 < m_{\,\PSQt} - m_{\chiz_1} < 80\GeV$                                      \\ [0.5ex]
\texttt{T2tt\_degen}   %& \ref{fig:T2tt_degen_feyn}
                       & $\text{pp}\ra\PSQt\PASQt$      
                       & $\PSQt\ra\cPqb\PW^*\chiz_1$
                       & $10 < m_{\,\PSQt} - m_{\chiz_1} < 80\GeV$                                      \\ [0.5ex]
\texttt{T2tt\_mixed}   %& \ref{fig:T2tt_mixed_feyn}
                       & $\text{pp}\ra\PSQt\PASQt$      
                       & $\PSQt\ra\cPqc\chiz_1 \;\text{or}\; \cPqb\PW^*\chiz_1$
                       & $\mathcal{BR} = 50/50\%$, $10 < m_{\,\PSQt} - m_{\chiz_1} < 80\GeV$            \\ [0.5ex]
    \hline
  \end{tabular}
\end{table*}

Fifteen unique production and decay modes are considered, which yield
a range of topologies and final states (with only the all-jet final
state considered in this search). Scans in the gluino or squark
($m_{\sGlu} / m_{\PSQ}$) and LSP ($m_{\chiz_1}$) mass parameter space
are performed for each model. Each class of simplified model is
identified by a label that indicates the topology and final state. 
%and the production and decay mode is indicated by the diagrams shown
%in Fig.~\ref{fig:simplified-models}. 
Table~\ref{tab:simplified-models} summarises the production and decay
modes, as well as any additional assumptions that define each
simplified model. The models can be categorised according to the
following descriptions: the gluino-mediated and direct production of
light-flavour squarks; the gluino-mediated production of off-shell
third-generation squarks; the ``natural'' gluino-mediated production
of on-shell top squarks; and the direct production of on-shell
third-generation squarks. In the case of direct pair production of
light-flavour squarks, two different assumptions on the theory
production cross section are made. For the ``eightfold'' scenario
(\texttt{T2qq\_8fold}), the scalar partners to left- and right-handed
quarks of the u, d, s, and c flavours are assumed to be light and
degenerate in mass, with other squark states decoupled to a high
mass. For the ``onefold'' scenario (\texttt{T2qq\_1fold}), only a
single light squark is assumed to particate in the interaction and all
other squarks are decoupled to a high mass.

Under the background + signal hypothesis, and in the presence of a
non-zero signal contribution, a modified frequentist approach is used
to determine upper limits at 95\% confidence level (CL) on the cross
section, $\sigma_\text{UL}$ (pb), to produce pairs of supersymmetric
particles as a function of the parent sparticle and the LSP
masses. The potential contributions from a new-physics signal to each
of the signal and control regions are considered, even though the only
significant contribution occurs in the signal region and not the
control region (\ie signal contamination). The approach is based on
the one-sided (LHC-style) profile likelihood ratio as the test
statistic, the \cls criterion~\cite{junk, read}, and asymptotic
formulae~\cite{Cowan:2010js} are utilised to approximate the
distributions of the test statistics under the SM background-only and
signal + background hypotheses.
%The test statistic is $q_\mu = -2 \ln (\mathcal{L}_\mu /
%\mathcal{L}_\text{max})$, where $\mathcal{L}_\text{max}$ is the
%maximum likelihood determined by allowing all parameters including a
%multiplier $\mu$ on the production cross section (``signal strength'')
%to vary, and $\mathcal{L}_\mu$ is the maximum likelihood for a fixed
%signal strength. To set limits, we use asymptotic results for the test
%statistic [71] and the CLs method described in Refs. [72, 73]. More
%details are provided in Refs. [15, 74].

\begin{table*}[tb]
  \topcaption{A summary of benchmark simplified models, the most sensitive
    \njet categories, and representative values for the corresponding
    experimental acceptance times efficiency
    ($\mathcal{A}\times\varepsilon$), the dominant sysmtematic
    uncertainties, the theory production cross section
    ($\sigma_\text{theory}$), and the expected and observed upper limits
    on the production cross section, expressed in terms of the signal
    strength parameter ($\mu$).
  }
  \label{tab:signal-eff}
  \centering
  \footnotesize
  \begin{tabular}{ lllcrrrrrcc }
    \hline
    \multicolumn{2}{l}{Benchmark models} 
  & Most sensitive
  & $\mathcal{A}\times\varepsilon$
  & \multicolumn{4}{c}{Systematic uncertainties [\%]}
  & $\sigma_\text{theory}$
  & \multicolumn{2}{c}{$\mu$ (95\% CL)}                                                                         \\ [0.3ex]
    \cline{5-8}
    \multicolumn{2}{l}{$(m_{\text{SUSY}}, m_{\mathrm{LSP}})$ [GeV]} 
  & \njet categories
  & [\%]    
  & MC stat.
  & ISR 
  & JEC
  & $\text{SF}_\text{b-tag}$
  & \multicolumn{1}{c}{[fb]}
  & Exp.
  & Obs.                                                                                                        \\ [0.3ex] 
    \hline
    \multirow{2}{*}{\texttt{T1qqqq}} 
  & (1300, 100) & $\geq$5, 4, 3, 2         & \phantom{1}9.4 & 7-30  & 2-2   & 4-21  & 2-14 & 46.1 & 0.79 & 0.76 \\
  & (900, 700)  & $\geq$5, $\geq$5a, 4, 4a & \phantom{1}5.6 & 10-33 & 1-13  & 1-26  & 1-10 & 677  & 0.58 & 0.44 \\ [0.5ex]
    \multirow{2}{*}{\texttt{T2qq\_8fold}}
  & (1050, 100) & $\geq$5, 3, 2, 4         & 13.4           & 7-33  & 2-5   & 3-16  & 1-11 & 44.0 & 0.72 & 0.50 \\
  & (650, 550)  & $\geq$5, 4, $\geq$5a, 4a & \phantom{1}2.6 & 10-28 & 3-9   & 2-28  & 1-6  & 1080 & 0.74 & 0.64 \\ [0.5ex]
     \multirow{2}{*}{\texttt{T2qq\_1fold}}
  & (1050, 100) & $\geq$5, 3, 2, 4         & 17.9           & 7-33  & 2-5   & 3-16  & 1-11 & 44.0 & 0.72 & 0.50 \\
  & (650, 550)  & $\geq$5, 4, $\geq$5a, 4a & \phantom{1}2.6 & 10-28 & 3-9   & 2-28  & 1-6  & 1080 & 0.74 & 0.64 \\ [0.5ex]
    \multirow{2}{*}{\texttt{T1bbbb}}
  & (1500, 100) & $\geq$5, 4, 3, 2         & 10.1           & 5-17  & 1-2   & 1-12  & 2-22 & 14.2 & 0.81 & 0.79 \\
  & (1000, 800) & $\geq$5, 4, $\geq$5a, 4a & \phantom{1}4.9 & 8-31  & 1-17  & 1-40  & 1-14 & 325  & 0.33 & 0.32 \\ [0.5ex]
    \multirow{2}{*}{\texttt{T1tttt}}
  & (1300, 100) & $\geq$5, $\geq$5a, 4, 3  & \phantom{1}2.4 & 7-16  & 1-2   & 2-7   & 2-12 & 46.1 & 1.00 & 1.89 \\
  & (800, 400)  & $\geq$5, $\geq$5a, 4, 4a & \phantom{1}0.6 & 7-27  & 1-2   & 3-45  & 1-8  & 1490 & 0.56 & 1.03 \\ [0.5ex]
    \multirow{2}{*}{\texttt{T1ttbb}}
  & (1300, 100) & $\geq$5, 4, 3, $\geq$5a  & \phantom{1}3.8 & 9-32  & 1-2   & 3-16  & 2-19 & 46.1 & 0.60 & 0.91 \\
  & (1000, 700) & $\geq$5, $\geq$5a, 4, 3  & \phantom{1}3.4 & 9-30  & 1-9   & 3-65  & 1-14 & 325  & 0.51 & 0.70 \\ [0.5ex]
    \multirow{2}{*}{\texttt{T5tttt}}
  & (800, 100)  & $\geq$5, $\geq$5a, 3, 4  & \phantom{1}0.2 & 12-20 & 2-4   & 3-5   & 1-6  & 1490 & 0.69 & 1.19 \\
  & (700, 400)  & $\geq$5, $\geq$5a, 4, 4a & \phantom{1}0.2 & 20-29 & 2-10  & 8-10  & 1-2  & 3530 & 1.00 & 1.35 \\ [0.5ex]
% & (700, 400)  & $\geq$5, $\geq$5a, 4, 4a & \phantom{1}0.2 & 20-20 & 2-10  & 10-10 & 2-2  &      & 1.00 & 1.35 \\ [0.5ex]
% T5ttttDM175 (700,400)
%  Efficiency      0.2   0.3
%  isrSignalWeight 1.0   10.0
%  jecWeight       8.0   62.0
%  puWeight        1.0   22.0
%  bsfWeight       1.0   23.0
%  triggerWeight   6.0   19.0
%  mcStat          29.0  50.0
    \multirow{2}{*}{\texttt{T5ttcc}}  
  & (1200, 200) & $\geq$5, 4, 3, $\geq$5a  & \phantom{1}4.9 & 6-25  & 5-25  & 3-21  & 1-24 & 85.6 & 0.58 & 0.87 \\
  & (750, 600)  & $\geq$5, $\geq$5a, 4, 4a & \phantom{1}1.0 & 9-23  & 1-4   & 5-21  & 1-3  & 2270 & 0.89 & 0.72 \\ [0.5ex]
    \multirow{2}{*}{\texttt{T5tttt\_degen}} 
  & (1100, 100) & $\geq$5, 4, 3, 4a        & \phantom{1}1.3 & 15-21 & 12-16 & 6-11  & 4-15 & 16.3 & 0.89 & 2.55 \\
  & (800, 600)  & $\geq$5, $\geq$5a, 4a, 4 & \phantom{1}1.9 & 5-32  & 1-8   & 1-34  & 1-7  & 1490 & 0.66 & 0.70 \\ [0.5ex]
    \multirow{2}{*}{\texttt{T2bb}}
  & (800, 50)   & 2, 3, 4, $\geq$5         & \phantom{1}1.5 & 5-31  & 2-6   & 1-21  & 1-23 & 28.3 & 0.96 & 1.06 \\
  & (375, 300)  & $\geq$5, 4, 3a, 3        & \phantom{1}0.1 & 8-33  & 1-10  & 3-25  & 1-7  & 2610 & 0.67 & 0.87 \\ [0.5ex]
    \multirow{2}{*}{\texttt{T2tb}}
  & (600, 50)   & $\geq$5, 4, 3, 2         & \phantom{1}6.1 & 3-28  & 1-3   & 1-22  & 1-17 & 175  & 0.70 & 1.35 \\
  & (350, 225)  & $\geq$5, 4, 3, 3a        & \phantom{1}1.0 & 9-33  & 1-4   & 2-41  & 1-8  & 3790 & 0.79 & 0.88 \\ [0.5ex]
    \multirow{2}{*}{\texttt{T2tt}}
  & (700, 50)   & $\geq$5, 4, 3, $\geq$5a  & \phantom{1}8.1 & 8-33  & 1-4   & 2-22  & 1-21 & 67.0 & 0.90 & 1.19 \\
  & (350, 100)  & $\geq$5, $\geq$5a, 4a, 4 & \phantom{1}1.4 & 7-31  & 1-1   & 1-28  & 1-7  & 3790 & 0.44 & 0.50 \\ [0.5ex]
    \multirow{1}{*}{\texttt{T2cc}}
  & (325, 305)  & $\geq$5, 4, 3, 2         & \phantom{1}0.8 & 3-32  & 1-27  & 1-27  & 1-12 & 5600 & 0.92 & 0.68 \\ [0.5ex]
    \multirow{1}{*}{\texttt{T2\_degen}}
  & (300, 290)  & 3, 4, $\geq$5, 2         & \phantom{1}0.9 & 2-27  & 1-27  & 1-25  & 1-12 & 8520 & 0.56 & 0.41 \\ [0.5ex]
    \multirow{1}{*}{\texttt{T2tt\_mixed}}
  & (300, 250)  & $\geq$5, 4, $\geq$5a, 4a & \phantom{1}0.4 & 3-33  & 1-27  & 1-33  & 1-13 & 8520 & 0.99 & 0.58 \\ [0.5ex]
    \hline
  \end{tabular}
\end{table*}

\subsection{Acceptances and uncertainties}

The experimental acceptance times efficiency
($\mathcal{A}\times\varepsilon$) and its uncertainty are evaluated
independently for each model as a function of ($m_\text{SUSY},
m_\text{LSP}$). Table~\ref{tab:signal-eff} summarises
$\mathcal{A}\times\varepsilon$ for a number of benchmark models, each
chosen to be near the limit of search sensitivity. For each topology,
typically two different pairs of parent sparticle and LSP masses
($m_\text{SUSY}, m_\text{LSP}$) are chosen that are characterised by a
large and a smaller (\ie ``compressed'') difference in parent
sparticle and LSP masses. The four most sensitive event categories,
defined in terms of \njet, are used to determine
$\sigma_\text{UL}$. The categories used per benchmark model are listed
in Table~\ref{tab:signal-eff}, along with
$\mathcal{A}\times\varepsilon$ determined for these four categories.

Contributions from several sources to the uncertainty in
$\mathcal{A}\times\varepsilon$ are considered. Each source of
uncertainty is included in the likelihood function via alternative
shapes to the nominal \HTmiss templates evaluated from simulated
signal events categorised according to \njet, \nb, and
\scalht. Correlations are taken into account where appropriate,
including those relevant to signal contamination in the control
regions. The morphing scheme that interpolates between the nominal and
alternative \HTmiss templates, described in Section~\ref{sec:result},
is also used for the simulated signal samples.

In addition to the uncertainty in the integrated luminosity of
4.6\%~\cite{lumi}, the following sources of uncertainty are dominant:
the statistical uncertainty arising from the finite size of simulated
signal samples, the modelling of initial-state radiation (ISR), the
corrections to jet energies (JEC) evaluated in simulation, and the
modelling of scale factors ($\text{SF}_\text{b-tag}$) applied to
simulated event samples that correct for differences in the efficiency
and misidentification probability of b quark jets. The magnitude of
each contribution depends on the model and the masses of the parent
sparticle and LSP. 

The $\mathcal{A}\times\varepsilon$ for models with small mass
splittings (\eg $m_{\PSQ} - m_{\chiz_1} \lesssim m_{\text t}$) is due
in large part to ISR, the modelling of which is evaluated by comparing
the simulated and measured \Pt spectra of the system recoiling against
the ISR jets in \ttbar events, using the technique described in
Ref.~\cite{single-lepton-stop}. The uncertainty can be as large as
$\sim$30\%, and is the dominant systematic uncertainty for systems
with a compressed mass spectrum. Uncertainties in the jet energy
scale, as large as $\sim$40\%, can also be dominant for models
characterised by high jet multiplicities in the final state. The
uncertainties in $\text{SF}_\text{b-tag}$ can be as large as
$\sim$25\%. Table~\ref{tab:signal-eff} summarises these dominant
contributions to the uncertainty in $\mathcal{A}\times\varepsilon$ for
a range of benchmark models. Characteristic values for each model are
expressed in terms of a range that is representative of the values
across all bins of the signal region. The upper bound for each range
may be subject to moderate statistical fluctuations. 

Further uncertainties with subdominant contributions are considered on
a similar footing. The uncertainties in the efficiency of identifying
well-reconstructed, isolated leptons are considered, with a typical
magnitude of $\sim$5\% and treated as anti-correlated between the
signal and control regions. An uncertainty of 5\% in the minimum bias
cross section, $\sigma_\text{MB} = 69.0 \pm 3.5\unit{mb}$, is assumed
and propagated through to the reweighting procedure to account for
differences between the simulated and data-derived measurements of the
pileup distributions. Finally, uncertainties in the simulation
modelling of the efficiencies of the trigger strategy employed by the
search are typically $<$10\%.

The choice of PDF set, or variations therein, predominantly affects
$\mathcal{A}\times\varepsilon$ through changes in the \Pt spectrum of
the system recoil, which is covered by the ISR uncertainty, hence no
additional uncertainty is adopted. Uncertainties in
$\mathcal{A}\times\varepsilon$ due to variations in the
renormalisation and factorisation scales are determined to be
relatively small. In both cases, contributions to the uncertainty in
the theory production cross section are considered.

\subsection{Cross section and mass exclusions}

The upper limits on the signal production cross section are evaluated
at a 95\% CL for each of the aforementioned benchmark models. The
limits are expressed in terms of the signal strength parameter, $\mu$,
which is determined relative to the theory cross section that is
calculated at NLO+NLL accuracy. The limits are summarised in
Table~\ref{tab:signal-eff}. Expected limits on $\mu$ are also listed,
which are determined using an asimov data set. All benchmark models
are disfavoured based on expectations. The observed limits fluctuate
around the expected $\mu$ values, with some models exhibiting a
moderately weaker-than-expected limit due to fluctuations in data, as
discussed below.

\begin{figure*}[!h]
  \begin{center}
    \includegraphics[width=0.55\textwidth]{figures/limits/v1/mixSUMMARY.pdf}
    \includegraphics[width=0.55\textwidth]{figures/limits/v1/gluinoSUMMARY.pdf} 
    \caption{Observed and expected mass exclusions at 95\% CL
      (indicated, respectively, by solid and dashed contours) for
      a number of simplified models. (Top) The pair production of
      gluino-mediated or direct pair production of light-flavour
      squarks. The two scenarios involve, respectively, the
      decay $\sGlu\ra\cPaq\cPq\chiz_1$ (\texttt{T1qqqq}) and
      $\PSQ\ra\cPq\chiz_1$, and the latter involves two assumptions on
      the mass degeneracy of the squarks (\texttt{T2qq\_8fold} and
      \texttt{T2qq\_1fold}). (Bottom) Three scenarios involving the
      gluino-mediated pair production of off-shell third-generation
      squarks: $\sGlu\ra\cPaqb\cPqb\chiz_1$ (\texttt{T1bbbb}),
      $\sGlu\ra\cPaqt\PSQt^*\ra\cPaqt\cPqt\chiz_1$ (\texttt{T1tttt}),
      and $\sGlu\ra\cPaqt\cPqb\chipm_1\ra\cPaqt\cPqb\PW^*\chiz_1$
      (\texttt{T1tbb}). 
}
    \label{fig:limits-sms-1} 
    \vspace{1.0cm} % hack to remove text from same page
  \end{center}
\end{figure*}

\begin{figure*}[!h]
  \begin{center}
    \includegraphics[width=0.55\textwidth]{figures/limits/v1/naturalWT1SUMMARY.pdf}
%    \includegraphics[width=0.6\textwidth]{figures/limits/v1/naturalSUMMARY.pdf}
    \includegraphics[width=0.55\textwidth]{figures/limits/v1/allThirdGenSUMMARY.pdf} 
    \caption{ Observed and expected mass exclusions at 95\% CL
      (indicated, respectively, by solid and dashed contours) for a
      number of simplified models. (Top) Three scenarios involving the
      gluino-mediated pair production of on-shell top squarks:
      $\sGlu\ra\cPaqt\PSQt\ra\cPaqt\cPqt\chiz_1$ with $m_{\,\PSQt} -
      m_{\chiz_1} = 175\GeV$ (\texttt{T5tttt}),
      $\sGlu\ra\cPaqt\PSQt\ra\cPaqt\cPqc\chiz_1$ with $m_{\,\PSQt} -
      m_{\chiz_1} = 20\GeV$ (\texttt{T5ttcc}), and
      $\sGlu\ra\cPaqt\PSQt\ra\cPaqt\cPqb\PW^*\chiz_1$ with
      $m_{\,\PSQt} - m_{\chiz_1} = 20\GeV$
      (\texttt{T5tttt\_degen}). (Bottom) Six scenarios involving the
      direct pair production of third-generation squarks. The first
      scenario involves the pair production of bottom squarks,
      $\PSQb\ra\cPqb\chiz_1$ (\texttt{T2bb}). Two scenarios involve
      the decay of top squark pairs as follows: $\PSQt\ra\cPqt\chiz_1$
      or $\PSQt\ra\cPqb\chipm_1\ra\cPqb\PW^*\chiz_1$ with
      $m_{\chipm_1} - m_{\chiz_1} = 5\GeV$ and $\mathcal{BR} =
      50/50\%$ (\texttt{T2tb}), or $\PSQt\ra\cPqt\chiz_1$
      (\texttt{T2tt}). The final three sceanarios consider top squark
      decays under the assumption $10 < m_{\,\PSQt} - m_{\chiz_1} <
      80\GeV$: $\PSQt\ra\cPqc\chiz_1$ (\texttt{T2cc}),
      $\PSQt\ra\cPqb\PW^*\chiz_1$ (\texttt{T2tt\_degen}), and
      $\PSQt\ra\cPqc\chiz_1$ or $\PSQt\ra\cPqb\PW^*\chiz_1$ with
      $\mathcal{BR} = 50/50\%$ (\texttt{T2tt\_mixed}).  }
    \label{fig:limits-sms-2} 
  \end{center} 
\end{figure*}

Figures~\ref{fig:limits-sms-1} and~\ref{fig:limits-sms-2} summarise
the disfavoured regions of the mass parameter space for fifteen
simplified models. These regions are derived by comparing the upper
limits on the measured fiducial cross section, corrected for the
experimental $\mathcal{A}\times\varepsilon$, with the theory cross
sections calculated at NLO+NLL accuracy in $\alpha_\text{s}$. The
former cross section value is determined as a function of $m_{\sGlu}$
or $m_{\PSQ}$ and $m_{\chiz_1}$, while the latter has a dependence
solely on $m_{\sGlu}$ or $m_{\PSQ}$. For each simplifed model,
exclusion contours in the mass plane are shown when evaluated with the
observed data counts in the signal region (solid contours) and the
expected counts based on an asimov data set (dashed contours).

Figure~\ref{fig:limits-sms-1} (top) shows exclusion contours for
models that assume the gluino-mediated or direct production of
light-flavour squarks. The excluded region extends to higher masses
for the gluino-mediated production of light-flavour squarks
(\texttt{T1qqqq}), with respect to the direct pair production when
assuming an eightfold degeneracy in mass (\texttt{T2qq\_8fold}), due
to a combination of a higher gluino pair production cross section and
a final state characterised by higher jet multiplicities, which can be
exploited to provide better signal-to-background seperation. The
excluded mass region is significantly reduced when assuming only a
single light squark (\texttt{T2qq\_1fold}), with limits weakening due
to the lower production cross section, compounded by the reduced
signal-to-background ratios achieved in the core of distributions in
the discriminating variables.

Figure~\ref{fig:limits-sms-1} (bottom) shows exclusion contours for
models that assume the gluino-mediated pair production of off-shell
third-generation squarks. For the topologies \texttt{T1tttt} and
\texttt{T1bbbb}, each gluino is assumed to undergo a three-body decay
via, respectively, an off-shell top or bottom squark to a
quark-antiquark pair of the same flavour and the $\chiz_1$. In the
case of \texttt{T1ttbb}, each gluino is assumed to undergo a
three-body decay to an on-shell chargino, $\chipm_1$, a bottom quark,
and an antitop quark. The chargino mass is defined relative to the
neutralino mass via the expression $m_{\chipm_1} - m_{\chiz_1} =
5\GeV$. The chargino decays promptly to the $\chiz_1$ and an off-shell
W boson. The excluded mass regions differ significantly for these
topologies, primarily due to the different number of (on-shell) W
bosons in their final states, resulting in the highest $\mathcal{A}
\times \varepsilon$ for \texttt{T1bbbb} and lowest for
\texttt{T1tttt}. Further, $\mathcal{A} \times \varepsilon$ has a
strong dependence on jet multiplicity, which is highest for
\texttt{T1tttt}, due to the \bdphi variable. An additional feature for
\texttt{T1ttbb} is the weakening of the mass limit at low values of
$m_{\chiz_1}$, when $m_{\chipm_1} = m_{\chiz_1} + 5\GeV \lesssim
m_\text{t}$. In this scenario, the $\chipm_1$ (and hence $\chiz_1$) is
not highly Lorentz boosted relative to the top quark resulting from
the three-body decay of the gluino. Hence, two $\chiz_1$ sparticles do
not carry away significant \ptvecmiss, which is instead realised
through W boson decays to neutrinos and ``lost'' leptons or $\tau$
leptons that decay to neutrinos and hadrons. The observed mass limits
for these topologies are up to $\sim$2 standard deviations weaker than
the expected limits. These differences are due to upward fluctuations
in data for two contiguous bins that satisfy the requirements $\njet
\geq 5$, $\nb \geq 2$, and $\scalht > 800\GeV$. This region has the
highest sensitivity to models involving gluino production and decays
to third-generation quarks (via on- or off-shell squarks). The
observed counts are consistent with statistical fluctuations and the
events do not exhibit anomolous nonphysical behaviours. The events are
distributed in \HTmiss consistent with expectation, hence models
characterised by high values of \HTmiss, such as \texttt{T1bbbb} with
$m_{\sGlu} \gg m_{\chiz_1}$ or $m_{\sGlu} \approx m_{\chiz_1}$, are
less compatible with the data counts in this high-\njet, \nb, and
\scalht region.

Figure~\ref{fig:limits-sms-2} (top) shows exclusion contours for
models that assume gluino pair production, with each gluino decaying
to a top quark and an on-shell top squark, the latter of which decays
to SM particles and the LSP, $\chiz_1$. As discussed earlier, these
models can be considered as representations of a ``natural'' solution
to the little hierarchy problem. Three different scenarios are
considered for the decay of the top squarks. The \texttt{T5tttt} model
assumes a two-body decay to a top quark and the $\chiz_1$, with the
top squark mass defined relative to the $\chiz_1$ as $m_{\PSQt} -
m_{\chiz_1} = m_\text{t}$. The \texttt{T5ttcc} and
\texttt{T5tttt\_degen} models assume $m_{\PSQt} - m_{\chiz_1} =
20\GeV$ and two- and four-body decays to, respectively, a charm quark
and the $\chiz_1$, or to bf$\bar{\text{f}}'\chiz_1$ via an off-shell W
boson. These two decays are the only modes open to the top squark
under this near-mass-degenerate scenario.
%For comparison, the limit contours for \texttt{T1tttt}, which assumes
%an off-shell top squark that is decoupled to a high mass, are also
%shown in Fig.~\ref{fig:limits-sms-2} (top). 
%The expected mass exclusion regions for \texttt{T1ttcc} and
%\texttt{T5tttt\_degen} are comparable to that of \texttt{T1tttt}, but
%do exhibit some different behaviour. % NEED DETAILS HERE

Finally, Fig.~\ref{fig:limits-sms-2} (bottom) shows exclusion contours
for models that assume the direct production of pairs of
third-generation squarks. For the model \texttt{T2bb}, bottom squarks
are pair produced and each decays to a bottom quark and the
$\chiz_1$. The model \texttt{T2tt} assumes top squarks are pair
produced and each is assumed to undergo a two- or three-body decay to,
respectively, a top quark and the $\chiz_1$ when $m_{\PSQt} -
m_{\chiz_1} > m_\text{t}$ is satisfied, or a b quark, an on-shell W
boson, and the $\chiz_1$ for the condition $m_{\PW} < m_{\PSQt} -
m_{\chiz_1} < m_\text{t}$. Models that satisfy $\abs{m_{\PSQt} -
  m_\text{t} - m_{\chiz_1}} < 25\GeV$ are not considered here. The
model \texttt{T2tb} also assumes the pair production of top squarks,
with each undergoing a two-body decay to either a top quark and the
$\chiz_1$, or a bottom quark and the $\chipm_1$, with equal branching
ratios $\mathcal{BR}(\PSQt \ra \text{t}\chiz_1) = \mathcal{BR}(\PSQt
\ra \text{b}\chipm_1) = 50\%$. As for the \texttt{T1ttbb} model, the
chargino mass is defined relative to the neutralino mass via the
expression $m_{\chipm_1} - m_{\chiz_1} = 5\GeV$, and the chargino
decays promptly to the $\chiz_1$ and an off-shell W boson. The
excluded mass regions differ significantly for the \texttt{T2bb},
\texttt{T2tb}, and \texttt{T2tt} topologies, in an analogous way to
the \texttt{T1bbbb}, \texttt{T1ttbb}, and \texttt{T1tttt} models
described above. The difference in the mass exclusions is due
primarily to the different number of (on-shell) W bosons in the final
states, which affects $\mathcal{A} \times \varepsilon$ through the
presence of leptons from the decay of the W boson. Further, an
additional feature for \texttt{T2tb} is the weakening of the mass
limit at low values of $m_{\chiz_1}$, when $m_{\chipm_1} = m_{\chiz_1}
+ 5\GeV \lesssim m_\text{t}$. Moderately weaker-than-expected mass
limits are observed for all models involving two-body decays, which is
traced to mild upward fluctuations in data for events satisfying
$\njet = 2$, $\nb = 2$, and $350 < \scalht < 500\GeV$.

Figure~\ref{fig:limits-sms-2} (bottom) also shows exclusion contours
for models that assume the pair production of top squarks but a
near-mass-degenerate system that satisfies $10\GeV < m_{\PSQt} -
m_{\chiz_1} < m_\PW$. Two decays of the top squark are
considered. Analogous to the \texttt{T5ttcc} and
\texttt{T5tttt\_degen} models (but without the gluino), the
\texttt{T2cc} and \texttt{T2tt\_degen} models assume two- and
four-body decays of the top squark to, respectively, a charm quark and
the $\chiz_1$, or to via an off-shell W boson. A third model,
\texttt{T2tt\_mixed}, assumes both these decay modes with an equal
branching ratio, $\mathcal{BR}(\PSQt \ra \text{c}\chiz_1) =
\mathcal{BR}(\PSQt \ra \text{bf}\bar{\text{f}}'\chiz_1) = 50\%$. For
\texttt{T2cc}, the excluded mass region is relatively stable as a
function of the mass splitting $\dm = m_{\PSQt} - m_{\chiz_1}$, with
$\PSQt$ masses excluded up to 400\GeV. For \texttt{T2tt\_degen}, the
excluded mass region is strongly dependent on \dm, weakening
considerably for increasing values of \dm due to the increased
momentum phase space available to leptons produced in the four-body
decay. The model \texttt{T2tt\_mixed} exhibits an intermediate
behaviour. Mass limits for all three models converge for the smallest
mass splitting considered, $\dm = 10\GeV$, when the SM particles from
the \PSQt decay are extremely soft and outside the experimental
acceptance. An approximately contiguous mass exclusion limit is
observed across the transition from the \texttt{T2tt\_degen} four-body
to the \texttt{T2tt} three-body decay of the $\PSQt$, as the top quark
moves on-shell. The excluded mass region weakens further as $\dm \ra
m_\text{t}$.

Table~\ref{tab:simplified-models-limits} summarises the strongest
expected and observed excluded $m_{\sGlu}$ or $m_{\PSQ}$ and
$m_{\chiz_1}$ masses for each simplified model.

\begin{table}[tb]
  \topcaption{Summary of the mass limits obtained for the fifteen
    simplified models. The limits indicate the strongest observed and
    expected (in parentheses) mass exclusions in $\sGlu$, $\PSQ$,
    $\PSQb$, $\PSQt$, and $\chiz_1$. The quoted values have
    uncertainties of $\pm$25\GeV and $\pm$10\GeV for models involving
    the pair production of, respectively, gluinos and squarks.
  }
  \label{tab:simplified-models-limits}
  \centering
  \footnotesize
  \begin{tabular}{ lccc }
    \hline
    Topology               & Parent    & \multicolumn{2}{c}{Best mass limit [GeV]}     \\
    \cline{3-4}
                           & sparticle & $\sGlu / \PSQ / \PSQb / \PSQt$ & $\chiz_1$    \\ [0.5ex]
    \hline
%    \multicolumn{4}{l}{\it Gluinos and light-flavour squarks}                         \\ 
%    \multicolumn{4}{l}{\it Gluinos and off-shell stops and sbottoms}                  \\ 
%    \multicolumn{4}{l}{\it Gluinos and on-shell stops}                                \\ 
%    \multicolumn{4}{l}{\it Direct top squark production}                              \\ 
    \texttt{T1qqqq}        & $\sGlu$   & 1375 \ph(1350)                 & 875 \ph(850) \\ 
    \texttt{T2qq\_8fold}   & $\PSQ$    & 1150 \ph(1075)                 & 600 \ph(550) \\ 
    \texttt{T2qq\_1fold}   & $\PSQ$    & \ph575 \ph\ph(650)             & 275 \ph(275) \\ 
    \texttt{T1bbbb}        & $\sGlu$   & 1575 \ph(1575)                 & 975 (1025)   \\ 
    \texttt{T1tttt}        & $\sGlu$   & 1125 \ph(1325)                 & 475 \ph(600) \\ 
    \texttt{T1ttbb}        & $\sGlu$   & 1375 \ph(1450)                 & 750 \ph(850) \\ 
    \texttt{T5tttt}        & $\sGlu$   & \ph800 \ph(1000)               & 300 \ph(450) \\ 
    \texttt{T5ttcc}        & $\sGlu$   & 1350 \ph(1350)                 & 700 \ph(800) \\ 
    \texttt{T5tttt\_degen} & $\sGlu$   & 1150 \ph(1275)                 & 650 \ph(725) \\ 
    \texttt{T2bb}          & $\PSQb$   & \ph800 \ph\ph(800)             & 360 \ph(400) \\ 
    \texttt{T2tb}          & $\PSQt$   & \ph610 \ph\ph(690)             & 240 \ph(300) \\ 
    \texttt{T2tt} (3-body) & $\PSQt$   & \ph670 \ph\ph(720)             & 210 \ph(240) \\
    \texttt{T2tt} (2-body) & $\PSQt$   & \ph280 \ph\ph(280)             & 200 \ph(200) \\ 
    \texttt{T2cc}          & $\PSQt$   & \ph400 \ph\ph(350)             & 310 \ph(340) \\ 
    \texttt{T2tt\_degen}   & $\PSQt$   & \ph370 \ph\ph(360)             & 360 \ph(350) \\ 
    \texttt{T2tt\_mixed}   & $\PSQt$   & \ph360 \ph\ph(350)             & 350 \ph(340) \\ [0.5ex]
    \hline
  \end{tabular}
\end{table}

















%%%%%%%%%%%%%%%%%%%%%%%%%%%%%%%%%%%%%%%%%%%%%%%%%%%%%%%%%%%%%%%%%%%%%%%%%%%%%%%%
%%%%%%%%%%%%%%%%%%%%%%%%%%%%%%%%%%%%%%%%%%%%%%%%%%%%%%%%%%%%%%%%%%%%%%%%%%%%%%%%
%%%%%%%%%%%%%%%%%%%%%%%%%%%%%%%%%%%%%%%%%%%%%%%%%%%%%%%%%%%%%%%%%%%%%%%%%%%%%%%%
%%%%%%%%%%%%%%%%%%%%%%%%%%%%%%%%%%%%%%%%%%%%%%%%%%%%%%%%%%%%%%%%%%%%%%%%%%%%%%%%

%\clearpage
%\begin{figure*}[tb]
%  \begin{center}
%    \subfigure[\texttt{T1qqqq}]{
%      \includegraphics[height=0.15\textwidth]{figures/diagrams/CMS_logo}
%      \label{fig:T1qqqq_feyn}
%    } ~~
%    \subfigure[\texttt{T2qq}]{
%      \includegraphics[height=0.15\textwidth]{figures/diagrams/T2qq_feyn}
%      \label{fig:T2qq_feyn}
%    } ~~
%    \subfigure[\texttt{T1bbbb}]{
%      \includegraphics[height=0.15\textwidth]{figures/diagrams/CMS_logo}
%      \label{fig:T1bbbb_feyn}
%    } ~~
%    \subfigure[\texttt{T1tttt}]{
%      \includegraphics[height=0.15\textwidth]{figures/diagrams/T1tttt_feyn}
%      \label{fig:T1tttt_feyn}
%    } ~~
%    \subfigure[\texttt{T1ttbb}]{
%      \includegraphics[height=0.15\textwidth]{figures/diagrams/T5ttbb_feyn}
%      \label{fig:T1ttbb_feyn}
%    } \\
%    \subfigure[\texttt{T5tttt}]{
%      \includegraphics[height=0.15\textwidth]{figures/diagrams/T5tttt_feyn}
%      \label{fig:T5tttt_feyn}
%    } ~~
%    \subfigure[\texttt{T5ttcc}]{
%      \includegraphics[height=0.15\textwidth]{figures/diagrams/T5ttcc_feyn}
%      \label{fig:T5ttcc_feyn}
%    } ~~
%    \subfigure[\texttt{T5tttt\_degen}]{
%      \includegraphics[height=0.15\textwidth]{figures/diagrams/T5tttt_degen_feyn}
%      \label{fig:T5tttt_degen_feyn}
%    } ~~
%    \subfigure[\texttt{T2bb}]{
%      \includegraphics[height=0.15\textwidth]{figures/diagrams/T2bb_feyn}
%      \label{fig:T2bb_feyn}
%    } ~~
%    \subfigure[\texttt{T2tb}]{
%      \includegraphics[height=0.15\textwidth]{figures/diagrams/CMS_logo}
%      \label{fig:T2tb_feyn}
%    } \\
%    \subfigure[\texttt{T2tt}]{
%      \includegraphics[height=0.15\textwidth]{figures/diagrams/T2tt_feyn}
%      \label{fig:T2tt_feyn}
%    } ~~
%    \subfigure[\texttt{T2cc}]{
%      \includegraphics[height=0.15\textwidth]{figures/diagrams/CMS_logo}
%      \label{fig:T2cc_feyn}
%    } ~~
%    \subfigure[\texttt{T2tt\_degen}]{
%      \includegraphics[height=0.15\textwidth]{figures/diagrams/CMS_logo}
%      \label{fig:T2tt_degen_feyn}
%    } ~~
%    \subfigure[\texttt{T2tt\_mixed}]{
%      \includegraphics[height=0.15\textwidth]{figures/diagrams/CMS_logo}
%      \label{fig:T2tt_mixed_feyn}
%    } ~~
%    \caption{ Simplified model diagrams that represent unique
%      production and decay modes of supersymmetric
%      particles. Three-body decays of gluinos are assumed to proceed
%      through off-shell squarks. Additional assumptions concerning the
%      mass relations and branching ratios are specified in
%      Table~\ref{tab:simplified-models}. The diagrams labelled
%      \texttt{T1qqqq} and \texttt{T2qq} depict, respectively, the
%      gluino-mediated and direct production of light-flavour
%      squarks. The diagrams labelled \texttt{T1bbbb}, \texttt{T1tttt},
%      and \texttt{T1ttbb} depict models involving the gluino-mediated
%      production of off-shell third-generation squarks. The diagrams
%      labelled \texttt{T5tttt}, \texttt{T5ttcc}, and
%      \texttt{T5tttt\_degen} depict ``natural'' models comprising
%      gluino-mediated production of on-shell top squarks. Finally, the
%      remaining six diagrams depict the direct production of on-shell
%      third-generation squarks, decaying via a range of channels.  }
%    \label{fig:simplified-models}
%  \end{center}
%\end{figure*}




%& PU & Trigger & XS [pb] & XS [fb] (3sf)
%& 1-5  & 1-3  & 0.0460525 & 46.1             
%& 1-9  & 5-13 & 0.677478  & 677\phantom{.0}  
%                                             
%& 1-10 & 1-6  & 0.0439731 & 44.0             
%& 1-15 & 3-12 & 1.08047   & 1080\phantom{.0} 
%                                             
%& 1-4  & 1-3  & 0.0141903 & 14.2             
%& 1-20 & 1-15 & 0.325388  & 325\phantom{.0}  
%                                             
%& 1-4  & 1-3  & 0.0460525 & 46.1             
%& 1-13 & 1-10 & 1.4891    & 1490\phantom{.0} 
%                                             
%& 1-18 & 1-4  & 0.0460525 & 46.1             
%& 1-12 & 1-14 & 0.325388  & 325\phantom{.0}  
%                                             
%& 2-6  & 3-6  & 1.4891    & 1490\phantom{.0} 
%& 4-4  & 3-10 & 3.5251    & 3530\phantom{.0} 
%                                             
%& 1-9  & 2-6  & 0.0856418 & 85.6             
%& 1-8  & 4-7  & 2.26585   & 2270\phantom{.0} 
%                                             
%& 3-5  & 3-4  & 0.163491  & 16.3             
%& 1-20 & 1-11 & 1.4891    & 1490\phantom{.0} 
%                                             
%& 1-16 & 2-12 & 0.0283338 & 28.3             
%& 1-11 & 3-3  & 2.61162   & 2610\phantom{.0} 
%                                             
%& 1-8  & 1-12 & 0.174599  & 175\phantom{.0}  
%& 1-12 & 5-7  & 3.78661   & 3790\phantom{.0} 
%                                             
%& 1-13 & 2-11 & 0.0670476 & 67.0             
%& 1-10 & 5-9  & 3.78661   & 3790\phantom{.0} 
%                                             
%& 1-26 & 5-16 & 5.60471   & 5600\phantom{.0} 
%                                             
%& 1-11 & 2-18 & 8.51615   & 8520\phantom{.0} 
%                                             
%& 1-22 & 2-16 & 8.51615   & 8520\phantom{.0} 

%\clearpage
%\begin{figure*}[thp!]
%  \begin{center}
%    \includegraphics[width=0.49\textwidth]{figures/limits/v1/mixSUMMARY.pdf}
%    \includegraphics[width=0.49\textwidth]{figures/limits/v1/gluinoSUMMARY.pdf} \\
%    \includegraphics[width=0.49\textwidth]{figures/limits/v1/naturalSUMMARY.pdf}
%    \includegraphics[width=0.49\textwidth]{figures/limits/v1/allThirdGenSUMMARY.pdf} \\
%    \caption{Observed upper limit in cross section at 95\% CL
%      (indicated by the colour scale) for simplified models that
%      assume the pair production of gluinos, as a function of the
%      gluino and $\chiz_{1}$ masses for gluino three-body decays to
%      $b\bar{b}\chiz_{1}$ (top left), $q\bar{q}\chiz_{1}$ (top right) and $t\bar{t}\chiz_{1}$ (bottom center). 
%      The black solid thick (thin) line indicates the observed mass
%      exclusion region assuming the nominal (${\pm}1 \sigma$ theory
%      uncertainty) production cross section. The red dashed thick
%      (thin) line indicates the median (${\pm}1 \sigma$ experimental
%      uncertainty) expected exclusion.
%    }
%    \label{fig:limits-sms} 
%  \end{center}
%\end{figure*}


