\section{Summary}
\label{sec:summary}

An inclusive search for new-physics phenomena is reported, based on
data from pp collisions at $\sqrt{s} = 13\TeV$. The data are recorded
with the CMS detector and correspond to an integrated luminosity of
$2.3 \pm 0.1 \fbinv$. The final states analysed contain one or more
jets with large transverse momenta and a significant imbalance of
transverse momentum, as expected from the production of massive
coloured SUSY particles, each decaying to SM particles and the
lightest stable, weakly-interacting, SUSY particle.

%Signal candidate events are categorised according to the number of
%reconstructed jets, the number of jets identified as originating from
%b quarks, and the scalar (\scalht) and the magnitude of the vector
%(\HTmiss) sums of the transverse momenta of jets. 
%The search employs the use of several kinematic variables, including
%\alphat and \bdphi, to suppress the background from QCD multijet
%production to the percent level with respect to other nonmultijet SM
%backgrounds, which are dominated by vector boson and top quark pair
%production.
%The \alphat variable is also employed in the trigger logic that is
%used to record the candidate signal events, which allows the use of
%low thresholds for the momentum sums $\scalht > 200\GeV$ and $\HTmiss
%\gtrsim 130\GeV$. These low thresholds, in addition to the inclusion
%of final states containing a single jet, maximise the experimental
%acceptance to new-physics processes, such as low-mass squark
%signatures, nearly mass-degenerate SUSY models, and other new-physics
%phenomena, such as DM models that postulate the direct production of
%stable, weakly interacting, massive particles in pp collisions.

The sums of the standard model backgrounds are estimated from a
simultaneous binned likelihood fit to the observed yields for samples
of events categorised according to the number of reconstructed jets,
the number of jets identified as originating from b quarks, and the
scalar and the magnitude of the vector sums of the transverse momenta
of jets. In addition to the signal region, \mj, \mmj, and \gj control
regions are included in the likelihood fit. The observed yields are
found to be in agreement with the expected contributions from standard
model processes.  The search result is interpreted in the mass
parameter space of fourteen simplified SUSY models,
%that assume the pair production of gluinos or squarks and a range of
%decay modes.
%The models cover scenarios that involve the gluino-mediated or direct
%production of light- or heavy-flavour squarks, spectra with
%intermediate SUSY particle states and nonunity branching fractions,
%``natural'' spectra with gluinos and on-shell top squarks, and nearly
%mass-degenerate spectra.
which cover scenarios that involve the gluino-mediated or direct
production of light- or heavy-flavour squarks, intermediate SUSY
particle states, as well as natural and nearly mass-degenerate
spectra.

The increase in the centre-of-mass energy of the LHC, from 8 to
13\TeV, provides a significant gain in sensitivity to heavy particle
states such as gluinos. In the case of pair-produced gluinos, each
decaying via an off-shell b squark to the b quark and the LSP, models
with masses up to $\sim$1.6 and $\sim$1.0\TeV are excluded for,
respectively, the gluino and LSP. These limits improve on those
obtained at $\sqrt{s} = 8\TeV$ by, respectively, $\sim$250 and
$\sim$300\GeV. In the case of direct pair production, models with
masses up to $\sim$800 and $\sim$350\GeV are excluded for,
respectively, the b squark and LSP. These mass limits are sensitive to
the assumptions on the squark flavour and the presence of intermediate
states such as charginos.

Finally, a comprehensive study of nearly mass-degenerate models
involving top squark pair production is performed. The two decay modes
of the top squark are the loop-induced two-body decay to the
neutralino and one c quark, and the four-body decay to the neutralino,
one b quark, and an off-shell W boson. A third scenario is considered
in which the two modes are simultaneously open, each with a branching
fraction of 50\%. Masses of the top squark and LSP up to,
respectively, 400 and 360\GeV are excluded, depending on the decay
modes considered.

In conclusion, the analysis provides sensitivity across a large region
of the natural SUSY parameter space, as characterised by
interpretations with several simplified models. In particular, these
studies improve on existing limits for nearly mass-degenerate models
involving the production of pairs of top squarks.
