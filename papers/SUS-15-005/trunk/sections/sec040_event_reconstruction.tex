%%__________________________________________________________________||
\section{Event reconstruction}
\label{sec:event_reconstruction}

The global event reconstruction (also called particle-flow event
reconstruction~\cite{CMS-PAS-PFT-09-001,CMS-PAS-PFT-10-001}) consists
in reconstructing and identifying each single particle with an
optimized combination of all subdetector information. In this process,
the identification of the particle type (photon, electron, muon,
charged hadron, neutral hadron) plays an important r\^ole in the
determination of the particle direction and energy. Photons (\eg
coming from \Pgpz\ decays or from electron bremsstrahlung) are
identified as ECAL energy clusters not linked to the extrapolation of
any charged particle trajectory to the ECAL. Electrons (\eg coming
from photon conversions in the tracker material or from \cPqb-hadron
semileptonic decays) are identified as a primary charged particle
track and potentially many ECAL energy clusters corresponding to this
track extrapolation to the ECAL and to possible bremsstrahlung photons
emitted along the way through the tracker material. Muons (\eg from
\cPqb-hadron semileptonic decays) are identified as a track in the
central tracker consistent with either a track or several hits in the
muon system, associated with an energy deficit in the
calorimeters. Charged hadrons are identified as charged particle
tracks neither identified as electrons, nor as muons. Finally, neutral
hadrons are identified as HCAL energy clusters not linked to any
charged hadron trajectory, or as ECAL and HCAL energy excesses with
respect to the expected charged hadron energy deposit.

The energy of photons is directly obtained from the ECAL measurement,
corrected for zero-suppression effects. The energy of electrons is
determined from a combination of the track momentum at the main
interaction vertex, the corresponding ECAL cluster energy, and the
energy sum of all bremsstrahlung photons attached to the track. The
energy of muons is obtained from the corresponding track momentum. The
energy of charged hadrons is determined from a combination of the
track momentum and the corresponding ECAL and HCAL energy, corrected
for zero-suppression effects and for the response function of the
calorimeters to hadronic showers. Finally, the energy of neutral
hadrons is obtained from the corresponding corrected ECAL and HCAL
energy.

For each event, hadronic jets are clustered from these reconstructed
particles with the infrared and collinear safe anti-\kt algorithm,
operated with a size parameter $R$ of 0.5. The jet momentum is
determined as the vectorial sum of all particle momenta in this jet,
and is found in the simulation to be within 5 to 10\% of the true
momentum over the whole \pt spectrum and detector acceptance. Jet
energy corrections are derived from the simulation, and are confirmed
with in situ measurements with the energy balance of dijet and photon
+ jet events~\cite{Chatrchyan:2011ds}. The jet energy resolution
amounts typically to 15\% at 10\GeV, 8\% at 100\GeV, and 4\% at 1\TeV,
to be compared to about 40\%, 12\%, and 5\% obtained when the
calorimeters alone are used for jet clustering.

The missing transverse momentum vector \ptvecmiss is defined as the
projection on the plane perpendicular to the beams of the negative
vector sum of the momenta of all reconstructed particles in an
event. Its magnitude is referred to as \ETmiss.
