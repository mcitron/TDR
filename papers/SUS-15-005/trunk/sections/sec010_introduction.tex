\section{Introduction}
\label{sec:introduction}

The standard model (SM) of particle physics is successful in
describing a wide range of phenomena, although it is widely believed
to be only an effective approximation of a more complete theory that
supersedes it at a higher energy scale. Supersymmetry
(SUSY)~\cite{ref:SUSY-1, ref:SUSY0, ref:SUSY3, ref:SUSY1} is a
modification to the SM that extends its underlying space-time symmetry
group. For each boson (fermion) in the SM, a fermionic (bosonic)
superpartner, which differs in spin by one-half unit, is introduced.

Experimentally, SUSY is testable through the prediction of an
extensive array of new observable states (of unknown
masses)~\cite{ref:SUSY4, ref:SUSY2}. In the minimal supersymmetric
extension to the SM~\cite{ref:SUSY2}, the gluinos $\sGlu$, light- and
heavy-flavour squarks $\PSQ, \PSQb, \PSQt$, and sleptons $\PSl$ are,
respectively, the superpartners to gluons, quarks, and leptons. An
extended Higgs sector is also predicted, as well as four neutralino
$\chiz_{1,2,3,4}$ and two chargino $\chipm_{1,2}$ states that arise
from mixing between the higgsino and gaugino states, which are the
superpartners of the Higgs and electroweak gauge bosons.  The
assumption of $R$-parity conservation~\cite{Farrar:1978xj} has
important consequences for cosmology and collider
phenomenology. Supersymmetric particles are expected to be produced in
pairs at the LHC, with heavy states decaying eventually to the stable
lightest SUSY particle (LSP). The LSP is generally assumed to be the
$\chiz_1$, which is weakly interacting and massive. This SUSY particle
is considered to be a candidate for dark matter
(DM)~\cite{Jungman:1995df}, the existence of which is supported by
astrophysical data~\cite{1674-1137-38-9-090001}.  Hence, the
characteristic signature of natural SUSY production at the LHC is a
final state containing an abundance of jets, possibly originating from
top or bottom quarks, accompanied by a significant transverse momentum
imbalance, \ptvecmiss.

The proposed supersymmetric extension of the SM is also compelling
from a theoretical perspective, as the addition of superpartners to SM
particles can modify the running of the gauge coupling constants such
that their unification can be achieved at a high energy
scale~\cite{Dimopoulos:1981yj, Ibanez:1981yh, Marciano:1981un}. A
more topical perspective, given the recently discovered Higgs
boson~\cite{ref:atlashiggsdiscovery, ref:cmshiggsdiscovery,
  ref:cmshiggsdiscoverylong}, is the possibility that the
superpartners can alleviate the gauge hierarchy problem of the SM by
compensating for radiative corrections to the Higgs boson mass from
loop processes involving SM particles~\cite{ref:hierarchy1,
  ref:hierarchy2}. These corrections have a power-law dependence on
scale, which can only be accommodated in the SM through an extreme
level of fine tuning of the bare Higgs boson mass. 
%A ``natural'' solution from SUSY, with minimal fine-tuning, requires
%the gluino, third-generation squarks, and the (higgsino-like) LSP to
%have masses at or near the electroweak
%scale~\cite{ref:barbierinsusy}. 
%A ``natural'' solution from SUSY, with minimal fine-tuning, requires
%the mass parameters associated with the $\sGlu$, $\PSQt_\text{L}$,
%$\PSQt_\text{R}$, $\PSQb_\text{L}$, and higgsinos to be at or near the
%electroweak scale~\cite{ref:barbierinsusy}. These natural constraints,
%which can be relaxed at the expense of some level of fine tuning,
%imply the presence of states that may be produced by the CERN LHC.
A ``natural'' solution from SUSY, with minimal fine-tuning, implies
the presence of the gluino, third-generation squarks, and a
(higgsino-like) $\chiz_1$ to be at or near the electroweak
scale~\cite{ref:barbierinsusy}.

The lack of evidence to date for SUSY has also focused attention on
regions of the parameter space with sparse coverage, which includes
natural models~\cite{Delgado:2012eu, Boehm:1999tr, Carena:2008mj,
  Grober:2014aha, Grober:2015fia}. For example, models in which both
the $\PSQt$ and the $\chiz_1$ are light and nearly degenerate in mass
are phenomenologically well motivated~\cite{Boehm:1999bj,
  Balazs:2004bu, Martin:2007gf, Martin:2007hn}. This class of models,
with ``compressed'' mass spectra, typically yield SM particles with
low transverse momenta (\Pt) from the decays of SUSY particles. Hence,
searches rely on the associated production of jets, often resulting
from initial-state radiation (ISR), to achieve experimental
acceptance.

This paper presents an inclusive search for new-physics phenomena in
hadronic final states with one or more energetic jets and an imbalance
in \ptvecmiss, performed in proton-proton (pp) collisions at a
centre-of-mass energy $\sqrt{s} = 13\TeV$. The analysed data sample
corresponds to an integrated luminosity of $2.3 \pm 0.1
\fbinv$ %~\cite{lumi}
collected by the CMS experiment. Earlier searches using the same
technique have been performed in pp collisions at both $\sqrt{s} =
7$~\cite{RA1Paper, RA1Paper2011, RA1Paper2011FULL} and
$8\TeV$~\cite{RA1Paper2012, RA1Parked} by the CMS Collaboration.
%Previous iterations of this search have been performed in pp
%collisions at both $\sqrt{s} = 7$~\cite{RA1Paper, RA1Paper2011,
%  RA1Paper2011FULL} and $8\TeV$~\cite{RA1Paper2012, RA1Parked}.
The increase in the centre-of-mass energy of the LHC, from $\sqrt{s} =
8$ to 13\TeV, provides a unique opportunity to search for the
characteristic signatures of new physics at the TeV scale. For
example, the increase in $\sqrt{s}$ leads to a factor $\gtrsim$35
increase in the parton luminosity~\cite{susynlo} for the pair
production of coloured SUSY particles, each of mass 1.5\TeV, which
were beyond the reach of searches performed at $\sqrt{s} = 8\TeV$ by
the ATLAS~\cite{Aad:2015iea, Aad:2015pfx} (and references therein) and
CMS~\cite{CMS:2014dpa, Khachatryan:2015vra, Khachatryan:2016oia,
  Chatrchyan:2013wxa, Chatrchyan:2014lfa, Khachatryan:2015pwa,
  Khachatryan:2015wza, Khachatryan:2016zcu} Collaborations. Several
searches in this final state, interpreted within the context of SUSY,
have already provided results with the first data at this new energy
frontier~\cite{Aad:2016jxj, Aaboud:2016tnv, Aaboud:2016zdn,
  Aad:2016eki, Aaboud:2016nwl, Khachatryan:2016kdk, cms-13}.

An inclusive strategy is adopted to ensure sensitivity to the broadest
possible region of the SUSY parameter space. The strategy focuses on
maintaining high acceptance through the application of selection
criteria with the lowest possible thresholds, and the categorisation
of candidate signal events according to multiple discriminating
variables.  The search is sufficiently generic and inclusive to
provide sensitivity to a wide range of non-SUSY %supersymmetric
models that postulate the existence of a stable, weakly interacting,
massive particle. In addition to the jets$\,$+$\,$\ptvecmiss topology,
the search considers final states containing a ``monojet'' topology,
which is expected to improve the sensitivity to DM particle production
in pp collisions~\cite{Fox:2012ee, Buchmueller:2015eea}.
%The coverage of the broad parameter space, as characterised by a
%simplified model approach that assumes \eg vector, axialvector,
%scalar, or pseudoscalar mediated interactions~\cite{Fox:2012ee,
%  Buchmueller:2015eea}, is complementary to non-collider-based
%experiments that aim to observe DM interactions either indirectly,
%through the potential signatures of DM annihilations in astrophysical
%data, or directly, through the search for nonnegligible DM-nucleon
%scattering cross sections.

This search is based on an examination of the number of reconstructed
jets per event, the number of these jets identified as originating
from bottom quarks, and the scalar and vector \Pt sums of these
jets. These variables provide sensitivity to the different production
mechanisms (squark-squark, squark-gluino, and gluino-gluino) of
massive coloured SUSY particles at hadron colliders, third-generation
squark signatures, and both large and small mass splittings between
the parent SUSY particle and the LSP.

The dominant background process for a search in all-jet final states
is multijet production, a manifestation of quantum chromodynamics
(QCD). An accurate estimate of this background is difficult to
achieve, given the lack of precise theoretical predictions for the
multijet production cross section and kinematic properties. 
%Further, the discovery potential of a search in the all-jets channel
%can be particularly sensitive to the uncertainties in the estimation
%of the multijet background. 
Hence, this search adopts a strategy that employs several variables to
reduce the multijet contribution to a low level with respect to other
SM backgrounds, rather than estimate a significant contribution with
high precision.

%The search is built around two variables that are designed to provide
%robust discrimination against multijet events at the energy and
%luminosity frontier. A dimensionless kinematic variable
%\alphat~\cite{Randall:2008rw, RA1Paper} provides powerful
%discrimination against multijet production. The \alphat variable is
%constructed from jet-based quantities and provides discrimination
%between genuine sources of \ptvecmiss from stable, weakly interacting
%particles such as neutrinos or neutralinos that escape the detector,
%and instrumental sources such as the mismeasurements of jet energies.

The search is built around two variables that are designed to provide
robust discrimination against multijet events at the energy and
luminosity frontier. A dimensionless kinematic variable
\alphat~\cite{Randall:2008rw, RA1Paper} is constructed from jet-based
quantities and provides discrimination between genuine sources of
\ptvecmiss from stable, weakly interacting particles such as neutrinos
or neutralinos that escape the detector, and instrumental sources such
as the mismeasurements of jet energies. The \bdphi~\cite{RA1Paper}
variable exploits azimuthal angular information and also provides
strong rejection power against multijet events, including rare
energetic events in which neutrinos carry a significant fraction of
the energy of a jet due to semileptonic decays of heavy-flavour
mesons. Very restrictive requirements on the \alphat and \dphi
variables are employed in this search to ensure a low level of
contamination from the multijet background.

The organisation of this paper is as
follows. Sections~\ref{sec:detector} and~\ref{sec:simulation}
describe, respectively, the CMS apparatus and the simulated event
samples. Sections~\ref{sec:event_reconstruction}
and~\ref{sec:event_selection} describe the event reconstruction and
selection criteria used to identify candidate signal events and
control region samples. Section~\ref{sec:backgrounds} provides details
on the estimation of the multijet and all other SM
backgrounds. Finally, the search results and interpretations, in terms
of simplified SUSY models, are described in
Sections~\ref{sec:result} and~\ref{sec:interpretations}, and
summarised in Section~\ref{sec:summary}.


