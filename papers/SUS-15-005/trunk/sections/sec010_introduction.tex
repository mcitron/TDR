\section{Introduction}
\label{sec:introduction}

Supersymmetry (SUSY)~\cite{ref:SUSY-1, ref:SUSY0, ref:SUSY1,
  ref:SUSY2, ref:SUSY3, ref:SUSY4, ref:hierarchy1, ref:hierarchy2}
is a complete, renormalisable extension to the Standard Model
that can provide a solution to the hierarchy problem of the SM Higgs
boson~\cite{ref:atlashiggsdiscovery, ref:cmshiggsdiscovery} if
SUSY is realised at the TeV scale. Further, the assumption of R-parity
conservation~\cite{Farrar:1978xj} has important consequences for
collider phenomenology and cosmology. Supersymmetric particles
(sparticles) such as gluinos and squarks are expected to be produced
in pairs at the LHC and promptly decay to the lightest stable
supersymmetric particle (LSP), which is generally assumed to be a
weakly interacting, massive neutralino and a dark matter
candidate. The characteristic signature of SUSY production at the LHC
is a final state of multijets accompanied by significant missing
transverse momentum, \ptvecmiss.

The new energy frontier of the LHC during Run~2 provides a unique
opportunity to search for the characteristic signatures of TeV-scale
sparticles. This paper presents an inclusive search
for the pair production of massive coloured sparticles in hadronic
final states with two or more energetic jets and missing transverse
momentum, performed in pp collisions at a centre-of-mass energy
$\sqrt{s} = 13\TeV$. The analysed data sample corresponds to an
integrated luminosity of $2.3 \pm 0.1 \fbinv$~\cite{lumi} collected by
the Compact Muon Solenoid (CMS) experiment. Previous iterations of
this search have been performed in pp collisions at both $\sqrt{s} =
7$~\cite{RA1Paper, RA1Paper2011, RA1Paper2011FULL} and
$8\TeV$~\cite{RA1Paper2012}.

The search strategy is based around two key aspects in order to
achieve a robust, inclusive search capable of exploiting the potential
of the new LHC energy frontier under the challenging conditions of new
beam and detector configurations early in Run~2. First, multiple tight
selection criteria are employed to suppress multijet production, a
manifestation of quantum chromodynamics (QCD), to a negligible level
relative to all other SM background processes. Second, the
experimental acceptance to a potential signal is maximised through the
use of trigger conditions that maintain the same low thresholds
employed during Run~1.

The strategy is built around the use of the kinematic variable
\alphat~\cite{Randall:2008rw, RA1Paper}, which provides powerful
discrimination against multijet production. The \alphat variable is
constructed from jet-based quantities to provide robust discriminating
power between sources of genuine and misreconstructed missing
transverse momentum, making it suitable for early searches operating
at new energy and luminosity frontiers. The \alphat variable is
utilised as part of the trigger conditions, providing high performance
in terms of maintaining low thresholds for a given trigger
bandwidth. Further variables are also employed to discriminate against
multijet production and suppress this background process to a
negligible level. The $\Delta\phi^{*}_{\rm min}$~\cite{RA1Paper}
variable exploits azimuthal angular information and also provides
strong rejection power against multijet events, including rare
energetic events in which neutrinos carry a significant fraction of a
jet's energy due to semileptonic decays of heavy-flavour mesons.


%%% @@
The high-energy, high-luminosity hadron collider environment at the
LHC, coupled with the lack of precise theoretical predictions for the
cross section and kinematic properties of multijet events, ensures
that estimating the backgorund
This difficulty is compounded by the 
The measurement of \ptvecmiss, relative to other physics objects, is
particularly sensitive to the beam conditions and detector
performance. 
Given these difficulties, this search adopts a strategy that employs
several variables in an attempt to reduce the multijet contribution to
a negligible level with respect to other SM backgrounds, rather than
estimate with high precision a nonnegligible contribution. 


%The search is devised around the kinematic variable
%\alphat~\cite{Randall:2008rw, RA1Paper} that provides powerful
%discrimination against multijet production, a manifestation of quantum
%chromodynamics (QCD), and adheres to an inclusive strategy with the
%aim of providing sensitivity to the widest possible range of SUSY
%models. The \alphat variable is constructed from jet-based quantities
%to provide robust discriminating power between sources of genuine and
%misreconstructed missing transverse momentum, making it suitable for early
%searches. A further variable that exploits azimuthal angular
%information, known as $\Delta\phi^{*}_{\rm min}$~\cite{RA1Paper}, is
%also employed to suppress QCD multijet production, including potential
%contributions from semileptonic heavy-flavour decays, to a negligible
%level.

The search is based on an examination of the number of reconstructed
jets per event, the number of these jets identified as originating
from bottom quarks, and the scalar and vector sums of transverse
momenta of these jets. 
%The search exploits the use of \mht templates derived from simulation,
%which are extensively validated in data control regions. 
These discriminating variables provide sensitivity to the different
production mechanisms of massive coloured sparticles at hadron
colliders (\ie squark-squark, squark-gluino, and gluino-gluino),
third-generation squark signatures, and a large range of mass
splittings between the parent sparticle and the LSP,
respectively. Interpretations of the result are provided in the
parameter space of simplified models~\cite{Alwall:2008ag,
  Alwall:2008va, sms} that represent the pair production of gluinos
and their subsequent prompt decays to four quarks and two LSPs via
off-shell squarks of light or heavy flavour, as illustrated in
Fig.~\ref{fig:feyn}.


