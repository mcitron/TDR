%%__________________________________________________________________||
\section{Background estimation for SM processes with genuine \ETmiss}
\label{sec:ewk_background}

In the absence of multijet events, the background counts in the signal
region arise from SM processes with significant \ETmiss in the final
state. In events with low counts of jets and b-quark jets, the largest
backgrounds with genuine \ETmiss are from the associated production of
W or Z bosons with jets, followed by either the weak decays \znunu or
\wtaunu, where the $\tau$ decays hadronically and is identified as a
jet, or by leptonic decays that are not rejected by the dedicated
electron or muon vetoes. The veto of events containing isolated tracks
is efficient at further suppressing these backgrounds as well as the
single-prong hadronic decay of the tau lepton. At higher jet and
b-quark jet multiplicities, top quark production followed by
semileptonic weak top quark decay becomes important.

The production of W and Z bosons in association with jets, \ttbar and
\gj processes are simulated with the \MADGRAPH V5~\cite{madgraph}
event generator. The production of single-top quark events is
generated with \POWHEG~\cite{powheg}, and diboson events are produced
with \PYTHIA8.1~\cite{pythia8}. For all simulated samples, \PYTHIA8.1
is used to describe parton showering and hadronisation. All samples
are generated using the \textsc{cteq6l1}~\cite{Pumplin:2002vw} parton
distribution functions (PDF). The description of the detector response
is implemented using the \GEANTfour~\cite{geant} package. The
simulated samples are normalised using the most accurate cross section
calculations currently available, usually with
next-to-leading-order (NLO) accuracy (NNLO in the case of \ttbar). \\
Corrections to the normalisation of the \gj, W+jets, Z+jets and \ttbar+jets 
processes are derived using sidebands in the \mht variable in data control samples. 
To model the effects of pileup, the simulated events are generated with a nominal distribution
of pp interactions per bunch crossing and then reweighted to match the
pileup distribution as measured in data.

The method to estimate the non-multijet backgrounds in the signal
region relies on the use of transfer factors, which are constructed
per bin (in terms of \njet, \nb, and \scalht), per data control
sample and per background. The transfer factors are determined from the simulated event
samples and are ratios of expected yields in the corresponding bins of
the signal region and control samples. The transfer factors are used
to extrapolate from the event yields measured in data control samples
to an expectation for the total background event yields in the signal
region.

Three disjoint data control regions, binned identically to the signal
region, are used to estimate the contributions from the various
remaining SM background processes. The control regions are defined by
a selection of \mj, \mmj, and \gj events. The selection criteria are
chosen such that the SM processes and their kinematic properties
resemble as closely as possible the SM background behaviour in the
signal region once the muon, dimuon system, or photon are ignored when
computing quantities such as \scalht, \mht and \alphat. The baseline
selection criteria and binning definition described in
Table~\ref{tab:selections} are applied to all control samples, except
for the lepton and photon vetoes, which are inverted and tightened to
improve purity. The event selection criteria are defined to also
ensure that any potential contamination from multijet events or a wide
variety of SUSY models (\ie signal contamination) is negligible.

The \mj sample is recorded using a trigger condition that requires an
isolated muon and the event selection criteria are chosen in order to
ensure high trigger efficiency. Furthermore, the muon is required to
be well separated from the jets in the event and the transverse mass
($M_{\rm T}$) of the muon and \ETmiss~\cite{CMS-PAS-PFT-09-001,
  CMS-PAS-PFT-10-001} system must satisfy $30 < M_{\rm T} < 125\gev$
to ensure a sample rich in W bosons (produced promptly or from the
decay of top quarks). The \mmj sample uses similar selection criteria
as the \mj sample and the same trigger condition. Exactly two
oppositely-charged, isolated muons are required, the muons must be
distanced from the jets in the event, and the invariant mass of the
dimuon system must be within a window of $\pm 25\GeV$ around the mass
of the Z boson. For both the muon and dimuon samples, no requirement
is made on the variable \alphat in order to increase the statistical
precision of the predictions derived from these samples, in constrast
to the identical \alphat requirements made for the signal region and
photon control sample. The \gj sample is recorded using a single
photon trigger condition. The event selection criteria comprise an
isolated photon with $\Et > 200\gev$ and $\scalht > 400\GeV$.

Three independent estimates of the irreducible background of \znunu +
jets events are determined from the \gj, \mmj, amd \mj data control
samples. The \gj and \zmumu + jets processes have similar kinematic
properties when the photon or muons are ignored~\cite{Bern:2011pa}, 
albeit different acceptances. In addition, the \gj process has a
larger production cross section than \znunu + jets events. The \mj
data sample is used to provide an estimate for the \znunu\ + jets
contribution as well as the other dominant SM processes, \ttbar and W
boson production. Residual contributions from processes such as
single-top-quark, diboson, and Drell-Yan production are also included.

\begin{table}[h!]
  \caption{
    Systematic uncertainties (percent) in the estimates of the
    normalisation of the SM background components using transfer factors. 
    The quoted ranges correspond to the variations across \scalht bins 
    and jet topology. 
  } 
  \label{tab:bkgd_systs}
  \centering
  \footnotesize
  \begin{tabular}{ ccccc }
    \hline
    \hline
    Systematic source & \multicolumn{4}{c}{Uncertainty} \\    
     & $\mj \Rightarrow \znunu$  & $\mmj \Rightarrow \znunu$ & $\gj \Rightarrow \znunu$ & $\mj \Rightarrow \ttbar+W$\\
    \hline
    \alphat/\bdphi extrapolation & $5-80\%$ & $50-80\%$ & - & $5-80\%$ \\
    W/Z ratio & $10-30\%$ & - & - & -  \\
    Z/$\gamma$ ratio & - & - & $10-30\%$ & -  \\
    W/\ttbar admixture & - & - & - & $10-100\%$  \\
    W/Z acceptance & $5-50\%$ & - & - & $5-50\%$  \\
    Jet energy scale & $<15\%$ & $<10\%$ & $<15\%$ & $<15\%$ \\
    B-tagging efficiency & $<5\%$ & $<2\%$ & $<2\%$ & $<5\%$ \\
    Pileup & $<6\%$ & $<4\%$ & $<3\%$ & $<10\%$ \\
    Top $p_{T}$ weights & $<20\%$  & $<4\%$ & - & $<5\%$ \\
    Lepton selection & - & - & - & $2-5\%$ \\
    \hline
    \hline
  \end{tabular}
\end{table}

Several sources of uncertainties are considered in the trasfer factors 
derived from simulation and they are summarised in Tab.~\ref{tab:bkgd_systs}. 
When the simulation is corrected to match the data, 
like for the b-tagging efficiency or the jet energy response, 
the uncertainty on those weights is propagated to the transfer factors 
and taken as fully correlated across the bins. 
%% Some of them, like b-tagging efficiency, jet energy scale, lepton efficiency, 
%% are estimated by varying the corresponding weights in the simulation and 
%% recomputing the transfer factors. These uncertainties are taken as fully correlated 
%% across all the analysis bins. 
Other systematics effects are inspected through data-driven tests probing 
the compatibility of yields in two disjoint data control samples and 
a corresponding transfer factor derived from simulation~\cite{RA1Paper2012}. 
Each of these tests is built to target specific key ingredients in the simulation modelling, 
like the $Z / \gamma$ ratio or the \alphat cut efficiency. 
The level of closure and its statistical precision are 
used to extract systematic uncertainties as a function of \scalht, 
while considering ``asymmetric'' and ``symmetric'' jet topologies separately, 
which are taken as uncorrelated. \\
The systematic uncertainties in the transfer factors is propagated  
to the yields for a given component of background in the signal region, 
in each \njet, \nb, and \scalht bin. 

Templates derived from simulation are used to predict the background
counts in the \mht dimension. Multiple data control regions are used
to evaluate the degree to which the simulation describes the \mht
distributions observed in data, and to assign appropriate systematic
uncertainties that can be in excess of $>100\%$ in the most sensitive
\mht bins. Independent systematic uncertainties in the templates for
the \znunu + jets background and the W + jets and \ttbar backgrounds
are treated as fully uncorrelated across \njet and \nb categories and
\scalht bins, and with respect to the ``normalisation'' systematic
uncertainties summarised in Table~\ref{tab:bkgd_systs}.

%%__________________________________________________________________||
