%%____________________________________________________________________________||
\section{Triggers}
\label{sec:triggers}


\subsection{Hadronic signal region}

%% Additional items to discuss?
%%
%% $\alphaT$ and $\ht$ computed on central jets restricted to $|\eta| < 3$ with 40 GeV threshold.
%% Static alphaT
%% List of samples, description of method. Pileup mixed QCD pTHAT samples (crosssections) for the three beam scenarios.
%% Trigger thresholds optimised to be efficient to a range of signal models and SM backgrounds.
%% Table of triggers, rates and eff for (compressed model/backgrounds)

In Run 2 the RA1 analysis will aim to retain the low-thresholds of Run 1 with developments to the trigger selection, maintaining sensitivity to signatures of new physics with hadronic energy as low as $\scalht = 200$ GeV. This in part is achieved by a migration to PF-based online jet reconstruction with a reduced radius parameter $\Delta R = 0.4$, which provides improvements in jet energy resolution in high-pileup conditions over calorimeter-based reconstruction and mitigates the effects of pileup contamination within the jet cone.

The hadronic signal selection is performed with $\scalht$-$\alphat$ cross triggers with a second jet threshold requirement (\verb!HLT_PFDijetXXXHTYYYAlphaTZZZ!), which suppresses QCD multijet events whilst maintaining signal acceptance.  A loose calorimeter trigger prefilter is utilised to reduce the pass-through rate prior to track-based reconstruction, ensuring the PF-based filters meet the HLT mean timing limit of $\sim$10 ms. The calorimeter prefilter utlises loose $\scalht$ and dijet $\pt$ requirements in addition to a new variable $\alphat$', defined as $\alphat$ in the limit $\Delta\scalht \rightarrow 0$, which better correlates $\alphat$ between calorimeter and PF-based reconstruction.

Each of the signal triggers uniquely seed a single offline analysis bin with the exception of the highest-$\scalht$ trigger which is utilised for analysis bins above $\scalht > 400$ GeV. Analysis bins at very high-$\scalht$ are seeded by the \verb!HLT_PFHT900! trigger with no explicit dependence on $\alphat$ or second jet threshold. A list of Run 2 triggers for the hadronic signal region in the Run 2 HLT menu are shown in Table~\ref{tab:2015_Hadronic_Signal_Triggers}. The thresholds of the triggers will remain unchanged through run conditions and are measured to give high efficiencies in all three proposed run conditions: PU40bx50, PU20bx25 and PU40bx25. The Level-1 seeds for the HLT paths are given by the disjunction of the lowest unprescaled Level-1 hadronic scalar energy and missing energy sum seeds (\verb!HTTXXX_OR_ETMYYY!) for the run scenario.

Studies are currently underway to extend the trigger strategy to increase acceptance to monojet-like signatures of compressed spectrum and DM models with the transition from a pure dijet $\pt$ requirement to a dijet average $\pt$ trigger, enabling an increase in efficiency in the selection of events exhibiting asymmetric jet topologies.


% TABLE : 2015 triggers
%----------------------------------------------------------------------
\begin{table}[h!]
\footnotesize
\centering
\begin{tabular}{c|ccc|c} 
\hline
\hline
HLT path & L1 seed & HLT calo-prefilter                      & HLT PF-filter                          & Rate \\[0.7 ex] 
         &         & ($\scalht$, $\alphat$', $\pt^{\rm j2}$) & ($\scalht$, $\alphat$, $\pt^{\rm j2}$) & (Hz) \\[0.7 ex] 
\hline
\verb!HLT_PFDijet90HT200AlphaT0p57! & \verb!HTT175 OR ETM70! & 150, 0.540, 70 & 200, 0.570, 90 & 11.0 $\pm$ 3.0 \\
\verb!HLT_PFDijet90HT250AlphaT0p55! & \verb!HTT175 OR ETM70! & 200, 0.535, 70 & 250, 0.550, 90 & 8.5  $\pm$ 3.0 \\
\verb!HLT_PFDijet90HT300AlphaT0p53! & \verb!HTT175 OR ETM70! & 250, 0.525, 70 & 300, 0.530, 90 & 9.5  $\pm$ 3.0 \\
\verb!HLT_PFDijet90HT350AlphaT0p52! & \verb!HTT175 OR ETM70! & 300, 0.520, 70 & 350, 0.520, 90 & 10.0 $\pm$ 3.0 \\
\verb!HLT_PFDijet90HT400AlphaT0p51! & \verb!HTT175 OR ETM70! & 370, 0.510, 70 & 400, 0.510, 90 & 13.5 $\pm$ 3.5 \\
\hline
\multicolumn{4}{c|}{Exclusive rate (Hz)} & 34 $\pm$ 6 \\
\hline
\hline

\end{tabular}
\caption{Hadronic signal region HLT paths for the Run 2 PU40bx25 scenario, lower threshold Level-1 seeds, {\verb!HTT150 OR ETM60}, are utilised for the PU40bx50 and PU20bx50 scenarios. }
\label{tab:2015_Hadronic_Signal_Triggers}
\end{table}




%Prescaled control region triggers
\subsection{Control samples}
Prescaled $\scalht$-dijet triggers, each with an exclusive rate of $\sim$1 Hz, are utilised in selecting events for the hadronic control region. These share the same Level-1 seeds and $\scalht$ threshold of the signal triggers and also map to a unique offline bin. The non-hadronic control regions will be seeded by the lowest-threshold unprescaled triggers of the given run scenario, these include: \verb!HLT_Ele27_eta2p1_WP75_Gsf! seeding the electron control region ($e$, $ee$), \verb!HLT_IsoMu20_eta2p1! the muon control region ($\mu$, $\mu\mu$) and \verb!HLT_Photon175! the photon control region.



%% % TABLE : 2015 control triggers
%% %----------------------------------------------------------------------
%% \begin{table}[h!]
%% \footnotesize
%% \centering
%% \begin{tabular}{|c||c|c|c||c|} 
%% \hline
%% HLT path & L1 seed & Prescale \\[0.7 ex] 
%% \hline
%% \verb!HLT_PFDijet90HT200! & \verb!HTT175 OR ETM70! & ? \\
%% \verb!HLT_PFDijet90HT250! & \verb!HTT175 OR ETM70! & ? \\
%% \verb!HLT_PFDijet90HT300! & \verb!HTT175 OR ETM70! & ? \\
%% \verb!HLT_PFDijet90HT350! & \verb!HTT175 OR ETM70! & ? \\
%% \verb!HLT_PFDijet90HT400! & \verb!HTT175 OR ETM70! & ? \\
%% \hline

%% \end{tabular}
%% \caption{Hadronic control region HLT paths for the Run 2 PU40bx25 scenario. }
%% \label{tab:2015_Hadronic_Control_Triggers}
%% \end{table}



\subsection{Low-$\scalht$ Level-1 seed}

Some models of new physics, such as compressed spectrum models of supersymmetry, are difficult to select at the trigger level due to low-hadronic energy visible in the final state. Sensitivity to such models requires low trigger thresholds which is difficult to achieve without incurring a large increase in the Level-1 trigger rate. Several new Level-1 seeds which maintain low thresholds by actively vetoing QCD event topologies have been studied and have been shown to retain higher efficiencies for such models than can be achieved with the current Level-1 menu.

One such seed vetoes events exhibiting dijet topologies by vetoing events where the azimuthal separation of the leading jets exceeds eight calorimeter regions corresponding to a veto on events satisfiying: $\Delta\phi(j_{1}^{L1},j_{2}^{L1}) \ge 160^{\circ}$, where a jet threshold of $\pt > 30$ GeV is imposed on the second jet.

The $\mht/\scalht$ exploits the correlation in missing hadronic energy and visible hadronic energy typical of signal models and SM backgrounds with geniune $\met$ and the decorrelation of the quantities for QCD with fake-$\met$.

The logical conjunction of these two seeds can further improve signal efficiency whilst maintaining an effective suppression of QCD events, retaining ISR-dominated final states which typically have large $\mht/\scalht$ but may fail the $\Delta\phi(j_{1}^{L1},j_{2}^{L1})$ trigger and events with high jet multiplicities which have high $\Delta\phi(j_{1}^{L1},j_{2}^{L1})$ but may fail the $\mht/\scalht$ threshold.

Repurpose $\mht$ and $\phi({\mht})$ trigger bits with $\mht/\scalht$ and $\Delta\phi(j_{1}^{L1},j_{2}^{L1})$ respectively.


%%____________________________________________________________________________||
