%%____________________________________________________________________________||
\section{Changes in the methods from the previous analysis}
\label{sec:strategy}


This seciton needs to describe:

\begin{itemize}
 \item steps needed to produce a publishable result
\end{itemize}

Please somebody write this section!!

%% motivations moved from Event selection

\subsection{Changes from the Run~1 analysis}

%new alphaT
High \alphat cuts can have a significant effect on signal acceptance, so it is
advantageous to tune the cut this way to maintain a high signal to background
ratio, while remaining QCD free. The low \HT bins are most likely to be 
contaminated by QCD so the \alphat threshold 
is scaled up until the bin is QCD free. The value of \mht that this
approximately corresponds to is calculated and the \alphat cut for the
subsequent bins is chosen to keep this \mht approximately constant. This ensures
we remain QCD free while maximising signal acceptance.

%HT jet bins
In the SUS-14-006 analysis the maximum \HT bin was 1075. With the increase in
centre of mass energy in Run~2, the prevalance of higher \HT events will be
increased. To increase the sensitivity
of the analysis to high \HT signal events, the \HT bins can be extended to higher
values than in Run~1.

%Asym jet bin
As SUSY models with compressed spectra in general only produce soft SM
particles, they are dwarfed by background. It is possible to regain sensitivity
to these models by relying on their production in conjunction with ISR. These
sort of events typically have one high \PT jet and other lower \PT jets. It is
therefore proposed to add an extra analysis category in which events with the
second leading jet below the lead jet are included.

%njet bin
With the increase in events available from electroweak processes in Run~2, it
now seems possible introduce a bin per \njet category, rather than requiring
just $\njet \leq 3$ and $\njet \geq 4$. This aids in categorising low \njet signal
processes, in which an invisible particle system is made in conjunction with ISR.
%%____________________________________________________________________________||
