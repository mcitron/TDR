%%__________________________________________________________________||
\section{The CMS detector}
\label{sec:detector}

A more detailed description of the CMS detector can be found in
Ref.~\cite{ref:CMS}. The central feature of the CMS detector is a
superconducting solenoid providing an axial magnetic field of
3.8~T. The CMS detector is nearly hermetic, which allows for
momentum-balance measurements in the plane transverse to the beam
axis. The bore of the solenoid is instrumented with several particle
detection systems.

Charged particle trajectories are measured by a
silicon pixel and strip tracker system, with full azimuthal ($\phi$)
coverage and a pseudorapidity acceptance of $|\eta| < 2.5$. Here,
$\eta \equiv -\ln [ \tan (\theta/2) ]$ and $\theta$ is the polar angle
with respect to the counterclockwise beam direction. The resolutions
on the transverse momentum (\pt) and impact parameter of a charged
particle with $\pt < 40\gev$ are typically 1\% and 15\mum,
respectively.

A lead tungstate crystal electromagnetic calorimeter (ECAL) and a
brass/scintillator hadron calorimeter surround the tracking volume and
provide coverage over $|\eta| < 3.0$. The forward hadron calorimeter
extends symmetrically the coverage by an additional two units in
$\eta$.  The ECAL has an energy resolution of better than 0.5\% for
unconverted photons with transverse energies above 100\GeV.  In the
$\eta$-$\phi$ plane, and for $\abs{\eta}< 1.48$, the HCAL cells map
onto $5 \times 5$ arrays of ECAL crystals to form calorimeter towers
projecting radially outwards from close to the nominal interaction
point. At larger values of $\abs{ \eta }$, the size of the towers
increases and the matching ECAL arrays contain fewer crystals. Within
each tower, the energy deposits in ECAL and HCAL cells are summed to
define the calorimeter tower energies, subsequently used to provide
the energies and directions of hadronic jets. The HCAL, when combined
with the ECAL, measures jets with a resolution $\Delta E/E \approx
100\% / \sqrt{E\,[\GeVns]} \oplus 5\%$.

% https://twiki.cern.ch/twiki/bin/viewauth/CMS/Internal/PubDetector

%The central feature of the CMS apparatus is a superconducting solenoid
%of 6\unit{m} internal diameter, providing a magnetic field of
%3.8\unit{T}. Within the superconducting solenoid volume are a silicon
%pixel and strip tracker, a lead tungstate crystal electromagnetic
%calorimeter (ECAL), and a brass and scintillator hadron calorimeter
%(HCAL), each composed of a barrel and two endcap sections. Forward
%calorimeters extend the pseudorapidity~\cite{Chatrchyan:2008zzk}
%coverage provided by the barrel and endcap detectors. Muons are measured
%in gas-ionization detectors embedded in the steel flux-return yoke
%outside the solenoid.

%%__________________________________________________________________||
