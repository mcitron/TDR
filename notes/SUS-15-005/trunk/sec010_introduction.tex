%%__________________________________________________________________||
\section{Introduction}
\label{sec:introduction}

The standard model (SM) of particle physics is widely considered to be
only an effective approximation of a more complete theory, such as
supersymmetry (SUSY)~\cite{ref:SUSY-1,ref:SUSY0,ref:SUSY1,ref:SUSY2,ref:SUSY3,ref:SUSY4,ref:hierarchy1,ref:hierarchy2},
that would supersede the SM at a higher energy scale. For R-parity conserving
SUSY~\cite{Farrar:1978xj}, supersymmetric particles (sparticles) such
as squarks and gluinos are produced in pairs and decay to the
lightest, stable supersymmetric particle (LSP), which is generally
assumed to be a weakly interacting and massive neutralino, the $\chiz_1$. 
A characteristic signature is a final state of multijets accompanied by
significant missing transverse momentum, $\ptvecmiss$.

This document presents an inclusive search for the pair production of
massive coloured sparticles in hadronic final states with two or more
energetic jets and \met, performed in pp collisions at a
centre-of-mass energy $\sqrt{s} = 13\TeV$. The analysed data sample
corresponds to an integrated luminosity of $2.1 \pm 0.3
\fbinv$~\cite{lumi} collected by the Compact Muon Solenoid (CMS)
experiment. Previous iterations of this search have been performed in
pp collisions at both $\sqrt{s} = 7$~\cite{RA1Paper, RA1Paper2011,
  RA1Paper2011FULL} and $8\TeV$~\cite{RA1Paper2012}.


The search is devised around the kinematic variable
\alphat~\cite{Randall:2008rw, RA1Paper} that provides powerful
discrimination against multijet production, a manifestation of quantum
chromodynamics (QCD), and adheres to an inclusive strategy with the
aim of providing sensitivity to the widest possible range of SUSY
models. The \alphat variable is constructed from jet-based quantities
to provide robust discriminating power between sources of genuine and
misreconstructed missing transverse momentum, making it suitable for early
searches. A further variable that exploits azimuthal angular
information, known as $\Delta\phi^{*}_{\rm min}$~\cite{RA1Paper}, is
also employed to suppress QCD multijet production, including potential
contributions from semileptonic heavy-flavour decays, to a negligible
level.

The search is based on an examination of the number of reconstructed
jets per event, the scalar (\scalht) and vector (\mht) sums of
transverse momenta of these jets, and the number of these jets
identified as originating from bottom quarks. The search exploits the
use of \mht templates derived from simulation, which are extensively
validated in data control regions. These discriminating variables
provide sensitivity to the different production mechanisms of massive
coloured sparticles at hadron colliders (\ie squark-squark,
squark-gluino, and gluino-gluino), a large range of mass splittings
between the parent sparticle and the LSP, and third-generation squark
signatures, respectively. Interpretations of the analysis are provided
in the parameter space of simplified models~\cite{Alwall:2008ag,
  Alwall:2008va, sms}. Limits are derived in the mass parameter space
of simplified models of pair-produced gluinos, decaying inclusively to
four quarks and additionally decaying to four top and bottom
quarks. Figure~\ref{fig:feyn} illustrates the simplified models
considered.

\begin{figure*}[thb]
\centering
\includegraphics[width=0.32\linewidth]{T1bbbb.pdf}
\includegraphics[width=0.32\linewidth]{T1tttt.pdf}
\includegraphics[width=0.32\linewidth]{T1qqqq.pdf}
\caption{
Feynman diagrams for the simplified models.
}
\label{fig:feyn}
\end{figure*}


%%__________________________________________________________________||
