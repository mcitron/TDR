%%__________________________________________________________________||
\section{Multijet backgrounds}
\label{sec:qcd_background}

The signal region is defined so that the expected contribution from
multijet events is suppressed to the percent level with respect to
the total expected background counts from other SM processes for all
event categories and \scalht bins. This is achieved through very tight
requirements on the variables \alphat and \dphi, as described
above. 
%With this level of suppression, any potential contamination is
%assumed to be fully absorbed by the systematic uncertainties
%associated with the estimation of the non-multijet backgrounds,
%described below.
%The necessary requirements on \alphat are determined by the following
%method that relies on data control samples.
%
%The method discussed here differs with respect to that used in the
%previous search~\cite{RA1Paper2012}. 
Potential contamination from multijet events due to 
%instrumental effects 
limitations in reconstruction and analysis acceptance is estimated 
by exploiting the ratio of multijet events 
that satisfy or fail the requirement $\mhtmet < 1.25$. This ratio is taken from simulation and used with the yield of events in the $\mhtmet > 1.25$ sideband region to predict the background.

%The dependence of the ratio on \alphat is modelled by a falling
%exponential function with a fit to data performed within the sideband 
%$0.5 < \alphat < 0.55$ for each signal region bin. An estimate for the multijet
%contamination is then determined as a function of the threshold
%$\alphat^{\rm cut}$ from the product of the extrapolated value of the
%ratio and the observed event yields in a data sideband defined by
%$\alphat > \alphat^{\rm cut}$ and $\mhtmet > 1.25$. The value of
%$\alphat^{\rm cut}$ required to suppress the multijet contribution to
%the total SM background count to the percent level is determined
%individually for each \scalht bin.
% in the region $200 < \scalht < 475
%\gev$. 
%The corresponding \alphat thresholds are summarised in
%Table~\ref{tab:thresholds} and are identical for all \njet and \nb
%categories. Higher thresholds on \alphat than
%reported in Ref.~\cite{RA1Paper2012} are required in order to
%guarantee a negligible contamination from this type of events in the 
%new signal region bins at low \scalht and under higher pileup conditions.
%
%Various checks on the level of closure in both simulation and data are
%performed and the exponential function models accurately the behaviour
%in both simulation and data over all bins containing a sufficient
%population of multijet events. Systematic uncertainties are derived to
%encapsulate alternative fit functions and these uncertainties are
%considered in the determination of the \alphat thresholds. 

%Following the \alphat requirements determined by the procedure
%described above, 
%any residual contamination from energetic 
%multijet events that yield significant \met from neutrinos produced in
%semileptonic heavy-flavour decays are efficiently suppressed by the
%requirement $\dphi > 0.3$. This is validated in both simulation and
%data with events satisfying $\scalht > 775 \gev$ and either of the sideband
%requirements $0.51 < \alphat < 0.55$ or $\mhtmet > 1.25$. 
%These events are selected with an unprescaled \scalht trigger which
%allows to study the performance of the selection requirements in the low \alphat
%region around 0.51, corresponding to similar \mht values as employed 
%in the lowest \scalht bins used in the analysis.
%From these studies the remaining multijet background is found to be 
%compatible with zero.

%%__________________________________________________________________||
