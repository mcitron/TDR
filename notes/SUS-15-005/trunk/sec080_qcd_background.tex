%%__________________________________________________________________||
\section{Multijet backgrounds}
\label{sec:qcd_background}

The signal region is defined so that the expected contribution from
multijet events is suppressed to the percent level with respect to
the total expected background counts from other SM processes for all
event categories and \scalht bins. This is achieved through very tight
requirements on the variables \alphat and \dphi, as described
above. 

Potential contamination from multijet events in the signal region is
estimated by exploiting the ratio of multijet events that satisfy or
fail the requirement $\mhtmet < 1.25$. This ratio is taken from
simulated events, which is validated in multijet-enriched data
sidebands. The ratio is then used with the yield of events in the
$\mhtmet > 1.25$ sideband region, corrected to account for
contamination from vector boson and \ttbar production (plus residual
contributions from other SM processes), to predict the background in
the signal region, which is found to be small compared to other
backgrounds. In addition, data control variables are inspected to
provide confidence that multijet contamination due to instrumental
effects is negligible.

%%__________________________________________________________________||
