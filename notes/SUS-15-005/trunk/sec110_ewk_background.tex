%%__________________________________________________________________||
\section{Non-multijet backgrounds}
\label{sec:ewk_background}

In the absence of multijet events, the background counts in the signal
region arise from SM processes with significant \met in the final
state. In events with low counts of jets and b-quark jets, the largest
backgrounds with genuine \met are from the associated production of W
or Z bosons with jets, followed by either the weak decays \znunu or
\wtaunu, where the $\tau$ decays hadronically and is identified as a
jet; or by leptonic decays that are not rejected by the dedicated
electron or muon vetoes. The veto of events containing isolated tracks
is efficient at further suppressing these backgrounds as well as the
single-prong hadronic decay of the tau lepton. At higher jet and
b-quark jet multiplicities, top quark production followed by
semileptonic weak top quark decay becomes important. 

The production of W and Z bosons in association with jets, \ttbar and \gj processes are simulated
with the \MADGRAPH V5~\cite{madgraph} event generator. The production
of single-top quark events is generated with
\POWHEG~\cite{powheg}, and diboson events are produced with
\PYTHIA8.1~\cite{pythia8}. For all simulated samples, \PYTHIA8.1 is
used to describe parton showering and hadronisation. All samples are
generated using the \textsc{cteq6l1}~\cite{Pumplin:2002vw} parton
distribution functions (PDF). The description of the detector response
is implemented using the \GEANTfour~\cite{geant} package. The
simulated samples are normalised using the most accurate cross section
calculations currently available, usually with
next-to-next-to-leading-order (NNLO) accuracy. To model the effects of
pileup, the simulated events are generated with a nominal distribution
of pp interactions per bunch crossing and then reweighted to match the
pileup distribution as measured in data.

The method to estimate the non-multijet backgrounds in the signal
region relies on the use of transfer factors (TF), which are
constructed per bin (in terms of \njet, \nb, and \scalht) per data
control sample. The TFs are determined from the simulated event
samples and are ratios of expected yields in the corresponding bins of
the signal region and control samples. The TFs are used to extrapolate
from the event yields measured in a data control samples to an
expectation for the total background event yields in the signal
region. 

Three independent estimates of the irreducible background of \znunu +
jets events are determined from the \gj, \mmj, amd \mj data control
samples.  The \gj and \zmumu + jets processes have similar kinematic
properties when the photon or muons are ignored~\cite{Bern:2011pa}
albeit different acceptances. In addition, the \gj process has a
larger production cross section than \znunu + jets events. The \mj
data sample is also used to provide an estimate for the \znunu\ + jets
contribution as well as the other dominant SM processes, \ttbar and W
boson production. Residual contributions from processes such as
single-top-quark, diboson, and Drell-Yan production are also included.

The uncertainty in the transfer factors derived from simulation is
probed through closure tests based on data control
samples~\cite{RA1Paper2012}. Each closure test inspects the
compatibility of yields in two disjoint control samples and a
corresponding TF derived from simulation. A large ensemble of tests
are performed to probe key ingredients of the simulation modelling
that may introduce a source of mismodelling in the transfer
factors. The closure tests reveal no significant biases or
dependencies on \njet or \scalht for all individual tests. Systematic
uncertainties in the transfer factors are determined from the
statistically-weighted variance in the level of closure for
collections of tests. Templates derived from simulation are used to
predict the background counts in the \mht dimension. Multiple data
control regions are used to evaluate the degree to which the
simulation describes the \mht distributions observed in data and to
assign appropriate systematic uncertainties.

%%__________________________________________________________________||
