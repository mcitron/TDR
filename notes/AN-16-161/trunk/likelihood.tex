%%____________________________________________________________________________||
\section{Likelihood model}
\label{sec:likelihood}

Consider a given category of event as defined by \njet, \nb~and \scalht, which are in the following identified with \htcat. 
In each category, the signal is extracted using the discriminating variable \mht. 
Histogram templates of the \mht distribution are built for the signal and the background processes 
using the MC samples described in Sec.~\ref{sec:datasets}. \\
The binning of the templates is chosen taking into account both the limited statistics in the simulation and 
the control of the background in data. 
A maximum MC statistical uncertainty of 50\% (corresponding to 4 unweighted events) is required in each bin of the MC histogram template, 
in order to ensure a statistically meaningful prediction. The uncertainty on the MC statistics in each bin is anyway 
taken into account with a dedicated nuisance parameter, one per template bin, as explained below. 
A minimum bin width constraint of 50 GeV is applied, 
in order to reduce the bin-by-bin migration due to the finite \mht resolution.\\

For each category \htcat and \mht bin, $i$, in the signal region, let $n^{\htcat}_{\mathrm{had},i}$ be the number of observed events, 
$b^{\htcat}_{\mathrm{had},i}$ the number of predicted background events and $s^{\htcat}_{\mathrm{had},i}$ the expected number of signal events. \\
The likelihood function for the hadronic signal region is, in each \htcat:

\begin{equation}
\mathcal{L}^{\htcat}_{\mathrm{had}}=\prod_i \mathrm{Poisson}(n^{\htcat}_{\mathrm{had},i} |\, b^{\htcat}_{\mathrm{had},i} + s^{\htcat}_{\mathrm{had},i})
\label{eq:hadronicLikelihood}
\end{equation}

The \mht dimension is not used for the control region, and their information is only used to constrain the normalisation of the background processes 
in the signal region, as described in Sec.~\ref{sec:backgroundmet}. 
Their likelihood is therefore written as:

\begin{equation}
\mathcal{L}^{\htcat}_{\mathrm{CR,j}}=\mathrm{Pois}(n^{\htcat}_{\mathrm{CR,j}} |\, b^{\htcat}_{\mathrm{CR,j}} + s^{\htcat}_{\mathrm{CR,j}})
\label{eq:controlLikelihood}
\end{equation}

In Eq.~\ref{eq:controlLikelihood}, $n^{\htcat}_{\mathrm{CR,j}}$ is the number of observed events, $b^{\htcat}_{\mathrm{CR,j}}$ the number of predicted 
background events and $s^{\htcat}_{\mathrm{CR,j}}$ the expected number of signal events in the control region $j$. \\
Eq.~\ref{eq:controlLikelihood} applies to all the control region used for the background estimation, 
as defined in Sec.~\ref{sec:selection}. \\
Notice that the signal contribution $s^{\htcat}_{\mathrm{CR,j}}$ in Eq.~\ref{eq:controlLikelihood}, as estimated from MC, is in general negligible. 
Where the signal contamination is sizeable, it is included in the likelihood as an additional process contributing to the event yield in that particular bin.

The prediction of the background yields in the signal region and in the corresponding control samples are connected 
by means of transfer factors, as explained in Sec.~\ref{sec:backgroundmet}. 
This connection is implemented by introducing a floating parameter, which is correlated 
between the signal region and the control regions. 
One floating parameter is used for both background 
processes ($Z_{\mathrm{inv.}}$ and \ttbar/W) \footnote{In the following we use $Z_{\mathrm{inv}}$ to indicate 
the $Z\to \mathrm{inv}$ process and \ttbar/W to indicate the sum of the yields of the $t\bar{t}$ and $W+\mathrm{jets}$ processes.}
giving a prediction for each background sources depending on the relevant nuisances described below. \\
This binds the background yields in the signal and control regions to float together, 
taking into account the statistical uncertainty associated with the counts in the control samples. 

The systematic uncertainties affecting the transfer factors, described in Sec.~\ref{sec:systematics}, 
are incorporated in the likelihood by means of guassian nuisance parameters, which act on the floating
parameter. These depend on the background process and the control region used for the prediction and
are taken as uncorrelated between each of the \htcat bins. Additionally, systematic effects studied 
through variations in simulation, like b-tag SF and jet energy corrections, are included as shape uncertainties on the transfer factors. 
The are taken as fully correlated across all bins but have different sizes depending on the bin. The effects 
of these systematics are determined using the method described in Sec~\ref{sec:systematics}. \\
The uncertainties on the signal efficiency times acceptance, described in Sec.~\ref{sec:susy}, 
are also taken as shape uncertainties correlated across all the \htcat bins. 
The uncertainty that encapsulates the potential for bin migration within the \mht distribution of events is implemented 
providing alternative templates corresponding to up/down variation of each source, separately for each \htcat bin. 
The procedure to assess the alternative templates is described in detail in Sec.~\ref{sec:syst-on-shape}. \\
The uncertainty due to the limited statistics in the MC samples used to populate the template histograms is incorporated 
as additional nuisance parameters, one per template bin, taken as uncorrelated across the histogram bins. 

The total likelihood is the product over all the \htcat bins and all the control regions, and can be written as:

\begin{equation}
\label{eq:total_likelihood}
\mathcal{L} = \prod_{\htcat} \mathcal{L}_{\text{had}}^{\htcat} \times \prod_{\text{j}} \mathcal{L}_{\text{CR,j}}^{\htcat}
\end{equation}

The fit is carried on in two steps. 
In the first, the background prediction for the signal region $b^{\htcat}_{\mathrm{had},i}$ are extracted from a likelihood fit 
using the control regions only in Eq.~\ref{eq:controlLikelihood}. 
This fit provides the best knowledge of the background yields in the signal region, 
and they can therefore be used to re-interpret the analysis. 
In the second step, the fit is done using the full likelihood (Eq.~\ref{eq:total_likelihood}) and all 
the correlations between the backgrounds in control regions and signal region are taken into account, 
together with the statistical uncertainty associated to the finite number of events in the control region. \\
The likelihood is profiled against all nuisance parameters in order to derive expected exclusion limits and sensitivity, 
which will be discussed in Sec.\ref{sec:susy}, \ref{sec:darkmatter}.

For technical limitations, only a subset of jet multiplicity categories are combined to derive the exclusion limits. 
Four out of nine \nj categories are included, while considering all the \nb, \scalht, \mht bins within each \nj category. 
These four categories are chosen as the ones providing the best expected exclusion for each signal point separately. \\
%% In App.~\ref{app:jetRanking}, the details of this ranking are given for all the models considered in the analysis. 
%% The four categories used in the limits are also shown in tabular format for some benchmark models in Tab.~\ref{tab:sig-eff-bestCat}.




%%____________________________________________________________________________||
