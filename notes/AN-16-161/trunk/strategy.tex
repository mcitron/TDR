%%____________________________________________________________________________||
\section{Analysis strategy}
\label{sec:strategy}

\subsection{The analysis during Run~1}

While the analysis strategy has evolved significantly since the
beginning of Run~1, certain design features have remained central to
the RA1 analysis throughout Run~1 and will continue to do so during
Run~2. These features, along with some of the more important recent
developments, are outlined below.

{\bf Model-independent search for new physics:}

The analysis is designed to be sensitive to a range of beyond standard
model (BSM) physics scenarios with signatures that are characterised
by jets and significant \met in the final state. The search is one of
the original Reference Analyses (RAs) that formed the basis of the CMS
SUSY search program at the beginning of Run~1. The program was
designed to be inclusive, model-independent (as much as possible), and
topology-based. 

The current incarnation of the RA1 search still retains its generic,
inclusive nature. A binned signal region, defined by a set of
inclusive selection criteria, is employed to provide sensitivity to a
range of new physics models. By the end of Run~1, three discriminating
variables were used to categorise signal event candidates: \scalht,
\njet, and \nb. In the case of SUSY, this choice of variables provides
sensitivity to the direct pair-production of all flavours of squarks
(including the third generation), as well as pair-produced gluinos and
their decays via all favours of quarks. The large dynamic range in
\scalht provides sensitivity to a large range of mass splittings
between the parent sparticle and the LSP.

{\bf Robust performance under Start-Up conditions:}

The search relies on the \alphat variable, which strongly suppresses
the dominant background of multijet production from QCD, while
maintaining acceptance to new physics signatures with genuine \met in
the final state. At the beginning of Run~1, the \alphat variable was
chosen because it was shown to be extremely effective at operating
under Start-Up conditions, when the beam conditions and detector
performance were poorly understood. For example, multijet
events with significant \met from jet mismeasurements due to
instrumental effects are penalised to lower values of \alphat, hence
the variable exhibits an intrinsic robustness against potential
sources of ``fake'' \met.

% Further, the analysis did not use directly the \met variables and
% relied instead solely on calorimeter-based jets (with relatively
% high thresholds of \Pt > 50\gev) to

While the Collaboration now has a good understanding of the detector
performance under Run~1 conditions, the strategy we employ is still
 very relevant when exploring the new high-energy, high-luminosity
frontier in Run~2.

{\bf Inclusive trigger strategy with low thresholds:}

The trigger strategy has made use of the \alphat variable at the HLT
since 2011. Given the rejection power of \alphat for multijet events,
we employ this variable (crossed with \scalht) at the trigger level,
which allows to push to lower thresholds for a given trigger rate with
respect to other discriminating variables typically used in jets +
\met searches (\eg \met, \mht, etc). A suite of \scalht-\alphat cross
triggers allowed the RA1 search to push as low as $\scalht > 200\gev$
and (effectively) $\mht \gtrsim 100\gev$ at the trigger, and as low as
$\scalht > 200\gev$ and (effectively) $\mht \gtrsim 130\gev$ in the
signal region definition, with near-maximum signal efficiency at the
cost of $\sim25$~Hz (of which $\sim$15 was ``parked''). 

{\bf Multijet-free signal region.}

As the analysis has evolved, the event selection has been tailored to
provide a multijet-free sample of signal event candidates. This is
achieved by applying a \scalht-dependent \alphat threshold, tuned to
suppress the multijet contribution to the sub-percent level with
respect to the total background counts per signal region bin. This is
extremely important for the lowest \scalht bins, which provide a
significant contribution to the overall sensitivity to compressed SUSY
or Dark Matter models. Additional studies of the ``parked data'' led
to the addition of another powerful variable to further discriminate
against heavy flavour decays in QCD multijet events, known as \bdphi,
which inspects the azimuthal separation of jets and \met.

{\bf Access to new physics via initial state radiation:}

The analysis acceptance to compressed SUSY models relies not on
the detection of very soft objects (often failing acceptance cuts)
produced in the decay of the sparticles, but instead on the presence
of jets from initial state radiation. The signal acceptance times
efficiency for this mechanism is typically at the sub-percent level.
Further, the signal is characterised by a very broad shape in \scalht,
with significant counts at low \scalht, as well low values of \njet,
\nb, and \mht. Hence, the highest signal counts are typically found in
the bins with the highest background counts, typically populated by
events from W + jets and Z$\ra\nu\nu$ + jets, which results in a poor
S/B. Hence, the signal shape over many bins has to be considered in
order to achieve reasonable sensitivity to such signatures. One
distinguishing feature with respect to the SM backgrounds, which is
most pronounced for (near-)degenerate spectra, is that while the
values of \scalht and \mht are typically small, the ratio $\mht /
\scalht$ is large with respect to SM backgrounds.

This mechanism of signal acceptance was the focus of the ``parked
analysis'' in Ref~\cite{CMS_AN_2013-366}. While the interpretations of
this result focused on strongly-produced compressed SUSY models, the
same mechanism can be used to access the direct production of a Dark
Matter candidate. Sensitivity to this
class of models relies heavily on tight background control and low
thresholds. This is the motivation for the strategies of developing
high-performance triggers with low thresholds, tuning the event
selection to provide a multijet-free signal region, and relying on a
large ensemble of closure tests in data to demonstrate excellent
control of the non-multijet backgrounds in the signal region.

{\bf Data-driven background estimation methods and closure tests:}

The analysis relies on multiple data control samples to predict the
non-multijet backgrounds. The control samples have a composition
similar to the signal region, cover a similar kinematic phase
space, have a large acceptance, and are signal-depleted. The analysis
relies on transfer factors constructed from MC to extrapolate from the
control regions to the signal region. The extrapolation is minimised
by using identical binning in all samples. Potential residual biases
in the transfer factors, \eg from MC mismodelling, are probed though a
large ensemble of closure tests performed between numerous data
control sub-samples. Several hundred statistically independent tests
are performed, which cover a wide range of potential biases. 

These closure tests are one of the most crucial components of this
analysis, as they provide an understanding of the reliance on MC (in
the form of ratios, \ie transfer factors) that is {\it founded on
  comparison with data}. Any potential biases or dependencies on the
discriminating variables used to categorise the signal events must be
understood prior to unblinding the signal region. There are concrete
examples in the past when the closure tests have revealed problems,
which had to be understood and solved prior to continuing with the
analysis.

In the absence of any significant biases, the closure tests are then
used to determine systematic uncertainties on the transfer factors,
which are propagated through to the analysis result via the likelihood
model. These systematic uncertainties are therefore motivated through
comparisons with control data and are, ultimately, statistically
limited by the data counts in the control samples. With increasing
integrated luminosity, we expect the closure tests to be increasingly
statistically-significant probes of potential biases and, in the
absence of bias, the systematic uncertainties will correspondingly
reduce.

\subsection{Developments for Run~2}
\label{sec:changes}

Building on the experience gained during Run~1, we developed the
analysis to provide improved sensitivity to a range of new physics
models. These developments fall broadly under three categories:

\begin{itemize}
\item general optimisations concerning trigger, event reconstruction and
  selection;
\item improvements concerning simplified Dark Matter models and compressed SUSY
  models. 
\item improvements in the sensitivity to heavy objects, such as gluinos 
\end{itemize}

Some of these changes have already been studied and/or implemented
during supplementary studies to the Run~1 ``parked analysis'', as
described in Ref.~\cite{CMS_AN_2013-366}. Others, like an extensive addition of new
jet categories are presented for the first time. These additions constitute a significant effort and aim to achieve very good 
sensitivity for DM signatures recommended by the ATLAS-CMS DM Forum~\cite{Abercrombie:2015wmb}.
The status will be noted at appropriate points in this document. A summary of these changes are
listed below. This list is not exhaustive and some developments are
still subject to (minor) change.

{\bf Particle Flow event reconstruction:}

The analysis switched from relying on jets reconstructed solely
from calorimeter information to the use of jets from the Particle Flow
(PF) reconstruction algorithm. This change provides performance
improvements for jet-based variables (including \scalht, \mht, and
\alphat) and, in particular, better mitigate any effects relating to
pileup.

{\bf General event selection optimisation:}

In addition to the switch to PF, all event selection criteria for the
signal region and control samples have been reviewed. The jet
\Pt threshold is no longer scaled at low \scalht as during Run~1, and
the binning in \scalht has been simplified. Also binning choice in the
signal region had evolved to consider the anticipated integrated luminosity and 
data taking conditions.  We also plan to investigate developments concerning 
\eg boosted objects.

{\bf Trigger strategy:}

The trigger strategy is similar to that of Run~1, and continues
to rely on a suite of \scalht-\alphat cross triggers at the HLT. The
trigger decision is now based on PF jets, rather than
calorimeter-based jets as during Run~1. The \HT and \alphat thresholds
used at the end of Run~1 have been maintained, which fall in the range
of $200 < \scalht < 400\gev$ and $0.57 > \alphat > 0.51$. The \Pt
requirements on the lead two jets have been raised, in order to help
control rate, while maintaining acceptance to events with soft second
jets. 

The highest \scalht bins in the analysis is seeded by the lowest
unprescaled \scalht trigger, which currently has a threshold of
$\scalht > 800\gev$.

Single object muon, electron, photon, and (prescaled) \HT triggers
are used to seed the control samples. 

There have also been developments of novel L1 seeds to improve
efficiencies at low \scalht. The performance of the
L1 cross triggers based on these seeds provides very high efficiency 
down to $\scalht > 200\gev$.

{\bf \scalht-dependent \alphat thresholds:}

During Run-1, the \alphat thresholds were raised as high as $\alphat >
0.65$ for the lowest \scalht bins to suppress potential contamination
from multijet events to the sub-percent level with respect to the
total background counts. A constant threshold of $\alphat > 0.55$ was
used for all bins satisfying $\scalht > 325\gev$. However, \alphat is
a relative quantity that relates \scalht and \mht (and \dht) such that
there is an effective lower bound on \mht that increases linearly with
\scalht. In order to maximise signal acceptance, while maintaining at
least the same level of multijet rejection as in the lowest \scalht
bins, the \alphat threshold is lowered as a function of \scalht
in line with the thresholds for the suite of HLT triggers. This
approach allows an \alphat threshold as low as 0.52 for the
region $\scalht > 400\gev$, which corresponds to a lower bound on \mht
of $\sim110\gev$. This (approximate) requirement on \mht is
(indirectly) imposed on all bins in the low \scalht region through an
appropriate choice of \alphat threshold.

This dependence on \scalht of the \alphat thresholds allows improve acceptance, 
particularly to ``jetty'' signatures which are a
characteristic of gluino pair-production and decay. The approach
allows to maintain an approximately constant \mht requirement of
$\sim130\gev$ through the full range of $\scalht > 200\gev$.

For the region $\scalht > 800\gev$, events are collected with the
lowest unprescaled \scalht trigger in the HLT menu. As there is no
\alphat requirement in the trigger, we choose to drop the \alphat
requirement in the event selection and instead employ only a
requirement on the variable \bdphi to control any potential multijet
contamination, in conjunction with a loose baseline requirement of
$\mht > 130\gev$ (which corresponds to an \alphat threshold of
$\sim0.507$). 
While the use of \alphat is necessary to push to low
values of \scalht and \mht, we opt for only the \bdphi and \mht
requirements at high \scalht, as this can improve the signal
efficiency for some models while still
controlling multijet contamination. 
More studies will be carried on in the future to optimise the cuts in 
this region of the phase space.

{\bf Adding the \mht dimension:}

During Run~1, events in the signal region were categorised according
to three discriminating variables: \scalht, \njet, and \nb. While
\scalht has historically been used in many searches for new physics at
the LHC as an estimator for the mass scale of new physics, another
characteristic signature for (R-parity conserving) SUSY and other
models involving the direct production of a Dark Matter candidate is
the presence of significant \met.

Early in the \texttt{PHYS14} exercise, we explored the options for an
additional discriminating variable that improves the analysis
sensitivity to both the strong production of heavy objects such as
gluinos and squarks as well as compressed SUSY models and Dark Matter
candidates. The variable \mht was chosen, given its generally good
performance with respect to the other variables that were
investigated, plus this variable fits naturally with the current
``robust'' analysis strategy of relying solely on jets as input
(rather than \eg \met). Events are binned according to this
variable, which supersedes the ``effective'' single bin approach
employed in Run~1 via the \alphat requirement that, depending on the
\HT scale, translated into an ``effective'' minimum \mht requirement
of $\sim$130--500\gev.

{\bf Additional \njet categories:}

We adopted a more granular jet multiplicity binning in Run~2, in
which events are categorised according to whether they contain
exactly one, two, three, four, or at least five jets. This compares with
the two categories of $\njet = 2-3$ and $\njet \geq 4$ used during
Run~1. Events are also categorised according to the number of jets
that are tagged as originating from b-quarks, with up to as many as
four b-tags (as during Run~1). This improved granularity in the number of
jets helps to improve the sensitivity to a wide range of models,
including compressed SUSY scenarios, collider production of Dark
Matter, and gluino-mediated models.

{\bf ``Asymmetric'' jet \Pt thresholds:}

The nominal event selection criteria for the signal region include an increased
 threshold of $100\gev$ for the two hardest jets in the
event. This raised threshold is aimed at reducing the V + jets
backgrounds while maintaining acceptance to SUSY models due to the
assumed pair-production of sparticles. (The decay of each
sparticle is assumed to result in at least one hard jet.) 

However, for compressed SUSY and Dark Matter models, signal acceptance
is largely due to the presence of jets from initial state
radiation. Under this scenario, it is beneficial to relax the
requirements on the second leading jet in the event in order to
maximise acceptance. Hence, in addition to each ``nominal'' \njet
category, there is now a new orthogonal ``asymmetric'' \njet category
that still requires a hard leading jet but also requires a soft second
jet. The ``asymmetric'' dijet category is therefore comparable to the
dijet bin of the monojet analysis~\cite{Chatrchyan:2011nd,
Chatrchyan:2012me, Khachatryan:2014rra}. Given that a minimum of two
jets are still required, the method of using \alphat to suppress
multijet contamination is still applicable.

{\bf ``Monojet'' selection:}

In order to achieve a complete coverage of the jet multiplicity phase space,
and further improve acceptance to low WIMP and mediator masses
in simplified SUSY and DM models, we also lowered the jet multiplicity 
requirement of the analysis to include the ``monojet'' category, in which
exactly one jet above the $100\gev$ is required. In this case, a requirement on
$\bdphi$ alone, and not \alphat, is sufficient to suppress the multijet
background to a negligible level.


{\bf Developments concerning closure tests:}

Two new closure test have been developed to test the estimations of the $Z \to \nu \nu$
background (single-muon predicting double-muon) and the $W$ charge asymmetry (positive muon predicting negative muon). 

Moreover, the procedure to derive uncertainties on the transfer factor has been slightly revised, 
with each test being considered separately and addressing one source of uncertainty. 
The tests are still performed in bins of \nj and \scalht, 
in order to spot trends and biases as a function of this variables. 
To derive the systematics uncertainty that is propagated to the analysis, however, 
tests are collapsed into ``asymmetric'' and ``symmetric'' topologies. 
More details are given in Sec.~\ref{sec:systematics}


\subsection{Developments for the 2016 run \label{sec:devs_2016}}
During the preparation for the restart of the data taking in spring 2016 
several developments to the 2015 analysis have been considered and studied in some detail. 
However, as the aim of the early analysis described in this document is to deliver 
robust results in a short timescale, it has been decided to postpone most of these key enhancements 
for the analysis of the full dataset, foreseen for the end of 2016. 
Moreover, the beneficial effects of new features in the analysis will be further enhanced 
with the larger dataset that will be available by then. 

Therefore, only few changes are made with respect to the analysis carried on during 2015, namely:
\begin{itemize}
\item a more inclusive strategy for the signal triggers has been employed, allowing to recover most of the inefficiency 
of the pure \scalht trigger path used in the region $\scalht > 800 \,\mathrm{GeV}$. More details are given in Sec.~\ref{sec:hadronic_signal_region};
\item while the method remains unchanged, the yields of the QCD multi-jet background are now extracted from a maximum-likelihood fit, as documented in Sec.~\ref{sec:qcd}. 
This allows to correctly propagate the uncertainties in the non-multijet component in the multi-jet sideband with its correlations. 
\end{itemize}

%%____________________________________________________________________________||
