%%____________________________________________________________________________||

\section{Background estimation for processes with genuine \met}
\label{sec:backgroundmet}
\subsection{Overview of SM background processes}

Once all the signal region selection requirements have been imposed,
the contribution from QCD multijet events is expected to be
negligible, as demonstrated in Appendix~\ref{sec:kisigplot}. In the absence of
multijet events, the background counts in the signal region arise from
SM processes with significant \met in the final state. In events with
low statistics of jets and b-quark jets, the largest backgrounds with
genuine \met are from the associated production of W or Z bosons with
jets, followed by either the weak decays \znunu or \wtaunu, where the
$\tau$ decays hadronically and is identified as a jet; or by leptonic
decays that are not rejected by the dedicated electron or muon
vetoes. The veto of events containing isolated tracks is efficient at
further suppressing these backgrounds as well as the single-prong
hadronic decay of the tau lepton. At higher jet and b-quark jet
multiplicities, top quark production followed by semileptonic weak top
quark decay becomes important.  Residual contributions from processes
such as single-top-quark, $\ttbar$V or $\ttbar$H, diboson, and
Drell-Yan production are also expected. These SM processes are
collectively referred to as the non-multijet backgrounds.

%The production of W and Z bosons in association with jets is simulated
%with the \MADGRAPH V5~\cite{madgraph} event generator. The production
%of \ttbar and single-top quark events is generated with
%\POWHEG~\cite{powheg}, and diboson events are produced with
%\PYTHIA6.4~\cite{pythia}. For all simulated samples, \PYTHIA6.4 is
%used to describe parton showering and hadronisation. All samples are
%generated using the \textsc{cteq6l1}~\cite{Pumplin:2002vw} parton
%distribution functions (PDF). The description of the detector response
%is implemented using the \GEANTfour~\cite{geant} package. 
The simulated samples are normalised using the most accurate cross
section calculations currently available, usually with
next-to-next-to-leading-order (NNLO) accuracy. To model the effects of
pileup, the simulated events are generated with a nominal distribution
of pp interactions per bunch crossing, which are then reweighted
to match the pileup distribution as measured in data. 

\subsection{The ``transfer factor'' method}
\label{sec:ewk-method}

The method used to estimate the aforementioned SM background
contributions in the hadronic signal region relies on the use of a
transfer factor (TF) determined from MC samples to transform the
observed yield in a given \scalht, jet (\njet) and b-tag (\nb)
multiplicity bin of a control sample, $\nobs^{\rm
  control}(\njet,\nb,\scalht)$, into a predicted yield for the
corresponding bin of the hadronic signal region, $\npre^{\rm
  signal}(\njet,\nb,\scalht)$. The choice of \njet and \nb~event
categorisation and \scalht binning in the control samples is identical
to that for the signal region, as defined in
Sec.~\ref{sec:selection}. 

Each transfer factor is simply a ratio of the yields obtained from MC
simulation for the same bin of the signal region and a given control
sample:

\begin{equation}
  \label{equ:tf-ratio}
  {\rm TF} = \frac{N_{\rm MC}^{\rm signal}(\njet,\nb,\scalht)}{N_{\rm
      MC}^{\rm control}(\njet,\nb,\scalht)} 
\end{equation}

In this way, predictions of background counts from SM processes can be
made based on the various control samples:

\begin{equation}
  \label{equ:pred-method}
  \npre^{\rm signal}(\njet,\nb,\scalht) = \frac{N_{\rm MC}^{\rm
      signal}(\njet,\nb,\scalht)}{N_{\rm MC}^{\rm
      control}(\njet,\nb,\scalht)} \times \nobs^{\rm
    control}(\njet,\nb,\scalht)   
\end{equation}

When constructing the transfer factors, the MC expectations for the
following SM processes are considered: W + jets ($N_{\rm W}$), \ttbar
+ jets ($N_{\ttbar}$), \znunu\ + jets ($N_{\znunu}$), DY + jets
($N_{\mathrm DY}$), \gj ($N_\gamma$), single top + jets
production via the $s$, $t$, and $tW$-channels ($N_{\rm top}$), $WW+$~jets, $WZ~+$~jets, and $ZZ + \textrm{jets}$ ($N_{\rm di-boson}$), and $\ttbar$V or
$\ttbar$H ($N_{\rm {\ttbar}X}$). Details on the MC
samples used are given in Sec.~\ref{sec:datasets}. All MC samples
are normalised to the integrated luminosity of the appropriate data
sample.

The selection criteria for the data control samples closely resemble
those for the signal region, differing mainly through the use of a
lepton or photon object {\it tag} (that is ignored in the calculation
of jet-based kinematic variables such as \scalht, \mht, \alphat, \etc)
and minimal additional kinematic requirements (\eg invariant or
transverse mass windows) to obtain W, Z, and \ttbar-enriched event
samples. The same selection criteria are designed to suppress signal
contamination in the control samples so that unbiased data-driven
estimates for the SM backgrounds in the signal region can be
made. More detail on the selection criteria can be found in Sec.~\ref{sec:selection}.

The transfer factors account for differences in cross sections and
branching ratios, acceptance and reconstruction efficiencies, and/or
kinematic requirements between the signal and control regions. Any
dependence on \njet, \nb, or \HT is largely attributable to
differences in acceptance due to the presence or otherwise of \alphat
or \mht requirements.

Many systematic effects are expected to cancel largely in the transfer
factor. However, a systematic uncertainty is assigned to each transfer
factor to account for theoretical uncertainties and effects such as
the mismodelling of kinematics (\eg acceptances) and instrumental
effects (\eg reconstruction efficiencies).

In the end, a fitting procedure that provides the final result is
defined formally by the likelihood model described in
Sec.~\ref{sec:likelihood}. In summary, the observation in each bin
(defined in terms of the variables \njet, \nb, and \scalht) of the
signal sample is modelled as Poisson-distributed about the sum of a SM
expectation (and a potential signal contribution). The components of
this SM expectation are related to the expected yields in the control
samples via transfer factors derived from simulation. The observations
in each bin (again defined by \njet, \nb, and \scalht) of the control
samples are similarly modelled as Poisson-distributed about the
expected yields for each control sample. In this way, for a given
bin, the observed yields in the signal and control samples are
connected via the transfer factors derived from simulation. 

The transfer factors are shown in Tables~\ref{tab:tf_mu_zinv_sym}-
\ref{tab:tf_mumu_zinv_mono}. The procedure to determine the systematic
uncertainties associated with these transfer factors is described in
Sec.~\ref{sec:systematics}.


The transfer factors are shown in tables below.

\begin{table}[h!]
\tiny
\centering
\caption{Transfer factors from the \mj control region to the \zInv~ background for symmetric categories.\label{tab:tf_mu_zinv_sym}}
\scalebox{0.85}{\begin{tabular}{ccccccccc}
	\hline\hline
	& \multicolumn{8}{c}{\scalht (\gev)} \\ 
	 (\njet,  \nb) & 200-250 & 250-300 & 300-350 & 350-400 & 400-500 & 500-600 & 600-800 & 800-$\infty$ \\ [0.8ex] 
\hline
	(2, 0) & $1.00\pm 0.02$ & $0.76\pm 0.01$ & $0.57\pm 0.01$ & $0.39\pm 0.01$ & $0.30\pm 0.01$ & $0.21\pm 0.01$ & $0.14\pm 0.00$ & $0.28\pm 0.01$ \\[0.5ex] 
	(2, 1) & $0.48\pm 0.03$ & $0.51\pm 0.02$ & $0.54\pm 0.03$ & $0.44\pm 0.03$ & $0.33\pm 0.02$ & $0.26\pm 0.02$ & $0.20\pm 0.01$ & $0.43\pm 0.03$ \\[0.5ex] 
	(2, 2) & $0.68\pm 0.13$ & $0.77\pm 0.14$ & $0.68\pm 0.14$ & $0.48\pm 0.12$ & $0.33\pm 0.07$ & $0.37\pm 0.12$ & $0.22\pm 0.06$ & -- \\[0.5ex] 
	(3, 0) & $0.19\pm 0.08$ & $0.45\pm 0.01$ & $0.53\pm 0.01$ & $0.53\pm 0.01$ & $0.42\pm 0.01$ & $0.27\pm 0.01$ & $0.19\pm 0.00$ & $0.26\pm 0.00$ \\[0.5ex] 
	(3, 1) & -- & $0.10\pm 0.01$ & $0.16\pm 0.01$ & $0.17\pm 0.01$ & $0.20\pm 0.01$ & $0.18\pm 0.01$ & $0.15\pm 0.01$ & $0.26\pm 0.01$ \\[0.5ex] 
	(3, 2) & -- & $0.08\pm 0.02$ & $0.10\pm 0.01$ & $0.08\pm 0.01$ & $0.08\pm 0.01$ & $0.07\pm 0.01$ & $0.07\pm 0.01$ & $0.14\pm 0.02$ \\[0.5ex] 
	(3, $\ge3$) & -- & -- & -- & -- & $0.03\pm 0.03$ & -- & -- & -- \\[0.5ex] 
	(4, 0) & -- & -- & $0.51\pm 0.03$ & $0.54\pm 0.02$ & $0.44\pm 0.01$ & $0.33\pm 0.01$ & $0.23\pm 0.00$ & $0.24\pm 0.00$ \\[0.5ex] 
	(4, 1) & -- & -- & $0.13\pm 0.01$ & $0.11\pm 0.01$ & $0.11\pm 0.00$ & $0.11\pm 0.01$ & $0.10\pm 0.00$ & $0.17\pm 0.01$ \\[0.5ex] 
	(4, 2) & -- & -- & $0.07\pm 0.02$ & $0.04\pm 0.01$ & $0.05\pm 0.00$ & $0.04\pm 0.00$ & $0.05\pm 0.00$ & $0.09\pm 0.01$ \\[0.5ex] 
	(4, $\ge3$) & -- & -- & -- & $0.06\pm 0.04$ & $0.07\pm 0.02$ & $0.03\pm 0.02$ & $0.03\pm 0.01$ & $0.12\pm 0.05$ \\[0.5ex] 
	($\ge5$, 0) & -- & -- & -- & $0.31\pm 0.04$ & $0.36\pm 0.01$ & $0.27\pm 0.01$ & $0.19\pm 0.00$ & $0.18\pm 0.00$ \\[0.5ex] 
	($\ge5$, 1) & -- & -- & -- & $0.07\pm 0.02$ & $0.07\pm 0.01$ & $0.05\pm 0.00$ & $0.05\pm 0.00$ & $0.06\pm 0.00$ \\[0.5ex] 
	($\ge5$, 2) & -- & -- & -- & $0.01\pm 0.01$ & $0.03\pm 0.00$ & $0.02\pm 0.00$ & $0.02\pm 0.00$ & $0.03\pm 0.00$ \\[0.5ex] 
	($\ge5$, $\ge3$) & -- & -- & -- & -- & $0.03\pm 0.02$ & $0.02\pm 0.01$ & $0.03\pm 0.01$ & $0.03\pm 0.01$ \\[0.5ex] 
	\hline
	\hline
\end{tabular}}
\end{table}

\begin{table}[h!]
\tiny
\centering
\caption{Transfer factors from the \mj control region to the \zInv~ background for asymmetric categories.\label{tab:tf_mu_zinv_asym}}
\begin{tabular}
{ccccccccc}
	\hline\hline
	& \multicolumn{8}{c}{\scalht (\gev)} \\ 
	 (\njet,  \nb) & 200-250 & 250-300 & 300-350 & 350-400 & 400-500 & 500-600 & 600-800 & 800-$\infty$ \\ [0.8ex] 
\hline
	(2a, 0) & $0.55^{+ 0.01 }_{- 0.01 }$ & $0.34^{+ 0.01 }_{- 0.01 }$ & $0.30^{+ 0.01 }_{- 0.01 }$ & $0.29^{+ 0.01 }_{- 0.01 }$ & $0.30^{+ 0.01 }_{- 0.01 }$ & $0.26^{+ 0.02 }_{- 0.02 }$ & $0.23^{+ 0.02 }_{- 0.02 }$ & -- \\[0.5ex] 
	(2a, 1) & $0.22^{+ 0.01 }_{- 0.01 }$ & $0.18^{+ 0.01 }_{- 0.01 }$ & $0.17^{+ 0.01 }_{- 0.01 }$ & $0.21^{+ 0.02 }_{- 0.02 }$ & $0.18^{+ 0.02 }_{- 0.02 }$ & $0.25^{+ 0.05 }_{- 0.05 }$ & -- & -- \\[0.5ex] 
	(2a, 2) & $0.23^{+ 0.02 }_{- 0.02 }$ & $0.13^{+ 0.02 }_{- 0.02 }$ & $0.16^{+ 0.04 }_{- 0.04 }$ & $0.10^{+ 0.04 }_{- 0.04 }$ & $0.07^{+ 0.04 }_{- 0.04 }$ & -- & -- & -- \\[0.5ex] 
	(3a, 0) & $0.55^{+ 0.01 }_{- 0.01 }$ & $0.43^{+ 0.01 }_{- 0.01 }$ & $0.41^{+ 0.01 }_{- 0.01 }$ & $0.32^{+ 0.01 }_{- 0.01 }$ & $0.23^{+ 0.01 }_{- 0.01 }$ & $0.18^{+ 0.02 }_{- 0.02 }$ & $0.17^{+ 0.02 }_{- 0.02 }$ & -- \\[0.5ex] 
	(3a, 1) & $0.10^{+ 0.01 }_{- 0.01 }$ & $0.09^{+ 0.00 }_{- 0.00 }$ & $0.09^{+ 0.01 }_{- 0.01 }$ & $0.11^{+ 0.01 }_{- 0.01 }$ & $0.08^{+ 0.01 }_{- 0.01 }$ & $0.03^{+ 0.01 }_{- 0.01 }$ & $0.08^{+ 0.02 }_{- 0.02 }$ & -- \\[0.5ex] 
	(3a, 2) & $0.05^{+ 0.01 }_{- 0.01 }$ & $0.04^{+ 0.00 }_{- 0.00 }$ & $0.05^{+ 0.01 }_{- 0.01 }$ & $0.04^{+ 0.01 }_{- 0.01 }$ & $0.07^{+ 0.02 }_{- 0.02 }$ & $0.06^{+ 0.03 }_{- 0.03 }$ & -- & -- \\[0.5ex] 
	(3a, $\ge3$) & $0.03^{+ 0.03 }_{- 0.03 }$ & $0.05^{+ 0.03 }_{- 0.03 }$ & -- & -- & -- & -- & -- & -- \\[0.5ex] 
	(4a, 0) & $0.10^{+ 0.03 }_{- 0.03 }$ & $0.28^{+ 0.02 }_{- 0.02 }$ & $0.41^{+ 0.02 }_{- 0.02 }$ & $0.42^{+ 0.02 }_{- 0.02 }$ & $0.37^{+ 0.02 }_{- 0.02 }$ & $0.23^{+ 0.02 }_{- 0.02 }$ & $0.13^{+ 0.02 }_{- 0.02 }$ & -- \\[0.5ex] 
	(4a, 1) & $0.04^{+ 0.02 }_{- 0.02 }$ & $0.05^{+ 0.01 }_{- 0.01 }$ & $0.06^{+ 0.00 }_{- 0.00 }$ & $0.07^{+ 0.01 }_{- 0.01 }$ & $0.09^{+ 0.01 }_{- 0.01 }$ & $0.05^{+ 0.01 }_{- 0.01 }$ & $0.03^{+ 0.01 }_{- 0.01 }$ & -- \\[0.5ex] 
	(4a, 2) & -- & $0.03^{+ 0.01 }_{- 0.01 }$ & $0.02^{+ 0.00 }_{- 0.00 }$ & $0.03^{+ 0.01 }_{- 0.01 }$ & $0.03^{+ 0.01 }_{- 0.01 }$ & $0.01^{+ 0.01 }_{- 0.01 }$ & $0.00^{+ 0.00 }_{- 0.00 }$ & -- \\[0.5ex] 
	(4a, $\ge3$) & -- & $0.01^{+ 0.01 }_{- 0.01 }$ & $0.05^{+ 0.03 }_{- 0.03 }$ & $0.02^{+ 0.02 }_{- 0.02 }$ & -- & -- & -- & -- \\[0.5ex] 
	($\ge5$a, 0) & -- & $0.13^{+ 0.13 }_{- 0.13 }$ & $0.36^{+ 0.05 }_{- 0.05 }$ & $0.37^{+ 0.03 }_{- 0.03 }$ & $0.30^{+ 0.02 }_{- 0.02 }$ & $0.20^{+ 0.02 }_{- 0.02 }$ & $0.19^{+ 0.03 }_{- 0.03 }$ & -- \\[0.5ex] 
	($\ge5$a, 1) & -- & $0.06^{+ 0.06 }_{- 0.06 }$ & $0.06^{+ 0.01 }_{- 0.01 }$ & $0.04^{+ 0.01 }_{- 0.01 }$ & $0.04^{+ 0.00 }_{- 0.00 }$ & $0.04^{+ 0.01 }_{- 0.01 }$ & $0.02^{+ 0.01 }_{- 0.01 }$ & -- \\[0.5ex] 
	($\ge5$a, 2) & -- & -- & $0.02^{+ 0.01 }_{- 0.01 }$ & $0.02^{+ 0.01 }_{- 0.01 }$ & $0.02^{+ 0.00 }_{- 0.00 }$ & $0.02^{+ 0.01 }_{- 0.01 }$ & $0.00^{+ 0.00 }_{- 0.00 }$ & -- \\[0.5ex] 
	($\ge5$a, $\ge3$) & -- & -- & -- & $0.00^{+ 0.00 }_{- 0.00 }$ & $0.02^{+ 0.01 }_{- 0.01 }$ & $0.03^{+ 0.03 }_{- 0.03 }$ & -- & -- \\[0.5ex] 
	\hline
	\hline
\end{tabular}
\end{table}

\begin{table}[h!]
\tiny
\centering
\caption{Transfer factors from the \mj control region to the \zInv~ background for monojet categories.\label{tab:tf_mu_zinv_mono}}
\scalebox{0.85}{\begin{tabular}{ccccccccc}
	\hline\hline
	& \multicolumn{8}{c}{\scalht (\gev)} \\ 
	 (\njet,  \nb) & 200-250 & 250-300 & 300-350 & 350-400 & 400-500 & 500-600 & 600-800 & 800-$\infty$ \\ [0.8ex] 
\hline
	(1, 0) & $1.36\pm 0.01$ & $1.33\pm 0.01$ & $1.30\pm 0.02$ & $1.24\pm 0.02$ & $1.09\pm 0.02$ & $1.17\pm 0.06$ & $1.45\pm 0.17$ & -- \\[0.5ex] 
	(1, 1) & $1.38\pm 0.03$ & $1.45\pm 0.06$ & $1.33\pm 0.08$ & $1.29\pm 0.11$ & $1.11\pm 0.11$ & $0.68\pm 0.15$ & -- & -- \\[0.5ex] 
	\hline
	\hline
\end{tabular}}
\end{table}

\begin{table}[h!]
\tiny
\centering
\caption{Transfer factors from the \gj control region to the \zInv~ background for symmetric categories.\label{tab:tf_gj_zinv_sym}}
\scalebox{0.85}{\begin{tabular}{ccccc}
	\hline\hline
	& \multicolumn{4}{c}{\scalht (\gev)} \\ 
	 (\njet,  \nb) & 400-500 & 500-600 & 600-800 & 800-$\infty$ \\ [0.8ex] 
\hline
	(2, 0) & $0.71\pm 0.03$ & $0.62\pm 0.04$ & $0.64\pm 0.03$ & $0.31\pm 0.01$ \\[0.5ex] 
	(2, 1) & $0.63\pm 0.08$ & $0.59\pm 0.09$ & $0.60\pm 0.09$ & $0.27\pm 0.02$ \\[0.5ex] 
	(2, 2) & $1.01\pm 0.61$ & $0.89\pm 0.72$ & $0.32\pm 0.17$ & -- \\[0.5ex] 
	(3, 0) & $0.69\pm 0.02$ & $0.63\pm 0.03$ & $0.58\pm 0.02$ & $0.29\pm 0.01$ \\[0.5ex] 
	(3, 1) & $0.78\pm 0.08$ & $0.62\pm 0.07$ & $0.55\pm 0.05$ & $0.31\pm 0.02$ \\[0.5ex] 
	(3, 2) & $0.93\pm 0.28$ & $1.00\pm 0.38$ & $0.65\pm 0.19$ & $0.30\pm 0.06$ \\[0.5ex] 
	(3, $\ge3$) & $5.85\pm 4.78$ & -- & -- & -- \\[0.5ex] 
	(4, 0) & $0.85\pm 0.04$ & $0.56\pm 0.03$ & $0.56\pm 0.02$ & $0.32\pm 0.01$ \\[0.5ex] 
	(4, 1) & $0.89\pm 0.10$ & $0.64\pm 0.08$ & $0.54\pm 0.05$ & $0.30\pm 0.02$ \\[0.5ex] 
	(4, 2) & $1.04\pm 0.34$ & $0.68\pm 0.18$ & $0.45\pm 0.10$ & $0.34\pm 0.06$ \\[0.5ex] 
	(4, $\ge3$) & $0.64\pm 0.45$ & $21.42\pm 43.71$ & $0.43\pm 0.38$ & $0.27\pm 0.14$ \\[0.5ex] 
	($\ge5$, 0) & $0.86\pm 0.09$ & $0.68\pm 0.06$ & $0.56\pm 0.03$ & $0.30\pm 0.01$ \\[0.5ex] 
	($\ge5$, 1) & $0.51\pm 0.09$ & $0.55\pm 0.09$ & $0.47\pm 0.04$ & $0.32\pm 0.02$ \\[0.5ex] 
	($\ge5$, 2) & $0.61\pm 0.22$ & $0.61\pm 0.19$ & $0.38\pm 0.07$ & $0.32\pm 0.04$ \\[0.5ex] 
	($\ge5$, $\ge3$) & $0.86\pm 0.52$ & $0.57\pm 0.50$ & $0.52\pm 0.23$ & $0.54\pm 0.24$ \\[0.5ex] 
	\hline
	\hline
\end{tabular}}
\end{table}

\begin{table}[h!]
\tiny
\centering
\caption{Transfer factors from the \gj control region to the \zInv~ background for asymmetric and monojet categories. The letter ``a'' in jet \eg ``2a''  indicates the asymmetric jet bins. All entries are non-zero but are truncated to one decimal place.\label{tab:tf_gj_zinv_asym}}
\begin{tabular}
{ccccc}
	\hline\hline
&	& \multicolumn{4}{c}{\scalht (\gev)} \\ 
	 (\njet,  \nb) & 400-500 & 500-600 & 600-800 & 800-$\infty$ \\ [0.8ex] 
\hline
	(1, 0) & $0.33^{+ 0.01 }_{- 0.01 }$ & $0.32^{+ 0.02 }_{- 0.02 }$ & $0.28^{+ 0.01 }_{- 0.01 }$ & -- \\[0.5ex] 
	(1, 1) & $0.36^{+ 0.06 }_{- 0.06 }$ & $0.30^{+ 0.07 }_{- 0.07 }$ & $0.28^{+ 0.06 }_{- 0.06 }$ & -- \\[0.5ex] 
	(2a, 0) & $0.62^{+ 0.04 }_{- 0.04 }$ & $0.56^{+ 0.06 }_{- 0.06 }$ & $0.43^{+ 0.05 }_{- 0.05 }$ & -- \\[0.5ex] 
	(2a, 1) & $0.84^{+ 0.20 }_{- 0.20 }$ & $0.60^{+ 0.22 }_{- 0.22 }$ & $1.27^{+ 0.46 }_{- 0.46 }$ & -- \\[0.5ex] 
	(2a, 2) & -- & -- & -- & -- \\[0.5ex] 
	(3a, 0) & $0.63^{+ 0.05 }_{- 0.05 }$ & $0.59^{+ 0.09 }_{- 0.09 }$ & $0.64^{+ 0.11 }_{- 0.11 }$ & -- \\[0.5ex] 
	(3a, 1) & $0.51^{+ 0.12 }_{- 0.12 }$ & $0.32^{+ 0.13 }_{- 0.13 }$ & $0.32^{+ 0.13 }_{- 0.13 }$ & -- \\[0.5ex] 
	(3a, 2) & $0.58^{+ 0.37 }_{- 0.37 }$ & $0.49^{+ 0.54 }_{- 0.54 }$ & $0.44^{+ 0.50 }_{- 0.50 }$ & -- \\[0.5ex] 
	(3a, $\ge3$) & -- & -- & -- & -- \\[0.5ex] 
	(4a, 0) & $0.65^{+ 0.05 }_{- 0.05 }$ & $0.58^{+ 0.10 }_{- 0.10 }$ & $0.49^{+ 0.12 }_{- 0.12 }$ & -- \\[0.5ex] 
	(4a, 1) & $0.74^{+ 0.15 }_{- 0.15 }$ & $0.30^{+ 0.11 }_{- 0.11 }$ & $0.46^{+ 0.26 }_{- 0.26 }$ & -- \\[0.5ex] 
	(4a, 2) & $0.70^{+ 0.36 }_{- 0.36 }$ & -- & -- & -- \\[0.5ex] 
	(4a, $\ge3$) & -- & -- & -- & -- \\[0.5ex] 
	($\ge5$a, 0) & $0.60^{+ 0.07 }_{- 0.07 }$ & $0.60^{+ 0.12 }_{- 0.12 }$ & $0.51^{+ 0.14 }_{- 0.14 }$ & -- \\[0.5ex] 
	($\ge5$a, 1) & $0.90^{+ 0.29 }_{- 0.29 }$ & $0.40^{+ 0.15 }_{- 0.15 }$ & $0.56^{+ 0.29 }_{- 0.29 }$ & -- \\[0.5ex] 
	($\ge5$a, 2) & $0.71^{+ 0.44 }_{- 0.44 }$ & $3.83^{+ 3.60 }_{- 3.60 }$ & -- & -- \\[0.5ex] 
	($\ge5$a, $\ge3$) & -- & -- & -- & -- \\[0.5ex] 
	\hline
	\hline
\end{tabular}
\end{table}

\begin{table}[h!]
\tiny
\centering
\caption{Transfer factors from the \gj control region to the \zInv~ background for monojet categories.\label{tab:tf_gj_zinv_mono}}
\scalebox{0.85}{\begin{tabular}{ccccc}
	\hline\hline
	& \multicolumn{4}{c}{\scalht (\gev)} \\ 
	 (\njet,  \nb) & 400-500 & 500-600 & 600-800 & 800-$\infty$ \\ [0.8ex] 
\hline
	(1, 0) & $0.59\pm 0.02$ & $0.54\pm 0.03$ & $0.58\pm 0.03$ & -- \\[0.5ex] 
	(1, 1) & $1.00\pm 0.18$ & $1.10\pm 0.32$ & -- & -- \\[0.5ex] 
	\hline
	\hline
\end{tabular}}
\end{table}

\begin{table}[h!]
\tiny
\centering
\caption{Transfer factors from the \mj control region to the \ttbar/W background for symmetric categories.\label{tab:tf_mu_ttw_sym}}
\scalebox{0.85}{\begin{tabular}{ccccccccc}
	\hline\hline
	& \multicolumn{8}{c}{\scalht (\gev)} \\ 
	 (\njet,  \nb) & 200-250 & 250-300 & 300-350 & 350-400 & 400-500 & 500-600 & 600-800 & 800-$\infty$ \\ [0.8ex] 
\hline
	(2, 0) & $0.96\pm 0.02$ & $0.64\pm 0.01$ & $0.42\pm 0.01$ & $0.26\pm 0.01$ & $0.17\pm 0.01$ & $0.10\pm 0.00$ & $0.06\pm 0.00$ & $0.12\pm 0.00$ \\[0.5ex] 
	(2, 1) & $0.72\pm 0.03$ & $0.55\pm 0.03$ & $0.32\pm 0.02$ & $0.24\pm 0.02$ & $0.13\pm 0.01$ & $0.07\pm 0.01$ & $0.05\pm 0.01$ & $0.09\pm 0.01$ \\[0.5ex] 
	(2, 2) & $0.43\pm 0.09$ & $0.52\pm 0.12$ & $0.42\pm 0.12$ & $0.19\pm 0.09$ & $0.03\pm 0.02$ & $0.10\pm 0.10$ & $0.02\pm 0.01$ & -- \\[0.5ex] 
	(3, 0) & $0.50\pm 0.21$ & $0.43\pm 0.02$ & $0.52\pm 0.01$ & $0.49\pm 0.01$ & $0.33\pm 0.01$ & $0.17\pm 0.01$ & $0.09\pm 0.00$ & $0.12\pm 0.00$ \\[0.5ex] 
	(3, 1) & -- & $0.34\pm 0.02$ & $0.34\pm 0.01$ & $0.29\pm 0.01$ & $0.19\pm 0.01$ & $0.09\pm 0.01$ & $0.05\pm 0.00$ & $0.09\pm 0.01$ \\[0.5ex] 
	(3, 2) & -- & $0.16\pm 0.02$ & $0.24\pm 0.02$ & $0.20\pm 0.02$ & $0.15\pm 0.01$ & $0.06\pm 0.01$ & $0.03\pm 0.01$ & $0.02\pm 0.01$ \\[0.5ex] 
	(3, $\ge3$) & -- & $0.07\pm 0.07$ & -- & -- & $0.13\pm 0.07$ & -- & -- & -- \\[0.5ex] 
	(4, 0) & -- & -- & $0.68\pm 0.04$ & $0.62\pm 0.02$ & $0.44\pm 0.01$ & $0.26\pm 0.01$ & $0.13\pm 0.00$ & $0.12\pm 0.00$ \\[0.5ex] 
	(4, 1) & -- & -- & $0.48\pm 0.03$ & $0.41\pm 0.02$ & $0.27\pm 0.01$ & $0.15\pm 0.01$ & $0.07\pm 0.00$ & $0.08\pm 0.01$ \\[0.5ex] 
	(4, 2) & -- & -- & $0.40\pm 0.04$ & $0.37\pm 0.02$ & $0.22\pm 0.01$ & $0.10\pm 0.01$ & $0.05\pm 0.01$ & $0.05\pm 0.01$ \\[0.5ex] 
	(4, $\ge3$) & -- & -- & $0.57\pm 0.27$ & $0.27\pm 0.09$ & $0.27\pm 0.06$ & $0.10\pm 0.04$ & $0.05\pm 0.03$ & $0.06\pm 0.04$ \\[0.5ex] 
	($\ge5$, 0) & -- & -- & -- & $0.37\pm 0.05$ & $0.52\pm 0.02$ & $0.30\pm 0.01$ & $0.18\pm 0.01$ & $0.13\pm 0.00$ \\[0.5ex] 
	($\ge5$, 1) & -- & -- & -- & $0.39\pm 0.05$ & $0.37\pm 0.01$ & $0.20\pm 0.01$ & $0.09\pm 0.00$ & $0.08\pm 0.00$ \\[0.5ex] 
	($\ge5$, 2) & -- & -- & -- & $0.33\pm 0.06$ & $0.36\pm 0.02$ & $0.18\pm 0.01$ & $0.08\pm 0.01$ & $0.06\pm 0.00$ \\[0.5ex] 
	($\ge5$, $\ge3$) & -- & -- & -- & -- & $0.27\pm 0.05$ & $0.14\pm 0.03$ & $0.06\pm 0.01$ & $0.05\pm 0.01$ \\[0.5ex] 
	\hline
	\hline
\end{tabular}}
\end{table}

\begin{table}[h!]
\tiny
\centering
\caption{Transfer factors from the \mj control region to the \ttbar/W background for asymmetric categories.\label{tab:tf_mu_ttw_asym}}
\scalebox{0.85}{\begin{tabular}{ccccccccc}
	\hline\hline
	& \multicolumn{8}{c}{\scalht (\gev)} \\ 
	 (\njet,  \nb) & 200-250 & 250-300 & 300-350 & 350-400 & 400-500 & 500-600 & 600-800 & 800-$\infty$ \\ [0.8ex] 
\hline
	(2a, 0) & $0.49^{+ 0.01 }_{- 0.01 }$ & $0.25^{+ 0.01 }_{- 0.01 }$ & $0.20^{+ 0.01 }_{- 0.01 }$ & $0.16^{+ 0.01 }_{- 0.01 }$ & $0.13^{+ 0.01 }_{- 0.01 }$ & $0.15^{+ 0.02 }_{- 0.02 }$ & $0.08^{+ 0.01 }_{- 0.01 }$ & -- \\[0.5ex] 
	(2a, 1) & $0.34^{+ 0.01 }_{- 0.01 }$ & $0.22^{+ 0.02 }_{- 0.02 }$ & $0.14^{+ 0.02 }_{- 0.02 }$ & $0.10^{+ 0.02 }_{- 0.02 }$ & $0.10^{+ 0.02 }_{- 0.02 }$ & $0.11^{+ 0.05 }_{- 0.05 }$ & -- & -- \\[0.5ex] 
	(2a, 2) & $0.25^{+ 0.03 }_{- 0.03 }$ & $0.14^{+ 0.03 }_{- 0.03 }$ & $0.12^{+ 0.06 }_{- 0.06 }$ & $0.11^{+ 0.08 }_{- 0.08 }$ & $0.08^{+ 0.06 }_{- 0.06 }$ & -- & -- & -- \\[0.5ex] 
	(3a, 0) & $0.62^{+ 0.02 }_{- 0.02 }$ & $0.45^{+ 0.01 }_{- 0.01 }$ & $0.39^{+ 0.02 }_{- 0.02 }$ & $0.24^{+ 0.02 }_{- 0.02 }$ & $0.13^{+ 0.01 }_{- 0.01 }$ & $0.08^{+ 0.01 }_{- 0.01 }$ & $0.05^{+ 0.01 }_{- 0.01 }$ & -- \\[0.5ex] 
	(3a, 1) & $0.42^{+ 0.01 }_{- 0.01 }$ & $0.33^{+ 0.01 }_{- 0.01 }$ & $0.29^{+ 0.01 }_{- 0.01 }$ & $0.21^{+ 0.02 }_{- 0.02 }$ & $0.08^{+ 0.01 }_{- 0.01 }$ & $0.07^{+ 0.02 }_{- 0.02 }$ & $0.02^{+ 0.01 }_{- 0.01 }$ & -- \\[0.5ex] 
	(3a, 2) & $0.26^{+ 0.02 }_{- 0.02 }$ & $0.21^{+ 0.01 }_{- 0.01 }$ & $0.26^{+ 0.02 }_{- 0.02 }$ & $0.18^{+ 0.02 }_{- 0.02 }$ & $0.05^{+ 0.02 }_{- 0.02 }$ & $0.06^{+ 0.05 }_{- 0.05 }$ & -- & -- \\[0.5ex] 
	(3a, $\ge3$) & $0.17^{+ 0.07 }_{- 0.07 }$ & $0.20^{+ 0.05 }_{- 0.05 }$ & $0.22^{+ 0.08 }_{- 0.08 }$ & -- & -- & -- & -- & -- \\[0.5ex] 
	(4a, 0) & $0.09^{+ 0.05 }_{- 0.05 }$ & $0.41^{+ 0.03 }_{- 0.03 }$ & $0.50^{+ 0.03 }_{- 0.03 }$ & $0.53^{+ 0.03 }_{- 0.03 }$ & $0.32^{+ 0.02 }_{- 0.02 }$ & $0.13^{+ 0.02 }_{- 0.02 }$ & $0.05^{+ 0.01 }_{- 0.01 }$ & -- \\[0.5ex] 
	(4a, 1) & $0.08^{+ 0.03 }_{- 0.03 }$ & $0.21^{+ 0.01 }_{- 0.01 }$ & $0.37^{+ 0.01 }_{- 0.01 }$ & $0.35^{+ 0.02 }_{- 0.02 }$ & $0.27^{+ 0.02 }_{- 0.02 }$ & $0.09^{+ 0.02 }_{- 0.02 }$ & $0.01^{+ 0.01 }_{- 0.01 }$ & -- \\[0.5ex] 
	(4a, 2) & $0.08^{+ 0.05 }_{- 0.05 }$ & $0.14^{+ 0.01 }_{- 0.01 }$ & $0.31^{+ 0.02 }_{- 0.02 }$ & $0.31^{+ 0.02 }_{- 0.02 }$ & $0.21^{+ 0.02 }_{- 0.02 }$ & $0.06^{+ 0.02 }_{- 0.02 }$ & $0.01^{+ 0.01 }_{- 0.01 }$ & -- \\[0.5ex] 
	(4a, $\ge3$) & -- & $0.26^{+ 0.08 }_{- 0.08 }$ & $0.31^{+ 0.06 }_{- 0.06 }$ & $0.25^{+ 0.07 }_{- 0.07 }$ & $0.22^{+ 0.08 }_{- 0.08 }$ & -- & -- & -- \\[0.5ex] 
	($\ge5$a, 0) & -- & $1.05^{+ 0.52 }_{- 0.52 }$ & $0.66^{+ 0.10 }_{- 0.10 }$ & $0.61^{+ 0.06 }_{- 0.06 }$ & $0.55^{+ 0.04 }_{- 0.04 }$ & $0.29^{+ 0.04 }_{- 0.04 }$ & $0.15^{+ 0.04 }_{- 0.04 }$ & -- \\[0.5ex] 
	($\ge5$a, 1) & -- & $0.26^{+ 0.12 }_{- 0.12 }$ & $0.51^{+ 0.05 }_{- 0.05 }$ & $0.45^{+ 0.03 }_{- 0.03 }$ & $0.39^{+ 0.02 }_{- 0.02 }$ & $0.22^{+ 0.02 }_{- 0.02 }$ & $0.08^{+ 0.02 }_{- 0.02 }$ & -- \\[0.5ex] 
	($\ge5$a, 2) & -- & $0.00^{+ 0.00 }_{- 0.00 }$ & $0.38^{+ 0.05 }_{- 0.05 }$ & $0.44^{+ 0.03 }_{- 0.03 }$ & $0.37^{+ 0.02 }_{- 0.02 }$ & $0.18^{+ 0.03 }_{- 0.03 }$ & $0.10^{+ 0.03 }_{- 0.03 }$ & -- \\[0.5ex] 
	($\ge5$a, $\ge3$) & -- & -- & $0.22^{+ 0.09 }_{- 0.09 }$ & $0.36^{+ 0.08 }_{- 0.08 }$ & $0.41^{+ 0.07 }_{- 0.07 }$ & $0.18^{+ 0.07 }_{- 0.07 }$ & -- & -- \\[0.5ex] 
	\hline
	\hline
\end{tabular}}
\end{table}

\begin{table}[h!]
\tiny
\centering
\caption{Transfer factors from the \mj control region to the \ttbar/W background for monojet categories.\label{tab:tf_mu_ttw_mono}}
\scalebox{0.85}{\begin{tabular}{ccccccccc}
	\hline\hline
	& \multicolumn{8}{c}{\scalht (\gev)} \\ 
	 (\njet,  \nb) & 200-250 & 250-300 & 300-350 & 350-400 & 400-500 & 500-600 & 600-800 & 800-$\infty$ \\ [0.8ex] 
\hline
	(1, 0) & $1.15\pm 0.02$ & $0.95\pm 0.02$ & $0.72\pm 0.03$ & $0.56\pm 0.03$ & $0.52\pm 0.03$ & $0.39\pm 0.03$ & $0.33\pm 0.03$ & -- \\[0.5ex] 
	(1, 1) & $0.84\pm 0.07$ & $0.71\pm 0.09$ & $0.65\pm 0.13$ & $0.42\pm 0.10$ & $0.72\pm 0.16$ & $0.38\pm 0.15$ & -- & -- \\[0.5ex] 
	\hline
	\hline
\end{tabular}}
\end{table}

\begin{table}[h!]
\tiny
\centering
\caption{Transfer factors from the \mmj control region to the \zInv~ background for symmetric categories.\label{tab:tf_mumu_zinv_sym}}
\scalebox{0.85}{\begin{tabular}{ccccccccc}
	\hline\hline
	& \multicolumn{8}{c}{\scalht (\gev)} \\ 
	 (\njet,  \nb) & 200-250 & 250-300 & 300-350 & 350-400 & 400-500 & 500-600 & 600-800 & 800-$\infty$ \\ [0.8ex] 
\hline
	(2, 0) & $9.92^{+ 0.85 }_{- 0.85 }$ & $7.47^{+ 0.52 }_{- 0.52 }$ & $4.96^{+ 0.33 }_{- 0.33 }$ & $4.04^{+ 0.25 }_{- 0.25 }$ & $3.16^{+ 0.11 }_{- 0.11 }$ & $2.12^{+ 0.09 }_{- 0.09 }$ & $1.26^{+ 0.05 }_{- 0.05 }$ & $2.70^{+ 0.11 }_{- 0.11 }$ \\[0.5ex] 
	(2, 1) & $8.98^{+ 2.45 }_{- 2.45 }$ & $4.61^{+ 0.96 }_{- 0.96 }$ & $3.19^{+ 0.61 }_{- 0.61 }$ & $2.70^{+ 0.53 }_{- 0.53 }$ & $2.46^{+ 0.24 }_{- 0.24 }$ & $2.13^{+ 0.31 }_{- 0.31 }$ & $1.64^{+ 0.20 }_{- 0.20 }$ & $2.53^{+ 0.31 }_{- 0.31 }$ \\[0.5ex] 
	(2, 2) & $10.56^{+ 8.41 }_{- 8.41 }$ & $2.79^{+ 1.38 }_{- 1.38 }$ & $4.71^{+ 2.91 }_{- 2.91 }$ & $1.41^{+ 0.78 }_{- 0.78 }$ & $2.76^{+ 1.30 }_{- 1.30 }$ & $5.16^{+ 2.68 }_{- 2.68 }$ & $1.28^{+ 0.61 }_{- 0.61 }$ & -- \\[0.5ex] 
	(3, 0) & $15.26^{+ 16.46 }_{- 16.46 }$ & $7.28^{+ 1.16 }_{- 1.16 }$ & $6.32^{+ 0.58 }_{- 0.58 }$ & $6.05^{+ 0.42 }_{- 0.42 }$ & $4.71^{+ 0.18 }_{- 0.18 }$ & $3.14^{+ 0.12 }_{- 0.12 }$ & $1.95^{+ 0.06 }_{- 0.06 }$ & $2.48^{+ 0.09 }_{- 0.09 }$ \\[0.5ex] 
	(3, 1) & -- & $5.22^{+ 1.70 }_{- 1.70 }$ & $3.55^{+ 0.66 }_{- 0.66 }$ & $5.26^{+ 0.89 }_{- 0.89 }$ & $3.32^{+ 0.35 }_{- 0.35 }$ & $2.14^{+ 0.22 }_{- 0.22 }$ & $1.57^{+ 0.14 }_{- 0.14 }$ & $2.32^{+ 0.23 }_{- 0.23 }$ \\[0.5ex] 
	(3, 2) & -- & $1.11^{+ 0.54 }_{- 0.54 }$ & $2.10^{+ 0.80 }_{- 0.80 }$ & $1.96^{+ 0.63 }_{- 0.63 }$ & $2.49^{+ 0.58 }_{- 0.58 }$ & $1.32^{+ 0.40 }_{- 0.40 }$ & $0.77^{+ 0.22 }_{- 0.22 }$ & $1.13^{+ 0.46 }_{- 0.46 }$ \\[0.5ex] 
	(3, $\ge3$) & -- & -- & -- & -- & $4.38^{+ 4.56 }_{- 4.56 }$ & -- & -- & -- \\[0.5ex] 
	(4, 0) & -- & -- & $4.62^{+ 1.05 }_{- 1.05 }$ & $8.17^{+ 1.21 }_{- 1.21 }$ & $5.81^{+ 0.34 }_{- 0.34 }$ & $4.18^{+ 0.20 }_{- 0.20 }$ & $2.66^{+ 0.10 }_{- 0.10 }$ & $2.56^{+ 0.10 }_{- 0.10 }$ \\[0.5ex] 
	(4, 1) & -- & -- & $4.96^{+ 2.99 }_{- 2.99 }$ & $6.20^{+ 1.44 }_{- 1.44 }$ & $3.70^{+ 0.46 }_{- 0.46 }$ & $2.29^{+ 0.27 }_{- 0.27 }$ & $1.92^{+ 0.20 }_{- 0.20 }$ & $2.16^{+ 0.21 }_{- 0.21 }$ \\[0.5ex] 
	(4, 2) & -- & -- & $5.41^{+ 4.16 }_{- 4.16 }$ & $1.36^{+ 0.64 }_{- 0.64 }$ & $2.42^{+ 0.69 }_{- 0.69 }$ & $1.89^{+ 0.50 }_{- 0.50 }$ & $0.92^{+ 0.24 }_{- 0.24 }$ & $1.93^{+ 0.44 }_{- 0.44 }$ \\[0.5ex] 
	(4, $\ge3$) & -- & -- & -- & $1.85^{+ 2.28 }_{- 2.28 }$ & $2.80^{+ 2.82 }_{- 2.82 }$ & $7.31^{+ 7.31 }_{- 7.31 }$ & $1.79^{+ 1.61 }_{- 1.61 }$ & $0.06^{+ 0.05 }_{- 0.05 }$ \\[0.5ex] 
	($\ge5$, 0) & -- & -- & -- & $17.09^{+ 6.72 }_{- 6.72 }$ & $6.31^{+ 0.95 }_{- 0.95 }$ & $4.80^{+ 0.45 }_{- 0.45 }$ & $3.23^{+ 0.19 }_{- 0.19 }$ & $2.55^{+ 0.10 }_{- 0.10 }$ \\[0.5ex] 
	($\ge5$, 1) & -- & -- & -- & $14.04^{+ 14.48 }_{- 14.48 }$ & $4.03^{+ 1.06 }_{- 1.06 }$ & $3.28^{+ 0.71 }_{- 0.71 }$ & $2.24^{+ 0.27 }_{- 0.27 }$ & $1.68^{+ 0.14 }_{- 0.14 }$ \\[0.5ex] 
	($\ge5$, 2) & -- & -- & -- & $2.03^{+ 2.52 }_{- 2.52 }$ & $1.46^{+ 0.55 }_{- 0.55 }$ & $1.27^{+ 0.35 }_{- 0.35 }$ & $1.25^{+ 0.31 }_{- 0.31 }$ & $1.02^{+ 0.18 }_{- 0.18 }$ \\[0.5ex] 
	($\ge5$, $\ge3$) & -- & -- & -- & -- & $0.61^{+ 0.61 }_{- 0.61 }$ & $2.37^{+ 2.80 }_{- 2.80 }$ & $2.08^{+ 1.49 }_{- 1.49 }$ & $1.54^{+ 0.85 }_{- 0.85 }$ \\[0.5ex] 
	\hline
	\hline
\end{tabular}}
\end{table}

\begin{table}[h!]
\tiny
\centering
\caption{Transfer factors from the \mmj control region to the \zInv~ background for asymmetric categories.\label{tab:tf_mumu_zinv_asym}}
\scalebox{0.85}{\begin{tabular}{ccccccccc}
	\hline\hline
	& \multicolumn{8}{c}{\scalht (\gev)} \\ 
	 (\njet,  \nb) & 200-250 & 250-300 & 300-350 & 350-400 & 400-500 & 500-600 & 600-800 & 800-$\infty$ \\ [0.8ex] 
\hline
	(2a, 0) & $5.88\pm 0.16$ & $3.31\pm 0.12$ & $2.63\pm 0.13$ & $2.12\pm 0.13$ & $2.47\pm 0.13$ & $1.86\pm 0.14$ & $1.84\pm 0.16$ & -- \\[0.5ex] 
	(2a, 1) & $3.99\pm 0.33$ & $2.47\pm 0.28$ & $1.69\pm 0.26$ & $2.42\pm 0.49$ & $2.07\pm 0.32$ & $1.98\pm 0.58$ & -- & -- \\[0.5ex] 
	(2a, 2) & $3.04\pm 0.63$ & $1.63\pm 0.51$ & $7.30\pm 3.08$ & $9.91\pm 10.07$ & $1.17\pm 0.81$ & -- & -- & -- \\[0.5ex] 
	(3a, 0) & $7.86\pm 0.49$ & $5.31\pm 0.27$ & $5.61\pm 0.39$ & $3.47\pm 0.27$ & $2.14\pm 0.13$ & $1.48\pm 0.16$ & $1.34\pm 0.15$ & -- \\[0.5ex] 
	(3a, 1) & $5.56\pm 0.83$ & $3.78\pm 0.46$ & $3.22\pm 0.47$ & $2.17\pm 0.39$ & $1.52\pm 0.24$ & $0.43\pm 0.15$ & $1.18\pm 0.35$ & -- \\[0.5ex] 
	(3a, 2) & $3.08\pm 0.86$ & $2.15\pm 0.53$ & $3.05\pm 0.79$ & $3.05\pm 1.13$ & $1.75\pm 0.59$ & $1.46\pm 0.85$ & -- & -- \\[0.5ex] 
	(3a, $\ge3$) & $1.06\pm 1.33$ & $1.10\pm 1.25$ & -- & -- & -- & -- & -- & -- \\[0.5ex] 
	(4a, 0) & $1.89\pm 1.31$ & $4.34\pm 0.65$ & $7.32\pm 0.83$ & $7.17\pm 0.82$ & $5.18\pm 0.46$ & $2.49\pm 0.38$ & $1.19\pm 0.25$ & -- \\[0.5ex] 
	(4a, 1) & $1.30\pm 1.43$ & $5.14\pm 1.89$ & $3.84\pm 0.83$ & $4.93\pm 1.18$ & $4.14\pm 0.62$ & $2.03\pm 0.64$ & $0.79\pm 0.34$ & -- \\[0.5ex] 
	(4a, 2) & -- & $10.86\pm 5.46$ & $3.25\pm 1.38$ & $6.05\pm 2.19$ & $3.70\pm 1.28$ & $0.46\pm 0.32$ & $0.06\pm 0.05$ & -- \\[0.5ex] 
	(4a, $\ge3$) & -- & $1332.61\pm 1844.69$ & $3.00\pm 2.92$ & -- & -- & -- & -- & -- \\[0.5ex] 
	($\ge5$a, 0) & -- & -- & $6.17\pm 2.52$ & $11.59\pm 3.27$ & $7.60\pm 1.32$ & $3.83\pm 0.60$ & $2.83\pm 0.57$ & -- \\[0.5ex] 
	($\ge5$a, 1) & -- & -- & $6.88\pm 3.75$ & $2.83\pm 1.04$ & $3.25\pm 0.79$ & $4.80\pm 1.65$ & $1.43\pm 0.66$ & -- \\[0.5ex] 
	($\ge5$a, 2) & -- & -- & $12.08\pm 12.96$ & $9.24\pm 6.67$ & $3.89\pm 1.52$ & $2.68\pm 1.42$ & $0.37\pm 0.39$ & -- \\[0.5ex] 
	($\ge5$a, $\ge3$) & -- & -- & -- & -- & $6.04\pm 5.54$ & $49.96\pm 67.51$ & -- & -- \\[0.5ex] 
	\hline
	\hline
\end{tabular}}
\end{table}

\begin{table}[h!]
\tiny
\centering
\caption{Transfer factors from the \mmj control region to the \zInv~ background for monojet categories. The letter ``a'' in jet \eg ``2a''  indicates the asymmetric jet bins. All entries are non-zero but are truncated to one decimal place.\label{tab:tf_mumu_zinv_mono}}
\begin{tabular}
{ccccccccc}
	\hline\hline
&	& \multicolumn{8}{c}{\scalht (\gev)} \\ 
	 (\njet,  \nb) & 200-250 & 250-300 & 300-350 & 350-400 & 400-500 & 500-600 & 600-800 & 800-$\infty$ \\ [0.8ex] 
\hline
	(1, 0) & $7.87^{+ 0.16 }_{- 0.16 }$ & $6.98^{+ 0.22 }_{- 0.22 }$ & $6.65^{+ 0.30 }_{- 0.30 }$ & $6.01^{+ 0.34 }_{- 0.34 }$ & $5.75^{+ 0.25 }_{- 0.25 }$ & $5.71^{+ 0.29 }_{- 0.29 }$ & $5.18^{+ 0.31 }_{- 0.31 }$ & -- \\[0.5ex] 
	(1, 1) & $6.24^{+ 0.56 }_{- 0.56 }$ & $5.29^{+ 0.68 }_{- 0.68 }$ & $4.35^{+ 0.79 }_{- 0.79 }$ & $4.61^{+ 1.07 }_{- 1.07 }$ & $6.55^{+ 1.09 }_{- 1.09 }$ & $6.36^{+ 1.52 }_{- 1.52 }$ & $4.86^{+ 1.12 }_{- 1.12 }$ & -- \\[0.5ex] 
	\hline
	\hline
\end{tabular}
\end{table}


\clearpage

\subsection{Adding the \mht dimension}

The aforementioned description of the TFs provide an estimate of the
total SM background as a function of the (\njet,\nb,\HT) bin that is
integrated over \mht. However, the analysis takes advantage of \mht
distribution obtained from simulation. This information is propagated
to the likelihood model via an \mht template per (\njet,\nb,\HT) bin,
which is equivalent to dicing the numerator of the TF according to
\mht, \ie $N_{\rm MC}^{\rm signal}(\njet,\nb,\scalht,\mht)$. In this
regard, the TFs described above provide an estimate of the
normalisation for each \mht template.

\subsection{Data control samples used in the method}

To estimate the contributions from these backgrounds, three data
control regions are used, which are binned identically to the signal
region: \mj, \mmj and \gj.  Their definitions are provided
in Sec.~\ref{sec:selection}. The \ej and \eej control regions are not
used now, but are investigated to be used analogously to the \mj and \mmj regions in
the future. The selection criteria for these
control regions are defined such that any potential contamination from
new physics processes or QCD multijets is negligible.

In previous versions of this analysis, the \mmj and \gj control
samples are used to predict the \znunu +jets background. We plan to
extend this approach by relying on all (and not just a sub-set of)
relevant control samples to predict the two dominant components of the
total SM background (\wj and \ttbar, or \znunu + jets). Specifically,
we are using the \mj sample to predict the \wj and \ttbar backgrounds
(across all \nb bins) and up to three samples comprising \zmmj,
\gj and \wmj to predict the \znunu + jets background for events
containing exactly zero or one b-tagged jets. Any correlations are
appropriately handled by the likelihood model (via the
\texttt{Combine} tool).

The predictions of the \znunu + jets background based on the \zmmj
control samples exhibits significantly larger statistical
uncertainties at high \njet, \nb, \scalht, or \mht due to lower event
counts arising from the lower Z cross section (w.r.t. \gj and
\wj). Regardless, these samples are included in the likelihood fit
to provide additional confidence in the control of the \znunu + jets
background.

Concerning the use of $W$-enriched samples to predict the \znunu +jets
background, we have studied this approach
in detail with the 8\TeV dataset. Based on the outcome of these studies,
we have decided to proceed with this approach, as part
of the baseline likelihood description. Studies were
based on data-driven tests with 8\TeV data
(as described in Sec.~\ref{sec:closure-tests}). Closure tests are a critical
tool to determine which samples can be used to predict the SM
background components. In particular, we have studied the effect of
new closure tests designed to test the $W$-enriched to $Z$-enriched
extrapolation, specifically \mj to \gj, \mj to \mmj
and $\mu^{+}$ to $\mu^{-}$ closure tests. This study is detailed in Appendix~\ref{app:zInvBgControl}.
If further studies of the closure tests with $13\tev$
data suggest that using the \mj control region to predict the \znunu
background is not feasible, we will revert back to the approach
used in Run~I analysis (\ie relying solely on the \zll and \gj
samples). Early investigations suggest there are not any major problems.
%The closure tests provide important event samples for probing the
%accuracy of the simulation modelling implicit in the transfer factors.
%Specific examples include ``\mj to predict \mmj'' and ``zeej to
%predict \gj'' with events containing exactly zero or one b-tagged
%jets. The former test relies on a \wmj-enriched sample to predict
%yields in the \zmmj sample (in the presence of some ``\ttbar
%contamination'' for events with $\nb = 1$). The latter test is a
%consistency check between a dilepton and a \gj sample (as done in
%previous iterations of the analysis).
Most importantly, we retain full flexibility in our approach and the
control samples used to predict the various background components.

Currently, the projected sensitivity, as described in the current
version of this note, is based on predictions of SM background
components made from control regions as follows. For events containing
exactly zero or one b-tagged jets, the \mj (enriched in \wej), \gj and
\mmj control samples are used to estimate the irreducible \znunu + jets
background, while the \mj control sample is used to estimate all
remaining SM processes (predominately \wj and \ttbar). For events
containing two or more b-tagged jets, the \mj sample is
used to predict the total SM background (dominated by \ttbar).

