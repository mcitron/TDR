%%____________________________________________________________________________||
\section{Physics objects}
\label{sec:objects}
The definitions of the physics objects used in this analysis follow the recommendations of the various Physics Object Groups (POGs). 
During data taking these recommendations are subject to change and will be be updated if necessary.

\subsection{Jets}
\label{sec:jetreco}
Jets are defined as sets of particle-flow (PF) candidates clustered by the
anti-$k_{T}$ jet clustering algorithm \cite{Cacciari:2008gp} with a distance parameter of 0.4
(PFJets). Charged Hadron Subtraction (CHS) is applied, i.e., charged
hadrons that can be traced back to pileup vertices are not clustered.
The four-momenta of jets are initially defined as the four-vector sum of
the four-momenta of the constituent particle-flow candidates and then
scaled by the jet energy correction factors designated as L1FastJet,
L2Relative, and L3Absolute \cite{Chatrchyan:2011ds}.

The ``loose'' working point Jet-Id selection criteria is chosen. 
The cuts are listed in Tab.~\ref{tab:loose-jet-id}. 
In addition, a dedicated selection is applied to reject ``beam halo'' candidate events, 
as described thoroughly in section \ref{sec:had-signal}.

\begin{table}[ht!]
  \caption{The ``loose'' jet ID requirements. \label{tab:loose-jet-id}}
  \centering
  \begin{tabular}{ ccc }
    \hline
    \hline
    Variable & cut & notes \\ \hline
    \multicolumn{3}{c}{$-3.0 < \eta_{\mathrm{jet}} < 3.0$} \\ \hline    
    Neutral Hadron Fraction & $<0.99$ & - \\
    Neutral EM Fraction & $<0.99$ & - \\
    Number of constituents & $>1$ & - \\
    Charged Hadron Fraction & $>0$ & only for $|\eta_{\mathrm{jet}}| < 2.4$ \\
    Charged Multiplicity & $>0$ & only for $|\eta_{\mathrm{jet}}| < 2.4$ \\
    Charged EM Fraction & $<0.99$ & only for $|\eta_{\mathrm{jet}}| < 2.4$ \\ \hline
    \multicolumn{3}{c}{$|\eta_{\mathrm{jet}}| > 3.0$} \\ \hline        
    Neutral EM Fraction & $<0.90$ & - \\
    Number of Neutral Particles & $>10$ & - \\
    \hline
    \hline
  \end{tabular}
\end{table}

\subsection{b-tagged jets}
\label{sec:btags}
Jets originating from bottom quarks are identified through vertices that are displaced with respect to the primary interaction \cite{Chatrchyan:2012jua}. The algorithm used to tag b-jets is the Combined Secondary Vertex tagger V2, using the ``medium'' working point, which is achieved by requiring a cut of $>$ 0.80 on the algorithm discriminator variable. 
This results in a gluon/light-quark mis-tag rate of $\sim$1 \% (where ``light'' means $u$, $d$ and $s$ quarks) with a b-tag signal efficiency of about 80 \%. 


\subsection{Muons}
\label{sec:muon-id}
For the purpose of vetoing muons in the signal region, the ``loose'' working point 
definition of the recommended identification algorithm is used, which provides $\sim$ 98 $\%$ efficiency. 
Muons are also required to be well isolated, i.e. with a low activity in the vicinity of their track. 
The transverse momenta of PF neutral and charged candidates, as well as photons, lying within a cone around the lepton are summed. 
The relative combined isolation $I^{rel}_{comb}$ is then defined as 
the ratio of this scalar sum to the transverse momentum of the lepton
candidate. Additionally, $\rho\times A_{eff}$ corrections are applied to
remove the effects of pileup.

In the hadronic signal region a variable cone size for the isolation is used, which is referred to as ``mini-isolation''. 
This isolation algorithm helps in recovering some efficiency in the lepton selection for boosted topology of top quark decays, 
in which the muon's track may be found close to the jet activity due to the boost of the parent top. 
Therefore, the cone size used for the calculation of the lepton isolation is reduced as a function of 
the lepton \Pt, as follows: $R=0.2$ for $\Pt_{\ell}\leq50\gev$,
$R=10\gev/\Pt_{\ell}$ for $50 \gev < \Pt_{\ell} < 200\gev$ and $R=0.05$ for $\Pt_{\ell} > 200 \gev$.
In the signal region, identified muons with mini-isolation satisfying $I^{rel}_{comb} < 0.2$ are vetoed.

In the \mj and \mmj samples, the standard PF-based relative isolation is used, with a cone size of 0.4. 
Muons are defined to be isolated if they fulfill the criterium $I^{rel}_{comb} < 0.12$. 
Additionally muons are required to satisfy the ``tight'' working
point definition of the recommended identification algorithm. This
slightly reduces the efficiency but additionally reduces the fake
rate.

Muons are vetoed in the definition of the hadronic signal region, 
as described in Section \ref{sec:preSelection}, while 
control regions with one muon (``\mj'') or two muons (``\mmj'') are defined for the purpose of the background estimation, 
as described in Sec.~\ref{subsec:mucontrolSelection}.


\subsection{Photons}
\label{sec:photon-id}
Photons are identified according to the ``tight'' working point definition ($\sim$ 71 $\%$ efficiency) 
of the simple cut-based photon identification algorithm \cite{photon-id} 
and required to be well isolated. 
A PF-based isolation is used with a cone size $\Delta R$ $<$ 0.3 and
$\rho\times A_{eff}$ corrections are applied to remove the effects of pileup \cite{pf-photon}. 
Table \ref{tab:photon-id-gamma} summarises the identification and isolation selection used. 

Photons are vetoed in the definition of the hadronic signal region, 
as described in Sec.~\ref{sec:preSelection}, while a 
control region with one photon (``\gj'') is defined for the purpose of the background estimation, 
as described in Sec.~\ref{subsec:photoncontrolSelection}.


\begin{table}[ht!]
  \caption{Photon identification (``tight'' working point).\label{tab:photon-id-gamma}}
  \centering
  \footnotesize
  \begin{tabular}{ ccc }
    \hline
    \hline
    Categories                    & Barrel                             & EndCap                             \\
    \hline
    Conversion safe electron veto & Yes                                & Yes                                \\
    Single Tower H/E              & 0.01                               & 0.015                               \\
    $\sigma_{i\eta i\eta}$        & 0.01                               & 0.0265                               \\
    PF charged hadron isolation   & 1.66                               & 1.04                               \\
    PF neutral hadron isolation   & 0.14 + $ e^{0.0028 \times \pt^{\gamma} + 0.5408}$  &  3.89 + 0.0172 $\times$ $\pt^{\gamma}$\\
    PF photon isolation           & 1.40 + 0.0014 $\times$ $\pt^{\gamma}$ & 1.40 + 0.0091 $\times$ $\pt^{\gamma}$ \\
    \hline
    \hline
  \end{tabular}
  \end{table}


\subsection{Electrons}
\label{sec:electron-id}
Electrons are identified according to the ``loose'' working point definition ($\sim$ 90 $\%$ efficiency) 
of the cut-based identification \cite{electron-id} for the purpose of the electron veto in the signal region.

Electrons are also require required to be well isolated. 
Similar to muons, in the hadronic signal regions, a PF-based isolation
\cite{pf-photon} is used with a cone size determined by the mini
isolation algorithm (see Sec.~\ref{sec:muon-id}) and $\rho\times A_{eff}$ 
corrections are applied to remove the effects of pileup.
Isolated electrons are defined by $I^{rel}_{comb} < 0.1$. 

Table \ref{tab:ele-id} summarises the identification 
selection used. 

Electrons are vetoed in the definition of the hadronic signal region, 
as described in Sec.~\ref{sec:preSelection}.


\begin{table}[h!]
  \caption{Electron identification (``tight'' working point).\label{tab:ele-id}}
  \centering
  \footnotesize
  \begin{tabular}{ lcc }
    \hline
    \hline
    Categories                                               & Barrel    & EndCap    \\
    \hline
    $\Delta \eta_{In}$                                       & 0.0105   & 0.00814  \\
    $\Delta \phi_{In}$                                       & 0.115    & 0.182  \\
    $\sigma_{i\eta i\eta}$                                   & 0.0103    & 0.0301  \\
    H/E                                                      & 0.104    & 0.0897   \\
    d0 (vtx)                                                 & 0.0261    & 0.118  \\
    dZ (vtx)                                                 & 0.041    & 0.822  \\
    $\lvert(1/E_{\textrm{ECAL}} - 1/p_{\textrm{trk}})\rvert$ & 0.102     & 0.126  \\
    Missing hits (inner tracker)                             & 2         & 1         \\
    Conversion veto                                          & yes       & yes   \\
    \hline
    \hline
  \end{tabular}
  \end{table}


\subsection{Single isolated tracks}
\label{sec:SIT}

A single isolated track (SIT) can be used to identify W bosons through their leptonic decays: 
W $\rightarrow$ $\mu \nu$, W $\rightarrow$ $e\nu$, and W $\rightarrow$ $\tau$($\rightarrow l$) $\nu$. 
Single prong decays of the tau lepton can also be identified: W $\rightarrow$ $\tau$ ($\rightarrow$ h$^{\pm}$ + n$\pi^{0}$) $\nu$. 
A single isolated track comprises a charged PF candidate with $\Pt > 10 \gev$, $\Delta z(\mathrm{track}, \mathrm{PV}) < 0.05 \, \mathrm{cm}$ 
and with a relative isolation smaller than 0.1, where the isolation is determined from the sum 
of the \Pt of the charged PF candidates within $\Delta R < 0.3$.

SIT are vetoed in the definition of the hadronic signal region, 
as described in Sec.~\ref{sec:preSelection}.


\subsection{Missing transverse momentum}
Missing transverse momentum (\met) is defined as the negative vector sum
of the transverse momentum of all particle-flow candidates in the event.
The Type-I \met correction \cite{Khachatryan:2014gga} is applied, i.e., the transverse momentum of
the particle-flow candidates clustered as jets are replaced with the
transverse momentum of the jets that are scaled by the jet energy
correction factors.

The \met is used in the definition of 
the transverse mass, $M_{T}$, which is in turn used as part of
the selection criteria that define the single muon control sample 
(Sec.~\ref{subsec:mucontrolSelection}), and for the $\mhtmet$ cleaning filter, 
as described in Sec.~\ref{sec:selection}.



%%____________________________________________________________________________||
