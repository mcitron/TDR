%%____________________________________________________________________________||
\section{Systematic uncertainties in the transfer factors}
\label{sec:systematics}

This section addresses the estimation of systematic uncertainties
affecting the transfer factors (equation~\ref{equ:tf-ratio})
for non-multijet backgrounds. 
These uncertainties will be often referred to as \textit{``normalisation uncertainties''}, 
as opposed to the \textit{``template uncertainties''} described in Sec.~\ref{sec:syst-on-shape}. 
The former affect the total number of events in each (\njet,~\nb,~\scalht) bin (integrating over \mht), 
while the latter encode the limited knowledge on how these events distribute in the \mht dimension.

Two approaches are used to derive uncertainties from specific sources:
one is based on variations in simulation (Sec.~\ref{sec:mc-variations}), the other makes use of control samples in data (Sec.~\ref{sec:closure-tests}).
Each systematic source considered in the analysis is described below, 
together with the method used to derive the corresponding uncertainties and the correlation model. 
A summary of the all the uncertainties is given in Tab.~\ref{tab:systs}.



%% %%%%%%%%%%%%%%%%%%%%%%%%%%%%%%%%%%%%%%%%%%%%%%%%%%%%%%%%%%%%%%%%
%% % MC-based systematics
%% %%%%%%%%%%%%%%%%%%%%%%%%%%%%%%%%%%%%%%%%%%%%%%%%%%%%%%%%%%%%%%%%
\subsection{Systematic variations in simulation}
\label{sec:mc-variations}
A set of corrections are applied to the MC samples in order to improve the modelling 
of detector effects (b-tagging efficiency, jet energy response, etc.) 
as well as the simulation of the kinematics of certain processes (top $p_{T}$). 
These corrections are described in Sec.~\ref{sec:datasets}. \\
In this section the effect of varying these corrections within their uncertainties is 
presented, focusing on the relative change in the 4 type of transfer factors which are of interest 
for the background prediction, namely: $\mj \rightarrow (\znunu)$,
$\mmj \rightarrow (\znunu)$, $\gj \rightarrow (\znunu)$ and $\mj
\rightarrow \mathrm{\ttbar+W}$. 
The binning of the analysis is chosen in order to minimise the impact of 
these systematic sources, which are expected to be sub-dominant. 
However, they are propagated to the final results, taking into account correlations and bin migration effects.


\subsubsection*{Jet energy scale}
\label{sec:tfSyst_jec}
The effect of varying the jet energy scale in
the \mj and \mmj control regions is investigated.  The energies of
jets used in the analysis are corrected as a function of their \pt and
$\eta$ via the procedure recommended by the JetMET POG. These
corrections have an associated uncertainty, which is propagated through the analysis. 
As the \scalht and jet multiplicity binning is mirrored in signal and control regions, 
the effect of jet energy scale on the transfer factor is expected to be small. 
However, the jet energy scale can still have an
effect due to jets moving in and out acceptance (above and below
$40\gev$). The relative change in the transfer factors is presented as a function of \scalht and jet category 
in Fig. ~\ref{fig:tfSyst_jec_muToZinv}-\ref{fig:tfSyst_jec_muToTtw}.
The changes are typically in the range of $1-15\%$.


\begin{figure}[!h]
  \centering
  \subfigure[JEC up variation]{
    \includegraphics[width=0.5\textwidth]{figures/mcSystematics6p26Fb/Zinv/mu/ratiotfh_ht_mht_alljecWeight_Up.pdf}
  } ~~
  \subfigure[JEC down variation]{
    \includegraphics[width=0.5\textwidth]{figures/mcSystematics6p26Fb/Zinv/mu/ratiotfh_ht_mht_alljecWeight_Down.pdf}
  }\\

  \caption{\label{fig:tfSyst_jec_muToZinv} The relative change in the
  $\mj \rightarrow (\znunu)$ transfer
  factors when varying JEC in MC within its uncertainties, as a function of \scalht and jet category. 
  Variations corresponding to $+1\sigma$ ($-1\sigma$) are shown in the left (right) figure. 
  }
\end{figure}

\begin{figure}[!h]
  \centering
  \subfigure[JEC up variation]{
    \includegraphics[width=0.5\textwidth]{figures/mcSystematics6p26Fb/Zinv/mumu/ratiotfh_ht_mht_alljecWeight_Up.pdf}
  } ~~
  \subfigure[JEC down variation]{
    \includegraphics[width=0.5\textwidth]{figures/mcSystematics6p26Fb/Zinv/mumu/ratiotfh_ht_mht_alljecWeight_Down.pdf}
  }\\

  \caption{\label{fig:tfSyst_jec_mumuToZinv} The relative change in
  the $\mmj \rightarrow (\znunu)$ transfer
  factors when varying JEC in MC within its uncertainties, as a function of \scalht and jet category. 
  Variations corresponding to $+1\sigma$ ($-1\sigma$) are shown in the left (right) figure. 
  }
\end{figure}

\begin{figure}[!h]
  \centering
  \subfigure[JEC up variation]{
    \includegraphics[width=0.5\textwidth]{figures/mcSystematics6p26Fb/Zinv/gj/ratiotfh_ht_mht_alljecWeight_Up.pdf}
  } ~~
  \subfigure[JEC down variation]{
    \includegraphics[width=0.5\textwidth]{figures/mcSystematics6p26Fb/Zinv/gj/ratiotfh_ht_mht_alljecWeight_Down.pdf}
  }\\

  \caption{\label{fig:tfSyst_jec_gjToZinv} The relative change in the
  $\gj \rightarrow (\znunu)$ transfer
  factors when varying JEC in MC within its uncertainties, as a function of \scalht and jet category. 
  Variations corresponding to $+1\sigma$ ($-1\sigma$) are shown in the left (right) figure. 
  }
\end{figure}

\begin{figure}[!h]
  \centering
  \subfigure[JEC up variation]{
    \includegraphics[width=0.5\textwidth]{figures/mcSystematics6p26Fb/Ttw/mu/ratiotfh_ht_mht_alljecWeight_Up.pdf}
  } ~~
  \subfigure[JEC down variation]{
    \includegraphics[width=0.5\textwidth]{figures/mcSystematics6p26Fb/Ttw/mu/ratiotfh_ht_mht_alljecWeight_Down.pdf}
  }\\

  \caption{\label{fig:tfSyst_jec_muToTtw} The relative change in the
  $\mj \rightarrow \mathrm{\ttbar+W}$ transfer
  factors when varying JEC in MC within its uncertainties, as a function of \scalht and jet category. 
  Variations corresponding to $+1\sigma$ ($-1\sigma$) are shown in the left (right) figure. 
  }
\end{figure}




\subsubsection*{B-tagging efficiency}
\label{sec:tfSyst_btag}
\textbf{This section include results from 2015 data, as the scale factors for 2016 have not yet been provided.}\\
Scale factors provided by the BTV POG are applied to the MC samples
to correct for differences in the b-tagging efficiencies and 
misidentifications between simulation and data. 
The method employed is
based on simple event reweighting as described in
Ref.~\cite{btagSFMethods}. As the scale factors for the data collected in
2016 are not yet available the systematics shown below are based 
on those from 2015. These uncertainties are not applied for the results in \ref{sec:results},
but will be rederived and included when the necessary scale factors are available.
Events are reweighted according to the probability of obtaining a particular jet configuration in data
and simulation, as determined by the b-tagging efficiencies computed
in the MC samples and the scale factors measured in data.
Since no extrapolation is performed in the background prediction across different 
\nb  multiplicities, the analysis is expected to be robust against variation in the 
b-tag efficiency. 
To test this effect the change in the transfer factors is measured
by varying the scale factors within their uncertainties. 
The relative change in the transfer factors is presented as a function of \scalht and jet category 
in Fig. ~\ref{fig:tfSyst_bsf_muToZinv}-\ref{fig:tfSyst_bsf_muToTtw}.
They are typically in the range of $1-5\%$.


\begin{figure}[!h]
  \centering
  \subfigure[b-tag SF up variation]{
    \includegraphics[width=0.5\textwidth]{figures/mcSystematics6p26Fb/Zinv/mu/ratiotfh_ht_mht_allbsfWeight_Up.pdf}
  } ~~
  \subfigure[b-tag SF down variation]{
    \includegraphics[width=0.5\textwidth]{figures/mcSystematics6p26Fb/Zinv/mu/ratiotfh_ht_mht_allbsfWeight_Down.pdf}
  }\\

  \caption{\label{fig:tfSyst_bsf_muToZinv} The relative change in the
  $\mj \rightarrow (\znunu)$ transfer
  factors when varying b-tag SF in MC within its uncertainties, as a function of \scalht and jet category. 
  Variations corresponding to $+1\sigma$ ($-1\sigma$) are shown in the left (right) figure. 
  }
\end{figure}

\begin{figure}[!h]
  \centering
  \subfigure[b-tag SF up variation]{
    \includegraphics[width=0.5\textwidth]{figures/mcSystematics6p26Fb/Zinv/mumu/ratiotfh_ht_mht_allbsfWeight_Up.pdf}
  } ~~
  \subfigure[b-tag SF down variation]{
    \includegraphics[width=0.5\textwidth]{figures/mcSystematics6p26Fb/Zinv/mumu/ratiotfh_ht_mht_allbsfWeight_Down.pdf}
  }\\

  \caption{\label{fig:tfSyst_bsf_mumuToZinv} The relative change in
  the $\mmj \rightarrow (\znunu)$ transfer
  factors when varying b-tag SF in MC within its uncertainties, as a function of \scalht and jet category. 
  Variations corresponding to $+1\sigma$ ($-1\sigma$) are shown in the left (right) figure. 
  }
\end{figure}

\begin{figure}[!h]
  \centering
  \subfigure[b-tag SF up variation]{
    \includegraphics[width=0.5\textwidth]{figures/mcSystematics6p26Fb/Zinv/gj/ratiotfh_ht_mht_allbsfWeight_Up.pdf}
  } ~~
  \subfigure[b-tag SF down variation]{
    \includegraphics[width=0.5\textwidth]{figures/mcSystematics6p26Fb/Zinv/gj/ratiotfh_ht_mht_allbsfWeight_Down.pdf}
  }\\

  \caption{\label{fig:tfSyst_bsf_gjToZinv} The relative change in the
  $\gj \rightarrow (\znunu)$ transfer
  factors when varying b-tag SF in MC within its uncertainties, as a function of \scalht and jet category. 
  Variations corresponding to $+1\sigma$ ($-1\sigma$) are shown in the left (right) figure. 
  }
\end{figure}

\begin{figure}[!h]
  \centering
  \subfigure[b-tag SF up variation]{
    \includegraphics[width=0.5\textwidth]{figures/mcSystematics6p26Fb/Ttw/mu/ratiotfh_ht_mht_allbsfWeight_Up.pdf}
  } ~~
  \subfigure[b-tag SF down variation]{
    \includegraphics[width=0.5\textwidth]{figures/mcSystematics6p26Fb/Ttw/mu/ratiotfh_ht_mht_allbsfWeight_Down.pdf}
  }\\

  \caption{\label{fig:tfSyst_bsf_muToTtw} The relative change in the $\mj \rightarrow \mathrm{tt+W}$ transfer
  factors when varying b-tag SF in MC within its uncertainties, as a function of \scalht and jet category. 
  Variations corresponding to $+1\sigma$ ($-1\sigma$) are shown in the left (right) figure. 
  }
\end{figure}





\subsubsection*{Lepton and photon trigger/identification/isolation efficiency}
\label{sec:tfSyst_lepton}
\textbf{This section include results from 2015 data, as the scale factors for 2016 have not yet been provided.}\\
Leptons out of $p_{T}$ and $\eta$ acceptance, or within detector
acceptance but not identified properly by lepton identification or isolation
requirements contribute to the so-called ``lost-lepton background'', 
which mainly stem from W and \ttbar events. 
The fraction of events with leptons out of acceptance ($f_{sample}$)
is calculated from generator truth level information for each MC
sample. Differences in efficiencies between data and simulation are
accounted for with data/MC scale
factors for trigger, lepton identification and isolation~\cite{twiki-leptonSF}. 
The uncertainties on the transfer factors associated 
with these scale factors are determined as described in the following.
These studies were caried out with data and MC from 2015 and will be updated 
for the final result. To cover issues in the efficiency of the
muon trigger a $5\%$ systematic is assigned for both the \mj and \mmj 
control regions as described in \ref{sec:disclaimers}. To cover 
issues in the photon trigger (as described in \ref{sec:disclaimers}) a $5\%$ is applied in the
\gj control region.

The effect of lepton reconstruction in the $\mj \rightarrow \mathrm{tt+W}$ 
transfer factors is factorised in the following expression: 
\begin{equation}
    \label{eq:lostLepTF}
    y = \frac{\sum_{sample} [ R_{sample} \times f_{sample} \times N^{GEN}_{sample} \times ( 1 - \epsilon_{Loose} ) + ( 1 - f_{sample} ) \times N^{GEN}_{sample} ]}{ \sum_{sample} N^{GEN}_{sample} \times \epsilon_{Tight} \times R_{sample} }
\end{equation}
where $R_{sample}$ is the cross section reweighting factor for each sample, 
$N^{GEN}_{sample}$ is the total MC events for the category, $\epsilon_{Tight}$
and $\epsilon_{Loose}$ are the lepton efficiency for Tight and Loose working 
point. The variable $y$ is computed for each category. For the numerator, full
signal selection except the lepton veto to mimic signal region phase space as
closely as possible. For the denominator, the full selection for the \mj 
control sample is applied.

The variation on the variable $y$ is computed by varying the lepton scale factor
up and down according to each source of uncertainty. 
A systematic uncertainty of 2\% on the muon trigger efficiency is taken. 
It is doubled with respect to the recommendations from the Muon POG, because the latter is derived using the 
``OR'' of the \verb!HLT_IsoMu20! and \verb!HLT_IsoTkMu20!, while the analysis makes use of the \verb!HLT_IsoMu20! path only. 
A ~1\% systematic is propagated for the ``tight'' muon selection. 
This uncertainty are taken as correlated between signal and control regions. 

Systematic uncertainties of 1\% and 5\% are assigned to the efficiency of the muon and electron veto working point efficiency and taken as correlated across all the signal region bins. 
Data/MC lepton scale factors have negligible statistical uncertainties. 
The up and down variations are added in quadrature respectively. 
The procedure is repeated separately for muons and
electrons. 

The relative change in the $\mj \rightarrow \mathrm{tt+W}$ transfer factors 
is presented in Fig.~\ref{fig:lostLepton}. The
total change is typically in the range $2-5\%$

\begin{figure}[!h]
  \centering
  \subfigure[Lepton scale factors varied up (muons)]{
    \includegraphics[width=0.5\textwidth]{figures/mcSystematics/lostLepton/muonUp.pdf}
  } ~~
  \subfigure[Lepton scale factors varied down (muons)]{
    \includegraphics[width=0.5\textwidth]{figures/mcSystematics/lostLepton/muonDown.pdf}
  }\\
  \subfigure[Scale factors varied up (electrons)]{
    \includegraphics[width=0.5\textwidth]{figures/mcSystematics/lostLepton/electronUp.pdf}
  } ~~
  \subfigure[Scale factors varied down (electrons)]{
    \includegraphics[width=0.5\textwidth]{figures/mcSystematics/lostLepton/electronDown.pdf}
  }\\
  \caption{\label{fig:lostLepton} 
    Relative uncertainties for the ``lost-lepton'' ($\mathrm{tt+W}$) background due to lepton efficiency. 
    The total uncertainty on the muons (electrons) is shown in the top (bottom) row.
  }
  
\end{figure}



\subsubsection*{Photon trigger uncertainty}
\label{sec:tfSyst_photonTrigger}

Variations in the trigger weight for the signal region are studied. A conservative systematic
uncertainty on this correction is taken as the size of the inefficiency. 
The relative change in the \gj transfer factor is presented in Fig.~\ref{fig:tfSyst_photonTrigger_gjToZinv}
variation is typically in the range $0-3\%$.


\begin{figure}[!h]
  \centering
  \subfigure[Photon trigger weight up variation]{
    \includegraphics[width=0.5\textwidth]{figures/mcSystematics6p26Fb/Zinv/gj/ratiotfh_ht_mht_allphotonTriggerWeight_Up.pdf}
  } ~~
  \subfigure[Photon trigger weight down variation]{
    \includegraphics[width=0.5\textwidth]{figures/mcSystematics6p26Fb/Zinv/gj/ratiotfh_ht_mht_allphotonTriggerWeight_Down.pdf}
  }\\

  \caption{\label{fig:tfSyst_photonTrigger_gjToZinv} The relative change in
  the $\gj \rightarrow (\znunu)$ transfer
  factors when varying photon trigger weight in MC within its uncertainties, as a function of \scalht and jet category. 
  Variations corresponding to $+1\sigma$ ($-1\sigma$) are shown in the left (right) figure. 
  }
\end{figure}

\subsubsection*{Signal trigger uncertainty}
\label{sec:tfSyst_topPt}

Variations in the trigger weight for the signal region are studied. A systematic is taken
as the difference in the efficiency measured using muon and electron reference triggers.
The relative change in transfer factors is presented in Fig.
~\ref{fig:tfSyst_trigger_muToZinv}-\ref{fig:tfSyst_trigger_muToTtw}. The
variation is typically in the range $0-5\%$.

\begin{figure}[!h]
  \centering
  \subfigure[trigger weight up variation]{
    \includegraphics[width=0.5\textwidth]{figures/mcSystematics6p26Fb/Zinv/mu/ratiotfh_ht_mht_alltriggerWeight_Up.pdf}
  } ~~
  \subfigure[trigger weight down variation]{
    \includegraphics[width=0.5\textwidth]{figures/mcSystematics6p26Fb/Zinv/mu/ratiotfh_ht_mht_alltriggerWeight_Down.pdf}
  }\\

  \caption{\label{fig:tfSyst_trigger_muToZinv} The relative change in
  the $\mj \rightarrow (\znunu)$ transfer
  factors when varying trigger weight in MC within its uncertainties, as a function of \scalht and jet category. 
  Variations corresponding to $+1\sigma$ ($-1\sigma$) are shown in the left (right) figure. 
  }
\end{figure}
\begin{figure}[!h]
  \centering
  \subfigure[trigger weight up variation]{
    \includegraphics[width=0.5\textwidth]{figures/mcSystematics6p26Fb/Zinv/mumu/ratiotfh_ht_mht_alltriggerWeight_Up.pdf}
  } ~~
  \subfigure[trigger weight down variation]{
    \includegraphics[width=0.5\textwidth]{figures/mcSystematics6p26Fb/Zinv/mumu/ratiotfh_ht_mht_alltriggerWeight_Down.pdf}
  }\\

  \caption{\label{fig:tfSyst_trigger_mumuToZinv} The relative change in
  the $\mmj \rightarrow (\znunu)$ transfer
  factors when varying trigger weight in MC within its uncertainties, as a function of \scalht and jet category. 
  Variations corresponding to $+1\sigma$ ($-1\sigma$) are shown in the left (right) figure. 
  }
\end{figure}

\begin{figure}[!h]
  \centering
  \subfigure[trigger weight up variation]{
    \includegraphics[width=0.5\textwidth]{figures/mcSystematics6p26Fb/Zinv/gj/ratiotfh_ht_mht_alltriggerWeight_Up.pdf}
  } ~~
  \subfigure[trigger weight down variation]{
    \includegraphics[width=0.5\textwidth]{figures/mcSystematics6p26Fb/Zinv/gj/ratiotfh_ht_mht_alltriggerWeight_Down.pdf}
  }\\

  \caption{\label{fig:tfSyst_trigger_gjToZinv} The relative change in
  the $\gj \rightarrow (\znunu)$ transfer
  factors when varying trigger weight in MC within its uncertainties, as a function of \scalht and jet category. 
  Variations corresponding to $+1\sigma$ ($-1\sigma$) are shown in the left (right) figure. 
  }
\end{figure}

\begin{figure}[!h]
  \centering
  \subfigure[trigger weight up variation]{
    \includegraphics[width=0.5\textwidth]{figures/mcSystematics6p26Fb/Ttw/mu/ratiotfh_ht_mht_alltriggerWeight_Up.pdf}
  } ~~
  \subfigure[trigger weight down variation]{
    \includegraphics[width=0.5\textwidth]{figures/mcSystematics6p26Fb/Ttw/mu/ratiotfh_ht_mht_alltriggerWeight_Down.pdf}
  }\\

  \caption{\label{fig:tfSyst_trigger_muToTtw} The relative change in the $\mj \rightarrow \mathrm{tt+W}$ transfer
  factors when varying trigger weight in MC within its uncertainties, as a function of \scalht and jet category. 
  Variations corresponding to $+1\sigma$ ($-1\sigma$) are shown in the left (right) figure. 
  }
\end{figure}
% \subsubsection*{Top $p_T$ reweighting}
% \label{sec:tfSyst_topPt}
%
% Variations in the reweighting of top $p_{T}$ distribution, as first outlined in 
% Sec.~\ref{sec:SMxs}, are studied. A conservative systematic
% uncertainty on this correction is taken as the size of the correction itself. 
% The relative change in transfer factors is presented in Fig.
% ~\ref{fig:tfSyst_topPt_muToZinv}-\ref{fig:tfSyst_topPt_muToTtw}. The
% variation is typically in the range $0-15\%$.
% \begin{figure}[!h]
%   \centering
%   \subfigure[top $p_{T}$ weight up variation]{
%     \includegraphics[width=0.5\textwidth]{figures/mcSystematics6p26Fb/Zinv/mu/ratiotfh_ht_mht_alltopPtWeight_Up.pdf}
%   } ~~
%   \subfigure[top $p_{T}$ weight down variation]{
%     \includegraphics[width=0.5\textwidth]{figures/mcSystematics6p26Fb/Zinv/mu/ratiotfh_ht_mht_alltopPtWeight_Down.pdf}
%   }\\
%
%   \caption{\label{fig:tfSyst_topPt_muToZinv} The relative change in
%   the $\mj \rightarrow (\znunu)$ transfer
%   factors when varying top $p_{T}$ weight in MC within its uncertainties, as a function of \scalht and jet category. 
%   Variations corresponding to $+1\sigma$ ($-1\sigma$) are shown in the left (right) figure. 
%   }
% \end{figure}
% \begin{figure}[!h]
%   \centering
%   \subfigure[top $p_{T}$ weight up variation]{
%     \includegraphics[width=0.5\textwidth]{figures/mcSystematics6p26Fb/Zinv/mumu/ratiotfh_ht_mht_alltopPtWeight_Up.pdf}
%   } ~~
%   \subfigure[top $p_{T}$ weight down variation]{
%     \includegraphics[width=0.5\textwidth]{figures/mcSystematics6p26Fb/Zinv/mumu/ratiotfh_ht_mht_alltopPtWeight_Down.pdf}
%   }\\
%
%   \caption{\label{fig:tfSyst_topPt_mumuToZinv} The relative change in
%   the $\mmj \rightarrow (\znunu)$ transfer
%   factors when varying top $p_{T}$ weight in MC within its uncertainties, as a function of \scalht and jet category. 
%   Variations corresponding to $+1\sigma$ ($-1\sigma$) are shown in the left (right) figure. 
%   }
% \end{figure}
%
% \begin{figure}[!h]
%   \centering
%   \subfigure[top $p_{T}$ weight up variation]{
%     \includegraphics[width=0.5\textwidth]{figures/mcSystematics6p26Fb/Zinv/gj/ratiotfh_ht_mht_alltopPtWeight_Up.pdf}
%   } ~~
%   \subfigure[top $p_{T}$ weight down variation]{
%     \includegraphics[width=0.5\textwidth]{figures/mcSystematics6p26Fb/Zinv/gj/ratiotfh_ht_mht_alltopPtWeight_Down.pdf}
%   }\\
%
%   \caption{\label{fig:tfSyst_topPt_gjToZinv} The relative change in
%   the $\gj \rightarrow (\znunu)$ transfer
%   factors when varying top $p_{T}$ weight in MC within its uncertainties, as a function of \scalht and jet category. 
%   Variations corresponding to $+1\sigma$ ($-1\sigma$) are shown in the left (right) figure. 
%   }
% \end{figure}
%
% \begin{figure}[!h]
%   \centering
%   \subfigure[top $p_{T}$ weight up variation]{
%     \includegraphics[width=0.5\textwidth]{figures/mcSystematics6p26Fb/Ttw/mu/ratiotfh_ht_mht_alltopPtWeight_Up.pdf}
%   } ~~
%   \subfigure[top $p_{T}$ weight down variation]{
%     \includegraphics[width=0.5\textwidth]{figures/mcSystematics6p26Fb/Ttw/mu/ratiotfh_ht_mht_alltopPtWeight_Down.pdf}
%   }\\
%
%   \caption{\label{fig:tfSyst_topPt_muToTtw} The relative change in the $\mj \rightarrow \mathrm{tt+W}$ transfer
%   factors when varying top $p_{T}$ weight in MC within its uncertainties, as a function of \scalht and jet category. 
%   Variations corresponding to $+1\sigma$ ($-1\sigma$) are shown in the left (right) figure. 
%   }
% \end{figure}


\subsubsection*{QCD contamination}
\label{sec:tfSyst_qcdCont}

A check has also been performed on the systematic effect on the
background prediction due to QCD contamination in the control samples,
which has been found to be at the ~5\% level for the \gj
control region. Applying an arbitrarily large variation of $\pm
100\%$ on the number of Monte Carlo QCD events leads to a systematic
variation on the transfer factors of at most 5\% in the majority of bins.
This preliminary study suggests that effect from QCD
contamination in the \gj control region is small compared 
to the total uncertainty assigned to transfer factors. 
This systematic source is covered in the data-driven study  
using the photon control region, described in Sec. ~\ref{sec:tfSyst_ZGratio}.



\subsubsection*{PU reweighting}
\label{sec:tfSyst_pu}

Events in simulation are reweighted in order to match the distribution 
of the primary vertex multiplicity observed in data, as described in Sec. ~\ref{sec:pileup-reweighting}.
A systematic uncertainty is derived by propagating 
the 5\% uncertainty on the minimum bias cross section used in the reweighting procedure. 
The relative change in the transfer factors under this variation is
small (\~1-5\%)
and shown in each analysis bin in Fig. ~\ref{fig:tfSyst_pu_muToZinv}-\ref{fig:tfSyst_pu_muToTtw}.

\begin{figure}[!h]
  \centering
  \subfigure[PU weight up variation]{
    \includegraphics[width=0.5\textwidth]{figures/mcSystematics6p26Fb/Zinv/mu/ratiotfh_ht_mht_allpuWeight_Up.pdf}
  } ~~
  \subfigure[PU weight down variation]{
    \includegraphics[width=0.5\textwidth]{figures/mcSystematics6p26Fb/Zinv/mu/ratiotfh_ht_mht_allpuWeight_Down.pdf}
  }\\

  \caption{\label{fig:tfSyst_pu_muToZinv} The relative change in the
  $\mj \rightarrow (\znunu)$ transfer
  factors when varying PU weight in MC within its uncertainties, as a function of \scalht and jet category. 
  Variations corresponding to $+1\sigma$ ($-1\sigma$) are shown in the left (right) figure. 
  }
\end{figure}

\begin{figure}[!h]
  \centering
  \subfigure[PU weight up variation]{
    \includegraphics[width=0.5\textwidth]{figures/mcSystematics6p26Fb/Zinv/mumu/ratiotfh_ht_mht_allpuWeight_Up.pdf}
  } ~~
  \subfigure[PU weight down variation]{
    \includegraphics[width=0.5\textwidth]{figures/mcSystematics6p26Fb/Zinv/mumu/ratiotfh_ht_mht_allpuWeight_Down.pdf}
  }\\

  \caption{\label{fig:tfSyst_pu_mumuToZinv} The relative change in the
  $\mmj \rightarrow (\znunu)$ transfer
  factors when varying PU weight in MC within its uncertainties, as a function of \scalht and jet category. 
  Variations corresponding to $+1\sigma$ ($-1\sigma$) are shown in the left (right) figure. 
  }
\end{figure}

\begin{figure}[!h]
  \centering
  \subfigure[PU weight up variation]{
    \includegraphics[width=0.5\textwidth]{figures/mcSystematics6p26Fb/Zinv/gj/ratiotfh_ht_mht_allpuWeight_Up.pdf}
  } ~~
  \subfigure[PU weight down variation]{
    \includegraphics[width=0.5\textwidth]{figures/mcSystematics6p26Fb/Zinv/gj/ratiotfh_ht_mht_allpuWeight_Down.pdf}
  }\\

  \caption{\label{fig:tfSyst_pu_gjToZinv} The relative change in the
  $\gj \rightarrow (\znunu)$ transfer
  factors when varying PU weight in MC within its uncertainties, as a function of \scalht and jet category. 
  Variations corresponding to $+1\sigma$ ($-1\sigma$) are shown in the left (right) figure. 
  }
\end{figure}

\begin{figure}[!h]
  \centering
  \subfigure[PU weight up variation]{
    \includegraphics[width=0.5\textwidth]{figures/mcSystematics6p26Fb/Ttw/mu/ratiotfh_ht_mht_allpuWeight_Up.pdf}
  } ~~
  \subfigure[PU weight down variation]{
    \includegraphics[width=0.5\textwidth]{figures/mcSystematics6p26Fb/Ttw/mu/ratiotfh_ht_mht_allpuWeight_Down.pdf}
  }\\

  \caption{\label{fig:tfSyst_pu_muToTtw} The relative change in the $\mj \rightarrow \mathrm{tt+W}$ transfer
  factors when varying PU weight in MC within its uncertainties, as a function of \scalht and jet category. 
  Variations corresponding to $+1\sigma$ ($-1\sigma$) are shown in the left (right) figure. 
  }
\end{figure}



\newpage
%% %%%%%%%%%%%%%%%%%%%%%%%%%%%%%%%%%%%%%%%%%%%%%%%%%%%%%%%%%%%%%%%%
%% % Closure tests
%% %%%%%%%%%%%%%%%%%%%%%%%%%%%%%%%%%%%%%%%%%%%%%%%%%%%%%%%%%%%%%%%%

\subsection{Systematics uncertainties from data-driven tests}
\label{sec:closure-tests}
This analysis aims to rely as much as possible on the data control samples
to check for sources of bias in the transfer factors due to potential limitations in
the simulation modelling. 
Therefore, along with the MC variations mentioned above, tests are performed 
in which the number of events in a given data control sample is predicted 
using events from another data control sample and the corresponding transfer factor built in MC. 
The agreement between the predicted and observed yields is
expressed as the ratio $(\nobs - \npre)/\npre$ while considering only
the statistical uncertainties on \npre and \nobs. Therefore, the level
of closure is defined by the statistical significance of a deviation
in the ratio from zero.
These tests are performed separately for each \njet category, as a function of \scalht. 
The systematic uncertainty in each \scalht bin is derived by summing in quadrature the ratio 
defined above with its statistical error, after merging the \njet categories into symmetric and asymmetric topologies. 
Pairs of \scalht bins are merged when the $\mu\mu$+jets sample is used, in order to gain statistics. \\
Since the uncertainties derived with this approach are statistical in nature, 
these systematics are considered un-correlated in each \scalht bin and jet topology. 


\subsubsection*{Extrapolation in \alphat and \bdphi}
\label{sec:tfSyst_alphaT}
The modelling of the \alphat and
\bdphi extrapolations are also tested with dedicated tests in data. 
In both cases, it is checked that events with genuine \met found in the core
of the variable distribution below some threshold value can be used to
predict the events in the tail (above the same threshold value).
This is important to verify the
approach of using \mj and \mmj samples without an \alphat requirement
to make background predictions in the signal region. The tests
confront data yields in a \mj  samples with an \alphat /\bdphi
requirement against predictions determined in a \mj sample with
the \alphat /\bdphi requirement inverted. 
The contribution to the systematic error for \met extrapolation is taken
from the \alphat closure tests for bins with $\scalht<800\gev$ and from 
the \bdphi tests for bins with $\scalht>800\gev$. 

The result of these tests are shown in Fig.~\ref{fig:closureAlphaT} as a function of \scalht and \njet. 
The grey band is the systematic uncertainty propagated through the analysis, 
taken as un-correlated per each \scalht bin and jet topology
(symmetric/asymmetric). The systematic derived from these tests is
in the range $5-40\%$.


\begin{figure}[h!]
  \begin{center}
    \subfigure[]{\includegraphics[width=0.5\textwidth]{figures/closureTests/alphaTsym__noFit.pdf}}
    ~~
    \subfigure[]{\includegraphics[width=0.5\textwidth]{figures/closureTests/alphaTasym__noFit.pdf}}\\
    \subfigure[]{\includegraphics[width=0.5\textwidth]{figures/closureTests/bDPhisym__noFit.pdf}}
    ~~
    \subfigure[]{\includegraphics[width=0.5\textwidth]{figures/closureTests/bDPhiasym__noFit.pdf}} 

    \caption{Data-driven tests probing the \alphat (top row) and \bdphi (bottom row) extrapolation for each
      \njet category (open symbols) overlaid on top of the systematic
      uncertainty estimates used for each of the seven \scalht bins (shaded bands). 
      The symmetric (asymmetric) jet topologies are shown in the left (right) plot. 
    }
    \label{fig:closureAlphaT}
  \end{center} 
\end{figure}



\subsubsection*{Modelling of the W/Z ratio}
\label{sec:tfSyst_WZratio}
To validate the use of \wmj and \ttbar dominated \mj events to predict the \znunu
background, tests are performed in data using single-muon and double-muon control regions. 
The events in the \mj control are used to predict events in the \mmj control regions, 
using transfer factors from simulation. 
These tests target the modelling of the W/Z ratio in simulation and 
also indirectly test muon acceptance effects, which 
are expected to be sub-dominant and whose uncertainties are already addressed elsewhere.

The result are shown in Fig.~\ref{fig:closureMuToMuMu} as a function of \scalht and \njet. 
The grey band is the systematic uncertainty propagated through the analysis, 
taken as un-correlated per each \scalht bin and jet topology (symmetric/asymmetric). The systematic derived from these tests is
in the range $10-30\%$.



\begin{figure}[h!]
  \begin{center}
    \subfigure[]{\includegraphics[width=0.5\textwidth]{figures/closureTests/mu_mumusym_half_noFit.pdf}}
    ~~
    \subfigure[]{\includegraphics[width=0.5\textwidth]{figures/closureTests/mu_mumuasym_half_noFit.pdf}} 
    \caption{Data-driven tests probing the use of the \mj control sample
      to predict the \znunu background for each
      \njet category (open symbols) overlaid on top of the systematic
      uncertainty estimates used for each of the seven \scalht bins (shaded bands).  
      The symmetric (asymmetric) jet topologies are shown in the left (right) plot. 
    }
    \label{fig:closureMuToMuMu}
  \end{center} 
\end{figure}




\subsubsection*{Modelling of the W/Z acceptance due to polarisation effects}
\label{sec:tfSyst_Wpol}
A data-driven test is introduced to check the modelling of the W polarisation in simulation. 
In this study, carried on using events in the single-muon control region, $\mu^{+}$ events 
are used to predict $\mu^{-}$ events, using transfer factor built in MC. 
The polarisation of the W boson has an impact on the prediction 
of the \znunu background using the \mj control region, as explained in the following. 
The production mechanism of $W$ from pp
collisions means high $p_T$ $W$ bosons are predominantly left handed
\cite{WPol}.  For high $p_T$ bosons, this implies that $W^+$ decays to
the left handed neutrino along its direction of motion while the
lepton is pointing backward. The opposite behaviour is expected for
the $W^-$. The lepton is therefore more boosted (and the neutrino less
boosted) in $W^+$ decays than $W^-$ decays.  This leads to a larger
number of $W^+$ decays in the single lepton control regions (which
relies on the lepton $p_T$ for acceptance) than in the signal region
(which relies on the neutrino $p_T$ for acceptance).

The results are shown in Fig.~\ref{fig:closureMuPToMuM} as a function of \scalht and \njet. 
The grey band is the systematic uncertainty propagated through the analysis, 
taken as un-correlated per each \scalht bin and jet topology (symmetric/asymmetric). The systematic derived from these tests is
in the range $5-50\%$.



\begin{figure}[h!]
  \begin{center}
    \subfigure[]{\includegraphics[width=0.5\textwidth]{figures/closureTests/muplus_muminussym__noFit.pdf}}
    ~~
    \subfigure[]{\includegraphics[width=0.5\textwidth]{figures/closureTests/muplus_muminusasym__noFit.pdf}} 
    \caption{Data-driven tests probing the W polarisation effects. 
      These are shown for each
      \njet category (open symbols) overlaid on top of the systematic
      uncertainty estimates used for each of the seven \scalht bins
      (shaded bands). 
      The symmetric (asymmetric) jet topologies are shown in the left (right) plot.       
    }
    \label{fig:closureMuPToMuM}
  \end{center} 
\end{figure}



\subsubsection*{Modelling of the Z/$\gamma$ ratio}
\label{sec:tfSyst_ZGratio}
To validate the use of \gj events to predict the \znunu
background, tests are performed in data using the photon and double-muon control regions. 
The events in the \gj control are used to predict events in the \mmj control regions, 
using transfer factors from simulation. 
These tests target the modelling of the Z/$\gamma$ ratio in simulation and 
also indirectly test muon/photon acceptance effects, which, however, 
are expected to be sub-dominant and whose uncertainties are already addressed elsewhere. 

The result are shown in Fig.~\ref{fig:closurePhoToMuMu} as a function of \scalht and \njet. 
The grey band is the systematic uncertainty propagated through the analysis, 
taken as un-correlated per each \scalht bin and jet topology (symmetric/asymmetric). The systematic derived from these tests is
in the range $10-30\%$.


\begin{figure}[h!]
  \begin{center}
    \subfigure[]{\includegraphics[width=0.5\textwidth]{figures/closureTests/phot_mumusym_half_noFit.pdf}}
    ~~
    \subfigure[]{\includegraphics[width=0.5\textwidth]{figures/closureTests/phot_mumuasym_half_noFit.pdf}} 
    \caption{Data-driven tests probing the Z/$\gamma$ ratio for each
      \njet category (open symbols) overlaid on top of the systematic
      uncertainty estimates used for each of the seven \scalht bins
      (shaded bands). 
      The symmetric (asymmetric) jet topologies are shown in the left (right) plot.      
    }
    \label{fig:closurePhoToMuMu}
  \end{center} 
\end{figure}





\subsubsection*{Modelling of the W/\ttbar admixture}
\label{sec:tfSyst_WttAd}
The $0$ b-tag $\rightarrow1$ b-tag data-driven tests in the \mj control region 
probe the sensitivity of the transfer factors to the relative
admixture of events from the $W$ + jets and \ttbar processes, 
since they utilise a W-eniched sample to predict a \ttbar-enriched sample. 
These tests indirectly probe also the modelling of the b-tagging efficiency, 
although this systematic effect is expected to be smaller and is already addressed 
by the dedicated study presented in Sec. ~\ref{sec:tfSyst_btag}.
These tests are believed to give a conservative uncertainty, 
as the admixture changes little between the \mj sample and the signal region, 
given that no extrapolation between different b-tag multiplicities is performed 
in the estimation of the background. 
The uncertainty derived is therefore driven by the limited statistics available in the control sample. 

The result are shown in Fig.~\ref{fig:closureBTag} as a function of \scalht and \njet. 
The grey band is the systematic uncertainty propagated through the analysis, 
taken as un-correlated per each \scalht bin and jet topology (symmetric/asymmetric). The systematic derived from these tests is
in the range $5-60\%$.

\begin{figure}[h!]
  \begin{center}
    \subfigure[]{\includegraphics[width=0.5\textwidth]{figures/closureTests/eq0b_eq1b_muonsym__noFit.pdf}}
    ~~
    \subfigure[]{\includegraphics[width=0.5\textwidth]{figures/closureTests/eq0b_eq1b_muonasym__noFit.pdf}} 
    \caption{Data-driven tests probing the W and \ttbar admixture 
      in each \njet category (open symbols) overlaid on top of the systematic
      uncertainty estimates used for each of the seven \scalht bins
      (shaded bands). 
      The symmetric (asymmetric) jet topologies are shown in the left (right) plot.      
    }
    \label{fig:closureBTag}
  \end{center} 
\end{figure}


\newpage
\begin{landscape}
\begin{table}[h!]
  \caption{Summary of the systematics on the transfer factors considered in the analysis, 
    with representatives ranges of uncertainties and the correlation assummed, 
    for the predictions of the $\ttbar$, W and $\znunu$  background
    components.}
  \label{tab:systs}
  \centering
  \footnotesize
  \begin{tabular}{ ccccccc }
    \hline
    \hline
    Systematic & Method & \multicolumn{4}{c}{Relative uncertainty on transfer factor} & Correlation model \\    
     & & $\mj \rightarrow \znunu$  & $\mmj \rightarrow \znunu$ & $\gj \rightarrow \znunu$ & $\mj \rightarrow \ttbar+W$ & \\
    \hline
    \alphat/\bdphi extrapolation & data-driven tests & $5-40\%$ &
    $5-40\%$ & - & $5-40\%$ & un-correlated across \scalht/jet top. \\
    W/Z ratio & data-driven tests & $10-30\%$ & - & - & - & un-correlated across \scalht/jet top. \\
    Z/$\gamma$ ratio & data-driven tests & - & - & $10-40\%$ & - & un-correlated across \scalht/jet top. \\
    W/\ttbar admixture & data-driven tests & - & - & - & $5-60\%$ & un-correlated across \scalht/jet top. \\
    W polarisation & data-driven tests & $5-50\%$ & - & - & $5-50\%$ & un-correlated across \scalht/jet top. \\
    Jet energy scale & MC variations & $<15\%$ & $<10\%$ & $<15\%$ &
    $<15\%$ & fully correlated \\
    B-tagging efficiency & MC variations & $<5\%$ & $<2\%$ & $<2\%$
    & $<5\%$ & fully correlated \\
    Pileup weights & MC variations & $<5\%$ & $<6\%$ & $<4\%$ & $<9\%$ & fully correlated \\
    Top $p_{T}$ weights & MC variations & $<20\%$  & $<4\%$ & - &
    $<5\%$ & fully correlated \\
    Lepton selection & MC variations & - & - & - & $2-5\%$ & fully correlated \\
    Lepton trigger efficiency & MC variations & $5\%$ & $5\%$ & - & $5\%$ & fully correlated \\
    Photon trigger efficiency & MC variations & - & - & $5\%$ & - & fully correlated \\
    \hline
    \hline
  \end{tabular}
\end{table}

\end{landscape}
