%____________________________________________________________________________||
\section{Event selection for signal and control regions}
\label{sec:selection}

This section first outlines the set of ``pre-selection'' requirements
that are common to all signal and control regions, before defining the
selection criteria that are specific to each region.

%%____________________________________________________________________________||
\subsection{Pre-selection}
\label{sec:preSelection}

{\bf Removing instrumental sources of ``fake'' \met.} 

A number of beam- and detector-related effects can induce significant
\met. Examples include beam halo, reconstruction failures, spurious
detector noise, or event misreconstruction due to detector
inefficiencies. These events, with large, non-physical values of \met,
are rejected with high efficiency by applying a range of dedicated
vetoes. All ``MET filters'' recommended by the JetMET POG and SUSY PAG
are applied by default in this analysis and listed in Table~\ref{tab:pre-selections}.

{\bf Jet requirements.} 

Jets considered in the analysis are required to satisfy $\PT>40\gev$
and $|\eta|<3.0$. Events containing jets in the forward region that
satisfy the requirements $\PT>40\gev$ and $|\eta|>3.0$ are rejected in
order to control background contributions from SM processes, without
introducing a significant reduction in signal acceptance. The jets
that are selected are used in the calculation of all jet-based
event-level variables, such as \HT, \mht, and \alphat.

Raised $\PT$ thresholds and tighter $\eta$ requirements on the lead jets 
are also required. The lead jet is required to satisfy $\PT > 100\gev$
and $|\eta|<2.5$. This helps to ensure high trigger efficiencies,
but also helps to improve the S/B for a wide
range of models with respect to SM processes, such as V + jets
production. Events are then classified based on the
second leading jet. In the case that a second leading jet satisfies $\PT > 100\gev$ 
events are assigned to a ``symmetric'' \njet category. If the second
jet satisfies $40 < \PT < 100\gev$ events are assigned to an
``asymmetric'' \njet category. Finally, if there is no second leading
jet with $\PT>40\gev$, events are assigned to the ``mono-jet''
category. This categorisation is heavily utilised throughout all the document. \\
The asymmetric and mono-jet categories have been added to
the analysis to help improve acceptance to a range of DM models and compressed
SUSY.

{\bf Event categorisation according to \njet and \nb.} 

Events in the hadronic signal and all control regions (described
below) are categorised identically and according to the number of jets
(\njet) reconstructed in each event and the number of jets identified
as originating from bottom quarks (\nb) in each event. As a baseline,
the resulting sub-samples comprise events containing exactly one, two,
three, four, or at least five jets. These are further split into
``monojet'' (only in the $\njet=1$ case),
``symmetric'' or ``asymmetric'' \njet categories according to the
second leading jet \Pt, as defined above.

Events are also categorised according to the the number of b-tagged
jets (``b-jets''). As a baseline, the sub-samples are defined by
requiring exactly zero, one, two, or at least three b-tagged jets. By
construction, $\nb \leq \njet$. Standard Model background events containing three or more b-tags
typically arise from an additional source of b-jets like 
``gluon splitting'' or from 
mistag of a jet from a light-flavoured parton. 


{\bf \HT,\mht requirements and binning.} 

Events are required to have significant hadronic activity by requiring
$\scalht > 200\GeV$. Despite an increase in both multijet production
cross sections and pileup in Run~2, the lowest \HT threshold is
kept at the same value of the Run~1 analysis~\cite{Chatrchyan:2013lya}
in order to maintain acceptance to DM models or compressed
SUSY. Events in all samples are binned identically, according to the
\HT variable. The choice of binning in \HT is driven primarily by the
trigger strategy employed by the analysis, as described in
Section~\ref{sec:triggers}, and can be summarised as follows: 50\gev
bins in the range $200 < \HT < 400\gev$, 100\gev bins in the range
$400 < \HT < 600\gev$, a 200\gev bin $600<\HT<800\gev$ and a final 
inclusive bin $\HT > 800\gev$.

Events are also required to have appreciable missing hadronic energy
by requiring $\mht>130\gev$. This ensures that all events used within
the analysis have a degree of missing energy similar to that required
in the signal region via an \alphat cut. This is outlined in
Sec.~\ref{sec:had-signal}. 

The lower threshold of the last (inclusive) \HT bin is not forced to
$800\gev$, it is instead determined
independently for each (\njet,\nb) event category and is chosen to
always align with one of the ``default'' boundaries defined above. The
metric for choosing the final bin threshold is based on the number of
events in the corresponding event category and \HT bin of the data
control samples. This ensures that all bins in the data control
samples are sufficiently populated to ensure a statistically
significant prediction in each of the corresponding signal region
bins. Currently, this is achieved by requiring at least $10$ predicted
events in each \HT and \njet category of the control regions
along with the requirement that there is at least one observed
event per \nb category. If this is not the case, events from high \HT bins are combined
into a single inclusive bin that satisfies this metric. This metric
ensures that there are sufficient events in the control samples
to probe for potential systematic effects with closure tests between
simulation and data, as described in Sec.~\ref{sec:systematics}.

\begin{table}[h!]
  \topcaption{Summary of the pre-selection criteria.}
  \label{tab:pre-selections}
  \centering
  \footnotesize
  \begin{tabular}{ ll }
    \hline
    \hline
    Selection                     & Requirement                                                                          \\
    \hline
    ``MET filters''               & Primary Vertex, CSC Beam Halo, HBHE Noise and Isolation, ECAL Endcap SC Noise        \\
    Jet acceptance                & $\PT > 40\gev$, $|\eta| < 3$                                                         \\
%    \njet                         & $\geq2$                                                                \\
    Lead jet acceptance           & $\PT > 100\gev$, $|\eta| <    2.5$                                     \\
    Second jet acceptance         & $\PT > 100\gev$ \texttt{OR} $40 < \PT < 100\gev$                       \\
    Loosest \HT requirement       & $\HT > 200\gev$                                                        \\
    Loosest \mht requirement      & $>130\gev$                                                     \\  
    Baseline \HT binning          & 200--250, 250--300, 300--350, 350--400, 400--500, 500--600, 600--800, $>$800\gev \\
    Baseline \njet multiplicities & 1 (mono-jet), 2, 3, 4, $\geq$5 (both symmetric and asymmetric)                       \\
    Baseline \nb multiplicities   & 0, 1, 2, $\geq3$ ($\nb \leq \njet$)                                    \\
    \hline
    \hline
  \end{tabular}
\end{table}

{\bf Summary of pre-selection requirements.} 

Table~\ref{tab:pre-selections} summarises the pre-selection
requirements and default categorisation and binning scheme. The
threshold of the final \HT bin per (\njet,\nb) category is summarised
in Table~\ref{tab:binning-3fb}. An identical scheme is used for the
signal region and all control regions. No extrapolation is performed
in the variables \njet, \nb, and \HT in this analysis. For each of the
signal and control regions, thirty event categories are considered,
each with up to seven bins in \HT, assuming an integrated luminosity
of 2.1\ifb. 

\begin{table}[h!]
  \caption{Threshold (GeV) of the final \HT bin as a function of event
    category (\njet,\nb), which is always aligned with respect
    to one of the baseline boundaries (motivated primarily by the trigger) of
    200, 250, 300, 350, 400, 600, 800\gev. 
    %This is the projected choice for 3\fbinv.
    }
  \label{tab:binning-3fb}
  \centering
  \footnotesize
  \hspace*{-1cm}\begin{tabular}{ l|cc|ccc|cccc|cccc|cccc }
    \hline
    \hline
    \njet      & \multicolumn{2}{c|}{1} & \multicolumn{3}{c|}{2} & \multicolumn{4}{c|}{3} & \multicolumn{4}{c|}{4} & \multicolumn{4}{c}{$\geq5$} \\ 
    \nb        & 0   & 1   & 0   & 1   & 2   & 0   & 1   & 2   & 3   & 0   & 1   & 2   & $\geq3$ & 0   & 1   & 2   & $\geq3$ \\
    \hline
    Symmetric  & 600 & 500 & 800 & 800 & 600 & 800 & 800 & 800 & 400 & 800 & 800 & 800 & 800     & 800 & 800 & 800 & 800     \\
    Asymmetric & -   & -   & 600 & 500 & 400 & 600 & 600 & 500 & 300 & 600 & 600 & 600 & 400     & 600 & 600 & 600 & 500     \\
    \hline
    \hline
  \end{tabular}
\end{table}


\subsection{Lepton and photon vetoes \label{sec:vetoes}}

To select a sample of events in the hadronic final state and to
suppress SM processes with genuine \met from neutrinos, events
containing an isolated electron with $\pt > 10\GeV$ and $|\eta| < 2.5$
or an isolated muon with $\pt > 10\GeV$ and $|\eta| < 2.5$ are
vetoed. Further, to reduce the ``lost leptons'' backgrounds from \wj
and \ttbar, events containing single isolated tracks with $\pt >
10\GeV$ and $|\eta| < 2.5$, as defined in Sec.~\ref{sec:objects},
are vetoed as part of the signal region selection criteria. In the
case of the single and di-lepton control samples, a further
requirement is made such that events are not vetoed due to the
presence of a track from the well identified leptons, by requiring
$\Delta R(\textrm{track},\textrm{lepton}) > 0.02$.

Finally, to select a purely hadronic topology and to allow for a
orthogonal control region, events are vetoed in which an isolated photon
with $\pt > 25\GeV$ and $|\eta| < 2.5$ is identified.

\subsection{The hadronic signal region}
\label{sec:had-signal}

The lepton and photon vetoes are applied to select hadronic final
states. All pre-selection criteria are also applied. Following these
selections, the multijet background from QCD is still several orders
of magnitude larger than the typical signal expected from SUSY.

{\bf \HT-dependent \alphat requirements.}

Background events from multijet production populate the region
$\alphat \lesssim 0.5$ and therefore can be rejected with very high
efficiency by requiring an appropriate cut on \alphat. \\
A useful approximate conversion between \alphat and \mht can be
obtained by calculating \alphat, as described by Eq.~\ref{eq:alphat3}, 
while forcing $\dht = 0\gev$. Hence, using this metric, the
dependence of the \alphat requirement as a function of the \HT bin can
be determined such that the effective requirement on \mht is
comparable, \ie roughly constant, across all \HT bins. The values
typically fall in the range $\sim110 < \mht < \sim160\gev$. This
approximate levelling of the ``effective'' \mht threshold implies
increasingly tighter requirements against instrumental effects versus
\HT, while maximising signal
acceptance. Table~\ref{tab:alphat-thresholds} summarises the 
\alphat thresholds and corresponding ``effective'' \mht thresholds for
each \HT bin. The \alphat threshold is dependent only on \HT and not
on \njet nor \nb that are used to define the event categories.

\begin{table}[h!]
  \caption{\alphat and corresponding ``effective'' \mht (GeV) thresholds versus
    lower bound of \scalht bin. For all \HT bins satisfying $\HT > 800
    \gev$, the direct requirement of $\mht > 130\gev$ is imposed rather
    than a requirement on \alphat. No \alphat requirement is imposed in the
    monojet bins.}
  \label{tab:alphat-thresholds}
  \centering
  \footnotesize
  \begin{tabular}{ lcccccccc }
    \hline
    \hline
    \scalht            & 200       & 250       & 300       & 350       & 400       & 500       & 600       \\
    \hline                                                                                     
    \alphat threshold  & 0.65      & 0.60      & 0.55      & 0.53      & 0.52      & 0.52      & 0.52      \\
    ``Effective'' \mht & $\sim$128 & $\sim$138 & $\sim$125 & $\sim$123 & $\sim$110 & $\sim$138 & $\sim$162 \\
    \hline
    \hline
  \end{tabular}
\end{table}

For all signal region bins satisfying $\HT > 800\gev$  no \alphat
threshold is required, which removes the inefficiencies of this
variable for high jet multiplicity events. Instead, the following
$\mht >130\gev$ requirement helps to control the multijet background
along with the imposition of $\bdphi > 0.5$ (described below).

{\bf \bdphi requirement.} 

Further, an additional powerful variable \bdphi is used to suppress
multijet contamination due to both instrumental effects and
semi-leptonic heavy-flavour decays with genuine \met in the final
state. The variable is determined as follows. The jet-based estimate
of the missing transverse energy, ${\mhtvec}$, is recomputed while
ignoring one of the reconstructed jets (the ``test'' jet). The
difference in the azimuthal angle between the recomputed $\mhtvec$
and the ``test'' jet is then determined. This process is repeated for
each jet in the event and the minimum of all the azimuthal
differences, \bdphi, is determined. For monojet events, the calculation is 
performed using all jets with $\Pt > 20\gev$. The ``test'' jet whose subtraction
from the calculation $\mhtvec$ yields this minimum value, is
identified as the jet that is most likely to have given rise to the
missing transverse energy in the event. Events with significant \mht
due to instrumental effects or heavy flavour decays populate the
region at $\bdphi$ and so candidate signal event are accepted
only if they satisfy $\bdphi > 0.5$. The use of the \bdphi and \alphat
variables provide an extremely powerful rejection factor against
contamination from multijet events and allow to maintain low jet \PT,
\HT, and \mht thresholds, which in turn maximises signal acceptance
for a large range of DM and SUSY models with final states
characterised by the presence of significant \met.

{\bf $\mht/\met$ cleaning filter.} 

To protect against multiple jets failing the $\Et$ threshold or
falling out of detector acceptance, the jet-based
estimate of the missing transverse energy, \mht, is compared to the
missing transverse energy variable, $\met$, and events with $R_{\rm
  miss}=\mht/\met > 1.25$ are rejected.
  
%% {\bf Detector dead cell control.}

%% Masked regions in the ECAL (which amount to about 1\% of the ECAL channel count)
%% or HCAL, or by missing instrumentation in the barrel-endcap gap, could cause 
%% severe energy losses. A data-driven method is developed to identify dead cells. The 
%% procedure is carried out on events that pass a loose selection of one good primary vertex,
%% $\njet>1$ and $\scalht>150\gev$. For each identified jet with
%% $\PT > 20 \gev$ in data, the azimuthal angle ($\Delta\phi_{jet}$) between the jet and the 
%% recomputed ${\mhtvec}$ is determined, in the same way as in the procedure to compute the \bdphi 
%% variable. The positions of all jets which give $\bdphi < 0.3$ are plotted in
%% an $\eta-\phi$ map. Subsequently, the positions of all jets with
%% $\pt>20\gev$ (woth no \bdphi requirement) are plotted in a second $\eta-\phi$ map.
%% These two maps are then divided to form a 2D ratio map, taking the
%% first map as the numerator and second as the denominator.
%% Jets pointing to dead cells are likely to give $\bdphi < 0.3$, so the
%% location of dead cells in $\eta$ and $\phi$ have higher values in this 2D ratio map. 

%% Figure~\ref{fig:2dRatioMap} is the 2D ratio map made with unblinded signal region with luminosity of 
%% 149.49~$\text{pb}^{-1}$. In the lower left region of the plot, two
%% areas with significant instrumental effects are clear. To ensure these
%% effects are filtered out by all cleaning cuts within the analysis, an
%% $\eta-\phi$ map of all jets with $\pt>20\gev$ and $\bdphi<0.3$ are
%% plotted after the signal region selection (as described in
%% Sec.~\ref{sec:had-signal}), shown in
%% Fig.~\ref{fig:jetMapPostSignalSelection}. The areas with significant
%% instrumental effects visible in Fig.~\ref{fig:2dRatioMap} do not appear
%% to remain in Fig.~\ref{fig:jetMapPostSignalSelection}. This implies the current suite of cleaning cuts
%% is enough to remove any localised detector effects.

%% \begin{figure}[h!]
%%     \begin{center}
%%         {\includegraphics[width=0.7\textwidth]{figures/selection/EtaPhiMap.pdf}}
%%         \caption{The 2D ratio of jets ($\pt>20\gev$) with $\bdphi<0.3$ plotted in the
%%         $\eta-\phi$ plane, divided by all jets that satisfy $\pt>20\gev$. 
%%         Made with 149.49~$\text{pb}^{-1}$ of
%%         events that pass a loose selection of one good primary vertex,
%%         $\njet>1$ and $\scalht>150\gev$.}
%%         \label{fig:2dRatioMap}
%%     \end{center}
%% \end{figure}

%% \begin{figure}[h!]
%%     \begin{center}
%%         {\includegraphics[width=0.7\textwidth]{figures/selection/bDPhilt0p3AfterSelection.pdf}}
%%         \caption{Jets ($\pt>20\gev$) with $\bdphi<0.3$ plotted in the
%%         $\eta-\phi$ plane. 
%%         Made with $1.26\ifb$ of
%%         events that pass the full signal region selection.}
%%         \label{fig:jetMapPostSignalSelection}
%%     \end{center}
%% \end{figure}


{\bf Beam halo.}

The CSC beam halo filter has been found to be less efficient during the early
Run 2 data-taking period compared to the previous run.

Beam halo events manifest themselves as single energy deposits in the
calorimeters, which introduces large amounts of ``fake'' \met. This effect is
especially prominent in the signal region monojet category, particularly at
$\phi$ coordinates of 0 and $\pi$ because of the tendency of halo particles to
lie within the plane of the LHC ring. 

Such spurious events are suppressed by requiring at least 10\% of the leading
jet's energy to originate from charged hadrons, $CHF>0.1$. The effectiveness of this selection
is demonstrated in Fig.~\ref{fig:leadJetCleaning}, with the 2015 data. 

There is no need for this selection in the control regions, 
as the requirement of well identified physics objects, like muons 
and photons naturally removes spurious events of this kind. 

This study has not been updated yet with the 2016 data. 
However, this requirement is still in place, and has to be regarded as 
an additional tight ID requirement employed in the analysis, 
with efficiency close to one for real jets and thus not affecting the 
signal efficiency significantly. 

%% Beyond the filter available in the data ntuples, the JetMET POG have
%% provided lists of events that should fail the CSC beam halo and bad
%% ECAL super cluster filters. These extra events are vetoed and the
%% efficiency of events in the signal region (Sec.~\ref{sec:had-signal})
%% and single muon control region (Sec.~\ref{subsec:mucontrolSelection})
%% that pass this veto for each analysis bin are plotted in
%% Fig.~\ref{fig:cscFilterEfficiencies}. In the vast majority of bins the
%% extra filters are $100\%$ efficient, with a few bins with an
%% inefficiency at the $2-3\%$ level. This confirms that the $CHF$ cut is
%% already effectively removing spurious events that are present due to
%% beam halo effects. 

\begin{figure}[h!]
    \begin{center}
        {\includegraphics[width=0.32\textwidth]{figures/selection/leadJetChf_all_before.pdf}}
        {\includegraphics[width=0.32\textwidth]{figures/selection/leadJetPhi_all_before.pdf}}
        {\includegraphics[width=0.32\textwidth]{figures/selection/leadJetPhi_all_after.pdf}}
        \caption{Distributions in the signal region of the lead jet charged hadron
        energy fraction (CHF) (Left), lead jet $\phi$ direction (Centre), and lead jet $\phi$
        direction after applying a requirement of {CHF~$>0.1$}. The large excess in data
        at charged hadron fractions close to zero and ${\phi = 0, \pi}$ is consistent with beam
        halo effects, and is effectively suppressed by the aforementioned selection.}
        \label{fig:leadJetCleaning}
    \end{center}
\end{figure}

%% \begin{figure}[h!]
%%   \begin{center}
%%     \subfigure[Signal region
%%     selection]{\includegraphics[width=0.5\textwidth]{figures/selection/Signal_Data_CSCEfficiency.pdf}} ~~
%%     \subfigure[Single muon control region
%%     selection]{\includegraphics[width=0.5\textwidth]{figures/selection/SingleMu_Data_CSCEfficiency.pdf}} \\
%%     \caption{Efficiencies of the CSC beam halo and bad ECAL super
%%     cluster filters applied on $1.26\ifb$ of data passing the single
%%     muon and signal region selections.}
%%     \label{fig:cscFilterEfficiencies}
%%   \end{center} 
%% \end{figure}

{\bf Summary of signal region selection.} 

The requirements that define the hadronic signal region are summarised
in Table~\ref{tab:sr-selections}.

\begin{table}[h!]
  \topcaption{Summary of the signal region selection criteria, applied
    in addition to the pre-selection summarised in
    Table~\ref{tab:pre-selections}.}
  \label{tab:sr-selections}
  \centering
  \footnotesize
  \begin{tabular}{ ll }
    \hline
    \hline
    Selection             & Requirement                                                    \\
    \hline
    \alphat               & $>$0.52--0.65 (\HT-dependent) for region $200 < \HT < 800\gev$ \\
    \bdphi                & $>0.5$                                                         \\
    \mht/\met             & $<1.25$                                                        \\
    ``Dead ECAL filter''  & (see text)                                                     \\
    ``Beam Halo Filter''  &  $CHF(\textrm{leading jet})>0.1$                                \\

    \hline
    \hline
  \end{tabular}
\end{table}

\subsection{Adding the \texorpdfstring{\mht}{MHT} dimension}
\label{sec:had-shape}

As described above, and as used in Run~1, the analysis takes advantage
of three discriminating variables, \njet, \nb, and \HT, to provide
sensitivity to a large range of SUSY (and DM) models. No extrapolation
in these variables is performed, with predictions of SM background
yields in the (\njet,\nb,\HT) bins of the signal region based on both
observed counts and transfer factors derived from simulated yields in
the corresponding (\njet,\nb,\HT) bins of the control samples. Each
prediction is statistically and systematically independent.

%% In Run~1, for each (\njet,\nb,\HT) bin in the signal region, an
%% extrapolation in the variable \alphat was necessary to obtain
%% background predictions based on the muon control samples, which did
%% not impose any \alphat requirement. No extrapolation in \alphat was
%% performed for the photon control sample, which used the same \alphat
%% requirements as the signal region. The \alphat requirements used in
%% Run~1 for the signal region correspond loosely to \mht thresholds in
%% the range $\sim$130 to $\sim$500\gev depending on the \HT
%% bin. Uncertainties in this extrapolation were determined through
%% closure tests with respect to data, including one dedicated to the
%% \alphat extrapolation, plus additional cross checks.

In Run~2, we additionally bin event counts in the signal region
according to the variable \mht in order to provide further
discriminating power between any potential signal and the SM
background counts. Hence, while no extrapolation is performed in
\njet, \nb, nor \HT, the analysis relies on information obtained
from simulation to extrapolate from counts (integrated over \mht) in
the control samples to a predicted distribution in \mht for each
corresponding (\njet,\nb,\HT) bin in the signal region.

The \mht dimension is included in the likelihood model by using
templates determined per (\njet,\nb,\HT) bin from simulation. The
templates use an \mht bin width of 50\gev. The threshold of the final
\mht bin used by the templates is determined based on the size of the
available simulated samples: a statistical uncertainty of not more
than 50\% for the sum of all SM background processes is required. 
Alternative templates are used to encode
the systematic uncertainty in the \mht distribution obtained from
simulation, as described in Sec.~\ref{sec:syst-on-shape}. 

%% \subsection{The hadronic control region}

%% A hadronic control region that is enriched in multijet events and
%% disjoint with respect to the signal region is obtained by applying
%% both the pre-selection criteria and lepton/photon vetoes, as defined
%% above, and inverting the (\HT-dependent) \alphat and/or \mhtmet
%% requirements. 
%% The sample of events populating this control region are used primarily
%% to estimate any residual background contamination from QCD multijet
%% events, described in Sec.~\ref{sec:qcd}.

\subsection{Commonalities between the data control regions}

There are three control regions with leptons or photons in the final
state: \mj, \mmj, and \gj. 
The full pre-selection is applied as part of the definition of each of these control
regions. The cuts on event-level jet-based quantities, with the exceptions of the jet CHF cut, 
are identical to those applied in the hadronic search region and the same \njet, \nb,
and \scalht binning is used. The lepton(s) or photon is not considered
in the calculation of the event-level variables.

The selection criteria of the various control regions are defined such
that the background composition and event kinematics of the control
regions mirror as closely as possible those for the signal
region. This is done in order to minimise the reliance on the
simulation to model correctly the backgrounds and event kinematics in
the control and signal samples.

Two exceptions are made. First, no \bdphi requirement is imposed as
part of the selection criteria defining the control regions. Second,
in the case of the four leptonic control regions, no requirement is
made on \alphat. This is made possible by the remaining kinematic
selection criteria, which are sufficiently selective to ensure that
the leptonic event samples remain rich in events from the \wj, \ttbar
and \zll processes with negligible contamination from QCD multijet
events. Thus, the acceptance of the leptonic control regions can be
significantly increased, which simultaneously improves their
predictive power and further reduces the effect of any potential
signal contamination.

The lepton event samples can be used to predict components of the SM
background across all \scalht bins, while the \gj sample is only 
used for the region $\HT > 400\gev$. 
This choice has been made to ensure that the photon control sample 
is used to predict the $Z\rightarrow\nu\nu$ only in that 
kinematic regime where the $\gamma/Z$ ratio is better under control in the simulation. 

\subsection{The \texorpdfstring{\mj}{muon plus jets} control sample}
\label{subsec:mucontrolSelection}

The selection criteria for the \mj sample are chosen to identify W
bosons decaying to a muon and a neutrino in the phase-space of the
signal. In order to select events containing W bosons, exactly one
tight isolated muon within an acceptance of \PT $>$ 30 \gev and
$|\eta| <$ 2.1 is required (due to the trigger), and the transverse
mass of the W candidate must satisfy $30 < \mt(\mu,\pfmet) < 125\gev$
(to suppress QCD multijet and potential signal events). Events are
vetoed if $\Delta R(\mu,\textrm{jet}_i) < 0.5$ running over all jets
$i$. The single isolated track veto, described in
Sections~\ref{sec:objects} and~\ref{sec:vetoes}, is also applied,
which considers all single isolated tracks in the event except that
associated with the identified, isolated muon. Finally, the cleaning
cut $\mht/\met$ is also applied, as done in the signal region, where
the \met is adjusted to account for the transverse momentum of the
identified, isolated muon.

\subsection{The \texorpdfstring{\mmj}{di-muon plus jets} control sample}
\label{subsec:mumucontrolSelection}

The selection criteria are identical to those for the \mj sample, with
the following exceptions that are tuned to identify Z bosons decaying
to two muons in the kinematic phase space of the signal region. 
In order to select an event sample containing Z bosons, exactly two
tight isolated muons within an acceptance of $\Pt > 30\gev$ and
$|\eta| < 2.1$ are required (due to the trigger). The invariant mass
of the two muons must satisfy $m_{Z} - 25 < M_{\mu_1\mu_2} < m_{Z} +
25$ and they must have opposite charge. Events are vetoed if $\Delta
R(\mu_{i},\textrm{jet}_j) < 0.5$ is satisfied, running over all muons
$i$ and all jets $j$. The single isolated track veto is also applied
considering all single isolated tracks in the event except those
associated with the two identified, isolated muons. Finally, the
cleaning cut $\mht/\met$ is also applied, as done in the signal
region, where the \met is adjusted to account for the transverse
momenta of the two identified, isolated muons. 

% \subsection{The \texorpdfstring{\ej}{electron plus jets} control sample}
% \label{subsec:elecontrolSelection}
%
% The selection criteria that define the \ej control sample
% mirror those of the \mj sample, \ie, they are tuned
% to identify W boson decaying to an electron and a neutrino in the
% kinematic phase space of the signal region.
%
% Electrons are required to satisfy the Tight working point and satisfy
% the requirements $\Pt> 30\gev$ and $|\eta| < 2.1$. The tightening of
% the Loose working point defined in Sec.~\ref{sec:electron-id} was
% found to greatly reduce multijet contamination without a large
% reduction in statistics within the electron control samples. The
% transverse mass of the W candidate must satisfy $30 < \mt(e,\pfmet)
% < 125\gev$.
%
% \subsection{The \texorpdfstring{\eej}{electron plus jets} control sample}
% \label{subsec:dielecontrolSelection}
%
% The selection criteria that define the \eej control sample
% mirror those of the \mmj sample. They are tuned
% to identify Z boson decaying to a pair of electrons in the
% kinematic phase space of the signal region.
%
% Both electrons need to satisfy Tight working point and the
% requirements $\Pt> 30\gev$ and $|\eta| < 2.1$. The invariant mass of
% the two electrons must satisfy $m_{Z} - 25 < M_{e_1e_2} < m_{Z} + 25$
% and they must have opposite charge.

\subsection{The \texorpdfstring{\gj}{photon plus jets} control sample}
\label{subsec:photoncontrolSelection}

The \gj sample is defined by requiring exactly one photon satisfying
tight isolation criteria and within an acceptance of $\pt > 200\gev$
(limited by trigger requirements) and $|\eta| < 1.45$. Furthermore,
events are vetoed if $\Delta R(\gamma,\textrm{jet}_i) < 1.0$ is
satisfied, running over all jets $i$. One important difference with
respect to the leptonic control samples is the application of the
\HT-dependent \alphat requirements imposed as part of the signal
region definition. This is done primarily to ensure that the photon
control sample and signal region are subject to identical kinematic
requirements and the photon carries sufficient transverse energy so
that the mass of the Z boson becomes a negligible effect when using
the \gj sample to predict the kinematic distributions of the \znunu
background. The cleaning cut $\mht/\met$ is also applied, as done in
the signal region, where the \met is adjusted to account for the
transverse energy of the identified, isolated photon. 
