%%____________________________________________________________________________||
\section{Introduction}
\label{sec:intro}

In this note we summarise the \alphat analysis, which searches for
signatures of physics beyond the standard model in events with jets
and missing transverse momentum (\met). This analysis uses the
kinematic variable \alphat to efficiently select candidate signal
events with genuine missing transverse momentum while providing robust
rejection against QCD multijet background events. With the \alphat
variable, CMS has been searching for supersymmetry (SUSY) in
proton-proton collisions data collected during LHC Run~1. With data at
a centre-of-mass energy of 7 TeV collected in 2010 and 2011, the
\alphat analysis has excluded a large parameter space of the
constrained minimal supersymmetric extension of the standard model
(CMSSM) \cite{Khachatryan:2011tk, Chatrchyan:2011zy,
Chatrchyan:2012wa} and a parameter space of simplified models
\cite{Chatrchyan:2012wa}. With data at a centre-of-mass energy of 8
TeV collected and promptly reconstructed in 2012, the \alphat analysis
further excluded a parameter space of simplified models
\cite{Chatrchyan:2013lya}. Additional sets of data which were
collected in 2012 but were reconstructed later during the LHC Long
Shutdown 1 (LS1) are called the ``parked data'', which contain data
for events triggered with lower energy
thresholds~\cite{Khachatryan:2016pxa}. The search strategy in this note is
an extension of that in Ref. \cite{CMS:2015dbr}.

After two years of the LS1, in June 2015, the LHC started its Run 2 and is
delivering proton-proton collisions to CMS at a higher centre-of-mass
energy of 13 TeV. The higher energy collision considerably increases
the production cross sections of heavy particles, providing a strong
motivation to continue the search for supersymmetry. The \alphat
analysis is particularly suited for this search and has potential for
discovery in early data. The reconstruction of missing transverse
momentum in different collider environment or at new collider energy
is typically very challenging and often requires an extended period of
development. However, the \alphat variable is designed to provide
robust discriminating power between sources of genuine and ``fake''
(\eg instrumental) \met, as well as using an estimator of \met that
relies solely on the vector sum of jet transverse momenta (\mht).

The nature of dark matter (DM) is one of the outstanding problems in
particle physics and a massive weakly interactive particle (WIMP) is
highly motivated. A WIMP is not a specific elementary particle, but
rather a broad class of possible particles. The most highly
scrutinized thermal relic DM candidate is the lightest neutralino
particle of supersymmetric (SUSY) theories. The neutralino is
particularly well-motivated since, in addition to solving the DM
problem, SUSY extensions of the SM contain a number of other
attractive features both for particle physics and in early Universe
cosmology. Most prominently known, is that SUSY does not only provide
an excellent DM candidate but also solves the fine-tuning problem.
However, not only SUSY but also non-supersymmetric models of physics
beyond the standard model predict the production of DM at LHC Run~2.
The \alphat variable efficiently selects events that potentially
contain DM candidates produced in the collisions. In Run~2, we extend
the \alphat analysis to significantly improve the acceptance to dark
matter production at the LHC. 

Section \ref{sec:strategy} discusses changes in the search methods
made since Run~1. Sec.~\ref{sec:alphatdef} defines the \alphat
variable. Section \ref{sec:datasets} lists the data sets used in this
note. Section \ref{sec:triggers} describes the preparation of the
triggers for Run~2. Section \ref{sec:objects} defines the physics
objects used in this note. Section \ref{sec:selection} describes how
events are selected. Section \ref{sec:yields} shows the expected event
yields in the signal and control regions. While Section \ref{sec:qcd}
discusses how multijet background events are controlled, Section
\ref{sec:backgroundmet} estimates background from other processes.
Then, Sec.~\ref{sec:systematics} discusses systematic uncertainties in
the estimates. With likelihood models introduced in Section
\ref{sec:likelihood}, we interpret the results in simplified SUSY
models in Sec.~\ref{sec:susy}. Finally Sec.~\ref{sec:summary} provides
a summarise.

%%____________________________________________________________________________||
