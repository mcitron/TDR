%____________________________________________________________________________||
\section{Event selection for signal and control regions}
\label{sec:selection}

This section first outlines the set of ``pre-selection'' requirements
that are common to all signal and control regions, before defining the
selection criteria that are specific to each region.

%%____________________________________________________________________________||
\subsection{Pre-selection}
\label{sec:preSelection}

{\bf Removing instrumental sources of ``fake'' \met.} 

A number of beam- and detector-related effects can induce significant
\met. Examples include beam halo, reconstruction failures, spurious
detector noise, or event misreconstruction due to detector
inefficiencies. These events, with large, non-physical values of \met,
are rejected with high efficiency by applying a range of dedicated
vetoes. All ``MET filters'' recommended by the JetMET POG and SUSY PAG
will be applied by default in this analysis.

{\bf Jet requirements.} 

Jets considered in the analysis are required to satisfy $\PT>40\gev$
and $|\eta|<3.0$. Events containing jets in the forward region that
satisfy the requirements $\PT>40\gev$ and $|\eta|>3.0$ are rejected in
order to control background contributions from SM processes, without
introducing a significant reduction in signal acceptance. The jets
that are selected are used in the calculation of all jet-based
event-level variables, such as \HT, \mht, and \alphat.

Raised $\PT$ thresholds on the lead jets are also required. The lead
jet is required to satisfy $\PT > 100\gev$, which is required to
ensure trigger efficiency but also helps to improve the S/B for a wide
range of models with respect to SM processes such as V + jets
production. Events are then classified according to the second hardest
jet, which is required to satisfy either $\PT > 100\gev$ (leading to a
``symmetric'' \njet category), or $40 < \PT < 100\gev$ (an
``asymmetric'' \njet category). The latter of these two requirements
helps to improve acceptance to a range of DM models and compressed
SUSY.

{\bf Event categorisation according to \njet and \nb.} 

Events in the hadronic signal and all control regions (described
below) are categorised identically and according to the number of jets
(\njet) reconstructed in each event and the number of jets identified
as originating from bottom quarks (\nb) in each event. As a baseline,
the resulting sub-samples comprise events containing exactly two,
three, four, or at least five jets. These are further split into
``symmetric'' or ``asymmetric'' \njet categories according to the
second leading jet \Pt, as defined above.

Events are also categorised according to the the number of b-tagged
jets (``b-jets''). As a baseline, the sub-samples are defined by
requiring exactly zero, one, two, or at least three b-tagged jets. By
construction, $\nb \leq \njet$. Events containing three or more b-tags
typically implies either an additional source of b-jets beyond those
from the \ttbar process (\eg via gluon splitting) or at least one
mistag of a jet from a light-flavoured parton. The number of b-tagged
jets is currently taken directly from simulation.

In the near future, the ``b-tag formula method'' (described below and
detailed in Ref.~\cite{Chatrchyan:2013lya}) will again be employed to
improve the statistical precision of predictions involving events
containing high multiplicities of b-tagged jets, which will allow for
the addition of further categories that contain at least four b-tagged
jets. In order to maximise sensitivity to potential new physics
signatures in final states with a high b-quark content, a method that
improves the statistical power of the predictions from simulation,
particularly for $n_\cPqb \ge 2$, is employed~\cite{RA1Paper2012}. The
distribution of $n_\cPqb$ is estimated from generator-level
information contained in the simulation, namely the number of
reconstruction-level jets matched to underlying bottom quarks
($n_\cPqb^\text{gen}$), charm quarks ($n_\cPqc^\text{gen}$), and
light-flavoured partons ($n_\cPq^\text{gen}$) per event. All relevant
combinations of $n_\cPqb^{\rm gen}$, $n_\cPqc^{\rm gen}$, and
$n_{\cPq}^{\rm gen}$ are considered, and event counts are recorded
according to the categorisation of events in terms of \njet and
\scalht. The efficiency $\epsilon$ with which b-quark jets are
identified and the mistag probabilities $f_\cPqc$ and $f_\cPq$ are
also determined from simulation per event category, with each quantity
averaged over jet \pt and $\eta$. Corrections are applied on a
jet-by-jet basis to $\epsilon$, $f_\cPqc$, and $f_\cPq$ in order to
match the corresponding measurements from
data~\cite{CMS-PAS-BTV-12-001}. This information is sufficient to
predict $n_\cPqb$ and thus also determine the event yield from
simulation for a given event category. The event yields for a given
b-quark jet multiplicity can be predicted with a higher statistical
precision than obtained directly from simulation, particularly for
events with high counts of b-quark jets.

{\bf \HT requirements and binning.} 

Events are required to have significant hadronic activity by requiring
$\scalht > 200\GeV$. Despite an increase in both multijet production
cross sections and pileup in Run~2, the lowest \HT threshold will be
kept at the same value of the Run~1 analysis~\cite{Chatrchyan:2013lya}
in order to maintain acceptance to DM models or compressed
SUSY. Events in all samples are binned identically, according to the
\HT variable. The choice of binning in \HT is driven primarily by the
trigger strategy employed by the analysis, as described in
Section~\ref{sec:triggers}, and can be summarised as follows: 50\gev
bins in the range $200 < \HT < 400\gev$, 200\gev bins in the range
$400 < \HT < 800\gev$, and a final inclusive bin $\HT > 800\gev$.

The lower threshold of the last (inclusive) \HT bin is determined
independently for each (\njet,\nb) event category and is chosen to
always align with one of the ``default'' boundaries defined above. The
metric for choosing the final bin threshold is based on the number of
events in the corresponding event category and \HT bin of the data
control samples. This ensures that all bins in the data control
samples are sufficiently populated to ensure a statistically
significant prediction in each of the corresponding signal region
bins. Currently, this is achieved by requiring sufficient events
($\sim 10$) to yield a statistical uncertainty on the total background
prediction of $\lesssim30\%$. Events from high \HT bins are combined
into a single inclusive bin that satisfies this metric. This metric
also ensures that there are sufficient events in the control samples
to probe for potential systematic effects with closure tests between
simulation and data, as described in Sec.~\ref{sec:systematics}.

\begin{table}[h!]
  \topcaption{Summary of the pre-selection criteria for 3\fbinv and 10\fbinv.}
  \label{tab:pre-selections}
  \centering
  \footnotesize
  \begin{tabular}{ ll }
    \hline
    \hline
    Selection                     & Requirement                                                            \\
    \hline
    ``MET filters''               & Various                                                                \\
    Jet acceptance                & $\PT > 40\gev$, $|\eta| < 3$                                           \\
%    \njet                         & $\geq2$                                                                \\
    Lead jet acceptance           & $\PT > 100\gev$, $|\eta| <    2.5$                                     \\
    Second jet acceptance         & $\PT > 100\gev$ \texttt{OR} $40 < \PT < 100\gev$                       \\
    Loosest \HT requirement       & $\HT > 200\gev$                                                        \\
    Baseline \HT binning          & 200--250, 250--300, 300--350, 350--400, 400--600, 600--800, $>$800\gev \\
    Baseline \njet multiplicities & 2, 3, 4, $\geq$5 (both symmetric and asymmetric)                       \\
    Baseline \nb multiplicities   & 0, 1, 2, $\geq3$ ($\nb \leq \njet$)                                    \\
    \hline
    \hline
  \end{tabular}
\end{table}

{\bf Summary of pre-selection requirements for 3\fbinv and 10\fbinv.} 

Table~\ref{tab:pre-selections} summarises the pre-selection
requirements and default categorisation and binning scheme. The
threshold of the final \HT bin per (\njet,\nb) category is summarised
in Table~\ref{tab:binning-3fb}. An identical scheme is used for the
signal region and all control regions. No extrapolation is performed
in the variables \njet, \nb, and \HT in this analysis. For each of the
signal and control regions, thirty event categories are considered,
each with up to seven bins in \HT, assuming an integrated luminosity
of 3\fbinv. The same binning scheme is currently also used for the
sensitivity projections with 10\fbinv. However, this scheme will
evolve with integrated luminosity as discussed below.

\begin{table}[h!]
  \caption{Threshold (GeV) of the final \HT bin as a function of event
    category (\njet,\nb), which is always aligned with respect
    to one of the baseline boundaries (motivated primarily by the trigger) of
    200, 250, 300, 350, 400, 600, 800\gev. This is the projected choice for 3\fbinv.}
  \label{tab:binning-3fb}
  \centering
  \footnotesize
  \begin{tabular}{ l|ccc|cccc|cccc|cccc }
    \hline
    \hline
    \njet      & \multicolumn{3}{c}{2} & \multicolumn{4}{c}{3} & \multicolumn{4}{c}{4} & \multicolumn{4}{c}{$\geq5$}                                                   \\ 
    \nb        & 0                     & 1                     & 2                     & 0   & 1   & 2   & 3   & 0   & 1   & 2   & $\geq3$ & 0   & 1   & 2   & $\geq3$ \\
    \hline
    Symmetric  & 800                   & 800                   & 600                   & 800 & 800 & 800 & 400 & 800 & 800 & 800 & 600     & 800 & 800 & 800 & 800     \\
    Asymmetric & 800                   & 600                   & 400                   & 800 & 600 & 400 & 300 & 600 & 600 & 600 & 400     & 600 & 600 & 600 & 400     \\
    \hline
    \hline
  \end{tabular}
\end{table}

{\bf Evolution of event categorisation and \HT binning versus integrated luminosity.} 

The choice of how events are categorised according to the number of
\njet and \nb and binned in \HT will evolve as a function of
integrated luminosity. Additional categories with higher jet
multiplicities maybe added for data samples corresponding to higher
integrated luminosities (\eg $\sim$10\fbinv). Additional categories
with higher b-tag multiplicities will also be considered once the
b-tag ``formula method'' (described above and detailed in
Ref.~\cite{Chatrchyan:2013lya}) has been commissioned with data. This
will allow for categories defined by the requirement of at least four
b-tagged jets. Additional bins at high \HT will also be considered for
higher integrated luminosity scenarios. 

\subsection{Lepton and photon vetoes \label{sec:vetoes}}

To select a sample of events in the hadronic final state and to
suppress SM processes with genuine \met from neutrinos, events
containing an isolated electron with $\pt > 10\GeV$ and $|\eta| < 2.5$
or an isolated muon with $\pt > 10\GeV$ and $|\eta| < 2.5$ are
vetoed. Further, to reduce the ``lost leptons'' backgrounds from \wj
and \ttbar, events containing single isolated tracks with $\pt >
10\GeV$ and $|\eta| < 2.5$, as defined in Section~\ref{sec:objects},
are vetoed as part of the signal region selection criteria. In the
case of the single and di-lepton control samples, a further
requirement is made such that events are not vetoed due to the
presence of a track from the well identified leptons, by requiring
$\Delta R(\textrm{track},\textrm{lepton}) > 0.02$.

Finally, to select a pure multijet topology and to allow for a
disjoint control region, events are vetoed in which an isolated photon
with $\pt > 25\GeV$ and $|\eta| < 2.5$ is identified.

\subsection{The hadronic signal region}
\label{sec:had-signal}

The lepton and photon vetoes are applied to select hadronic final
states. All pre-selection criteria are also applied. Following these
selections, the multijet background from QCD is still several orders
of magnitude larger than the typical signal expected from SUSY.

{\bf \HT-dependent \alphat requirements.}

Background events from multijet production populate the region
$\alphat \lesssim 0.5$ and therefore can be rejected with very high
efficiency by requiring an appropriate cut on \alphat. In the Run~1
analyses, a minimum requirement of $\alphat > 0.55$ was imposed, with
a raised threshold (as high as 0.65) used at low \HT to control
against a potential increase in contamination from multijet production
due to worsening resolution and jet \PT threshold effects.  (The
latter effect refers to ``fake'' \mht arising from the rare occurrence
of multiple soft jets below threshold that are relatively collinear.)
The minimum threshold of 0.55 was motivated primarily by the control
of the multijet background and considered conservative for mid to high
regions in \HT, while maintaining good signal acceptance. However,
improving jet resolutions and a reduced jet threshold effect relative
to an increasing \HT scale means that looser \alphat thresholds can be
used at higher values of \HT.

A useful approximate conversion between \alphat and \mht can be
obtained by calculating \alphat while forcing $\dht = 0\gev$, as
described by Equ.~\ref{eq:alphat3}. Hence, using this metric, the
dependence of the \alphat requirement as a function of the \HT bin can
be determined such that the effective requirement on \mht is
comparable, \ie roughly constant, across all \HT bins. The values
typically fall in the range $\sim110 < \mht < \sim160\gev$. This
approximate levelling of the ``effective'' \mht threshold implies
increasingly tighter requirements against instrumental effects versus
\HT, while maximising signal
acceptance. Table~\ref{tab:alphat-thresholds} summarises the expected
\alphat thresholds and corresponding ``effective'' \mht thresholds for
each \HT bin. The \alphat threshold is dependent only on \HT and not
on \njet nor \nb that are used to define the event categories.

\begin{table}[h!]
  \caption{\alphat and corresponding ``effective'' \mht (GeV) thresholds versus
    lower bound of \scalht bin. For all \HT bins satisfying $\HT > 800
    \gev$, a direct requirement of $\mht > 130\gev$ is imposed rather
    than a requirement on \alphat.}
  \label{tab:alphat-thresholds}
  \centering
  \footnotesize
  \begin{tabular}{ lcccccccc }
    \hline
    \hline
    \scalht            & 200       & 250       & 300       & 350       & 400       & 600       \\
    \hline                                                                     
    \alphat threshold  & 0.65      & 0.60      & 0.55      & 0.53      & 0.52      & 0.52      \\
    ``Effective'' \mht & $\sim$128 & $\sim$138 & $\sim$125 & $\sim$123 & $\sim$110 & $\sim$162 \\
    \hline
    \hline
  \end{tabular}
\end{table}

For Run~2, all signal region bins satisfying $\HT > 800\gev$ will be
seeded by the single-object \texttt{HLT\_HT800} trigger, which is
expected to be unprescaled. For these high \HT bins, no \alphat
threshold is required, which removes the inefficiencies of this
variable for high jet multiplicity events. Instead, the following
requirements are imposed to control the multijet background: $\mht >
130\gev$ and $\bdphi > 0.3$ (described below).

{\bf \bdphi requirement.} 

Further, an additional powerful variable \bdphi is used to suppress
multijet contamination due to both instrumental effects and
semi-leptonic heavy-flavour decays with genuine \met in the final
state. The variable is determined as follows. The jet-based estimate
of the missing transverse energy, $\vec{\mht}$, is recomputed while
ignoring one of the reconstructed jets (the ``test'' jet). The
difference in the azimuthal angle between the recomputed $\vec{\mht}$
and the ``test'' jet is then determined. This process is repeated for
each jet in the event and the minimum of all the azimuthal
differences, \bdphi, is determined. The ``test'' jet whose subtraction
from the calculation $\vec{\mht}$ yields this minimum value, is
identified as the jet that is most likely to have given rise to the
missing transverse energy in the event. Events with significant \mht
due to instrumental effects or heavy flavour decays populate the
region $\bdphi \approx 0$ and so candidate signal event are accepted
only if they satisfy $\bdphi > 0.3$. The use of the \bdphi and \alphat
variables provide an extremely powerful rejection factor against
contamination from multijet events and allow to maintain low jet \PT,
\HT, and \mht thresholds, which in turn maximises signal acceptance
for a large range of DM and SUSY models with final states
characterised by the presence of significant \met.

{\bf Additional cleaning filters.} 

Two additional dedicated cleaning cuts are applied. First, to protect
against multiple jets failing the $\Et$ threshold, the jet-based
estimate of the missing transverse energy, \mht, is compared to the
missing transverse energy variable, $\met$, and events with $R_{\rm
  miss}=\mht/\met > 1.25$ are rejected. Second, to protect against
severe energy losses caused by masked regions in the ECAL (which
amount to about 1\% of the ECAL channel count) or HCAL, or by missing
instrumentation in the barrel-endcap gap, events with $\bdphi < 0.5$
are rejected if the distance in the ($\eta,\phi$) plane between the
selected jet and the closest masked ECAL region, $\Delta R_{\rm
  ECAL}$, is smaller than 0.3. Similarly, events are rejected if the
jet points within 0.3 in $\eta$ of the ECAL barrel-endcap gap at
$|\eta| = 1.5$.

{\bf Summary of signal region selection.} 

The requirements that define the hadronic signal region are summarised
in Table~\ref{tab:sr-selections}.

\begin{table}[h!]
  \topcaption{Summary of the signal region selection criteria, applied
    in addition to the pre-selection summarised in
    Table~\ref{tab:pre-selections}.}
  \label{tab:sr-selections}
  \centering
  \footnotesize
  \begin{tabular}{ ll }
    \hline
    \hline
    Selection             & Requirement                                                    \\
    \hline
    \alphat               & $>$0.52--0.65 (\HT-dependent) for region $200 < \HT < 800\gev$ \\
    \mht                  & $>130\gev$ for region $\HT > 800\gev$                          \\  
    \bdphi                & $>0.3$                                                         \\
    \mht/\met             & $<1.25$                                                        \\
    ``Dead ECAL filter''  & (see text)                                                     \\
    \hline
    \hline
  \end{tabular}
\end{table}

\subsection{Adding the \texorpdfstring{\mht}{MHT} dimension}
\label{sec:had-shape}

As described above, and as used in Run~1, the analysis takes advantage
of three discriminating variables, \njet, \nb, and \HT, to provide
sensitivity to a large range of SUSY (and DM) models. No extrapolation
in these variables is performed, with predictions of SM background
yields in the (\njet,\nb,\HT) bins of the signal region based on both
observed counts and transfer factors derived from simulated yields in
the corresponding (\njet,\nb,\HT) bins of the control samples. Each
prediction is statistically and systematically independent.

In Run~1, for each (\njet,\nb,\HT) bin in the signal region, an
extrapolation in the variable \alphat was necessary to obtain
background predictions based on the muon control samples, which did
not impose any \alphat requirement. No extrapolation in \alphat was
performed for the photon control sample, which used the same \alphat
requirements as the signal region. The \alphat requirements used in
Run~1 for the signal region correspond loosely to \mht thresholds in
the range $\sim$130 to $\sim$500\gev depending on the \HT
bin. Uncertainties in this extrapolation were determined through
closure tests with respect to data, including one dedicated to the
\alphat extrapolation, plus additional cross checks.

In Run~2, we will additionally bin event counts in the signal region
according to the variable \mht in order to provide further
discriminating power between any potential signal and the SM
background counts. Hence, while no extrapolation is performed in
\njet, \nb, nor \HT, the analysis will rely on information obtained
from simulation to extrapolate from counts (integrated over \mht) in
the control samples to a predicted distribution in \mht for each
corresponding (\njet,\nb,\HT) bin in the signal region.

The \mht dimension is included in the likelihood model using templates
determined per (\njet,\nb,\HT) bin from simulation. An associated
normalisation nuisance is determined from closure tests between
simulation and data, as described in
Sec.~\ref{sec:systematics}. Alternative templates are used to encode
the systematic uncertainty in the \mht distribution obtained from
simulation. 
%Preliminary studies and estimates of the shape systematics are also
%discussed Sec.~\ref{sec:systematics}. The likelihood model is
%described in Sec.~\ref{sec:likelihood}.

The templates use \mht bins of 50\gev in width. A metric is used to
determine the threshold of the final \mht bin used by the
templates. This metric is currently based on requiring a minimum
number of both observed counts in the {\it data control samples} and
(unweighted) simulated events with \mht values higher than the final
bin threshold. While the counts in the (\njet,\nb,\HT) bins of each
data control sample will not be binned according to \mht (as in the
signal region), the former requirement ensures that there are
sufficient events in each (\njet,\nb,\HT) bin of the control samples
to probe for potential systematic effects {\it across all bins in
  \mht} using closure tests between simulation and data, as described
in Sec.~\ref{sec:systematics}. Hence, a data-driven cross check for
potential biases and on the systematic uncertainty in the \mht
template is possible. The latter requirement minimises the statistical
uncertainties associated with the finite number of simulated events
for the various SM background processes. Both requirement are based on
counts according to \njet and \nb but inclusive with respect to \nb.

\subsection{The hadronic control region}

A hadronic control region that is enriched in multijet events and
disjoint with respect to the signal region is obtained by applying
both the pre-selection criteria and lepton/photon vetoes, as defined
above, and inverting the (\HT-dependent) \alphat and/or \mhtmet
requirements. 
%None of the cleaning filters are not applied to allow the study of
%instrumental effects. 
The sample of events populating this control region are used primarily
to estimate any residual background contamination from QCD multijet
events, described in Sec.~\ref{sec:qcd}.

\subsection{Commonalities between the data control regions}

There are five control regions with leptons or photons in the final
state: \mj, \mmj, \ej, \eej, and \gj. The full pre-selection is
applied as part of the definition of each of these control
regions. The cuts on event-level jet-based quantities are identical to
those applied in the hadronic search region and the same \njet, \nb,
and \scalht binning is used. The lepton(s) or photon is not considered
in the calculation of the event-level variables.

The selection criteria of the various control regions are defined such
that the background composition and event kinematics of the control
regions mirror as closely as possible those for the signal
region. This is done in order to minimise the reliance on the
simulation to model correctly the backgrounds and event kinematics in
the control and signal samples.

Two exceptions are made. First, no \bdphi requirement is imposed as
part of the selection criteria defining the control regions. Second,
in the case of the four leptonic control regions, no requirement is
made on \alphat. This is made possible by the remaining kinematic
selection criteria, which are sufficiently selective to ensure that
the leptonic event samples remain rich in events from the \wj, \ttbar
and \zll processes with negligible contamination from QCD multijet
events. Thus, the acceptance of the leptonic control regions can be
significantly increased, which simultaneously improves their
predictive power and further reduces the effect of any potential
signal contamination.

The lepton event samples can be used to predict components of the SM
background across all \scalht bins, while the \gj sample can only be
used for the region $\HT > 400\gev$ due to the photon trigger
requirements.

\subsection{The \texorpdfstring{\mj}{muon plus jets} control sample}
\label{subsec:mucontrolSelection}

%Events from the \wj and \ttbar processes are found in the hadronic
%signal sample due to unidentified leptons (either out of acceptance or
%not reconstructed) and hadronic tau decays originating from
%high-p$_{T}$ W bosons. An estimate of these background processes is
%obtained through the use of a \mj sample. 

The selection criteria for the \mj sample are chosen to identify W
bosons decaying to a muon and a neutrino in the phase-space of the
signal. In order to select events containing W bosons, exactly one
tight isolated muon within an acceptance of \PT $>$ 30 \gev and
$|\eta| <$ 2.1 is required (due to the trigger), and the transverse
mass of the W candidate must satisfy $30 < \mt(\mu,\pfmet) < 125\gev$
(to suppress QCD multijet and potential signal events). Events are
vetoed if $\Delta R(\mu,\textrm{jet}_i) < 0.5$ running over all jets
$i$. The single isolated track veto, described in
Sections~\ref{sec:objects} and~\ref{sec:vetoes}, is also applied,
which considers all single isolated tracks in the event except that
associated with the identified, isolated muon. Finally, the cleaning
cut $\mht/\met$ is also applied, as done in the signal region, where
the \met is adjusted to account for the transverse momentum of the
identified, isolated muon.

\subsection{The \texorpdfstring{\mmj}{di-muon plus jets} control sample}

%The \znunu\ + jets process forms an irreducible background and can be
%estimated using the \zmumu + jets process, which has similar kinematic
%properties but a different acceptance and a smaller branching ratio. A
%background estimate is obtained through the use of a \mmj sample. 

The selection criteria are identical to those for the \mj sample, with
the following exceptions that are tuned to identify Z bosons decaying
to two muons in the kinematic phase space of the signal region. 
In order to select an event sample containing Z bosons, exactly two
tight isolated muons within an acceptance of $\Pt > 30\gev$ and
$|\eta| < 2.1$ are required (due to the trigger). The invariant mass
of the two muons must satisfy $m_{Z} - 25 < M_{\mu_1\mu_2} < m_{Z} +
25$ and they must have opposite charge. Events are vetoed if $\Delta
R(\mu_{i},\textrm{jet}_j) < 0.5$ is satisfied, running over all muons
$i$ and all jets $j$. The single isolated track veto is also applied
considering all single isolated tracks in the event except those
associated with the two identified, isolated muons. Finally, the
cleaning cut $\mht/\met$ is also applied, as done in the signal
region, where the \met is adjusted to account for the transverse
momenta of the two identified, isolated muons. 

\subsection{The \texorpdfstring{\ej}{electron plus jets} control sample}
\label{subsec:elecontrolSelection}

The selection criteria that define the \ej and \eej control samples
mirror those of the \mj and \mmj samples, respectively, and are tuned
to identify W and Z bosons, respectively, decaying to electrons in the
kinematic phase space of the signal region. 

Electrons are required to satisfy the Tight working point and satisfy
the requirements $\Pt> 30\gev$ and $|\eta| < 2.1$. The tightening of
the Loose working point defined in Sec.~\ref{sec:electron-id} was
found to greatly reduce multijet contamination without a large
reduction in statistics within the electron control samples.

\subsection{The \texorpdfstring{\gj}{photon plus jets} control sample}
\label{subsec:photoncontrolSelection}

%The \znunu\ + jets process can also be estimated using the \gj
%process, which has a larger cross section and kinematic properties
%similar to those of \znunu\ events when the photon is
%ignored~\cite{PAS-SUS-08-002,Bern:2011pa}. 

The \gj sample is defined by requiring exactly one photon satisfying
tight isolation criteria and within an acceptance of $\pt > 175\gev$
(limited by trigger requirements) and $|\eta| < 1.45$. Furthermore,
events are vetoed if $\Delta R(\gamma,\textrm{jet}_i) < 1.0$ is
satisfied, running over all jets $i$. One important difference with
respect to the leptonic control samples is the application of the
\HT-dependent \alphat requirements imposed as part of the signal
region definition. This is done primarily to ensure that the photon
control sample and signal region are subject to identical kinematic
requirements and the photon carries sufficient transverse energy so
that the mass of the Z boson becomes a negligible effect when using
the \gj sample to predict the kinematic distributions of the \znunu
background. The cleaning cut $\mht/\met$ is also applied, as done in
the signal region, where the \met is adjusted to account for the
transverse energy of the identified, isolated photon. As stated above,
the \gj sample can only be used to predict background components in
the region $\HT > 400\gev$ due to trigger requirements.
