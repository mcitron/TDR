%%____________________________________________________________________________||
\section{Interpretation in Simplified SUSY models}
\label{sec:susy}

\subsection{Signal models and efficiencies}
\label{subsec:susy_models}
To interpret the results of this search, simplified
models~\cite{Alwall:2008ag,Alwall:2008va,sms} are used. These
effective models use only a limited set of sparticles (production and
decay) and enable comprehensive studies of individual SUSY event
topologies. The simplified model studies can be performed in terms of
fundamental properties such as decay modes, production cross sections,
and sparticle masses. 
In order to gauge the early discovery potential, simplified models corresponding to both gluino and squark pair production 
are considered in this exercise. The full list of models considered is in table \ref{tab:simplified-models}. 

Signal efficiency times acceptance is determined per model per event
category (\njet,\nb) per \scalht bin. Systematic
uncertainties on the signal efficiencies are determined per model
following the 2012 analysis (see Section~\ref{sec:sig-syst}). Due to CPU
limitations, only a subset of event categories (but all \scalht bins
within a category) are considered for interpretation with each
simplified model and are listed in table \ref{tab:simplified-models}. 
The choice of which categories to include in the fit 
is made by comparing the expected upper limit on the signal cross section 
per model mass point achieved for each individual event category. 
The categories are then ranked according to their expected (exclusion) reach 
and the 10 most sensitive categories are used. 

\begin{table}[h!]
  \caption{A summary of the simplified models considered for
    interpretation. The event categories considered for each model are
    listed.}  
  \label{tab:simplified-models}
  \setlength{\extrarowheight}{2.5pt}
  \centering
  %% \begin{tabular*}{\textwidth}{ llcc }
  \begin{tabular*}{\textwidth}{ llcc }
    \hline
    \hline
    %% Model & Production/decay mode & ($m_{\rm SUSY},m_{\rm LSP}$) [GeV] & (\njet,\nb) event categories \\ 
    Model & Production/decay mode & ($m_{\rm SUSY},m_{\rm LSP}$) [GeV] \\%& (\njet,\nb) event categories \\ 
    \hline
    \texttt{SMS\_T1bbbb\_mGl1500\_mLSP100} & \Tonebbbb & (1500,100) \\
    \texttt{SMS\_T1bbbb\_mGl1000\_mLSP900} & \Tonebbbb & (1000,900) \\
    \texttt{SMS\_T1qqqq\_mGl1400\_mLSP100} & \Toneqqqq & (1400,100) \\
    \texttt{SMS\_T1qqqq\_mGl1000\_mLSP800} & \Toneqqqq & (1000,800) \\
    \texttt{SMS\_T1tttt\_mGl1500\_mLSP100} & \Tonetttt & (1500,100) \\
    \texttt{SMS\_T1tttt\_mGl1200\_mLSP800} & \Tonetttt & (1200,800) \\
    \texttt{SMS\_T2tt\_mStop650\_mLSP325}  & \Ttwott   & (650,550) \\
    \texttt{SMS\_T2tt\_mStop500\_mLSP325}  & \Ttwott   & (500,325) \\
    \texttt{SMS\_T2tt\_mStop425\_mLSP325}  & \Ttwott   & (425,325) \\
    \texttt{SMS\_T2qq\_mStop600\_mLSP550}  & \Ttwoqq   & (600,550) \\
    \texttt{SMS\_T2qq\_mStop1200\_mLSP100} & \Ttwoqq   & (1200,100) \\
    \texttt{SMS\_T2bb\_mStop900\_mLSP100}  & \Ttwobb   & (900,100) \\
    \texttt{SMS\_T2bb\_mStop600\_mLSP580}  & \Ttwobb   & (600,580) \\
    \hline
    \hline
  \end{tabular*}
\end{table}

\subsection{Systematic uncertainties on signal efficiency times}
\label{sec:sig-syst}

For the Phys14 exercise the systematic uncertainty for the signal region 
is assumed to be unchanged from that in the previous analysis. 
Thus a systematic uncertainty of $15\%$ is applied. The systematic 
uncertainty in the signal acceptance times efficiency is determined per mass 
point per event category (\njet,\nb) per
\scalht bin (inclusive on $\scalht > 200$). The effect of
uncertainties on the luminosity measurement, the parton distribution
functions, the jet energy scale, initial state radiation, and the
efficiencies of various cuts (including the \mht/\met filter, the
``dead ECAL'' filter, etc)
%and the lepton/photon vetoes)
used in the candidate signal event selection are considered. Each
contribution is considered to be independent and all contributions are
summed in quadrature to obtain a total systematic uncertainty per mass
point per category per \scalht bin. A 10\% uncertainty on the signal
acceptance is assumed. 


\subsection{Expected exclusion limits and discovery significance}
\label{subsec:susy_results}

%Tables~\ref{tab:t1bbbb_1500_100} and~\ref{tab:t1bbbb_1000_900} show
Table~\ref{tab:results} shows
the expected performance for the model \verb!T1bbbb! with two mass
scenarios, ($m_{\rm gluino} = 1500,m_{\rm LSP} = 100\gev$) and
($m_{\rm gluino} = 1000,m_{\rm LSP} = 900\gev$), and an integrated
luminosity of 4\fbinv. The tables list both significances and R-values
for the \scalht and \alphat variables, along with a range of
alternative variables, including \mht, \met, effective mass (\meff),
and finally the projection of the \mht vector onto the plane
transverse to the event thrust axis (\mhttt). The two most sensitive
bins in each variable are used, and each result is based on a
combination of bins from four events categories: (\njet,\nb) = (4,2),
(4,$\geq$3), ($\geq$5,2), and ($\geq$5,$\geq$3). All signal region
selection criteria are applied, including $\scalht > 900\gev$ and
$\mht > 130\gev$. The background systematic uncertainty is assumed to
be 30\% for this selection and a systematic uncertainty on signal
acceptance of 10\% is assumed. The results were produced with the
Higgs combination tool and rely on the use of \cls and
pseudo-experiments.

The results with the \scalht and \alphat requirements listed in 
%Tables~\ref{tab:t1bbbb_1500_100} and~\ref{tab:t1bbbb_1000_900} represent
Table~\ref{tab:results} represents 
the baseline performance with no optimisation beyond a minimal choice
in categories and binning. Further optimisations are possible, such as
increasing the number of categories and bins used in the statistical
interpretation. Some example alternatives to \scalht and \alphat are
also shown, which increase the significance to as much as
$\sim4\sigma$. Investigations concerning the choice of an additional
discriminating variable(s) are ongoing.

It should be noted that, while encouraging, such optimisations are
sensitive to the assumed level of control when extrapolating into the
tails of kinematic distributions, particularly for this class of
models (involving very heavy objects). Ultimately, these assumptions
are limited by event counts in data control samples, which are the
only way to validate the use of transfer factors (as done in this
analysis), or the accuracy of MC modelling of kinematic shapes, or the
assumptions implicit in the parameterisation of shapes in data or MC,
etc. The reliance on data-driven methods, cross-checks, and control
variables is of paramount important as we probe a new mass regime in
Run~2.

%\begin{table}
%  \centering
%  \caption{Significances and R-values for \texttt{T1bbbb} ($m_{\rm
%      gluino} = 1500, m_{\rm LSP} = 100\gev$).}
%  \label{tab:t1bbbb_1500_100}
%  \footnotesize
%  \begin{tabular}{lccccccccccc}
%    \hline
%    \hline
%    Variable & Binning                & Significance & R-value ($\pm1\sigma$ exptal.) \\
%    \hline
%    \scalht  & 1400--1700, $\geq$1700 & 1.9$\sigma$  & $1.08^{+0.69}_{-0.42}$         \\
%    \alphat  & 0.55--0.65, $\geq$0.65 & 2.1$\sigma$  & $1.04^{+0.74}_{-0.43}$         \\
%    \mhttt   & 130--200, $\geq$200    & 2.9$\sigma$  & $0.74^{+0.53}_{-0.27}$         \\
%    \meff    & 1750--2500, $\geq$2500 & 3.2$\sigma$  & $0.64^{+0.40}_{-0.27}$         \\
%    \mht     & 500--700, $\geq$700    & 4.1$\sigma$  & $0.45^{+0.30}_{-0.17}$         \\
%    \met     & 500--700, $\geq$700    & 4.2$\sigma$  & $0.43^{+0.28}_{-0.15}$         \\
%    \hline
%    \hline
%  \end{tabular} 
%\end{table}

%\begin{table}
%  \centering
%  \caption{Significances and R-values for \texttt{T1bbbb} ($m_{\rm
%      gluino} = 1000, m_{\rm LSP} = 900\gev$).}
%  \label{tab:t1bbbb_1000_900}
%  \footnotesize
%  \begin{tabular}{lccccccccccc}
%    \hline
%    \hline
%    Variable & Binning                & Significance & R-value    ($\pm1\sigma$ exptal.) \\
%    \hline
%    \scalht  & 1400--1700, $\geq$1700 & 0.2$\sigma$  & $8.93^{+6.26}_{-3.87}$            \\
%    \alphat  & 0.55--0.65, $\geq$0.65 & 2.2$\sigma$  & $0.93^{+0.63}_{-0.40}$            \\
%    \mhttt   & 130--200, $\geq$200    & 1.0$\sigma$  & $2.38^{+1.60}_{-0.98}$           \\
%    \meff    & 1750--2500, $\geq$2500 & 0.7$\sigma$  & $2.90^{+1.77}_{-1.24}$            \\
%    \mht     & 500--700, $\geq$700    & 3.1$\sigma$  & $0.62^{+0.38}_{-0.24}$            \\
%    \met     & 500--700, $\geq$700    & 3.2$\sigma$  & $0.60^{+0.36}_{-0.24}$            \\
%    \hline
%    \hline
%  \end{tabular} 
%\end{table}

\begin{table}
  \centering
  \caption{Significances and R-values for various models.}
  \label{tab:results}
  \footnotesize
  \begin{tabular}{llcccc}
    \hline
    \hline
    &  & \multicolumn{2}{c}{\texttt{T1bbbb} (1500,100)} & \multicolumn{2}{c}{\texttt{T1bbbb} (1000,900)} \\
    Variable & Binning                & Significance & R-value ($\pm1\sigma_{\rm exptal}$) & Significance & R-value ($\pm1\sigma_{\rm exptal}$) \\
    \hline                                                                                                                              
    \scalht  & 1400--1700, $\geq$1700 & 1.9$\sigma$  & $1.08^{+0.69}_{-0.42}$              & 0.2$\sigma$  & $8.93^{+6.26}_{-3.87}$              \\
    \alphat  & 0.55--0.65, $\geq$0.65 & 2.1$\sigma$  & $1.04^{+0.74}_{-0.43}$              & 2.2$\sigma$  & $0.93^{+0.63}_{-0.40}$              \\
    \mhttt   & 130--200, $\geq$200    & 2.9$\sigma$  & $0.74^{+0.53}_{-0.27}$              & 1.0$\sigma$  & $2.38^{+1.60}_{-0.98}$              \\
    \meff    & 1750--2500, $\geq$2500 & 3.2$\sigma$  & $0.64^{+0.40}_{-0.27}$              & 0.7$\sigma$  & $2.90^{+1.77}_{-1.24}$              \\
    \mht     & 500--700, $\geq$700    & 4.1$\sigma$  & $0.45^{+0.30}_{-0.17}$              & 3.1$\sigma$  & $0.62^{+0.38}_{-0.24}$              \\
    \met     & 500--700, $\geq$700    & 4.2$\sigma$  & $0.43^{+0.28}_{-0.15}$              & 3.2$\sigma$  & $0.60^{+0.36}_{-0.24}$              \\
    \hline
    \hline
  \end{tabular} 
\end{table}



%%____________________________________________________________________________||
