%%____________________________________________________________________________||
\section{Intepretation in Simplified SUSY models}
\label{sec:susy}
\textbf{FIXME: Preliminary cut&paste}\\

\subsection{Signal models and efficiencies}
\label{subsec:susy_models}
To interpret the results of this search, simplified
models~\cite{Alwall:2008ag,Alwall:2008va,sms} are used. These
effective models use only a limited set of sparticles (production and
decay) and enable comprehensive studies of individual SUSY event
topologies. The simplified model studies can be performed in terms of
fundamental properties such as decay modes, production cross sections,
and sparticle masses. As gluinos provide an early discovery potential, the model
T1bbbb is used here to demonstrate sensitivity.

Signal efficiency times acceptance is determined per model per event
category (\njet,\nb) per \scalht bin. Systematic
uncertainties on the signal efficiencies are determined per model
following the 2012 analysis (see Section~\ref{sec:sig-syst}). Due to CPU
limitations, only a subset of event categories (but all \scalht bins
within a category) are considered for interpretation with each
simplified model. 

%The efficiency time acceptance for an inclusive

%selection on \scalht and for the relevant categories are shown.
The simplified models considered for the Phys14 exercise are summarised in
Table~\ref{tab:simplified-models}. The event categories considered for
each model interpretation are listed. The choice of which categories
is made by comparing the expected upper limit on the signal cross section 
per model mass point achieved for each individual event category. 
The categories are then ranked according to their expected (exclusion) reach 
and the most sensitive categories are used. 

%The second method is based on a signal
%injection test that highlights the expected signal significance per
%signal region bin. The event categories that demonstrate the largest
%significances across all \scalht bins are chosen. The former method is
%presently used.

\begin{table}[h!]
  \caption{A summary of the simplified models considered for
    interpretation. The event categories considered for each model are
    listed.}  
  \label{tab:simplified-models}
  \setlength{\extrarowheight}{2.5pt}
  \centering
  \begin{tabular}{ llccc }
    \hline
    \hline
    Model                  & Production/decay mode & ($m_{\rm SUSY},m_{\rm LSP}$) [GeV]  & (\njet,\nb) event categories considered        \\ 
    \hline
    \texttt{T1bbbb}           & \Tonebbbb & (1500,100)               & (4,2), (4,$\geq 3$), ($\geq 5$,2), ($\geq 5$,$\geq 3$) \\ % (2--3,1), 
    % \texttt{T2bw (x=0.25)}  & \Ttwobw               & FIXME: ANA-CATS \\
    % \texttt{T2bw (x=0.75)}  & \Ttwobw               & FIXME: ANA-CATS \\
    % \texttt{T2tt}           & \Ttwott               & FIXME: ANA-CATS \\
    \hline
    \hline
  \end{tabular}
\end{table}

\subsection{Systematic uncertainties on signal efficiency times}
\label{sec:sig-syst}

For the Phys14 exercise the systematic uncertainty for the signal region 
is assumed to be unchanged from that in the previous analysis. 
Thus a systematic uncertainty of $10\%$ is applied. The systematic 
uncertainty in the signal acceptance times efficiency is determined per mass 
point per event category (\njet,\nb) per
\scalht bin (inclusive on $\scalht > 200$). The effect of
uncertainties on the luminosity measurement, the parton distribution
functions, the jet energy scale, initial state radiation, and the
efficiencies of various cuts (including the \mht/\met filter, the
``dead ECAL'' filter, etc)
%and the lepton/photon vetoes)
used in the candidate signal event selection are considered. Each
contribution is considered to be independent and all contributions are
summed in quadrature to obtain a total systematic uncertainty per mass
point per category per \scalht bin. A 10\% uncertainty on the signal
acceptance is assumed. 


\subsection{Expected exclusion limits and discovery significance}
\label{subsec:susy_results}


%%____________________________________________________________________________||
