%%____________________________________________________________________________||
\section{Introduction}
\label{sec:intro}

In this note, we summarise the Early Analysis exercise for the \alphat
analysis, which searches for signatures of physics beyond the standard
model in events with jets and missing transverse momentum (\met). This
analysis uses the kinematic variable \alphat to efficiently select
candidate signal events with genuine missing transverse momentum while
providing robust rejection against QCD multijet background
events. With the \alphat variable, CMS has been searching for
supersymmetry (SUSY) in proton-proton collisions data collected during
LHC Run~1. With data at a centre-of-mass energy of 7 TeV collected in
2010 and 2011, the \alphat analysis has excluded a large parameter
space of the constrained minimal supersymmetric extension of the
standard model (CMSSM) \cite{Khachatryan:2011tk, Chatrchyan:2011zy,
  Chatrchyan:2012wa} and a parameter space of simplified models
\cite{Chatrchyan:2012wa}. With data at a centre-of-mass energy of 8
TeV collected and promptly reconstructed in 2012, the \alphat analysis
further excluded a parameter space of simplified models
\cite{Chatrchyan:2013lya}. Additional sets of data which were
collected in 2012 but were reconstructed later during the LHC Long
Shutdown 1 (LS1) are called the ``parked data'', which contain data
for events triggered with lower energy
thresholds~\cite{CMS_AN_2013-366}.
%The review of the \alphat analysis of the parked data \cite{CMS_AN_2013-366} is ongoing.  
The search strategy in this note is an extension of that in Ref.
\cite{CMS_AN_2013-366}.

After two years of the LS1, in June 2015, LHC started its Run 2 and is
delivering proton-proton collisions to CMS at a higher centre-of-mass
energy of 13 TeV. The higher energy collision considerably increases
the production cross sections of heavy particles. It is highly
motivated to continue the search for supersymmetry. The \alphat
analysis is particularly suited for this search and has potential for
discovery in early data. The reconstruction of missing transverse
momentum in different collider environment or at new collider energy
is typically very challenging and often requires an extended period of
development. However, the \alphat variable is designed to provide
robust discriminating power between sources of genuine and ``fake''
(\eg instrumental) \met, as well as using an estimator of \met that
relies solely on the vector sum of jet transverse momenta (\mht).
%can effectively select events with missing transverse momentum without
%directly using the reconstructed missing transverse momentum.

The identity of dark matter (DM) is one of the outstanding problems in
particle physics. The production of DM at LHC Run~2 has been predicted
by many models of physics beyond the standard model, by both
supersymmetric and non-supersymmetric models. The \alphat variable
efficiently selects events that potentially contain DM candidates
produced in the collisions. In Run~2, we extend the \alphat analysis
to significantly improve the acceptance to dark matter production at
the LHC.

The study in this note started as part of the PHYS14 exercise
\cite{PHYS14}. The PHYS14 exercise, initiated in October 2014, was
focused on preparation of the Run~2 physics analyses. In this note, we
use MC event samples generated for this exercise. Events in these
samples are simulated under conditions anticipated in early LHC Run~2,
e.g, the centre-of-mass energy, the bunch-spacing, the number of the
interactions per bunch crossing.

Section \ref{sec:strategy} discusses changes in the search methods
made since Run~1. Section \ref{sec:alphatdef} defines the \alphat
variable.  Section \ref{sec:datasets} lists the data sets used in this
note.  Section \ref{sec:triggers} describes the preparation of the
triggers for Run~2. Section \ref{sec:objects} defines the physics
objects used in this note. Section \ref{sec:selection} describes how
events are selected. Section \ref{sec:yields} shows the expected event
yields in the signal and control regions. While Section \ref{sec:qcd}
discusses how multijet background events are controlled, Section
\ref{sec:backgroundmet} estimates background from other
processes. Then, Section \ref{sec:systematics} discusses systematic
uncertainties in the estimates. With likelihood models introduced in
Section \ref{sec:likelihood}, we interpret the results in simplified
SUSY models in Section \ref{sec:susy} and Dark Matter models in
\ref{sec:darkmatter}. Section \ref{sec:summary} summarises this
exercise.

%%____________________________________________________________________________||
