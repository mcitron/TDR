%%____________________________________________________________________________||
\section{Interpretation in Dark Matter models}
\label{sec:darkmatter}

In Run~1, the \alphat analysis searched for supersymmetry. Its results
were used to interpret models with supersymmetry and their simplified
models. In Run~2, we plan to extend this analysis to include searches
for the production of dark matter (DM) in the proton-proton collisions.

The properties of DM and their connections to known particles and
physical laws remain unknown. Although the Weakly Interacting Massive
Particles (WIMPs) hypothesis has guided the quest to understand this
fundamental problem, rapid experimental progress continues to exclude
large regions of the most popular dark matter models. In light of this
uncertainty, more model-independent searches for dark matter have gained
significant traction. It is important to be able to combine collider,
direct detection, and indirect detection searches in a complementary way
if we are to determine that a signal observed experimentally is indeed
ssssfrom dark matter \cite{Bauer:2013ihz}.

In particular, the use of simplified models provides allows one to relate the different experimental
signatures in a relatively model-independent way~\cite{Buchmueller:2014yoa}. 

%Here it is assumed that new particles mediate the interactions between dark matter and standard model
%particles, but that these mediators are too heavy to be produced directly in experiments. Hence, the interactions can be described by a contact
%operator \cite{Beltran:2010ww}. For each operator and dark matter mass $m_\textrm{DM}$ the relic abundance, direct detection signal, and collider predictions depend on a single parameter $M_*$, which parameterizes the coupling strength of the contact interaction. 


The collider search for dark matter is a critical element of this
approach, and can provide the strongest constraints in cases where the
dark matter mass is $\lesssim 10\ \GeV$ or when the operator has
suppressed direct detection signals, e.g. in the case of
isospin-violating dark matter~\cite{Feng:2011vu} or pseudo-scalar
couplings~\cite{Buckley:2014fba}. The generic signature of dark matter
pair production in colliders is missing transverse momentum from the
dark matter and energetic visible particles that are used to tag the
event. First analyses using this final state used contact
operators~\cite{Goodman:2010ku} with couplings between dark matter and
quarks. The resulting ``monojet'' final state consisting of one or two
hard jets plus large missing transverse momentum from the dark matter.
Experimental limits using monojet final states have been published using
7 and 8 TeV LHC data \cite{Chatrchyan:2012me,ATLAS:2012ky} for a variety
of operators.
%In addition, similar final states with the form $\chi \bar \chi + X$, where $\chi$ is the dark matter particle
%and $X$ can be a photon, jet, or other particle, have been studied in the context of effective operators These additional particles are initial state radiation (ISR) 
%radiated off from the interacting partons. (e.g. \cite{Goodman:2010yf,Fox:2011pm,Petriello:2008pu,Fox:2011fx}).


%Previous CMS analyses performed the search for new physics in the monojet final state, as detailed in References~\cite{Aad:2011xw, ATLAS:2012ky} using up to $4.7\,\fbi$ 
%of data with proton-proton collision at a center-of-mass energy of $\sqrt{s}=7\,\tev$. A conference note~\cite{ATLAS-CONF-2012-147} using 8$\,\tev$ data has also been published, 
%based on half of the 2012 dataset. 


The present document also describes the planned analysis of WIMP pair production in association with light and heavy quarks using the upcoming 13~TeV data set corresponding. 
Hadronic final states of the form $\chi \bar \chi + X$, where $\chi$ is the dark matter particle and $X$ can be one or several jets have been studied in the context of effective operators. These jets are either directly produced due in the interaction between SM and DM or are initial state radiation (ISR)  radiated off from the interacting partons (e.g. \cite{Goodman:2010yf,Fox:2011pm,Petriello:2008pu,Fox:2011fx}). As detailed in Ref.~\cite{Lin:2013sca, Artoni:2013zba} particular the use of heavy quarks, namely $b$- and top quarks take advantage of quark-mass dependency of scalar couplings. The quark mass dependency comes from minimal flavor violation assumptions. Further advantages gained by the use of third generation quarks is to extend the analysis to larger jet multiplicities and therefore larger signal acceptance compared to the monojet analysis. Thus accessing a unique and orthogonal phase space. 


This analysis will also set strongest constraints for low mass dark matter, and the strongest collider constraints across a wide range of masses. We will also start to probe excesses observed in direct detection experiment at energies of about 10 GeV in the DAMA (2008), CoGent (2010/11), CREST-II (2012) and CDMS (2013) experiments but also by the Fermi-LAT (2013) satellite indicating a DM particle of about 50 GeV mass. We expect to place stringent constraints on this phase space in the near future with the $13$~TeV run.


\subsection{Signal models and efficiencies}
\label{subsec:darkmatter_models}


\subsection{Expected exclusion limits and discovery significance}
\label{subsec:darkmatter_results}


%%____________________________________________________________________________||
