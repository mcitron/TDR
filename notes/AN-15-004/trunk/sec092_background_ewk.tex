%%____________________________________________________________________________||

\section{Estimation for processes with genuine \met}


In the absence of multijet events, the background counts in the signal
region arise from SM processes with significant \met in the final
state. In events with low counts of jets and b-quark jets, the largest
backgrounds with genuine \met are from the associated production of W
or Z bosons with jets, followed by either the weak decays \znunu or
\wtaunu, where the $\tau$ decays hadronically and is identified as a
jet; or by leptonic decays that are not rejected by the dedicated
electron or muon vetoes. The veto of events containing isolated tracks
is efficient at further suppressing these backgrounds as well as the
single-prong hadronic decay of the tau lepton. At higher jet and
b-quark jet multiplicities, top quark production followed by
semileptonic weak top quark decay becomes important. 
%Residual contributions from processes such as single-top-quark,
%diboson, and Drell-Yan production are also expected.

The production of W and Z bosons in association with jets is simulated
with the \MADGRAPH V5~\cite{madgraph} event generator. The production
of \ttbar and single-top quark events is generated with
\POWHEG~\cite{powheg}, and diboson events are produced with
\PYTHIA6.4~\cite{pythia}. For all simulated samples, \PYTHIA6.4 is
used to describe parton showering and hadronisation. All samples are
generated using the \textsc{cteq6l1}~\cite{Pumplin:2002vw} parton
distribution functions (PDF). The description of the detector response
is implemented using the \GEANTfour~\cite{geant} package. The
simulated samples are normalised using the most accurate cross section
calculations currently available, usually with
next-to-next-to-leading-order (NNLO) accuracy. To model the effects of
pileup, the simulated events are generated with a nominal distribution
of pp interactions per bunch crossing and then reweighted to match the
pileup distribution as measured in data.

The method to estimate the non-multijet backgrounds in the signal
region relies on the use of transfer factors (TF), which are
constructed per bin (in terms of \njet, \nb, and \scalht) per data
control sample. The TFs are determined from the simulated event
samples and are ratios of expected yields in the corresponding bins of
the signal region and control samples. The TFs are used to extrapolate
from the event yields measured in a data control samples to an
expectation for the total background event yields in the signal
region. 
%Two independent estimates of the irreducible background of \znunu +
%jets events are determined from the \gj and \mmj data control samples,
%both of which have similar kinematic properties when the muons or
%photon are ignored~\cite{Bern:2011pa} but different acceptances. The
%\gj process has a larger production cross section while the \mmj
%process has kinematic properties that are more similar to $\znunu +
%\text{jets}$.

Two independent estimates of the irreducible background of \znunu +
jets events are determined from the \gj and \mmj data control samples.
The \gj and \zmumu + jets processes have similar kinematic properties
when the photon or muons are ignored~\cite{Bern:2011pa} albeit
different acceptances. In addition, the \gj process has a larger
production cross section than \znunu + jets events.

The \mj data sample provides an estimate for the total
contribution from all other SM processes, which is dominated by \ttbar
and W-boson production. Residual contributions from processes such as
single-top-quark, diboson, and Drell-Yan production are also
estimated. For the category of events satisfying $\nb \geq 2$, in
which the contribution from $\znunu + \text{jets}$ events is
suppressed to a negligible level, the \mj sample is also used to
estimate this small contribution rather than using the statistically
limited \mmj and \gj samples. Hence, only the \mj sample is used to
estimate the total SM background for events satisfying $\nb \geq 2$,
whereas all three data control samples are used for events satisfying
$\nb \leq 1$.

In order to maximise sensitivity to potential new physics signatures
in final states with a high b-quark content, a method that improves
the statistical power of the predictions from simulation, particularly
for $n_\cPqb \ge 2$, is employed~\cite{RA1Paper2012}. The distribution
of $n_\cPqb$ is estimated from generator-level information contained
in the simulation, namely the number of reconstruction-level jets
matched to underlying bottom quarks ($n_\cPqb^\text{gen}$), charm
quarks ($n_\cPqc^\text{gen}$), and light-flavoured partons
($n_\cPq^\text{gen}$) per event. All relevant combinations of
$n_\cPqb^{\rm gen}$, $n_\cPqc^{\rm gen}$, and $n_{\cPq}^{\rm gen}$ are
considered, and event counts are recorded according to the
categorisation of events in terms of \njet and \scalht. The efficiency
$\epsilon$ with which b-quark jets are identified and the mistag
probabilities $f_\cPqc$ and $f_\cPq$ are also determined from
simulation per event category, with each quantity averaged over jet
\pt and $\eta$. Corrections are applied on a jet-by-jet basis to
$\epsilon$, $f_\cPqc$, and $f_\cPq$ in order to match the
corresponding measurements from data~\cite{CMS-PAS-BTV-12-001}. This
information is sufficient to predict $n_\cPqb$ and thus also determine
the event yield from simulation for a given event category. The event
yields for a given b-quark jet multiplicity can be predicted with a
higher statistical precision than obtained directly from simulation,
particularly for events with high counts of b-quark jets, and are
subsequently used to determine the transfer factors binned according
to \nb (in addition to \njet and \scalht).

\subsection{Introduction}

The remaining backgrounds in the signal region, after the imposition
of \alphat and \mhtmet cuts, are from SM processes with genuine
\met. The main backgrounds include $\znunu\ + \textrm{jets}$, \wj,
\zj, \ttbar. Contributions from SM processes such as single-top,
Drell-Yan, and diboson production are also expected.

To estimate the contributions from these backgrounds, three data
control sample are used, which are binned identically to the signal
region: $\mu$ + jets, $\mu\mu$ + jets and $\gamma$ + jets. Their
definitions are in Section \ref{sec:selection}. $\mu\mu$ + jets and
$\gamma$ + jets control sample are used to estimate the irreducible
background \znunu + jets events, while $\mu$ + jets control sample is
used to estimate all other SM processes. The selection criteria for
these control regions are defined such that any potential
contaminations from SUSY models and QCD multijets are negligible. The
possiblity to add an additional control sample with $e$ + jets is
being investigated.

\subsection{Method}
\label{sec:ewk-method}

The method to estimate these SM background contributions relies on the
use of a transfer factor. A transfer factor is the ratio of the yields
obtained from MC simulation, defined for each \scalht, $n_{jet}$,
$n_b$ bin of a control sample:

\begin{equation}
  \label{equ:tf-ratio}
  {\rm TF} = \frac{N_{\rm MC}^{\rm signal}(\scalht,\njet,\nb)}{N_{\rm
      MC}^{\rm control}(\scalht,\njet,\nb)} 
\end{equation}

where $\npre^{\rm signal}(\scalht,\njet,\nb)$ is the predicted yield
for the corresponding bin of the signal region, and $\npre^{\rm
  control}(\scalht,\njet,\nb)$ is the one for the control region.

By this transfer factor, the ``na\"ive'' prediction for the total SM
background can be calculated, with $\nobs^{\rm
  control}(\scalht,\njet,\nb)$, the observed yield in the control
region, by:

\begin{equation}
  \label{equ:pred-method}
  \npre^{\rm signal}(\scalht,\njet,\nb) = \frac{N_{\rm MC}^{\rm
      signal}(\scalht,\njet,\nb)}{N_{\rm MC}^{\rm
      control}(\scalht,\njet,\nb)} \times \nobs^{\rm
    control}(\scalht,\njet,\nb)   
\end{equation}

When constructing the transfer factors, the MC expectations for the
following SM processes are considered: W + jets ($N_{\rm W}$), \ttbar
+ jets ($N_{\ttbar}$), \znunu\ + jets ($N_{\znunu}$), DY + jets
($N_{\mathrm DY}$), \gj ($N_\gamma$), single top + jets production via
the s, t, and tW-channels ($N_{\rm top}$), and WW + jets, WZ + jets,
and ZZ + jets ($N_{\rm di-boson}$). Details on the MC samples used are
given in Sec.~\ref{sec:datasets}. All MC samples are normalised to
appropriate intergrated luminosities for PHYS14 studies.

The selection criteria for the three control samples closely resemble
those for the signal region, differing mainly through the use of a
muon, di-muon, or photon {\it tag} (that is ignored in the calculation
of jet-based kinematic variables such as \scalht, \mht, \alphat, \etc)
and some minimal additional kinematic requirements (\eg invariant or
transerve mass windows) to obtain W, Z, and \ttbar-enriched event
samples. The same selection criteria are designed to suppress signal
contamination in the control samples so that unbiased data-driven
estimates for the SM backgrounds in the signal region can be
made. More detail on the selection criteria can be found in Section
\ref{sec:selection}.

Many systematic effects are expected to cancel largely in the transfer
factor. However, a systematic uncertainty is assigned to each transfer
factor to account for theoretical uncertainties and effects such as
the mismodelling of kinematics (\eg acceptances) and instrumental
effects (\eg reconstruction efficiencies).

In the end, a fitting prcedure that provides the final result is
defined formally by the likelihood model described in
Sec.~\ref{sec:sensitivity}. In summary, the observation in each bin
(defined in terms of the variables \njet, \nb, and \scalht) of the
signal sample is modelled as Poisson-distributed about the sum of a SM
expectation (and a potential signal contribution). The components of
this SM expectation are related to the expected yields in the control
samples via transfer factors derived from simulation. The observations
in each bin (again defined by \njet, \nb, and \scalht) of the control
samples are similarly modelled as Poisson-distributed about the
expectated yields for each control sample. In this way, for a given
bin, the observed yields in the signal and control samples are all
connected via the transfer factors derived from simulation.
