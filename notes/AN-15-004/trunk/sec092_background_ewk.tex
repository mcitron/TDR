%%____________________________________________________________________________||

\section{Background estimation for processes with genuine \met}
\label{sec:backgroundmet}}
\subsection{Overview of SM background processes}

Once all the signal region selection requirements have been imposed,
the contribution from QCD multijet events is expected to be
negligible, as demonstrated in Section~\ref{}. In the absence of
multijet events, the background counts in the signal region arise from
SM processes with significant \met in the final state. In events with
low counts of jets and b-quark jets, the largest backgrounds with
genuine \met are from the associated production of W or Z bosons with
jets, followed by either the weak decays \znunu or \wtaunu, where the
$\tau$ decays hadronically and is identified as a jet; or by leptonic
decays that are not rejected by the dedicated electron or muon
vetoes. The veto of events containing isolated tracks is efficient at
further suppressing these backgrounds as well as the single-prong
hadronic decay of the tau lepton. At higher jet and b-quark jet
multiplicities, top quark production followed by semileptonic weak top
quark decay becomes important.  Residual contributions from processes
such as single-top-quark, $\ttbar$V or $\ttbar$H, diboson, and
Drell-Yan production are also expected. These SM processes are
collectively referred to as the non-multijet backgrounds.

%The production of W and Z bosons in association with jets is simulated
%with the \MADGRAPH V5~\cite{madgraph} event generator. The production
%of \ttbar and single-top quark events is generated with
%\POWHEG~\cite{powheg}, and diboson events are produced with
%\PYTHIA6.4~\cite{pythia}. For all simulated samples, \PYTHIA6.4 is
%used to describe parton showering and hadronisation. All samples are
%generated using the \textsc{cteq6l1}~\cite{Pumplin:2002vw} parton
%distribution functions (PDF). The description of the detector response
%is implemented using the \GEANTfour~\cite{geant} package. 
The simulated samples are normalised using the most accurate cross
section calculations currently available, usually with
next-to-next-to-leading-order (NNLO) accuracy. To model the effects of
pileup, the simulated events are generated with a nominal distribution
of pp interactions per bunch crossing, which will then be reweighted
to match the pileup distribution as measured in data. 

\subsection{The ``transfer factor'' method}
\label{sec:ewk-method}

The method used to estimate the aforementioned SM background
contributions in the hadronic signal region relies on the use of a
transfer factor (TF) determined from MC samples to transform the
observed yield in a given \scalht, jet (\njet) and b-tag (\nb)
multiplicity bin of a control sample, $\nobs^{\rm
  control}(\njet,\nb,\scalht)$, into a predicted yield for the
corresponding bin of the hadronic signal region, $\npre^{\rm
  signal}(\njet,\nb,\scalht)$. The choice of \njet and \nb event
categorisation and \scalht binning in the control samples is identical
to that for the signal region, as defined in
Section~\ref{sec:selection}. 

Each transfer factor is simply a ratio of the yields obtained from MC
simulation for the same bin of the signal region and a given control
sample:

\begin{equation}
  \label{equ:tf-ratio}
  {\rm TF} = \frac{N_{\rm MC}^{\rm signal}(\njet,\nb,\scalht)}{N_{\rm
      MC}^{\rm control}(\njet,\nb,\scalht)} 
\end{equation}

In this way, predictions of background counts from SM processes can be
made based on the various control samples:

\begin{equation}
  \label{equ:pred-method}
  \npre^{\rm signal}(\njet,\nb,\scalht) = \frac{N_{\rm MC}^{\rm
      signal}(\njet,\nb,\scalht)}{N_{\rm MC}^{\rm
      control}(\njet,\nb,\scalht)} \times \nobs^{\rm
    control}(\njet,\nb,\scalht)   
\end{equation}

When constructing the transfer factors, the MC expectations for the
following SM processes are considered: W + jets ($N_{\rm W}$), \ttbar
+ jets ($N_{\ttbar}$), \znunu\ + jets ($N_{\znunu}$), DY + jets
($N_{\mathrm DY}$), \gj ($N_\gamma$), single top + jets
production via the s, t, and tW-channels ($N_{\rm top}$), WW +
jets, WZ + jets, and ZZ + jets ($N_{\rm di-boson}$), and $\ttbar$V or
$\ttbar$H ($N_{\rm {\ttbar}X}$). Details on the MC
samples used are given in Sec.~\ref{sec:mc-samples}. All MC samples
are normalised to the intergrated luminosity of the appopriate data
sample.

The selection criteria for the data control samples closely resemble
those for the signal region, differing mainly through the use of a
lepton or photon object {\it tag} (that is ignored in the calculation
of jet-based kinematic variables such as \scalht, \mht, \alphat, \etc)
and minimal additional kinematic requirements (\eg invariant or
transerve mass windows) to obtain W, Z, and \ttbar-enriched event
samples. The same selection criteria are designed to suppress signal
contamination in the control samples so that unbiased data-driven
estimates for the SM backgrounds in the signal region can be
made. More detail on the selection criteria can be found in Section
\ref{sec:selection}.

The transfer factors account for differences in cross sections and
branching ratios, acceptance and reconstruction efficiencies, and/or
kinematic requirements between the signal and control regions. Any
dependence on \njet, \nb, or \HT is largely attributible to
differences in acceptance due to the presence or otherwise of \alphat
or \mht requirements.

Many systematic effects are expected to cancel largely in the transfer
factor. However, a systematic uncertainty is assigned to each transfer
factor to account for theoretical uncertainties and effects such as
the mismodelling of kinematics (\eg acceptances) and instrumental
effects (\eg reconstruction efficiencies).

In the end, a fitting prcedure that provides the final result is
defined formally by the likelihood model described in
Sec.~\ref{sec:sensitivity}. In summary, the observation in each bin
(defined in terms of the variables \njet, \nb, and \scalht) of the
signal sample is modelled as Poisson-distributed about the sum of a SM
expectation (and a potential signal contribution). The components of
this SM expectation are related to the expected yields in the control
samples via transfer factors derived from simulation. The observations
in each bin (again defined by \njet, \nb, and \scalht) of the control
samples are similarly modelled as Poisson-distributed about the
expectated yields for each control sample. In this way, for a given
bin, the observed yields in the signal and control samples are
connected via the transfer factors derived from simulation. 

\subsection{Adding the \mht dimension}

The aforementioned description of the TFs provide an estimate of the
total SM background as a function of the (\njet,\nb,\HT) bin that is
integrated over \mht. However, the analysis takes advantage of \mht
distribution obtained from simulation. This information is propagated
to the likelihood model via an \mht template per (\njet,\nb,\HT) bin,
which is equivalent to dicing the numerator of the TF according to
\mht, \ie $N_{\rm MC}^{\rm signal}(\njet,\nb,\scalht,\mht)$. In this
regard, the TFs described above provide an estimate of the
normalisation for each \mht template.

\subsection{Data control samples used in the method}

To estimate the contributions from these backgrounds, five data
control regions are used, which are binned identically to the signal
region: \mj, \mmj, ej, \eej, and \gj. Their definitions are provided
in Section \ref{sec:selection}. The selection criteria for these
control regions are defined such that any potential contamination from
new physics processes or QCD multijets is negligible.

Currently, the projected sensitivity of the analysis is based on
predictions of SM background components made from control regions as
follows. For events containing exactly zero or one b-tagged jets, the
\ej (enriched in \wej) and \gj control samples are used to estimate
the irreducible \znunu + jets background, while the \mj control sample
is used to estimate all remaining SM processes (predominately \wj and
\ttbar). For events containing two or more b-tagged jets, both the \mj
and \ej samples are used to predict the total SM background (dominated
by \ttbar). 

It is planned to use also the \mj sample (enriched in \wmj) to predict
the \znunu + jets background (as well as \wej and \gj events) and also
the \ej sample to predict the \wj and \ttbar backgrounds plus residual
contributions (as well as \mj).

%The \gj and \zmumu + jets processes have similar kinematic properties
%when the photon or muons are ignored~\cite{Bern:2011pa} albeit
%different acceptances. In addition, the \gj process has a larger
%production cross section than \znunu + jets events.

%The \mj data sample provides an estimate for the total contribution
%from all other SM processes, which is dominated by \ttbar and W-boson
%production. Residual contributions from processes such as
%single-top-quark, diboson, and Drell-Yan production are also
%estimated. For the category of events satisfying $\nb \geq 2$, in
%which the contribution from $\znunu + \text{jets}$ events is
%suppressed to a negligible level, the \mj sample is also used to
%estimate this small contribution rather than using the statistically
%limited \mmj and \gj samples. Hence, only the \mj sample is used to
%estimate the total SM background for events satisfying $\nb \geq 2$,
%whereas all three data control samples are used for events satisfying
%$\nb \leq 1$.
