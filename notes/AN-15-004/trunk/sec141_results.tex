%%____________________________________________________________________________||
\section{Results}
\label{sec:results}

In this section the main results concerning the fit are summarised. 

Tables ~\ref{tab:predallqcd_sig_comb_mono}-~\ref{tab:predallqcd_sig_comb_asym} summarise 
the predicted and observed yields in the signal region 
for ``monojet'',``symmetric'' and ``asymmetric'' topologies respectively, 
corresponding to an integrated luminosity of 2.3 \ifb.
The predicted yields are based on a fit to multiple data control samples to obtain predictions in (\nj,\nb,\scalht). 
The observed counts in the signal region are not considered in this fit. 
The uncertainties reflect all statistical and (pre-fit) systematic sources added in quadrature. 
The $\ttbar+W$ and \znunu components are also shown (the former of which contains all residual contributions from sub-dominant processes such as e.g. diboson production). 
This table summarises our best knowledge of the SM background rates in the signal region while not considering counts in the signal region itself. 

In Fig.~\ref{fig:summaryPlot_Monojet}-~\ref{fig:summaryPlot_Symmetric} the pre-fit/post-fit background yield predictions 
are shown compared to the observations in data, with the breakdown for background processes, for each (\njet,\nb,\scalht) analysis bin. 
Expected yields for some benchmark signal models are also shown in the histogram. 

Tables ~\ref{tab:predallqcdpost_sig_comb_mono}-~\ref{tab:predallqcdpost_sig_comb_asym} summarises the post-fit (background-only fit) predicted backgrounds and the 
observed yields in the signal region corresponding to an integrated luminosity of 2.3 \ifb. 

Examples of \mht templates for the background and the some benchmark signals are shown in Fig.~\ref{fig:mht-templates}, 
for some (\njet,\nb,\scalht) analysis bins. 

In Figures ~\ref{fig:nuisPull_AlphaT}-~\ref{fig:nuisPull_Correlated} the pull 
of the nuisance parameters with respect to the pre-fit value is shown, 
grouped into different type of nuisances and, where relevant, 
split between ``asymmetric'' and ``symmetric'' topologies 
(``monojet'' topologies are included in the asymmetric). \\
No significant post-fit pull is observed across all set of parameters. 

Correction are derived per \scalht, \njet, \nb bin from the fit to the control regions (as 
is used to make the pre-fit predictions). In Figures~\ref{fig:shapesht}-~\ref{fig:shapesbjet} the results of applying 
these corrections to the signal region predictions of several kinematic variables 
for symmetric, asymmetric and monojet categories are shown. 
This is compared to the agreement when the prediction for the signal region
is taken directly from MC (uncorrected). As can be seen the agreement for a range of key variables,
including \scalht, \mht, jet \pt, \njet and \nb, is substantially improved after applying 
the corrections from the control region. This provides additional evidence for the 'scale anchoring'
approach as measuring the scale in the control regions allows considerable improvement
in the data/prediction agreement for variables associated with the kinematics of the event.
In general variables which are inclusively correlated strongly with scale such as \mht are more
effected than those that are less scale dependant such as \alphat and \bdphi.

\clearpage
\begin{figure}[tbhp]
    \caption{ \mht templates for some (\njet,\nb,\scalht) analysis bins. Benchmark models are also shown in the stack. \label{fig:mht-templates} }
  \begin{center}
    \subfigure[$\njet \geq 5$, $\nb \geq 3$, $\scalht > 800$ GeV]{ \includegraphics[width=0.45\textwidth]{figures/postFitResults/shapePlots/postFitShape_ge3b_ge5j_800_Inf_prefit_T1bbbb_1400_100} } ~~
    \subfigure[$\njet \geq 5$, $\nb \geq 3$, $600 < \scalht < 800$ GeV]{ \includegraphics[width=0.45\textwidth]{figures/postFitResults/shapePlots/postFitShape_eq0b_ge5j_600_800_prefit_T1qqqq_900_700} } \\
    \subfigure[$\njet \geq 5$, $\nb = 1$, $\scalht > 800$ GeV]   { \includegraphics[width=0.45\textwidth]{figures/postFitResults/shapePlots/postFitShape_eq1b_ge5j_800_Inf_prefit_T1qqqq_1200_100} } ~~
    \subfigure[$\njet \geq 5$, $\nb = 2$, $\scalht > 800$ GeV]{ \includegraphics[width=0.45\textwidth]{figures/postFitResults/shapePlots/postFitShape_eq2b_ge5j_800_Inf_prefit_T1tttt_800_400} } \\
  \end{center}
\end{figure}


\clearpage
\begin{table}[h!]
\tiny
\centering
\caption{Pre fit Predictions and Data in the signal region for 2.24\ifb for monojet categories. All entries are non-zero but are truncated to one decimal place.\label{tab:predallqcd_sig_comb_mono}}
\scalebox{0.85}{\begin{tabular}{cccccccccc}
	\hline\hline
	&	& \multicolumn{8}{c}{\scalht (\gev)}\\ 
	&	 (\njet, \nb) & 200-250 & 250-300 & 300-350 & 350-400 & 400-500 & 500-600 & 600-800 & 800-$\infty$ \\ [0.8ex] 
\hline
	Data & (1, 0) & 13094 & 4130 & 1477 & 663 & 461 & 118 & 50 & -- \\[0.5ex] 
	SM & (1, 0) & $12319.3\pm 914.4$ & $4167.9\pm 386.4$ & $1474.1\pm 139.7$ & $559.8\pm 89.5$ & $468.2\pm 75.1$ & $145.6\pm 29.0$ & $60.4\pm 26.6$ & -- \\[0.5ex] 
	Ttw & (1, 0) & $5275.2\pm 508.1$ & $1615.9\pm 227.2$ & $504.5\pm 58.9$ & $169.0\pm 29.3$ & $135.6\pm 19.3$ & $33.8\pm 8.9$ & $13.9\pm 7.1$ & -- \\[0.5ex] 
	Zinv & (1, 0) & $7044.1\pm 495.7$ & $2552.0\pm 219.1$ & $969.6\pm 96.4$ & $390.8\pm 64.3$ & $325.0\pm 63.5$ & $111.7\pm 24.3$ & $46.5\pm 21.8$ & -- \\[0.5ex] 
	QCD & (1, 0) & $0.0\pm 5.1$ & $0.0\pm 1.5$ & $0.0\pm 2.6$ & $0.0\pm 2.8$ & $2.6\pm 6.5$ & $0.0\pm 137.0$ & $0.0\pm 193.4$ & -- \\[0.5ex] 
	Data & (1, 1) & 475 & 151 & 57 & 25 & 24 & 6 & -- & -- \\[0.5ex] 
	SM & (1, 1) & $505.3\pm 69.4$ & $169.6\pm 26.2$ & $61.3\pm 10.2$ & $21.3\pm 4.3$ & $21.2\pm 4.6$ & $4.7\pm 1.1$ & -- & -- \\[0.5ex] 
	Ttw & (1, 1) & $172.2\pm 27.2$ & $51.2\pm 9.9$ & $17.3\pm 3.4$ & $5.3\pm 1.1$ & $7.1\pm 1.4$ & $1.0\pm 0.3$ & -- & -- \\[0.5ex] 
	Zinv & (1, 1) & $333.2\pm 44.2$ & $118.4\pm 18.0$ & $44.0\pm 7.1$ & $16.0\pm 3.3$ & $13.8\pm 3.4$ & $3.7\pm 0.9$ & -- & -- \\[0.5ex] 
	QCD & (1, 1) & $0.0\pm 0.2$ & $0.0\pm 0.1$ & $0.0\pm 0.1$ & $0.0\pm 0.1$ & $0.1\pm 0.3$ & $0.0\pm 4.5$ & -- & -- \\[0.5ex] 
	\hline
	\hline
\end{tabular}}
\end{table}

\clearpage
\begin{table}[h!]
\tiny
\centering
\caption{Pre fit Predictions and Data in the signal region for 2.24\ifb for symmetric categories. All entries are non-zero but are truncated to one decimal place.\label{tab:predallqcd_sig_comb_sym}}
\scalebox{0.85}{\begin{tabular}{cccccccccc}
	\hline\hline
	&	& \multicolumn{8}{c}{\scalht (\gev)}\\ 
	&	 (\njet, \nb) & 200-250 & 250-300 & 300-350 & 350-400 & 400-500 & 500-600 & 600-800 & 800-$\infty$ \\ [0.8ex] 
\hline
	Data & (2, 0) & 1167 & 1155 & 760 & 442 & 335 & 119 & 58 & 57 \\[0.5ex] 
	SM & (2, 0) & $1213.7\pm 272.5$ & $1210.9\pm 210.0$ & $802.2\pm 111.2$ & $463.2\pm 79.7$ & $381.2\pm 65.2$ & $120.6\pm 23.5$ & $48.5\pm 7.4$ & $57.7\pm 12.1$ \\[0.5ex] 
	Ttw & (2, 0) & $515.6\pm 116.2$ & $519.1\pm 89.6$ & $325.2\pm 45.9$ & $158.9\pm 32.7$ & $128.2\pm 26.1$ & $39.7\pm 10.5$ & $14.8\pm 3.4$ & $18.2\pm 4.0$ \\[0.5ex] 
	Zinv & (2, 0) & $534.6\pm 134.6$ & $610.2\pm 135.8$ & $409.5\pm 58.8$ & $237.6\pm 26.4$ & $214.7\pm 32.3$ & $76.5\pm 15.6$ & $33.7\pm 5.5$ & $39.5\pm 9.0$ \\[0.5ex] 
	QCD & (2, 0) & $60.7\pm 68.6$ & $29.0\pm 30.4$ & $24.3\pm 26.1$ & $25.0\pm 26.5$ & $14.3\pm 14.9$ & $1.5\pm 1.9$ & $0.0\pm 0.0$ & $0.0\pm 0.0$ \\[0.5ex] 
	Data & (2, 1) & 137 & 115 & 76 & 40 & 39 & 5 & 4 & 2 \\[0.5ex] 
	SM & (2, 1) & $124.8\pm 28.5$ & $95.8\pm 17.8$ & $57.1\pm 9.8$ & $35.4\pm 7.1$ & $33.7\pm 6.9$ & $12.0\pm 2.6$ & $4.9\pm 1.0$ & $4.4\pm 1.1$ \\[0.5ex] 
	Ttw & (2, 1) & $68.9\pm 16.0$ & $47.7\pm 9.6$ & $24.6\pm 4.7$ & $11.5\pm 2.7$ & $10.5\pm 2.5$ & $4.0\pm 1.1$ & $1.3\pm 0.3$ & $1.3\pm 0.3$ \\[0.5ex] 
	Zinv & (2, 1) & $38.3\pm 9.7$ & $41.1\pm 9.6$ & $27.1\pm 4.6$ & $18.5\pm 2.7$ & $19.5\pm 3.6$ & $7.7\pm 1.8$ & $3.7\pm 0.7$ & $3.1\pm 0.8$ \\[0.5ex] 
	QCD & (2, 1) & $6.2\pm 7.1$ & $2.3\pm 2.4$ & $1.7\pm 1.9$ & $1.9\pm 2.0$ & $1.3\pm 1.3$ & $0.1\pm 0.2$ & $0.0\pm 0.0$ & $0.0\pm 0.0$ \\[0.5ex] 
	Data & (2, 2) & 8 & 6 & 3 & 5 & 3 & 0 & 0 & -- \\[0.5ex] 
	SM & (2, 2) & $6.2\pm 1.5$ & $3.8\pm 0.8$ & $7.4\pm 1.4$ & $1.4\pm 0.4$ & $1.5\pm 0.4$ & $1.6\pm 0.5$ & $0.3\pm 0.1$ & -- \\[0.5ex] 
	Ttw & (2, 2) & $2.7\pm 0.7$ & $1.5\pm 0.3$ & $3.9\pm 0.9$ & $0.4\pm 0.1$ & $0.5\pm 0.1$ & $1.0\pm 0.4$ & $0.1\pm 0.0$ & -- \\[0.5ex] 
	Zinv & (2, 2) & $2.7\pm 0.7$ & $2.0\pm 0.5$ & $3.0\pm 0.6$ & $0.6\pm 0.1$ & $0.8\pm 0.2$ & $0.5\pm 0.2$ & $0.2\pm 0.1$ & -- \\[0.5ex] 
	QCD & (2, 2) & $0.3\pm 0.4$ & $0.1\pm 0.1$ & $0.2\pm 0.2$ & $0.1\pm 0.1$ & $0.1\pm 0.1$ & $0.0\pm 0.0$ & $0.0\pm 0.0$ & -- \\[0.5ex] 
	Data & (3, 0) & 4 & 205 & 592 & 577 & 624 & 215 & 97 & 79 \\[0.5ex] 
	SM & (3, 0) & $0.9\pm 0.4$ & $236.4\pm 41.7$ & $708.7\pm 127.1$ & $626.7\pm 143.0$ & $661.8\pm 103.1$ & $227.1\pm 42.6$ & $102.3\pm 15.9$ & $76.2\pm 16.3$ \\[0.5ex] 
	Ttw & (3, 0) & $0.6\pm 0.2$ & $105.2\pm 19.1$ & $305.5\pm 47.9$ & $238.4\pm 49.0$ & $257.7\pm 50.3$ & $80.9\pm 21.3$ & $33.8\pm 7.6$ & $23.8\pm 5.6$ \\[0.5ex] 
	Zinv & (3, 0) & $0.4\pm 0.2$ & $114.6\pm 26.4$ & $299.5\pm 47.6$ & $254.6\pm 30.4$ & $332.6\pm 50.5$ & $126.3\pm 25.6$ & $68.5\pm 11.4$ & $52.4\pm 12.0$ \\[0.5ex] 
	QCD & (3, 0) & $0.0\pm 0.0$ & $6.0\pm 7.0$ & $40.8\pm 43.1$ & $53.6\pm 55.3$ & $26.3\pm 27.0$ & $7.2\pm 8.1$ & $0.0\pm 0.2$ & $0.0\pm 0.0$ \\[0.5ex] 
	Data & (3, 1) & -- & 46 & 114 & 114 & 93 & 32 & 18 & 10 \\[0.5ex] 
	SM & (3, 1) & -- & $49.6\pm 9.3$ & $120.2\pm 24.5$ & $142.6\pm 33.6$ & $132.2\pm 24.6$ & $35.8\pm 7.4$ & $20.7\pm 3.9$ & $11.4\pm 2.8$ \\[0.5ex] 
	Ttw & (3, 1) & -- & $34.9\pm 7.3$ & $72.8\pm 14.9$ & $75.1\pm 17.8$ & $69.4\pm 15.6$ & $16.6\pm 4.6$ & $8.0\pm 2.0$ & $3.7\pm 1.0$ \\[0.5ex] 
	Zinv & (3, 1) & -- & $11.2\pm 2.7$ & $29.2\pm 5.1$ & $38.3\pm 5.5$ & $49.7\pm 8.8$ & $16.1\pm 3.7$ & $12.7\pm 2.5$ & $7.6\pm 2.0$ \\[0.5ex] 
	QCD & (3, 1) & -- & $1.3\pm 1.5$ & $6.9\pm 7.3$ & $12.2\pm 12.6$ & $5.2\pm 5.4$ & $1.1\pm 1.3$ & $0.0\pm 0.0$ & $0.0\pm 0.0$ \\[0.5ex] 
	Data & (3, 2) & -- & 11 & 12 & 14 & 16 & 5 & 1 & 1 \\[0.5ex] 
	SM & (3, 2) & -- & $7.6\pm 1.5$ & $25.0\pm 5.5$ & $27.5\pm 6.8$ & $17.3\pm 4.3$ & $5.4\pm 1.3$ & $1.2\pm 0.3$ & $1.3\pm 0.4$ \\[0.5ex] 
	Ttw & (3, 2) & -- & $5.1\pm 1.2$ & $17.5\pm 4.2$ & $18.4\pm 4.8$ & $11.1\pm 3.4$ & $2.9\pm 1.0$ & $0.3\pm 0.1$ & $0.5\pm 0.2$ \\[0.5ex] 
	Zinv & (3, 2) & -- & $1.8\pm 0.4$ & $4.2\pm 0.8$ & $4.1\pm 0.7$ & $4.3\pm 0.8$ & $2.0\pm 0.5$ & $0.9\pm 0.2$ & $0.8\pm 0.3$ \\[0.5ex] 
	QCD & (3, 2) & -- & $0.2\pm 0.2$ & $1.4\pm 1.5$ & $2.4\pm 2.4$ & $0.7\pm 0.7$ & $0.2\pm 0.2$ & $0.0\pm 0.0$ & $0.0\pm 0.0$ \\[0.5ex] 
	Data & (3, $\ge3$) & -- & 0 & -- & -- & 1 & -- & -- & -- \\[0.5ex] 
	SM & (3, $\ge3$) & -- & $0.2\pm 0.1$ & -- & -- & $0.5\pm 0.2$ & -- & -- & -- \\[0.5ex] 
	Ttw & (3, $\ge3$) & -- & $0.2\pm 0.1$ & -- & -- & $0.3\pm 0.1$ & -- & -- & -- \\[0.5ex] 
	Zinv & (3, $\ge3$) & -- & $0.0\pm 0.0$ & -- & -- & $0.2\pm 0.1$ & -- & -- & -- \\[0.5ex] 
	QCD & (3, $\ge3$) & -- & $0.0\pm 0.0$ & -- & -- & $0.0\pm 0.0$ & -- & -- & -- \\[0.5ex] 
	Data & (4, 0) & -- & -- & 77 & 181 & 369 & 175 & 120 & 68 \\[0.5ex] 
	SM & (4, 0) & -- & -- & $63.8\pm 10.6$ & $201.3\pm 32.3$ & $377.7\pm 54.9$ & $170.3\pm 35.6$ & $117.8\pm 18.7$ & $68.6\pm 14.5$ \\[0.5ex] 
	Ttw & (4, 0) & -- & -- & $33.2\pm 6.2$ & $102.9\pm 23.5$ & $190.6\pm 39.9$ & $70.8\pm 18.8$ & $44.0\pm 9.9$ & $23.9\pm 5.5$ \\[0.5ex] 
	Zinv & (4, 0) & -- & -- & $25.2\pm 3.7$ & $85.1\pm 11.7$ & $182.6\pm 28.9$ & $99.0\pm 20.8$ & $73.8\pm 12.6$ & $44.6\pm 10.2$ \\[0.5ex] 
	QCD & (4, 0) & -- & -- & $1.8\pm 2.2$ & $4.8\pm 5.5$ & $1.6\pm 1.8$ & $0.2\pm 0.2$ & $0.0\pm 0.3$ & $0.0\pm 0.0$ \\[0.5ex] 
	Data & (4, 1) & -- & -- & 19 & 93 & 134 & 39 & 18 & 10 \\[0.5ex] 
	SM & (4, 1) & -- & -- & $32.9\pm 6.3$ & $90.3\pm 19.4$ & $115.5\pm 23.3$ & $49.6\pm 11.9$ & $25.9\pm 5.0$ & $13.9\pm 3.3$ \\[0.5ex] 
	Ttw & (4, 1) & -- & -- & $25.8\pm 5.4$ & $67.8\pm 17.3$ & $84.6\pm 21.2$ & $30.8\pm 9.0$ & $13.3\pm 3.5$ & $5.5\pm 1.4$ \\[0.5ex] 
	Zinv & (4, 1) & -- & -- & $4.9\pm 0.8$ & $16.3\pm 2.5$ & $29.4\pm 5.3$ & $18.8\pm 4.3$ & $12.6\pm 2.5$ & $8.4\pm 2.2$ \\[0.5ex] 
	QCD & (4, 1) & -- & -- & $0.9\pm 1.1$ & $2.1\pm 2.5$ & $0.5\pm 0.6$ & $0.0\pm 0.1$ & $0.0\pm 0.1$ & $0.0\pm 0.0$ \\[0.5ex] 
	Data & (4, 2) & -- & -- & 8 & 30 & 39 & 12 & 7 & 2 \\[0.5ex] 
	SM & (4, 2) & -- & -- & $7.8\pm 1.7$ & $23.1\pm 5.9$ & $42.6\pm 10.9$ & $10.8\pm 3.1$ & $3.6\pm 0.9$ & $3.3\pm 1.0$ \\[0.5ex] 
	Ttw & (4, 2) & -- & -- & $6.2\pm 1.5$ & $19.9\pm 5.5$ & $36.4\pm 10.5$ & $8.6\pm 2.9$ & $2.2\pm 0.7$ & $1.6\pm 0.6$ \\[0.5ex] 
	Zinv & (4, 2) & -- & -- & $1.0\pm 0.2$ & $1.5\pm 0.3$ & $5.8\pm 1.1$ & $2.1\pm 0.5$ & $1.4\pm 0.3$ & $1.7\pm 0.5$ \\[0.5ex] 
	QCD & (4, 2) & -- & -- & $0.2\pm 0.3$ & $0.5\pm 0.6$ & $0.2\pm 0.2$ & $0.0\pm 0.0$ & $0.0\pm 0.0$ & $0.0\pm 0.0$ \\[0.5ex] 
	Data & (4, $\ge3$) & -- & -- & 0 & 3 & 0 & 2 & 0 & 0 \\[0.5ex] 
	SM & (4, $\ge3$) & -- & -- & $0.3\pm 0.1$ & $2.1\pm 0.6$ & $2.8\pm 0.9$ & $1.0\pm 0.3$ & $0.1\pm 0.0$ & $0.1\pm 0.0$ \\[0.5ex] 
	Ttw & (4, $\ge3$) & -- & -- & $0.3\pm 0.1$ & $1.6\pm 0.5$ & $2.5\pm 0.9$ & $0.7\pm 0.3$ & $0.0\pm 0.0$ & $0.1\pm 0.0$ \\[0.5ex] 
	Zinv & (4, $\ge3$) & -- & -- & $0.0\pm 0.0$ & $0.3\pm 0.1$ & $0.2\pm 0.1$ & $0.2\pm 0.1$ & $0.0\pm 0.0$ & $0.0\pm 0.0$ \\[0.5ex] 
	QCD & (4, $\ge3$) & -- & -- & $0.0\pm 0.0$ & $0.0\pm 0.1$ & $0.0\pm 0.0$ & $0.0\pm 0.0$ & $0.0\pm 0.0$ & $0.0\pm 0.0$ \\[0.5ex] 
	Data & ($\ge5$, 0) & -- & -- & -- & 8 & 109 & 100 & 94 & 64 \\[0.5ex] 
	SM & ($\ge5$, 0) & -- & -- & -- & $18.8\pm 4.4$ & $126.4\pm 25.7$ & $120.2\pm 30.4$ & $91.9\pm 15.6$ & $61.6\pm 13.9$ \\[0.5ex] 
	Ttw & ($\ge5$, 0) & -- & -- & -- & $12.1\pm 3.4$ & $68.5\pm 14.1$ & $49.2\pm 13.7$ & $42.2\pm 10.0$ & $24.5\pm 6.3$ \\[0.5ex] 
	Zinv & ($\ge5$, 0) & -- & -- & -- & $6.5\pm 1.5$ & $41.8\pm 7.8$ & $46.2\pm 9.8$ & $48.2\pm 8.6$ & $37.1\pm 8.7$ \\[0.5ex] 
	QCD & ($\ge5$, 0) & -- & -- & -- & $0.1\pm 2.8$ & $6.2\pm 7.1$ & $10.2\pm 11.0$ & $0.5\pm 0.6$ & $0.0\pm 0.0$ \\[0.5ex] 
	Data & ($\ge5$, 1) & -- & -- & -- & 6 & 62 & 48 & 35 & 21 \\[0.5ex] 
	SM & ($\ge5$, 1) & -- & -- & -- & $3.6\pm 1.0$ & $77.5\pm 17.8$ & $62.2\pm 17.5$ & $38.4\pm 8.9$ & $23.7\pm 6.2$ \\[0.5ex] 
	Ttw & ($\ge5$, 1) & -- & -- & -- & $3.1\pm 0.9$ & $58.9\pm 13.6$ & $40.3\pm 12.2$ & $27.0\pm 8.1$ & $14.3\pm 4.6$ \\[0.5ex] 
	Zinv & ($\ge5$, 1) & -- & -- & -- & $0.4\pm 0.1$ & $9.0\pm 1.9$ & $9.4\pm 2.1$ & $10.7\pm 2.1$ & $9.4\pm 2.4$ \\[0.5ex] 
	QCD & ($\ge5$, 1) & -- & -- & -- & $0.0\pm 0.5$ & $3.8\pm 4.4$ & $5.3\pm 5.7$ & $0.2\pm 0.2$ & $0.0\pm 0.0$ \\[0.5ex] 
	Data & ($\ge5$, 2) & -- & -- & -- & 0 & 27 & 18 & 10 & 16 \\[0.5ex] 
	SM & ($\ge5$, 2) & -- & -- & -- & $2.7\pm 0.8$ & $27.1\pm 7.0$ & $24.9\pm 7.6$ & $11.0\pm 3.2$ & $7.0\pm 2.2$ \\[0.5ex] 
	Ttw & ($\ge5$, 2) & -- & -- & -- & $2.4\pm 0.8$ & $22.2\pm 5.6$ & $17.8\pm 5.9$ & $8.9\pm 3.0$ & $5.3\pm 1.9$ \\[0.5ex] 
	Zinv & ($\ge5$, 2) & -- & -- & -- & $0.2\pm 0.1$ & $1.3\pm 0.3$ & $2.1\pm 0.5$ & $1.9\pm 0.4$ & $1.7\pm 0.5$ \\[0.5ex] 
	QCD & ($\ge5$, 2) & -- & -- & -- & $0.0\pm 0.4$ & $1.3\pm 1.5$ & $2.1\pm 2.3$ & $0.1\pm 0.1$ & $0.0\pm 0.0$ \\[0.5ex] 
	Data & ($\ge5$, $\ge3$) & -- & -- & -- & -- & 1 & 1 & 1 & 3 \\[0.5ex] 
	SM & ($\ge5$, $\ge3$) & -- & -- & -- & -- & $1.6\pm 0.4$ & $3.3\pm 1.1$ & $1.5\pm 0.5$ & $0.9\pm 0.4$ \\[0.5ex] 
	Ttw & ($\ge5$, $\ge3$) & -- & -- & -- & -- & $1.3\pm 0.4$ & $2.6\pm 0.9$ & $1.1\pm 0.4$ & $0.6\pm 0.3$ \\[0.5ex] 
	Zinv & ($\ge5$, $\ge3$) & -- & -- & -- & -- & $0.1\pm 0.0$ & $0.2\pm 0.1$ & $0.3\pm 0.1$ & $0.2\pm 0.1$ \\[0.5ex] 
	QCD & ($\ge5$, $\ge3$) & -- & -- & -- & -- & $0.1\pm 0.1$ & $0.3\pm 0.3$ & $0.0\pm 0.0$ & $0.0\pm 0.0$ \\[0.5ex] 
	\hline
	\hline
\end{tabular}}
\end{table}

\clearpage
\begin{table}[h!]
\tiny
\centering
\caption{Pre fit Predictions and Data in the signal region for 2.2\ifb for asymmetric categories. The letter ``a'' in jet \eg ``2a''  indicates the asymmetric jet bins. All entries are non-zero but are truncated to one decimal place.\label{tab:predallqcd_sig_comb_asym}}
\scalebox{0.85}{\begin{tabular}{cccccccccc}
	\hline\hline
	&	& \multicolumn{8}{c}{\scalht (\gev)}\\ 
	&	 (\njet, \nb) & 200-250 & 250-300 & 300-350 & 350-400 & 400-500 & 500-600 & 600-800 & 800-$\infty$ \\ [0.8ex] 
\hline
	Data & (2a, 0) & 5788 & 1585 & 584 & 232 & 139 & 26 & 16 & -- \\[0.5ex] 
	SM & (2a, 0) & $5681.6\pm 512.7$ & $1662.2\pm 195.7$ & $601.4\pm 63.5$ & $226.2\pm 39.7$ & $144.3\pm 26.5$ & $36.3\pm 7.3$ & $19.0\pm 8.2$ & -- \\[0.5ex] 
	Ttw & (2a, 0) & $2705.4\pm 292.2$ & $725.3\pm 115.2$ & $250.8\pm 30.8$ & $83.2\pm 16.6$ & $44.7\pm 7.9$ & $13.0\pm 3.5$ & $4.2\pm 1.8$ & -- \\[0.5ex] 
	Zinv & (2a, 0) & $2873.7\pm 251.3$ & $920.2\pm 100.9$ & $350.6\pm 39.9$ & $143.0\pm 24.8$ & $99.6\pm 21.6$ & $23.3\pm 5.4$ & $13.8\pm 6.9$ & -- \\[0.5ex] 
	QCD & (2a, 0) & $102.5\pm 98.8$ & $16.7\pm 19.0$ & $0.0\pm 0.0$ & $0.0\pm 0.0$ & $0.0\pm 0.0$ & $0.0\pm 0.0$ & $1.1\pm 1.1$ & -- \\[0.5ex] 
	Data & (2a, 1) & 536 & 152 & 51 & 18 & 7 & 4 & -- & -- \\[0.5ex] 
	SM & (2a, 1) & $524.5\pm 57.4$ & $158.5\pm 23.5$ & $49.3\pm 6.7$ & $20.5\pm 3.7$ & $12.6\pm 2.5$ & $4.3\pm 1.0$ & -- & -- \\[0.5ex] 
	Ttw & (2a, 1) & $306.7\pm 39.4$ & $81.9\pm 15.9$ & $21.8\pm 3.6$ & $6.3\pm 1.3$ & $4.8\pm 1.0$ & $1.2\pm 0.4$ & -- & -- \\[0.5ex] 
	Zinv & (2a, 1) & $208.3\pm 21.9$ & $75.1\pm 9.9$ & $27.5\pm 3.8$ & $14.2\pm 2.7$ & $7.8\pm 1.8$ & $3.0\pm 0.8$ & -- & -- \\[0.5ex] 
	QCD & (2a, 1) & $9.5\pm 9.1$ & $1.6\pm 1.8$ & $0.0\pm 0.0$ & $0.0\pm 0.0$ & $0.0\pm 0.0$ & $0.0\pm 0.0$ & -- & -- \\[0.5ex] 
	Data & (2a, 2) & 31 & 10 & 3 & 1 & 0 & -- & -- & -- \\[0.5ex] 
	SM & (2a, 2) & $28.6\pm 3.6$ & $7.0\pm 1.2$ & $6.5\pm 1.0$ & $2.0\pm 0.4$ & $0.8\pm 0.2$ & -- & -- & -- \\[0.5ex] 
	Ttw & (2a, 2) & $14.0\pm 2.0$ & $3.3\pm 0.7$ & $2.9\pm 0.5$ & $1.0\pm 0.3$ & $0.3\pm 0.1$ & -- & -- & -- \\[0.5ex] 
	Zinv & (2a, 2) & $14.1\pm 1.8$ & $3.6\pm 0.5$ & $3.6\pm 0.6$ & $0.9\pm 0.2$ & $0.5\pm 0.1$ & -- & -- & -- \\[0.5ex] 
	QCD & (2a, 2) & $0.5\pm 0.5$ & $0.1\pm 0.1$ & $0.0\pm 0.0$ & $0.0\pm 0.0$ & $0.0\pm 0.0$ & -- & -- & -- \\[0.5ex] 
	Data & (3a, 0) & 1599 & 1609 & 777 & 239 & 95 & 15 & 9 & -- \\[0.5ex] 
	SM & (3a, 0) & $1605.7\pm 160.4$ & $1477.9\pm 193.8$ & $757.0\pm 80.3$ & $250.6\pm 43.6$ & $111.7\pm 19.5$ & $19.7\pm 4.0$ & $9.3\pm 4.2$ & -- \\[0.5ex] 
	Ttw & (3a, 0) & $847.9\pm 97.3$ & $764.3\pm 128.5$ & $361.3\pm 39.2$ & $104.5\pm 20.9$ & $42.2\pm 6.5$ & $6.0\pm 1.7$ & $2.4\pm 1.1$ & -- \\[0.5ex] 
	Zinv & (3a, 0) & $726.3\pm 71.7$ & $689.2\pm 78.7$ & $363.7\pm 41.7$ & $135.8\pm 24.1$ & $69.5\pm 15.3$ & $13.8\pm 3.1$ & $6.8\pm 3.5$ & -- \\[0.5ex] 
	QCD & (3a, 0) & $31.6\pm 25.7$ & $24.4\pm 23.9$ & $31.9\pm 32.1$ & $10.3\pm 12.2$ & $0.0\pm 0.0$ & $0.0\pm 0.0$ & $0.0\pm 0.0$ & -- \\[0.5ex] 
	Data & (3a, 1) & 340 & 299 & 152 & 59 & 15 & 1 & 1 & -- \\[0.5ex] 
	SM & (3a, 1) & $327.0\pm 36.9$ & $346.4\pm 57.8$ & $143.9\pm 18.8$ & $42.0\pm 7.9$ & $14.7\pm 2.8$ & $2.3\pm 0.6$ & $1.2\pm 0.5$ & -- \\[0.5ex] 
	Ttw & (3a, 1) & $250.7\pm 30.9$ & $259.7\pm 50.2$ & $99.7\pm 15.1$ & $25.3\pm 5.6$ & $6.9\pm 1.5$ & $1.5\pm 0.5$ & $0.3\pm 0.2$ & -- \\[0.5ex] 
	Zinv & (3a, 1) & $69.9\pm 7.5$ & $81.0\pm 10.1$ & $38.1\pm 4.7$ & $15.0\pm 2.7$ & $7.8\pm 1.8$ & $0.8\pm 0.2$ & $0.8\pm 0.4$ & -- \\[0.5ex] 
	QCD & (3a, 1) & $6.4\pm 5.2$ & $5.7\pm 5.6$ & $6.1\pm 6.1$ & $1.7\pm 2.0$ & $0.0\pm 0.0$ & $0.0\pm 0.0$ & $0.0\pm 0.0$ & -- \\[0.5ex] 
	Data & (3a, 2) & 52 & 62 & 29 & 12 & 1 & 0 & -- & -- \\[0.5ex] 
	SM & (3a, 2) & $57.2\pm 7.3$ & $59.8\pm 11.2$ & $31.6\pm 5.1$ & $10.2\pm 2.5$ & $1.9\pm 0.5$ & $0.4\pm 0.1$ & -- & -- \\[0.5ex] 
	Ttw & (3a, 2) & $46.6\pm 6.5$ & $47.9\pm 10.0$ & $24.1\pm 4.4$ & $7.8\pm 2.1$ & $0.6\pm 0.2$ & $0.2\pm 0.1$ & -- & -- \\[0.5ex] 
	Zinv & (3a, 2) & $9.4\pm 1.0$ & $11.0\pm 1.5$ & $6.2\pm 0.9$ & $2.1\pm 0.4$ & $1.3\pm 0.3$ & $0.3\pm 0.1$ & -- & -- \\[0.5ex] 
	QCD & (3a, 2) & $1.1\pm 0.9$ & $1.0\pm 1.0$ & $1.3\pm 1.3$ & $0.4\pm 0.5$ & $0.0\pm 0.0$ & $0.0\pm 0.0$ & -- & -- \\[0.5ex] 
	Data & (3a, $\ge3$) & 3 & 1 & 1 & -- & -- & -- & -- & -- \\[0.5ex] 
	SM & (3a, $\ge3$) & $0.9\pm 0.2$ & $1.8\pm 0.4$ & $0.7\pm 0.2$ & -- & -- & -- & -- & -- \\[0.5ex] 
	Ttw & (3a, $\ge3$) & $0.7\pm 0.2$ & $1.4\pm 0.4$ & $0.7\pm 0.2$ & -- & -- & -- & -- & -- \\[0.5ex] 
	Zinv & (3a, $\ge3$) & $0.1\pm 0.0$ & $0.4\pm 0.1$ & $0.0\pm 0.0$ & -- & -- & -- & -- & -- \\[0.5ex] 
	QCD & (3a, $\ge3$) & $0.0\pm 0.0$ & $0.0\pm 0.0$ & $0.0\pm 0.0$ & -- & -- & -- & -- & -- \\[0.5ex] 
	Data & (4a, 0) & 3 & 178 & 412 & 246 & 119 & 15 & 2 & -- \\[0.5ex] 
	SM & (4a, 0) & $3.8\pm 0.5$ & $150.4\pm 20.7$ & $406.0\pm 48.5$ & $259.6\pm 47.5$ & $132.3\pm 21.3$ & $14.7\pm 3.2$ & $2.6\pm 1.3$ & -- \\[0.5ex] 
	Ttw & (4a, 0) & $1.8\pm 0.6$ & $88.7\pm 15.0$ & $212.1\pm 25.8$ & $140.4\pm 29.2$ & $59.3\pm 9.4$ & $5.6\pm 1.6$ & $0.6\pm 0.4$ & -- \\[0.5ex] 
	Zinv & (4a, 0) & $2.0\pm 0.4$ & $60.3\pm 7.5$ & $166.7\pm 20.8$ & $106.5\pm 18.9$ & $68.1\pm 14.9$ & $9.1\pm 2.2$ & $2.0\pm 1.0$ & -- \\[0.5ex] 
	QCD & (4a, 0) & $0.0\pm 0.0$ & $1.4\pm 1.6$ & $27.2\pm 30.4$ & $12.6\pm 10.6$ & $4.9\pm 3.8$ & $0.0\pm 0.0$ & $0.0\pm 0.0$ & -- \\[0.5ex] 
	Data & (4a, 1) & 1 & 53 & 180 & 96 & 51 & 4 & 0 & -- \\[0.5ex] 
	SM & (4a, 1) & $1.4\pm 0.3$ & $50.5\pm 8.3$ & $165.7\pm 22.6$ & $98.3\pm 19.4$ & $52.0\pm 9.4$ & $3.1\pm 0.8$ & $0.6\pm 0.3$ & -- \\[0.5ex] 
	Ttw & (4a, 1) & $1.0\pm 0.3$ & $39.0\pm 7.3$ & $130.4\pm 18.6$ & $75.5\pm 16.6$ & $35.5\pm 7.6$ & $1.7\pm 0.6$ & $0.1\pm 0.1$ & -- \\[0.5ex] 
	Zinv & (4a, 1) & $0.4\pm 0.1$ & $11.1\pm 1.5$ & $24.2\pm 3.2$ & $18.0\pm 3.2$ & $14.7\pm 3.3$ & $1.3\pm 0.3$ & $0.5\pm 0.3$ & -- \\[0.5ex] 
	QCD & (4a, 1) & $0.0\pm 0.0$ & $0.5\pm 0.5$ & $11.1\pm 12.4$ & $4.8\pm 4.0$ & $1.9\pm 1.5$ & $0.0\pm 0.0$ & $0.0\pm 0.0$ & -- \\[0.5ex] 
	Data & (4a, 2) & 0 & 11 & 44 & 30 & 8 & 0 & 0 & -- \\[0.5ex] 
	SM & (4a, 2) & $0.3\pm 0.1$ & $14.4\pm 2.6$ & $51.9\pm 8.4$ & $27.2\pm 6.1$ & $14.8\pm 3.2$ & $0.6\pm 0.2$ & $0.1\pm 0.1$ & -- \\[0.5ex] 
	Ttw & (4a, 2) & $0.3\pm 0.1$ & $11.7\pm 2.4$ & $44.2\pm 7.5$ & $23.3\pm 5.7$ & $12.3\pm 2.9$ & $0.4\pm 0.2$ & $0.0\pm 0.0$ & -- \\[0.5ex] 
	Zinv & (4a, 2) & $0.0\pm 0.0$ & $2.5\pm 0.4$ & $4.2\pm 0.6$ & $2.5\pm 0.5$ & $1.9\pm 0.5$ & $0.1\pm 0.0$ & $0.1\pm 0.0$ & -- \\[0.5ex] 
	QCD & (4a, 2) & $0.0\pm 0.0$ & $0.1\pm 0.2$ & $3.5\pm 3.9$ & $1.3\pm 1.1$ & $0.5\pm 0.4$ & $0.0\pm 0.0$ & $0.0\pm 0.0$ & -- \\[0.5ex] 
	Data & (4a, $\ge3$) & -- & 0 & 0 & 2 & 2 & -- & -- & -- \\[0.5ex] 
	SM & (4a, $\ge3$) & -- & $1.8\pm 0.4$ & $3.0\pm 0.6$ & $2.6\pm 0.8$ & $1.8\pm 0.5$ & -- & -- & -- \\[0.5ex] 
	Ttw & (4a, $\ge3$) & -- & $1.6\pm 0.4$ & $2.4\pm 0.6$ & $2.1\pm 0.7$ & $1.7\pm 0.5$ & -- & -- & -- \\[0.5ex] 
	Zinv & (4a, $\ge3$) & -- & $0.1\pm 0.0$ & $0.4\pm 0.1$ & $0.3\pm 0.1$ & $0.0\pm 0.0$ & -- & -- & -- \\[0.5ex] 
	QCD & (4a, $\ge3$) & -- & $0.0\pm 0.0$ & $0.2\pm 0.2$ & $0.1\pm 0.1$ & $0.1\pm 0.1$ & -- & -- & -- \\[0.5ex] 
	Data & ($\ge5$a, 0) & -- & 3 & 40 & 96 & 105 & 20 & 3 & -- \\[0.5ex] 
	SM & ($\ge5$a, 0) & -- & $3.9\pm 1.0$ & $49.0\pm 7.9$ & $113.6\pm 22.1$ & $127.0\pm 21.2$ & $21.4\pm 4.9$ & $4.5\pm 2.0$ & -- \\[0.5ex] 
	Ttw & ($\ge5$a, 0) & -- & $3.6\pm 1.4$ & $31.6\pm 4.8$ & $63.7\pm 13.2$ & $77.8\pm 14.1$ & $12.8\pm 3.7$ & $2.0\pm 0.9$ & -- \\[0.5ex] 
	Zinv & ($\ge5$a, 0) & -- & $0.3\pm 0.6$ & $17.2\pm 3.6$ & $37.6\pm 7.3$ & $42.7\pm 9.3$ & $8.3\pm 2.0$ & $2.5\pm 1.3$ & -- \\[0.5ex] 
	QCD & ($\ge5$a, 0) & -- & $0.0\pm 0.0$ & $0.1\pm 0.1$ & $12.3\pm 9.5$ & $6.5\pm 6.7$ & $0.3\pm 0.5$ & $0.0\pm 0.0$ & -- \\[0.5ex] 
	Data & ($\ge5$a, 1) & -- & 0 & 24 & 60 & 74 & 15 & 0 & -- \\[0.5ex] 
	SM & ($\ge5$a, 1) & -- & $1.2\pm 0.3$ & $21.9\pm 3.4$ & $51.6\pm 10.5$ & $72.2\pm 13.7$ & $17.3\pm 4.8$ & $1.9\pm 0.8$ & -- \\[0.5ex] 
	Ttw & ($\ge5$a, 1) & -- & $1.0\pm 0.4$ & $19.5\pm 3.0$ & $41.8\pm 8.7$ & $61.0\pm 12.4$ & $14.0\pm 4.4$ & $1.3\pm 0.6$ & -- \\[0.5ex] 
	Zinv & ($\ge5$a, 1) & -- & $0.2\pm 0.5$ & $2.4\pm 0.5$ & $4.2\pm 0.9$ & $7.5\pm 1.7$ & $3.0\pm 0.7$ & $0.5\pm 0.3$ & -- \\[0.5ex] 
	QCD & ($\ge5$a, 1) & -- & $0.0\pm 0.0$ & $0.1\pm 0.1$ & $5.6\pm 4.3$ & $3.7\pm 3.8$ & $0.3\pm 0.4$ & $0.0\pm 0.0$ & -- \\[0.5ex] 
	Data & ($\ge5$a, 2) & -- & 0 & 11 & 27 & 29 & 6 & 1 & -- \\[0.5ex] 
	SM & ($\ge5$a, 2) & -- & $0.0\pm 0.0$ & $6.7\pm 1.1$ & $25.6\pm 5.4$ & $29.1\pm 6.1$ & $6.1\pm 1.8$ & $0.5\pm 0.2$ & -- \\[0.5ex] 
	Ttw & ($\ge5$a, 2) & -- & $0.0\pm 0.0$ & $6.4\pm 1.0$ & $21.8\pm 4.7$ & $26.0\pm 5.8$ & $5.2\pm 1.7$ & $0.5\pm 0.2$ & -- \\[0.5ex] 
	Zinv & ($\ge5$a, 2) & -- & $0.0\pm 0.0$ & $0.3\pm 0.1$ & $1.1\pm 0.2$ & $1.5\pm 0.3$ & $0.8\pm 0.2$ & $0.0\pm 0.0$ & -- \\[0.5ex] 
	QCD & ($\ge5$a, 2) & -- & $0.0\pm 0.0$ & $0.0\pm 0.0$ & $2.8\pm 2.1$ & $1.5\pm 1.5$ & $0.1\pm 0.1$ & $0.0\pm 0.0$ & -- \\[0.5ex] 
	Data & ($\ge5$a, $\ge3$) & -- & -- & 0 & 2 & 5 & 1 & -- & -- \\[0.5ex] 
	SM & ($\ge5$a, $\ge3$) & -- & -- & $0.5\pm 0.1$ & $3.0\pm 0.7$ & $4.5\pm 1.2$ & $0.8\pm 0.3$ & -- & -- \\[0.5ex] 
	Ttw & ($\ge5$a, $\ge3$) & -- & -- & $0.5\pm 0.1$ & $2.7\pm 0.7$ & $4.0\pm 1.1$ & $0.7\pm 0.3$ & -- & -- \\[0.5ex] 
	Zinv & ($\ge5$a, $\ge3$) & -- & -- & $0.0\pm 0.0$ & $0.0\pm 0.0$ & $0.2\pm 0.1$ & $0.1\pm 0.0$ & -- & -- \\[0.5ex] 
	QCD & ($\ge5$a, $\ge3$) & -- & -- & $0.0\pm 0.0$ & $0.3\pm 0.3$ & $0.2\pm 0.2$ & $0.0\pm 0.0$ & -- & -- \\[0.5ex] 
	\hline
	\hline
\end{tabular}}
\end{table}


\clearpage
\begin{table}[h!]
\tiny
\centering
\caption{Post fit Predictions and Data in the signal region for 12.9\ifb for monojet categories. All entries are non-zero but are truncated to one decimal place.\label{tab:predallqcdpost_sig_comb_mono}}
\scalebox{0.85}{\begin{tabular}{cccccccccc}
	\hline\hline
	&	& \multicolumn{8}{c}{\scalht (\gev)}\\ 
	&	 (\njet, \nb) & 200-250 & 250-300 & 300-350 & 350-400 & 400-500 & 500-600 & 600-800 & 800-$\infty$ \\ [0.8ex] 
\hline
	Data & (1, 0) & 70743 & 23504 & 8886 & 3692 & 2596 & 663 & 339 & -- \\[0.5ex] 
	SM & (1, 0) & $70661.9\pm 243.2$ & $23453.4\pm 125.3$ & $8889.6\pm 84.8$ & $3730.3\pm 53.8$ & $2555.4\pm 45.2$ & $622.7\pm 16.3$ & $342.3\pm 18.5$ & -- \\[0.5ex] 
	Ttw & (1, 0) & $31637.0\pm 108.9$ & $9377.6\pm 50.7$ & $3237.4\pm 31.1$ & $1262.7\pm 18.2$ & $771.9\pm 14.3$ & $166.9\pm 4.3$ & $85.4\pm 4.7$ & -- \\[0.5ex] 
	Zinv & (1, 0) & $39023.0\pm 137.2$ & $14075.8\pm 75.2$ & $5650.1\pm 53.5$ & $2467.3\pm 35.7$ & $1783.5\pm 31.4$ & $452.5\pm 11.4$ & $256.0\pm 13.9$ & -- \\[0.5ex] 
	QCD & (1, 0) & $2.0\pm 1.8$ & $0.0\pm 0.0$ & $2.0\pm 2.2$ & $0.3\pm 0.3$ & $0.0\pm 0.0$ & $3.3\pm 3.9$ & $0.9\pm 1.2$ & -- \\[0.5ex] 
	Data & (1, 1) & 2704 & 927 & 378 & 154 & 111 & 78 & -- & -- \\[0.5ex] 
	SM & (1, 1) & $2743.1\pm 49.1$ & $953.7\pm 26.0$ & $381.8\pm 18.0$ & $157.6\pm 9.3$ & $116.9\pm 6.4$ & $65.7\pm 4.8$ & -- & -- \\[0.5ex] 
	Ttw & (1, 1) & $983.5\pm 18.3$ & $332.7\pm 9.2$ & $126.7\pm 6.0$ & $41.7\pm 2.5$ & $43.7\pm 2.4$ & $18.5\pm 1.4$ & -- & -- \\[0.5ex] 
	Zinv & (1, 1) & $1759.5\pm 31.4$ & $620.9\pm 17.0$ & $255.0\pm 12.1$ & $115.9\pm 6.8$ & $73.2\pm 4.0$ & $46.9\pm 3.5$ & -- & -- \\[0.5ex] 
	QCD & (1, 1) & $0.1\pm 0.1$ & $0.0\pm 0.0$ & $0.1\pm 0.1$ & $0.0\pm 0.0$ & $0.0\pm 0.0$ & $0.3\pm 0.3$ & -- & -- \\[0.5ex] 
	\hline
	\hline
\end{tabular}}
\end{table}

\clearpage
\begin{table}[h!]
\tiny
\centering
\caption{Post fit Predictions and Data in the signal region for 2.6\ifb for symmetric categories. All entries are non-zero but are truncated to one decimal place.\label{tab:predallqcdpost_sig_comb_sym}}
\scalebox{0.85}{\begin{tabular}{cccccccccc}
	\hline\hline
	&	& \multicolumn{8}{c}{\scalht (\gev)}\\ 
	&	 (\njet, \nb) & 200-250 & 250-300 & 300-350 & 350-400 & 400-500 & 500-600 & 600-800 & 800-$\infty$ \\ [0.8ex] 
\hline
	Data & (2, 0) & 1366 & 1350 & 911 & 521 & 452 & 131 & 88 & 76 \\[0.5ex] 
	SM & (2, 0) & $1369.6\pm 50.0$ & $1357.5\pm 45.3$ & $879.3\pm 27.4$ & $502.2\pm 25.1$ & $449.3\pm 15.0$ & $121.0\pm 6.4$ & $68.9\pm 3.6$ & $73.5\pm 6.0$ \\[0.5ex] 
	Ttw & (2, 0) & $628.9\pm 34.3$ & $604.0\pm 23.0$ & $372.6\pm 10.9$ & $194.0\pm 8.1$ & $163.2\pm 5.8$ & $39.7\pm 1.9$ & $20.1\pm 1.0$ & $22.7\pm 1.9$ \\[0.5ex] 
	Zinv & (2, 0) & $634.7\pm 35.2$ & $701.2\pm 26.9$ & $492.3\pm 14.9$ & $280.9\pm 11.8$ & $279.0\pm 9.7$ & $78.6\pm 3.8$ & $46.6\pm 2.4$ & $49.5\pm 4.1$ \\[0.5ex] 
	QCD & (2, 0) & $105.9\pm 80.1$ & $52.2\pm 48.1$ & $14.4\pm 17.0$ & $27.3\pm 27.4$ & $7.1\pm 6.5$ & $2.6\pm 3.5$ & $2.2\pm 2.2$ & $1.2\pm 1.0$ \\[0.5ex] 
	Data & (2, 1) & 116 & 110 & 74 & 25 & 26 & 6 & 3 & 4 \\[0.5ex] 
	SM & (2, 1) & $118.5\pm 10.8$ & $104.0\pm 8.8$ & $71.3\pm 5.9$ & $37.6\pm 4.3$ & $24.0\pm 2.4$ & $9.3\pm 1.2$ & $4.7\pm 0.8$ & $7.6\pm 1.2$ \\[0.5ex] 
	Ttw & (2, 1) & $67.5\pm 7.6$ & $53.3\pm 4.7$ & $26.9\pm 2.2$ & $13.1\pm 1.5$ & $6.5\pm 0.7$ & $2.1\pm 0.3$ & $0.9\pm 0.2$ & $1.3\pm 0.2$ \\[0.5ex] 
	Zinv & (2, 1) & $42.1\pm 4.7$ & $46.6\pm 4.1$ & $43.3\pm 3.6$ & $22.4\pm 2.5$ & $17.0\pm 1.7$ & $7.1\pm 0.9$ & $3.6\pm 0.6$ & $6.2\pm 1.0$ \\[0.5ex] 
	QCD & (2, 1) & $8.9\pm 6.8$ & $4.1\pm 3.8$ & $1.2\pm 1.4$ & $2.1\pm 2.1$ & $0.5\pm 0.5$ & $0.2\pm 0.3$ & $0.2\pm 0.2$ & $0.1\pm 0.1$ \\[0.5ex] 
	Data & (2, 2) & 5 & 6 & 8 & 0 & 1 & 0 & 0 & -- \\[0.5ex] 
	SM & (2, 2) & $5.5\pm 1.7$ & $7.6\pm 2.1$ & $5.0\pm 1.7$ & $1.9\pm 0.5$ & $0.8\pm 0.3$ & $0.5\pm 0.2$ & $0.1\pm 0.1$ & -- \\[0.5ex] 
	Ttw & (2, 2) & $2.0\pm 0.7$ & $2.8\pm 0.8$ & $2.1\pm 0.7$ & $0.5\pm 0.1$ & $0.0\pm 0.0$ & $0.1\pm 0.0$ & $0.0\pm 0.0$ & -- \\[0.5ex] 
	Zinv & (2, 2) & $3.0\pm 1.0$ & $4.4\pm 1.3$ & $2.9\pm 1.0$ & $1.3\pm 0.4$ & $0.7\pm 0.3$ & $0.4\pm 0.2$ & $0.1\pm 0.1$ & -- \\[0.5ex] 
	QCD & (2, 2) & $0.6\pm 0.4$ & $0.3\pm 0.3$ & $0.1\pm 0.1$ & $0.1\pm 0.1$ & $0.0\pm 0.0$ & $0.0\pm 0.0$ & $0.0\pm 0.0$ & -- \\[0.5ex] 
	Data & (3, 0) & 0 & 248 & 685 & 687 & 718 & 220 & 119 & 118 \\[0.5ex] 
	SM & (3, 0) & $1.4\pm 0.6$ & $246.3\pm 13.8$ & $668.8\pm 26.6$ & $691.6\pm 26.0$ & $721.3\pm 22.5$ & $227.5\pm 12.4$ & $132.6\pm 6.2$ & $112.0\pm 6.6$ \\[0.5ex] 
	Ttw & (3, 0) & $0.9\pm 0.4$ & $120.6\pm 7.2$ & $309.3\pm 16.7$ & $316.8\pm 12.6$ & $311.5\pm 9.9$ & $85.3\pm 4.8$ & $42.8\pm 2.1$ & $34.8\pm 1.8$ \\[0.5ex] 
	Zinv & (3, 0) & $0.5\pm 0.2$ & $122.2\pm 7.0$ & $319.7\pm 17.2$ & $342.9\pm 13.6$ & $392.3\pm 12.5$ & $131.7\pm 7.4$ & $89.0\pm 4.2$ & $72.7\pm 3.8$ \\[0.5ex] 
	QCD & (3, 0) & $0.0\pm 0.0$ & $3.5\pm 3.4$ & $39.7\pm 31.0$ & $31.8\pm 24.0$ & $17.6\pm 13.6$ & $10.5\pm 7.7$ & $0.9\pm 0.8$ & $4.4\pm 4.3$ \\[0.5ex] 
	Data & (3, 1) & 2 & 40 & 97 & 88 & 87 & 17 & 15 & 7 \\[0.5ex] 
	SM & (3, 1) & $0.5\pm 0.2$ & $45.1\pm 4.2$ & $114.4\pm 7.7$ & $102.8\pm 6.4$ & $100.3\pm 5.1$ & $25.8\pm 2.4$ & $14.6\pm 1.4$ & $12.0\pm 1.5$ \\[0.5ex] 
	Ttw & (3, 1) & $0.3\pm 0.1$ & $34.4\pm 3.2$ & $72.5\pm 4.6$ & $62.0\pm 3.8$ & $49.1\pm 2.7$ & $8.5\pm 0.8$ & $3.3\pm 0.3$ & $3.1\pm 0.4$ \\[0.5ex] 
	Zinv & (3, 1) & $0.2\pm 0.1$ & $10.0\pm 0.9$ & $35.6\pm 2.3$ & $36.1\pm 2.3$ & $48.9\pm 2.8$ & $16.2\pm 1.6$ & $11.2\pm 1.1$ & $8.4\pm 1.1$ \\[0.5ex] 
	QCD & (3, 1) & $0.0\pm 0.0$ & $0.7\pm 0.6$ & $6.3\pm 4.9$ & $4.7\pm 3.5$ & $2.3\pm 1.8$ & $1.2\pm 0.9$ & $0.1\pm 0.1$ & $0.5\pm 0.5$ \\[0.5ex] 
	Data & (3, 2) & -- & 5 & 14 & 15 & 18 & 1 & 1 & 2 \\[0.5ex] 
	SM & (3, 2) & -- & $4.6\pm 0.9$ & $15.4\pm 2.3$ & $16.1\pm 2.1$ & $14.8\pm 1.5$ & $2.7\pm 0.4$ & $1.4\pm 0.3$ & $0.5\pm 0.2$ \\[0.5ex] 
	Ttw & (3, 2) & -- & $3.1\pm 0.6$ & $10.5\pm 1.6$ & $11.0\pm 1.4$ & $10.5\pm 1.1$ & $0.9\pm 0.2$ & $0.3\pm 0.1$ & $0.1\pm 0.0$ \\[0.5ex] 
	Zinv & (3, 2) & -- & $1.5\pm 0.3$ & $3.8\pm 0.6$ & $4.4\pm 0.6$ & $4.0\pm 0.5$ & $1.7\pm 0.3$ & $1.1\pm 0.2$ & $0.4\pm 0.2$ \\[0.5ex] 
	QCD & (3, 2) & -- & $0.1\pm 0.1$ & $1.1\pm 0.8$ & $0.7\pm 0.5$ & $0.3\pm 0.3$ & $0.1\pm 0.1$ & $0.0\pm 0.0$ & $0.0\pm 0.0$ \\[0.5ex] 
	Data & (3, $\ge3$) & -- & 0 & -- & 0 & 0 & -- & -- & -- \\[0.5ex] 
	SM & (3, $\ge3$) & -- & $0.1\pm 0.1$ & -- & $0.3\pm 0.3$ & $0.1\pm 0.1$ & -- & -- & -- \\[0.5ex] 
	Ttw & (3, $\ge3$) & -- & $0.1\pm 0.1$ & -- & $0.2\pm 0.2$ & $0.1\pm 0.1$ & -- & -- & -- \\[0.5ex] 
	Zinv & (3, $\ge3$) & -- & $0.0\pm 0.0$ & -- & $0.1\pm 0.1$ & $0.0\pm 0.0$ & -- & -- & -- \\[0.5ex] 
	QCD & (3, $\ge3$) & -- & $0.0\pm 0.0$ & -- & $0.0\pm 0.0$ & $0.0\pm 0.0$ & -- & -- & -- \\[0.5ex] 
	Data & (4, 0) & -- & 3 & 74 & 272 & 511 & 208 & 135 & 82 \\[0.5ex] 
	SM & (4, 0) & -- & $1.5\pm 0.8$ & $84.1\pm 7.0$ & $254.8\pm 13.1$ & $495.1\pm 19.3$ & $197.9\pm 9.8$ & $126.9\pm 7.4$ & $86.2\pm 4.6$ \\[0.5ex] 
	Ttw & (4, 0) & -- & $1.1\pm 0.5$ & $47.3\pm 4.0$ & $134.7\pm 7.2$ & $224.6\pm 9.4$ & $88.2\pm 4.3$ & $45.6\pm 2.3$ & $29.8\pm 1.6$ \\[0.5ex] 
	Zinv & (4, 0) & -- & $0.5\pm 0.3$ & $36.7\pm 3.1$ & $115.9\pm 6.2$ & $232.1\pm 9.8$ & $106.7\pm 5.1$ & $76.1\pm 3.7$ & $55.1\pm 2.9$ \\[0.5ex] 
	QCD & (4, 0) & -- & $0.0\pm 0.0$ & $0.1\pm 0.1$ & $4.2\pm 4.4$ & $38.5\pm 21.4$ & $3.0\pm 2.8$ & $5.2\pm 5.5$ & $1.3\pm 1.3$ \\[0.5ex] 
	Data & (4, 1) & -- & 0 & 27 & 87 & 127 & 36 & 23 & 21 \\[0.5ex] 
	SM & (4, 1) & -- & $0.6\pm 0.5$ & $31.3\pm 3.8$ & $80.3\pm 5.5$ & $137.5\pm 6.9$ & $48.3\pm 3.4$ & $22.8\pm 2.4$ & $17.9\pm 1.8$ \\[0.5ex] 
	Ttw & (4, 1) & -- & $0.5\pm 0.5$ & $24.6\pm 3.0$ & $62.1\pm 4.3$ & $89.2\pm 4.4$ & $27.7\pm 2.0$ & $9.0\pm 1.0$ & $5.9\pm 0.6$ \\[0.5ex] 
	Zinv & (4, 1) & -- & $0.0\pm 0.0$ & $6.7\pm 0.8$ & $16.9\pm 1.2$ & $38.5\pm 2.1$ & $20.0\pm 1.5$ & $12.9\pm 1.4$ & $11.7\pm 1.2$ \\[0.5ex] 
	QCD & (4, 1) & -- & $0.0\pm 0.0$ & $0.0\pm 0.0$ & $1.3\pm 1.4$ & $9.8\pm 5.5$ & $0.6\pm 0.6$ & $0.9\pm 1.0$ & $0.2\pm 0.2$ \\[0.5ex] 
	Data & (4, 2) & -- & -- & 5 & 23 & 40 & 10 & 1 & 3 \\[0.5ex] 
	SM & (4, 2) & -- & -- & $8.4\pm 1.8$ & $22.3\pm 2.8$ & $35.4\pm 2.8$ & $12.0\pm 1.4$ & $4.7\pm 0.7$ & $1.8\pm 0.4$ \\[0.5ex] 
	Ttw & (4, 2) & -- & -- & $7.2\pm 1.5$ & $19.8\pm 2.6$ & $26.1\pm 2.2$ & $9.0\pm 1.1$ & $2.4\pm 0.3$ & $0.6\pm 0.1$ \\[0.5ex] 
	Zinv & (4, 2) & -- & -- & $1.2\pm 0.3$ & $2.1\pm 0.3$ & $6.5\pm 0.6$ & $2.9\pm 0.3$ & $2.1\pm 0.3$ & $1.2\pm 0.3$ \\[0.5ex] 
	QCD & (4, 2) & -- & -- & $0.0\pm 0.0$ & $0.4\pm 0.4$ & $2.8\pm 1.6$ & $0.1\pm 0.1$ & $0.2\pm 0.2$ & $0.0\pm 0.0$ \\[0.5ex] 
	Data & (4, $\ge3$) & -- & -- & -- & 0 & 1 & 0 & 0 & 0 \\[0.5ex] 
	SM & (4, $\ge3$) & -- & -- & -- & $0.7\pm 0.4$ & $1.7\pm 0.6$ & $0.5\pm 0.2$ & $0.1\pm 0.1$ & $0.1\pm 0.1$ \\[0.5ex] 
	Ttw & (4, $\ge3$) & -- & -- & -- & $0.5\pm 0.3$ & $1.2\pm 0.5$ & $0.3\pm 0.1$ & $0.1\pm 0.0$ & $0.0\pm 0.0$ \\[0.5ex] 
	Zinv & (4, $\ge3$) & -- & -- & -- & $0.1\pm 0.1$ & $0.3\pm 0.1$ & $0.1\pm 0.1$ & $0.0\pm 0.0$ & $0.1\pm 0.0$ \\[0.5ex] 
	QCD & (4, $\ge3$) & -- & -- & -- & $0.0\pm 0.0$ & $0.2\pm 0.1$ & $0.0\pm 0.0$ & $0.0\pm 0.0$ & $0.0\pm 0.0$ \\[0.5ex] 
	Data & ($\ge5$, 0) & -- & -- & -- & 18 & 139 & 114 & 84 & 99 \\[0.5ex] 
	SM & ($\ge5$, 0) & -- & -- & -- & $11.8\pm 2.3$ & $134.8\pm 8.0$ & $108.2\pm 7.8$ & $101.2\pm 5.3$ & $88.1\pm 7.9$ \\[0.5ex] 
	Ttw & ($\ge5$, 0) & -- & -- & -- & $6.6\pm 1.3$ & $77.6\pm 4.7$ & $54.0\pm 3.7$ & $47.0\pm 2.5$ & $33.9\pm 2.0$ \\[0.5ex] 
	Zinv & ($\ge5$, 0) & -- & -- & -- & $5.2\pm 1.0$ & $53.5\pm 3.3$ & $48.6\pm 3.3$ & $53.0\pm 2.8$ & $47.1\pm 2.8$ \\[0.5ex] 
	QCD & ($\ge5$, 0) & -- & -- & -- & $0.0\pm 0.0$ & $3.8\pm 2.8$ & $5.7\pm 6.5$ & $1.2\pm 1.4$ & $7.2\pm 7.9$ \\[0.5ex] 
	Data & ($\ge5$, 1) & -- & -- & -- & 2 & 63 & 53 & 36 & 26 \\[0.5ex] 
	SM & ($\ge5$, 1) & -- & -- & -- & $6.1\pm 1.6$ & $66.1\pm 4.7$ & $49.6\pm 3.6$ & $36.4\pm 2.6$ & $27.2\pm 2.8$ \\[0.5ex] 
	Ttw & ($\ge5$, 1) & -- & -- & -- & $5.2\pm 1.3$ & $53.8\pm 4.0$ & $37.0\pm 2.5$ & $23.5\pm 1.7$ & $14.3\pm 1.4$ \\[0.5ex] 
	Zinv & ($\ge5$, 1) & -- & -- & -- & $0.9\pm 0.2$ & $10.4\pm 0.8$ & $10.2\pm 0.7$ & $12.5\pm 0.9$ & $10.8\pm 1.2$ \\[0.5ex] 
	QCD & ($\ge5$, 1) & -- & -- & -- & $0.0\pm 0.0$ & $1.8\pm 1.3$ & $2.3\pm 2.6$ & $0.4\pm 0.4$ & $2.1\pm 2.3$ \\[0.5ex] 
	Data & ($\ge5$, 2) & -- & -- & -- & 3 & 19 & 19 & 6 & 6 \\[0.5ex] 
	SM & ($\ge5$, 2) & -- & -- & -- & $3.0\pm 0.9$ & $23.9\pm 2.6$ & $16.7\pm 1.8$ & $10.7\pm 1.2$ & $8.9\pm 1.1$ \\[0.5ex] 
	Ttw & ($\ge5$, 2) & -- & -- & -- & $2.9\pm 0.9$ & $21.3\pm 2.3$ & $13.9\pm 1.5$ & $8.6\pm 1.0$ & $5.9\pm 0.7$ \\[0.5ex] 
	Zinv & ($\ge5$, 2) & -- & -- & -- & $0.1\pm 0.0$ & $1.8\pm 0.2$ & $1.9\pm 0.2$ & $1.9\pm 0.2$ & $2.3\pm 0.3$ \\[0.5ex] 
	QCD & ($\ge5$, 2) & -- & -- & -- & $0.0\pm 0.0$ & $0.8\pm 0.6$ & $0.9\pm 1.0$ & $0.1\pm 0.2$ & $0.6\pm 0.7$ \\[0.5ex] 
	Data & ($\ge5$, $\ge3$) & -- & -- & -- & -- & 0 & 0 & 1 & 1 \\[0.5ex] 
	SM & ($\ge5$, $\ge3$) & -- & -- & -- & -- & $1.4\pm 0.6$ & $1.1\pm 0.4$ & $1.2\pm 0.3$ & $0.8\pm 0.3$ \\[0.5ex] 
	Ttw & ($\ge5$, $\ge3$) & -- & -- & -- & -- & $1.2\pm 0.5$ & $1.0\pm 0.3$ & $0.7\pm 0.2$ & $0.5\pm 0.2$ \\[0.5ex] 
	Zinv & ($\ge5$, $\ge3$) & -- & -- & -- & -- & $0.2\pm 0.1$ & $0.1\pm 0.0$ & $0.4\pm 0.1$ & $0.3\pm 0.1$ \\[0.5ex] 
	QCD & ($\ge5$, $\ge3$) & -- & -- & -- & -- & $0.0\pm 0.0$ & $0.1\pm 0.1$ & $0.0\pm 0.0$ & $0.1\pm 0.1$ \\[0.5ex] 
	\hline
	\hline
\end{tabular}}
\end{table}

\clearpage
\begin{table}[h!]
\tiny
\centering
\caption{Post fit Predictions and Data in the signal region for 2.24\ifb for asymmetric categories. The letter ``a'' in jet \eg ``2a''  indicates the asymmetric jet bins. All entries are non-zero but are truncated to one decimal place.\label{tab:predallqcdpost_sig_comb_asym}}
\scalebox{0.85}{\begin{tabular}{cccccccccc}
	\hline\hline
	&	& \multicolumn{8}{c}{\scalht (\gev)}\\ 
	&	 (\njet, \nb) & 200-250 & 250-300 & 300-350 & 350-400 & 400-500 & 500-600 & 600-800 & 800-$\infty$ \\ [0.8ex] 
\hline
	Data & (2a, 0) & 5788 & 1585 & 584 & 232 & 139 & 26 & 16 & -- \\[0.5ex] 
	SM & (2a, 0) & $5831.5\pm 81.6$ & $1621.6\pm 35.0$ & $581.0\pm 17.9$ & $227.5\pm 9.5$ & $136.5\pm 6.4$ & $30.0\pm 2.6$ & $18.3\pm 3.2$ & -- \\[0.5ex] 
	Ttw & (2a, 0) & $2765.7\pm 96.3$ & $673.4\pm 40.7$ & $236.7\pm 13.7$ & $81.1\pm 5.7$ & $42.0\pm 3.7$ & $10.2\pm 1.6$ & $3.1\pm 1.0$ & -- \\[0.5ex] 
	Zinv & (2a, 0) & $2901.1\pm 78.1$ & $913.7\pm 38.8$ & $344.4\pm 14.4$ & $146.5\pm 7.4$ & $94.5\pm 5.6$ & $19.8\pm 1.9$ & $12.6\pm 2.2$ & -- \\[0.5ex] 
	QCD & (2a, 0) & $107.2\pm 110.8$ & $16.5\pm 17.2$ & $0.0\pm 0.9$ & $0.0\pm 0.8$ & $0.0\pm 1.4$ & $0.0\pm 1.6$ & $1.1\pm 4.3$ & -- \\[0.5ex] 
	Data & (2a, 1) & 536 & 152 & 51 & 18 & 7 & 4 & -- & -- \\[0.5ex] 
	SM & (2a, 1) & $541.0\pm 16.4$ & $153.8\pm 7.1$ & $49.7\pm 3.7$ & $19.7\pm 2.2$ & $10.7\pm 1.4$ & $4.0\pm 1.0$ & -- & -- \\[0.5ex] 
	Ttw & (2a, 1) & $320.7\pm 14.2$ & $78.2\pm 5.7$ & $22.3\pm 1.8$ & $6.8\pm 0.8$ & $4.0\pm 0.6$ & $1.2\pm 0.4$ & -- & -- \\[0.5ex] 
	Zinv & (2a, 1) & $205.1\pm 8.7$ & $72.4\pm 4.4$ & $27.4\pm 2.3$ & $13.0\pm 1.6$ & $6.8\pm 0.9$ & $2.8\pm 0.7$ & -- & -- \\[0.5ex] 
	QCD & (2a, 1) & $9.9\pm 10.3$ & $1.6\pm 1.6$ & $0.0\pm 0.1$ & $0.0\pm 0.1$ & $0.0\pm 0.1$ & $0.0\pm 0.2$ & -- & -- \\[0.5ex] 
	Data & (2a, 2) & 31 & 10 & 3 & 1 & 0 & -- & -- & -- \\[0.5ex] 
	SM & (2a, 2) & $29.5\pm 3.5$ & $7.4\pm 1.2$ & $5.0\pm 1.1$ & $1.8\pm 0.6$ & $0.6\pm 0.3$ & -- & -- & -- \\[0.5ex] 
	Ttw & (2a, 2) & $14.7\pm 1.9$ & $3.3\pm 0.6$ & $2.2\pm 0.5$ & $1.1\pm 0.4$ & $0.3\pm 0.1$ & -- & -- & -- \\[0.5ex] 
	Zinv & (2a, 2) & $14.0\pm 1.7$ & $3.9\pm 0.7$ & $2.7\pm 0.6$ & $0.8\pm 0.2$ & $0.3\pm 0.2$ & -- & -- & -- \\[0.5ex] 
	QCD & (2a, 2) & $0.5\pm 0.6$ & $0.1\pm 0.1$ & $0.0\pm 0.0$ & $0.0\pm 0.0$ & $0.0\pm 0.0$ & -- & -- & -- \\[0.5ex] 
	Data & (3a, 0) & 1599 & 1609 & 777 & 239 & 95 & 15 & 9 & -- \\[0.5ex] 
	SM & (3a, 0) & $1624.8\pm 39.6$ & $1561.5\pm 44.5$ & $781.7\pm 28.8$ & $258.7\pm 13.4$ & $100.9\pm 5.4$ & $16.5\pm 1.7$ & $8.0\pm 1.5$ & -- \\[0.5ex] 
	Ttw & (3a, 0) & $849.7\pm 32.7$ & $740.6\pm 47.7$ & $352.1\pm 20.7$ & $100.3\pm 7.8$ & $37.4\pm 3.4$ & $5.2\pm 0.9$ & $1.9\pm 0.7$ & -- \\[0.5ex] 
	Zinv & (3a, 0) & $718.1\pm 24.8$ & $705.6\pm 32.7$ & $347.7\pm 17.2$ & $134.7\pm 7.3$ & $63.5\pm 4.1$ & $11.3\pm 1.3$ & $6.1\pm 1.2$ & -- \\[0.5ex] 
	QCD & (3a, 0) & $32.6\pm 34.9$ & $26.2\pm 27.3$ & $34.4\pm 35.9$ & $11.1\pm 12.1$ & $0.0\pm 0.2$ & $0.0\pm 1.2$ & $0.0\pm 193.6$ & -- \\[0.5ex] 
	Data & (3a, 1) & 340 & 299 & 152 & 59 & 15 & 1 & 1 & -- \\[0.5ex] 
	SM & (3a, 1) & $340.7\pm 12.6$ & $335.8\pm 13.1$ & $148.6\pm 7.9$ & $47.1\pm 3.6$ & $13.0\pm 1.4$ & $2.0\pm 0.5$ & $1.0\pm 0.4$ & -- \\[0.5ex] 
	Ttw & (3a, 1) & $260.9\pm 11.3$ & $237.2\pm 15.6$ & $96.9\pm 6.4$ & $27.3\pm 2.6$ & $6.1\pm 0.8$ & $1.4\pm 0.3$ & $0.3\pm 0.1$ & -- \\[0.5ex] 
	Zinv & (3a, 1) & $68.0\pm 3.5$ & $73.5\pm 4.5$ & $36.0\pm 2.4$ & $15.2\pm 1.4$ & $6.8\pm 0.7$ & $0.7\pm 0.1$ & $0.7\pm 0.3$ & -- \\[0.5ex] 
	QCD & (3a, 1) & $6.8\pm 7.3$ & $5.6\pm 5.9$ & $6.5\pm 6.8$ & $2.0\pm 2.2$ & $0.0\pm 0.0$ & $0.0\pm 0.1$ & $0.0\pm 24.0$ & -- \\[0.5ex] 
	Data & (3a, 2) & 52 & 62 & 29 & 12 & 1 & 0 & -- & -- \\[0.5ex] 
	SM & (3a, 2) & $58.3\pm 4.6$ & $60.1\pm 3.8$ & $31.3\pm 2.7$ & $11.1\pm 1.5$ & $1.5\pm 0.4$ & $0.4\pm 0.2$ & -- & -- \\[0.5ex] 
	Ttw & (3a, 2) & $47.4\pm 3.8$ & $45.5\pm 3.6$ & $22.8\pm 2.4$ & $8.2\pm 1.2$ & $0.6\pm 0.2$ & $0.1\pm 0.1$ & -- & -- \\[0.5ex] 
	Zinv & (3a, 2) & $9.0\pm 0.7$ & $10.0\pm 0.8$ & $5.2\pm 0.6$ & $1.9\pm 0.3$ & $0.9\pm 0.2$ & $0.2\pm 0.1$ & -- & -- \\[0.5ex] 
	QCD & (3a, 2) & $1.2\pm 1.3$ & $1.0\pm 1.1$ & $1.4\pm 1.4$ & $0.5\pm 0.5$ & $0.0\pm 0.0$ & $0.0\pm 0.0$ & -- & -- \\[0.5ex] 
	Data & (3a, $\ge3$) & 3 & 1 & 1 & -- & -- & -- & -- & -- \\[0.5ex] 
	SM & (3a, $\ge3$) & $1.3\pm 0.5$ & $1.5\pm 0.6$ & $0.8\pm 0.4$ & -- & -- & -- & -- & -- \\[0.5ex] 
	Ttw & (3a, $\ge3$) & $1.1\pm 0.4$ & $1.1\pm 0.4$ & $0.7\pm 0.4$ & -- & -- & -- & -- & -- \\[0.5ex] 
	Zinv & (3a, $\ge3$) & $0.1\pm 0.1$ & $0.3\pm 0.1$ & $0.0\pm 0.0$ & -- & -- & -- & -- & -- \\[0.5ex] 
	QCD & (3a, $\ge3$) & $0.0\pm 0.0$ & $0.0\pm 0.0$ & $0.0\pm 0.0$ & -- & -- & -- & -- & -- \\[0.5ex] 
	Data & (4a, 0) & 3 & 178 & 412 & 246 & 119 & 15 & 2 & -- \\[0.5ex] 
	SM & (4a, 0) & $4.1\pm 1.0$ & $159.3\pm 10.8$ & $418.0\pm 19.5$ & $256.7\pm 14.2$ & $128.0\pm 7.6$ & $12.9\pm 1.7$ & $2.2\pm 0.6$ & -- \\[0.5ex] 
	Ttw & (4a, 0) & $2.1\pm 0.5$ & $88.9\pm 6.7$ & $205.7\pm 13.9$ & $132.1\pm 10.1$ & $55.5\pm 4.9$ & $4.9\pm 0.9$ & $0.5\pm 0.2$ & -- \\[0.5ex] 
	Zinv & (4a, 0) & $2.0\pm 0.5$ & $64.3\pm 4.2$ & $160.1\pm 9.3$ & $102.3\pm 6.8$ & $63.0\pm 4.1$ & $8.0\pm 1.1$ & $1.7\pm 0.5$ & -- \\[0.5ex] 
	QCD & (4a, 0) & $0.0\pm 0.0$ & $1.5\pm 1.8$ & $30.0\pm 31.2$ & $13.1\pm 13.9$ & $4.9\pm 5.5$ & $0.0\pm 0.1$ & $0.0\pm 0.2$ & -- \\[0.5ex] 
	Data & (4a, 1) & 1 & 53 & 180 & 96 & 51 & 4 & 0 & -- \\[0.5ex] 
	SM & (4a, 1) & $1.6\pm 0.5$ & $50.7\pm 4.6$ & $172.6\pm 9.2$ & $98.9\pm 6.8$ & $49.0\pm 3.9$ & $2.9\pm 0.6$ & $0.5\pm 0.1$ & -- \\[0.5ex] 
	Ttw & (4a, 1) & $1.1\pm 0.3$ & $38.3\pm 3.2$ & $127.7\pm 8.5$ & $74.3\pm 6.3$ & $33.4\pm 3.2$ & $1.7\pm 0.4$ & $0.1\pm 0.0$ & -- \\[0.5ex] 
	Zinv & (4a, 1) & $0.5\pm 0.1$ & $10.4\pm 1.0$ & $23.0\pm 1.7$ & $16.1\pm 1.4$ & $11.8\pm 1.2$ & $1.1\pm 0.2$ & $0.4\pm 0.1$ & -- \\[0.5ex] 
	QCD & (4a, 1) & $0.0\pm 0.0$ & $0.5\pm 0.6$ & $12.4\pm 12.9$ & $5.0\pm 5.4$ & $1.9\pm 2.1$ & $0.0\pm 0.0$ & $0.0\pm 0.0$ & -- \\[0.5ex] 
	Data & (4a, 2) & 0 & 11 & 44 & 30 & 8 & 0 & 0 & -- \\[0.5ex] 
	SM & (4a, 2) & $0.3\pm 0.2$ & $13.9\pm 1.5$ & $51.6\pm 4.3$ & $28.7\pm 2.9$ & $12.7\pm 1.6$ & $0.6\pm 0.2$ & $0.1\pm 0.0$ & -- \\[0.5ex] 
	Ttw & (4a, 2) & $0.3\pm 0.2$ & $11.1\pm 1.2$ & $41.7\pm 3.8$ & $24.1\pm 2.7$ & $10.3\pm 1.4$ & $0.4\pm 0.2$ & $0.0\pm 0.0$ & -- \\[0.5ex] 
	Zinv & (4a, 2) & $0.0\pm 0.0$ & $2.3\pm 0.3$ & $3.6\pm 0.3$ & $2.2\pm 0.3$ & $1.5\pm 0.2$ & $0.1\pm 0.0$ & $0.0\pm 0.0$ & -- \\[0.5ex] 
	QCD & (4a, 2) & $0.0\pm 0.0$ & $0.1\pm 0.2$ & $3.7\pm 3.8$ & $1.5\pm 1.6$ & $0.5\pm 0.5$ & $0.0\pm 0.0$ & $0.0\pm 0.0$ & -- \\[0.5ex] 
	Data & (4a, $\ge3$) & -- & 0 & 0 & 2 & 2 & -- & -- & -- \\[0.5ex] 
	SM & (4a, $\ge3$) & -- & $1.3\pm 0.5$ & $2.4\pm 0.8$ & $2.3\pm 0.7$ & $2.1\pm 0.6$ & -- & -- & -- \\[0.5ex] 
	Ttw & (4a, $\ge3$) & -- & $1.2\pm 0.5$ & $1.9\pm 0.7$ & $1.9\pm 0.6$ & $1.8\pm 0.6$ & -- & -- & -- \\[0.5ex] 
	Zinv & (4a, $\ge3$) & -- & $0.1\pm 0.0$ & $0.2\pm 0.1$ & $0.3\pm 0.1$ & $0.0\pm 0.0$ & -- & -- & -- \\[0.5ex] 
	QCD & (4a, $\ge3$) & -- & $0.0\pm 0.0$ & $0.2\pm 0.2$ & $0.1\pm 0.1$ & $0.1\pm 0.1$ & -- & -- & -- \\[0.5ex] 
	Data & ($\ge5$a, 0) & -- & 3 & 40 & 96 & 105 & 20 & 3 & -- \\[0.5ex] 
	SM & ($\ge5$a, 0) & -- & $2.9\pm 1.3$ & $43.5\pm 4.7$ & $107.8\pm 8.9$ & $114.4\pm 8.6$ & $19.6\pm 2.6$ & $3.3\pm 0.9$ & -- \\[0.5ex] 
	Ttw & ($\ge5$a, 0) & -- & $2.5\pm 1.1$ & $28.3\pm 3.3$ & $57.4\pm 5.6$ & $65.8\pm 6.7$ & $11.1\pm 1.9$ & $1.3\pm 0.5$ & -- \\[0.5ex] 
	Zinv & ($\ge5$a, 0) & -- & $0.4\pm 0.3$ & $14.8\pm 1.7$ & $33.7\pm 3.0$ & $35.7\pm 3.0$ & $7.5\pm 1.0$ & $2.0\pm 0.5$ & -- \\[0.5ex] 
	QCD & ($\ge5$a, 0) & -- & $0.0\pm 0.0$ & $0.1\pm 0.9$ & $13.1\pm 14.8$ & $6.2\pm 7.0$ & $0.3\pm 0.4$ & $0.0\pm 44.7$ & -- \\[0.5ex] 
	Data & ($\ge5$a, 1) & -- & 0 & 24 & 60 & 74 & 15 & 0 & -- \\[0.5ex] 
	SM & ($\ge5$a, 1) & -- & $0.8\pm 0.5$ & $22.2\pm 3.0$ & $57.4\pm 5.9$ & $71.9\pm 5.7$ & $15.4\pm 2.3$ & $1.4\pm 0.5$ & -- \\[0.5ex] 
	Ttw & ($\ge5$a, 1) & -- & $0.6\pm 0.4$ & $19.7\pm 2.7$ & $44.4\pm 5.0$ & $57.6\pm 5.8$ & $12.3\pm 2.2$ & $1.0\pm 0.4$ & -- \\[0.5ex] 
	Zinv & ($\ge5$a, 1) & -- & $0.1\pm 0.1$ & $2.4\pm 0.3$ & $4.5\pm 0.5$ & $6.4\pm 0.6$ & $2.3\pm 0.4$ & $0.4\pm 0.1$ & -- \\[0.5ex] 
	QCD & ($\ge5$a, 1) & -- & $0.0\pm 0.0$ & $0.1\pm 0.5$ & $7.0\pm 7.9$ & $3.9\pm 4.4$ & $0.3\pm 0.3$ & $0.0\pm 19.7$ & -- \\[0.5ex] 
	Data & ($\ge5$a, 2) & -- & 0 & 11 & 27 & 29 & 6 & 1 & -- \\[0.5ex] 
	SM & ($\ge5$a, 2) & -- & $0.0\pm 0.0$ & $7.9\pm 1.7$ & $27.3\pm 3.3$ & $28.9\pm 3.3$ & $5.5\pm 1.2$ & $0.4\pm 0.2$ & -- \\[0.5ex] 
	Ttw & ($\ge5$a, 2) & -- & $0.0\pm 0.0$ & $7.5\pm 1.6$ & $22.1\pm 2.9$ & $24.4\pm 3.1$ & $4.6\pm 1.1$ & $0.4\pm 0.2$ & -- \\[0.5ex] 
	Zinv & ($\ge5$a, 2) & -- & $0.0\pm 0.0$ & $0.3\pm 0.1$ & $1.1\pm 0.2$ & $1.3\pm 0.2$ & $0.6\pm 0.1$ & $0.0\pm 0.0$ & -- \\[0.5ex] 
	QCD & ($\ge5$a, 2) & -- & $0.0\pm 0.0$ & $0.0\pm 0.2$ & $3.3\pm 3.8$ & $1.6\pm 1.8$ & $0.1\pm 0.1$ & $0.0\pm 6.0$ & -- \\[0.5ex] 
	Data & ($\ge5$a, $\ge3$) & -- & -- & 0 & 2 & 5 & 1 & -- & -- \\[0.5ex] 
	SM & ($\ge5$a, $\ge3$) & -- & -- & $0.5\pm 0.3$ & $3.0\pm 0.9$ & $4.5\pm 1.1$ & $0.9\pm 0.4$ & -- & -- \\[0.5ex] 
	Ttw & ($\ge5$a, $\ge3$) & -- & -- & $0.5\pm 0.3$ & $2.5\pm 0.9$ & $3.8\pm 1.0$ & $0.8\pm 0.3$ & -- & -- \\[0.5ex] 
	Zinv & ($\ge5$a, $\ge3$) & -- & -- & $0.0\pm 0.0$ & $0.0\pm 0.0$ & $0.2\pm 0.0$ & $0.1\pm 0.0$ & -- & -- \\[0.5ex] 
	QCD & ($\ge5$a, $\ge3$) & -- & -- & $0.0\pm 0.0$ & $0.4\pm 0.4$ & $0.2\pm 0.3$ & $0.0\pm 0.0$ & -- & -- \\[0.5ex] 
	\hline
	\hline
\end{tabular}}
\end{table}





%%%% Summary plots
\clearpage
\begin{landscape}
  \begin{center}
    \begin{figure}[h!]
      \caption{Background yield predictions and data obervation for the (\njet,\nb,\scalht) analysis bins (integrated over \MHT) in the monojet topologies. \label{fig:summaryPlot_Monojet}}.
      \includegraphics[width=0.8\linewidth]{figures/postFitResults/summaryPlots/summaryPlot_Monojet_prefit_overlay_fit_b}
    \end{figure}
  \end{center}
\end{landscape}

\clearpage
\begin{landscape}
  \begin{center}
    \begin{figure}[h!]
      \caption{Background yield predictions and data obervation for the (\njet,\nb,\scalht) analysis bins (integrated over \MHT) in the asymmetric topologies. \label{fig:summaryPlot_Asymmetric}}.
      \includegraphics[width=0.8\linewidth]{figures/postFitResults/summaryPlots/summaryPlot_Asymmetric_prefit_overlay_fit_b}
    \end{figure}
  \end{center}
\end{landscape}

\clearpage
\begin{landscape}
  \begin{center}
    \begin{figure}[h!]
      \caption{Background yield predictions and data obervation for the (\njet,\nb,\scalht) analysis bins (integrated over \MHT) in the symmetric topologies. \label{fig:summaryPlot_Symmetric}}.
      \includegraphics[width=0.8\linewidth]{figures/postFitResults/summaryPlots/summaryPlot_Symmetric_prefit_overlay_fit_b}
    \end{figure}
  \end{center}
\end{landscape}





\clearpage
\begin{figure}[tbhp]
    \caption{ Pull of the nuisances parameters associated to the \alt-extrapolation systematic uncertainty, 
      for the asymmetric (symmetric) categories on the left (right).
      \label{fig:nuisPull_AlphaT}}
  \begin{center}
    \subfigure[Asymmetric categories]{ \includegraphics[width=0.45\textwidth]{figures/postFitResults/nuisancePlots/AlphaT_asym_nuisances} } ~~
    \subfigure[Symmetric categories]{ \includegraphics[width=0.45\textwidth]{figures/postFitResults/nuisancePlots/AlphaT_sym_nuisances} }
  \end{center}
\end{figure}

\begin{figure}[tbhp]
    \caption{ Pull of the nuisances parameters associated to the $\gamma/Z$ ratio systematic uncertainty, 
      for the asymmetric (symmetric) categories on the left (right).
      \label{fig:nuisPull_gamma_Z_ratio}}
  \begin{center}
    \subfigure[Asymmetric categories]{ \includegraphics[width=0.45\textwidth]{figures/postFitResults/nuisancePlots/gamma_Z_ratio_asym_nuisances} } ~~
    \subfigure[Symmetric categories]{ \includegraphics[width=0.45\textwidth]{figures/postFitResults/nuisancePlots/gamma_Z_ratio_sym_nuisances} }
  \end{center}
\end{figure}

\begin{figure}[tbhp]
    \caption{ Pull of the nuisances parameters associated to the $W/Z$ ratio systematic uncertainty, 
      for the asymmetric (symmetric) categories on the left (right).
      \label{fig:nuisPull_W_Z_ratio}}
  \begin{center}
    \subfigure[Asymmetric categories]{ \includegraphics[width=0.45\textwidth]{figures/postFitResults/nuisancePlots/W_Z_ratio_asym_nuisances} } ~~
    \subfigure[Symmetric categories]{ \includegraphics[width=0.45\textwidth]{figures/postFitResults/nuisancePlots/W_Z_ratio_sym_nuisances} }
  \end{center}
\end{figure}


\begin{figure}[tbhp]
    \caption{ Pull of the nuisances parameters associated to the $\ttbar/W$ admixture systematic uncertainty, 
      for the asymmetric (symmetric) categories on the left (right).
      \label{fig:nuisPull_tt_W_admixture}}
  \begin{center}
    \subfigure[Asymmetric categories]{ \includegraphics[width=0.45\textwidth]{figures/postFitResults/nuisancePlots/tt_W_admixture_asym_nuisances} } ~~
    \subfigure[Symmetric categories]{ \includegraphics[width=0.45\textwidth]{figures/postFitResults/nuisancePlots/tt_W_admixture_sym_nuisances} }
  \end{center}
\end{figure}


\begin{figure}[tbhp]
    \caption{ Pull of the nuisances parameters associated to the W polarisation systematic uncertainty, 
      for the asymmetric (symmetric) categories on the left (right).
      \label{fig:nuisPull_WPol}}
  \begin{center}
    \subfigure[Asymmetric categories]{ \includegraphics[width=0.45\textwidth]{figures/postFitResults/nuisancePlots/WPol_asym_nuisances} } ~~
    \subfigure[Symmetric categories]{ \includegraphics[width=0.45\textwidth]{figures/postFitResults/nuisancePlots/WPol_sym_nuisances} }
  \end{center}
\end{figure}


\begin{figure}[tbhp]
    \caption{ Pull of the nuisances parameters associated to the $\ttbar+W$ template systematic uncertainty, 
      for the asymmetric (symmetric) categories on the left (right).
      \label{fig:nuisPull_TemplateTtw}}
  \begin{center}
    \subfigure[Asymmetric categories]{ \includegraphics[width=0.8\textwidth]{figures/postFitResults/nuisancePlots/TemplateTtw_asym_nuisances} } \\
    \subfigure[Symmetric categories]{ \includegraphics[width=0.8\textwidth]{figures/postFitResults/nuisancePlots/TemplateTtw_sym_nuisances} }
  \end{center}
\end{figure}


\begin{figure}[tbhp]
    \caption{ Pull of the nuisances parameters associated to the $\ttbar+W$ template systematic uncertainty, 
      for the asymmetric (symmetric) categories on the left (right).
      \label{fig:nuisPull_TemplateZinv}}
  \begin{center}
    \subfigure[Asymmetric categories]{ \includegraphics[width=0.8\textwidth]{figures/postFitResults/nuisancePlots/TemplateZinv_asym_nuisances} } \\
    \subfigure[Symmetric categories]{ \includegraphics[width=0.8\textwidth]{figures/postFitResults/nuisancePlots/TemplateZinv_sym_nuisances} }
  \end{center}
\end{figure}


\begin{figure}[tbhp]
    \caption{ Pull of the nuisances parameters associated to the QCD systematic uncertainty, 
      for the asymmetric (symmetric) categories on the left (right).
      \label{fig:nuisPull_qcd}}
  \begin{center}
    \subfigure[Asymmetric categories]{ \includegraphics[width=0.45\textwidth]{figures/postFitResults/nuisancePlots/qcd_asym_nuisances} } ~~
    \subfigure[Symmetric categories]{ \includegraphics[width=0.45\textwidth]{figures/postFitResults/nuisancePlots/qcd_sym_nuisances} }
  \end{center}
\end{figure}


\begin{figure}[tbhp]
    \caption{ Pull of the nuisances parameters associated to the correlated systematic uncertainties. 
      \label{fig:nuisPull_Correlated}}
  \begin{center}
    \includegraphics[width=0.8\textwidth]{figures/postFitResults/nuisancePlots/Correlated_nuisances}
  \end{center}
\end{figure}

\clearpage

\begin{figure}[tbhp]
    \begin{center}
        \subfigure[Symmetric Uncorrected]{\includegraphics[width=0.38\textwidth]{figures/uncorrectedShapes/sym/all/ht40_sym_all}} ~~
        \subfigure[Symmetric Corrected] {\includegraphics[width=0.38\textwidth]{figures/correctedShapes/sym/all/ht40_sym_all}}\\
        \subfigure[Asymmetric Uncorrected] {\includegraphics[width=0.38\textwidth]{figures/uncorrectedShapes/asym/all/ht40_asym_all}} ~~
        \subfigure[Asymmetric Corrected] {\includegraphics[width=0.38\textwidth]{figures/correctedShapes/asym/all/ht40_asym_all}}\\
        \subfigure[Monojet Unorrected] {\includegraphics[width=0.38\textwidth]{figures/uncorrectedShapes/mono/all/ht40_mono_all}} ~~
        \subfigure[Monojet Corrected]{\includegraphics[width=0.38\textwidth]{figures/correctedShapes/mono/all/ht40_mono_all}}\\
        \caption{The \scalht distribution comparing data and prediction agreement before and after applying the correction from the control region only fit. Also shown are the results of a linear fit to the data/prediction ratio (p0,p1 parameters) which confirms the agreeement is significantly improved through the correction applied in the control regions}
        \label{fig:shapesht}
    \end{center}
\end{figure}

\begin{figure}[tbhp]
    \begin{center}
        \subfigure[Symmetric Uncorrected]{\includegraphics[width=0.38\textwidth]{figures/uncorrectedShapes/sym/all/mht40_pt_sym_all}} ~~
        \subfigure[Symmetric Corrected] {\includegraphics[width=0.38\textwidth]{figures/correctedShapes/sym/all/mht40_pt_sym_all}}\\
        \subfigure[Asymmetric Uncorrected] {\includegraphics[width=0.38\textwidth]{figures/uncorrectedShapes/asym/all/mht40_pt_asym_all}} ~~
        \subfigure[Asymmetric Corrected] {\includegraphics[width=0.38\textwidth]{figures/correctedShapes/asym/all/mht40_pt_asym_all}}\\
        \subfigure[Monojet Unorrected] {\includegraphics[width=0.38\textwidth]{figures/uncorrectedShapes/mono/all/mht40_pt_mono_all}} ~~
        \subfigure[Monojet Corrected]{\includegraphics[width=0.38\textwidth]{figures/correctedShapes/mono/all/mht40_pt_mono_all}}\\
        \caption{The \mht distribution comparing data and prediction agreement before and after applying the correction from the control region only fit. Also shown are the results of a linear fit to the data/prediction ratio (p0,p1 parameters) which confirms the agreeement is significantly improved through the correction applied in the control regions}
    \end{center}
\end{figure}
\begin{figure}[tbhp]
    \begin{center}
        \subfigure[Symmetric Uncorrected]{\includegraphics[width=0.38\textwidth]{figures/uncorrectedShapes/sym/all/alphaT_sym_all}} ~~
        \subfigure[Symmetric Corrected] {\includegraphics[width=0.38\textwidth]{figures/correctedShapes/sym/all/alphaT_sym_all}}\\
        \subfigure[Asymmetric Uncorrected] {\includegraphics[width=0.38\textwidth]{figures/uncorrectedShapes/asym/all/alphaT_asym_all}} ~~
        \subfigure[Asymmetric Corrected] {\includegraphics[width=0.38\textwidth]{figures/correctedShapes/asym/all/alphaT_asym_all}}\\
        \caption{The \mht distribution comparing data and prediction agreement before and after applying the correction from the control region only fit. Also shown are the results of a linear fit to the data/prediction ratio (p0,p1 parameters) which confirms the agreeement is significantly improved through the correction applied in the control regions}
    \end{center}
\end{figure}
\begin{figure}[tbhp]
    \begin{center}
        \subfigure[Symmetric Uncorrected]{\includegraphics[width=0.38\textwidth]{figures/uncorrectedShapes/sym/all/biasedDPhi_sym_all}} ~~
        \subfigure[Symmetric Corrected] {\includegraphics[width=0.38\textwidth]{figures/correctedShapes/sym/all/biasedDPhi_sym_all}}\\
        \subfigure[Asymmetric Uncorrected] {\includegraphics[width=0.38\textwidth]{figures/uncorrectedShapes/asym/all/biasedDPhi_asym_all}} ~~
        \subfigure[Asymmetric Corrected] {\includegraphics[width=0.38\textwidth]{figures/correctedShapes/asym/all/biasedDPhi_asym_all}}\\
        \subfigure[Monojet Unorrected] {\includegraphics[width=0.38\textwidth]{figures/uncorrectedShapes/mono/all/biasedDPhi20_mono_all}} ~~
        \subfigure[Monojet Corrected]{\includegraphics[width=0.38\textwidth]{figures/correctedShapes/mono/all/biasedDPhi20_mono_all}}\\
        \caption{The \bdphi distribution (including jets down to \pt = 20 GeV for monojet) comparing data and prediction agreement before and after applying the correction from the control region only fit. Also shown are the results of a linear fit to the data/prediction ratio (p0,p1 parameters) which confirms the agreeement is significantly improved through the correction applied in the control regions}
    \end{center}
\end{figure}
\begin{figure}[tbhp]
    \begin{center}
        \subfigure[Symmetric uncorrected]{\includegraphics[width=0.38\textwidth]{figures/uncorrectedShapes/sym/all/jet_pt[0]_sym_all}} ~~
        \subfigure[Symmetric corrected] {\includegraphics[width=0.38\textwidth]{figures/correctedShapes/sym/all/jet_pt[0]_sym_all}}\\
        \subfigure[Asymmetric uncorrected] {\includegraphics[width=0.38\textwidth]{figures/uncorrectedShapes/asym/all/jet_pt[0]_asym_all}} ~~
        \subfigure[Asymmetric corrected] {\includegraphics[width=0.38\textwidth]{figures/correctedShapes/asym/all/jet_pt[0]_asym_all}}\\
        \subfigure[Monojet uncorrected] {\includegraphics[width=0.38\textwidth]{figures/uncorrectedShapes/mono/all/jet_pt[0]_mono_all}} ~~
        \subfigure[Monojet corrected]{\includegraphics[width=0.38\textwidth]{figures/correctedShapes/mono/all/jet_pt[0]_mono_all}}\\
        \caption{The \bdphi distribution comparing data and prediction agreement before and after applying the correction from the control region only fit. Also shown are the results of a linear fit to the data/prediction ratio (p0,p1 parameters) which confirms the agreeement is significantly improved through the correction applied in the control regions}
    \end{center}
\end{figure}
\begin{figure}[tbhp]
    \begin{center}
        \subfigure[Symmetric Uncorrected]{\includegraphics[width=0.38\textwidth]{figures/uncorrectedShapes/sym/all/jet_pt[2]_sym_all}} ~~
        \subfigure[Symmetric Corrected] {\includegraphics[width=0.38\textwidth]{figures/correctedShapes/sym/all/jet_pt[2]_sym_all}}\\
        \subfigure[Asymmetric Uncorrected] {\includegraphics[width=0.38\textwidth]{figures/uncorrectedShapes/asym/all/jet_pt[2]_asym_all}} ~~
        \subfigure[Asymmetric Corrected] {\includegraphics[width=0.38\textwidth]{figures/correctedShapes/asym/all/jet_pt[2]_asym_all}}\\
        \caption{The third jet \pt distribution comparing data and prediction agreement before and after applying the correction from the control region only fit. Also shown are the results of a linear fit to the data/prediction ratio (p0,p1 parameters) which confirms the agreeement is significantly improved through the correction applied in the control regions}
    \end{center}
\end{figure}
\begin{figure}[tbhp]
    \begin{center}
        \subfigure[Symmetric uncorrected]{\includegraphics[width=0.38\textwidth]{figures/uncorrectedShapes/sym/all/nJet40_sym_all}} ~~
        \subfigure[Symmetric corrected] {\includegraphics[width=0.38\textwidth]{figures/correctedShapes/sym/all/nJet40_sym_all}}\\
        \subfigure[Asymmetric uncorrected] {\includegraphics[width=0.38\textwidth]{figures/uncorrectedShapes/asym/all/nJet40_asym_all}} ~~
        \subfigure[Asymmetric corrected] {\includegraphics[width=0.38\textwidth]{figures/correctedShapes/asym/all/nJet40_asym_all}}\\
        \caption{The \njet distribution comparing data and prediction agreement before and after applying the correction from the control region only fit. Also shown are the results of a linear fit to the data/prediction ratio (p0,p1 parameters) which confirms the agreeement is significantly improved through the correction applied in the control regions}
    \end{center}
\end{figure}
\begin{figure}[tbhp]
    \begin{center}
        \subfigure[Symmetric uncorrected]{\includegraphics[width=0.38\textwidth]{figures/uncorrectedShapes/sym/all/nBJet40_sym_all}} ~~
        \subfigure[Symmetric corrected] {\includegraphics[width=0.38\textwidth]{figures/correctedShapes/sym/all/nBJet40_sym_all}}\\
        \subfigure[Asymmetric uncorrected] {\includegraphics[width=0.38\textwidth]{figures/uncorrectedShapes/asym/all/nBJet40_asym_all}} ~~
        \subfigure[Asymmetric corrected] {\includegraphics[width=0.38\textwidth]{figures/correctedShapes/asym/all/nBJet40_asym_all}}\\
        \subfigure[Monojet uncorrected] {\includegraphics[width=0.38\textwidth]{figures/uncorrectedShapes/mono/all/nBJet40_mono_all}} ~~
        \subfigure[Monojet corrected]{\includegraphics[width=0.38\textwidth]{figures/correctedShapes/mono/all/nBJet40_mono_all}}\\
        \caption{The \nb distribution comparing data and prediction agreement before and after applying the correction from the control region only fit. Also shown are the results of a linear fit to the data/prediction ratio (p0,p1 parameters) which confirms the agreeement is significantly improved through the correction applied in the control regions}
        \label{fig:shapesbjet}
    \end{center}
\end{figure}
