%%____________________________________________________________________________||
\section{Likelihood model}
\label{sec:likelihood}

Consider a given category of event as defined by \njet, \nb and \HT, which will be in the following identified with \htcat. 
In each category, the signal is extracted using the \mht shape as a discriminating variable. 
Histogram templates of the \mht distribution are built for the signal and the background processes 
using the MC samples described in section \ref{sec:datasets}. 
The binning of the templates is chosen taking into account both the limited statistics of the MC samples and 
the control of the background. To fulfill both these objectives, each bin should have, at the same time:
\begin{itemize}
\item at least 10 unweighted MC counts
\item at least 10 events in the corresponding bin of each of the control samples relevant for the background prediction (inclusively on the b-tag requirement)
\end{itemize}

The first requirement ensures that the MC statistical uncertainty is negligible with respect to the other sources of systematic uncertainties. \textbf{FIXME: this sounds too aggressive!} \\
The second requirement allows the perform a statistically meaningful closure test in data up to the highest \mht values. \\
Furthermore, a minimum bin width constraint of 50 GeV is applied, in order to have negligible bin-by-bin migration. \\ 

For each \mht bin $i$, let $n^{i}$ be the number of observed events, $b^{i}$ the number of predicted 
background events and $s^{i}$ the expected number of signal events. \\
The likelihood function for the hadronic signal region is:

\begin{equation}
\mathcal{L}_{\mathrm{hadronic}}=\prod_i \mathrm{Pois}(n^i |\, b^i + s^i)
\label{eq:hadronicLikelihood}
\end{equation}

The control regions are not diced in the \mht dimension, and their information is only used to constrain the normalisation of the background processes 
in the signal region, as described in section \ref{sec:background}. 
Their likelihood is therefore written as:

\begin{equation}
\mathcal{L}_{\mathrm{control}}=\mathrm{Pois}(n |\, b + s)
\label{eq:controlLikelihood}
\end{equation}

Equation \ref{eq:controlLikelihood} applies to any of the control region, 
as they are defined in \ref{sec:selection}, namely: $e$ + jets, $ee$ + jets, $\mu$ + jets, $\mu\mu$ + jets and $\gamma$ + jets. \\
Notice that the signal contribution $s$ in equation \ref{eq:controlLikelihood}, as estimated from MC, is in general negligible. 
Where the signal contamination is sizeable, it is included in the likelihood as an additional process contributing the event yield in that particular bin. \\

The prediction of the background yields in the signal region and in the corresponding control samples are connected 
by means of transfer factors, as explained in \ref{sec:background}. 
This connection is implemented by introducing a log-uniform nuisance parameter, which is correlated 
between the signal region and the control region, separately for each background process ($Z_{\mathrm{inv.}}$, $ttW$ \footnote{In the following we will use $Z_{\mathrm{inv}}$ to indicate the $Z\to \mathrm{inv}$ process and $ttW$ to indicate the sum of the yields of the $t\bar{t}$ and $W+\mathrm{jets}$ processes.}). \\
This way the background yields in the signal and control region are bound to float together, 
taking into account this way the statistical uncertainty associated with the counts in the control samples, which is then propagated to the 
prediction of the background in the signal region. \\

The systematic uncertainties affecting the transfer factors, described in \ref{subsec:tfunc}, are incorporated in the likelihood by means of log-normal nuisance parameters, which are taken as uncorrelated between each of the \htcat bin. \\
The shape uncertainties for the \mht templates are implemented providing alternative templates corresponding to up/down variation of each source, separately for each \htcat bin.\\

The total likelihood can be written as:
\begin{equation}
\label{eq:total_likelihood}
\mathcal{L} = \mathcal{L}_{\text{hadronic}} \times \mathcal{L}_{\text{syst.}} \times \sum _{\text{control}} \mathcal{L}_{\text{control}}
\end{equation}
The likelihood is profiled with respect all the nuisance parameters in order to derive exclusion limits, as the ones shown in sections \ref{sec:susy} and \ref{sec:darkmatter}. 




\newpage

\subsection{Results}

%Tables~\ref{tab:t1bbbb_1500_100} and~\ref{tab:t1bbbb_1000_900} show
Table~\ref{tab:results} shows
the expected performance for the model \verb!T1bbbb! with two mass
scenarios, ($m_{\rm gluino} = 1500,m_{\rm LSP} = 100\gev$) and
($m_{\rm gluino} = 1000,m_{\rm LSP} = 900\gev$), and an integrated
luminosity of 4\fbinv. The tables list both significances and R-values
for the \scalht and \alphat variables, along with a range of
alternative variables, including \mht, \met, effective mass (\meff),
and finally the projection of the \mht vector onto the plane
transverse to the event thrust axis (\mhttt). The two most sensitive
bins in each variable are used, and each result is based on a
combination of bins from four events categories: (\njet,\nb) = (4,2),
(4,$\geq$3), ($\geq$5,2), and ($\geq$5,$\geq$3). All signal region
selection criteria are applied, including $\scalht > 900\gev$ and
$\mht > 130\gev$. The background systematic uncertainty is assumed to
be 30\% for this selection and a systematic uncertainty on signal
acceptance of 10\% is assumed. The results were produced with the
Higgs combination tool and rely on the use of \cls and
pseudo-experiments.

The results with the \scalht and \alphat requirements listed in 
%Tables~\ref{tab:t1bbbb_1500_100} and~\ref{tab:t1bbbb_1000_900} represent
Table~\ref{tab:results} represents 
the baseline performance with no optimisation beyond a minimal choice
in categories and binning. Further optimisations are possible, such as
increasing the number of categories and bins used in the statistical
interpretation. Some example alternatives to \scalht and \alphat are
also shown, which increase the significance to as much as
$\sim4\sigma$. Investigations concerning the choice of an additional
discriminating variable(s) are ongoing.

It should be noted that, while encouraging, such optimisations are
sensitive to the assumed level of control when extrapolating into the
tails of kinematic distributions, particularly for this class of
models (involving very heavy objects). Ultimately, these assumptions
are limited by event counts in data control samples, which are the
only way to validate the use of transfer factors (as done in this
analysis), or the accuracy of MC modelling of kinematic shapes, or the
assumptions implicit in the parameterisation of shapes in data or MC,
etc. The reliance on data-driven methods, cross-checks, and control
variables is of paramount important as we probe a new mass regime in
Run~2.

%\begin{table}
%  \centering
%  \caption{Significances and R-values for \texttt{T1bbbb} ($m_{\rm
%      gluino} = 1500, m_{\rm LSP} = 100\gev$).}
%  \label{tab:t1bbbb_1500_100}
%  \footnotesize
%  \begin{tabular}{lccccccccccc}
%    \hline
%    \hline
%    Variable & Binning                & Significance & R-value ($\pm1\sigma$ exptal.) \\
%    \hline
%    \scalht  & 1400--1700, $\geq$1700 & 1.9$\sigma$  & $1.08^{+0.69}_{-0.42}$         \\
%    \alphat  & 0.55--0.65, $\geq$0.65 & 2.1$\sigma$  & $1.04^{+0.74}_{-0.43}$         \\
%    \mhttt   & 130--200, $\geq$200    & 2.9$\sigma$  & $0.74^{+0.53}_{-0.27}$         \\
%    \meff    & 1750--2500, $\geq$2500 & 3.2$\sigma$  & $0.64^{+0.40}_{-0.27}$         \\
%    \mht     & 500--700, $\geq$700    & 4.1$\sigma$  & $0.45^{+0.30}_{-0.17}$         \\
%    \met     & 500--700, $\geq$700    & 4.2$\sigma$  & $0.43^{+0.28}_{-0.15}$         \\
%    \hline
%    \hline
%  \end{tabular} 
%\end{table}

%\begin{table}
%  \centering
%  \caption{Significances and R-values for \texttt{T1bbbb} ($m_{\rm
%      gluino} = 1000, m_{\rm LSP} = 900\gev$).}
%  \label{tab:t1bbbb_1000_900}
%  \footnotesize
%  \begin{tabular}{lccccccccccc}
%    \hline
%    \hline
%    Variable & Binning                & Significance & R-value    ($\pm1\sigma$ exptal.) \\
%    \hline
%    \scalht  & 1400--1700, $\geq$1700 & 0.2$\sigma$  & $8.93^{+6.26}_{-3.87}$            \\
%    \alphat  & 0.55--0.65, $\geq$0.65 & 2.2$\sigma$  & $0.93^{+0.63}_{-0.40}$            \\
%    \mhttt   & 130--200, $\geq$200    & 1.0$\sigma$  & $2.38^{+1.60}_{-0.98}$           \\
%    \meff    & 1750--2500, $\geq$2500 & 0.7$\sigma$  & $2.90^{+1.77}_{-1.24}$            \\
%    \mht     & 500--700, $\geq$700    & 3.1$\sigma$  & $0.62^{+0.38}_{-0.24}$            \\
%    \met     & 500--700, $\geq$700    & 3.2$\sigma$  & $0.60^{+0.36}_{-0.24}$            \\
%    \hline
%    \hline
%  \end{tabular} 
%\end{table}

\begin{table}
  \centering
  \caption{Significances and R-values for various models.}
  \label{tab:results}
  \footnotesize
  \begin{tabular}{llcccc}
    \hline
    \hline
    &  & \multicolumn{2}{c}{\texttt{T1bbbb} (1500,100)} & \multicolumn{2}{c}{\texttt{T1bbbb} (1000,900)} \\
    Variable & Binning                & Significance & R-value ($\pm1\sigma_{\rm exptal}$) & Significance & R-value ($\pm1\sigma_{\rm exptal}$) \\
    \hline                                                                                                                              
    \scalht  & 1400--1700, $\geq$1700 & 1.9$\sigma$  & $1.08^{+0.69}_{-0.42}$              & 0.2$\sigma$  & $8.93^{+6.26}_{-3.87}$              \\
    \alphat  & 0.55--0.65, $\geq$0.65 & 2.1$\sigma$  & $1.04^{+0.74}_{-0.43}$              & 2.2$\sigma$  & $0.93^{+0.63}_{-0.40}$              \\
    \mhttt   & 130--200, $\geq$200    & 2.9$\sigma$  & $0.74^{+0.53}_{-0.27}$              & 1.0$\sigma$  & $2.38^{+1.60}_{-0.98}$              \\
    \meff    & 1750--2500, $\geq$2500 & 3.2$\sigma$  & $0.64^{+0.40}_{-0.27}$              & 0.7$\sigma$  & $2.90^{+1.77}_{-1.24}$              \\
    \mht     & 500--700, $\geq$700    & 4.1$\sigma$  & $0.45^{+0.30}_{-0.17}$              & 3.1$\sigma$  & $0.62^{+0.38}_{-0.24}$              \\
    \met     & 500--700, $\geq$700    & 4.2$\sigma$  & $0.43^{+0.28}_{-0.15}$              & 3.2$\sigma$  & $0.60^{+0.36}_{-0.24}$              \\
    \hline
    \hline
  \end{tabular} 
\end{table}

%Comment to let me make a commit called first draft



%%____________________________________________________________________________||
