%%____________________________________________________________________________||
\section{Likelihood model}
\label{sec:likelihood}

Consider a given category of event as defined by \njet, \nb and \HT, which will be in the following identified with \htcat. 
In each category, the signal is extracted using the \mht shape as a discriminating variable. 
Histogram templates of the \mht distribution are built for the signal and the background processes 
using the MC samples described in section \ref{sec:datasets}. 
The binning of the templates is chosen taking into account both the limited statistics of the MC samples and 
the control of the background. To fulfill both these objectives, each bin should have, at the same time:
\begin{itemize}
\item at least 10 unweighted MC counts
\item at least 10 events in the corresponding bin of each of the control samples relevant for the background prediction (inclusively on the b-tag requirement)
\end{itemize}

The first requirement ensures that the MC statistical uncertainty is negligible with respect to the other sources of systematic uncertainties. \\
The second requirement allows the perform a statistically meaningful closure test in data up to the highest \mht values. \\
Furthermore, a minimum bin width constraint of 50 GeV is applied, in order to have negligible bin-by-bin migration. \\ 

For each \mht bin $i$, let $n^{i}$ be the number of observed events, $b^{i}$ the number of predicted 
background events and $s^{i}$ the expected number of signal events. \\
The likelihood function for the hadronic signal region is:

\begin{equation}
\mathcal{L}_{\mathrm{hadronic}}=\prod_i \mathrm{Pois}(n^i |\, b^i + s^i)
\label{eq:hadronicLikelihood}
\end{equation}

The control regions are not diced in the \mht dimension, and their information is only used to constrain the normalisation of the background processes 
in the signal region, as described in section \ref{sec:background}. 
Their likelihood is therefore written as:

\begin{equation}
\mathcal{L}_{\mathrm{control}}=\mathrm{Pois}(n |\, b + s)
\label{eq:controlLikelihood}
\end{equation}

Equation \ref{eq:controlLikelihood} applies to any of the control region, 
as they are defined in \ref{sec:selection}, namely: $e$ + jets, $ee$ + jets, $\mu$ + jets, $\mu\mu$ + jets and $\gamma$ + jets. \\
Notice that the signal contribution $s$ in equation \ref{eq:controlLikelihood}, as estimated from MC, is in general negligible. 
Where the signal contamination is sizeable, it is included in the likelihood as an additional process contributing the event yield in that particular bin. \\

The prediction of the background yields in the signal region and in the corresponding control samples are connected 
by means of transfer factors, as explained in \ref{sec:background}. 
This connection is implemented by introducing a log-uniform nuisance parameter, which is correlated 
between the signal region and the control region, separately for each background process ($Z_{\mathrm{inv.}}$, $ttW$ \footnote{In the following we will use $Z_{\mathrm{inv}}$ to indicate the $Z\to \mathrm{inv}$ process and $ttW$ to indicate the sum of the yields of the $t\bar{t}$ and $W+\mathrm{jets}$ processes.}). \\
This way the background yields in the signal and control region are bound to float together, 
taking into account this way the statistical uncertainty associated with the counts in the control samples, which is then propagated to the 
prediction of the background in the signal region. \\

The systematic uncertainties affecting the transfer factors, described in \ref{subsec:tfunc}, are incorporated in the likelihood by means of log-normal nuisance parameters, which are taken as uncorrelated between each of the \htcat bin. \\
The unsertainty on the signal efficiency times acceptance, described in section \ref{subsec:susy_results}, is taken as correlated between all the \htcat bins. \\
The shape uncertainties for the \mht templates are implemented providing alternative templates corresponding to up/down variation of each source, separately for each \htcat bin. \\

The total likelihood can be written as:
\begin{equation}
\label{eq:total_likelihood}
\mathcal{L} = \mathcal{L}_{\text{hadronic}} \times \mathcal{L}_{\text{syst.}} \times \sum _{\text{control}} \mathcal{L}_{\text{control}}
\end{equation}
The likelihood is profiled with respect all the nuisance parameters in order to derive exclusion limits, 
as the ones shown in sections \ref{sec:susy} and \ref{sec:darkmatter}. 

%Comment to let me make a commit called first draft



%%____________________________________________________________________________||
