\clearpage
\section{W Polarisation Closure Test\label{app:wpol}}

As the LHC is a pp collider, high $p_T$ W bosons are predominantly
produced left handed \cite{WPol}.  For high $p_T$ bosons, this implies
that $W^+$ will decay to the left handed neutrino along its direction
of motion while the lepton will be backward. The opposite behaviour is
expected for the $W^-$. The lepton will therefore be more boosted (and
the neutrino less boosted) in $W^+$ decays than $W^-$ decays.  This
leads to a larger number of $W^+$ decays in the single lepton control
regions (which relies on the lepton $p_T$ for acceptance) than in the
signal region (which relies on the neutrino $p_T$ for acceptance). In
order to understand if this leads to a bias in the prediction of the
$W$ and unpolarised \zInv\ background in the signal region when using
a single lepton control region a closure test from $\mu^+$ + jets to
$\mu^-$ + jets will be added.  The \mj to \mmj closure tests will also
be important for probing the effect in the \zInv\ prediction. The
results from these closure tests using $8\TeV$ data can be seen in
Figure~\ref{fig:wpolCT}.  No significant bias or trend is observed.

\begin{figure}[h!]
 \begin{center}  
  \subfigure[Closure tests for $\le3$ jets]{\includegraphics[width=0.5\textwidth]{figures/wPol/summaryLe3J.pdf}} ~~
  \subfigure[Closure tests for $\ge4$ jets]{\includegraphics[width=0.5\textwidth]{figures/wPol/summaryGe4J.pdf}}
  \caption{Closure tests that probe the effects of W polarisation on
    experimental acceptance. The red circles correspond to the \mj to
    \mmj closure test while the blue crosses correspond to the
    $\mu^+$ + jets to $\mu^-$ + jets closure test.}
  \label{fig:wpolCT}
 \end{center}
\end{figure}          
