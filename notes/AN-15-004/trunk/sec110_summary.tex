%%____________________________________________________________________________||
\section{Summary}
\label{sec:summary}

We report on the prospects for a missing energy plus jet search that
is sensitive to models of Supersymmetry and Dark Matter. 
%The analysis follows an inclusive approach designed to capture a wide
%scope of possible final states, use of robust methods to be
%insensitive to multijet production, instrumental effects and MC
%mis-modelling. %Thus making it ideal for early data discoveries.
The Run I analysis has been been futher developed in several areas,
including an increase in the acceptance of the signal region and an
optimisation of the signal event categorisation. One prominant example
is the relaxed requirement on the second leading jet that leads to a
significantly improved acceptance to compressed SUSY and Dark Matter
models.
%asymetric jet selection. This selection uses an orthogonal requirement
%on the sub-leading jet to improve acceptance for ISR and monojet-type
%final states. This increases the acceptance of an example compressed
%SUSY model by about a factor of three and an example DM model by as
%much as a factor of five.
All signal selections region are binned according to the number of
reconstructed jets, the scalar sum of the transverse energy of jets,
and the number of jets identified to originate from bottom
quarks. Further discriminating variables are under investigation. The
sum of standard model backgrounds per bin is estimated from a
simultaneous binned likelihood fit to event yields in the signal
region and $\mu$ + jets, $\mu\mu$ + jets, and $\gamma$ + jets control
samples.

The addition of simplified and heavy quark flavored DM models yields
to strong results on DM models in the spin-dependent and
spin-independent nucleon cross section scattering plane in a largely
model-independent way. These results may provide the strongest
constraints with early 13~TeV data for low mass Dark Matter and the
strongest collider constraints across a wide range of masses. In
particular probe excesses observed in direct detection experiments at
energies of about 10 GeV but also by the Fermi-LAT (2013) satellite
indicating a DM particle of about 50 GeV mass. We expect to
conclusively probe this region with the full Run II dataset.

Preliminary results have been shown for a SUSY simplified model
comprising pair-produced 1500\gev gluinos, each decaying to a pair of
b-quarks and the LSP (of 100\gev). The expected sensitivity with
4\fbinv is as large as $\sim4\sigma$. However, a general statement can
be made that this particular interpretation is particularly sensitive
to the assumptions placed on the level of control when extrapolating
into the tails of any kinematic distribution.

Finally, it is worth repeating the concluding statement of the
previous Section: the reliance on data-driven methods, cross-checks,
and control variables is of paramount important as we probe a new mass
regime in Run~2.

%The analysis also has been adapted to latest standards of the CMS
%collaboration. All analysis tools have been ported and verified using
%the \textsc{PHYS14} prescription in {\tt cmgtools}, the use of calojet
%off- and online has been replaced by particle flow (PF) jet and the
%trigger requirements have been adapted to maintain 2012 thresholds.
%Further plans for this analysis will be to improve/increase the
%analysis binning for large $H_\textrm{T}$, improve and maintain
%trigger thresholds using PF trigger paths and evolve the selection
%according to the running conditions with increasing luminosity.

%\begin{itemize}
%  \item summarize the PHYS14 exercise described in the other sections
%  \item mention briefly further preparation plans for Run 2
%  \item conclude with the outlook for Run 2
%\end{itemize}

%%____________________________________________________________________________||
