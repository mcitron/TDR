%%____________________________________________________________________________||
\section{Characterisation of the signal and control regions}
\label{sec:yields}

%%____________________________________________________________________________||
% \subsection{Key distributions for the hadronic signal
%   region\label{sec:mc-data-comp}}
%
% %Distributions of key analysis variables 
% The hadronic signal region selection is detailed in Sec.~\ref{sec:hadSelection}.

%%____________________________________________________________________________||
\subsection{Breakdown of SM backgrounds in the hadronic signal
  region\label{sec:bkgd-comp}}

In the absence of multijet events from QCD, the remaining significant
backgrounds in the signal region are expected to stem from SM
processes with genuine \met in the final state. For the low jet
multiplicity categories, the largest backgrounds with genuine \met are
generally from the associated production of W or Z bosons with jets,
followed by either the weak decays \znunu\ or \wtaunu, where the
$\tau$ decays hadronically and is identified as a jet, or by leptonic
decays that are outside acceptance or not rejected by the dedicated
electron or muon vetoes. For the higher jet multiplicity categories,
top quark production followed by semileptonic weak top quark decay
becomes important. The relative contribution from \ttbar is enhanced
or suppressed depending on the number of b-jets required. 
% A breakdown
% of the relative contributions of the SM backgrounds, as given by
% simulation, in the different (\njet, \nb, \scalht) bins can be found
% in Table~\ref{tab:backgrounds}. 
Plots showing the yields for these electroweak backgrounds can be seen in Table~\ref{tab:ewk-bkgd}.
A breakdown of the three
dominant channels, \ttbar, \zInv~ and W~+~jets, are shown in Tables \ref{tab:tt-bkgd}, 
\ref{tab:zinv-bkgd} and \ref{tab:wjet-bkgd} respectively for 3\ifb. The contribution from
other sources, such as the single top and diboson channels, was found to be
negligible so are not shown.

%\begin{landscape}

\newpage
\begin{table}[h!]
\tiny
\centering
\topcaption{Yields for the Electroweak process in the signal region for 1.28\ifb. The letter ``a'' in jet \eg ``2a''  indicates the asymmetric jet bins.\label{tab:yields_ewk_sig}}
\begin{tabular}
{c|c|cccccccc}
	\hline\hline
   &     & \multicolumn{8}{c}{\scalht (\gev)} \\ 
	\njet & \nb & 200-250 & 250-300 & 300-350 & 350-400 & 400-500 & 500-600 & 600-800 & 800-$\infty$ \\ 
\hline
	1 & 0 & 3110.89 $\pm$17.17 & 1040.39 $\pm$8.55 & 416.02 $\pm$5.13 & 176.45 $\pm$2.95 & 131.07 $\pm$2.15 & 38.69 $\pm$1.02 & 17.18 $\pm$0.51 & 4.00 $\pm$0.34 \\ 
	1 & 1 & 107.85 $\pm$2.83 & 38.52 $\pm$1.51 & 17.61 $\pm$1.09 & 7.24 $\pm$0.57 & 6.19 $\pm$0.51 & 1.58 $\pm$0.33 & 0.81 $\pm$0.29 & 0.13 $\pm$0.28 \\ 
	2 & 0 & 383.58 $\pm$5.84 & 413.85 $\pm$5.69 & 278.55 $\pm$4.50 & 169.87 $\pm$3.13 & 155.22 $\pm$2.43 & 53.99 $\pm$1.22 & 26.75 $\pm$0.59 & 36.35 $\pm$0.62 \\ 
	2 & 1 & 42.62 $\pm$2.02 & 39.01 $\pm$1.80 & 26.60 $\pm$1.39 & 15.26 $\pm$0.97 & 12.76 $\pm$0.75 & 4.88 $\pm$0.47 & 2.80 $\pm$0.33 & 3.54 $\pm$0.33 \\ 
	2 & 2 & 2.27 $\pm$0.46 & 2.43 $\pm$0.43 & 1.65 $\pm$0.38 & 1.15 $\pm$0.35 & 0.81 $\pm$0.32 & 0.40 $\pm$0.30 & 0.13 $\pm$0.28 & 0.08 $\pm$0.28 \\ 
	3 & 0 & 0.61 $\pm$0.32 & 75.05 $\pm$2.47 & 205.23 $\pm$4.01 & 208.05 $\pm$3.72 & 250.08 $\pm$3.32 & 92.84 $\pm$1.66 & 53.22 $\pm$0.85 & 51.48 $\pm$0.74 \\ 
	3 & 1 & 0.23 $\pm$0.30 & 19.07 $\pm$1.44 & 43.60 $\pm$2.04 & 44.25 $\pm$2.04 & 43.28 $\pm$1.71 & 13.18 $\pm$0.77 & 7.32 $\pm$0.44 & 7.25 $\pm$0.41 \\ 
	3 & 2 & 0.00 $\pm$0.28 & 3.51 $\pm$0.66 & 8.33 $\pm$0.92 & 9.95 $\pm$1.02 & 7.02 $\pm$0.82 & 1.69 $\pm$0.36 & 0.84 $\pm$0.31 & 0.57 $\pm$0.29 \\ 
	3 & $\ge3$ & 0.00 $\pm$0.28 & 0.00 $\pm$0.28 & 0.05 $\pm$0.28 & 0.22 $\pm$0.29 & 0.13 $\pm$0.29 & 0.13 $\pm$0.29 & 0.03 $\pm$0.28 & 0.00 $\pm$0.28 \\ 
	4 & 0 & - & 0.56 $\pm$0.32 & 32.73 $\pm$1.73 & 92.10 $\pm$2.63 & 173.61 $\pm$3.03 & 83.93 $\pm$1.69 & 54.45 $\pm$0.94 & 43.71 $\pm$0.69 \\ 
	4 & 1 & - & 0.09 $\pm$0.29 & 12.43 $\pm$1.15 & 33.37 $\pm$1.84 & 58.68 $\pm$2.43 & 20.92 $\pm$1.22 & 12.13 $\pm$0.77 & 8.44 $\pm$0.50 \\ 
	4 & 2 & - & 0.00 $\pm$0.28 & 5.48 $\pm$0.85 & 12.64 $\pm$1.20 & 21.72 $\pm$1.51 & 6.11 $\pm$0.75 & 3.08 $\pm$0.59 & 1.68 $\pm$0.36 \\ 
	4 & $\ge3$ & - & 0.00 $\pm$0.28 & 0.09 $\pm$0.29 & 0.69 $\pm$0.34 & 1.04 $\pm$0.37 & 0.57 $\pm$0.32 & 0.25 $\pm$0.30 & 0.04 $\pm$0.28 \\ 
	$\ge5$ & 0 & - & - & 0.00 $\pm$0.28 & 4.64 $\pm$0.65 & 49.07 $\pm$1.84 & 46.56 $\pm$1.54 & 43.24 $\pm$1.14 & 39.29 $\pm$0.87 \\ 
	$\ge5$ & 1 & - & - & 0.00 $\pm$0.28 & 3.14 $\pm$0.62 & 31.81 $\pm$1.84 & 26.51 $\pm$1.62 & 19.37 $\pm$1.26 & 15.32 $\pm$1.02 \\ 
	$\ge5$ & 2 & - & - & 0.00 $\pm$0.28 & 1.42 $\pm$0.48 & 13.18 $\pm$1.22 & 10.97 $\pm$1.11 & 8.94 $\pm$0.98 & 5.35 $\pm$0.73 \\ 
	$\ge5$ & $\ge3$ & - & - & 0.00 $\pm$0.28 & 0.40 $\pm$0.32 & 1.65 $\pm$0.51 & 1.41 $\pm$0.47 & 0.61 $\pm$0.33 & 0.64 $\pm$0.33 \\ 
	2a & 0 & 1972.45 $\pm$13.73 & 578.07 $\pm$6.52 & 225.64 $\pm$3.97 & 93.68 $\pm$2.27 & 61.58 $\pm$1.52 & 15.82 $\pm$0.69 & 6.58 $\pm$0.38 & 6.01 $\pm$0.36 \\ 
	2a & 1 & 188.34 $\pm$4.18 & 51.77 $\pm$2.05 & 17.50 $\pm$1.10 & 7.67 $\pm$0.64 & 5.43 $\pm$0.52 & 1.18 $\pm$0.32 & 0.56 $\pm$0.29 & 0.45 $\pm$0.29 \\ 
	2a & 2 & 13.10 $\pm$1.06 & 2.57 $\pm$0.42 & 1.38 $\pm$0.37 & 0.63 $\pm$0.31 & 0.19 $\pm$0.29 & 0.03 $\pm$0.28 & 0.03 $\pm$0.28 & 0.01 $\pm$0.28 \\ 
	3a & 0 & 537.30 $\pm$7.10 & 537.23 $\pm$6.84 & 273.12 $\pm$4.68 & 90.73 $\pm$2.34 & 43.30 $\pm$1.37 & 8.57 $\pm$0.58 & 2.94 $\pm$0.33 & 2.61 $\pm$0.32 \\ 
	3a & 1 & 132.98 $\pm$3.79 & 131.21 $\pm$3.69 & 58.52 $\pm$2.39 & 18.55 $\pm$1.30 & 7.03 $\pm$0.72 & 0.82 $\pm$0.31 & 0.52 $\pm$0.30 & 0.31 $\pm$0.29 \\ 
	3a & 2 & 21.18 $\pm$1.50 & 24.01 $\pm$1.60 & 15.27 $\pm$1.31 & 3.09 $\pm$0.61 & 0.82 $\pm$0.32 & 0.17 $\pm$0.29 & 0.02 $\pm$0.28 & 0.01 $\pm$0.28 \\ 
	3a & $\ge3$ & 0.52 $\pm$0.33 & 0.42 $\pm$0.31 & 0.25 $\pm$0.30 & 0.00 $\pm$0.28 & 0.03 $\pm$0.28 & 0.00 $\pm$0.28 & 0.00 $\pm$0.28 & 0.00 $\pm$0.28 \\ 
	4a & 0 & - & 62.80 $\pm$2.38 & 157.01 $\pm$3.65 & 102.67 $\pm$2.80 & 57.37 $\pm$1.68 & 8.31 $\pm$0.60 & 1.59 $\pm$0.31 & 0.91 $\pm$0.30 \\ 
	4a & 1 & - & 24.06 $\pm$1.60 & 71.28 $\pm$2.90 & 41.26 $\pm$2.08 & 19.82 $\pm$1.37 & 1.45 $\pm$0.38 & 0.21 $\pm$0.29 & 0.17 $\pm$0.28 \\ 
	4a & 2 & - & 6.43 $\pm$0.86 & 22.81 $\pm$1.58 & 15.02 $\pm$1.33 & 6.44 $\pm$0.88 & 0.35 $\pm$0.30 & 0.01 $\pm$0.28 & 0.02 $\pm$0.28 \\ 
	4a & $\ge3$ & - & 0.14 $\pm$0.29 & 1.99 $\pm$0.55 & 0.96 $\pm$0.41 & 0.30 $\pm$0.31 & 0.08 $\pm$0.29 & 0.00 $\pm$0.28 & 0.00 $\pm$0.28 \\ 
	$\ge5$a & 0 & - & - & 15.24 $\pm$1.21 & 35.90 $\pm$1.76 & 47.83 $\pm$1.92 & 8.36 $\pm$0.68 & 2.29 $\pm$0.37 & 0.38 $\pm$0.29 \\ 
	$\ge5$a & 1 & - & - & 8.30 $\pm$0.91 & 27.49 $\pm$1.75 & 36.77 $\pm$2.03 & 5.62 $\pm$0.80 & 0.70 $\pm$0.33 & 0.06 $\pm$0.28 \\ 
	$\ge5$a & 2 & - & - & 5.54 $\pm$0.84 & 11.56 $\pm$1.17 & 19.24 $\pm$1.52 & 3.26 $\pm$0.64 & 0.36 $\pm$0.31 & 0.02 $\pm$0.28 \\ 
	$\ge5$a & $\ge3$ & - & - & 0.53 $\pm$0.34 & 1.77 $\pm$0.52 & 1.42 $\pm$0.47 & 0.90 $\pm$0.37 & 0.22 $\pm$0.30 & 0.00 $\pm$0.28 \\ 
	\hline
	\hline
\end{tabular}
\end{table}

\newpage
\begin{table}[h]
  \scriptsize
  \centering
  \topcaption{\ttbar background yields for each bin in the signal region for 3\ifb.
    The letter ``a'' in \njet \eg ``2a''  indicates the asymmetric \njet bins.
    \label{tab:tt-bkgd_3fb}}
  \begin{tabular}
    {c|c|ccccccc}
    \hline\hline
          &     & \multicolumn{7}{c}{\scalht (\gev)} \\ 
    \njet & \nb & 200-250 & 250-300 & 300-350 & 350-400 & 400-600 & 600-800 & 800-$\infty$ \\  
    \hline
	2 & 0 & 43.69 $\pm$2.04 & 38.44 $\pm$1.91 & 14.69 $\pm$1.18 & 6.20 $\pm$0.77 & 4.29 $\pm$0.64 & 0.10 $\pm$0.13 & 0.76 $\pm$0.27 \\ 
	2 & 1 & 47.31 $\pm$2.12 & 38.06 $\pm$1.91 & 16.60 $\pm$1.26 & 6.77 $\pm$0.80 & 3.43 $\pm$0.57 & 0.57 $\pm$0.23 & 0.19 $\pm$0.15 \\ 
	2 & 2 & 3.53 $\pm$0.58 & 2.67 $\pm$0.50 & 0.67 $\pm$0.25 & 0.95 $\pm$0.30 & 0.48 $\pm$0.21 & 0.00 $\pm$0.11 & 0.10 $\pm$0.13 \\ 
	3 & 0 & 0.19 $\pm$0.15 & 21.94 $\pm$1.45 & 45.50 $\pm$2.08 & 36.34 $\pm$1.86 & 41.21 $\pm$1.98 & 1.62 $\pm$0.39 & 2.67 $\pm$0.50 \\ 
	3 & 1 & 0.10 $\pm$0.13 & 33.39 $\pm$1.78 & 71.25 $\pm$2.61 & 62.76 $\pm$2.45 & 66.58 $\pm$2.52 & 3.05 $\pm$0.54 & 3.82 $\pm$0.60 \\ 
	3 & 2 & 0.00 $\pm$0.11 & 6.87 $\pm$0.81 & 20.99 $\pm$1.41 & 24.32 $\pm$1.52 & 31.38 $\pm$1.73 & 2.19 $\pm$0.46 & 1.53 $\pm$0.38 \\ 
	3 & $\ge3$ & 0.00 $\pm$0.11 & 0.10 $\pm$0.13 & 1.05 $\pm$0.32 & 1.53 $\pm$0.38 & 2.48 $\pm$0.49 & 0.29 $\pm$0.17 & 0.00 $\pm$0.11 \\ 
	4 & 0 & - & 0.19 $\pm$0.15 & 13.83 $\pm$1.15 & 40.63 $\pm$1.97 & 85.85 $\pm$2.86 & 7.92 $\pm$0.87 & 7.73 $\pm$0.86 \\ 
	4 & 1 & - & 0.57 $\pm$0.23 & 29.28 $\pm$1.67 & 81.56 $\pm$2.79 & 205.94 $\pm$4.43 & 18.70 $\pm$1.34 & 17.84 $\pm$1.30 \\ 
	4 & 2 & - & 0.29 $\pm$0.17 & 14.78 $\pm$1.19 & 49.79 $\pm$2.18 & 117.80 $\pm$3.35 & 10.78 $\pm$1.01 & 9.92 $\pm$0.97 \\ 
	4 & $\ge3$ & - & 0.00 $\pm$0.11 & 0.67 $\pm$0.25 & 4.20 $\pm$0.63 & 10.11 $\pm$0.98 & 1.43 $\pm$0.37 & 0.86 $\pm$0.29 \\ 
	$\ge5$ & 0 & - & - & 0.00 $\pm$0.11 & 4.10 $\pm$0.63 & 80.12 $\pm$2.76 & 25.47 $\pm$1.56 & 29.28 $\pm$1.67 \\ 
	$\ge5$ & 1 & - & - & 0.48 $\pm$0.21 & 8.39 $\pm$0.89 & 195.07 $\pm$4.31 & 67.15 $\pm$2.53 & 81.27 $\pm$2.78 \\ 
	$\ge5$ & 2 & - & - & 0.10 $\pm$0.13 & 4.86 $\pm$0.68 & 142.41 $\pm$3.69 & 51.22 $\pm$2.21 & 65.24 $\pm$2.49 \\ 
	$\ge5$ & $\ge3$ & - & - & 0.00 $\pm$0.11 & 0.76 $\pm$0.27 & 22.89 $\pm$1.48 & 11.35 $\pm$1.04 & 15.07 $\pm$1.20 \\ 
	2a & 0 & 204.89 $\pm$4.42 & 42.73 $\pm$2.02 & 9.63 $\pm$0.96 & 2.86 $\pm$0.52 & 0.86 $\pm$0.29 & 0.00 $\pm$0.11 & 0.00 $\pm$0.11 \\ 
	2a & 1 & 177.04 $\pm$4.11 & 39.49 $\pm$1.94 & 9.44 $\pm$0.95 & 2.10 $\pm$0.45 & 1.05 $\pm$0.32 & 0.00 $\pm$0.11 & 0.00 $\pm$0.11 \\ 
	2a & 2 & 11.45 $\pm$1.04 & 2.38 $\pm$0.48 & 0.67 $\pm$0.25 & 0.29 $\pm$0.17 & 0.10 $\pm$0.13 & 0.00 $\pm$0.11 & 0.10 $\pm$0.13 \\ 
	3a & 0 & 153.00 $\pm$3.82 & 143.84 $\pm$3.70 & 59.24 $\pm$2.38 & 14.98 $\pm$1.20 & 3.53 $\pm$0.58 & 0.10 $\pm$0.13 & 0.19 $\pm$0.15 \\ 
	3a & 1 & 209.95 $\pm$4.48 & 214.72 $\pm$4.53 & 105.97 $\pm$3.18 & 28.52 $\pm$1.65 & 5.44 $\pm$0.72 & 0.00 $\pm$0.11 & 0.00 $\pm$0.11 \\ 
	3a & 2 & 52.75 $\pm$2.24 & 53.70 $\pm$2.26 & 33.29 $\pm$1.78 & 10.97 $\pm$1.02 & 1.72 $\pm$0.40 & 0.00 $\pm$0.11 & 0.10 $\pm$0.13 \\ 
	3a & $\ge3$ & 2.29 $\pm$0.47 & 2.10 $\pm$0.45 & 1.43 $\pm$0.37 & 0.76 $\pm$0.27 & 0.10 $\pm$0.13 & 0.00 $\pm$0.11 & 0.00 $\pm$0.11 \\ 
	4a & 0 & - & 27.19 $\pm$1.61 & 80.51 $\pm$2.77 & 46.83 $\pm$2.11 & 22.99 $\pm$1.48 & 0.00 $\pm$0.11 & 0.00 $\pm$0.11 \\ 
	4a & 1 & - & 45.98 $\pm$2.09 & 162.44 $\pm$3.94 & 102.35 $\pm$3.12 & 48.17 $\pm$2.14 & 0.19 $\pm$0.15 & 0.10 $\pm$0.13 \\ 
	4a & 2 & - & 19.84 $\pm$1.38 & 81.75 $\pm$2.79 & 54.18 $\pm$2.27 & 29.09 $\pm$1.67 & 0.00 $\pm$0.11 & 0.00 $\pm$0.11 \\ 
	4a & $\ge3$ & - & 1.72 $\pm$0.40 & 7.25 $\pm$0.83 & 5.34 $\pm$0.71 & 2.48 $\pm$0.49 & 0.10 $\pm$0.13 & 0.00 $\pm$0.11 \\ 
	$\ge5$a & 0 & - & - & 13.64 $\pm$1.14 & 37.39 $\pm$1.89 & 56.85 $\pm$2.33 & 1.81 $\pm$0.42 & 0.48 $\pm$0.21 \\ 
	$\ge5$a & 1 & - & - & 23.85 $\pm$1.51 & 82.51 $\pm$2.81 & 152.14 $\pm$3.81 & 3.34 $\pm$0.56 & 0.57 $\pm$0.23 \\ 
	$\ge5$a & 2 & - & - & 16.02 $\pm$1.24 & 55.32 $\pm$2.30 & 103.69 $\pm$3.14 & 3.72 $\pm$0.60 & 0.19 $\pm$0.15 \\ 
	$\ge5$a & $\ge3$ & - & - & 2.19 $\pm$0.46 & 8.30 $\pm$0.89 & 16.69 $\pm$1.26 & 0.76 $\pm$0.27 & 0.00 $\pm$0.11 \\ 
\hline\hline
  \end{tabular}
\end{table}


\newpage
\begin{table}[h]
  \scriptsize
  \centering
  \topcaption{W~+~jets background yields for each bin in the signal region for 3\ifb.
    The letter ``a'' in \njet \eg ``2a''  indicates the asymmetric \njet bins.
    \label{tab:wjet-bkgd}}
  \begin{tabular}
    {c|c|ccccccc}
    \hline\hline
          &     & \multicolumn{7}{c}{\scalht (\gev)} \\ 
    \njet & \nb & 200-250 & 250-300 & 300-350 & 350-400 & 400-600 & 600-800 & 800-$\infty$ \\  
    \hline
	2 & 0 & 595.75 $\pm$15.92 & 589.10 $\pm$14.63 & 366.81 $\pm$10.99 & 217.75 $\pm$7.09 & 230.29 $\pm$4.12 & 21.13 $\pm$1.64 & 76.44 $\pm$1.86 \\ 
	2 & 1 & 44.42 $\pm$4.26 & 45.00 $\pm$4.22 & 33.60 $\pm$3.54 & 18.66 $\pm$2.45 & 22.02 $\pm$1.87 & 2.36 $\pm$1.53 & 9.12 $\pm$1.57 \\ 
	2 & 2 & 1.06 $\pm$1.59 & 3.17 $\pm$1.81 & 1.28 $\pm$1.59 & 1.06 $\pm$1.57 & 0.71 $\pm$1.53 & 0.09 $\pm$1.52 & 0.41 $\pm$1.52 \\ 
	3 & 0 & 0.71 $\pm$1.56 & 108.45 $\pm$6.38 & 342.08 $\pm$10.71 & 333.22 $\pm$8.78 & 491.72 $\pm$6.25 & 53.34 $\pm$1.81 & 110.50 $\pm$2.00 \\ 
	3 & 1 & 0.00 $\pm$1.52 & 13.88 $\pm$2.65 & 35.18 $\pm$3.64 & 43.00 $\pm$3.37 & 65.68 $\pm$2.60 & 7.91 $\pm$1.57 & 20.96 $\pm$1.62 \\ 
	3 & 2 & 0.35 $\pm$1.54 & 0.75 $\pm$1.56 & 3.44 $\pm$1.81 & 4.15 $\pm$1.72 & 5.61 $\pm$1.61 & 0.57 $\pm$1.52 & 1.33 $\pm$1.53 \\ 
	3 & $\ge3$ & 0.00 $\pm$1.52 & 0.00 $\pm$1.52 & 0.00 $\pm$1.52 & 0.02 $\pm$1.52 & 0.15 $\pm$1.52 & 0.02 $\pm$1.52 & 0.05 $\pm$1.52 \\ 
	4 & 0 & - & 0.00 $\pm$1.52 & 53.82 $\pm$4.50 & 143.21 $\pm$6.25 & 432.82 $\pm$6.64 & 65.53 $\pm$1.90 & 102.96 $\pm$1.97 \\ 
	4 & 1 & - & 0.00 $\pm$1.52 & 8.86 $\pm$2.27 & 21.81 $\pm$2.69 & 69.19 $\pm$2.77 & 14.30 $\pm$1.61 & 22.42 $\pm$1.63 \\ 
	4 & 2 & - & 0.00 $\pm$1.52 & 2.02 $\pm$1.82 & 3.88 $\pm$1.75 & 9.65 $\pm$1.73 & 1.78 $\pm$1.53 & 3.14 $\pm$1.54 \\ 
	4 & $\ge3$ & - & 0.00 $\pm$1.52 & 0.00 $\pm$1.52 & 0.71 $\pm$1.56 & 0.37 $\pm$1.53 & 0.09 $\pm$1.52 & 0.18 $\pm$1.52 \\ 
	$\ge5$ & 0 & - & - & 0.00 $\pm$1.52 & 7.32 $\pm$2.05 & 162.75 $\pm$4.47 & 66.85 $\pm$2.04 & 104.10 $\pm$2.01 \\ 
	$\ge5$ & 1 & - & - & 0.00 $\pm$1.52 & 2.07 $\pm$1.63 & 36.28 $\pm$2.53 & 16.08 $\pm$1.64 & 31.02 $\pm$1.67 \\ 
	$\ge5$ & 2 & - & - & 0.00 $\pm$1.52 & 0.13 $\pm$1.52 & 6.66 $\pm$1.69 & 2.94 $\pm$1.54 & 6.03 $\pm$1.55 \\ 
	$\ge5$ & $\ge3$ & - & - & 0.00 $\pm$1.52 & 0.00 $\pm$1.52 & 0.75 $\pm$1.55 & 0.41 $\pm$1.52 & 0.83 $\pm$1.53 \\ 
	2a & 0 & 3185.31 $\pm$39.22 & 855.15 $\pm$17.48 & 298.73 $\pm$10.01 & 111.65 $\pm$5.38 & 75.96 $\pm$2.83 & 5.11 $\pm$1.55 & 5.06 $\pm$1.55 \\ 
	2a & 1 & 215.63 $\pm$10.03 & 58.57 $\pm$4.75 & 26.76 $\pm$3.27 & 8.08 $\pm$1.87 & 6.47 $\pm$1.61 & 0.62 $\pm$1.52 & 0.65 $\pm$1.52 \\ 
	2a & 2 & 12.08 $\pm$2.53 & 2.53 $\pm$1.73 & 1.23 $\pm$1.59 & 0.53 $\pm$1.54 & 0.24 $\pm$1.52 & 0.02 $\pm$1.52 & 0.02 $\pm$1.52 \\ 
	3a & 0 & 911.15 $\pm$21.08 & 860.76 $\pm$17.78 & 439.76 $\pm$12.45 & 136.00 $\pm$5.93 & 57.60 $\pm$2.89 & 2.25 $\pm$1.53 & 2.51 $\pm$1.53 \\ 
	3a & 1 & 95.81 $\pm$6.66 & 99.33 $\pm$6.14 & 48.06 $\pm$4.28 & 17.52 $\pm$2.46 & 6.64 $\pm$1.64 & 0.33 $\pm$1.52 & 0.44 $\pm$1.52 \\ 
	3a & 2 & 11.50 $\pm$2.57 & 10.44 $\pm$2.40 & 4.58 $\pm$1.91 & 2.12 $\pm$1.66 & 0.57 $\pm$1.53 & 0.00 $\pm$1.52 & 0.00 $\pm$1.52 \\ 
	3a & $\ge3$ & 0.00 $\pm$1.52 & 0.00 $\pm$1.52 & 0.00 $\pm$1.52 & 0.04 $\pm$1.52 & 0.00 $\pm$1.52 & 0.00 $\pm$1.52 & 0.00 $\pm$1.52 \\ 
	4a & 0 & - & 88.53 $\pm$5.80 & 273.90 $\pm$10.01 & 186.05 $\pm$7.31 & 108.32 $\pm$3.88 & 1.27 $\pm$1.53 & 1.17 $\pm$1.53 \\ 
	4a & 1 & - & 14.19 $\pm$2.67 & 42.49 $\pm$4.07 & 30.80 $\pm$3.24 & 17.80 $\pm$2.03 & 0.21 $\pm$1.52 & 0.17 $\pm$1.52 \\ 
	4a & 2 & - & 0.71 $\pm$1.56 & 5.42 $\pm$2.00 & 4.59 $\pm$1.82 & 2.21 $\pm$1.57 & 0.05 $\pm$1.52 & 0.03 $\pm$1.52 \\ 
	4a & $\ge3$ & - & 0.00 $\pm$1.52 & 0.04 $\pm$1.52 & 0.09 $\pm$1.52 & 0.15 $\pm$1.52 & 0.00 $\pm$1.52 & 0.00 $\pm$1.52 \\ 
	$\ge5$a & 0 & - & - & 23.40 $\pm$3.18 & 68.79 $\pm$4.89 & 101.78 $\pm$4.40 & 3.16 $\pm$1.55 & 0.42 $\pm$1.52 \\ 
	$\ge5$a & 1 & - & - & 4.01 $\pm$1.88 & 13.14 $\pm$2.47 & 20.63 $\pm$2.24 & 0.94 $\pm$1.53 & 0.15 $\pm$1.52 \\ 
	$\ge5$a & 2 & - & - & 0.71 $\pm$1.56 & 1.72 $\pm$1.63 & 3.43 $\pm$1.65 & 0.09 $\pm$1.52 & 0.03 $\pm$1.52 \\ 
	$\ge5$a & $\ge3$ & - & - & 0.00 $\pm$1.52 & 0.02 $\pm$1.52 & 0.71 $\pm$1.55 & 0.00 $\pm$1.52 & 0.00 $\pm$1.52 \\ 
	
\hline\hline
  \end{tabular}
\end{table}

\newpage
\begin{table}[h!]
\tiny
\centering
\topcaption{Yields for the \zInv~ process in the signal region for 1.28\ifb. The letter ``a'' in jet \eg ``2a''  indicates the asymmetric jet bins.\label{tab:yields_zinv_sig}}
\begin{tabular}
{c|c|cccccccc}
	\hline\hline
   &     & \multicolumn{8}{c}{\scalht (\gev)} \\ 
	\njet & \nb & 200-250 & 250-300 & 300-350 & 350-400 & 400-500 & 500-600 & 600-800 & 800-$\infty$ \\ 
\hline
	1 & 0 & 1937.37 $\pm$9.57 & 692.55 $\pm$4.80 & 294.28 $\pm$3.09 & 134.66 $\pm$1.99 & 100.64 $\pm$1.56 & 31.70 $\pm$0.81 & 14.55 $\pm$0.41 & 3.60 $\pm$0.20 \\ 
	1 & 1 & 77.25 $\pm$1.85 & 29.02 $\pm$0.99 & 12.94 $\pm$0.64 & 5.88 $\pm$0.42 & 4.76 $\pm$0.34 & 1.24 $\pm$0.17 & 0.70 $\pm$0.12 & 0.12 $\pm$0.10 \\ 
	2 & 0 & 206.72 $\pm$2.82 & 237.00 $\pm$2.79 & 166.14 $\pm$2.28 & 106.61 $\pm$1.73 & 102.13 $\pm$1.55 & 37.58 $\pm$0.87 & 19.27 $\pm$0.44 & 26.17 $\pm$0.48 \\ 
	2 & 1 & 14.75 $\pm$0.79 & 15.86 $\pm$0.71 & 13.48 $\pm$0.65 & 8.86 $\pm$0.50 & 7.92 $\pm$0.43 & 3.04 $\pm$0.25 & 2.05 $\pm$0.17 & 2.58 $\pm$0.18 \\ 
	2 & 2 & 1.30 $\pm$0.22 & 1.56 $\pm$0.24 & 0.94 $\pm$0.19 & 0.61 $\pm$0.16 & 0.48 $\pm$0.13 & 0.25 $\pm$0.12 & 0.12 $\pm$0.10 & 0.06 $\pm$0.10 \\ 
	3 & 0 & 0.22 $\pm$0.11 & 39.41 $\pm$1.12 & 108.64 $\pm$1.87 & 112.10 $\pm$1.76 & 144.59 $\pm$1.83 & 59.23 $\pm$1.08 & 36.96 $\pm$0.61 & 36.47 $\pm$0.57 \\ 
	3 & 1 & 0.03 $\pm$0.10 & 3.79 $\pm$0.36 & 11.11 $\pm$0.58 & 11.90 $\pm$0.57 & 17.29 $\pm$0.65 & 6.98 $\pm$0.36 & 4.46 $\pm$0.22 & 4.84 $\pm$0.23 \\ 
	3 & 2 & 0.00 $\pm$0.09 & 0.58 $\pm$0.16 & 1.47 $\pm$0.23 & 1.30 $\pm$0.20 & 1.69 $\pm$0.21 & 0.81 $\pm$0.15 & 0.47 $\pm$0.11 & 0.42 $\pm$0.11 \\ 
	3 & $\ge3$ & 0.00 $\pm$0.09 & 0.00 $\pm$0.09 & 0.04 $\pm$0.10 & 0.08 $\pm$0.10 & 0.01 $\pm$0.09 & 0.01 $\pm$0.09 & 0.03 $\pm$0.09 & 0.00 $\pm$0.09 \\ 
	4 & 0 & - & 0.14 $\pm$0.11 & 13.59 $\pm$0.68 & 42.41 $\pm$1.11 & 87.13 $\pm$1.45 & 50.01 $\pm$1.00 & 35.06 $\pm$0.61 & 29.97 $\pm$0.52 \\ 
	4 & 1 & - & 0.04 $\pm$0.10 & 2.01 $\pm$0.27 & 5.91 $\pm$0.41 & 12.76 $\pm$0.55 & 8.11 $\pm$0.41 & 5.94 $\pm$0.27 & 5.31 $\pm$0.23 \\ 
	4 & 2 & - & 0.00 $\pm$0.09 & 0.46 $\pm$0.16 & 0.73 $\pm$0.17 & 2.45 $\pm$0.26 & 1.29 $\pm$0.22 & 0.85 $\pm$0.13 & 0.75 $\pm$0.12 \\ 
	4 & $\ge3$ & - & 0.00 $\pm$0.09 & 0.00 $\pm$0.09 & 0.13 $\pm$0.10 & 0.16 $\pm$0.10 & 0.09 $\pm$0.10 & 0.02 $\pm$0.09 & 0.02 $\pm$0.09 \\ 
	$\ge5$ & 0 & - & - & 0.00 $\pm$0.09 & 1.72 $\pm$0.23 & 20.05 $\pm$0.71 & 22.29 $\pm$0.69 & 23.54 $\pm$0.55 & 24.23 $\pm$0.47 \\ 
	$\ge5$ & 1 & - & - & 0.00 $\pm$0.09 & 0.28 $\pm$0.11 & 3.81 $\pm$0.31 & 3.86 $\pm$0.30 & 5.07 $\pm$0.27 & 5.13 $\pm$0.23 \\ 
	$\ge5$ & 2 & - & - & 0.00 $\pm$0.09 & 0.06 $\pm$0.10 & 0.73 $\pm$0.16 & 0.91 $\pm$0.16 & 0.95 $\pm$0.15 & 1.05 $\pm$0.13 \\ 
	$\ge5$ & $\ge3$ & - & - & 0.00 $\pm$0.09 & 0.00 $\pm$0.09 & 0.04 $\pm$0.10 & 0.04 $\pm$0.09 & 0.16 $\pm$0.10 & 0.14 $\pm$0.10 \\ 
	2a & 0 & 1081.21 $\pm$6.97 & 350.11 $\pm$3.38 & 142.84 $\pm$2.12 & 63.38 $\pm$1.34 & 44.57 $\pm$1.03 & 11.62 $\pm$0.48 & 5.10 $\pm$0.24 & 5.08 $\pm$0.23 \\ 
	2a & 1 & 75.27 $\pm$1.73 & 24.41 $\pm$0.88 & 10.30 $\pm$0.57 & 5.24 $\pm$0.40 & 3.26 $\pm$0.28 & 0.98 $\pm$0.17 & 0.45 $\pm$0.11 & 0.36 $\pm$0.11 \\ 
	2a & 2 & 6.83 $\pm$0.52 & 1.80 $\pm$0.25 & 0.82 $\pm$0.18 & 0.23 $\pm$0.11 & 0.17 $\pm$0.11 & 0.03 $\pm$0.09 & 0.01 $\pm$0.09 & 0.01 $\pm$0.09 \\ 
	3a & 0 & 264.48 $\pm$3.29 & 272.80 $\pm$3.00 & 143.94 $\pm$2.14 & 54.88 $\pm$1.25 & 28.02 $\pm$0.82 & 5.98 $\pm$0.38 & 2.37 $\pm$0.18 & 2.14 $\pm$0.17 \\ 
	3a & 1 & 25.33 $\pm$0.98 & 27.33 $\pm$0.93 & 15.32 $\pm$0.69 & 6.06 $\pm$0.42 & 3.03 $\pm$0.28 & 0.42 $\pm$0.12 & 0.22 $\pm$0.10 & 0.29 $\pm$0.11 \\ 
	3a & 2 & 2.95 $\pm$0.32 & 3.74 $\pm$0.35 & 2.67 $\pm$0.31 & 0.50 $\pm$0.14 & 0.50 $\pm$0.14 & 0.10 $\pm$0.10 & 0.02 $\pm$0.09 & 0.01 $\pm$0.09 \\ 
	3a & $\ge3$ & 0.04 $\pm$0.10 & 0.14 $\pm$0.11 & 0.00 $\pm$0.09 & 0.00 $\pm$0.09 & 0.00 $\pm$0.09 & 0.00 $\pm$0.09 & 0.00 $\pm$0.09 & 0.00 $\pm$0.09 \\ 
	4a & 0 & - & 27.26 $\pm$0.95 & 72.45 $\pm$1.51 & 48.17 $\pm$1.19 & 32.67 $\pm$0.91 & 5.06 $\pm$0.33 & 1.22 $\pm$0.15 & 0.77 $\pm$0.12 \\ 
	4a & 1 & - & 4.35 $\pm$0.38 & 9.58 $\pm$0.55 & 6.84 $\pm$0.45 & 4.91 $\pm$0.35 & 0.62 $\pm$0.14 & 0.15 $\pm$0.10 & 0.11 $\pm$0.10 \\ 
	4a & 2 & - & 1.01 $\pm$0.21 & 1.70 $\pm$0.24 & 0.94 $\pm$0.18 & 0.78 $\pm$0.15 & 0.08 $\pm$0.10 & 0.00 $\pm$0.09 & 0.02 $\pm$0.09 \\ 
	4a & $\ge3$ & - & 0.05 $\pm$0.10 & 0.10 $\pm$0.10 & 0.05 $\pm$0.10 & 0.00 $\pm$0.09 & 0.00 $\pm$0.09 & 0.00 $\pm$0.09 & 0.00 $\pm$0.09 \\ 
	$\ge5$a & 0 & - & - & 5.67 $\pm$0.44 & 14.21 $\pm$0.66 & 17.77 $\pm$0.68 & 3.86 $\pm$0.29 & 1.31 $\pm$0.17 & 0.27 $\pm$0.11 \\ 
	$\ge5$a & 1 & - & - & 0.86 $\pm$0.18 & 2.43 $\pm$0.28 & 2.81 $\pm$0.27 & 0.76 $\pm$0.15 & 0.22 $\pm$0.10 & 0.03 $\pm$0.09 \\ 
	$\ge5$a & 2 & - & - & 0.16 $\pm$0.11 & 0.46 $\pm$0.14 & 0.83 $\pm$0.17 & 0.28 $\pm$0.12 & 0.03 $\pm$0.09 & 0.02 $\pm$0.09 \\ 
	$\ge5$a & $\ge3$ & - & - & 0.00 $\pm$0.09 & 0.00 $\pm$0.09 & 0.11 $\pm$0.10 & 0.06 $\pm$0.10 & 0.00 $\pm$0.09 & 0.00 $\pm$0.09 \\ 
	\hline
	\hline
\end{tabular}
\end{table}


%%____________________________________________________________________________||
\newpage
\subsection{Yields in the control samples}

The yields in the \mj, \mmj, \ej, \eej and \gj control samples can be seen in
Tables~\ref{tab:mj-bkgd}, \ref{tab:ej-bkgd}, \ref{tab:mmj-bkgd}, \ref{tab:eej-bkgd}
and \ref{tab:gj-bkgd} respectively or 3 \ifb. 
The number of events in each of these bins is important for
working out how far we can extend in our analysis bins. We require there to be
enough events in the control samples to allow robust data driven prediction of
the backgrounds.

\begin{table}[h!]
\tiny
\centering
\topcaption{Yields  in the \mj control region for 1.28\ifb. The letter ``a'' in jet \eg ``2a''  indicates the asymmetric jet bins.\label{tab:yields_ewk_mu}}
\begin{tabular}
{c|c|cccccccc}
	\hline\hline
   &     & \multicolumn{8}{c}{\scalht (\gev)} \\ 
	\njet & \nb & 200-250 & 250-300 & 300-350 & 350-400 & 400-500 & 500-600 & 600-800 & 800-$\infty$ \\ 
\hline
	1 & 0 & 1854.36 $\pm$17.36 & 679.51 $\pm$9.71 & 295.58 $\pm$6.38 & 141.43 $\pm$3.94 & 113.75 $\pm$2.79 & 36.96 $\pm$1.33 & 18.22 $\pm$0.56 & 4.04 $\pm$0.33 \\ 
	1 & 1 & 75.46 $\pm$3.39 & 25.54 $\pm$1.80 & 12.10 $\pm$1.24 & 6.00 $\pm$0.79 & 4.92 $\pm$0.60 & 1.38 $\pm$0.36 & 0.95 $\pm$0.30 & 0.18 $\pm$0.28 \\ 
	2 & 0 & 201.18 $\pm$5.36 & 293.99 $\pm$6.35 & 271.80 $\pm$5.97 & 244.18 $\pm$5.05 & 306.72 $\pm$4.39 & 147.99 $\pm$2.46 & 126.79 $\pm$1.31 & 64.94 $\pm$0.81 \\ 
	2 & 1 & 40.85 $\pm$2.22 & 44.20 $\pm$2.32 & 35.98 $\pm$2.09 & 28.70 $\pm$1.75 & 39.32 $\pm$1.83 & 18.51 $\pm$1.10 & 15.45 $\pm$0.86 & 8.02 $\pm$0.60 \\ 
	2 & 2 & 1.81 $\pm$0.48 & 3.54 $\pm$0.66 & 1.84 $\pm$0.48 & 2.75 $\pm$0.54 & 3.92 $\pm$0.71 & 1.72 $\pm$0.49 & 1.68 $\pm$0.45 & 0.48 $\pm$0.31 \\ 
	3 & 0 & 0.57 $\pm$0.33 & 90.37 $\pm$3.52 & 189.13 $\pm$4.91 & 216.36 $\pm$4.79 & 327.41 $\pm$4.82 & 191.98 $\pm$3.08 & 176.36 $\pm$1.89 & 104.36 $\pm$1.27 \\ 
	3 & 1 & 0.68 $\pm$0.34 & 47.38 $\pm$2.46 & 101.29 $\pm$3.54 & 98.15 $\pm$3.35 & 121.55 $\pm$3.52 & 62.61 $\pm$2.45 & 51.75 $\pm$2.04 & 25.28 $\pm$1.31 \\ 
	3 & 2 & 0.18 $\pm$0.29 & 10.90 $\pm$1.14 & 32.10 $\pm$1.98 & 35.32 $\pm$2.15 & 45.85 $\pm$2.32 & 19.21 $\pm$1.52 & 13.36 $\pm$1.20 & 4.81 $\pm$0.70 \\ 
	3 & $\ge3$ & 0.00 $\pm$0.28 & 1.25 $\pm$0.51 & 1.11 $\pm$0.44 & 0.54 $\pm$0.33 & 2.25 $\pm$0.61 & 0.94 $\pm$0.40 & 0.33 $\pm$0.30 & 0.27 $\pm$0.30 \\ 
	4 & 0 & - & 0.28 $\pm$0.30 & 25.98 $\pm$1.79 & 79.75 $\pm$3.00 & 199.36 $\pm$4.19 & 139.40 $\pm$2.94 & 149.81 $\pm$2.42 & 93.43 $\pm$1.42 \\ 
	4 & 1 & - & 0.36 $\pm$0.31 & 26.08 $\pm$1.78 & 72.73 $\pm$2.93 & 164.56 $\pm$4.35 & 99.19 $\pm$3.28 & 83.62 $\pm$2.93 & 40.36 $\pm$1.79 \\ 
	4 & 2 & - & 0.20 $\pm$0.29 & 10.82 $\pm$1.16 & 38.20 $\pm$2.16 & 84.62 $\pm$3.18 & 50.25 $\pm$2.43 & 35.95 $\pm$2.00 & 17.11 $\pm$1.37 \\ 
	4 & $\ge3$ & - & 0.00 $\pm$0.28 & 0.94 $\pm$0.37 & 2.71 $\pm$0.63 & 6.47 $\pm$0.90 & 4.82 $\pm$0.83 & 3.36 $\pm$0.69 & 1.63 $\pm$0.50 \\ 
	$\ge5$ & 0 & - & - & 0.23 $\pm$0.30 & 5.18 $\pm$0.79 & 55.66 $\pm$2.36 & 82.74 $\pm$2.70 & 117.77 $\pm$2.75 & 106.40 $\pm$2.21 \\ 
	$\ge5$ & 1 & - & - & 0.49 $\pm$0.33 & 7.09 $\pm$0.93 & 81.40 $\pm$3.18 & 112.34 $\pm$3.63 & 142.51 $\pm$3.98 & 106.25 $\pm$3.33 \\ 
	$\ge5$ & 2 & - & - & 0.18 $\pm$0.29 & 3.31 $\pm$0.65 & 47.88 $\pm$2.40 & 68.33 $\pm$2.81 & 94.40 $\pm$3.35 & 66.37 $\pm$2.79 \\ 
	$\ge5$ & $\ge3$ & - & - & 0.00 $\pm$0.28 & 0.60 $\pm$0.34 & 5.80 $\pm$0.86 & 8.74 $\pm$1.06 & 12.75 $\pm$1.25 & 10.58 $\pm$1.16 \\ 
	2a & 0 & 1965.92 $\pm$17.27 & 1023.73 $\pm$11.91 & 479.16 $\pm$7.92 & 208.56 $\pm$4.67 & 145.81 $\pm$3.11 & 39.84 $\pm$1.31 & 20.13 $\pm$0.61 & 4.03 $\pm$0.33 \\ 
	2a & 1 & 344.07 $\pm$6.66 & 141.22 $\pm$4.17 & 66.65 $\pm$2.85 & 24.84 $\pm$1.65 & 18.57 $\pm$1.31 & 4.85 $\pm$0.65 & 2.18 $\pm$0.38 & 0.48 $\pm$0.29 \\ 
	2a & 2 & 33.49 $\pm$2.01 & 15.80 $\pm$1.36 & 5.69 $\pm$0.78 & 2.49 $\pm$0.53 & 1.03 $\pm$0.37 & 0.41 $\pm$0.32 & 0.43 $\pm$0.31 & 0.18 $\pm$0.29 \\ 
	3a & 0 & 467.55 $\pm$8.27 & 620.76 $\pm$9.15 & 340.28 $\pm$6.70 & 170.95 $\pm$4.27 & 122.52 $\pm$2.89 & 31.20 $\pm$1.33 & 13.65 $\pm$0.62 & 2.98 $\pm$0.33 \\ 
	3a & 1 & 258.51 $\pm$5.63 & 333.58 $\pm$6.41 & 160.59 $\pm$4.40 & 61.95 $\pm$2.62 & 37.83 $\pm$1.92 & 8.46 $\pm$0.85 & 3.59 $\pm$0.57 & 0.30 $\pm$0.29 \\ 
	3a & 2 & 69.34 $\pm$2.86 & 98.89 $\pm$3.45 & 49.47 $\pm$2.40 & 17.97 $\pm$1.43 & 10.61 $\pm$1.11 & 2.45 $\pm$0.54 & 0.73 $\pm$0.33 & 0.09 $\pm$0.29 \\ 
	3a & $\ge3$ & 2.66 $\pm$0.62 & 3.37 $\pm$0.68 & 1.69 $\pm$0.50 & 0.43 $\pm$0.32 & 0.49 $\pm$0.32 & 0.21 $\pm$0.29 & 0.00 $\pm$0.28 & 0.00 $\pm$0.28 \\ 
	4a & 0 & - & 95.76 $\pm$3.53 & 173.49 $\pm$4.81 & 110.69 $\pm$3.57 & 82.12 $\pm$2.66 & 22.46 $\pm$1.22 & 8.09 $\pm$0.57 & 1.24 $\pm$0.30 \\ 
	4a & 1 & - & 89.87 $\pm$3.25 & 153.11 $\pm$4.28 & 96.55 $\pm$3.39 & 60.92 $\pm$2.67 & 10.24 $\pm$1.02 & 3.54 $\pm$0.61 & 0.69 $\pm$0.32 \\ 
	4a & 2 & - & 39.91 $\pm$2.20 & 73.06 $\pm$2.97 & 46.96 $\pm$2.36 & 25.99 $\pm$1.75 & 6.97 $\pm$0.95 & 1.42 $\pm$0.50 & 0.08 $\pm$0.28 \\ 
	4a & $\ge3$ & - & 2.00 $\pm$0.51 & 5.29 $\pm$0.81 & 3.82 $\pm$0.73 & 1.24 $\pm$0.45 & 0.32 $\pm$0.30 & 0.17 $\pm$0.29 & 0.00 $\pm$0.28 \\ 
	$\ge5$a & 0 & - & - & 13.12 $\pm$1.27 & 37.80 $\pm$2.14 & 56.15 $\pm$2.44 & 16.92 $\pm$1.19 & 7.40 $\pm$0.78 & 0.52 $\pm$0.29 \\ 
	$\ge5$a & 1 & - & - & 20.01 $\pm$1.58 & 54.74 $\pm$2.61 & 76.68 $\pm$3.00 & 21.91 $\pm$1.59 & 9.00 $\pm$1.03 & 0.75 $\pm$0.34 \\ 
	$\ge5$a & 2 & - & - & 8.95 $\pm$1.02 & 25.69 $\pm$1.77 & 47.49 $\pm$2.42 & 15.51 $\pm$1.38 & 4.11 $\pm$0.72 & 0.55 $\pm$0.33 \\ 
	$\ge5$a & $\ge3$ & - & - & 1.00 $\pm$0.46 & 3.28 $\pm$0.68 & 3.30 $\pm$0.67 & 2.02 $\pm$0.58 & 0.82 $\pm$0.36 & 0.01 $\pm$0.28 \\ 
	\hline
	\hline
\end{tabular}
\end{table}

\newpage
\begin{table}[h!]
\tiny
\centering
\topcaption{Yields  in the \mmj control region for 1.28\ifb. The letter ``a'' in jet \eg ``2a''  indicates the asymmetric jet bins.\label{tab:yields_ewk_mumu}}
\begin{tabular}
{c|c|cccccccc}
	\hline\hline
   &     & \multicolumn{8}{c}{\scalht (\gev)} \\ 
	\njet & \nb & 200-250 & 250-300 & 300-350 & 350-400 & 400-500 & 500-600 & 600-800 & 800-$\infty$ \\ 
\hline
	1 & 0 & 255.86 $\pm$4.95 & 102.05 $\pm$3.01 & 45.27 $\pm$1.94 & 22.98 $\pm$1.22 & 17.90 $\pm$0.70 & 5.70 $\pm$0.28 & 2.84 $\pm$0.20 & 0.61 $\pm$0.17 \\ 
	1 & 1 & 13.53 $\pm$1.12 & 5.25 $\pm$0.67 & 3.01 $\pm$0.54 & 1.11 $\pm$0.25 & 0.73 $\pm$0.19 & 0.21 $\pm$0.17 & 0.14 $\pm$0.17 & 0.05 $\pm$0.16 \\ 
	2 & 0 & 20.83 $\pm$1.38 & 30.86 $\pm$1.64 & 32.26 $\pm$1.64 & 26.94 $\pm$1.24 & 34.55 $\pm$0.79 & 18.50 $\pm$0.45 & 15.34 $\pm$0.32 & 8.88 $\pm$0.30 \\ 
	2 & 1 & 2.58 $\pm$0.50 & 3.52 $\pm$0.57 & 3.72 $\pm$0.57 & 3.07 $\pm$0.46 & 3.79 $\pm$0.31 & 1.92 $\pm$0.26 & 1.57 $\pm$0.21 & 0.82 $\pm$0.17 \\ 
	2 & 2 & 0.08 $\pm$0.17 & 0.38 $\pm$0.20 & 0.23 $\pm$0.19 & 0.49 $\pm$0.21 & 0.13 $\pm$0.17 & 0.07 $\pm$0.16 & 0.09 $\pm$0.16 & 0.02 $\pm$0.16 \\ 
	3 & 0 & 0.05 $\pm$0.17 & 6.74 $\pm$0.78 & 15.83 $\pm$1.14 & 20.68 $\pm$1.10 & 31.37 $\pm$0.79 & 20.17 $\pm$0.47 & 19.65 $\pm$0.36 & 12.91 $\pm$0.30 \\ 
	3 & 1 & 0.00 $\pm$0.16 & 0.77 $\pm$0.25 & 2.81 $\pm$0.53 & 2.59 $\pm$0.39 & 6.14 $\pm$0.58 & 3.07 $\pm$0.27 & 2.98 $\pm$0.25 & 1.88 $\pm$0.19 \\ 
	3 & 2 & 0.00 $\pm$0.16 & 0.28 $\pm$0.19 & 0.56 $\pm$0.22 & 1.08 $\pm$0.28 & 1.29 $\pm$0.35 & 0.72 $\pm$0.23 & 0.29 $\pm$0.17 & 0.30 $\pm$0.19 \\ 
	3 & $\ge3$ & 0.00 $\pm$0.16 & 0.01 $\pm$0.16 & 0.00 $\pm$0.16 & 0.08 $\pm$0.17 & 0.02 $\pm$0.16 & 0.00 $\pm$0.16 & 0.00 $\pm$0.16 & 0.01 $\pm$0.16 \\ 
	4 & 0 & - & 0.00 $\pm$0.16 & 2.98 $\pm$0.54 & 5.90 $\pm$0.67 & 15.94 $\pm$0.68 & 12.79 $\pm$0.42 & 13.63 $\pm$0.33 & 10.50 $\pm$0.29 \\ 
	4 & 1 & - & 0.00 $\pm$0.16 & 0.31 $\pm$0.19 & 1.11 $\pm$0.27 & 3.35 $\pm$0.44 & 3.05 $\pm$0.39 & 3.06 $\pm$0.27 & 2.00 $\pm$0.19 \\ 
	4 & 2 & - & 0.00 $\pm$0.16 & 0.00 $\pm$0.16 & 0.63 $\pm$0.24 & 1.37 $\pm$0.36 & 0.77 $\pm$0.23 & 0.66 $\pm$0.20 & 0.41 $\pm$0.19 \\ 
	4 & $\ge3$ & - & 0.00 $\pm$0.16 & 0.00 $\pm$0.16 & 0.00 $\pm$0.16 & 0.03 $\pm$0.17 & 0.02 $\pm$0.16 & 0.03 $\pm$0.16 & 0.03 $\pm$0.16 \\ 
	$\ge5$ & 0 & - & - & 0.00 $\pm$0.16 & 0.09 $\pm$0.17 & 2.91 $\pm$0.37 & 4.39 $\pm$0.29 & 7.96 $\pm$0.33 & 8.96 $\pm$0.28 \\ 
	$\ge5$ & 1 & - & - & 0.00 $\pm$0.16 & 0.03 $\pm$0.17 & 1.37 $\pm$0.29 & 1.85 $\pm$0.37 & 2.15 $\pm$0.27 & 2.87 $\pm$0.36 \\ 
	$\ge5$ & 2 & - & - & 0.00 $\pm$0.16 & 0.03 $\pm$0.17 & 0.51 $\pm$0.22 & 0.42 $\pm$0.19 & 0.89 $\pm$0.26 & 1.06 $\pm$0.26 \\ 
	$\ge5$ & $\ge3$ & - & - & 0.00 $\pm$0.16 & 0.00 $\pm$0.16 & 0.00 $\pm$0.16 & 0.07 $\pm$0.17 & 0.28 $\pm$0.20 & 0.05 $\pm$0.16 \\ 
	2a & 0 & 207.84 $\pm$4.48 & 116.59 $\pm$3.29 & 60.10 $\pm$2.27 & 28.77 $\pm$1.39 & 18.49 $\pm$0.60 & 6.07 $\pm$0.29 & 2.97 $\pm$0.21 & 0.64 $\pm$0.17 \\ 
	2a & 1 & 23.33 $\pm$1.50 & 11.52 $\pm$1.04 & 6.84 $\pm$0.81 & 2.38 $\pm$0.38 & 1.58 $\pm$0.22 & 0.85 $\pm$0.24 & 0.26 $\pm$0.17 & 0.06 $\pm$0.16 \\ 
	2a & 2 & 3.24 $\pm$0.60 & 1.60 $\pm$0.39 & 0.24 $\pm$0.19 & 0.02 $\pm$0.16 & 0.06 $\pm$0.16 & 0.02 $\pm$0.16 & 0.01 $\pm$0.16 & 0.00 $\pm$0.16 \\ 
	3a & 0 & 38.65 $\pm$1.91 & 55.65 $\pm$2.29 & 29.70 $\pm$1.61 & 17.04 $\pm$1.03 & 14.80 $\pm$0.60 & 3.91 $\pm$0.25 & 2.07 $\pm$0.21 & 0.34 $\pm$0.17 \\ 
	3a & 1 & 7.29 $\pm$0.87 & 7.64 $\pm$0.89 & 5.45 $\pm$0.71 & 2.81 $\pm$0.46 & 2.05 $\pm$0.27 & 0.61 $\pm$0.19 & 0.41 $\pm$0.19 & 0.04 $\pm$0.17 \\ 
	3a & 2 & 1.81 $\pm$0.43 & 3.51 $\pm$0.67 & 1.83 $\pm$0.48 & 0.49 $\pm$0.20 & 0.63 $\pm$0.23 & 0.06 $\pm$0.16 & 0.01 $\pm$0.16 & 0.00 $\pm$0.16 \\ 
	3a & $\ge3$ & 0.05 $\pm$0.17 & 0.32 $\pm$0.20 & 0.06 $\pm$0.17 & 0.01 $\pm$0.16 & 0.01 $\pm$0.16 & 0.00 $\pm$0.16 & 0.00 $\pm$0.16 & 0.00 $\pm$0.16 \\ 
	4a & 0 & - & 5.71 $\pm$0.70 & 10.53 $\pm$0.92 & 8.46 $\pm$0.72 & 6.80 $\pm$0.44 & 2.22 $\pm$0.23 & 1.03 $\pm$0.20 & 0.16 $\pm$0.17 \\ 
	4a & 1 & - & 1.55 $\pm$0.40 & 3.34 $\pm$0.60 & 1.80 $\pm$0.36 & 2.14 $\pm$0.39 & 0.47 $\pm$0.19 & 0.26 $\pm$0.18 & 0.01 $\pm$0.16 \\ 
	4a & 2 & - & 0.24 $\pm$0.19 & 0.79 $\pm$0.26 & 0.48 $\pm$0.21 & 1.12 $\pm$0.37 & 0.33 $\pm$0.20 & 0.13 $\pm$0.18 & 0.19 $\pm$0.19 \\ 
	4a & $\ge3$ & - & 0.00 $\pm$0.16 & 0.01 $\pm$0.16 & 0.00 $\pm$0.16 & 0.00 $\pm$0.16 & 0.00 $\pm$0.16 & 0.00 $\pm$0.16 & 0.00 $\pm$0.16 \\ 
	$\ge5$a & 0 & - & - & 0.85 $\pm$0.26 & 1.88 $\pm$0.42 & 2.25 $\pm$0.27 & 1.04 $\pm$0.19 & 0.44 $\pm$0.17 & 0.10 $\pm$0.16 \\ 
	$\ge5$a & 1 & - & - & 0.19 $\pm$0.18 & 0.52 $\pm$0.22 & 1.57 $\pm$0.36 & 0.64 $\pm$0.24 & 0.19 $\pm$0.17 & 0.16 $\pm$0.18 \\ 
	$\ge5$a & 2 & - & - & 0.00 $\pm$0.16 & 0.26 $\pm$0.19 & 0.29 $\pm$0.19 & 0.15 $\pm$0.17 & 0.23 $\pm$0.19 & 0.00 $\pm$0.16 \\ 
	$\ge5$a & $\ge3$ & - & - & 0.00 $\pm$0.16 & 0.00 $\pm$0.16 & 0.24 $\pm$0.19 & 0.00 $\pm$0.16 & 0.00 $\pm$0.16 & 0.00 $\pm$0.16 \\ 
	\hline
	\hline
\end{tabular}
\end{table}

\newpage
\begin{table}[h!]
\tiny
\centering
\topcaption{Yields  in the \gj control region for 3.0\ifb. The letter ``a'' in jet \eg ``2a''  indicates the asymmetric jet bins.\label{tab:yields_ewk_gj_3fb}}
\begin{tabular}
{c|c|cccc}
	\hline\hline
   &     & \multicolumn{4}{c}{\scalht (\gev)} \\ 
	\njet & \nb & 400-500 & 500-600 & 600-800 & 800-$\infty$ \\ 
\hline
	1 & 0 & 550.71 $\pm$16.32 & 178.61 $\pm$8.10 & 93.67 $\pm$4.12 & 23.91 $\pm$2.01 \\ 
	1 & 1 & 22.97 $\pm$3.15 & 7.33 $\pm$1.66 & 4.52 $\pm$1.08 & 1.27 $\pm$0.85 \\ 
	2 & 0 & 260.59 $\pm$10.97 & 109.70 $\pm$6.32 & 71.72 $\pm$3.61 & 166.92 $\pm$5.04 \\ 
	2 & 1 & 21.74 $\pm$3.15 & 9.19 $\pm$1.96 & 4.92 $\pm$1.10 & 16.55 $\pm$1.74 \\ 
	2 & 2 & 1.63 $\pm$0.94 & 0.50 $\pm$0.77 & 0.24 $\pm$0.74 & 0.46 $\pm$0.75 \\ 
	3 & 0 & 407.84 $\pm$14.07 & 180.73 $\pm$8.12 & 112.58 $\pm$4.34 & 241.25 $\pm$5.98 \\ 
	3 & 1 & 40.80 $\pm$4.24 & 19.85 $\pm$2.71 & 9.77 $\pm$1.36 & 31.79 $\pm$2.29 \\ 
	3 & 2 & 3.64 $\pm$1.36 & 1.86 $\pm$0.89 & 1.24 $\pm$0.84 & 2.95 $\pm$0.99 \\ 
	3 & $\ge3$ & 0.00 $\pm$0.73 & 0.00 $\pm$0.73 & 0.04 $\pm$0.73 & 0.19 $\pm$0.74 \\ 
	4 & 0 & 219.39 $\pm$10.06 & 143.80 $\pm$7.37 & 124.20 $\pm$4.95 & 193.89 $\pm$5.41 \\ 
	4 & 1 & 32.41 $\pm$3.88 & 21.32 $\pm$2.92 & 16.71 $\pm$1.76 & 31.18 $\pm$2.24 \\ 
	4 & 2 & 7.82 $\pm$1.93 & 3.41 $\pm$1.10 & 3.58 $\pm$1.01 & 4.71 $\pm$1.08 \\ 
	4 & $\ge3$ & 0.05 $\pm$0.73 & 0.28 $\pm$0.76 & 0.12 $\pm$0.73 & 0.00 $\pm$0.73 \\ 
	$\ge5$ & 0 & 47.09 $\pm$5.57 & 63.76 $\pm$5.12 & 84.96 $\pm$4.29 & 148.69 $\pm$4.75 \\ 
	$\ge5$ & 1 & 8.28 $\pm$1.85 & 15.79 $\pm$2.54 & 21.15 $\pm$2.32 & 37.90 $\pm$2.46 \\ 
	$\ge5$ & 2 & 1.46 $\pm$0.91 & 2.67 $\pm$1.03 & 4.29 $\pm$1.14 & 6.95 $\pm$1.23 \\ 
	$\ge5$ & $\ge3$ & 0.00 $\pm$0.73 & 0.00 $\pm$0.73 & 0.12 $\pm$0.73 & 0.50 $\pm$0.75 \\ 
	2a & 0 & 128.17 $\pm$7.78 & 38.44 $\pm$3.83 & 21.99 $\pm$2.21 & 21.05 $\pm$1.85 \\ 
	2a & 1 & 7.35 $\pm$1.77 & 2.80 $\pm$1.01 & 0.66 $\pm$0.76 & 1.92 $\pm$0.90 \\ 
	2a & 2 & 0.00 $\pm$0.73 & 0.00 $\pm$0.73 & 0.00 $\pm$0.73 & 0.12 $\pm$0.73 \\ 
	3a & 0 & 81.16 $\pm$6.37 & 18.57 $\pm$2.62 & 6.75 $\pm$1.22 & 11.16 $\pm$1.46 \\ 
	3a & 1 & 11.01 $\pm$2.37 & 2.36 $\pm$0.96 & 1.31 $\pm$0.85 & 0.85 $\pm$0.78 \\ 
	3a & 2 & 1.54 $\pm$0.93 & 0.35 $\pm$0.77 & 0.09 $\pm$0.73 & 0.00 $\pm$0.73 \\ 
	3a & $\ge3$ & 0.00 $\pm$0.73 & 0.00 $\pm$0.73 & 0.00 $\pm$0.73 & 0.00 $\pm$0.73 \\ 
	4a & 0 & 90.41 $\pm$6.86 & 15.41 $\pm$2.55 & 4.62 $\pm$1.14 & 3.65 $\pm$1.05 \\ 
	4a & 1 & 12.25 $\pm$2.38 & 3.84 $\pm$1.40 & 0.58 $\pm$0.76 & 0.41 $\pm$0.75 \\ 
	4a & 2 & 2.04 $\pm$1.00 & 0.00 $\pm$0.73 & 0.00 $\pm$0.73 & 0.00 $\pm$0.73 \\ 
	4a & $\ge3$ & 0.13 $\pm$0.75 & 0.00 $\pm$0.73 & 0.22 $\pm$0.74 & 0.00 $\pm$0.73 \\ 
	$\ge5$a & 0 & 50.83 $\pm$5.71 & 10.89 $\pm$2.15 & 4.42 $\pm$1.19 & 0.90 $\pm$0.78 \\ 
	$\ge5$a & 1 & 4.85 $\pm$1.46 & 3.51 $\pm$1.37 & 0.72 $\pm$0.77 & 0.21 $\pm$0.74 \\ 
	$\ge5$a & 2 & 2.27 $\pm$1.04 & 0.14 $\pm$0.74 & 0.00 $\pm$0.73 & 0.00 $\pm$0.73 \\ 
	$\ge5$a & $\ge3$ & 0.00 $\pm$0.73 & 0.00 $\pm$0.73 & 0.00 $\pm$0.73 & 0.00 $\pm$0.73 \\ 
	\hline
	\hline
\end{tabular}
\end{table}

\newpage

%%____________________________________________________________________________||
\subsection{Signal acceptance from asymmetric jet \Pt thresholds}

For a typical compressed model, Fig.~\ref{fig:asymMotivation} shows the second jet \PT
distribution for two different \HT bins. In the low \HT case a large portion of
the events are killed by requiring the second jet to have $\ET>100\gev$. 

We therefore add an new analysis category where the leading jet is required to fulfil 
momenta of jet $\ET>100\gev$ and sub-leading jet $40\gev<\ET<100\gev$. This new category 
results in new asymmetric jet bins split also in $\njet$, $\nb$ and \HT. The asymmetric jet bin
results in much larger acceptance for monojet-type and ISR induced event topologies like DM signals
and compresses SUSY scenarios. 
For the simplified model T2tt with $m_{\rm stop}=425\gev$ and $m_{\rm LSP}=325\gev$, 
including these bins increases signal acceptance by around a factor 3, for the DM models a factor of five is observed.
The asymmetric bins will be added to the nominal selection, thus in combination the for all models the analysis power can only improve or remain constant.
\begin{figure}[h!]
  \centering
  \subfigure[Second jet \PT for $200\gev<\HT<250\gev$, $\alphat>0.65$]{
    \includegraphics[width=0.5\textwidth]{figures/asymPlots/secondJetPtlowHT}
  }~~
  \subfigure[Second jet \PT for $400\gev<\HT>500\gev$, $\alphat>0.52$]{
    \includegraphics[width=0.5\textwidth]{figures/asymPlots/secondJetPthigherHT}
  }
  \\
  \caption{\label{fig:asymMotivation} The second leading jet \PT for different
  cases of \HT after a baseline signal selection: $\njet\geq2$, lead jet
  $\ET>100\gev$, lepton vetoes. Made with the T2tt ($m_{\rm
    stop}=425\gev$, $m_{\rm LSP}=325\gev$) simplified model sample.}
\end{figure}

