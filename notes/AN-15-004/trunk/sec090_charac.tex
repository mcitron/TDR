%%____________________________________________________________________________||
\section{Characterisation of the signal and control regions}
\label{sec:yields}

%%____________________________________________________________________________||
\subsection{Distributions for key analysis variables \label{sec:mc-data-comp}}

The hadronic signal region selection and control regions are detailed in Sec.~\ref{sec:selection}. The relevant data/MC plots are presented here, corresponding to symmetric, asymmetric and monojet selection. The distributions help to understand the MC modeling of key variables. The MC distributions are normalised to their corresponding luminosities in Run 2015D. These luminosities are calculated  with the consideration of the HLT triggers used in each region. The data/MC scale factors are shown in all figures.

%%____________________________________________________________________________||
\begin{figure}
    \begin{center}
        \subfigure {\includegraphics[width=0.5\textwidth]{figures/distributions/Signal/alphaT_sym.pdf}} ~~
        \subfigure {\includegraphics[width=0.5\textwidth]{figures/distributions/Signal/ht40_sym.pdf}} \\
        \subfigure {\includegraphics[width=0.5\textwidth]{figures/distributions/Signal/mht40_pt_sym.pdf}} ~~
        \subfigure {\includegraphics[width=0.5\textwidth]{figures/distributions/Signal/jet_pt[0]_sym.pdf}} \\
        \caption{Key analysis variables for hadronic signal region (symmetric bins)}
        \label{fig:distribution_signal_sym}
    \end{center}
\end{figure}

\begin{figure}
    \begin{center}
        \subfigure {\includegraphics[width=0.5\textwidth]{figures/distributions/Signal/alphaT_asym.pdf}} ~~
        \subfigure {\includegraphics[width=0.5\textwidth]{figures/distributions/Signal/ht40_asym.pdf}} \\
        \subfigure {\includegraphics[width=0.5\textwidth]{figures/distributions/Signal/mht40_pt_asym.pdf}} ~~
        \subfigure {\includegraphics[width=0.5\textwidth]{figures/distributions/Signal/jet_pt[0]_asym.pdf}} \\
        \caption{Key analysis variables for hadronic signal region (asymmetric bins)}
        \label{fig:distribution_signal_asym}
    \end{center}
\end{figure}

\begin{figure}
    \begin{center}
        \subfigure {\includegraphics[width=0.5\textwidth]{figures/distributions/Signal/jet_pt[0]_eq1j.pdf}} ~~
        \subfigure {\includegraphics[width=0.5\textwidth]{figures/distributions/Signal/jet_eta[0]_eq1j.pdf}} \\
        \subfigure {\includegraphics[width=0.5\textwidth]{figures/distributions/Signal/met_pt_eq1j.pdf}} ~~
        \subfigure {\includegraphics[width=0.5\textwidth]{figures/distributions/Signal/biasedDPhi20_eq1j.pdf}} \\
        \caption{Key analysis variables for hadronic signal region (monojet bins)}
        \label{fig:distribution_signal_mono}
    \end{center}
\end{figure}
%%____________________________________________________________________________||
\begin{figure}
    \begin{center}
        \subfigure {\includegraphics[width=0.5\textwidth]{figures/distributions/SingleMu/alphaT_sym.pdf}} ~~
        \subfigure {\includegraphics[width=0.5\textwidth]{figures/distributions/SingleMu/ht40_sym.pdf}} \\
        \subfigure {\includegraphics[width=0.5\textwidth]{figures/distributions/SingleMu/mht40_pt_sym.pdf}} ~~
        \subfigure {\includegraphics[width=0.5\textwidth]{figures/distributions/SingleMu/jet_pt[0]_sym.pdf}} \\
        \caption{Key analysis variables for single muon control region (symmetric bins)}
        \label{fig:distribution_singlemu_sym}
    \end{center}
\end{figure}

\begin{figure}
    \begin{center}
        \subfigure {\includegraphics[width=0.5\textwidth]{figures/distributions/SingleMu/alphaT_asym.pdf}} ~~
        \subfigure {\includegraphics[width=0.5\textwidth]{figures/distributions/SingleMu/ht40_asym.pdf}} \\
        \subfigure {\includegraphics[width=0.5\textwidth]{figures/distributions/SingleMu/mht40_pt_asym.pdf}} ~~
        \subfigure {\includegraphics[width=0.5\textwidth]{figures/distributions/SingleMu/jet_pt[0]_asym.pdf}} \\
        \caption{Key analysis variables for single muon control region (asymmetric bins)}
        \label{fig:distribution_singlemu_asym}
    \end{center}
\end{figure}

\begin{figure}
    \begin{center}
        
        \subfigure {\includegraphics[width=0.5\textwidth]{figures/distributions/SingleMu/jet_pt[0]_eq1j.pdf}} ~~
        \subfigure {\includegraphics[width=0.5\textwidth]{figures/distributions/SingleMu/jet_eta[0]_eq1j.pdf}} \\
        \subfigure {\includegraphics[width=0.5\textwidth]{figures/distributions/SingleMu/metNoMu_pt_eq1j.pdf}} ~~
        \subfigure {\includegraphics[width=0.5\textwidth]{figures/distributions/SingleMu/muon_pt[0]_eq1j.pdf}} \\
        \caption{Key analysis variables for single muon control region (monojet bins)}
        \label{fig:distribution_singlemu_mono}
    \end{center}
\end{figure}
%%____________________________________________________________________________||
\begin{figure}
    \begin{center}
        \subfigure {\includegraphics[width=0.5\textwidth]{figures/distributions/DoubleMu/alphaT_sym.pdf}} ~~
        \subfigure {\includegraphics[width=0.5\textwidth]{figures/distributions/DoubleMu/ht40_sym.pdf}} \\
        \subfigure {\includegraphics[width=0.5\textwidth]{figures/distributions/DoubleMu/mht40_pt_sym.pdf}} ~~
        \subfigure {\includegraphics[width=0.5\textwidth]{figures/distributions/DoubleMu/jet_pt[0]_sym.pdf}} \\
        \caption{Key analysis variables for double muon control region (symmetric bins)}
        \label{fig:distribution_doublemu_sym}
    \end{center}
\end{figure}

\begin{figure}
    \begin{center}
        \subfigure {\includegraphics[width=0.5\textwidth]{figures/distributions/DoubleMu/alphaT_asym.pdf}} ~~
        \subfigure {\includegraphics[width=0.5\textwidth]{figures/distributions/DoubleMu/ht40_asym.pdf}} \\
        \subfigure {\includegraphics[width=0.5\textwidth]{figures/distributions/DoubleMu/mht40_pt_asym.pdf}} ~~
        \subfigure {\includegraphics[width=0.5\textwidth]{figures/distributions/DoubleMu/jet_pt[0]_asym.pdf}} \\
        \caption{Key analysis variables for double muon control region (asymmetric bins)}
        \label{fig:distribution_doublemu_asym}
    \end{center}
\end{figure}

\begin{figure}
    \begin{center} 
        \subfigure {\includegraphics[width=0.5\textwidth]{figures/distributions/DoubleMu/jet_pt[0]_eq1j.pdf}} ~~
        \subfigure {\includegraphics[width=0.5\textwidth]{figures/distributions/DoubleMu/jet_eta[0]_eq1j.pdf}} \\
        \subfigure {\includegraphics[width=0.5\textwidth]{figures/distributions/DoubleMu/metNoMu_pt_eq1j.pdf}} ~~
        \subfigure {\includegraphics[width=0.5\textwidth]{figures/distributions/DoubleMu/muon_pt[0]_eq1j.pdf}} \\
        \caption{Key analysis variables for double muon control region (monojet bins)}
        \label{fig:distribution_doublemu_mono}
    \end{center}
\end{figure}
%%____________________________________________________________________________||
\begin{figure}
    \begin{center}
        \subfigure {\includegraphics[width=0.5\textwidth]{figures/distributions/SinglePhoton/alphaT_sym.pdf}} ~~
        \subfigure {\includegraphics[width=0.5\textwidth]{figures/distributions/SinglePhoton/ht40_sym.pdf}} \\
        \subfigure {\includegraphics[width=0.5\textwidth]{figures/distributions/SinglePhoton/mht40_pt_sym.pdf}} ~~
        \subfigure {\includegraphics[width=0.5\textwidth]{figures/distributions/SinglePhoton/jet_pt[0]_sym.pdf}} \\
        \caption{Key analysis variables for single photon control region (symmetric bins)}
        \label{fig:distribution_singlephoton_sym}
    \end{center}
\end{figure}

\begin{figure}[h]
    \begin{center}
        \subfigure {\includegraphics[width=0.5\textwidth]{figures/distributions/SinglePhoton/alphaT_asym.pdf}} ~~
        \subfigure {\includegraphics[width=0.5\textwidth]{figures/distributions/SinglePhoton/ht40_asym.pdf}} \\
        \subfigure {\includegraphics[width=0.5\textwidth]{figures/distributions/SinglePhoton/mht40_pt_asym.pdf}} ~~
        \subfigure {\includegraphics[width=0.5\textwidth]{figures/distributions/SinglePhoton/jet_pt[0]_asym.pdf}} \\
        \caption{Key analysis variables for single photon control region (asymmetric bins)}
        \label{fig:distribution_singlephoton_asym}
    \end{center}
\end{figure}

\begin{figure}
    \begin{center} 
        \subfigure {\includegraphics[width=0.5\textwidth]{figures/distributions/SinglePhoton/jet_pt[0]_eq1j.pdf}} ~~
        \subfigure {\includegraphics[width=0.5\textwidth]{figures/distributions/SinglePhoton/jet_eta[0]_eq1j.pdf}} \\
        \subfigure {\includegraphics[width=0.5\textwidth]{figures/distributions/SinglePhoton/gamma_pt[0]_eq1j.pdf}} ~~
        \subfigure {\includegraphics[width=0.5\textwidth]{figures/distributions/SinglePhoton/gamma_eta[0]_eq1j.pdf}} \\
        \caption{Key analysis variables for single photon control region (monojet bins)}
        \label{fig:distribution_singlephoton_mono}
    \end{center}
\end{figure}

\clearpage

%%____________________________________________________________________________||
\subsection{Breakdown of SM backgrounds in the hadronic signal
  region\label{sec:bkgd-comp}}

In the absence of multijet events from QCD, the remaining significant
backgrounds in the signal region are expected to stem from SM
processes with genuine \met in the final state. For the low jet
multiplicity categories, the largest backgrounds with genuine \met are
 from the associated production of W or Z bosons with jets,
followed by either the weak decays \znunu\ or \wtaunu, where the
$\tau$ decays hadronically and is identified as a jet, or by leptonic
decays that are outside acceptance or not rejected by the dedicated
electron or muon vetoes. For the higher jet multiplicity categories,
top quark production followed by semileptonic weak top quark decay
becomes dominant. The relative contribution from \ttbar is depending on 
the jet multiplicity with increase importance for large jet multiplicities.

% A breakdown
% of the relative contributions of the SM backgrounds, as given by
% simulation, in the different (\njet, \nb, \scalht) bins can be found
% in Table~\ref{tab:backgrounds}. 
Tables showing the yields for these electroweak backgrounds can be seen in Table~\ref{tab:yields_ewk_sig}.
An overview of the three dominant channels, \ttbar, \zInv~ and W~+~jets, are shown in Tables \ref{tab:yields_tt_sig}, 
\ref{tab:yields_zinv_sig} and \ref{tab:yields_wjetstolnuht_sig} respectively for 1.28\ifb 
(yields for 3\ifb are in Appendix~\ref{app:yields3fb}). The contribution from
other sources, such as the single top and diboson channels, was found to be
negligible so are not shown.

%\begin{landscape}

% \newpage
% \input{tables/dataLumi/yields_sig_sym_ewk.tex}
% \newpage
% \input{tables/dataLumi/yields_sig_sym_tt.tex}
% \newpage
% \input{tables/dataLumi/yields_sig_s_wjetstolnuht.tex}
% \newpage
% \begin{table}[h!]
\tiny
\centering
\topcaption{Yields for the \zInv~ process in the signal region for 1.28\ifb. The letter ``a'' in jet \eg ``2a''  indicates the asymmetric jet bins.\label{tab:yields_zinv_sig}}
\begin{tabular}
{c|c|cccccccc}
	\hline\hline
   &     & \multicolumn{8}{c}{\scalht (\gev)} \\ 
	\njet & \nb & 200-250 & 250-300 & 300-350 & 350-400 & 400-500 & 500-600 & 600-800 & 800-$\infty$ \\ 
\hline
	1 & 0 & 1937.37 $\pm$9.57 & 692.55 $\pm$4.80 & 294.28 $\pm$3.09 & 134.66 $\pm$1.99 & 100.64 $\pm$1.56 & 31.70 $\pm$0.81 & 14.55 $\pm$0.41 & 3.60 $\pm$0.20 \\ 
	1 & 1 & 77.25 $\pm$1.85 & 29.02 $\pm$0.99 & 12.94 $\pm$0.64 & 5.88 $\pm$0.42 & 4.76 $\pm$0.34 & 1.24 $\pm$0.17 & 0.70 $\pm$0.12 & 0.12 $\pm$0.10 \\ 
	2 & 0 & 206.72 $\pm$2.82 & 237.00 $\pm$2.79 & 166.14 $\pm$2.28 & 106.61 $\pm$1.73 & 102.13 $\pm$1.55 & 37.58 $\pm$0.87 & 19.27 $\pm$0.44 & 26.17 $\pm$0.48 \\ 
	2 & 1 & 14.75 $\pm$0.79 & 15.86 $\pm$0.71 & 13.48 $\pm$0.65 & 8.86 $\pm$0.50 & 7.92 $\pm$0.43 & 3.04 $\pm$0.25 & 2.05 $\pm$0.17 & 2.58 $\pm$0.18 \\ 
	2 & 2 & 1.30 $\pm$0.22 & 1.56 $\pm$0.24 & 0.94 $\pm$0.19 & 0.61 $\pm$0.16 & 0.48 $\pm$0.13 & 0.25 $\pm$0.12 & 0.12 $\pm$0.10 & 0.06 $\pm$0.10 \\ 
	3 & 0 & 0.22 $\pm$0.11 & 39.41 $\pm$1.12 & 108.64 $\pm$1.87 & 112.10 $\pm$1.76 & 144.59 $\pm$1.83 & 59.23 $\pm$1.08 & 36.96 $\pm$0.61 & 36.47 $\pm$0.57 \\ 
	3 & 1 & 0.03 $\pm$0.10 & 3.79 $\pm$0.36 & 11.11 $\pm$0.58 & 11.90 $\pm$0.57 & 17.29 $\pm$0.65 & 6.98 $\pm$0.36 & 4.46 $\pm$0.22 & 4.84 $\pm$0.23 \\ 
	3 & 2 & 0.00 $\pm$0.09 & 0.58 $\pm$0.16 & 1.47 $\pm$0.23 & 1.30 $\pm$0.20 & 1.69 $\pm$0.21 & 0.81 $\pm$0.15 & 0.47 $\pm$0.11 & 0.42 $\pm$0.11 \\ 
	3 & $\ge3$ & 0.00 $\pm$0.09 & 0.00 $\pm$0.09 & 0.04 $\pm$0.10 & 0.08 $\pm$0.10 & 0.01 $\pm$0.09 & 0.01 $\pm$0.09 & 0.03 $\pm$0.09 & 0.00 $\pm$0.09 \\ 
	4 & 0 & - & 0.14 $\pm$0.11 & 13.59 $\pm$0.68 & 42.41 $\pm$1.11 & 87.13 $\pm$1.45 & 50.01 $\pm$1.00 & 35.06 $\pm$0.61 & 29.97 $\pm$0.52 \\ 
	4 & 1 & - & 0.04 $\pm$0.10 & 2.01 $\pm$0.27 & 5.91 $\pm$0.41 & 12.76 $\pm$0.55 & 8.11 $\pm$0.41 & 5.94 $\pm$0.27 & 5.31 $\pm$0.23 \\ 
	4 & 2 & - & 0.00 $\pm$0.09 & 0.46 $\pm$0.16 & 0.73 $\pm$0.17 & 2.45 $\pm$0.26 & 1.29 $\pm$0.22 & 0.85 $\pm$0.13 & 0.75 $\pm$0.12 \\ 
	4 & $\ge3$ & - & 0.00 $\pm$0.09 & 0.00 $\pm$0.09 & 0.13 $\pm$0.10 & 0.16 $\pm$0.10 & 0.09 $\pm$0.10 & 0.02 $\pm$0.09 & 0.02 $\pm$0.09 \\ 
	$\ge5$ & 0 & - & - & 0.00 $\pm$0.09 & 1.72 $\pm$0.23 & 20.05 $\pm$0.71 & 22.29 $\pm$0.69 & 23.54 $\pm$0.55 & 24.23 $\pm$0.47 \\ 
	$\ge5$ & 1 & - & - & 0.00 $\pm$0.09 & 0.28 $\pm$0.11 & 3.81 $\pm$0.31 & 3.86 $\pm$0.30 & 5.07 $\pm$0.27 & 5.13 $\pm$0.23 \\ 
	$\ge5$ & 2 & - & - & 0.00 $\pm$0.09 & 0.06 $\pm$0.10 & 0.73 $\pm$0.16 & 0.91 $\pm$0.16 & 0.95 $\pm$0.15 & 1.05 $\pm$0.13 \\ 
	$\ge5$ & $\ge3$ & - & - & 0.00 $\pm$0.09 & 0.00 $\pm$0.09 & 0.04 $\pm$0.10 & 0.04 $\pm$0.09 & 0.16 $\pm$0.10 & 0.14 $\pm$0.10 \\ 
	2a & 0 & 1081.21 $\pm$6.97 & 350.11 $\pm$3.38 & 142.84 $\pm$2.12 & 63.38 $\pm$1.34 & 44.57 $\pm$1.03 & 11.62 $\pm$0.48 & 5.10 $\pm$0.24 & 5.08 $\pm$0.23 \\ 
	2a & 1 & 75.27 $\pm$1.73 & 24.41 $\pm$0.88 & 10.30 $\pm$0.57 & 5.24 $\pm$0.40 & 3.26 $\pm$0.28 & 0.98 $\pm$0.17 & 0.45 $\pm$0.11 & 0.36 $\pm$0.11 \\ 
	2a & 2 & 6.83 $\pm$0.52 & 1.80 $\pm$0.25 & 0.82 $\pm$0.18 & 0.23 $\pm$0.11 & 0.17 $\pm$0.11 & 0.03 $\pm$0.09 & 0.01 $\pm$0.09 & 0.01 $\pm$0.09 \\ 
	3a & 0 & 264.48 $\pm$3.29 & 272.80 $\pm$3.00 & 143.94 $\pm$2.14 & 54.88 $\pm$1.25 & 28.02 $\pm$0.82 & 5.98 $\pm$0.38 & 2.37 $\pm$0.18 & 2.14 $\pm$0.17 \\ 
	3a & 1 & 25.33 $\pm$0.98 & 27.33 $\pm$0.93 & 15.32 $\pm$0.69 & 6.06 $\pm$0.42 & 3.03 $\pm$0.28 & 0.42 $\pm$0.12 & 0.22 $\pm$0.10 & 0.29 $\pm$0.11 \\ 
	3a & 2 & 2.95 $\pm$0.32 & 3.74 $\pm$0.35 & 2.67 $\pm$0.31 & 0.50 $\pm$0.14 & 0.50 $\pm$0.14 & 0.10 $\pm$0.10 & 0.02 $\pm$0.09 & 0.01 $\pm$0.09 \\ 
	3a & $\ge3$ & 0.04 $\pm$0.10 & 0.14 $\pm$0.11 & 0.00 $\pm$0.09 & 0.00 $\pm$0.09 & 0.00 $\pm$0.09 & 0.00 $\pm$0.09 & 0.00 $\pm$0.09 & 0.00 $\pm$0.09 \\ 
	4a & 0 & - & 27.26 $\pm$0.95 & 72.45 $\pm$1.51 & 48.17 $\pm$1.19 & 32.67 $\pm$0.91 & 5.06 $\pm$0.33 & 1.22 $\pm$0.15 & 0.77 $\pm$0.12 \\ 
	4a & 1 & - & 4.35 $\pm$0.38 & 9.58 $\pm$0.55 & 6.84 $\pm$0.45 & 4.91 $\pm$0.35 & 0.62 $\pm$0.14 & 0.15 $\pm$0.10 & 0.11 $\pm$0.10 \\ 
	4a & 2 & - & 1.01 $\pm$0.21 & 1.70 $\pm$0.24 & 0.94 $\pm$0.18 & 0.78 $\pm$0.15 & 0.08 $\pm$0.10 & 0.00 $\pm$0.09 & 0.02 $\pm$0.09 \\ 
	4a & $\ge3$ & - & 0.05 $\pm$0.10 & 0.10 $\pm$0.10 & 0.05 $\pm$0.10 & 0.00 $\pm$0.09 & 0.00 $\pm$0.09 & 0.00 $\pm$0.09 & 0.00 $\pm$0.09 \\ 
	$\ge5$a & 0 & - & - & 5.67 $\pm$0.44 & 14.21 $\pm$0.66 & 17.77 $\pm$0.68 & 3.86 $\pm$0.29 & 1.31 $\pm$0.17 & 0.27 $\pm$0.11 \\ 
	$\ge5$a & 1 & - & - & 0.86 $\pm$0.18 & 2.43 $\pm$0.28 & 2.81 $\pm$0.27 & 0.76 $\pm$0.15 & 0.22 $\pm$0.10 & 0.03 $\pm$0.09 \\ 
	$\ge5$a & 2 & - & - & 0.16 $\pm$0.11 & 0.46 $\pm$0.14 & 0.83 $\pm$0.17 & 0.28 $\pm$0.12 & 0.03 $\pm$0.09 & 0.02 $\pm$0.09 \\ 
	$\ge5$a & $\ge3$ & - & - & 0.00 $\pm$0.09 & 0.00 $\pm$0.09 & 0.11 $\pm$0.10 & 0.06 $\pm$0.10 & 0.00 $\pm$0.09 & 0.00 $\pm$0.09 \\ 
	\hline
	\hline
\end{tabular}
\end{table}


%%____________________________________________________________________________||
\newpage
\subsection{Yields in the control samples}

The yields in the \mj, \mmj, \ej, \eej and \gj control samples can be found in
Tables~\ref{tab:yields_ewk_mu}, \ref{tab:yields_ewk_mumu},
and \ref{tab:yields_ewk_gj} respectively for 1.28 \ifb (yields for 3\ifb are in Appendix~\ref{app:yields3fb}). 
The number of events in each of these regions is important to determine how these analysis regions can be extended. We require to 
each control region to have sufficiently populated control regions to enable a robust data driven background prediction.


% \begin{table}[h!]
\tiny
\centering
\topcaption{Yields  in the \mj control region for 1.28\ifb. The letter ``a'' in jet \eg ``2a''  indicates the asymmetric jet bins.\label{tab:yields_ewk_mu}}
\begin{tabular}
{c|c|cccccccc}
	\hline\hline
   &     & \multicolumn{8}{c}{\scalht (\gev)} \\ 
	\njet & \nb & 200-250 & 250-300 & 300-350 & 350-400 & 400-500 & 500-600 & 600-800 & 800-$\infty$ \\ 
\hline
	1 & 0 & 1854.36 $\pm$17.36 & 679.51 $\pm$9.71 & 295.58 $\pm$6.38 & 141.43 $\pm$3.94 & 113.75 $\pm$2.79 & 36.96 $\pm$1.33 & 18.22 $\pm$0.56 & 4.04 $\pm$0.33 \\ 
	1 & 1 & 75.46 $\pm$3.39 & 25.54 $\pm$1.80 & 12.10 $\pm$1.24 & 6.00 $\pm$0.79 & 4.92 $\pm$0.60 & 1.38 $\pm$0.36 & 0.95 $\pm$0.30 & 0.18 $\pm$0.28 \\ 
	2 & 0 & 201.18 $\pm$5.36 & 293.99 $\pm$6.35 & 271.80 $\pm$5.97 & 244.18 $\pm$5.05 & 306.72 $\pm$4.39 & 147.99 $\pm$2.46 & 126.79 $\pm$1.31 & 64.94 $\pm$0.81 \\ 
	2 & 1 & 40.85 $\pm$2.22 & 44.20 $\pm$2.32 & 35.98 $\pm$2.09 & 28.70 $\pm$1.75 & 39.32 $\pm$1.83 & 18.51 $\pm$1.10 & 15.45 $\pm$0.86 & 8.02 $\pm$0.60 \\ 
	2 & 2 & 1.81 $\pm$0.48 & 3.54 $\pm$0.66 & 1.84 $\pm$0.48 & 2.75 $\pm$0.54 & 3.92 $\pm$0.71 & 1.72 $\pm$0.49 & 1.68 $\pm$0.45 & 0.48 $\pm$0.31 \\ 
	3 & 0 & 0.57 $\pm$0.33 & 90.37 $\pm$3.52 & 189.13 $\pm$4.91 & 216.36 $\pm$4.79 & 327.41 $\pm$4.82 & 191.98 $\pm$3.08 & 176.36 $\pm$1.89 & 104.36 $\pm$1.27 \\ 
	3 & 1 & 0.68 $\pm$0.34 & 47.38 $\pm$2.46 & 101.29 $\pm$3.54 & 98.15 $\pm$3.35 & 121.55 $\pm$3.52 & 62.61 $\pm$2.45 & 51.75 $\pm$2.04 & 25.28 $\pm$1.31 \\ 
	3 & 2 & 0.18 $\pm$0.29 & 10.90 $\pm$1.14 & 32.10 $\pm$1.98 & 35.32 $\pm$2.15 & 45.85 $\pm$2.32 & 19.21 $\pm$1.52 & 13.36 $\pm$1.20 & 4.81 $\pm$0.70 \\ 
	3 & $\ge3$ & 0.00 $\pm$0.28 & 1.25 $\pm$0.51 & 1.11 $\pm$0.44 & 0.54 $\pm$0.33 & 2.25 $\pm$0.61 & 0.94 $\pm$0.40 & 0.33 $\pm$0.30 & 0.27 $\pm$0.30 \\ 
	4 & 0 & - & 0.28 $\pm$0.30 & 25.98 $\pm$1.79 & 79.75 $\pm$3.00 & 199.36 $\pm$4.19 & 139.40 $\pm$2.94 & 149.81 $\pm$2.42 & 93.43 $\pm$1.42 \\ 
	4 & 1 & - & 0.36 $\pm$0.31 & 26.08 $\pm$1.78 & 72.73 $\pm$2.93 & 164.56 $\pm$4.35 & 99.19 $\pm$3.28 & 83.62 $\pm$2.93 & 40.36 $\pm$1.79 \\ 
	4 & 2 & - & 0.20 $\pm$0.29 & 10.82 $\pm$1.16 & 38.20 $\pm$2.16 & 84.62 $\pm$3.18 & 50.25 $\pm$2.43 & 35.95 $\pm$2.00 & 17.11 $\pm$1.37 \\ 
	4 & $\ge3$ & - & 0.00 $\pm$0.28 & 0.94 $\pm$0.37 & 2.71 $\pm$0.63 & 6.47 $\pm$0.90 & 4.82 $\pm$0.83 & 3.36 $\pm$0.69 & 1.63 $\pm$0.50 \\ 
	$\ge5$ & 0 & - & - & 0.23 $\pm$0.30 & 5.18 $\pm$0.79 & 55.66 $\pm$2.36 & 82.74 $\pm$2.70 & 117.77 $\pm$2.75 & 106.40 $\pm$2.21 \\ 
	$\ge5$ & 1 & - & - & 0.49 $\pm$0.33 & 7.09 $\pm$0.93 & 81.40 $\pm$3.18 & 112.34 $\pm$3.63 & 142.51 $\pm$3.98 & 106.25 $\pm$3.33 \\ 
	$\ge5$ & 2 & - & - & 0.18 $\pm$0.29 & 3.31 $\pm$0.65 & 47.88 $\pm$2.40 & 68.33 $\pm$2.81 & 94.40 $\pm$3.35 & 66.37 $\pm$2.79 \\ 
	$\ge5$ & $\ge3$ & - & - & 0.00 $\pm$0.28 & 0.60 $\pm$0.34 & 5.80 $\pm$0.86 & 8.74 $\pm$1.06 & 12.75 $\pm$1.25 & 10.58 $\pm$1.16 \\ 
	2a & 0 & 1965.92 $\pm$17.27 & 1023.73 $\pm$11.91 & 479.16 $\pm$7.92 & 208.56 $\pm$4.67 & 145.81 $\pm$3.11 & 39.84 $\pm$1.31 & 20.13 $\pm$0.61 & 4.03 $\pm$0.33 \\ 
	2a & 1 & 344.07 $\pm$6.66 & 141.22 $\pm$4.17 & 66.65 $\pm$2.85 & 24.84 $\pm$1.65 & 18.57 $\pm$1.31 & 4.85 $\pm$0.65 & 2.18 $\pm$0.38 & 0.48 $\pm$0.29 \\ 
	2a & 2 & 33.49 $\pm$2.01 & 15.80 $\pm$1.36 & 5.69 $\pm$0.78 & 2.49 $\pm$0.53 & 1.03 $\pm$0.37 & 0.41 $\pm$0.32 & 0.43 $\pm$0.31 & 0.18 $\pm$0.29 \\ 
	3a & 0 & 467.55 $\pm$8.27 & 620.76 $\pm$9.15 & 340.28 $\pm$6.70 & 170.95 $\pm$4.27 & 122.52 $\pm$2.89 & 31.20 $\pm$1.33 & 13.65 $\pm$0.62 & 2.98 $\pm$0.33 \\ 
	3a & 1 & 258.51 $\pm$5.63 & 333.58 $\pm$6.41 & 160.59 $\pm$4.40 & 61.95 $\pm$2.62 & 37.83 $\pm$1.92 & 8.46 $\pm$0.85 & 3.59 $\pm$0.57 & 0.30 $\pm$0.29 \\ 
	3a & 2 & 69.34 $\pm$2.86 & 98.89 $\pm$3.45 & 49.47 $\pm$2.40 & 17.97 $\pm$1.43 & 10.61 $\pm$1.11 & 2.45 $\pm$0.54 & 0.73 $\pm$0.33 & 0.09 $\pm$0.29 \\ 
	3a & $\ge3$ & 2.66 $\pm$0.62 & 3.37 $\pm$0.68 & 1.69 $\pm$0.50 & 0.43 $\pm$0.32 & 0.49 $\pm$0.32 & 0.21 $\pm$0.29 & 0.00 $\pm$0.28 & 0.00 $\pm$0.28 \\ 
	4a & 0 & - & 95.76 $\pm$3.53 & 173.49 $\pm$4.81 & 110.69 $\pm$3.57 & 82.12 $\pm$2.66 & 22.46 $\pm$1.22 & 8.09 $\pm$0.57 & 1.24 $\pm$0.30 \\ 
	4a & 1 & - & 89.87 $\pm$3.25 & 153.11 $\pm$4.28 & 96.55 $\pm$3.39 & 60.92 $\pm$2.67 & 10.24 $\pm$1.02 & 3.54 $\pm$0.61 & 0.69 $\pm$0.32 \\ 
	4a & 2 & - & 39.91 $\pm$2.20 & 73.06 $\pm$2.97 & 46.96 $\pm$2.36 & 25.99 $\pm$1.75 & 6.97 $\pm$0.95 & 1.42 $\pm$0.50 & 0.08 $\pm$0.28 \\ 
	4a & $\ge3$ & - & 2.00 $\pm$0.51 & 5.29 $\pm$0.81 & 3.82 $\pm$0.73 & 1.24 $\pm$0.45 & 0.32 $\pm$0.30 & 0.17 $\pm$0.29 & 0.00 $\pm$0.28 \\ 
	$\ge5$a & 0 & - & - & 13.12 $\pm$1.27 & 37.80 $\pm$2.14 & 56.15 $\pm$2.44 & 16.92 $\pm$1.19 & 7.40 $\pm$0.78 & 0.52 $\pm$0.29 \\ 
	$\ge5$a & 1 & - & - & 20.01 $\pm$1.58 & 54.74 $\pm$2.61 & 76.68 $\pm$3.00 & 21.91 $\pm$1.59 & 9.00 $\pm$1.03 & 0.75 $\pm$0.34 \\ 
	$\ge5$a & 2 & - & - & 8.95 $\pm$1.02 & 25.69 $\pm$1.77 & 47.49 $\pm$2.42 & 15.51 $\pm$1.38 & 4.11 $\pm$0.72 & 0.55 $\pm$0.33 \\ 
	$\ge5$a & $\ge3$ & - & - & 1.00 $\pm$0.46 & 3.28 $\pm$0.68 & 3.30 $\pm$0.67 & 2.02 $\pm$0.58 & 0.82 $\pm$0.36 & 0.01 $\pm$0.28 \\ 
	\hline
	\hline
\end{tabular}
\end{table}

% \newpage
% \begin{table}[h!]
\tiny
\centering
\topcaption{Yields  in the \mmj control region for 1.28\ifb. The letter ``a'' in jet \eg ``2a''  indicates the asymmetric jet bins.\label{tab:yields_ewk_mumu}}
\begin{tabular}
{c|c|cccccccc}
	\hline\hline
   &     & \multicolumn{8}{c}{\scalht (\gev)} \\ 
	\njet & \nb & 200-250 & 250-300 & 300-350 & 350-400 & 400-500 & 500-600 & 600-800 & 800-$\infty$ \\ 
\hline
	1 & 0 & 255.86 $\pm$4.95 & 102.05 $\pm$3.01 & 45.27 $\pm$1.94 & 22.98 $\pm$1.22 & 17.90 $\pm$0.70 & 5.70 $\pm$0.28 & 2.84 $\pm$0.20 & 0.61 $\pm$0.17 \\ 
	1 & 1 & 13.53 $\pm$1.12 & 5.25 $\pm$0.67 & 3.01 $\pm$0.54 & 1.11 $\pm$0.25 & 0.73 $\pm$0.19 & 0.21 $\pm$0.17 & 0.14 $\pm$0.17 & 0.05 $\pm$0.16 \\ 
	2 & 0 & 20.83 $\pm$1.38 & 30.86 $\pm$1.64 & 32.26 $\pm$1.64 & 26.94 $\pm$1.24 & 34.55 $\pm$0.79 & 18.50 $\pm$0.45 & 15.34 $\pm$0.32 & 8.88 $\pm$0.30 \\ 
	2 & 1 & 2.58 $\pm$0.50 & 3.52 $\pm$0.57 & 3.72 $\pm$0.57 & 3.07 $\pm$0.46 & 3.79 $\pm$0.31 & 1.92 $\pm$0.26 & 1.57 $\pm$0.21 & 0.82 $\pm$0.17 \\ 
	2 & 2 & 0.08 $\pm$0.17 & 0.38 $\pm$0.20 & 0.23 $\pm$0.19 & 0.49 $\pm$0.21 & 0.13 $\pm$0.17 & 0.07 $\pm$0.16 & 0.09 $\pm$0.16 & 0.02 $\pm$0.16 \\ 
	3 & 0 & 0.05 $\pm$0.17 & 6.74 $\pm$0.78 & 15.83 $\pm$1.14 & 20.68 $\pm$1.10 & 31.37 $\pm$0.79 & 20.17 $\pm$0.47 & 19.65 $\pm$0.36 & 12.91 $\pm$0.30 \\ 
	3 & 1 & 0.00 $\pm$0.16 & 0.77 $\pm$0.25 & 2.81 $\pm$0.53 & 2.59 $\pm$0.39 & 6.14 $\pm$0.58 & 3.07 $\pm$0.27 & 2.98 $\pm$0.25 & 1.88 $\pm$0.19 \\ 
	3 & 2 & 0.00 $\pm$0.16 & 0.28 $\pm$0.19 & 0.56 $\pm$0.22 & 1.08 $\pm$0.28 & 1.29 $\pm$0.35 & 0.72 $\pm$0.23 & 0.29 $\pm$0.17 & 0.30 $\pm$0.19 \\ 
	3 & $\ge3$ & 0.00 $\pm$0.16 & 0.01 $\pm$0.16 & 0.00 $\pm$0.16 & 0.08 $\pm$0.17 & 0.02 $\pm$0.16 & 0.00 $\pm$0.16 & 0.00 $\pm$0.16 & 0.01 $\pm$0.16 \\ 
	4 & 0 & - & 0.00 $\pm$0.16 & 2.98 $\pm$0.54 & 5.90 $\pm$0.67 & 15.94 $\pm$0.68 & 12.79 $\pm$0.42 & 13.63 $\pm$0.33 & 10.50 $\pm$0.29 \\ 
	4 & 1 & - & 0.00 $\pm$0.16 & 0.31 $\pm$0.19 & 1.11 $\pm$0.27 & 3.35 $\pm$0.44 & 3.05 $\pm$0.39 & 3.06 $\pm$0.27 & 2.00 $\pm$0.19 \\ 
	4 & 2 & - & 0.00 $\pm$0.16 & 0.00 $\pm$0.16 & 0.63 $\pm$0.24 & 1.37 $\pm$0.36 & 0.77 $\pm$0.23 & 0.66 $\pm$0.20 & 0.41 $\pm$0.19 \\ 
	4 & $\ge3$ & - & 0.00 $\pm$0.16 & 0.00 $\pm$0.16 & 0.00 $\pm$0.16 & 0.03 $\pm$0.17 & 0.02 $\pm$0.16 & 0.03 $\pm$0.16 & 0.03 $\pm$0.16 \\ 
	$\ge5$ & 0 & - & - & 0.00 $\pm$0.16 & 0.09 $\pm$0.17 & 2.91 $\pm$0.37 & 4.39 $\pm$0.29 & 7.96 $\pm$0.33 & 8.96 $\pm$0.28 \\ 
	$\ge5$ & 1 & - & - & 0.00 $\pm$0.16 & 0.03 $\pm$0.17 & 1.37 $\pm$0.29 & 1.85 $\pm$0.37 & 2.15 $\pm$0.27 & 2.87 $\pm$0.36 \\ 
	$\ge5$ & 2 & - & - & 0.00 $\pm$0.16 & 0.03 $\pm$0.17 & 0.51 $\pm$0.22 & 0.42 $\pm$0.19 & 0.89 $\pm$0.26 & 1.06 $\pm$0.26 \\ 
	$\ge5$ & $\ge3$ & - & - & 0.00 $\pm$0.16 & 0.00 $\pm$0.16 & 0.00 $\pm$0.16 & 0.07 $\pm$0.17 & 0.28 $\pm$0.20 & 0.05 $\pm$0.16 \\ 
	2a & 0 & 207.84 $\pm$4.48 & 116.59 $\pm$3.29 & 60.10 $\pm$2.27 & 28.77 $\pm$1.39 & 18.49 $\pm$0.60 & 6.07 $\pm$0.29 & 2.97 $\pm$0.21 & 0.64 $\pm$0.17 \\ 
	2a & 1 & 23.33 $\pm$1.50 & 11.52 $\pm$1.04 & 6.84 $\pm$0.81 & 2.38 $\pm$0.38 & 1.58 $\pm$0.22 & 0.85 $\pm$0.24 & 0.26 $\pm$0.17 & 0.06 $\pm$0.16 \\ 
	2a & 2 & 3.24 $\pm$0.60 & 1.60 $\pm$0.39 & 0.24 $\pm$0.19 & 0.02 $\pm$0.16 & 0.06 $\pm$0.16 & 0.02 $\pm$0.16 & 0.01 $\pm$0.16 & 0.00 $\pm$0.16 \\ 
	3a & 0 & 38.65 $\pm$1.91 & 55.65 $\pm$2.29 & 29.70 $\pm$1.61 & 17.04 $\pm$1.03 & 14.80 $\pm$0.60 & 3.91 $\pm$0.25 & 2.07 $\pm$0.21 & 0.34 $\pm$0.17 \\ 
	3a & 1 & 7.29 $\pm$0.87 & 7.64 $\pm$0.89 & 5.45 $\pm$0.71 & 2.81 $\pm$0.46 & 2.05 $\pm$0.27 & 0.61 $\pm$0.19 & 0.41 $\pm$0.19 & 0.04 $\pm$0.17 \\ 
	3a & 2 & 1.81 $\pm$0.43 & 3.51 $\pm$0.67 & 1.83 $\pm$0.48 & 0.49 $\pm$0.20 & 0.63 $\pm$0.23 & 0.06 $\pm$0.16 & 0.01 $\pm$0.16 & 0.00 $\pm$0.16 \\ 
	3a & $\ge3$ & 0.05 $\pm$0.17 & 0.32 $\pm$0.20 & 0.06 $\pm$0.17 & 0.01 $\pm$0.16 & 0.01 $\pm$0.16 & 0.00 $\pm$0.16 & 0.00 $\pm$0.16 & 0.00 $\pm$0.16 \\ 
	4a & 0 & - & 5.71 $\pm$0.70 & 10.53 $\pm$0.92 & 8.46 $\pm$0.72 & 6.80 $\pm$0.44 & 2.22 $\pm$0.23 & 1.03 $\pm$0.20 & 0.16 $\pm$0.17 \\ 
	4a & 1 & - & 1.55 $\pm$0.40 & 3.34 $\pm$0.60 & 1.80 $\pm$0.36 & 2.14 $\pm$0.39 & 0.47 $\pm$0.19 & 0.26 $\pm$0.18 & 0.01 $\pm$0.16 \\ 
	4a & 2 & - & 0.24 $\pm$0.19 & 0.79 $\pm$0.26 & 0.48 $\pm$0.21 & 1.12 $\pm$0.37 & 0.33 $\pm$0.20 & 0.13 $\pm$0.18 & 0.19 $\pm$0.19 \\ 
	4a & $\ge3$ & - & 0.00 $\pm$0.16 & 0.01 $\pm$0.16 & 0.00 $\pm$0.16 & 0.00 $\pm$0.16 & 0.00 $\pm$0.16 & 0.00 $\pm$0.16 & 0.00 $\pm$0.16 \\ 
	$\ge5$a & 0 & - & - & 0.85 $\pm$0.26 & 1.88 $\pm$0.42 & 2.25 $\pm$0.27 & 1.04 $\pm$0.19 & 0.44 $\pm$0.17 & 0.10 $\pm$0.16 \\ 
	$\ge5$a & 1 & - & - & 0.19 $\pm$0.18 & 0.52 $\pm$0.22 & 1.57 $\pm$0.36 & 0.64 $\pm$0.24 & 0.19 $\pm$0.17 & 0.16 $\pm$0.18 \\ 
	$\ge5$a & 2 & - & - & 0.00 $\pm$0.16 & 0.26 $\pm$0.19 & 0.29 $\pm$0.19 & 0.15 $\pm$0.17 & 0.23 $\pm$0.19 & 0.00 $\pm$0.16 \\ 
	$\ge5$a & $\ge3$ & - & - & 0.00 $\pm$0.16 & 0.00 $\pm$0.16 & 0.24 $\pm$0.19 & 0.00 $\pm$0.16 & 0.00 $\pm$0.16 & 0.00 $\pm$0.16 \\ 
	\hline
	\hline
\end{tabular}
\end{table}

% \newpage
% \begin{table}[h!]
\tiny
\centering
\topcaption{Yields  in the \gj control region for 3.0\ifb. The letter ``a'' in jet \eg ``2a''  indicates the asymmetric jet bins.\label{tab:yields_ewk_gj_3fb}}
\begin{tabular}
{c|c|cccc}
	\hline\hline
   &     & \multicolumn{4}{c}{\scalht (\gev)} \\ 
	\njet & \nb & 400-500 & 500-600 & 600-800 & 800-$\infty$ \\ 
\hline
	1 & 0 & 550.71 $\pm$16.32 & 178.61 $\pm$8.10 & 93.67 $\pm$4.12 & 23.91 $\pm$2.01 \\ 
	1 & 1 & 22.97 $\pm$3.15 & 7.33 $\pm$1.66 & 4.52 $\pm$1.08 & 1.27 $\pm$0.85 \\ 
	2 & 0 & 260.59 $\pm$10.97 & 109.70 $\pm$6.32 & 71.72 $\pm$3.61 & 166.92 $\pm$5.04 \\ 
	2 & 1 & 21.74 $\pm$3.15 & 9.19 $\pm$1.96 & 4.92 $\pm$1.10 & 16.55 $\pm$1.74 \\ 
	2 & 2 & 1.63 $\pm$0.94 & 0.50 $\pm$0.77 & 0.24 $\pm$0.74 & 0.46 $\pm$0.75 \\ 
	3 & 0 & 407.84 $\pm$14.07 & 180.73 $\pm$8.12 & 112.58 $\pm$4.34 & 241.25 $\pm$5.98 \\ 
	3 & 1 & 40.80 $\pm$4.24 & 19.85 $\pm$2.71 & 9.77 $\pm$1.36 & 31.79 $\pm$2.29 \\ 
	3 & 2 & 3.64 $\pm$1.36 & 1.86 $\pm$0.89 & 1.24 $\pm$0.84 & 2.95 $\pm$0.99 \\ 
	3 & $\ge3$ & 0.00 $\pm$0.73 & 0.00 $\pm$0.73 & 0.04 $\pm$0.73 & 0.19 $\pm$0.74 \\ 
	4 & 0 & 219.39 $\pm$10.06 & 143.80 $\pm$7.37 & 124.20 $\pm$4.95 & 193.89 $\pm$5.41 \\ 
	4 & 1 & 32.41 $\pm$3.88 & 21.32 $\pm$2.92 & 16.71 $\pm$1.76 & 31.18 $\pm$2.24 \\ 
	4 & 2 & 7.82 $\pm$1.93 & 3.41 $\pm$1.10 & 3.58 $\pm$1.01 & 4.71 $\pm$1.08 \\ 
	4 & $\ge3$ & 0.05 $\pm$0.73 & 0.28 $\pm$0.76 & 0.12 $\pm$0.73 & 0.00 $\pm$0.73 \\ 
	$\ge5$ & 0 & 47.09 $\pm$5.57 & 63.76 $\pm$5.12 & 84.96 $\pm$4.29 & 148.69 $\pm$4.75 \\ 
	$\ge5$ & 1 & 8.28 $\pm$1.85 & 15.79 $\pm$2.54 & 21.15 $\pm$2.32 & 37.90 $\pm$2.46 \\ 
	$\ge5$ & 2 & 1.46 $\pm$0.91 & 2.67 $\pm$1.03 & 4.29 $\pm$1.14 & 6.95 $\pm$1.23 \\ 
	$\ge5$ & $\ge3$ & 0.00 $\pm$0.73 & 0.00 $\pm$0.73 & 0.12 $\pm$0.73 & 0.50 $\pm$0.75 \\ 
	2a & 0 & 128.17 $\pm$7.78 & 38.44 $\pm$3.83 & 21.99 $\pm$2.21 & 21.05 $\pm$1.85 \\ 
	2a & 1 & 7.35 $\pm$1.77 & 2.80 $\pm$1.01 & 0.66 $\pm$0.76 & 1.92 $\pm$0.90 \\ 
	2a & 2 & 0.00 $\pm$0.73 & 0.00 $\pm$0.73 & 0.00 $\pm$0.73 & 0.12 $\pm$0.73 \\ 
	3a & 0 & 81.16 $\pm$6.37 & 18.57 $\pm$2.62 & 6.75 $\pm$1.22 & 11.16 $\pm$1.46 \\ 
	3a & 1 & 11.01 $\pm$2.37 & 2.36 $\pm$0.96 & 1.31 $\pm$0.85 & 0.85 $\pm$0.78 \\ 
	3a & 2 & 1.54 $\pm$0.93 & 0.35 $\pm$0.77 & 0.09 $\pm$0.73 & 0.00 $\pm$0.73 \\ 
	3a & $\ge3$ & 0.00 $\pm$0.73 & 0.00 $\pm$0.73 & 0.00 $\pm$0.73 & 0.00 $\pm$0.73 \\ 
	4a & 0 & 90.41 $\pm$6.86 & 15.41 $\pm$2.55 & 4.62 $\pm$1.14 & 3.65 $\pm$1.05 \\ 
	4a & 1 & 12.25 $\pm$2.38 & 3.84 $\pm$1.40 & 0.58 $\pm$0.76 & 0.41 $\pm$0.75 \\ 
	4a & 2 & 2.04 $\pm$1.00 & 0.00 $\pm$0.73 & 0.00 $\pm$0.73 & 0.00 $\pm$0.73 \\ 
	4a & $\ge3$ & 0.13 $\pm$0.75 & 0.00 $\pm$0.73 & 0.22 $\pm$0.74 & 0.00 $\pm$0.73 \\ 
	$\ge5$a & 0 & 50.83 $\pm$5.71 & 10.89 $\pm$2.15 & 4.42 $\pm$1.19 & 0.90 $\pm$0.78 \\ 
	$\ge5$a & 1 & 4.85 $\pm$1.46 & 3.51 $\pm$1.37 & 0.72 $\pm$0.77 & 0.21 $\pm$0.74 \\ 
	$\ge5$a & 2 & 2.27 $\pm$1.04 & 0.14 $\pm$0.74 & 0.00 $\pm$0.73 & 0.00 $\pm$0.73 \\ 
	$\ge5$a & $\ge3$ & 0.00 $\pm$0.73 & 0.00 $\pm$0.73 & 0.00 $\pm$0.73 & 0.00 $\pm$0.73 \\ 
	\hline
	\hline
\end{tabular}
\end{table}

% \newpage

%%____________________________________________________________________________||
\subsection{Signal acceptance from asymmetric jet \Pt thresholds}

For a typical compressed model, Fig.~\ref{fig:asymMotivation} shows the second jet \PT
distribution for two different \HT bins. In the low \HT case a large portion of
the events are killed by requiring the second jet to have $\ET>100\gev$. 

We therefore add an new analysis category where the leading jet is required to fulfil 
momenta of jet $\ET>100\gev$ and sub-leading jet $40\gev<\ET<100\gev$. This new category 
results in new asymmetric jet bins split also in $\njet$, $\nb$ and \HT. The asymmetric jet bin
results in much larger acceptance for monojet-type and ISR induced event topologies like DM signals
and compresses SUSY scenarios. 
For the simplified model T2tt with $m_{\rm stop}=425\gev$ and $m_{\rm LSP}=325\gev$, 
including these bins increases signal acceptance by around a factor 3, for the DM models a factor of five is observed.
The addition of the asymmetric bin can only lead in combination to improvements or constant sensitivities.
\begin{figure}[h!]
  \centering
  \subfigure[Second jet \PT for $200\gev<\HT<250\gev$, $\alphat>0.65$]{
    \includegraphics[width=0.5\textwidth]{figures/asymPlots/secondJetPtlowHT}
  }~~
  \subfigure[Second jet \PT for $400\gev<\HT>500\gev$, $\alphat>0.52$]{
    \includegraphics[width=0.5\textwidth]{figures/asymPlots/secondJetPthigherHT}
  }
  \\
  \caption{\label{fig:asymMotivation} The second leading jet \PT for different
  cases of \HT after a baseline signal selection: $\njet\geq2$, lead jet
  $\ET>100\gev$, lepton vetoes. Made with the T2tt ($m_{\rm
    stop}=425\gev$, $m_{\rm LSP}=325\gev$) simplified model sample.}
\end{figure}

