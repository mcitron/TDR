%%____________________________________________________________________________||
\section{Data sets}
\label{sec:datasets}

\subsection{Data}


In this note, we use 2.1~\ifb of proton-proton collision data at
$\sqrt{s} =$ 13~TeV collected in 2015. The JSON file listed in
Table~\ref{tab:cert_json} specifies the periods of the time in which
these certified data are collected. Table~\ref{tab:datasets_data} lists
the names of the data sets relevant for the analysis, along with the
integrated luminosities for each data set.

We perform a blind analysis. In the previous version of the note, the
data in the signal region is blinded with the exception of 149.48~\ipb
of the certified data specified by the JSON file in
Table~\ref{tab:cert_unblind_json}. Data in the control regions is
never blinded. These blinding policies are agree in the SUS PAG.

In the review of the previous version of the note and related
presentations in the SUS PAG, it becomes clear that we understand the
data in the control regions and a small fraction of the unblinded data
in the signal region well enough to predict the backgrounds in the
signal region. As a result, no data in the signal region is blinded in
this version of the note.

\begin{table}[!h]
\topcaption{The JSON file that specifies the periods of the time in
which 2.1~\ifb of the certified data are collected} \footnotesize
%latex.default(d, title = NULL, booktabs = FALSE, width = 3, rowname = NULL,     helvetica = FALSE, caption.loc = "bottom", ...)%
\begin{center}
\begin{tabular}{c}
\hline\hline
\verb!Cert_246908-258750_13TeV_PromptReco_Collisions15_25ns_JSON.txt!\tabularnewline
\hline
\end{tabular}\end{center}
 \label{tab:cert_json}
\end{table}

\begin{table}[!h]
\topcaption{Data sets}
\footnotesize %latex.default(d, title = NULL, booktabs = FALSE, width = 3, rowname = NULL,     helvetica = FALSE, caption.loc = "bottom", ...)%
\begin{center}
\begin{tabular}{lr}
\hline\hline
\multicolumn{1}{c}{Data set}&\multicolumn{1}{c}{$\int\mathcal{L}\textrm{d}t [\textrm{pb}^{-1}]$}\tabularnewline
\hline
\verb!/HTMHT/Run2015D-05Oct2015-v1/MINIAOD! &$ 551.60$\tabularnewline
\verb!/HTMHT/Run2015D-PromptReco-v4/MINIAOD! &$1599.66$\tabularnewline
\verb!/JetHT/Run2015D-05Oct2015-v1/MINIAOD! &$ 552.67$\tabularnewline
\verb!/JetHT/Run2015D-PromptReco-v4/MINIAOD! &$1599.66$\tabularnewline
\verb!/MET/Run2015D-05Oct2015-v1/MINIAOD! &$ 552.67$\tabularnewline
\verb!/MET/Run2015D-PromptReco-v4/MINIAOD! &$1599.66$\tabularnewline
\verb!/SingleElectron/Run2015D-05Oct2015-v1/MINIAOD! &$ 552.63$\tabularnewline
\verb!/SingleElectron/Run2015D-PromptReco-v4/MINIAOD! &$1599.11$\tabularnewline
\verb!/SingleMuon/Run2015D-05Oct2015-v1/MINIAOD! &$ 552.67$\tabularnewline
\verb!/SingleMuon/Run2015D-PromptReco-v4/MINIAOD! &$1599.53$\tabularnewline
\verb!/SinglePhoton/Run2015D-05Oct2015-v1/MINIAOD! &$ 552.67$\tabularnewline
\verb!/SinglePhoton/Run2015D-PromptReco-v4/MINIAOD! &$1598.83$\tabularnewline
\hline
\end{tabular}\end{center}

\label{tab:datasets_data}
\end{table}

\begin{table}[!h]
\topcaption{The JSON file specifying the 149.48~\ipb of the certified
data that we never blind} \footnotesize
\input{tables/datasets/cert_unblind_json.tex}
\label{tab:cert_unblind_json}
\end{table}

\subsection{Simulation}

Table~\ref{tab:datasets_bkg} lists the data sets of simulated events
of the standard model background processes used in this note. Tables
from \ref{tab:datasets_SMS_T1qqqq} to \ref{tab:datasets_SMS_T1bbbb}
lists those of the signal processes. In these data sets, in addition
to the main interaction, each event contains on average 20 minimum
bias interactions which simulate multiple interactions per
bunch-crossing (in-time pileup). The expected detector signal from
previous or following bunch crossings (out-of-time pileup) with 25ns
bunch spacing is overlapped.

\begin{table}[!h]
 \centering
 \topcaption{Simulated background samples}
 \scriptsize
 \scalebox{.7}[1.0]{%latex.default(d, title = NULL, booktabs = FALSE, width = 3, rowname = NULL,     helvetica = FALSE, caption.loc = "bottom", ...)%
\begin{center}
\begin{tabular}{ll}
\hline\hline
\multicolumn{1}{c}{Data set}&\multicolumn{1}{c}{Cross section [pb]}\tabularnewline
\hline
\verb!/TT_TuneCUETP8M1_13TeV-powheg-pythia8/RunIISpring16MiniAODv2-PUSpring16_80X_mcRun2_asymptotic_2016_miniAODv2_v0_ext4-v1/MINIAODSIM! &$8.318\times 10^{+02}$\tabularnewline
\verb!/TTJets_HT-600to800_TuneCUETP8M1_13TeV-madgraphMLM-pythia8/RunIISpring16MiniAODv2-PUSpring16_80X_mcRun2_asymptotic_2016_miniAODv2_v0_ext1-v1/MINIAODSIM! &$2.667\times 10^{+00}$\tabularnewline
\verb!/TTJets_HT-800to1200_TuneCUETP8M1_13TeV-madgraphMLM-pythia8/RunIISpring16MiniAODv2-PUSpring16_80X_mcRun2_asymptotic_2016_miniAODv2_v0_ext1-v1/MINIAODSIM! &$1.098\times 10^{+00}$\tabularnewline
\verb!/TTJets_HT-1200to2500_TuneCUETP8M1_13TeV-madgraphMLM-pythia8/RunIISpring16MiniAODv2-PUSpring16_80X_mcRun2_asymptotic_2016_miniAODv2_v0_ext1-v1/MINIAODSIM! &$1.987\times 10^{-01}$\tabularnewline
\verb!/TTJets_HT-2500toInf_TuneCUETP8M1_13TeV-madgraphMLM-pythia8/RunIISpring16MiniAODv2-PUSpring16_80X_mcRun2_asymptotic_2016_miniAODv2_v0-v1/MINIAODSIM! &$2.368\times 10^{-03}$\tabularnewline
\verb!/TTJets_SingleLeptFromT_TuneCUETP8M1_13TeV-madgraphMLM-pythia8/RunIISpring16MiniAODv2-PUSpring16_80X_mcRun2_asymptotic_2016_miniAODv2_v0-v1/MINIAODSIM! &$1.827\times 10^{+02}$\tabularnewline
\verb!/TTJets_SingleLeptFromTbar_TuneCUETP8M1_13TeV-madgraphMLM-pythia8/RunIISpring16MiniAODv2-PUSpring16_80X_mcRun2_asymptotic_2016_miniAODv2_v0-v1/MINIAODSIM! &$1.827\times 10^{+02}$\tabularnewline
\verb!/TTJets_SingleLeptFromTbar_TuneCUETP8M1_13TeV-madgraphMLM-pythia8/RunIISpring16MiniAODv2-PUSpring16_80X_mcRun2_asymptotic_2016_miniAODv2_v0_ext1-v1/MINIAODSIM! &$1.827\times 10^{+02}$\tabularnewline
\verb!/TTJets_DiLept_TuneCUETP8M1_13TeV-madgraphMLM-pythia8/RunIISpring16MiniAODv2-PUSpring16_80X_mcRun2_asymptotic_2016_miniAODv2_v0_ext1-v1/MINIAODSIM! &$8.829\times 10^{+01}$\tabularnewline
\verb!/WJetsToLNu_TuneCUETP8M1_13TeV-madgraphMLM-pythia8/RunIISpring16MiniAODv1-PUSpring16_80X_mcRun2_asymptotic_2016_v3-v2/MINIAODSIM! &$6.153\times 10^{+04}$\tabularnewline
\verb!/WJetsToLNu_HT-100To200_TuneCUETP8M1_13TeV-madgraphMLM-pythia8/RunIISpring16MiniAODv2-PUSpring16_80X_mcRun2_asymptotic_2016_miniAODv2_v0_ext1-v1/MINIAODSIM! &$1.627\times 10^{+03}$\tabularnewline
\verb!/WJetsToLNu_HT-200To400_TuneCUETP8M1_13TeV-madgraphMLM-pythia8/RunIISpring16MiniAODv2-PUSpring16_80X_mcRun2_asymptotic_2016_miniAODv2_v0_ext1-v1/MINIAODSIM! &$4.352\times 10^{+02}$\tabularnewline
\verb!/WJetsToLNu_HT-400To600_TuneCUETP8M1_13TeV-madgraphMLM-pythia8/RunIISpring16MiniAODv2-PUSpring16_80X_mcRun2_asymptotic_2016_miniAODv2_v0-v1/MINIAODSIM! &$5.918\times 10^{+01}$\tabularnewline
\verb!/WJetsToLNu_HT-600To800_TuneCUETP8M1_13TeV-madgraphMLM-pythia8/RunIISpring16MiniAODv2-PUSpring16_80X_mcRun2_asymptotic_2016_miniAODv2_v0-v1/MINIAODSIM! &$1.458\times 10^{+01}$\tabularnewline
\verb!/WJetsToLNu_HT-800To1200_TuneCUETP8M1_13TeV-madgraphMLM-pythia8/RunIISpring16MiniAODv2-PUSpring16_80X_mcRun2_asymptotic_2016_miniAODv2_v0_ext1-v1/MINIAODSIM! &$6.656\times 10^{+00}$\tabularnewline
\verb!/WJetsToLNu_HT-1200To2500_TuneCUETP8M1_13TeV-madgraphMLM-pythia8/RunIISpring16MiniAODv2-PUSpring16_80X_mcRun2_asymptotic_2016_miniAODv2_v0-v1/MINIAODSIM! &$1.608\times 10^{+00}$\tabularnewline
\verb!/WJetsToLNu_HT-2500ToInf_TuneCUETP8M1_13TeV-madgraphMLM-pythia8/RunIISpring16MiniAODv2-PUSpring16_80X_mcRun2_asymptotic_2016_miniAODv2_v0-v1/MINIAODSIM! &$3.891\times 10^{-02}$\tabularnewline
\verb!/ZJetsToNuNu_HT-100To200_13TeV-madgraph/RunIISpring15DR74-Asympt25ns_MCRUN2_74_V9-v1/MINIAODSIM! &$3.450\times 10^{+02}$\tabularnewline
\verb!/ZJetsToNuNu_HT-200To400_13TeV-madgraph/RunIISpring15DR74-Asympt25ns_MCRUN2_74_V9-v1/MINIAODSIM! &$9.638\times 10^{+01}$\tabularnewline
\verb!/ZJetsToNuNu_HT-400To600_13TeV-madgraph/RunIISpring15DR74-Asympt25ns_MCRUN2_74_V9-v1/MINIAODSIM! &$1.346\times 10^{+01}$\tabularnewline
\verb!/ZJetsToNuNu_HT-600ToInf_13TeV-madgraph/RunIISpring15DR74-Asympt25ns_MCRUN2_74_V9-v1/MINIAODSIM! &$5.170\times 10^{+00}$\tabularnewline
\verb!/QCD_HT300to500_TuneCUETP8M1_13TeV-madgraphMLM-pythia8/RunIISpring16MiniAODv2-PUSpring16_80X_mcRun2_asymptotic_2016_miniAODv2_v0_ext1-v1/MINIAODSIM! &$3.477\times 10^{+05}$\tabularnewline
\verb!/QCD_HT700to1000_TuneCUETP8M1_13TeV-madgraphMLM-pythia8/RunIISpring16MiniAODv2-PUSpring16_80X_mcRun2_asymptotic_2016_miniAODv2_v0-v1/MINIAODSIM! &$6.831\times 10^{+03}$\tabularnewline
\verb!/QCD_HT700to1000_TuneCUETP8M1_13TeV-madgraphMLM-pythia8/RunIISpring16MiniAODv2-PUSpring16_80X_mcRun2_asymptotic_2016_miniAODv2_v0_ext1-v1/MINIAODSIM! &$6.831\times 10^{+03}$\tabularnewline
\verb!/QCD_HT1000to1500_TuneCUETP8M1_13TeV-madgraphMLM-pythia8/RunIISpring16MiniAODv2-PUSpring16_80X_mcRun2_asymptotic_2016_miniAODv2_v0-v2/MINIAODSIM! &$1.207\times 10^{+03}$\tabularnewline
\verb!/QCD_HT1000to1500_TuneCUETP8M1_13TeV-madgraphMLM-pythia8/RunIISpring16MiniAODv2-PUSpring16_80X_mcRun2_asymptotic_2016_miniAODv2_v0_ext1-v1/MINIAODSIM! &$1.207\times 10^{+03}$\tabularnewline
\verb!/QCD_HT1500to2000_TuneCUETP8M1_13TeV-madgraphMLM-pythia8/RunIISpring16MiniAODv2-PUSpring16_80X_mcRun2_asymptotic_2016_miniAODv2_v0-v3/MINIAODSIM! &$1.199\times 10^{+02}$\tabularnewline
\verb!/QCD_HT1500to2000_TuneCUETP8M1_13TeV-madgraphMLM-pythia8/RunIISpring16MiniAODv2-PUSpring16_80X_mcRun2_asymptotic_2016_miniAODv2_v0_ext1-v1/MINIAODSIM! &$1.199\times 10^{+02}$\tabularnewline
\verb!/QCD_HT2000toInf_TuneCUETP8M1_13TeV-madgraphMLM-pythia8/RunIISpring16MiniAODv2-PUSpring16_80X_mcRun2_asymptotic_2016_miniAODv2_v0-v1/MINIAODSIM! &$2.524\times 10^{+01}$\tabularnewline
\verb!/QCD_HT2000toInf_TuneCUETP8M1_13TeV-madgraphMLM-pythia8/RunIISpring16MiniAODv2-PUSpring16_80X_mcRun2_asymptotic_2016_miniAODv2_v0_ext1-v1/MINIAODSIM! &$2.524\times 10^{+01}$\tabularnewline
\verb!/QCD_HT100to200_TuneCUETP8M1_13TeV-madgraphMLM-pythia8/RunIISpring15DR74-Asympt25ns_MCRUN2_74_V9-v2/MINIAODSIM! &$2.785\times 10^{+07}$\tabularnewline
\verb!/QCD_HT200to300_TuneCUETP8M1_13TeV-madgraphMLM-pythia8/RunIISpring15DR74-Asympt25ns_MCRUN2_74_V9-v2/MINIAODSIM! &$1.717\times 10^{+06}$\tabularnewline
\verb!/QCD_HT300to500_TuneCUETP8M1_13TeV-madgraphMLM-pythia8/RunIISpring15DR74-Asympt25ns_MCRUN2_74_V9-v2/MINIAODSIM! &$3.513\times 10^{+05}$\tabularnewline
\verb!/QCD_HT500to700_TuneCUETP8M1_13TeV-madgraphMLM-pythia8/RunIISpring15DR74-Asympt25ns_MCRUN2_74_V9-v1/MINIAODSIM! &$3.163\times 10^{+04}$\tabularnewline
\verb!/QCD_HT700to1000_TuneCUETP8M1_13TeV-madgraphMLM-pythia8/RunIISpring15DR74-Asympt25ns_MCRUN2_74_V9-v1/MINIAODSIM! &$6.802\times 10^{+03}$\tabularnewline
\verb!/QCD_HT1000to1500_TuneCUETP8M1_13TeV-madgraphMLM-pythia8/RunIISpring15DR74-Asympt25ns_MCRUN2_74_V9-v2/MINIAODSIM! &$1.206\times 10^{+03}$\tabularnewline
\verb!/QCD_HT1500to2000_TuneCUETP8M1_13TeV-madgraphMLM-pythia8/RunIISpring15DR74-Asympt25ns_MCRUN2_74_V9-v1/MINIAODSIM! &$1.204\times 10^{+02}$\tabularnewline
\verb!/QCD_HT2000toInf_TuneCUETP8M1_13TeV-madgraphMLM-pythia8/RunIISpring15DR74-Asympt25ns_MCRUN2_74_V9-v1/MINIAODSIM! &$2.525\times 10^{+01}$\tabularnewline
\verb!/DYJetsToLL_M-50_TuneCUETP8M1_13TeV-amcatnloFXFX-pythia8/RunIISpring16MiniAODv2-PUSpring16_80X_mcRun2_asymptotic_2016_miniAODv2_v0-v1/MINIAODSIM! &$6.025\times 10^{+03}$\tabularnewline
\verb!/DYJetsToLL_M-50_TuneCUETP8M1_13TeV-madgraphMLM-pythia8/RunIISpring16MiniAODv2-PUSpring16_80X_mcRun2_asymptotic_2016_miniAODv2_v0_ext1-v1/MINIAODSIM! &$6.025\times 10^{+03}$\tabularnewline
\verb!/DYJetsToLL_M-50_HT-100to200_TuneCUETP8M1_13TeV-madgraphMLM-pythia8/RunIISpring16MiniAODv2-PUSpring16_80X_mcRun2_asymptotic_2016_miniAODv2_v0_ext1-v1/MINIAODSIM! &$1.813\times 10^{+02}$\tabularnewline
\verb!/DYJetsToLL_M-50_HT-200to400_TuneCUETP8M1_13TeV-madgraphMLM-pythia8/RunIISpring16MiniAODv2-PUSpring16_80X_mcRun2_asymptotic_2016_miniAODv2_v0_ext1-v1/MINIAODSIM! &$5.042\times 10^{+01}$\tabularnewline
\verb!/DYJetsToLL_M-50_HT-400to600_TuneCUETP8M1_13TeV-madgraphMLM-pythia8/RunIISpring16MiniAODv2-PUSpring16_80X_mcRun2_asymptotic_2016_miniAODv2_v0_ext1-v1/MINIAODSIM! &$6.984\times 10^{+00}$\tabularnewline
\verb!/DYJetsToLL_M-50_HT-600toInf_TuneCUETP8M1_13TeV-madgraphMLM-pythia8/RunIISpring16MiniAODv2-PUSpring16_80X_mcRun2_asymptotic_2016_miniAODv2_v0-v1/MINIAODSIM! &$2.704\times 10^{+00}$\tabularnewline
\verb!/DYJetsToLL_M-50_HT-600toInf_TuneCUETP8M1_13TeV-madgraphMLM-pythia8/RunIISpring16MiniAODv2-PUSpring16_80X_mcRun2_asymptotic_2016_miniAODv2_v0_ext1-v1/MINIAODSIM! &$2.704\times 10^{+00}$\tabularnewline
\verb!/GJets_HT-100To200_TuneCUETP8M1_13TeV-madgraphMLM-pythia8/RunIISpring16MiniAODv2-PUSpring16_80X_mcRun2_asymptotic_2016_miniAODv2_v0-v4/MINIAODSIM! &$9.238\times 10^{+03}$\tabularnewline
\verb!/GJets_HT-200To400_TuneCUETP8M1_13TeV-madgraphMLM-pythia8/RunIISpring16MiniAODv2-PUSpring16_80X_mcRun2_asymptotic_2016_miniAODv2_v0-v1/MINIAODSIM! &$2.305\times 10^{+03}$\tabularnewline
\verb!/GJets_HT-400To600_TuneCUETP8M1_13TeV-madgraphMLM-pythia8/RunIISpring16MiniAODv2-PUSpring16_80X_mcRun2_asymptotic_2016_miniAODv2_v0-v1/MINIAODSIM! &$2.744\times 10^{+02}$\tabularnewline
\verb!/GJets_HT-600ToInf_TuneCUETP8M1_13TeV-madgraphMLM-pythia8/RunIISpring16MiniAODv2-PUSpring16_80X_mcRun2_asymptotic_2016_miniAODv2_v0-v1/MINIAODSIM! &$9.346\times 10^{+01}$\tabularnewline
\verb!/ttHJetToNonbb_M125_13TeV_amcatnloFXFX_madspin_pythia8_mWCutfix/RunIISpring16MiniAODv1-PUSpring16RAWAODSIM_80X_mcRun2_asymptotic_2016_v3_ext1-v1/MINIAODSIM! &$2.151\times 10^{-01}$\tabularnewline
\verb!/ttHJetTobb_M125_13TeV_amcatnloFXFX_madspin_pythia8/RunIISpring16MiniAODv1-PUSpring16RAWAODSIM_80X_mcRun2_asymptotic_2016_v3_ext3-v1/MINIAODSIM! &$2.934\times 10^{-01}$\tabularnewline
\verb!/TTGJets_TuneCUETP8M1_13TeV-amcatnloFXFX-madspin-pythia8/RunIISpring16MiniAODv2-PUSpring16_80X_mcRun2_asymptotic_2016_miniAODv2_v0-v1/MINIAODSIM! &$3.697\times 10^{+00}$\tabularnewline
\verb!/TTWJetsToLNu_TuneCUETP8M1_13TeV-amcatnloFXFX-madspin-pythia8/RunIISpring16MiniAODv2-PUSpring16_80X_mcRun2_asymptotic_2016_miniAODv2_v0-v1/MINIAODSIM! &$2.043\times 10^{-01}$\tabularnewline
\verb!/TTWJetsToQQ_TuneCUETP8M1_13TeV-amcatnloFXFX-madspin-pythia8/RunIISpring16MiniAODv2-PUSpring16_80X_mcRun2_asymptotic_2016_miniAODv2_v0-v1/MINIAODSIM! &$4.062\times 10^{-01}$\tabularnewline
\verb!/TTZToLLNuNu_M-10_TuneCUETP8M1_13TeV-amcatnlo-pythia8/RunIISpring16MiniAODv2-PUSpring16_80X_mcRun2_asymptotic_2016_miniAODv2_v0-v1/MINIAODSIM! &$2.529\times 10^{-01}$\tabularnewline
\verb!/TTZToQQ_TuneCUETP8M1_13TeV-amcatnlo-pythia8/RunIISpring16MiniAODv2-PUSpring16_80X_mcRun2_asymptotic_2016_miniAODv2_v0-v1/MINIAODSIM! &$5.297\times 10^{-01}$\tabularnewline
\verb!/WW_TuneCUETP8M1_13TeV-pythia8/RunIISpring16MiniAODv2-PUSpring16_80X_mcRun2_asymptotic_2016_miniAODv2_v0-v1/MINIAODSIM! &$1.139\times 10^{+02}$\tabularnewline
\verb!/WZ_TuneCUETP8M1_13TeV-pythia8/RunIISpring16MiniAODv2-PUSpring16_80X_mcRun2_asymptotic_2016_miniAODv2_v0-v1/MINIAODSIM! &$4.713\times 10^{+01}$\tabularnewline
\verb!/ZZ_TuneCUETP8M1_13TeV-pythia8/RunIISpring16MiniAODv2-PUSpring16_80X_mcRun2_asymptotic_2016_miniAODv2_v0-v1/MINIAODSIM! &$1.652\times 10^{+01}$\tabularnewline
\verb!/ST_s-channel_4f_leptonDecays_13TeV-amcatnlo-pythia8_TuneCUETP8M1/RunIISpring16MiniAODv2-PUSpring16_80X_mcRun2_asymptotic_2016_miniAODv2_v0-v1/MINIAODSIM! &$3.681\times 10^{+00}$\tabularnewline
\verb!/ST_tW_antitop_5f_inclusiveDecays_13TeV-powheg-pythia8_TuneCUETP8M1/RunIISpring16MiniAODv2-PUSpring16_80X_mcRun2_asymptotic_2016_miniAODv2_v0-v1/MINIAODSIM! &$3.560\times 10^{+01}$\tabularnewline
\verb!/ST_tW_top_5f_inclusiveDecays_13TeV-powheg-pythia8_TuneCUETP8M1/RunIISpring16MiniAODv2-PUSpring16_80X_mcRun2_asymptotic_2016_miniAODv2_v0-v2/MINIAODSIM! &$3.560\times 10^{+01}$\tabularnewline
\hline
\end{tabular}\end{center}
}
 \label{tab:datasets_bkg}
\end{table}

\begin{table}[!p]
 \centering
 \topcaption{Simulated signal samples: SMS T1qqqq}
 \scriptsize
 \scalebox{.7}[1.0]{%latex.default(d, title = NULL, booktabs = FALSE, width = 3, rowname = NULL,     helvetica = FALSE, caption.loc = "bottom", ...)%
\begin{center}
\begin{tabular}{l}
\hline\hline
\multicolumn{1}{c}{Data set}\tabularnewline
\hline
\verb!/SMS-T1qqqq_mGluino-600_mLSP-1to300_TuneCUETP8M1_13TeV-madgraphMLM-pythia8/RunIISpring15FSPremix-MCRUN2_74_V9-v1/MINIAODSIM! \tabularnewline
\verb!/SMS-T1qqqq_mGluino-600_mLSP-400to475_TuneCUETP8M1_13TeV-madgraphMLM-pythia8/RunIISpring15FSPremix-MCRUN2_74_V9-v1/MINIAODSIM! \tabularnewline
\verb!/SMS-T1qqqq_mGluino-600_mLSP-500to575_TuneCUETP8M1_13TeV-madgraphMLM-pythia8/RunIISpring15FSPremix-MCRUN2_74_V9-v1/MINIAODSIM! \tabularnewline
\verb!/SMS-T1qqqq_mGluino-625_mLSP-425to525_TuneCUETP8M1_13TeV-madgraphMLM-pythia8/RunIISpring15FSPremix-MCRUN2_74_V9-v1/MINIAODSIM! \tabularnewline
\verb!/SMS-T1qqqq_mGluino-625to650_mLSP-475to550_TuneCUETP8M1_13TeV-madgraphMLM-pythia8/RunIISpring15FSPremix-MCRUN2_74_V9-v1/MINIAODSIM! \tabularnewline
\verb!/SMS-T1qqqq_mGluino-675_mLSP-500to650_TuneCUETP8M1_13TeV-madgraphMLM-pythia8/RunIISpring15FSPremix-MCRUN2_74_V9-v1/MINIAODSIM! \tabularnewline
\verb!/SMS-T1qqqq_mGluino-750to775_mLSP-450to750_TuneCUETP8M1_13TeV-madgraphMLM-pythia8/RunIISpring15FSPremix-MCRUN2_74_V9-v1/MINIAODSIM! \tabularnewline
\verb!/SMS-T1qqqq_mGluino-800to825_mLSP-1to750_TuneCUETP8M1_13TeV-madgraphMLM-pythia8/RunIISpring15FSPremix-MCRUN2_74_V9-v1/MINIAODSIM! \tabularnewline
\verb!/SMS-T1qqqq_mGluino-825to875_mLSP-400to825_TuneCUETP8M1_13TeV-madgraphMLM-pythia8/RunIISpring15FSPremix-MCRUN2_74_V9-v1/MINIAODSIM! \tabularnewline
\verb!/SMS-T1qqqq_mGluino-1000to1025_mLSP-1to1000_TuneCUETP8M1_13TeV-madgraphMLM-pythia8/RunIISpring15FSPremix-MCRUN2_74_V9-v1/MINIAODSIM! \tabularnewline
\verb!/SMS-T1qqqq_mGluino-1025to1075_mLSP-400to1050_TuneCUETP8M1_13TeV-madgraphMLM-pythia8/RunIISpring15FSPremix-MCRUN2_74_V9-v1/MINIAODSIM! \tabularnewline
\verb!/SMS-T1qqqq_mGluino-1075_mLSP-925to975_TuneCUETP8M1_13TeV-madgraphMLM-pythia8/RunIISpring15FSPremix-MCRUN2_74_V9-v1/MINIAODSIM! \tabularnewline
\verb!/SMS-T1qqqq_mGluino-1100to1125_mLSP-1to1100_TuneCUETP8M1_13TeV-madgraphMLM-pythia8/RunIISpring15FSPremix-MCRUN2_74_V9-v1/MINIAODSIM! \tabularnewline
\verb!/SMS-T1qqqq_mGluino-1125to1175_mLSP-925to1125_TuneCUETP8M1_13TeV-madgraphMLM-pythia8/RunIISpring15FSPremix-MCRUN2_74_V9-v1/MINIAODSIM! \tabularnewline
\verb!/SMS-T1qqqq_mGluino-1225to1250_mLSP-1to1225_TuneCUETP8M1_13TeV-madgraphMLM-pythia8/RunIISpring15FSPremix-MCRUN2_74_V9-v1/MINIAODSIM! \tabularnewline
\verb!/SMS-T1qqqq_mGluino-1250to1275_mLSP-550to1250_TuneCUETP8M1_13TeV-madgraphMLM-pythia8/RunIISpring15FSPremix-MCRUN2_74_V9-v1/MINIAODSIM! \tabularnewline
\verb!/SMS-T1qqqq_mGluino-1300to1325_mLSP-1to1275_TuneCUETP8M1_13TeV-madgraphMLM-pythia8/RunIISpring15FSPremix-MCRUN2_74_V9-v1/MINIAODSIM! \tabularnewline
\verb!/SMS-T1qqqq_mGluino-1325to1350_mLSP-1to1275_TuneCUETP8M1_13TeV-madgraphMLM-pythia8/RunIISpring15FSPremix-MCRUN2_74_V9-v1/MINIAODSIM! \tabularnewline
\verb!/SMS-T1qqqq_mGluino-1400to1450_mLSP-1to1275_TuneCUETP8M1_13TeV-madgraphMLM-pythia8/RunIISpring15FSPremix-MCRUN2_74_V9-v1/MINIAODSIM! \tabularnewline
\verb!/SMS-T1qqqq_mGluino-1450to1500_mLSP-1to1275_TuneCUETP8M1_13TeV-madgraphMLM-pythia8/RunIISpring15FSPremix-MCRUN2_74_V9-v1/MINIAODSIM! \tabularnewline
\verb!/SMS-T1qqqq_mGluino-1500to1550_mLSP-1to1250_TuneCUETP8M1_13TeV-madgraphMLM-pythia8/RunIISpring15FSPremix-MCRUN2_74_V9-v1/MINIAODSIM! \tabularnewline
\verb!/SMS-T1qqqq_mGluino-1550_mLSP-950_TuneCUETP8M1_13TeV-madgraphMLM-pythia8/RunIISpring15FSPremix-MCRUN2_74_V9-v1/MINIAODSIM! \tabularnewline
\verb!/SMS-T1qqqq_mGluino-1600to1650_mLSP-1to1250_TuneCUETP8M1_13TeV-madgraphMLM-pythia8/RunIISpring15FSPremix-MCRUN2_74_V9-v1/MINIAODSIM! \tabularnewline
\verb!/SMS-T1qqqq_mGluino-1650to1750_mLSP-1to1250_TuneCUETP8M1_13TeV-madgraphMLM-pythia8/RunIISpring15FSPremix-MCRUN2_74_V9-v1/MINIAODSIM! \tabularnewline
\verb!/SMS-T1qqqq_mGluino-1750_mLSP-50to1250_TuneCUETP8M1_13TeV-madgraphMLM-pythia8/RunIISpring15FSPremix-MCRUN2_74_V9-v1/MINIAODSIM! \tabularnewline
\verb!/SMS-T1qqqq_mGluino-1850to1950_mLSP-1to1250_TuneCUETP8M1_13TeV-madgraphMLM-pythia8/RunIISpring15FSPremix-MCRUN2_74_V9-v1/MINIAODSIM! \tabularnewline
\verb!/SMS-T1qqqq_mGluino-1950to2000_mLSP-1to1250_TuneCUETP8M1_13TeV-madgraphMLM-pythia8/RunIISpring15FSPremix-MCRUN2_74_V9-v1/MINIAODSIM! \tabularnewline
\verb!/SMS-T1qqqq_mGluino-2000_mLSP-50to950_TuneCUETP8M1_13TeV-madgraphMLM-pythia8/RunIISpring15FSPremix-MCRUN2_74_V9-v1/MINIAODSIM! \tabularnewline
\hline
\end{tabular}\end{center}
}
\label{tab:datasets_SMS_T1qqqq}
\end{table}

\begin{table}[!p]
 \centering
\topcaption{Simulated signal samples: SMS T1tttt}
 \scriptsize
 \scalebox{.7}[1.0]{\input{tables/datasets/c140607_c151022_l003_SMS-T1tttt.tex}}
%\label{tab:datasets_bkg}
\end{table}

\begin{table}[!p]
 \centering
\topcaption{Simulated signal samples: SMS T1bbbb}
 \scriptsize
 \scalebox{.7}[1.0]{\input{tables/datasets/c140607_c151022_l003_SMS-T1bbbb.tex}}
\label{tab:datasets_SMS_T1bbbb}
\end{table}

\clearpage

\subsection{Pileup reweighting}
\label{sec:pileup-reweighting}

The distribution of the numbers of the pileup interactions in the
simulated events is different from that in the data; the simulated
events contain, on average, a larger number of pileup interactions than
the data. We reweight simulated events to correct for this difference.
This procedure is called \textit{pileup reweighting}.

In deriving the pileup reweighting factors, we follow the
recommendation by the physics validation group
\cite{twiki-PdmVPileUpDescription, twiki-PileupJSONFileforData}. In
the recommendation, the reweighting factors are a function of the
variable called \verb!nTrueInt!.

This variable \verb!nTrueInt! is the parameter of the Poisson
distribution from which the numbers of pileup interactions are drown
as random numbers. In each simulated event, the number of the in-time
pileup interactions and the number of the interactions in each
neighbouring bunch crossing to simulate of the out-of-time pileup are
random numbers from the Poisson distribution with the same parameter,
\verb!nTrueInt!. The value of \verb!nTrueInt! is not a constant of the
data set. It is a random number from the distribution specified in
Ref. \cite{github-mix_2015_25ns_Startup_PoissonOOTPU_cfi}.

The \verb!nTrueInt! in the data is the average pileup interactions for
a colliding bunch pair in a lumi section. The distribution of
\verb!nTrueInt! in the data is derived from the measured instantaneous
luminosity for each colliding bunch pair in each lumi section and the
cross section of the total inelastic pp interaction. We use the method
in Ref. \cite{twiki-PileupJSONFileforData} in deriving the
distribution with the recommended value of 69~mb as the minimum bias
cross section.

The pileup reweighting factors are the ratios of the distributions of
\verb!nTrueInt! in the data and in the simulated events and are
normalised so as to preserved the number of the simulated events.

Figure~\ref{f044_corr_nTrueInt_data_mc_norm} shows the distributions
of \verb!nTrueInt! in the data, simulated events and reweighted
simulated events. The figure demonstrates that the reweighted
simulated events have the distribution of \verb!nTrueInt! nearly
identical to that in the data.

\begin{figure}[!b]
\centering
\includegraphics[scale=1.00]{figures/pileup_reweighting/f044_corr_nTrueInt_data_mc_norm}
\caption{The distribution of the average numbers of the inelastic
interactions per colliding bunch pair per lumi section in the data,
corresponding distribution in the simulated events, and that of the
reweighted simulated events.} \label{f044_corr_nTrueInt_data_mc_norm}
\end{figure}


\subsection{Cross sections for SM samples}
Several MC samples of individual SM processes are binned according to a generator level quantity, such as the partonic \HT or bosonic \PT.
This analysis chooses to use samples binned in partonic \HT, for the set of MC samples (W+jets, DY+jets, QCD, $\gamma$+jets, $Z\rightarrow \nu\nu$+jets) and $\hat{P_{T}}$
for only the QCD sample.
These binned samples are provided with LO cross sections. The \kfactors required to go from LO to NNLO cross section are typically determined using corresponding
inclusive samples applied to each \HT binned sample.
Further studies can provide additional corrections to the cross sections, which can prove important to the closure test procedures described in
Section \ref{sec:closure-tests-desc}. As can be seen in Section \ref{sec:sideband_corrections}, residual cross section
corrections are measured using data in sidebands designed to enriched specific processes.

In the $8\tev$ LHC results the shape of the top quark $p_{T}$ spectrum
was found to differ between simulation and data. A reweighting is
therefore applied to MC events that contain a generated top. The value of
this correction is provided from the $8\tev$ results, as described in
\cite{twiki-TopPtReweighting}.

Following an inclusive selection, the distribution of the MC samples with respect to the binning variable $H_{T}^{parton}$ are shown in Fig.~\ref{fig:Lhe_Ht}, with
also the $\hat{P_{T}}$ distribution for QCD.

As stated above, the distributions shown in Fig.~\ref{fig:Lhe_Ht} follow an inclusive selection, whereby there is no requirement on each event. This direct translation
from MC allows to merge the binned generator level variable, and exhibits a smooth shape with respect to the cross sections in question.

A more in depth, data-driven investigation of the cross sections is shown in Sec.~\ref{sec:sideband_corrections}. Of which, an important point to note is that
the corrections to the cross sections, derived with the data sidebands are only relevant for data/MC comparison plots and the suite of closure tests defined in Section \ref{sec:closure-tests-desc}.

\begin{figure}[!h]
  \begin{center}
    \subfigure[$Z\rightarrow \nu\nu$ +jets] {\includegraphics[width=0.45\textwidth]{figures/binnedMCsamples/Zinv.pdf}} ~~
    \subfigure[$W\rightarrow l \nu$ + jets]{\includegraphics[width=0.45\textwidth]{figures/binnedMCsamples/WJetsToLNu_HT.pdf}} \\
    \subfigure[$DY\rightarrow ll$ + jets]{\includegraphics[width=0.45\textwidth]{figures/binnedMCsamples/DYJetsToLL_M50_HT.pdf}} ~~
    \subfigure[QCD]{\includegraphics[width=0.45\textwidth]{figures/binnedMCsamples/QCD_HT.pdf}} \\
    \subfigure[$\gamma$+jets]{\includegraphics[width=0.45\textwidth]{figures/binnedMCsamples/GJets_HT.pdf}} ~~
    \subfigure[QCD $\hat{P_{T}}$]{\includegraphics[width=0.45\textwidth]{figures/binnedMCsamples/QCD.pdf}}\\
    \caption{Generator-level $H_{T}^{parton}$ distributions for SM process, $Z\rightarrow \nu\nu$ + jets, W+jets, DY+jets, QCD, $\gamma$+jets, and $\hat{P_{T}}$ for QCD.}
    \label{fig:Lhe_Ht}
  \end{center}
\end{figure}

%%____________________________________________________________________________||
