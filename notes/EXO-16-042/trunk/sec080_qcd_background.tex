%%__________________________________________________________________||
\section{Estimation of the QCD multijet background}
\label{sec:qcd_background}

The signal region is defined such that the expected contribution from
multijet events is suppressed to the percent level with respect to the
total expected background counts from other SM processes for all event
categories and \scalht bins. This is achieved through very tight
requirements on the variables \alphat, \dphi, and $\mht / \ETmiss$, as
described above. 

Potential contamination from multijet events in the signal region is
estimated by exploiting the ratio of multijet events that satisfy or
fail the requirement $\mhtmet < 1.25$. This ratio is determined from
simulated events categorised according to \njet and \scalht and
validated in a multijet-enriched data sideband defined by $\dphi <
0.5$. Estimates of the QCD multijet background counts, binned
according to \njet and \scalht, are determined in a further data
sideband, defined by the requirement $\mhtmet > 1.25$, by correcting
the observed counts in data to account for contamination from vector
boson and \ttbar production (plus residual contributions from other SM
processes). The product of the corrected counts and ratios provide an
independent estimate of the QCD multijet contamination as a function
of \njet and \scalht, with an assumed systematic uncertainty of 100\%
based on the validation in the multijet-enriched data sidebands. The
estimates are found to negligible, typically at the percent level,
relative to the sum of all other SM backgrounds in each bin across the
full signal region phase space. The distribution of any residual
contamination as a function of \nb and \mht is determined from
simulation. Finally, data control variables are inspected to provide
confidence that any multijet contamination due to instrumental effects
is negligible.

%%__________________________________________________________________||
