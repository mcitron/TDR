%%__________________________________________________________________||
\section{Estimation of the QCD multijet background}
\label{sec:qcd_background}

%The signal region is defined such that the expected contribution from
%multijet events is suppressed to the percent level with respect to the
%total expected background counts from other SM processes for bins of
%the signal region. This is achieved through very tight requirements on
%the variables \alphat, \dphi, and \HTmiss/\ETmiss, the latter of which
%is described above.

Potential contamination from multijet events in the signal region is
estimated by exploiting the ratio of multijet events that satisfy or
fail the requirement $\HTmiss/\ETmiss < 1.25$, as determined from
simulation, for events categorised according to \njet and
\scalht. Estimates of the QCD multijet background counts, binned
according to \njet and \scalht, are determined in a data sideband,
defined by the requirement $\HTmiss/\ETmiss > 1.25$, by correcting the
observed counts in data to account for contamination from non-multijet
backgrounds (vector boson and \ttbar production, plus residual
contributions from other SM processes). The products of the corrected
counts and ratios provide independent estimates of the multijet
background as a function of \njet and \scalht.

The multijet background estimates are found to be typically at the
percent level relative to the sum of all other SM backgrounds, in all
bins of the signal region. The distribution of these small
contributions as a function of \nb and \mht is determined from
simulation. Statistical uncertainties associated with the finite event
counts in data and simulated samples are propagated to each
estimate. The uncertainties in the estimates of the contamination from
non-multijet backgrounds in the \HTmiss/\ETmiss data sideband are
determined according to the prescription described in
Section~\ref{sec:ewk_background}. Finally, the ratios determined from
simulated events are validated in a further multijet-enriched data
sideband, defined by $\dphi < 0.5$, from which a systematic
uncertainty of 100\% is determined.

%%__________________________________________________________________||
