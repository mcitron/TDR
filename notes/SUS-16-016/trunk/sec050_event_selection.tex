%%__________________________________________________________________||
\section{Event reconstruction and selections} 
\label{sec:event_selection}

Global event reconstruction is provided by the particle-flow (PF)
algorithm~\cite{CMS-PAS-PFT-09-001, CMS-PAS-PFT-10-001}, which aims to
identify single candidate particles using an optimized combination of
information from all detector systems. In this process, the
identification of the particle type (photon, electron, muon, charged
hadron, neutral hadron) plays an important role in the determination
of the particle direction and energy.

In order to suppress SM processes with genuine \ptvecmiss from
neutrinos and select only multijet final states, events containing an
isolated electron~\cite{PAS-EGM-10-004} or muon~\cite{PAS-MUO-10-004}
with $\Pt > 10\GeV$ or isolated photon~\cite{PAS-EGM-10-006} with $\pt
> 25\GeV$ are vetoed. Furthermore, events containing an isolated track
with $\Pt > 10\gev$ are also vetoed in order to reduce the background
contribution from final states containing hadronically-decaying tau
leptons. Jets are reconstructed from candidate particles clustered by
the anti-$k_{\rm T}$ algorithm~\cite{antikt} with a size parameter of
$0.4$. The jet energies are corrected to account for the effects of
pileup and to establish a uniform relative response in $\eta$ and a
calibrated absolute response in transverse momentum
\pt~\cite{cms-jets}. Jets considered in the analysis are required to
have a transverse momentum above $40\gev$ and $|\eta| < 3$.

The mass scale of the physics processes being probed is characterised
by the scalar sum of the transverse momenta $\Pt$ of these jets,
defined as $\scalht = \sum_{i=1}^{N_{\rm jet}} \Pt^{\,\mathrm{j}_i}$,
where $N_{\rm jet}$ is the number of jets within the experimental
acceptance. The missing transverse momentum vector \ptvecmiss is
defined as the projection on the plane perpendicular to the beams of
the negative vector sum of the momenta of all candidate particles in
an event. Its magnitude is referred to as \ETmiss. The estimator for
\ETmiss used by this search is given by the magnitude of the vector
sum of the transverse momenta of these jets, $\mht =
|\sum_{i=1}^{N_\text{jet}} \ptvec^{\,\mathrm{j}_i}|$. Events are
vetoed if any additional jet satisfies $\Pt > 40\GeV$ and $|\eta| >
3$, in order to maintain the performance of the variable \mht as an
estimator of \ETmiss. Significant hadronic activity and \ptvecmiss in
the event is ensured by requiring $\scalht > 200\GeV$ and $\mht >
130\gev$, respectively. The most energetic jet in the event is
required to satisfy $\Pt > 100\gev$.

A number of beam- and detector-related effects can lead to events with
large values of \ETmiss, such as beam halo, reconstruction failures,
spurious detector noise, or event misreconstruction due to detector
inefficiencies. These events, with large, non-physical values of
\ETmiss, are rejected with high efficiency by applying a range of
dedicated vetoes~\cite{RA1Paper2012, cms-met}. An additional dedicated
veto is employed to deal with the circumstance in which several jets
with transverse momentum below the \Pt thresholds and collinear in
$\phi$ can result in significant \mht relative to \ETmiss, the latter
of which is less sensitive to jet thresholds. This type of background,
typical of multijet events, is suppressed while maintaining high
efficiency for SM or new physics processes with significant \ptvecmiss
by requiring $\mht / \ETmiss < 1.25$.

\begin{table*}[tb]
  \topcaption{Summary of the event selection requirements and
    categorisations used to define the signal region and control
    samples.}
  \label{tab:selections}
  \centering
  \footnotesize
  \begin{tabular}{ ll }
    \hline
    \multicolumn{2}{l}{\bf Baseline selection}\T\B                                                                                             \\
    \ETmiss cleaning             & Filters related to beam and instrumental effects                                                            \\ 
    Lepton/photon vetoes         & $\Pt > 10,\, 10,\, 25\GeV$ for isolated tracks, leptons, photons (respectively) and $\abs{\eta} < 2.5$      \\ 
    Jet $j_\text{i}$ acceptance  & Consider each jet $j_\text{i}$ that satisfies $\Pt^{j_\text{i}} > 40\GeV$ and $\abs{\eta^{j_\text{1}}} < 3$ \\
    Jet $j_\text{1}$ acceptance  & $\Pt^{j_\text{1}} > 100\GeV$ and $\abs{\eta^{j_\text{1}}} < 2.5$                                            \\
    Jet $j_\text{2}$ acceptance  & $\Pt^{j_\text{2}} < 40\GeV$ (monojet),                                                                      \\
                                 & $40 < \Pt^{j_\text{2}} < 100\GeV$ (asymmetric),                                                             \\
                                 & $\Pt^{j_\text{2}} > 100\GeV$ (symmetric)                                                                    \\
    Forward jet veto             & Veto events containing jet satisfying $\Pt > 40\GeV$ and $\abs{\eta} > 3$                                   \\
    Jets below threshold         & $\HTmiss / \ETmiss < 1.25$                                                                                  \\
    Energy sums                  & $\scalht > 200\GeV$ and $\HTmiss > 130\GeV$ \B                                                              \\
    \hline
    \multicolumn{2}{l}{\bf Event categorisation}\T\B                                                                                           \\
    \njet                        & 1 (monojet); 2, 3, 4, $\geq$5 (asymmetric); 2, 3, 4, $\geq$5 (symmetric)                                    \\
    \nb                          & 0, 1, 2, $\geq$3 ($\nb \leq \njet$)                                                                         \\
    \scalht (GeV)                & 200, 250, 300, 350, 400, 500, 600, $>$800\GeV (some bins are dropped/merged \vs \njet) \B                   \\
    \hline
    {\bf Signal region (SR)}     & Baseline selection + \T\B                                                                                   \\
    QCD multijet rejection \quad & $\alphat > 0.65$, 0.60, 0.55, 0.53, 0.52, 0.52, 0.52 (mapped to \scalht bins in range 200--800\GeV)         \\
    QCD multijet rejection       & $\bdphi > 0.5$\B                                                                                            \\[0.5ex]
    \hline
    {\bf Control samples (CS)}   & Baseline selection + \T\B                                                                                   \\
    Multijet-enriched            & SR + $\HTmiss/\ETmiss > 1.25$ (inverted)                                                                    \\  
    \gj                          & 
    1$\gamma$ with $\Pt > 200\GeV$, $\abs{\eta} < 1.45$, 
    $\Delta R(\gamma,j_{\text{i}}) > 1.0$, 
    $\scalht > 400\GeV$, same \alphat req. as SR                                                                                               \\[0.5ex]
    \mj                          & 
    1$\mu$ with $\Pt > 30\GeV$, $\abs{\eta} < 2.1$, 
%    $I^{\mu}_\text{rel} < 0.1$, 
    $\Delta R(\mu,j_{\text{i}}) > 0.5$,
    $30 < m_\text{T}(\ptvec^\mu,\ptvecmiss) < 125\GeV$                                                                                         \\[0.5ex]
    \mmj                         & 
    2$\mu$ with $\Pt > 30\GeV$, $\abs{\eta} < 2.1$, 
%    $I^{\mu}_\text{rel} < 0.1$, 
    $\Delta R(\mu_{1,2},j_{\text{i}}) > 0.5$, 
    $ \abs{m_{\mu\mu} - m_\text{Z}} < 25\GeV$ \B                                                                                               \\[0.5ex]
    \hline
  \end{tabular}
\end{table*}

The aforementioned selection requirements define a baseline set, as
summarised in Table~\ref{tab:selections}. Additional requirements,
described below, are utilised to define a sample of candidate signal
events, labelled henceforth as the signal region. Four additional
control samples of events are employed to estimate the background
contributions from SM processes, which modify and expand on the
baseline selection requirements. The first control sample is enriched
in multijet events and is used to estimate the multijet contribution
in the signal region. Three additional control samples comprising \gj,
\mj, or \mmj events, defined by the baseline set of selections and the
inversion of one of the photon or lepton vetoes, are used to estimate
background contributions from SM processes, predominantly \wlj,
\znunuj, and \ttbar production, that lead to final states containing
jets and significant \ptvecmiss. Additional kinematic requirements are
employed to ensure the control samples are enriched in the same SM
processes that contribute to background events in the signal region,
and are depleted in contributions from multijet production or a wide
variety of SUSY models (\ie so-called signal contamination).  The
control samples are defined such that the kinematic properties of
events in the control regions and the candidate signal events resemble
as closely as possible one another, once the photon, muon, or dimuon
system is ignored in the calculation of quantities such as \scalht and
\HTmiss. The event selection requirements for the four control samples
are summarised in Table~\ref{tab:selections}.

Events containing at least two jets are categorised according to {\it
  symmetric} or {\it asymmetric} topologies if the second-most
energetic jet satisfies, respectively, $\Pt > 100\gev$ or $40 < \Pt <
100\gev$. Events that contain only one jet satisfying the requirement
$\Pt > 40\gev$ are categorised as a {\it monojet} topology. The
symmetric topology targets the pair production of sparticles and their
cascade decays, while the monojet and asymmetric topologies target
nearly mass-degenerate SUSY models, as well as the direct production
of weakly interacting massive particles. Events are further
categorised according to the number of jets per event (\njet), the
number of reconstructed jets identified as originating from a b quark
(\nb), and \scalht. These categorisations, summarised in
Table~\ref{tab:selections}, are used identically for the signal region
and the four control samples. Finally, the search exploits the use of
the \mht variable as a discriminant between the dominant SM
backgrounds and new-physics signatures. The expected distribution of
events as a function of \mht is determined from simulation, an
approach that is validated in multiple data control samples.

For events satisfying the baseline selections described above,
summarised in Table~\ref{tab:selections}, the multijet background
dominates over all other SM backgrounds. The \alphat kinematic
variable, first introduced in Refs.~\cite{Randall:2008rw, RA1Paper},
is used to efficiently reject multijet events with transverse momentum
mismeasurements while retaining sensitivity to new physics with
genuine \ptvecmiss signatures. The variable \alphat depends solely on
the measurements of the transverse momenta and azimuthal angles of
jets and it is intrinsically robust against the presence of jet energy
mismeasurements in multijet systems. For dijet events, the \alphat
variable is defined as $\alphat = \Pt^{\rm j_2}/M_\text{T}$ where
$\Pt^{\rm j_2}$ is the transverse momentum of the less-energetic jet,
and $M_\text{T}$ is the transverse mass of the dijet system.  For a
perfectly measured dijet event with $\Pt^{\mathrm{j}_1} =
\Pt^{\mathrm{j}_2}$ and jets back-to-back in $\phi$, and in the limit
in which the momentum of each jet is large compared with its mass, the
value of \alphat is 0.5. For the case of an imbalance in the measured
transverse momenta of back-to-back jets, \alphat is reduced to a value
smaller than 0.5, which gives the variable its intrinsic
robustness. Values significantly greater than 0.5 are observed when
the two jets are not back to back and are recoiling against
significant, genuine \ptvecmiss. The definition of the \alphat
variable can be generalised for events with two or more jets, as
described in Ref.~\cite{RA1Paper2012}.

Multijet events typically populate the region $\alphat \lesssim 0.5$
and the \alphat distribution is characterised by a sharp edge at 0.5,
beyond which the multijet event yield falls by several orders of
magnitude. Multijet events with extremely rare but large stochastic
fluctuations in the calorimetric measurements of jet energies can lead
to values of \alphat slightly above 0.5. The edge at 0.5 sharpens with
increasing \scalht for multijet events, primarily due to a
corresponding increase in the average jet energy and thus an
improvement in the jet energy resolution, but also because the
threshold effect of jets below the \Pt threshold contributing
significantly to \mht decreases with increasing \scalht. This
motivates a \scalht-dependent \alphat requirement that varies in the
range 0.52--0.65 for the region $\scalht < 800\gev$.

The \dphi variable considers the minimum azimuthal angular separation
of a jet and the \mht vector derived from all other jets in the
event. The \dphi variable provides powerful discriminating power
between final states with genuine \ptvecmiss and mismeasured QCD
multijet events. The variable is also highly efficient at suppressing
any potential contribution from rare energetic multijet events that
yield high jet multiplicities and significant \ETmiss due to
high-multiplicity neutrino production in semileptonic heavy-flavour
decays. The neutrinos are typically collinear with respect to the axis
of a jet and carry a significant fraction of the energy. The
requirement $\dphi > 0.5$ is sufficient to suppress effectively the
multijet background. For the region $\scalht < 800\gev$, the
requirements on both the \alphat and \dphi variables are utilised,
whereas for the region $\scalht > 800\gev$, the necessary control of
the QCD multijet background is achieved solely with the \dphi
requirement.

The tight requirements on the variables \alphat, \dphi, and
\HTmiss/\ETmiss suppress the expected contribution from multijet
events to the percent level with respect to the total expected
background counts from other SM processes, for all bins of the signal
region. Further, control variables are inspected to provide confidence
that any multijet contamination due to instrumental effects is
negligible. The aforementioned requirements complete the definition of
the signal region, and are summarised in Table~\ref{tab:selections}.

%Figure~\ref{fig:alphat-bdphi} shows the \alphat and \dphi
%distributions observed in data for events that satisfy all other
%signal region selection criteria plus $\scalht > 300\gev$ and $\scalht
%> 800\gev$, respectively. In the case of the \alphat distribution, the
%events that satisfy $\alphat < 0.55$ must only fulfill the baseline
%selection criteria defined in Table~\ref{tab:selections}, no \mht
%requirement is made, and the events are recorded with an unbiased set
%of trigger \scalht conditions.

%\begin{figure*}[tbhp]
%  \begin{center}
%    \includegraphics[width=0.49\textwidth]{alphaT_v4} \,
%    \includegraphics[width=0.49\textwidth]{bDPhi_v4} \\
%  \end{center}
%  \caption{(Left) The \alphat distribution observed in data for events
%    that are recorded with unbiased trigger conditions and satisfy the
%    baseline (full signal region) selection criteria for the region
%    $\alphat < 0.55$ ($\alphat > 0.55$). (Right) The \dphi
%    distribution observed in data for events that satisfy the full
%    signal region selection criteria and $\scalht > 800\gev$.  The
%    distributions for the QCD multijet backgrounds are determined from
%    simulation while all other SM backgrounds are estimated using a
%    $\mu$ + jets data control sample. %The uncertainties in the SM
%    %expectation are dominated by the statistical uncertainties 
%    %associated with the limited sample of simulated multijet events.
%    \label{fig:alphat-bdphi} 
%  }
%\end{figure*}

The categorisation of candidate signal events, as a function of \njet,
\nb, \scalht, and \HTmiss, and the number of bins within the signal
region are determined primarily by the statistical power of the
multiple data control samples. The signal region and control samples
cover a large phase space, defined primarily by the loose requirements
$\scalht > 200\GeV$ and $\HTmiss > 130\GeV$, and candidate signal
events are categorised into 194 exclusive sub-regions according to
\njet, \nb, and \scalht. Within each sub-region, events are further
categorised according to \HTmiss: the first bin is defined by the
range $130 < \HTmiss < 200\GeV$, subsequent bins have a width of
100\GeV, up to a final open bin that satisfies $\HTmiss > 800\GeV$. If
the statistical power of the simulated or data control samples is
limited, the higher \HTmiss bins are merged, reducing the threshold on
the final open bin, to the limiting case of a single open bin defined
by $\HTmiss > 130\GeV$. This procedure ensures the information taken
from simulation is always adequately supported by checks in the data
control samples. On average, less than four bins in \HTmiss are
utilised per (\njet, \nb, \scalht) category.

Candidate signal events are recorded with multiple jet-based trigger
conditions that require both \scalht and \alphat to satisfy
predetermined thresholds. In addition, a trigger condition based
solely on \scalht is used to record candidate events for the region
$\scalht > 800\gev$. A dedicated trigger condition requiring the
presence of significant \mht and \ETmiss is used to record events
containing one or more jets. The trigger-level jet energies are
corrected to account for energy scale and pileup effects. The trigger
strategy provides efficiencies at or near 100\% for all bins of the
signal region.
