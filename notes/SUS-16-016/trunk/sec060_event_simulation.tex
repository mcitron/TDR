%%__________________________________________________________________||
\section{Event simulation}
\label{sec:event_simulation}

The search relies on several event samples, recorded by the CMS
experiment or generated from Monte Carlo (MC) simulations, to estimate
the contributions from SM backgrounds, as described in
Section~\ref{sec:event_selection}. 
%The search relies on multiple control regions in data and simulated
%event samples to estimate contributions from SM backgrounds. 
%The definitions of the control regions in data and the background
%estimation procedures are described in
%Section~\ref{needtoaddthisref}. 
The dominant SM backgrounds for the search are QCD multijet
production, and the associated production of jets and top-antitop
(\ttbar), single top, and vector boson (\wlnu, \znunu). Residual
contributions from other processes, such as WW, WZ, ZZ (diboson)
production and the associated production of \ttbar and a vector boson
(W and Z), are also considered. Other processes, such as Drell-Yan
($\cPq\bar{\cPq}\!  \rightarrow\! \PZ/\gamma^*\! \rightarrow\!
\ell^+\ell^-$) and \gj production, are also relevant for some control
regions, defined below.

The \MADGRAPH5 aMC@NLO 2.2.2~\cite{Alwall2014} event generator code is
used at leading order (LO) accuracy to produce samples of \wj, \zj,
\gj, \ttbar, and multijet events. The same code is used at
next-to-leading order (NLO) accuracy to generate samples of s-channel
production of single top, as well as {\ttbar}W and {\ttbar}Z
events. The NLO \POWHEG v2~\cite{powheg, powheg_top_Wt} generator is
used to describe the t- and tW-channel production of single top
events. The LO \PYTHIA 8.2~\cite{pythia} program is used to generate
diboson (WW, WZ, ZZ) production. The simulated samples are normalised
according to production cross sections that are calculated with NLO
and next-to-NLO precision~\cite{Alwall2014, wphys, fewz, wwxs, top++,
  nlotop, powheg_top_Wt}. The description of the detector response,
for these SM processes, is implemented using the
\GEANTfour~\cite{geant} package.

Event samples for signal models involving gluino or squark pair
production, in association with up to two additional partons, are
generated at leading order with \MADGRAPH5 aMC@NLO, and the decay of
the sparticles is performed with \PYTHIA 8.2~\cite{pythia}. Inclusive,
process-dependent, signal production cross sections are calculated
with NLO plus next-to-leading-logarithm (NLL)
accuracy~\cite{Beenakker:1996ch, PhysRevLett.102.111802,
  PhysRevD.80.095004, 1126-6708-2009-12-041,
  doi:10.1142/S0217751X11053560, susynlo}. The detector response is
provided by the CMS fast simulation package~\cite{fastsim}.

%which yields consistent results compared with the GEANT4-based
%simulation, except that we apply a correction of 1\% to account for
%differences in the efficiency of the jet quality requirements [32],
%and corrections of 3-10\% to account for differences in the b jet
%tagging efficiency. 

The \textsc{NNPDF}3.0 LO and \textsc{NNPDF}3.0 NLO~\cite{nnpdf} parton
distribution functions (PDF) are used, respectively, with the LO and
NLO generators described above. The LO \PYTHIA 8.2~\cite{pythia}
program is used to describe parton showering and hadronisation for all
simulated samples. To model the effects of multiple pp collisions
within the same or neighboring bunch crossings (pileup), all simulated
events are generated with a nominal distribution of pp interactions
per bunch crossing and then reweighted to match the pileup
distribution as measured in data. 

%Finally, (near-unity) corrections to the normalisation of the
%simulated samples for the \gj, \wlj, \ttbar, and \zllj, and,
%equivalently, \znunuj processes are derived using a data sideband to
%the control regions.  

%To model the effects of pileup, the simulated events are generated
%with a nominal distribution of pp interactions per bunch crossing and
%then reweighted to match the pileup distribution as measured in data.

%Simulated samples of SM events are used to construct and validate the
%procedures and to estimate a few of the smaller background
%components. The MADGRAPH5 aMC@NLO 2.2.2 [35] event generator at
%leading order is used to simulate tt, W+jets, Z+jets, gamma+jets, and
%QCD multijet events. This same generator at next-to-leading (NLO)
%order is used to describe single top events in the s channel, events
%with dibosons (WW, ZZ, and WH production, etc., with H a Higgs boson),
%and rare processes (ttW, ttZ, and WWZ production, etc.), except WW
%events in which both W bosons decay leptonically are described with
%the POWHEG v1.0 [36-40] program at NLO. Single top events in the t and
%tW channels are also described with POWHEG at NLO. Simulation of the
%detector response is based on the GEANT4 [41] package. The simulated
%samples are normalized using the most accurate cross section
%calculations currently available [35, 39, 40, 42-50], generally with
%NLO or next-to-NLO accuracy.
%
%Signal T1bbbb, T1tttt, T1qqqq, and T5qqqqVV events are generated for a
%range of gluino mg and LSP m 0 mass values, with m 0 < m . Similarly,
%T2tt and T2qq events are generated for a range of squark m and m 0
%mass values, with m 0 < m . For the T5qqqqVV model, the masses of the
%intermediate chi and chi states are taken to be the mean of m 0 and m
%. The signal samples are generated with the MADGRAPH5 aMC@NLO program
%at leading order, with up to two partons present in addition to the
%gluino pair. The decays of the gluino are described with a pure phase-
%space matrix element [51]. The signal production cross sections are
%computed [52-56] with NLO plus next-to-leading-logarithm (NLL)
%accuracy. To reduce computational requirements, the detector is
%modeled with the CMS fast simulation program [57, 58], which yields
%consistent results compared with the GEANT4-based simulation, except
%that we apply a correction of 1\% to account for differences in the
%efficiency of the jet quality requirements [30], and corrections of
%3-10\% to account for differences in the b jet tagging efficiency.
%
%The NNPDF3.0LO [59] parton distribution functions (PDF) are used for
%the simulated samples generated at leading order, and the NNPDF3.0NLO
%[59] PDFs for the samples generated at NLO. All simulated samples use
%the PYTHIA 8.2 [51] program to describe parton showering and
%hadronization. To model the effects of pileup, the simulated events
%are generated with a nominal distribution of pp interactions per bunch
%crossing and then reweighted to match the corresponding distribution
%in data.
%
%We evaluate systematic uncertainties in the signal model
%predictions. Those that are relevant for the selection efficiency are
%listed in Table 1. The uncertainty associated with the renormal-
%ization and factorization scales is determined by varying each scale
%independently by factors of 2.0 and 0.5 [60, 61]. An uncertainty
%related to the modeling of initial-state radiation (ISR) is determined
%by comparing the simulated and measured pT spectra of the system
%recoiling against the ISR jets in tt events, using the technique
%described in Ref. [62]. The two spectra are observed to agree. The
%statistical precision of the comparison is used to define an
%uncertainty of 15\% (30\%) for 400 < pT < 600 GeV (pT > 600 GeV),
%while no uncertainty is deemed neces- sary for pT < 400 GeV. The
%uncertainties associated with the renormalization and factorization
%scales, integrated over all search regions, typically lie below 0.1\%
%but can be as large as 1-3\% for m 0 ~m (we use the notation m0 ~ m to
%mean 0 +2m ~m, with the bottom quark mass, the top quark mass, or the
%mass of the ``V'' boson, respectively, for the T1bbbb, T1tttt, and
%T5qqqqVV models; for the T1qqqq model, m 0 ~ m means m 0 ~ m ). The
%uncertainties associated with the jet energy scale and jet energy
%resolution are evaluated as a function of jet pT and eta. Note that
%the isolated lepton and isolated track vetoes have a minimal impact on
%the T1bbbb and T1qqqq models because events in these models rarely
%contain an isolated lepton, and that the associated uncertainty is
%negligible (0.1\%).
%
%We also evaluate systematic uncertainties in the signal predictions
%related to the b jet tagging and misidentification efficiencies and to
%the statistical uncertainties in the signal event samples.  These
%sources of uncertainty do not affect the signal efficiency but can
%potentially alter the signal distribution shapes. Similarly, the
%sources of systematic uncertainty associated with the trigger
%efficiency, pileup reweighting, renormalization and factorization
%scales, ISR, jet energy scale, and jet energy resolution can affect
%the shapes of the signal distributions. These potential changes in
%shape, i.e., migration of events between search regions, are accounted
%for in the limit-setting procedure described in Section 6.
%
%The systematic uncertainty in the determination of the integrated
%luminosity is 4.6\%. 