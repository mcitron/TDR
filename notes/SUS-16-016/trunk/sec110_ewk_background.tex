%%__________________________________________________________________||
\section{Estimation of SM backgrounds with genuine \ptvecmiss}
\label{sec:ewk_background}

Following the suppression of multijet events, the background counts in
the signal region mainly arise from SM processes that produce
neutrinos, resulting in final states with significant \ptvecmiss. In
events with low counts of jets and b quark jets, the largest
backgrounds with genuine \ptvecmiss are from the associated production
of W or Z bosons with jets, followed by either the weak decays \znunu
or \wtaunu, where the $\tau$ decays hadronically and is identified as
a jet; or by leptonic decays that are not rejected by the dedicated
electron or muon vetoes. The veto of events containing isolated tracks
is efficient at further suppressing these backgrounds as well as the
single-prong hadronic decay of the tau lepton. At higher jet and b-tag
multiplicities, top quark production followed by semileptonic weak top
quark decay becomes important.

The simulated samples of \gj, \wlj, \zllj, and \ttbar production are
normalised to data using scale factors derived in data sidebands,
enriched in the relevant process. The definition of the sidebands, the
selection applied and the scale factors are listed in
Table~\ref{tab:sideband-corrs}. These factors are derived after all
other corrections are applied to the simulated samples.

\begin{table}[!h]
  \footnotesize
  \centering
  \topcaption{Cross section corrections for SM processes determined from
    data sidebands.}
  \label{tab:sideband-corrs}
  \begin{tabular}
    {lllc}
    \hline
    SM process & Control sample & Data sideband           & Corrrection\T\B   \\
    \hline                   
    \gj        & \gj            & $0.50 < \alphat < 0.52$ & $1.33 \pm 0.03$\T \\
    \wlj       & \mj            & $100 < \mht < 130\GeV$  & $1.13 \pm 0.01$   \\
    \zllj      & \mmj           & $100 < \mht < 130\GeV$  & $0.99 \pm 0.02$   \\
    \ttbar     & \mj, \mmj      & $100 < \mht < 130\GeV$  & $0.86 \pm 0.01$\B \\
    \hline
  \end{tabular}
\end{table}

The method to estimate the non-multijet backgrounds in the signal
region relies on the use of transfer factors, which are constructed
per bin (in terms of \njet, \nb, and \scalht) per data control
sample. The transfer factors are determined from the simulated event
samples and are ratios of expected yields in the corresponding bins of
the signal region and control samples. The transfer factors are used
to extrapolate from the event yields measured in data control samples
to an expectation for the total background event yields in the signal
region.  Three disjoint data control samples, binned identically to
the signal region, are used to estimate the contributions from the
non-multijet SM background processes, as summarised in
Table~~\ref{tab:selections}.

The \mj sample is recorded using a trigger condition that requires an
isolated muon and the event selection criteria are chosen in order to
ensure high trigger efficiency. Furthermore, the muon is required to
be well separated from the jets in the event and the transverse mass
($m_{\text{T}}$) of the muon and \ptvecmiss system must satisfy $30 <
m_\text{T}(\ptvec^\mu,\ptvecmiss) < 125\GeV$ to ensure a sample rich
in W bosons (produced promptly or from the decay of top quarks). The
\mmj sample uses similar selection criteria as the \mj sample and the
same trigger condition. Exactly two oppositely-charged, isolated muons
are required, the muons must be distanced from the jets in the event,
and the invariant mass of the dimuon system ($m_{\mu\mu}$) must be
within a window of $\pm 25\GeV$ around the mass of the Z boson, $
\abs{m_{\mu\mu} - m_\text{Z}} < 25\GeV$. For both the muon and dimuon
samples, no requirement is made on the variable \alphat in order to
increase the statistical precision of the predictions derived from
these samples, in contrast to the identical \alphat requirements made
for the signal region and photon control sample. The \gj sample is
recorded using a single photon trigger condition. The event selection
criteria comprise an isolated photon with $\Et > 200\gev$ and $\scalht
> 400\GeV$.

Three independent estimates of the irreducible background of \znunu +
jets events are determined from the \gj, \mmj, and \mj data control
samples. The \gj and \zmumu + jets processes have similar kinematic
properties when the photon or muons are ignored~\cite{Bern:2011pa}, 
albeit different acceptances. In addition, the \gj process has a
larger production cross section than \znunu + jets events. The \mj
data sample is used to provide an estimate for the \znunu\ + jets
contribution as well as the other dominant SM processes, \ttbar and W
boson production. Residual contributions from processes such as
single-top-quark, diboson, and Drell-Yan production are also included.

\newcommand{\phh}{\ensuremath{\phantom{1-}}}
\begin{table*}[h!]
  \caption{
    Systematic uncertainties in the transfer factors used in
    the method to estimate the SM backgrounds with genuine \ptvecmiss
    in the signal region. The quoted ranges provide the minimum and
    maximum values used across all bins in \njet and \scalht.
  } 
  \label{tab:bkgd_systs}
  \centering
  \footnotesize
  \begin{tabular}{ lrrrr }
    \hline
    Systematic source         & \multicolumn{4}{c}{Uncertainty in transfer factor [\%]}\T\B \\
    \cline{2-5} 
                              & $\mj \Rightarrow \ttbar/\PW$ 
                              & $\mj \Rightarrow \znunu$ 
                              & $\mmj \Rightarrow \znunu$ 
                              & $\gj \Rightarrow \znunu$\T\B                                \\
    \hline                                                    
    \multicolumn{5}{l}{\it Corrections applied to simulation:}\T\B                          \\
    Jet energy scale          & 1--5  & 1--5  & 1--5  & 1--5                                \\
    b-tag efficiency / mistag & 1--5  & 1--5  & 1--5  & 1--5                                \\
    Lepton scale factors      & 1--3  & 1--3  & 1--3  & -                                   \\
    Pileup                    & 0--2  & 0--2  & 0--2  & 0--2                                \\
    Signal trigger efficiency & 1--2  & 1--2  & 1--2  & 1--2                                \\
    Muon trigger efficiency   & 2     & 2     & 2     & -                                   \\
    Photon trigger efficiency & -     & -     & -     & 1--2                                \\
    Top quark \Pt             & 1--10 & 1--30 & 1--10 & -                                   \\ 
    \multicolumn{5}{l}{\it Derived from closure tests in data:}\T\B                         \\
    W/Z ratio                 & -     & 4--15 & -     & -                                   \\
    Z/$\gamma$ ratio          & -     & -     & -     & 6--11                               \\
    W/\ttbar composition      & 4--30 & -     & -     & -                                   \\
    W polarisation            & 2--10 & 2--10 & -     & -                                   \\
    \alphat / \bdphi          & 3--30 & 3--30 & 3--30 & -\B                                 \\
    \hline
  \end{tabular}
\end{table*}

Several sources of uncertainty in the transfer factors are evaluated.
The most relevant effects are discussed below, and generally fall into
one of two categories. The first category concerns uncertainties in
``scale factor'' corrections applied to simulation, which are
determined using inclusive data samples that are defined by loose
selection criteria, to account for the mismodelling of theoretical and
experimental parameters. The second category concerns ``closure
tests'' in data that probe various aspects of the accuracy of the
simulation to model correctly the transfer factors in the phase space
of this search.

The uncertainties in the transfer factors are studied for variations
in scale factors related to: the jet energy scale, the efficiency and
misidentification probability of b quark jets, the efficiency to
identify or veto well-reconstructed, isolated leptons, and the
modelling of the transverse momentum of top quarks. %~\cite{}. 
A 5\% uncertainty in the minimum bias cross section is assumed and
propagated through to the reweighting procedure to account for
differences between the simulated and data-derived measurements of the
pileup distributions.  Uncertainties in the trigger efficiency
measurements are also propagated to the transfer factors.  The
aforementioned systematic uncertainties, resulting from variations in
scale factors, are summarised in Table~\ref{tab:bkgd_systs}, along
with representative magnitudes.  Each source of uncertainty is assumed
to vary with a fully correlated behaviour across the full phase space
of the signal and control regions.

Sources of additional uncertainty are determined from sets of
``closure tests'' based on data control
samples~\cite{RA1Paper2012}. Each set uses the observed event counts
in up to eight bins in \scalht for each of the nine \njet event
categories in one of the three independent data control samples, along
with the corresponding transfer factors determined from simulation, to
obtain a prediction of the observed yields in another control sample
(or, in one case, \nb event category). 
Each set of tests is designed to target a specific (potential) source
of bias in the simulation modelling that may introduce an \njet- or
\scalht-dependent source of systematic bias in the transfer
factors~\cite{RA1Paper2012}. Several sets of tests are performed. The
$\PZ/\gamma$ ratio determined from simulation is tested against the
same ratio measured using \zmmj events and the \gj sample. The
$\PW/\PZ$ ratio is also probed using the \mj and \mmj samples. A
further set probes the modelling of the relative composition between
\wlj and \ttbar events using \mj events containing exactly zero or one
more b-tagged jets, which represents a larger extrapolation in
relative composition than used in the search.  The effects of W
polarisation are probed by using \mj events with a positively charged
muon to predict those containing a negatively charged muon. Finally,
the accuracy of the modelling of the efficiencies of the \alphat and
\bdphi requirements are estimated using the \mj sample.

For each set of tests, the level of closure, which considers only
statistical uncertainties, is inspected to ensure no statistically
significant biases are observed as a function of the nine \njet
categories or the eight \scalht bins. In the absence of such a bias,
the level of closure is recomputed by integrating over either all
monojet and asymmetric topologies, or the symmetric \njet
categories. The level of closure and its statistical uncertainty are
combined in quadrature to determine additional contributions to the
uncertainties in the transfer factors. These uncertainties are
considered to be fully correlated between the monojet and asymmetric
topologies or the symmetric topology, and fully uncorrelated between
these two regions in \njet and \scalht bins. If the closure tests use
the \mmj sample, the level of closure is determined by additionally
integrating over pairs of adjacent \scalht bins. These uncertainties,
derived from the closure tests in data, are summarised in
Table~\ref{tab:bkgd_systs}, along with representative
magnitudes. These uncertainties are the dominant contribution to the
total uncertainty in the transfer factors, due to the limited number
of events in the data control samples.

Templates determined from simulation are used to predict the
background counts in the \mht dimension. Uncertainties in the scale
factor corrections applied to simulated events, as discussed above in
the context of transfer factors, are propagated as uncertainties in
the \HTmiss templates. Uncertainties in the trigger efficiency
measurements are also propagated to the \HTmiss templates. These
sources of uncertainty are assumed to vary with a correlated behaviour
across the full phase space of the signal region. Multiple data
control samples are used to evaluate the degree to which the
simulation describes the \mht distributions observed in data, and to
assign appropriate systematic uncertainties, which can be significant
($\sim$50--100\%) in the most sensitive \mht bins.  The \mht templates
from simulation are compared to the distributions observed in the
control samples, and inspected for trends, by assuming a linear
behaviour of the ratio of observed and simulated counts as a function
of \HTmiss. Linear fits are performed independently for each bin
defined by \njet, \nb, and \scalht. No significant biases or trends
are observed, given the statistical power of the control samples, and
systematic uncertainties are determined from the constrained fit
parameters. These uncertainties are treated as fully uncorrelated
between bins defined in terms of \njet, \nb, and \scalht, and also
with respect to the systematic uncertainties in in the transfer
factors, summarised in Table~\ref{tab:bkgd_systs}.

%%__________________________________________________________________||
