%%__________________________________________________________________||
\section{Estimation of SM backgrounds with genuine \ptvecmiss}
\label{sec:ewk_background}

In the absence of multijet events, the background counts in the signal
region arise from SM processes with significant \ETmiss in the final
state. In events with low counts of jets and b-quark jets, the largest
backgrounds with genuine \ptvecmiss are from the associated production of
W or Z bosons with jets, followed by either the weak decays \znunu or
\wtaunu, where the $\tau$ decays hadronically and is identified as a
jet; or by leptonic decays that are not rejected by the dedicated
electron or muon vetoes. The veto of events containing isolated tracks
is efficient at further suppressing these backgrounds as well as the
single-prong hadronic decay of the tau lepton. At higher jet and
b-quark jet multiplicities, top quark production followed by
semileptonic weak top quark decay becomes important.

%The production of W and Z bosons in association with jets, \ttbar and
%\gj processes are simulated with the \MADGRAPH V5~\cite{madgraph}
%event generator. The production of single-top quark events is
%generated with \POWHEG~\cite{powheg}, and diboson events are produced
%with \PYTHIA8.1~\cite{pythia8}. For all simulated samples, \PYTHIA8.1
%is used to describe parton showering and hadronisation. All samples
%are generated using the \textsc{cteq6l1}~\cite{Pumplin:2002vw} parton
%distribution functions (PDF). The description of the detector response
%is implemented using the \GEANTfour~\cite{geant} package. The
%simulated samples are normalised using the most accurate cross section
%calculations currently available, usually with
%next-to-next-to-leading-order (NNLO) accuracy. 
%To model the effects of pileup, the simulated events are generated
%with a nominal distribution of pp interactions per bunch crossing and
%then reweighted to match the pileup distribution as measured in data.

%FIXME: need to change when we switch to use the fit numbers
The samples corresponding to the production of \gj, \wj, \zj and \ttj
are reweighted with scale factors derived from sidebands in data and
simulation, enriched in the relevant process.  The definition of the
sidebands, the selection applied and the scale factors are listed in
Table~\ref{tab:sideband-corrs}.  These factors are derived after all
corrections are applied to the simulated samples.  they are applied
throughout the analysis, including background predictions and results.
After these corrections are included, a better closure is observed in
the systematic studies described in this section. 

\begin{table}[!h]
  \scriptsize
  \centering
  \topcaption{The scale factors applied to the main SM backgrounds and the definition of the sidebands used to derive them. 
  The uncertainty shown is statistical only.}
  \label{tab:sideband-corrs}
  \begin{tabular}
    {cllc}
    \hline\hline
    \textbf{Process} & \textbf{Sideband} & \textbf{Selection} & \textbf{Corrrection} \\
    \hline
    \gj & $0.50 < \alphat < 0.52(0.53)$ & \gj & $1.25 \pm 0.10$ \\
    \wj & $100 < \mht < 130 \, \mathrm{GeV}$ & \mj, $\njet \leq 3$, $\nb = 0$ & $1.12 \pm 0.03$ \\
    \zj & $100 < \mht < 130 \, \mathrm{GeV}$ & \mmj, $\nb = 0$ & $1.08 \pm 0.05$ \\
    \ttbar + jets & $100 < \mht < 130 \, \mathrm{GeV}$ & \mj, $\njet \geq 2$, $\nb \geq 2$ & $0.86 \pm 0.03$ \\
    \hline \hline
  \end{tabular}
\end{table}


The method to estimate the non-multijet backgrounds in the signal
region relies on the use of transfer factors, which are constructed
per bin (in terms of \njet, \nb, and \scalht) per data control
sample. The transfer factors are determined from the simulated event
samples and are ratios of expected yields in the corresponding bins of
the signal region and control samples. The transfer factors are used
to extrapolate from the event yields measured in data control samples
to an expectation for the total background event yields in the signal
region.

Three disjoint data control regions, binned identically to the signal
region, are used to estimate the contributions from the 
non-multijet SM background processes. The control regions are defined by
a selection of \mj, \mmj, and \gj events. The selection criteria are
chosen such that the SM processes and their kinematic properties
resemble as closely as possible the SM background behaviour in the
signal region once the muon, dimuon system, or photon are ignored when
computing quantities such as \scalht, \mht and \alphat. The baseline
selection criteria and binning definition described in
Table~\ref{tab:selections} are applied to all control samples, except
for the lepton and photon vetoes, which are inverted and tightened to
improve purity. The event selection criteria are defined to also
ensure that any potential contamination from multijet events or a wide
variety of SUSY models (\ie signal contamination) is negligible.

The \mj sample is recorded using a trigger condition that requires an
isolated muon and the event selection criteria are chosen in order to
ensure high trigger efficiency. Furthermore, the muon is required to
be well separated from the jets in the event and the transverse mass
($M_{\rm T}$) of the muon and \ETmiss~\cite{CMS-PAS-PFT-09-001,
  CMS-PAS-PFT-10-001} system must satisfy $30 < M_{\rm T} < 125\gev$
to ensure a sample rich in W bosons (produced promptly or from the
decay of top quarks). The \mmj sample uses similar selection criteria
as the \mj sample and the same trigger condition. Exactly two
oppositely-charged, isolated muons are required, the muons must be
distanced from the jets in the event, and the invariant mass of the
dimuon system must be within a window of $\pm 25\GeV$ around the mass
of the Z boson. For both the muon and dimuon samples, no requirement
is made on the variable \alphat in order to increase the statistical
precision of the predictions derived from these samples, in constrast
to the identical \alphat requirements made for the signal region and
photon control sample. The \gj sample is recorded using a single
photon trigger condition. The event selection criteria comprise an
isolated photon with $\Et > 200\gev$ and $\scalht > 400\GeV$.

Three independent estimates of the irreducible background of \znunu +
jets events are determined from the \gj, \mmj, amd \mj data control
samples. The \gj and \zmumu + jets processes have similar kinematic
properties when the photon or muons are ignored~\cite{Bern:2011pa}, 
albeit different acceptances. In addition, the \gj process has a
larger production cross section than \znunu + jets events. The \mj
data sample is used to provide an estimate for the \znunu\ + jets
contribution as well as the other dominant SM processes, \ttbar and W
boson production. Residual contributions from processes such as
single-top-quark, diboson, and Drell-Yan production are also included.

%\begin{table}[thp!]
%  \caption{%CMS {\it Preliminary}, $\mathcal{L}_{\mathrm{int}} =
%    % 2.2\fbinv$, $\sqrt{s} = 13\TeV$. 
%    % \newline
%    Systematic uncertainties (percent) in the estimates of the
%    normalisation of the SM background components as a function of
%    \njet, as determined from ensembles of closure tests based on
%    multiple data control samples. The quoted ranges correspond to the 
%    variations determined across the \nb and \scalht bins of a given \njet
%    category. The additional contributions listed at the foot of the
%    table are added in quadrature to the \njet-dependent contributions
%    for each SM background component. } 
%  \label{tab:bkgd_systs}
%  \centering
%  \footnotesize
%  \begin{tabular}{ lcc }
%    \hline
%    \hline
%    \njet                         & \multicolumn{2}{c}{Uncertainty (\%) in background component} \\
%    \cline{2-3}
%                                  & \ttbar, W+jets, residual SM & \znunu\ + jets                 \\
%    \hline
%    \multicolumn{2}{l}{``Monojet'':}                                                             \\
%    1                             & 9--36                       & 9--36                          \\
%    \hline
%    \multicolumn{2}{l}{``Asymmetric'':}                                                          \\
%    2                             & 11--105                     & 9--46                          \\
%    3                             & 12--86                      & 12--78                         \\
%    4                             & 16--52                      & 13--43                         \\
%    $\geq$5                       & 19--47                      & 27--73                         \\
%    \hline
%    \multicolumn{2}{l}{``Symmetric'':}                                                           \\
%    2                             & 7--34                       & 11--30                         \\
%    3                             & 9--31                       & 13--44                         \\
%    4                             & 13--36                      & 8--34                          \\
%    $\geq$5                       & 15--22                      & 17--28                         \\
%    \hline
%    \multicolumn{2}{l}{Additional contributions:}                                                \\
%    \alphat ($\scalht < 800\gev$) & 10-27                       & 10-27                          \\
%    \dphi ($\scalht > 800\gev$)   & 22                          & 22                             \\
%    b-quark identification        & $<$5                        & $<$5                           \\
%    \hline
%    \hline
%  \end{tabular}
%\end{table}

%The uncertainty in the transfer factors derived from simulation is
%probed through closure tests based on data control
%samples~\cite{RA1Paper2012}. Each closure test inspects the
%compatibility of yields in two disjoint data control samples and a
%corresponding transfer factor derived from simulation. A large
%ensemble of tests are performed to probe the simulation modelling of a
%range of key physics effects that may lead to potential biases in the
%transfer factors~\cite{RA1Paper2012}. For the analysed data sample,
%the closure tests reveal no significant biases or dependencies on
%\njet or \scalht for all individual tests. Systematic uncertainties in
%the transfer factors are typically determined from ensembles of
%closure tests, which can be considered as ``normalisation''
%uncertainties for a given component of background events categorised
%according to \njet, \nb, and \scalht.  Table~\ref{tab:bkgd_systs}
%summarises typical values for the systematic uncertainties in
%experimental acceptance effects for the dominant SM background
%components. These uncertainties are assumed to be fully uncorrelated
%for events catogorised differently in \njet, \nb, and \scalht. The
%uncertainties associated with extrapolations in the \alphat and \dphi
%variables, and the simulation modelling of the efficiency and mistag
%rates for identifying jets originating from b quarks or light-flavour
%partons are also listed, which are assumed to be fully correlated
%across the \nb dimension only.

\newcommand{\phh}{\ensuremath{\phantom{1-}}}
\begin{table*}[h!]
  \caption{
    Systematic uncertainties (percent) in the transfer factors used in
    the method to estimate the SM backgrounds with genuine \ptvecmiss
    in the signal region. The quoted ranges provide representative
    values of the observed variations as a function of \njet and
    \scalht. \fixme{\it To be updated.}
  } 
  \label{tab:bkgd_systs}
  \centering
  \footnotesize
  \begin{tabular}{ lrrrr }
    \hline
    Systematic           & \multicolumn{4}{c}{Uncertainty in transfer factor [\%]} \\
    \cline{2-5} 
    source               & $\mj \Rightarrow \ttbar/\PW$ 
                         & $\mj \Rightarrow \znunu$ 
                         & $\mmj \Rightarrow \znunu$ 
                         & $\gj \Rightarrow \znunu$                                \\
    \hline                                                    
    \multicolumn{5}{l}{\it Scale factors (applied to simulation):
      \fixme{\it these are to be updated}}                 \\
    Jet energy scale     & $<15\%$    & $<15\%$   & $<10\%$   & $<15\%$            \\
    b-tag eff. \ mistag  & $<5\%$     & $<5\%$    & $<2\%$    & $<2\%$             \\
    Lepton SFs           & $2-5\%$    & $2-5\%$   & $2-5\%$   & $-$                \\
    Pileup               & $<10\%$    & $<6\%$    & $<4\%$    & $<3\%$             \\
    Top quark \Pt        & $<5\%$     & $<20\%$   & $<4\%$    & $-$                \\ [0.5ex]
    \multicolumn{5}{l}{\it Closure tests:}                                         \\
    W/Z ratio            & $-$        & $10-30\%$ & $-$       & $-$                \\
    Z/$\gamma$ ratio     & $-$        & $-$       & $-$       & $10-40\%$          \\
    W/\ttbar composition & $5-60\%$ & $-$       & $-$       & $-$                \\
    W polarisation       & $5-50\%$   & $5-50\%$  & $-$       & $-$                \\
    $\alphat\,/\,\bdphi$ & $5-40\%$   & $5-40\%$  & $5-40\%$ & $-$                \\
    \hline
  \end{tabular}
\end{table*}

Several sources of uncertainty in the transfer factors are evaluated.
The most relevant effects are discussed below, and generally fall into
one of two categories. The first category concerns uncertainties in
``scale factor'' corrections applied to simulation, which are
determined using inclusive data samples that are defined by loose
selection criteria, to account for the mismodelling of theoretical and
experimental parameters. The second category concerns ``closure
tests'' in data that probe various aspects of the accuracy of the
simulation to model correctly the transfer factors in the phase space
of this search.

%The uncertainties in the transfer factors are studied for variations
%in scale factors related to: the jet energy scale (that result in
%uncertainties in the transfer factors as large as $\sim$15\%), the
%efficiency and misidentification probability of b quark jets (up to
%5\%), and the efficiency to identify well-reconstructed, isolated
%leptons (up to $\sim$5\%). A 5\% uncertainty in the minimum bias cross
%section, $\sigma_\text{MB} = 69.0 \pm 3.5\unit{mb}$, is assumed and
%propagated through to the reweighting procedure to account for
%differences between the simulated and data-derived measurements of the
%pileup distributions, which results in changes of up to
%$\sim$10\%. The modelling of the transverse momentum of top quarks
%($\Pt^\text{t}$) is evaluated by comparing the simulated and measured
%\Pt spectra of reconstructed top objects in \ttbar events. 
%% , using the technique described in Ref.~\cite{}. 
%Simulated events are reweighted according to scale factors that
%decrease from a value of $\sim$1.2 to $\sim$0.7, with uncertainties of
%$\sim$0.1, within the range $0 < \Pt^\text{t} < 400\GeV$. The
%resulting change in the transfer factors is as large as
%$\sim$20\%. These uncertainties, resulting from variations in scale
%factors, are summarised in Table~\ref{tab:bkgd_systs}, along with
%representative magnitudes.  Each source of uncertainty is assumed to
%vary with a fully correlated behaviour across the full phase space of
%the signal and control regions.

%The second category of sources 
Sources of uncertainty are determined from sets of ``closure tests''
based on data control samples~\cite{RA1Paper2012}. Each set uses the
observed event counts in up to eight bins in \scalht for each of the
nine \njet event categories in one of the three independent data
control regions, along with the corresponding transfer factors
determined from simulation, to obtain a prediction of the observed
yields in another control sample (or, in one case, \nb event
category).

Each set of tests is designed to target a specific (potential) source
of bias in the simulation modelling that may introduce an \njet- or
\scalht-dependent source of systematic bias in the transfer
factors~\cite{RA1Paper2012}. Several sets of tests are performed. The
$\PZ/\gamma$ ratio determined from simulation is tested against the
same ratio measured using \zmmj events and the \gj sample. The
$\PW/\PZ$ ratio is also probed using the \mj and \mmj samples. A
further set probes the modelling of the relative composition between
\wlj and \ttbar events using \mj events containing exactly zero or one
more b-tagged jets, which represents a larger extrapolation in
relative composition than used in the search.  The effects of W
polarisation are probed by using \mj events with a positively charged
muon to predict those containing a negatively charged muon. Finally,
the accuracy of the modelling of the efficiencies of the \alphat and
\bdphi requirements are estimated using the \mj sample.

For each set of tests, the level of closure, which considers only
statistical uncertainties, is inspected to ensure no statistically
significant biases are observed as a function of the nine \njet
categories or the eight \scalht bins. In the absence of such a bias,
the level of closure is recomputed by integrating over either all
monojet and asymmetric \njet categories, or the symmetric \njet
categories. The level of closure and its statistical uncertainty are
combined in quadrature to determine additional contributions to the
uncertainties in the transfer factors. These uncertainties are
considered to be fully correlated between the monojet and asymmetric
\njet categories or the symmetric \njet categories, and fully
uncorrelated between these two regions in \njet and \scalht bins. If
the closure tests use the \mmj sample, the level of closure is
determined by additionally integrating over pairs of adjacent \scalht
bins. These uncertainties, derived from the closure tests in data, are
summarised in Table~\ref{tab:bkgd_systs}, along with representative
magnitudes. These uncertainties are the dominant contribution to the
total uncertainty in the transfer factors, due to the limited number
of events in the data control regions.

Templates derived from simulation are used to predict the background
counts in the \mht dimension. Multiple data control regions are used
to evaluate the degree to which the simulation describes the \mht
distributions observed in data, and to assign appropriate systematic
uncertainties that can be in excess of $>100\%$ in the most sensitive
\mht bins. 
The \mht shape in data is compared to the prediction from the simulation 
and inspected for trends by fitting linear slopes to the data-MC ratio, 
independently for each \njet, \nb, \scalht category. 
This study doesn't reveal any significant bias and alternative templates 
covering the systematic uncertainty are derived by varying the 
linear parameter within its statistical uncertainty. 
Independent uncertainty in the templates are implemented for
the \znunu + jets background and the W + jets and \ttbar backgrounds
and treated as fully uncorrelated across categories 
and with respect to the ``normalisation'' systematic
uncertainties summarised in Table~\ref{tab:bkgd_systs}. \\
Alternative \mht templates are also derived by varying the MC corrections within their uncertainties, 
namely jet energy corrections, b-tag scale factors, pile-up reweighting and trigger efficiency.

%%__________________________________________________________________||
