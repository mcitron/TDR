s%____________________________________________________________________________||
\section{Interpretation in Dark Matter models} \label{sec:darkmatter}



\subsection{Light flavour models} \label{sec:dm_lightjet}

The light flavour simplified models consist of a DM particle \pchi of mass
\mchi that is a Dirac fermion, and a spin-1 (vector or axial-vector) or spin-0
(scalar or pseudoscalar) mediating particle \pphi of mass \mphi in an
$s$-channel. The couplings of the mediator with the standard model and dark
matter particles are given by \gsm and \gdm, respectively. The recommendations
by the DM Forum on the choice of couplings is \gsm$=1$,\gdm$=1$ for
(pseudo)scalar models, and \gsm$=0.25$,\gdm$=1$ for (axial-)vector models.
Assuming that no additional visible or invisible particles contribute to the decay 
of the mediator, we impose the minimal width determined by the choice of couplings. 
An example Feynman diagram is shown in Fig.~\ref{fig:DMfeynman}.


\begin{figure}[h!] \centering
  \subfigure{\includegraphics[width=0.35\textwidth]{figures/DMplots/feynman_light_jet.pdf}}
  \caption{Feynman diagram of DM pair production in light jet hadronic final states. \cite{Abercrombie:2015wmb}}
  \label{fig:DMfeynman} 
\end{figure}



%Assuming 2~\ifb of data we list the cross sections, yields and selection  efficiencies for the four light jet models in Tables~\ref{summaryTableAN_DMV_xs10_g0p25_2p1fb_exp}-\ref{summaryTableAN_DMS_xs10_g1p0_2p1fb_exp}. 
The signal selection efficiencies are around $\sim 1$\% for mass points near the expected exclusion
region, and are correspondingly larger (smaller) for higher (lower) mass points.
The asymmetric and monojet categories are seen to almost double the acceptance
to these models compared to the Run~1 symmetric categories alone, justifying the
inclusion of these selections into the analysis.




\clearpage 
%\begin{table}
\small
\centering
\begin{tabular}{lllllll}
\hline
$m_\phi$ & $m_\chi$ & $\sigma$ [pb] & Yield (sym) & Yield (asy) & Yield (mon) & Efficiency [\%] \\ \hline
10000     &   100       &   1.99e-04  &   0.00      &   0.00      &   0.00      &   1.45      \\ 
10000     &   150       &   1.80e-04  &   0.00      &   0.00      &   0.00      &   1.59      \\ 
10000     &   500       &   7.68e-05  &   0.00      &   0.00      &   0.00      &   2.11      \\ 
10000     &   50        &   2.09e-04  &   0.00      &   0.00      &   0.00      &   1.12      \\ 
1000      &   1000      &   5.04e-03  &   0.06      &   0.07      &   0.13      &   2.38      \\ 
1000      &   100       &   2.73e+01  &   179.52    &   190.01    &   383.56    &   1.31      \\ 
1000      &   150       &   2.62e+01  &   160.27    &   208.11    &   403.66    &   1.40      \\ 
1000      &   1         &   2.71e+01  &   160.57    &   203.17    &   434.80    &   1.40      \\ 
1000      &   500       &   4.97e+00  &   33.21     &   47.36     &   81.68     &   1.55      \\ 
100       &   10        &   5.93e+04  &   7589.50   &   19108.63  &   31108.83  &   0.05      \\ 
100       &   1         &   5.91e+04  &   14800.01  &   12847.15  &   11821.32  &   0.03      \\ 
100       &   50        &   3.88e+03  &   539.13    &   1807.91   &   4031.74   &   0.08      \\ 
10        &   100       &   5.54e+01  &   103.87    &   126.63    &   248.14    &   0.41      \\ 
10        &   10        &   5.17e+04  &   1145.90   &   332.97    &   2451.58   &   0.00      \\ 
10        &   150       &   1.37e+01  &   47.10     &   58.73     &   111.65    &   0.76      \\ 
10        &   500       &   1.41e-01  &   1.16      &   1.34      &   2.76      &   1.78      \\ 
10        &   50        &   4.67e+02  &   230.50    &   378.34    &   743.68    &   0.14      \\ 
2000      &   1         &   1.21e+00  &   11.66     &   12.00     &   24.85     &   1.92      \\ 
200       &   10        &   6.95e+03  &   5415.53   &   7354.07   &   15915.05  &   0.20      \\ 
200       &   150       &   2.67e+01  &   73.94     &   84.18     &   215.47    &   0.67      \\ 
200       &   1         &   7.31e+03  &   5121.26   &   7686.21   &   18636.29  &   0.20      \\ 
200       &   1         &   7.31e+03  &   6238.76   &   9708.29   &   17140.23  &   0.22      \\ 
200       &   50        &   7.21e+03  &   4068.85   &   7543.76   &   16796.86  &   0.19      \\ 
20        &   1         &   3.89e+06  &   72829.16  &   0.00      &   0.00      &   0.00      \\ 
300       &   100       &   1.92e+03  &   3361.32   &   3813.36   &   9551.46   &   0.42      \\ 
300       &   10        &   2.14e+03  &   3897.55   &   4065.27   &   11657.41  &   0.44      \\ 
300       &   150       &   2.65e+02  &   500.37    &   656.11    &   1441.81   &   0.47      \\ 
300       &   1         &   2.07e+03  &   3791.08   &   4448.85   &   9664.55   &   0.41      \\ 
300       &   50        &   2.04e+03  &   3769.22   &   4258.83   &   10451.88  &   0.43      \\ 
500       &   100       &   3.42e+02  &   966.21    &   1385.64   &   3075.59   &   0.76      \\ 
500       &   10        &   3.45e+02  &   1118.46   &   1620.52   &   2888.59   &   0.78      \\ 
500       &   150       &   3.28e+02  &   985.14    &   1458.16   &   2740.83   &   0.75      \\ 
500       &   1         &   3.23e+02  &   1325.55   &   1776.04   &   3250.20   &   0.94      \\ 
500       &   500       &   1.97e-01  &   1.54      &   1.88      &   3.85      &   1.76      \\ 
500       &   50        &   3.43e+02  &   1013.67   &   1413.07   &   3283.48   &   0.79      \\ 
50        &   10        &   3.83e+05  &   4924.12   &   9435.89   &   10145.19  &   0.00      \\ 
50        &   1         &   3.96e+05  &   16034.59  &   32044.78  &   15910.13  &   0.01      \\ 
50        &   50        &   6.32e+02  &   360.15    &   560.45    &   947.73    &   0.14      \\ 
\hline
\end{tabular}
\caption{Cross section, yields at 2~\ifb (split according to symmetric, asymmetric, and monojet categories), and total selection efficiency for the vector \DMj samples.}
\label{summaryTableAN_DMV_xs10_g0p25_2p1fb_exp}
\end{table}
 \clearpage
%\input{tables/DM/summaryTableAN_DMA_xs10_g0p25_2p1fb_exp.tex} \clearpage
%\input{tables/DM/summaryTableAN_DMP_xs10_g1p0_2p1fb_exp.tex} \clearpage
%\begin{table}
\footnotesize
\centering
\begin{tabular}{lllllll}
\hline
$m_\phi$ & $m_\chi$ & $\sigma$ [pb] & Yield (sym) & Yield (asy) & Yield (mon) & Efficiency [\%] \\ \hline
10000     &   1000      &   6.76e-09  &   0.00      &   0.00      &   0.00      &   2.31      \\ 
10000     &   10        &   2.34e-06  &   0.00      &   0.00      &   0.00      &   0.76      \\ 
10000     &   150       &   1.34e-06  &   0.00      &   0.00      &   0.00      &   1.00      \\ 
10000     &   1         &   2.36e-06  &   0.00      &   0.00      &   0.00      &   0.74      \\ 
10000     &   500       &   1.08e-07  &   0.00      &   0.00      &   0.00      &   1.83      \\ 
10000     &   50        &   2.18e-06  &   0.00      &   0.00      &   0.00      &   0.82      \\ 
1000      &   1000      &   1.62e-06  &   0.00      &   0.00      &   0.00      &   2.29      \\ 
1000      &   100       &   1.84e-01  &   1.15      &   1.35      &   1.80      &   1.12      \\ 
1000      &   10        &   1.93e-01  &   1.53      &   1.37      &   1.73      &   1.14      \\ 
1000      &   150       &   1.68e-01  &   1.14      &   1.24      &   1.34      &   1.05      \\ 
1000      &   1         &   1.97e-01  &   1.29      &   1.49      &   2.00      &   1.15      \\ 
1000      &   500       &   2.58e-03  &   0.02      &   0.02      &   0.03      &   1.33      \\ 
1000      &   50        &   1.94e-01  &   1.35      &   1.49      &   1.82      &   1.15      \\ 
100       &   100       &   3.30e-01  &   0.69      &   1.12      &   1.34      &   0.45      \\ 
100       &   10        &   1.14e+02  &   66.39     &   112.95    &   145.53    &   0.14      \\ 
100       &   1         &   1.14e+02  &   73.56     &   89.42     &   125.17    &   0.12      \\ 
100       &   50        &   2.04e+00  &   2.07      &   3.23      &   3.86      &   0.21      \\ 
10        &   10        &   9.95e+00  &   4.64      &   7.25      &   7.81      &   0.09      \\ 
10        &   1         &   1.16e+03  &   33.19     &   105.71    &   120.59    &   0.01      \\ 
2000      &   1000      &   1.99e-05  &   0.00      &   0.00      &   0.00      &   2.04      \\ 
2000      &   100       &   2.92e-03  &   0.03      &   0.03      &   0.03      &   1.43      \\ 
2000      &   10        &   3.26e-03  &   0.03      &   0.03      &   0.03      &   1.37      \\ 
2000      &   150       &   2.61e-03  &   0.03      &   0.03      &   0.03      &   1.53      \\ 
2000      &   1         &   3.27e-03  &   0.03      &   0.03      &   0.04      &   1.38      \\ 
2000      &   500       &   1.16e-03  &   0.02      &   0.01      &   0.02      &   1.97      \\ 
2000      &   50        &   3.17e-03  &   0.03      &   0.03      &   0.03      &   1.40      \\ 
200       &   100       &   8.18e-01  &   1.47      &   2.16      &   3.13      &   0.39      \\ 
200       &   10        &   3.95e+01  &   52.07     &   83.83     &   95.54     &   0.28      \\ 
200       &   150       &   1.70e-01  &   0.42      &   0.63      &   0.73      &   0.50      \\ 
200       &   1         &   3.18e+01  &   48.47     &   60.01     &   87.26     &   0.29      \\ 
200       &   50        &   3.95e+01  &   52.99     &   74.38     &   121.02    &   0.30      \\ 
20        &   10        &   1.21e+01  &   5.11      &   6.31      &   10.55     &   0.09      \\ 
20        &   1         &   7.18e+02  &   54.37     &   100.26    &   202.57    &   0.02      \\ 
300       &   100       &   2.32e+01  &   38.43     &   72.30     &   87.53     &   0.41      \\ 
300       &   10        &   2.32e+01  &   48.98     &   68.83     &   77.31     &   0.40      \\ 
300       &   150       &   4.86e-01  &   1.21      &   1.63      &   1.76      &   0.45      \\ 
300       &   1         &   2.32e+01  &   43.48     &   62.04     &   88.43     &   0.40      \\ 
300       &   50        &   2.32e+01  &   42.00     &   65.55     &   97.24     &   0.42      \\ 
5000      &   1000      &   3.29e-07  &   0.00      &   0.00      &   0.00      &   2.47      \\ 
5000      &   100       &   3.02e-05  &   0.00      &   0.00      &   0.00      &   0.93      \\ 
5000      &   10        &   3.94e-05  &   0.00      &   0.00      &   0.00      &   0.89      \\ 
5000      &   150       &   2.24e-05  &   0.00      &   0.00      &   0.00      &   1.07      \\ 
5000      &   1         &   3.94e-05  &   0.00      &   0.00      &   0.00      &   0.81      \\ 
5000      &   500       &   2.41e-06  &   0.00      &   0.00      &   0.00      &   1.83      \\ 
5000      &   50        &   3.72e-05  &   0.00      &   0.00      &   0.00      &   0.85      \\ 
500       &   100       &   6.69e+00  &   20.54     &   27.96     &   32.65     &   0.58      \\ 
500       &   10        &   7.47e+00  &   19.03     &   27.47     &   37.66     &   0.54      \\ 
500       &   150       &   5.56e+00  &   20.23     &   23.12     &   25.59     &   0.59      \\ 
500       &   1         &   7.47e+00  &   21.75     &   28.18     &   35.85     &   0.55      \\ 
500       &   500       &   3.02e-04  &   0.00      &   0.00      &   0.00      &   1.51      \\ 
500       &   50        &   7.31e+00  &   22.52     &   28.77     &   33.05     &   0.55      \\ 
50        &   10        &   2.87e+02  &   83.24     &   73.88     &   131.87    &   0.05      \\ 
50        &   1         &   2.87e+02  &   52.47     &   109.61    &   155.16    &   0.05      \\ 
\hline
\end{tabular}
\caption{Cross section, yields at 2~\ifb (split according to symmetric, asymmetric, and monojet categories), and total selection efficiency for the pseudo-scalar \DMtt samples.}
\label{summaryTableAN_DMS_xs10_g1p0_2p1fb_exp}
\end{table}
 \clearpage

\clearpage
For comparisons with past results we also include the axial-vector samples using $g_{\rm SM} = 1.0$ although they are not included in the 
'ATLAS-CMS DM' Forum. The key characteristics for this sample are shown in Table~\ref{summaryTableAN_DMA_xs10_g1p0_2p1fb_exp}.

%\input{tables/DM/summaryTableAN_DMA_xs10_g1p0_2p1fb_exp.tex} \clearpage



\subsubsection{Projected sensitivities}

The expected 95\% CL signal strength limits at an integrated luminosity of 2~\ifb for
the available samples of the four light jet simplified dark matter models are
presented in
%Tables~\ref{tab:dm_V_g1_2fb_limits}-\ref{tab:dm_P_g1_2fb_limits}.
Tables~\ref{limits_DMV_xs10_g0p25_2p1fb_exp}-\ref{limits_DMP_xs10_g1p0_2p1fb_exp}.
Currently we are assuming an uncertainty of 20\% for the vector and axial-vector samples, and 30\% are used 
for the scalar and pseudo-scalar samples.


%Figures~\ref{fig:dm_A_g1_2fb_2dlimits} and \ref{fig:dm_P_g1_2fb_2dlimits} show
%the corresponding interpolated vector and pseudo-scalar expected exclusion 
%contours in the {\mphi-\mchi} mass plane. For all samples $g_{\rm}=0.25$ is used.



%gDM=1 results
\clearpage

%\begin{table}
\begin{center}
\caption{DMV $g_{\rm SM}=0.25$ 2.1\ifb exp 95\% CL upper limits}
\begin{tabular}{lccccccccccc}
\label{limits_DMV_xs10_g0p25_2p1fb_exp}
\multirow{7}{*}{\rotatebox{90}{$m_{\rm{DM}}$ (GeV)}}
& \multicolumn{1}{c|}{1000} &  &  &  &  &  &  &  & 920.03 &  & \\ 
& \multicolumn{1}{c|}{500} & 65.33 &  &  &  &  &  & 39.91 & 1.95 &  & 7.66e+04\\ 
& \multicolumn{1}{c|}{150} & 1.30 &  &  &  & 1.05 & 0.10 & 0.06 & 0.40 &  & 4.37e+04\\ 
& \multicolumn{1}{c|}{100} & 0.66 &  &  & 0.45 & -1.00 & 0.02 & -1.00 & -1.00 & -1.00 & 5.21e+04\\ 
& \multicolumn{1}{c|}{50} & 0.19 &  & 0.15 & 0.03 & 0.01 & 0.02 & -1.00 & 0.47 &  & 5.74e+04\\ 
& \multicolumn{1}{c|}{10} & 0.01 & -1.00 & 0.00 & 0.00 & 0.01 & -1.00 & 0.06 &  & 5.11 & \\ 
& \multicolumn{1}{c|}{1} & -1.00 & 0.00 & 0.00 & 0.00 & 0.01 & 0.01 & 0.06 & 0.40 & 5.35 & \\ 
\cline{2-12}
& \multicolumn{1}{c|}{} & 10 & 20 & 50 & 100 & 200 & 300 & 500 & 1000 & 2000 & 10000\\ 
& & \multicolumn{9}{c}{$M_{\rm{Med}}$ (GeV)}
\end{tabular}
\end{center}
\end{table}

%\begin{table}
\begin{center}
\caption{DMA 2.1\ifb exp 95\% CL upper limits}
\begin{tabular}{lccccccccccc}
\label{limits_DMA_xs10_g0p25_2p1fb_exp}
\multirow{6}{*}{\rotatebox{90}{$m_{\rm{DM}}$ (GeV)}}
& \multicolumn{1}{c|}{500} & 173.73 &  &  &  &  &  & 118.94 & 19.94 &  & \\ 
& \multicolumn{1}{c|}{150} &  &  &  &  & 2.33 & 0.84 &  & 0.44 &  & 6.35e+04\\ 
& \multicolumn{1}{c|}{100} & 1.49 &  &  & 0.75 &  &  & 0.06 &  &  & 5.49e+04\\ 
& \multicolumn{1}{c|}{50} & 0.34 &  &  &  &  &  & 0.05 & 0.33 & 5.64 & \\ 
& \multicolumn{1}{c|}{10} & 0.02 & 0.02 & 0.00 & 0.00 & 0.01 &  &  &  &  & \\ 
& \multicolumn{1}{c|}{1} & -1.00 &  & 0.00 & 0.00 & 0.01 & 0.01 & 0.06 &  & 4.99 & 4.65e+04\\ 
\cline{2-12}
& \multicolumn{1}{c|}{} & 10 & 20 & 50 & 100 & 200 & 300 & 500 & 1000 & 2000 & 10000\\ 
& & \multicolumn{9}{c}{$M_{\rm{Med}}$ (GeV)}
\end{tabular}
\end{center}
\end{table}
 
%\begin{table}
\footnotesize
\begin{center}
\caption{DMS xs10 g1p0 2p1fb exp 95\% CL upper limits}
\begin{tabular}{lcccccccccccc}
\label{limits_DMS_xs10_g1p0_2p1fb_exp}
\multirow{7}{*}{\rotatebox{90}{$m_{\rm{DM}}$ (GeV)}}
& \multicolumn{1}{c|}{1000} &  &  &  &  &  &  &  & 2.64e+06 & 2.46e+05 & 1.10e+07 & 5.71e+08\\ 
& \multicolumn{1}{c|}{500} &  &  &  &  &  &  & 2.99e+04 & 3.56e+03 & 4.37e+03 & 2.45e+06 & 5.52e+07\\ 
& \multicolumn{1}{c|}{150} &  &  &  &  & 162.38 & 51.75 & 4.73 & 66.24 & 2.72e+03 & 4.90e+05 & 8.59e+06\\ 
& \multicolumn{1}{c|}{100} &  &  &  & 98.03 & 46.44 & 1.06 & 3.02 & 66.29 & 2.48e+03 & 4.23e+05 & \\ 
& \multicolumn{1}{c|}{50} &  &  &  & 35.98 & 1.00 & 1.85 & 2.86 & 71.19 & 2.50e+03 & 3.71e+05 & 7.48e+06\\ 
& \multicolumn{1}{c|}{10} & 9.64 & 12.48 & 0.75 & 0.67 & 1.18 & 1.70 & 3.70 & 41.43 & 2.30e+03 & 3.72e+05 & 6.70e+06\\ 
& \multicolumn{1}{c|}{1} & 0.48 & 0.39 & 1.02 & 0.69 & 1.63 & 1.66 & 3.04 & 64.29 & 2.59e+03 & 3.26e+05 & 6.54e+06\\ 
\cline{2-13}
& \multicolumn{1}{c|}{} & 10 & 20 & 50 & 100 & 200 & 300 & 500 & 1000 & 2000 & 5000 & 10000\\ 
& & \multicolumn{10}{c}{$M_{\rm{Med}}$ (GeV)}
\end{tabular}
\end{center}
\end{table}
 
%\begin{table}
\footnotesize
\begin{center}
\caption{DMP xs10 g1p0 2p1fb exp 95\% CL upper limits}
\begin{tabular}{lcccccccccccc}
\label{limits_DMP_xs10_g1p0_2p1fb_exp}
\multirow{7}{*}{\rotatebox{90}{$m_{\rm{DM}}$ (GeV)}}
& \multicolumn{1}{c|}{1000} &  &  &  &  &  &  &  & 4.52e+05 & 1.65e+04 & 5.13e+06 & 1.71e+08\\ 
& \multicolumn{1}{c|}{500} &  &  &  &  &  &  & 6.53e+03 & 299.10 & 2.69e+03 & 1.18e+06 & -1.00\\ 
& \multicolumn{1}{c|}{150} &  &  &  &  & 24.33 & 4.19 & 2.32 & 0.31 & 1.55e+03 & 2.73e+05 & 4.62e+06\\ 
& \multicolumn{1}{c|}{100} &  &  &  & 19.60 & 4.91 & 0.92 & 1.67 & 46.34 &  & 2.14e+05 & 3.80e+06\\ 
& \multicolumn{1}{c|}{50} &  &  & 11.85 & 3.38 & 0.38 & 0.54 & 2.21 & 39.22 & 1.52e+03 & 2.07e+05 & 7.11e+03\\ 
& \multicolumn{1}{c|}{10} & 3.12 & 2.29 & 0.34 & 0.30 & 0.48 & 0.79 & 1.87 & 49.00 & 3.16 & 2.39e+05 & 4.87e+06\\ 
& \multicolumn{1}{c|}{1} & 0.11 & 0.09 & 0.31 &  & 0.45 &  & 1.53 & 39.06 & 1.76e+03 & 1.88e+05 & 3.29e+06\\ 
\cline{2-13}
& \multicolumn{1}{c|}{} & 10 & 20 & 50 & 100 & 200 & 300 & 500 & 1000 & 2000 & 5000 & 10000\\ 
& & \multicolumn{10}{c}{$M_{\rm{Med}}$ (GeV)}
\end{tabular}
\end{center}
\end{table}

 

\clearpage
%For comparison we show the expected sensitivity for $g_{\rm SM}=1.0$ for the axial-vector sample in Tab.~\ref{limits_DMA_xs10_g1p0_2p1fb_exp}.

%\begin{table}
\label{limits_DMA_xs10_g1p0_2p1fb_exp}
\begin{center}
\caption{DMA xs10 g1p0 2p1fb exp 95\% CL upper limits}
\begin{tabular}{lcccccccccccc}
\multirow{6}{*}{\rotatebox{90}{$m_{\rm{DM}}$ (GeV)}}
& \multicolumn{1}{c|}{500} &  &  &  &  &  &  & 10.08 &  &  & 757.99 & \\ 
& \multicolumn{1}{c|}{150} &  &  &  &  &  & 0.07 & 0.07 & 0.30 & 3.45 & 224.75 & \\ 
& \multicolumn{1}{c|}{100} &  &  &  & 0.06 &  & 0.02 & 0.04 &  & 3.36 & 224.20 & \\ 
& \multicolumn{1}{c|}{50} &  &  & 0.02 & 0.01 & 0.00 & 0.02 & 0.04 & 0.28 &  & 172.37 & 3.82e+03\\ 
& \multicolumn{1}{c|}{10} & 0.00 & 0.00 & 0.00 & 0.00 & 0.00 & 0.01 & 0.04 & 0.30 &  & 194.11 & 3.87e+03\\ 
& \multicolumn{1}{c|}{1} & -1.00 & -1.00 & 0.00 & 0.00 & 0.00 & 0.01 & 0.03 & 0.24 & 2.95 & 172.43 & 3.67e+03\\ 
\cline{2-13}
& \multicolumn{1}{c|}{} & 10 & 20 & 50 & 100 & 200 & 300 & 500 & 1000 & 2000 & 5000 & 10000\\ 
& & \multicolumn{10}{c}{$M_{\rm{Med}}$ (GeV)}
\end{tabular}
\end{center}
\end{table}
 
\clearpage




%\begin{figure}
%\begin{center}
%\includegraphics[width=0.75\textwidth]{figures/DMplots/DMV_finalCanvasExpLimit.pdf} \\
%\caption{Expected 95\% CL upper limit on the cross section, and exclusion
%contour for the vector light flavour model with unity couplings at 2~\ifb.}
%\label{fig:dm_A_g1_2fb_2dlimits}
%\end{center}
%\end{figure}





%\begin{figure}
%\begin{center}
%\includegraphics[width=0.75\textwidth]{figures/DMplots/DMP_finalCanvasExpLimit.pdf} \\
%\caption{Expected 95\% CL upper limit on the cross section, and exclusion
%contour for the pseudoscalar light flavour model with unity couplings at 2~\ifb.}
%\label{fig:dm_P_g1_2fb_2dlimits}
%\end{center}
%\end{figure}




\clearpage
\subsection{Heavy flavour models} \label{sec:dm_heavyjet}

Owing to the principal of Minimal Flavor Violation (MFV), top and bottom quarks
can play important roles in the phenomenology of dark matter. Scalar and
pseudoscalar models predict not only the `monojet' processes described in
Sec.~\ref{sec:dm_lightjet} but also the production of dark matter in association
with top (or bottom) pairs. This results in signatures with relatively large jet
multiplicities, in particular for \DMtt production. The \alphat analysis is well 
suited to searching for such signatures. An example Feynman diagram for the pair
production of dark matter particles in association with pairs of heavy quarks is
shown in Fig.~\ref{fig:feynman_hf}.


\begin{figure}[h!] \centering
\subfigure{\includegraphics[width=0.35\textwidth]{figures/DMplots/feynman_hf.pdf}}
\caption{Feynman diagram of the pair production of Dark Matter particles in
association with $t\bar{t}$ or $b\bar{b}$. \cite{Abercrombie:2015wmb}}
\label{fig:feynman_hf} \end{figure}


%The cross sections, signal yields and efficiencies for scalar and pseudoscalar
%\DMtt models expected with 2~\ifb of data are shown in 
%Tables~\ref{summaryTableAN_DMttP_xs10_2p1fb_exp}~and~\ref{summaryTableAN_DMttS_xs10_2p1fb_exp} for \DMtt. 

The selection efficiencies for these models are around $\sim 10$\%, and the improvement
provided by the new asymmetric and monojet categories is again evident.

\clearpage 
%\input{tables/DM/summaryTableAN_DMttP_xs10_2p1fb_exp.tex}
%\input{tables/DM/summaryTableAN_DMttS_xs10_2p1fb_exp.tex} 
\clearpage


%The yields and efficiencies for \DMbb are shown in Tables~\ref{summaryTableAN_DMbbP_xs10_2p1fb_exp}~and~\ref{summaryTableAN_DMbbS_xs10_2p1fb_exp}, respectively. 

%\begin{table}
\small
\centering
\begin{tabular}{lllllll}
\label{summaryTableAN_DMbbP_xs10_2p1fb_exp}
\hline
$m_\phi$ & $m_\chi$ & $\sigma$ [pb] & Yield (sym) & Yield (asy) & Yield (mon) & Efficiency [\%] \\ \hline
1000      &   1000      &   2.31e-09  &   0.00      &   0.00      &   0.00      &   12.97     \\ 
10        &   1000      &   1.52e-09  &   0.00      &   0.00      &   0.00      &   13.18     \\ 
100       &   10        &   5.36e-01  &   0.08      &   0.17      &   0.61      &   0.08      \\ 
10        &   10        &   9.36e-02  &   0.00      &   0.01      &   0.02      &   0.01      \\ 
15        &   10        &   1.30e-01  &   0.00      &   0.01      &   0.02      &   0.01      \\ 
50        &   10        &   1.94e+00  &   0.04      &   0.11      &   0.50      &   0.02      \\ 
200       &   150       &   1.52e-04  &   0.00      &   0.00      &   0.00      &   1.46      \\ 
295       &   150       &   1.67e-03  &   0.01      &   0.01      &   0.02      &   1.03      \\ 
500       &   150       &   1.26e-03  &   0.01      &   0.01      &   0.04      &   2.11      \\ 
1000      &   1         &   5.48e-05  &   0.00      &   0.00      &   0.00      &   4.22      \\ 
100       &   1         &   5.37e-01  &   0.04      &   0.21      &   0.79      &   0.09      \\ 
10        &   1         &   6.25e+00  &   0.02      &   0.01      &   0.22      &   0.00      \\ 
200       &   1         &   8.59e-02  &   0.10      &   0.15      &   0.40      &   0.36      \\ 
20        &   1         &   4.82e+00  &   0.00      &   0.00      &   0.27      &   0.00      \\ 
300       &   1         &   2.25e-02  &   0.04      &   0.08      &   0.27      &   0.82      \\ 
500       &   1         &   1.51e-03  &   0.01      &   0.01      &   0.04      &   1.89      \\ 
50        &   1         &   1.95e+00  &   0.07      &   0.17      &   0.31      &   0.01      \\ 
10        &   500       &   2.12e-07  &   0.00      &   0.00      &   0.00      &   7.26      \\ 
500       &   500       &   3.10e-07  &   0.00      &   0.00      &   0.00      &   7.30      \\ 
995       &   500       &   6.70e-06  &   0.00      &   0.00      &   0.00      &   6.17      \\ 
10        &   50        &   3.19e-03  &   0.00      &   0.00      &   0.01      &   0.20      \\ 
200       &   50        &   8.56e-02  &   0.09      &   0.14      &   0.39      &   0.34      \\ 
50        &   50        &   4.19e-03  &   0.00      &   0.00      &   0.01      &   0.21      \\ 
95        &   50        &   2.21e-02  &   0.01      &   0.02      &   0.03      &   0.12      \\ 
\hline
\end{tabular}
\caption{Cross section, yields at 2~\ifb (split according to symmetric, asymmetric, and monojet categories), and total selection efficiency for the pseudo-scalar \DMbb samples.}
\label{summaryTableAN_DMbbP_xs10_2p1fb_exp}
\end{table}

%\begin{table}
\small
\label{summaryTableAN_DMbbS_xs10_2p1fb_exp}
\centering
\begin{tabular}{lllllll}
\hline
$m_\phi$ & $m_\chi$ & $\sigma$ [pb] & Yield (sym) & Yield (asy) & Yield (mon) & Efficiency [\%] \\ \hline
1000      &   1000      &   4.09e-10  &   0.00      &   0.00      &   0.00      &   13.48     \\ 
10        &   1000      &   2.88e-10  &   0.00      &   0.00      &   0.00      &   13.87     \\ 
100       &   10        &   1.24e+00  &   0.29      &   0.32      &   1.41      &   0.08      \\ 
10        &   10        &   2.68e-01  &   0.01      &   0.03      &   0.07      &   0.02      \\ 
15        &   10        &   3.52e-01  &   0.02      &   0.04      &   0.10      &   0.02      \\ 
50        &   10        &   7.66e+00  &   0.29      &   0.81      &   2.34      &   0.02      \\ 
200       &   150       &   5.91e-05  &   0.00      &   0.00      &   0.00      &   1.78      \\ 
295       &   150       &   2.58e-04  &   0.00      &   0.00      &   0.00      &   1.43      \\ 
500       &   150       &   1.50e-03  &   0.01      &   0.01      &   0.04      &   2.04      \\ 
1000      &   1         &   5.41e-05  &   0.00      &   0.00      &   0.00      &   4.44      \\ 
100       &   1         &   1.25e+00  &   0.37      &   0.49      &   1.75      &   0.10      \\ 
200       &   1         &   1.31e-01  &   0.08      &   0.23      &   0.63      &   0.34      \\ 
20        &   1         &   4.07e+01  &   0.00      &   1.41      &   0.56      &   0.00      \\ 
300       &   1         &   2.93e-02  &   0.06      &   0.11      &   0.32      &   0.79      \\ 
500       &   1         &   2.21e-03  &   0.01      &   0.02      &   0.06      &   2.00      \\ 
50        &   1         &   7.66e+00  &   0.87      &   0.44      &   1.10      &   0.02      \\ 
10        &   500       &   5.36e-08  &   0.00      &   0.00      &   0.00      &   8.61      \\ 
500       &   500       &   7.40e-08  &   0.00      &   0.00      &   0.00      &   8.27      \\ 
995       &   500       &   6.51e-07  &   0.00      &   0.00      &   0.00      &   6.72      \\ 
10        &   50        &   2.71e-03  &   0.00      &   0.00      &   0.01      &   0.32      \\ 
50        &   50        &   3.37e-03  &   0.00      &   0.01      &   0.01      &   0.29      \\ 
95        &   50        &   1.04e-02  &   0.01      &   0.01      &   0.03      &   0.19      \\ 
\hline
\end{tabular}
\caption{Cross section, yields at 2~\ifb (split according to symmetric, asymmetric, and monojet categories), and total selection efficiency for the pseudo-scalar \DMtt samples.}
\label{summaryTableAN_DMbbS_xs10_2p1fb_exp}
\end{table}
 
\clearpage


\subsubsection{Expected and observed sensitivities for DM+$t\bar{t}$}

The expected 95\% CL signal strength limits for simplified \DMtt models with scalar and
pseudo-scalar couplings are calculated for 2~\ifb. An uncertainty of 20\% is assumed for all 
heavy quark samples.



\clearpage
%Expected limits obtained for DM+$t\bar{t}$ are given in Tables~\ref{limits_DMttP_xs10_2p1fb_exp}-\ref{limits_DMttS_xs10_2p1fb_exp}.
%\begin{table}
\begin{center}
\caption{DMttP 2.1\ifb exp 95\% CL upper limits}
\begin{tabular}{lcccccccc}
\label{limits_DMttP_xs10_2p1fb_exp}
\multirow{5}{*}{\rotatebox{90}{$m_{\rm{DM}}$ (GeV)}}
& \multicolumn{1}{c|}{500} &  &  &  &  &  &  & 2.83e+04\\ 
& \multicolumn{1}{c|}{150} &  &  &  &  & 509.12 &  & 37.06\\ 
& \multicolumn{1}{c|}{50} &  &  & 116.37 &  & 3.90 & 7.22 & \\ 
& \multicolumn{1}{c|}{10} & 42.02 &  & 2.27 & 2.80 &  &  & \\ 
& \multicolumn{1}{c|}{1} & 1.83 & 2.21 & 2.37 & 2.69 & 4.15 & 5.98 & 35.21\\ 
\cline{2-9}
& \multicolumn{1}{c|}{} & 10 & 20 & 50 & 100 & 200 & 300 & 500\\ 
& & \multicolumn{6}{c}{$M_{\rm{Med}}$ (GeV)}
\end{tabular}
\end{center}
\end{table}


%\begin{table}
\begin{center}
\caption{DMttS xs10 2p1fb exp 95\% CL upper limits}
\begin{tabular}{lccccccccc}
\label{limits_DMttS_xs10_2p1fb_exp}
\multirow{5}{*}{\rotatebox{90}{$m_{\rm{DM}}$ (GeV)}}
& \multicolumn{1}{c|}{500} &  &  &  &  &  &  & 9.38e+04 & \\ 
& \multicolumn{1}{c|}{150} &  &  &  &  & 1.24e+03 &  & 42.92 & \\ 
& \multicolumn{1}{c|}{50} &  &  & 207.56 &  & 4.29 & 8.25 &  & \\ 
& \multicolumn{1}{c|}{10} & 24.40 &  & 0.89 & 1.97 &  &  &  & \\ 
& \multicolumn{1}{c|}{1} & 0.38 & 0.49 & 0.77 & 2.06 & 4.97 & 8.77 & 34.88 & 304.60\\ 
\cline{2-10}
& \multicolumn{1}{c|}{} & 10 & 20 & 50 & 100 & 200 & 300 & 500 & 1000\\ 
& & \multicolumn{7}{c}{$M_{\rm{Med}}$ (GeV)}
\end{tabular}
\end{center}
\end{table}




Appendix~\ref{sec:dm_checklist} contains additional validation of the \DMj DM samples like signal and background yields for the most sensitive bins and sensitivities.

\subsubsection{Expected and observed sensitivities for DM+$b(\bar{b})$}

The expected 95\% CL signal strength limits for simplified DM+$t(\bar{t})$ models with scalar and
pseudo-scalar couplings are calculated for 2~\ifb. An uncertainty of 20\% is assumed for all 
heavy quark samples.


\clearpage
Expected limits obtained for DM+$t\bar{t}$ are given in Tables~\ref{limits_DMbbP_xs10_2p1fb_exp}-\ref{limits_DMbbS_xs10_2p1fb_exp}. 

%\begin{table}
\begin{center}
\tiny
\caption{DMbbP 2.1\ifb exp 95\% CL upper limits}
\begin{tabular}{lccccccccccccc}
\label{limits_DMbbP_xs10_2p1fb_exp}
\multirow{6}{*}{\rotatebox{90}{$m_{\rm{DM}}$ (GeV)}}
& \multicolumn{1}{c|}{1000} & 2.03e+08 &  &  &  &  &  &  &  &  &  &  & 1.44e+08\\ 
& \multicolumn{1}{c|}{500} & 3.76e+06 &  &  &  &  &  &  &  &  & 2.79e+06 & 1.67e+05 & \\ 
& \multicolumn{1}{c|}{150} &  &  &  &  &  &  & 3.14e+04 & 5.07e+03 &  & 3.16e+03 &  & \\ 
& \multicolumn{1}{c|}{50} & 6.87e+03 &  &  & 5.53e+03 & 1.50e+03 &  & 162.84 &  &  &  &  & \\ 
& \multicolumn{1}{c|}{10} & 1.60e+03 & 827.41 &  & 156.65 &  & 105.24 &  &  &  &  &  & \\ 
& \multicolumn{1}{c|}{1} & 177.94 &  & 78.47 & 70.34 &  & 93.38 & 180.58 &  & 415.85 & 2.64e+03 &  & 3.36e+04\\ 
\cline{2-14}
& \multicolumn{1}{c|}{} & 10 & 15 & 20 & 50 & 95 & 100 & 200 & 295 & 300 & 500 & 995 & 1000\\ 
& & \multicolumn{11}{c}{$M_{\rm{Med}}$ (GeV)}
\end{tabular}
\end{center}
\end{table}

%\begin{table}
\begin{center}
\tiny
\caption{DMbbS xs10 2p1fb exp 95\% CL upper limits}
\begin{tabular}{lccccccccccccc}
\label{limits_DMbbS_xs10_2p1fb_exp}
\multirow{6}{*}{\rotatebox{90}{$m_{\rm{DM}}$ (GeV)}}
& \multicolumn{1}{c|}{1000} & 1.03e+09 &  &  &  &  &  &  &  &  &  &  & 7.52e+08\\ 
& \multicolumn{1}{c|}{500} & 1.27e+07 &  &  &  &  &  &  &  &  & 1.08e+07 & 1.48e+06 & \\ 
& \multicolumn{1}{c|}{150} &  &  &  &  &  &  & 6.97e+04 & 2.12e+04 &  & 2.72e+03 &  & \\ 
& \multicolumn{1}{c|}{50} & 7.24e+03 &  &  & 5.96e+03 & 2.09e+03 &  &  &  &  &  &  & \\ 
& \multicolumn{1}{c|}{10} & 828.51 & 210.29 &  & 24.07 &  & 43.11 &  &  &  &  &  & \\ 
& \multicolumn{1}{c|}{1} & -1.00 &  & 21.19 & 11.30 &  & 38.25 & 161.04 &  & 290.34 & 1.85e+03 &  & 3.10e+04\\ 
\cline{2-14}
& \multicolumn{1}{c|}{} & 10 & 15 & 20 & 50 & 95 & 100 & 200 & 295 & 300 & 500 & 995 & 1000\\ 
& & \multicolumn{11}{c}{$M_{\rm{Med}}$ (GeV)}
\end{tabular}
\end{center}
\end{table}




Appendix~\ref{sec:dm_checklist} contains additional validation for the \DMtt and \DMtt DM samples like signal and background yields for the most sensitive bins and sensitivities.




\subsection{Delphes simulated models}

In a multidimensional analysis such as ours, it is not straightforward to
perform a reweighting of the coarse DM Forum grid to interpolate limits
across the entire plane. In addition, the 50,000 events generated per sample
provides insufficient statistics in certain analysis bins, which can lead to
erratic large-weight events inflating the sensitivity to these models.

Hence, we plan to utilise a finer grid of samples ($\sim200$ mass points per
model) with approximately 10 times more statistics, in order to aid the
interpolation of limits in the mass plane. These are produced privately using the same 
configurations in POWHEG-BOX as the CMS central samples. Pythia 8 is then used
for parton showering and hadronisation. Finally, the detector simulation is 
performed in Delphes 3 with a CMS detector card.

The validation of our Delphes-based production is shown in Fig.~\ref{fig:delphesVSofficial}. We
obtain very good agreement in all key analysis variables. Additionally, we
also obtain consistent limits -- for a pseudoscalar sample with \mchi$=50$~GeV
and \mphi$=1000$~GeV, generated with the same number of events and generator
cuts as the corresponding central sample, we obtain an upper limit on the
signal strength of 1.11 (for a `dummy' cross section of 10 pb), compared to 1.07 
for the central sample.

\begin{figure}
    \begin{center}
        \subfigure {\includegraphics[width=0.33\textwidth]{figures/DMplots/outputDelphesVsOfficial/athist.pdf}} ~~
        \subfigure {\includegraphics[width=0.33\textwidth]{figures/DMplots/outputDelphesVsOfficial/mhthist.pdf}} ~~
        \subfigure {\includegraphics[width=0.33\textwidth]{figures/DMplots/outputDelphesVsOfficial/njets.pdf}} ~~
        %\subfigure {\includegraphics[width=0.5\textwidth]{figures/distributions/DoubleMu/ht40_sym.pdf}} \\
        %\subfigure {\includegraphics[width=0.5\textwidth]{figures/distributions/DoubleMu/mht40_pt_sym.pdf}} ~~
        %\subfigure {\includegraphics[width=0.5\textwidth]{figures/distributions/DoubleMu/nBJetMedium40_sym.pdf}} \\
        \caption{Comparison of key analysis variables between Delphes and official CMS samples for a pseudoscalar model with \mchi$=50$~GeV and \mphi$=1000$~GeV}
        \label{fig:delphesVSofficial}
    \end{center}
\end{figure}

We show cross sections, yields, efficiencies, limits, and most sensitive 
categories for the four models simulated using Delphes in 
% FIXME: Tables~\ref{summary_ScorpionDMV}-\ref{msb_Scorpion_DMP}.
Tables~\ref{tab:summary_ScorpionDMA}-\ref{tab:msb_ScorpionDMP}.

%\input{tables/DM/summaryTableAN_ScorpionDMV_xs10_2p2fb_wMCStat_try2_exp_reduced}
\begin{table}
\footnotesize
\centering
\begin{tabular}{ccccccc}
\hline\hline
$m_\phi$ & $m_\chi$ & $\sigma$ [pb] & Yield (sym) & Yield (asy) & Yield (mon) & Efficiency [\%] \\ \hline
1000      &   150       &   4.07e+00  &   252.94    &   293.33    &   582.94    &   12.62     \\ 
10        &   150       &   7.14e-01  &   29.35     &   35.53     &   71.55     &   8.69      \\ 
1500      &   150       &   8.29e-01  &   62.59     &   69.70     &   136.76    &   14.76     \\ 
2000      &   150       &   2.20e-01  &   18.52     &   19.75     &   39.16     &   16.00     \\ 
500       &   150       &   3.23e+01  &   1277.49   &   1576.36   &   3159.76   &   8.47      \\ 
1000      &   1         &   4.52e+00  &   271.53    &   314.74    &   640.26    &   12.33     \\ 
10        &   1         &   3.87e+04  &   8882.12   &   14815.99  &   12139.32  &   4.21e-02  \\ 
1500      &   1         &   8.78e-01  &   65.21     &   72.27     &   143.37    &   14.54     \\ 
2000      &   1         &   2.31e-01  &   19.35     &   21.03     &   40.37     &   15.86     \\ 
500       &   1         &   4.88e+01  &   1859.62   &   2315.78   &   4695.14   &   8.27      \\ 
1000      &   250       &   3.36e+00  &   211.34    &   243.40    &   486.22    &   12.75     \\ 
10        &   250       &   1.28e-01  &   7.11      &   8.34      &   16.76     &   11.44     \\ 
1500      &   250       &   7.61e-01  &   58.35     &   64.96     &   127.15    &   14.96     \\ 
2000      &   250       &   2.07e-01  &   17.65     &   18.97     &   37.32     &   16.22     \\ 
1000      &   350       &   2.24e+00  &   140.59    &   163.24    &   323.28    &   12.75     \\ 
10        &   350       &   3.59e-02  &   2.36      &   2.69      &   5.37      &   13.19     \\ 
1500      &   350       &   6.67e-01  &   51.63     &   56.56     &   111.81    &   14.99     \\ 
2000      &   350       &   1.93e-01  &   16.35     &   17.63     &   34.97     &   16.28     \\ 
500       &   350       &   6.42e-02  &   4.07      &   4.73      &   9.29      &   12.81     \\ 
1000      &   500       &   7.64e-02  &   5.25      &   6.01      &   11.81     &   13.72     \\ 
10        &   500       &   7.93e-03  &   0.61      &   0.67      &   1.33      &   14.96     \\ 
1500      &   500       &   4.62e-01  &   35.76     &   39.68     &   77.80     &   15.07     \\ 
2000      &   500       &   1.62e-01  &   14.05     &   15.03     &   29.35     &   16.42     \\ 
500       &   500       &   1.04e-02  &   0.79      &   0.87      &   1.73      &   14.86     \\ 
\hline\hline
\end{tabular}
\caption{Cross section, yields at 2.2~\ifb (split according to symmetric, asymmetric, and monojet categories), and total selection efficiency for the Delphes axial-vector samples.}
\label{tab:summmary_ScorpionDMA}
\end{table}

\begin{table}
\footnotesize
\centering
\begin{tabular}{ccccccc}
\hline\hline
$m_\phi$ & $m_\chi$ & $\sigma$ [pb] & Yield (sym) & Yield (asy) & Yield (mon) & Efficiency [\%] \\ \hline
300       &   100       &   1.85e+00  &   76.62     &   93.40     &   86.59     &   6.29      \\ 
500       &   100       &   6.03e-01  &   33.90     &   37.37     &   31.50     &   7.74      \\ 
10        &   150       &   9.75e-03  &   0.56      &   0.60      &   0.52      &   7.84      \\ 
300       &   150       &   5.72e-02  &   2.68      &   3.12      &   2.71      &   6.76      \\ 
500       &   150       &   4.98e-01  &   28.03     &   30.82     &   25.94     &   7.73      \\ 
10        &   1         &   1.18e+01  &   144.06    &   198.94    &   195.53    &   2.08      \\ 
150       &   1         &   3.79e+00  &   104.28    &   135.60    &   134.67    &   4.49      \\ 
300       &   1         &   1.85e+00  &   76.86     &   91.46     &   84.71     &   6.22      \\ 
50        &   1         &   8.83e+00  &   134.41    &   181.73    &   177.99    &   2.54      \\ 
500       &   1         &   6.68e-01  &   37.00     &   41.03     &   34.73     &   7.67      \\ 
10        &   20        &   1.89e-01  &   5.20      &   6.52      &   6.20      &   4.31      \\ 
150       &   20        &   3.79e+00  &   106.37    &   135.36    &   134.16    &   4.50      \\ 
300       &   20        &   1.85e+00  &   75.87     &   91.45     &   85.24     &   6.21      \\ 
50        &   20        &   8.99e+00  &   132.74    &   186.09    &   183.86    &   2.54      \\ 
500       &   20        &   6.66e-01  &   36.79     &   41.43     &   34.56     &   7.70      \\ 
10        &   50        &   6.59e-02  &   2.63      &   3.11      &   2.86      &   5.94      \\ 
150       &   50        &   3.84e+00  &   106.29    &   139.01    &   135.69    &   4.51      \\ 
300       &   50        &   1.85e+00  &   76.39     &   93.05     &   85.80     &   6.28      \\ 
50        &   50        &   7.38e-02  &   2.84      &   3.42      &   3.15      &   5.80      \\ 
500       &   50        &   6.54e-01  &   36.31     &   40.21     &   34.18     &   7.70      \\ 
\hline\hline
\end{tabular}
\caption{Cross section, yields at 2.2~\ifb (split according to symmetric, asymmetric, and monojet categories), and total selection efficiency for the Delphes scalar samples.}
\label{tab:summmary_ScorpionDMS}
\end{table}

\begin{table}
\footnotesize
\centering
\begin{tabular}{ccccccc}
\hline\hline
$m_\phi$ & $m_\chi$ & $\sigma$ [pb] & Yield (sym) & Yield (asy) & Yield (mon) & Efficiency [\%] \\ \hline
300       &   100       &   5.55e+00  &   222.79    &   267.79    &   249.86    &   6.06      \\ 
10        &   150       &   5.05e-02  &   2.53      &   2.89      &   2.55      &   7.16      \\ 
300       &   150       &   1.68e+01  &   676.13    &   802.06    &   753.08    &   6.03      \\ 
500       &   150       &   8.77e-01  &   51.43     &   54.57     &   49.02     &   8.03      \\ 
10        &   1         &   2.66e+01  &   329.48    &   444.75    &   426.99    &   2.06      \\ 
10        &   1         &   2.66e+01  &   284.73    &   378.14    &   372.48    &   1.77      \\ 
150       &   1         &   8.94e+00  &   241.70    &   320.41    &   307.47    &   4.42      \\ 
300       &   1         &   5.57e+00  &   225.06    &   271.26    &   246.44    &   6.06      \\ 
50        &   1         &   2.00e+01  &   293.77    &   416.10    &   395.05    &   2.51      \\ 
500       &   1         &   1.02e+00  &   51.94     &   56.70     &   51.04     &   7.13      \\ 
10        &   20        &   6.03e-01  &   14.95     &   19.05     &   18.33     &   3.94      \\ 
150       &   20        &   8.95e+00  &   247.29    &   317.61    &   312.20    &   4.46      \\ 
300       &   20        &   5.58e+00  &   226.36    &   272.63    &   249.95    &   6.10      \\ 
50        &   20        &   2.01e+01  &   298.24    &   405.80    &   397.42    &   2.49      \\ 
500       &   20        &   1.02e+00  &   59.41     &   63.90     &   56.32     &   8.04      \\ 
10        &   50        &   2.43e-01  &   8.69      &   10.49     &   9.88      &   5.44      \\ 
150       &   50        &   8.94e+00  &   244.10    &   319.95    &   308.82    &   4.44      \\ 
300       &   50        &   5.57e+00  &   226.89    &   268.95    &   247.41    &   6.06      \\ 
50        &   50        &   2.83e-01  &   9.74      &   11.88     &   11.20     &   5.26      \\ 
500       &   50        &   1.01e+00  &   57.76     &   62.95     &   55.90     &   7.98      \\ 
\hline\hline
\end{tabular}
\caption{Cross section, yields at 2.2~\ifb (split according to symmetric, asymmetric, and monojet categories), and total selection efficiency for the Delphes pseudoscalar samples.}
\label{tab:summmary_ScorpionDMP}
\end{table}


%\input{tables/DM/limits_ScorpionDMV_xs10_2p2fb_wMCStat_try2_exp}
\begin{table}
\tiny
\renewcommand{\arraystretch}{2.0}\begin{center}
\caption{Delphes axial-vector expected 95\% CL upper limits for 2.2~\ifb}
\begin{tabular}{lcccccccccccccc}
\multirow{13}{*}{\rotatebox{90}{$m_{\rm{DM}}$ (GeV)}}
& \multicolumn{1}{c|}{1000} &  &  &  &  &  &  &  &  & 3.55e+03 &  &  &  & \\ 
& \multicolumn{1}{c|}{600} &  &  &  &  &  &  &  &  & 150.19 & 17.22 & 5.32 & 6.05 & 8.75\\ 
& \multicolumn{1}{c|}{500} & 185.56 &  &  &  &  &  & 144.20 & 95.35 & 23.76 & 2.85 & 3.14 & 4.66 & 7.37\\ 
& \multicolumn{1}{c|}{450} & 125.99 &  &  &  &  & 114.83 & 91.86 & 54.41 & 3.21 & 2.01 & 2.75 & 4.22 & 6.92\\ 
& \multicolumn{1}{c|}{400} & 83.05 &  &  &  &  & 73.96 & 56.75 & 23.66 & 1.44 & 1.60 & 2.45 & 3.96 & 6.52\\ 
& \multicolumn{1}{c|}{350} & 54.13 &  & 54.01 & 53.82 &  & 46.67 & 32.62 & 2.40 & 0.98 & 1.37 & 2.24 & 3.73 & 6.31\\ 
& \multicolumn{1}{c|}{300} & 34.35 &  & 33.91 & 33.34 &  & 28.04 & 16.30 & 0.69 &  &  &  &  & \\ 
& \multicolumn{1}{c|}{250} & 20.74 &  & 20.62 & 20.28 &  & 15.41 &  &  & 0.66 & 1.12 & 1.96 & 3.41 & 5.75\\ 
& \multicolumn{1}{c|}{200} & 12.00 &  & 11.94 & 11.50 &  & 7.39 &  &  &  &  &  &  & \\ 
& \multicolumn{1}{c|}{150} & 6.29 &  & 6.31 & 5.93 & 4.53 & 1.55 & 0.16 & 0.28 & 0.55 & 1.02 & 1.83 & 3.27 & 5.52\\ 
& \multicolumn{1}{c|}{50} & 1.12 &  & 1.00 & 0.43 & 0.04 & 0.05 &  &  &  &  &  &  & \\ 
& \multicolumn{1}{c|}{10} & 0.20 & 0.18 & 0.02 & 0.02 &  &  &  &  &  &  &  &  & \\ 
& \multicolumn{1}{c|}{1} & 0.01 & 0.01 & 0.02 & 0.02 & 0.03 & 0.05 & 0.11 &  & 0.51 & 0.96 & 1.77 & 3.15 & 5.52\\ 
\cline{2-15}
& \multicolumn{1}{c|}{} & 10 & 20 & 50 & 100 & 200 & 300 & 500 & 750 & 1000 & 1250 & 1500 & 1750 & 2000\\ 
& & \multicolumn{12}{c}{$M_{\rm{Med}}$ (GeV)}
\end{tabular}
\end{center}
\label{tab:limits_ScorpionDMA}
\end{table}

\begin{table}
\tiny
\renewcommand{\arraystretch}{2.0}\begin{center}
\caption{Delphes scalar expected 95\% CL upper limits for 2.2~\ifb}
\begin{tabular}{lccccccccccccc}
\multirow{10}{*}{\rotatebox{90}{$m_{\rm{DM}}$ (GeV)}}
& \multicolumn{1}{c|}{200} &  &  &  &  &  &  &  &  & 182.74 & 22.24 & 19.34 & 21.62\\ 
& \multicolumn{1}{c|}{150} & 539.98 &  &  &  &  & 392.74 &  & 130.36 & 6.92 & 9.01 & 11.60 & 15.21\\ 
& \multicolumn{1}{c|}{125} &  &  &  &  &  &  & 114.77 & 4.84 & 5.90 & 7.91 & 10.47 & 13.80\\ 
& \multicolumn{1}{c|}{100} &  &  &  &  &  & 97.08 & 4.52 & 4.89 & 5.56 & 7.31 & 9.79 & 12.79\\ 
& \multicolumn{1}{c|}{75} &  &  &  &  & 79.00 & 4.12 & 4.63 & 4.95 & 5.28 & 7.05 & 9.26 & 12.32\\ 
& \multicolumn{1}{c|}{50} & 140.25 &  & 127.53 & 63.09 & 3.77 & 4.19 & 4.64 & 4.81 & 5.24 & 6.76 & 9.17 & 11.98\\ 
& \multicolumn{1}{c|}{35} &  & 104.12 & 90.27 & 3.35 & 3.80 & 4.10 & 4.57 & 4.82 & 5.15 & 6.73 & 8.94 & 11.80\\ 
& \multicolumn{1}{c|}{20} & 73.45 & 70.87 & 3.05 & 3.39 & 3.73 & 4.20 & 4.64 & 4.86 & 5.24 & 6.77 & 8.90 & 11.86\\ 
& \multicolumn{1}{c|}{10} & 51.94 & 33.89 & 3.01 & 3.31 & 3.73 & 4.20 & 4.68 & 4.86 & 5.21 & 6.63 &  & 11.67\\ 
& \multicolumn{1}{c|}{1} & 2.78 & 2.96 & 3.03 & 3.37 & 3.80 & 4.20 & 4.68 & 4.86 & 5.15 & 6.59 & 8.84 & 11.76\\ 
\cline{2-14}
& \multicolumn{1}{c|}{} & 10 & 20 & 50 & 100 & 150 & 200 & 250 & 300 & 400 & 450 & 500 & 550\\ 
& & \multicolumn{11}{c}{$M_{\rm{Med}}$ (GeV)}
\end{tabular}
\end{center}
\label{tab:limits_ScorpionDMS}
\end{table}

\begin{table}
\tiny
\renewcommand{\arraystretch}{2.0}\begin{center}
\caption{Delphes pseudoscalar expected 95\% CL upper limits for 2.2~\ifb}
\begin{tabular}{lcccccccccccccccccc}
\multirow{11}{*}{\rotatebox{90}{$m_{\rm{DM}}$ (GeV)}}
& \multicolumn{1}{c|}{250} &  &  &  &  &  &  &  &  &  &  & 24.41 & 14.11 & 15.42 & 18.88 &  &  & -1.00\\ 
& \multicolumn{1}{c|}{200} &  &  &  &  &  &  &  &  & 9.93 & 6.11 & 7.47 & -1.00 & -1.00 & -1.00 &  &  & \\ 
& \multicolumn{1}{c|}{150} & 129.64 &  &  &  &  & 76.08 &  & 0.54 & 2.92 & 4.32 & 6.10 & 8.49 & 11.08 & 14.55 &  &  & \\ 
& \multicolumn{1}{c|}{125} &  &  &  &  &  &  & 0.52 & 1.64 & 2.70 & 4.11 &  & 8.02 & -1.00 & 14.11 &  &  & \\ 
& \multicolumn{1}{c|}{100} &  &  &  &  &  & 0.47 & 1.81 & 1.66 & 2.60 & 3.89 &  & -1.00 & -1.00 & 13.99 &  &  & \\ 
& \multicolumn{1}{c|}{75} &  &  &  &  & 0.42 & 1.76 & -1.00 & 1.64 & 2.53 & 3.85 & 5.59 & 7.51 & 10.54 & 13.79 &  &  & \\ 
& \multicolumn{1}{c|}{50} & 42.65 &  & 38.35 & 0.37 & 1.61 & 1.73 & 1.81 & 1.61 & 2.46 & 3.81 & 5.40 & 7.54 & 10.23 & 13.69 & 72.01 & 0.45 & \\ 
& \multicolumn{1}{c|}{35} &  & 33.54 & 27.66 & 1.50 & 1.64 & 1.75 & 1.80 & 1.67 &  &  &  &  &  &  &  &  & \\ 
& \multicolumn{1}{c|}{20} & 25.86 & 23.97 & 1.37 & 1.48 & 1.62 & 1.77 & 1.82 & 1.66 & 2.42 & 3.74 & 5.31 & 7.43 & 10.15 & 13.42 &  &  & \\ 
& \multicolumn{1}{c|}{10} & 18.80 & 0.31 & 1.36 & 1.48 &  &  &  &  &  &  &  &  &  &  &  &  & \\ 
& \multicolumn{1}{c|}{1} & 1.44 & 1.28 & 1.39 & 1.72 & 1.64 & 1.76 & 1.79 & 1.64 & -1.00 & -1.00 & 5.97 & 7.51 & 10.08 & 13.62 &  &  & \\ 
\cline{2-19}
& \multicolumn{1}{c|}{} & 10 & 20 & 50 & 100 & 150 & 200 & 250 & 300 & 400 & 450 & 500 & 550 & 600 & 650 & 1000 & 4500 & 5000\\ 
& & \multicolumn{16}{c}{$M_{\rm{Med}}$ (GeV)}
\end{tabular}
\end{center}
\label{tab:limits_ScorpionDMP}
\end{table}


%\input{tables/DM/msb_ScorpionDMV_xs10_2p2fb_wMCStat_try2_reduced}
\begin{table}
\begin{center}
\footnotesize
\caption{Most sensitive jet categories for Delphes axial-vector at 2.2~\ifb. The numbers quoted are the upper limits on the signal strength for a particular category alone}
\begin{tabular}{|l|c|c|c|c|}
\hline
 A mDM1 mPhi10 0p25 & eq4j: 0.03 & eq3j: 0.03 & eq2j: 0.03 & ge5j: 0.04 \\ 
 A mDM150 mPhi10 0p25 & eq1j: 9.48 & eq2j: 13.07 & eq2a: 14.75 & eq3j: 14.99 \\ 
 A mDM250 mPhi10 0p25 & eq1j: 31.66 & eq2j: 42.81 & eq2a: 47.53 & eq3j: 48.52 \\ 
 A mDM350 mPhi10 0p25 & eq1j: 84.64 & eq2j: 109.75 & eq3j: 123.77 & eq2a: 127.39 \\ 
 A mDM500 mPhi10 0p25 & eq1j: 299.31 & eq2j: 364.22 & eq3j: 410.27 & eq2a: 423.86 \\ 
 A mDM1 mPhi500 0p25 & eq1j: 0.16 & eq2j: 0.23 & eq3j: 0.27 & eq2a: 0.27 \\ 
 A mDM150 mPhi500 0p25 & eq1j: 0.23 & eq2j: 0.33 & eq2a: 0.38 & eq3j: 0.40 \\ 
 A mDM350 mPhi500 0p25 & eq1j: 51.69 & eq2j: 66.39 & eq2a: 74.61 & eq3j: 75.95 \\ 
 A mDM500 mPhi500 0p25 & eq1j: 235.40 & eq2j: 285.33 & eq3j: 326.76 & eq2a: 327.82 \\ 
 A mDM1 mPhi1000 0p25 & eq1j: 0.77 & eq2j: 1.05 & eq3j: 1.15 & eq2a: 1.19 \\ 
 A mDM150 mPhi1000 0p25 & eq1j: 0.85 & eq2j: 1.11 & eq3j: 1.28 & eq2a: 1.28 \\ 
 A mDM250 mPhi1000 0p25 & eq1j: 1.00 & eq2j: 1.35 & eq3j: 1.53 & eq2a: 1.54 \\ 
 A mDM350 mPhi1000 0p25 & eq1j: 1.51 & eq2j: 2.01 & eq3j: 2.24 & eq2a: 2.24 \\ 
 A mDM500 mPhi1000 0p25 & eq1j: 37.54 & eq2j: 47.39 & eq3j: 54.96 & eq2a: 55.54 \\ 
 A mDM1 mPhi1500 0p25 & eq1j: 2.83 & eq2j: 3.61 & eq3j: 3.90 & eq2a: 4.05 \\ 
 A mDM150 mPhi1500 0p25 & eq1j: 2.97 & eq2j: 3.67 & eq3j: 4.11 & eq2a: 4.21 \\ 
 A mDM250 mPhi1500 0p25 & eq1j: 3.18 & eq2j: 3.92 & eq3j: 4.36 & eq2a: 4.42 \\ 
 A mDM350 mPhi1500 0p25 & eq1j: 3.56 & eq2j: 4.49 & eq3j: 5.04 & eq2a: 5.17 \\ 
 A mDM500 mPhi1500 0p25 & eq1j: 5.09 & eq2j: 6.30 & eq3j: 7.09 & eq2a: 7.23 \\ 
 A mDM1 mPhi2000 0p25 & eq1j: 8.96 & eq2j: 11.00 & eq3j: 12.03 & eq2a: 12.48 \\ 
 A mDM150 mPhi2000 0p25 & eq1j: 9.13 & eq2j: 11.08 & eq3j: 12.26 & eq2a: 12.74 \\ 
 A mDM250 mPhi2000 0p25 & eq1j: 9.49 & eq2j: 11.76 & eq3j: 12.70 & eq2a: 13.02 \\ 
 A mDM350 mPhi2000 0p25 & eq1j: 10.22 & eq2j: 12.66 & eq3j: 13.90 & eq2a: 14.11 \\ 
 A mDM500 mPhi2000 0p25 & eq1j: 12.16 & eq2j: 14.68 & eq3j: 15.87 & eq2a: 16.68 \\ 
\hline
\end{tabular}
\end{center}
\label{tab:msb_ScorpionDMA}
\end{table}

\begin{table}
\begin{center}
\footnotesize
\caption{Most sensitive jet categories for Delphes scalar at 2.2~\ifb. The numbers quoted are the upper limits on the signal strength for a particular category alone}
\begin{tabular}{|l|c|c|c|c|}
\hline
 S mDM1 mPhi10 & eq2j: 4.71 & eq3j: 4.94 & eq3a: 5.86 & eq4j: 6.68 \\ 
 S mDM20 mPhi10 & eq3j: 125.97 & eq2j: 131.43 & eq4j: 158.72 & eq3a: 171.46 \\ 
 S mDM50 mPhi10 & eq3j: 235.49 & eq2j: 262.24 & eq4j: 288.33 & eq3a: 340.49 \\ 
 S mDM150 mPhi10 & eq3j: 9.39e+02 & eq4j: 1.03e+03 & eq2j: 1.09e+03 & ge5j: 1.28e+03 \\ 
 S mDM1 mPhi50 & eq2j: 5.08 & eq3j: 5.35 & eq3a: 6.70 & eq4j: 7.18 \\ 
 S mDM20 mPhi50 & eq2j: 5.37 & eq3j: 5.56 & eq3a: 6.37 & eq4j: 6.75 \\ 
 S mDM50 mPhi50 & eq3j: 216.03 & eq2j: 243.40 & eq4j: 257.39 & ge5j: 313.29 \\ 
 S mDM1 mPhi150 & eq3j: 6.55 & eq2j: 6.60 & eq4j: 8.55 & eq3a: 8.55 \\ 
 S mDM20 mPhi150 & eq3j: 6.23 & eq2j: 6.68 & eq4j: 8.31 & eq3a: 8.76 \\ 
 S mDM50 mPhi150 & eq2j: 6.52 & eq3j: 6.65 & eq4j: 8.49 & eq3a: 8.58 \\ 
 S mDM1 mPhi300 & eq3j: 8.30 & eq2j: 8.81 & eq4j: 10.00 & eq3a: 11.67 \\ 
 S mDM20 mPhi300 & eq3j: 8.16 & eq2j: 8.91 & eq4j: 10.10 & eq3a: 11.49 \\ 
 S mDM50 mPhi300 & eq3j: 8.30 & eq2j: 8.54 & eq4j: 10.28 & eq3a: 11.49 \\ 
 S mDM100 mPhi300 & eq3j: 8.23 & eq2j: 8.83 & eq4j: 10.24 & eq3a: 11.91 \\ 
 S mDM150 mPhi300 & eq3j: 223.16 & eq2j: 242.71 & eq4j: 256.21 & eq3a: 323.07 \\ 
 S mDM1 mPhi500 & eq3j: 15.04 & eq2j: 16.97 & eq4j: 17.29 & ge5j: 22.43 \\ 
 S mDM20 mPhi500 & eq3j: 15.16 & eq4j: 17.35 & eq2j: 17.35 & ge5j: 21.61 \\ 
 S mDM50 mPhi500 & eq3j: 15.45 & eq2j: 17.56 & eq4j: 18.02 & ge5j: 22.68 \\ 
 S mDM100 mPhi500 & eq3j: 16.45 & eq2j: 19.15 & eq4j: 19.58 & ge5j: 23.71 \\ 
 S mDM150 mPhi500 & eq3j: 19.40 & eq4j: 22.68 & eq2j: 22.68 & ge5j: 28.03 \\ 
\hline
\end{tabular}
\end{center}
\label{tab:msb_ScorpionDMS}
\end{table}


\begin{table}
\begin{center}
\footnotesize
\caption{Most sensitive jet categories for Delphes pseudoscalar at 2.2~\ifb. The numbers quoted are the upper limits on the signal strength for a particular category alone}
\begin{tabular}{|l|c|c|c|c|}
\hline
 P mDM1 mPhi10 & eq2j: 2.18 & eq3j: 2.32 & eq3a: 2.73 & eq4j: 2.81 \\ 
 P mDM1 mPhi10 PU40 & eq2j: 2.38 & eq3j: 2.54 & eq3a: 3.25 & eq4j: 3.48 \\ 
 P mDM20 mPhi10 & eq3j: 45.17 & eq2j: 46.03 & eq4j: 54.86 & eq3a: 59.57 \\ 
 P mDM50 mPhi10 & eq3j: 70.62 & eq2j: 81.06 & eq4j: 88.14 & eq3a: 103.71 \\ 
 P mDM150 mPhi10 & eq3j: 223.58 & eq2j: 245.69 & eq4j: 255.87 & ge5j: 318.77 \\ 
 P mDM1 mPhi50 & eq2j: 2.33 & eq3j: 2.60 & eq3a: 3.00 & eq4j: 3.17 \\ 
 P mDM20 mPhi50 & eq2j: 2.37 & eq3j: 2.44 & eq3a: 2.95 & eq4j: 3.11 \\ 
 P mDM50 mPhi50 & eq3j: 65.18 & eq2j: 70.64 & eq4j: 79.73 & eq3a: 92.47 \\ 
 P mDM1 mPhi150 & eq3j: 2.83 & eq2j: 2.91 & eq3a: 3.66 & eq4j: 3.67 \\ 
 P mDM20 mPhi150 & eq3j: 2.78 & eq2j: 2.85 & eq3a: 3.62 & eq4j: 3.64 \\ 
 P mDM50 mPhi150 & eq3j: 2.74 & eq2j: 2.95 & eq4j: 3.57 & eq3a: 3.58 \\ 
 P mDM1 mPhi300 & eq3j: 2.72 & eq2j: 3.02 & eq4j: 3.56 & eq3a: 3.95 \\ 
 P mDM20 mPhi300 & eq3j: 2.83 & eq2j: 2.97 & eq4j: 3.47 & eq3a: 3.96 \\ 
 P mDM50 mPhi300 & eq3j: 2.72 & eq2j: 2.94 & eq4j: 3.35 & eq3a: 4.03 \\ 
 P mDM100 mPhi300 & eq3j: 2.92 & eq2j: 3.01 & eq4j: 3.33 & eq3a: 3.98 \\ 
 P mDM150 mPhi300 & eq3j: 0.92 & eq2j: 0.96 & eq4j: 1.18 & eq3a: 1.35 \\ 
 P mDM1 mPhi500 PU40 & eq3j: 10.26 & eq2j: 10.85 & eq4j: 12.20 & ge5j: 15.24 \\ 
 P mDM20 mPhi500 & eq3j: 9.18 & eq2j: 10.28 & eq4j: 10.53 & ge5j: 12.48 \\ 
 P mDM50 mPhi500 & eq3j: 9.10 & eq2j: 10.72 & eq4j: 10.81 & ge5j: 12.77 \\ 
 P mDM150 mPhi500 & eq3j: 10.53 & eq4j: 11.91 & eq2j: 12.10 & ge5j: 14.16 \\ 
\hline
\end{tabular}
\end{center}
\label{tab:msb_ScorpionDMP}
\end{table}

The N$_{j}$ categories are ranked according to sensitivity, a ranking of 1 =  most sensitive category, for the entire plane of DM mass vs mediator mass for all 4 models. These are shown in Figure~\ref{fig:sensitivityAV}. The most sensitive jet categories for the vector and axialvector models are the lowest jet multiplicities, Nj=1, Nj=2, while for the scalar and pseudoscalar models the most sensitive categories are Nj=2, Nj=3. 

\begin{figure}[h!] \centering
  \subfigure{\includegraphics[width=0.65\textwidth]{figures/DMplots/jetrankings_AV.pdf}}
  \caption{Sensitivity ranking of the jet categories for the axial vector model, a ranking of 1 denotes the most sensitive category. }
  \label{fig:sensitivityAV} 
\end{figure}

\begin{figure}[h!] \centering
  \subfigure{\includegraphics[width=0.65\textwidth]{figures/DMplots/jetrankings_PS.pdf}}
  \caption{Sensitivity ranking of the jet categories for the pseudoscalar model, a ranking of 1 denotes the most sensitive category. }
  \label{fig:sensitivityPS} 
\end{figure}

