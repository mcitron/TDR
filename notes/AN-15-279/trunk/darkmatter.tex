s%____________________________________________________________________________||
\section{Interpretation in Dark Matter models} \label{sec:darkmatter}



\subsection{Light flavour models} \label{sec:dm_lightjet}

The light flavour simplified models consist of a DM particle \pchi of mass
\mchi that is a Dirac fermion, and a spin-1 (vector or axial-vector) or spin-0
(scalar or pseudoscalar) mediating particle \pphi of mass \mphi in an
$s$-channel. The couplings of the mediator with the standard model and dark
matter particles are given by \gsm and \gdm, respectively. The recommendations
by the DM Forum on the choice of couplings is \gsm$=1$,\gdm$=1$ for
(pseudo)scalar models, and \gsm$=0.25$,\gdm$=1$ for (axial-)vector models.
Assuming that no additional visible or invisible particles contribute to the decay 
of the mediator, we impose the minimal width determined by the choice of couplings. 
An example Feynman diagram is shown in Fig.~\ref{fig:DMfeynman}.


\begin{figure}[h!] \centering
  \subfigure{\includegraphics[width=0.35\textwidth]{figures/DMplots/feynman_light_jet.pdf}}
  \caption{Feynman diagram of DM pair production in light jet hadronic final states. \cite{Abercrombie:2015wmb}}
  \label{fig:DMfeynman} 
\end{figure}



Assuming 2~\ifb of data we list the cross sections, yields and selection 
efficiencies for the four light jet models in Tables~\ref{summaryTableAN_DMV_xs10_g0p25_2p1fb_exp}-\ref{summaryTableAN_DMS_xs10_g1p0_2p1fb_exp}. The signal selection
efficiencies are around $\sim 1$\% for mass points near the expected exclusion
region, and are correspondingly larger (smaller) for higher (lower) mass points.
The asymmetric and monojet categories are seen to almost double the acceptance
to these models compared to the Run~1 symmetric categories alone, justifying the
inclusion of these selections into the analysis.




\clearpage 
%\begin{table}
\small
\centering
\begin{tabular}{lllllll}
\hline
$m_\phi$ & $m_\chi$ & $\sigma$ [pb] & Yield (sym) & Yield (asy) & Yield (mon) & Efficiency [\%] \\ \hline
10000     &   100       &   1.99e-04  &   0.00      &   0.00      &   0.00      &   1.45      \\ 
10000     &   150       &   1.80e-04  &   0.00      &   0.00      &   0.00      &   1.59      \\ 
10000     &   500       &   7.68e-05  &   0.00      &   0.00      &   0.00      &   2.11      \\ 
10000     &   50        &   2.09e-04  &   0.00      &   0.00      &   0.00      &   1.12      \\ 
1000      &   1000      &   5.04e-03  &   0.06      &   0.07      &   0.13      &   2.38      \\ 
1000      &   100       &   2.73e+01  &   179.52    &   190.01    &   383.56    &   1.31      \\ 
1000      &   150       &   2.62e+01  &   160.27    &   208.11    &   403.66    &   1.40      \\ 
1000      &   1         &   2.71e+01  &   160.57    &   203.17    &   434.80    &   1.40      \\ 
1000      &   500       &   4.97e+00  &   33.21     &   47.36     &   81.68     &   1.55      \\ 
100       &   10        &   5.93e+04  &   7589.50   &   19108.63  &   31108.83  &   0.05      \\ 
100       &   1         &   5.91e+04  &   14800.01  &   12847.15  &   11821.32  &   0.03      \\ 
100       &   50        &   3.88e+03  &   539.13    &   1807.91   &   4031.74   &   0.08      \\ 
10        &   100       &   5.54e+01  &   103.87    &   126.63    &   248.14    &   0.41      \\ 
10        &   10        &   5.17e+04  &   1145.90   &   332.97    &   2451.58   &   0.00      \\ 
10        &   150       &   1.37e+01  &   47.10     &   58.73     &   111.65    &   0.76      \\ 
10        &   500       &   1.41e-01  &   1.16      &   1.34      &   2.76      &   1.78      \\ 
10        &   50        &   4.67e+02  &   230.50    &   378.34    &   743.68    &   0.14      \\ 
2000      &   1         &   1.21e+00  &   11.66     &   12.00     &   24.85     &   1.92      \\ 
200       &   10        &   6.95e+03  &   5415.53   &   7354.07   &   15915.05  &   0.20      \\ 
200       &   150       &   2.67e+01  &   73.94     &   84.18     &   215.47    &   0.67      \\ 
200       &   1         &   7.31e+03  &   5121.26   &   7686.21   &   18636.29  &   0.20      \\ 
200       &   1         &   7.31e+03  &   6238.76   &   9708.29   &   17140.23  &   0.22      \\ 
200       &   50        &   7.21e+03  &   4068.85   &   7543.76   &   16796.86  &   0.19      \\ 
20        &   1         &   3.89e+06  &   72829.16  &   0.00      &   0.00      &   0.00      \\ 
300       &   100       &   1.92e+03  &   3361.32   &   3813.36   &   9551.46   &   0.42      \\ 
300       &   10        &   2.14e+03  &   3897.55   &   4065.27   &   11657.41  &   0.44      \\ 
300       &   150       &   2.65e+02  &   500.37    &   656.11    &   1441.81   &   0.47      \\ 
300       &   1         &   2.07e+03  &   3791.08   &   4448.85   &   9664.55   &   0.41      \\ 
300       &   50        &   2.04e+03  &   3769.22   &   4258.83   &   10451.88  &   0.43      \\ 
500       &   100       &   3.42e+02  &   966.21    &   1385.64   &   3075.59   &   0.76      \\ 
500       &   10        &   3.45e+02  &   1118.46   &   1620.52   &   2888.59   &   0.78      \\ 
500       &   150       &   3.28e+02  &   985.14    &   1458.16   &   2740.83   &   0.75      \\ 
500       &   1         &   3.23e+02  &   1325.55   &   1776.04   &   3250.20   &   0.94      \\ 
500       &   500       &   1.97e-01  &   1.54      &   1.88      &   3.85      &   1.76      \\ 
500       &   50        &   3.43e+02  &   1013.67   &   1413.07   &   3283.48   &   0.79      \\ 
50        &   10        &   3.83e+05  &   4924.12   &   9435.89   &   10145.19  &   0.00      \\ 
50        &   1         &   3.96e+05  &   16034.59  &   32044.78  &   15910.13  &   0.01      \\ 
50        &   50        &   6.32e+02  &   360.15    &   560.45    &   947.73    &   0.14      \\ 
\hline
\end{tabular}
\caption{Cross section, yields at 2~\ifb (split according to symmetric, asymmetric, and monojet categories), and total selection efficiency for the vector \DMj samples.}
\label{summaryTableAN_DMV_xs10_g0p25_2p1fb_exp}
\end{table}
 \clearpage
%\input{tables/DM/summaryTableAN_DMA_xs10_g0p25_2p1fb_exp.tex} \clearpage
%\input{tables/DM/summaryTableAN_DMP_xs10_g1p0_2p1fb_exp.tex} \clearpage
%\begin{table}
\footnotesize
\centering
\begin{tabular}{lllllll}
\hline
$m_\phi$ & $m_\chi$ & $\sigma$ [pb] & Yield (sym) & Yield (asy) & Yield (mon) & Efficiency [\%] \\ \hline
10000     &   1000      &   6.76e-09  &   0.00      &   0.00      &   0.00      &   2.31      \\ 
10000     &   10        &   2.34e-06  &   0.00      &   0.00      &   0.00      &   0.76      \\ 
10000     &   150       &   1.34e-06  &   0.00      &   0.00      &   0.00      &   1.00      \\ 
10000     &   1         &   2.36e-06  &   0.00      &   0.00      &   0.00      &   0.74      \\ 
10000     &   500       &   1.08e-07  &   0.00      &   0.00      &   0.00      &   1.83      \\ 
10000     &   50        &   2.18e-06  &   0.00      &   0.00      &   0.00      &   0.82      \\ 
1000      &   1000      &   1.62e-06  &   0.00      &   0.00      &   0.00      &   2.29      \\ 
1000      &   100       &   1.84e-01  &   1.15      &   1.35      &   1.80      &   1.12      \\ 
1000      &   10        &   1.93e-01  &   1.53      &   1.37      &   1.73      &   1.14      \\ 
1000      &   150       &   1.68e-01  &   1.14      &   1.24      &   1.34      &   1.05      \\ 
1000      &   1         &   1.97e-01  &   1.29      &   1.49      &   2.00      &   1.15      \\ 
1000      &   500       &   2.58e-03  &   0.02      &   0.02      &   0.03      &   1.33      \\ 
1000      &   50        &   1.94e-01  &   1.35      &   1.49      &   1.82      &   1.15      \\ 
100       &   100       &   3.30e-01  &   0.69      &   1.12      &   1.34      &   0.45      \\ 
100       &   10        &   1.14e+02  &   66.39     &   112.95    &   145.53    &   0.14      \\ 
100       &   1         &   1.14e+02  &   73.56     &   89.42     &   125.17    &   0.12      \\ 
100       &   50        &   2.04e+00  &   2.07      &   3.23      &   3.86      &   0.21      \\ 
10        &   10        &   9.95e+00  &   4.64      &   7.25      &   7.81      &   0.09      \\ 
10        &   1         &   1.16e+03  &   33.19     &   105.71    &   120.59    &   0.01      \\ 
2000      &   1000      &   1.99e-05  &   0.00      &   0.00      &   0.00      &   2.04      \\ 
2000      &   100       &   2.92e-03  &   0.03      &   0.03      &   0.03      &   1.43      \\ 
2000      &   10        &   3.26e-03  &   0.03      &   0.03      &   0.03      &   1.37      \\ 
2000      &   150       &   2.61e-03  &   0.03      &   0.03      &   0.03      &   1.53      \\ 
2000      &   1         &   3.27e-03  &   0.03      &   0.03      &   0.04      &   1.38      \\ 
2000      &   500       &   1.16e-03  &   0.02      &   0.01      &   0.02      &   1.97      \\ 
2000      &   50        &   3.17e-03  &   0.03      &   0.03      &   0.03      &   1.40      \\ 
200       &   100       &   8.18e-01  &   1.47      &   2.16      &   3.13      &   0.39      \\ 
200       &   10        &   3.95e+01  &   52.07     &   83.83     &   95.54     &   0.28      \\ 
200       &   150       &   1.70e-01  &   0.42      &   0.63      &   0.73      &   0.50      \\ 
200       &   1         &   3.18e+01  &   48.47     &   60.01     &   87.26     &   0.29      \\ 
200       &   50        &   3.95e+01  &   52.99     &   74.38     &   121.02    &   0.30      \\ 
20        &   10        &   1.21e+01  &   5.11      &   6.31      &   10.55     &   0.09      \\ 
20        &   1         &   7.18e+02  &   54.37     &   100.26    &   202.57    &   0.02      \\ 
300       &   100       &   2.32e+01  &   38.43     &   72.30     &   87.53     &   0.41      \\ 
300       &   10        &   2.32e+01  &   48.98     &   68.83     &   77.31     &   0.40      \\ 
300       &   150       &   4.86e-01  &   1.21      &   1.63      &   1.76      &   0.45      \\ 
300       &   1         &   2.32e+01  &   43.48     &   62.04     &   88.43     &   0.40      \\ 
300       &   50        &   2.32e+01  &   42.00     &   65.55     &   97.24     &   0.42      \\ 
5000      &   1000      &   3.29e-07  &   0.00      &   0.00      &   0.00      &   2.47      \\ 
5000      &   100       &   3.02e-05  &   0.00      &   0.00      &   0.00      &   0.93      \\ 
5000      &   10        &   3.94e-05  &   0.00      &   0.00      &   0.00      &   0.89      \\ 
5000      &   150       &   2.24e-05  &   0.00      &   0.00      &   0.00      &   1.07      \\ 
5000      &   1         &   3.94e-05  &   0.00      &   0.00      &   0.00      &   0.81      \\ 
5000      &   500       &   2.41e-06  &   0.00      &   0.00      &   0.00      &   1.83      \\ 
5000      &   50        &   3.72e-05  &   0.00      &   0.00      &   0.00      &   0.85      \\ 
500       &   100       &   6.69e+00  &   20.54     &   27.96     &   32.65     &   0.58      \\ 
500       &   10        &   7.47e+00  &   19.03     &   27.47     &   37.66     &   0.54      \\ 
500       &   150       &   5.56e+00  &   20.23     &   23.12     &   25.59     &   0.59      \\ 
500       &   1         &   7.47e+00  &   21.75     &   28.18     &   35.85     &   0.55      \\ 
500       &   500       &   3.02e-04  &   0.00      &   0.00      &   0.00      &   1.51      \\ 
500       &   50        &   7.31e+00  &   22.52     &   28.77     &   33.05     &   0.55      \\ 
50        &   10        &   2.87e+02  &   83.24     &   73.88     &   131.87    &   0.05      \\ 
50        &   1         &   2.87e+02  &   52.47     &   109.61    &   155.16    &   0.05      \\ 
\hline
\end{tabular}
\caption{Cross section, yields at 2~\ifb (split according to symmetric, asymmetric, and monojet categories), and total selection efficiency for the pseudo-scalar \DMtt samples.}
\label{summaryTableAN_DMS_xs10_g1p0_2p1fb_exp}
\end{table}
 \clearpage

\clearpage
For comparisons with past results we also include the axial-vector samples using $g_{\rm SM} = 1.0$ although they are not included in the 
'ATLAS-CMS-Pheno DM' Forum. The key characteristics for this sample are shown in Table~\ref{summaryTableAN_DMA_xs10_g1p0_2p1fb_exp}.

%\input{tables/DM/summaryTableAN_DMA_xs10_g1p0_2p1fb_exp.tex} \clearpage



\subsubsection{Projected sensitivities}

The expected 95\% CL signal strength limits at an integrated luminosity of 2~\ifb for
the available samples of the four light jet simplified dark matter models are
presented in
%Tables~\ref{tab:dm_V_g1_2fb_limits}-\ref{tab:dm_P_g1_2fb_limits}.
Tables~\ref{limits_DMV_xs10_g0p25_2p1fb_exp}-\ref{limits_DMP_xs10_g1p0_2p1fb_exp}.
Currently we are assuming an uncertainty of 20\% for the vector and axial-vector samples, and 30\% are used 
for the scalar and pseudo-scalar samples.


Figures~\ref{fig:dm_A_g1_2fb_2dlimits} and \ref{fig:dm_P_g1_2fb_2dlimits} show
the corresponding interpolated vector and pseudo-scalar expected exclusion 
contours in the {\mphi-\mchi} mass plane. For all samples $g_{\rm}=0.25$ is used.



%gDM=1 results
\clearpage

%\begin{table}
\begin{center}
\caption{DMV $g_{\rm SM}=0.25$ 2.1\ifb exp 95\% CL upper limits}
\begin{tabular}{lccccccccccc}
\label{limits_DMV_xs10_g0p25_2p1fb_exp}
\multirow{7}{*}{\rotatebox{90}{$m_{\rm{DM}}$ (GeV)}}
& \multicolumn{1}{c|}{1000} &  &  &  &  &  &  &  & 920.03 &  & \\ 
& \multicolumn{1}{c|}{500} & 65.33 &  &  &  &  &  & 39.91 & 1.95 &  & 7.66e+04\\ 
& \multicolumn{1}{c|}{150} & 1.30 &  &  &  & 1.05 & 0.10 & 0.06 & 0.40 &  & 4.37e+04\\ 
& \multicolumn{1}{c|}{100} & 0.66 &  &  & 0.45 & -1.00 & 0.02 & -1.00 & -1.00 & -1.00 & 5.21e+04\\ 
& \multicolumn{1}{c|}{50} & 0.19 &  & 0.15 & 0.03 & 0.01 & 0.02 & -1.00 & 0.47 &  & 5.74e+04\\ 
& \multicolumn{1}{c|}{10} & 0.01 & -1.00 & 0.00 & 0.00 & 0.01 & -1.00 & 0.06 &  & 5.11 & \\ 
& \multicolumn{1}{c|}{1} & -1.00 & 0.00 & 0.00 & 0.00 & 0.01 & 0.01 & 0.06 & 0.40 & 5.35 & \\ 
\cline{2-12}
& \multicolumn{1}{c|}{} & 10 & 20 & 50 & 100 & 200 & 300 & 500 & 1000 & 2000 & 10000\\ 
& & \multicolumn{9}{c}{$M_{\rm{Med}}$ (GeV)}
\end{tabular}
\end{center}
\end{table}

%\begin{table}
\begin{center}
\caption{DMA 2.1\ifb exp 95\% CL upper limits}
\begin{tabular}{lccccccccccc}
\label{limits_DMA_xs10_g0p25_2p1fb_exp}
\multirow{6}{*}{\rotatebox{90}{$m_{\rm{DM}}$ (GeV)}}
& \multicolumn{1}{c|}{500} & 173.73 &  &  &  &  &  & 118.94 & 19.94 &  & \\ 
& \multicolumn{1}{c|}{150} &  &  &  &  & 2.33 & 0.84 &  & 0.44 &  & 6.35e+04\\ 
& \multicolumn{1}{c|}{100} & 1.49 &  &  & 0.75 &  &  & 0.06 &  &  & 5.49e+04\\ 
& \multicolumn{1}{c|}{50} & 0.34 &  &  &  &  &  & 0.05 & 0.33 & 5.64 & \\ 
& \multicolumn{1}{c|}{10} & 0.02 & 0.02 & 0.00 & 0.00 & 0.01 &  &  &  &  & \\ 
& \multicolumn{1}{c|}{1} & -1.00 &  & 0.00 & 0.00 & 0.01 & 0.01 & 0.06 &  & 4.99 & 4.65e+04\\ 
\cline{2-12}
& \multicolumn{1}{c|}{} & 10 & 20 & 50 & 100 & 200 & 300 & 500 & 1000 & 2000 & 10000\\ 
& & \multicolumn{9}{c}{$M_{\rm{Med}}$ (GeV)}
\end{tabular}
\end{center}
\end{table}
 
%\begin{table}
\footnotesize
\begin{center}
\caption{DMS xs10 g1p0 2p1fb exp 95\% CL upper limits}
\begin{tabular}{lcccccccccccc}
\label{limits_DMS_xs10_g1p0_2p1fb_exp}
\multirow{7}{*}{\rotatebox{90}{$m_{\rm{DM}}$ (GeV)}}
& \multicolumn{1}{c|}{1000} &  &  &  &  &  &  &  & 2.64e+06 & 2.46e+05 & 1.10e+07 & 5.71e+08\\ 
& \multicolumn{1}{c|}{500} &  &  &  &  &  &  & 2.99e+04 & 3.56e+03 & 4.37e+03 & 2.45e+06 & 5.52e+07\\ 
& \multicolumn{1}{c|}{150} &  &  &  &  & 162.38 & 51.75 & 4.73 & 66.24 & 2.72e+03 & 4.90e+05 & 8.59e+06\\ 
& \multicolumn{1}{c|}{100} &  &  &  & 98.03 & 46.44 & 1.06 & 3.02 & 66.29 & 2.48e+03 & 4.23e+05 & \\ 
& \multicolumn{1}{c|}{50} &  &  &  & 35.98 & 1.00 & 1.85 & 2.86 & 71.19 & 2.50e+03 & 3.71e+05 & 7.48e+06\\ 
& \multicolumn{1}{c|}{10} & 9.64 & 12.48 & 0.75 & 0.67 & 1.18 & 1.70 & 3.70 & 41.43 & 2.30e+03 & 3.72e+05 & 6.70e+06\\ 
& \multicolumn{1}{c|}{1} & 0.48 & 0.39 & 1.02 & 0.69 & 1.63 & 1.66 & 3.04 & 64.29 & 2.59e+03 & 3.26e+05 & 6.54e+06\\ 
\cline{2-13}
& \multicolumn{1}{c|}{} & 10 & 20 & 50 & 100 & 200 & 300 & 500 & 1000 & 2000 & 5000 & 10000\\ 
& & \multicolumn{10}{c}{$M_{\rm{Med}}$ (GeV)}
\end{tabular}
\end{center}
\end{table}
 
%\begin{table}
\footnotesize
\begin{center}
\caption{DMP xs10 g1p0 2p1fb exp 95\% CL upper limits}
\begin{tabular}{lcccccccccccc}
\label{limits_DMP_xs10_g1p0_2p1fb_exp}
\multirow{7}{*}{\rotatebox{90}{$m_{\rm{DM}}$ (GeV)}}
& \multicolumn{1}{c|}{1000} &  &  &  &  &  &  &  & 4.52e+05 & 1.65e+04 & 5.13e+06 & 1.71e+08\\ 
& \multicolumn{1}{c|}{500} &  &  &  &  &  &  & 6.53e+03 & 299.10 & 2.69e+03 & 1.18e+06 & -1.00\\ 
& \multicolumn{1}{c|}{150} &  &  &  &  & 24.33 & 4.19 & 2.32 & 0.31 & 1.55e+03 & 2.73e+05 & 4.62e+06\\ 
& \multicolumn{1}{c|}{100} &  &  &  & 19.60 & 4.91 & 0.92 & 1.67 & 46.34 &  & 2.14e+05 & 3.80e+06\\ 
& \multicolumn{1}{c|}{50} &  &  & 11.85 & 3.38 & 0.38 & 0.54 & 2.21 & 39.22 & 1.52e+03 & 2.07e+05 & 7.11e+03\\ 
& \multicolumn{1}{c|}{10} & 3.12 & 2.29 & 0.34 & 0.30 & 0.48 & 0.79 & 1.87 & 49.00 & 3.16 & 2.39e+05 & 4.87e+06\\ 
& \multicolumn{1}{c|}{1} & 0.11 & 0.09 & 0.31 &  & 0.45 &  & 1.53 & 39.06 & 1.76e+03 & 1.88e+05 & 3.29e+06\\ 
\cline{2-13}
& \multicolumn{1}{c|}{} & 10 & 20 & 50 & 100 & 200 & 300 & 500 & 1000 & 2000 & 5000 & 10000\\ 
& & \multicolumn{10}{c}{$M_{\rm{Med}}$ (GeV)}
\end{tabular}
\end{center}
\end{table}

 

\clearpage
For comparison again we show the expected sensitivity for $g_{\rm SM}=1.0$ for the axial-vector sample in Tab.~\ref{limits_DMA_xs10_g1p0_2p1fb_exp}.

%\begin{table}
\label{limits_DMA_xs10_g1p0_2p1fb_exp}
\begin{center}
\caption{DMA xs10 g1p0 2p1fb exp 95\% CL upper limits}
\begin{tabular}{lcccccccccccc}
\multirow{6}{*}{\rotatebox{90}{$m_{\rm{DM}}$ (GeV)}}
& \multicolumn{1}{c|}{500} &  &  &  &  &  &  & 10.08 &  &  & 757.99 & \\ 
& \multicolumn{1}{c|}{150} &  &  &  &  &  & 0.07 & 0.07 & 0.30 & 3.45 & 224.75 & \\ 
& \multicolumn{1}{c|}{100} &  &  &  & 0.06 &  & 0.02 & 0.04 &  & 3.36 & 224.20 & \\ 
& \multicolumn{1}{c|}{50} &  &  & 0.02 & 0.01 & 0.00 & 0.02 & 0.04 & 0.28 &  & 172.37 & 3.82e+03\\ 
& \multicolumn{1}{c|}{10} & 0.00 & 0.00 & 0.00 & 0.00 & 0.00 & 0.01 & 0.04 & 0.30 &  & 194.11 & 3.87e+03\\ 
& \multicolumn{1}{c|}{1} & -1.00 & -1.00 & 0.00 & 0.00 & 0.00 & 0.01 & 0.03 & 0.24 & 2.95 & 172.43 & 3.67e+03\\ 
\cline{2-13}
& \multicolumn{1}{c|}{} & 10 & 20 & 50 & 100 & 200 & 300 & 500 & 1000 & 2000 & 5000 & 10000\\ 
& & \multicolumn{10}{c}{$M_{\rm{Med}}$ (GeV)}
\end{tabular}
\end{center}
\end{table}
 
\clearpage




%\begin{figure}
%\begin{center}
%\includegraphics[width=0.75\textwidth]{figures/DMplots/DMV_finalCanvasExpLimit.pdf} \\
%\caption{Expected 95\% CL upper limit on the cross section, and exclusion
%contour for the vector light flavour model with unity couplings at 2~\ifb.}
%\label{fig:dm_A_g1_2fb_2dlimits}
%\end{center}
%\end{figure}





%\begin{figure}
%\begin{center}
%\includegraphics[width=0.75\textwidth]{figures/DMplots/DMP_finalCanvasExpLimit.pdf} \\
%\caption{Expected 95\% CL upper limit on the cross section, and exclusion
%contour for the pseudoscalar light flavour model with unity couplings at 2~\ifb.}
%\label{fig:dm_P_g1_2fb_2dlimits}
%\end{center}
%\end{figure}




\clearpage
\subsection{Heavy flavour models} \label{sec:dm_heavyjet}

Owing to the principal of Minimal Flavor Violation (MFV), top and bottom quarks
can play important roles in the phenomenology of dark matter. Scalar and
pseudoscalar models predict not only the `monojet' processes described in
Sec.~\ref{sec:dm_lightjet} but also the production of dark matter in association
with top (or bottom) pairs. This results in signatures with relatively large jet
multiplicities, in particular for \DMtt production. The \alphat analysis is well 
suited to searching for such signatures. An example Feynman diagram for the pair
production of dark matter particles in association with pairs of heavy quarks is
shown in Fig.~\ref{fig:feynman_hf}.


\begin{figure}[h!] \centering
\subfigure{\includegraphics[width=0.35\textwidth]{figures/DMplots/feynman_hf.pdf}}
\caption{Feynman diagram of the pair production of Dark Matter particles in
association with $t\bar{t}$ or $b\bar{b}$. \cite{Abercrombie:2015wmb}}
\label{fig:feynman_hf} \end{figure}


The cross sections, signal yields and efficiencies for scalar and pseudoscalar
\DMtt models expected with 2~\ifb of data are shown in 
Tables~\ref{summaryTableAN_DMttP_xs10_2p1fb_exp}~and~\ref{summaryTableAN_DMttS_xs10_2p1fb_exp} for \DMtt. 

The selection efficiencies for these models are around $\sim 10$\%, and the improvement
provided by the new asymmetric and monojet categories is again evident.

\clearpage 
%\input{tables/DM/summaryTableAN_DMttP_xs10_2p1fb_exp.tex}
%\input{tables/DM/summaryTableAN_DMttS_xs10_2p1fb_exp.tex} 
\clearpage


The yields and efficiencies for \DMbb are shown in Tables~\ref{summaryTableAN_DMbbP_xs10_2p1fb_exp}~and~\ref{summaryTableAN_DMbbS_xs10_2p1fb_exp}, respectively. 

%\begin{table}
\small
\centering
\begin{tabular}{lllllll}
\label{summaryTableAN_DMbbP_xs10_2p1fb_exp}
\hline
$m_\phi$ & $m_\chi$ & $\sigma$ [pb] & Yield (sym) & Yield (asy) & Yield (mon) & Efficiency [\%] \\ \hline
1000      &   1000      &   2.31e-09  &   0.00      &   0.00      &   0.00      &   12.97     \\ 
10        &   1000      &   1.52e-09  &   0.00      &   0.00      &   0.00      &   13.18     \\ 
100       &   10        &   5.36e-01  &   0.08      &   0.17      &   0.61      &   0.08      \\ 
10        &   10        &   9.36e-02  &   0.00      &   0.01      &   0.02      &   0.01      \\ 
15        &   10        &   1.30e-01  &   0.00      &   0.01      &   0.02      &   0.01      \\ 
50        &   10        &   1.94e+00  &   0.04      &   0.11      &   0.50      &   0.02      \\ 
200       &   150       &   1.52e-04  &   0.00      &   0.00      &   0.00      &   1.46      \\ 
295       &   150       &   1.67e-03  &   0.01      &   0.01      &   0.02      &   1.03      \\ 
500       &   150       &   1.26e-03  &   0.01      &   0.01      &   0.04      &   2.11      \\ 
1000      &   1         &   5.48e-05  &   0.00      &   0.00      &   0.00      &   4.22      \\ 
100       &   1         &   5.37e-01  &   0.04      &   0.21      &   0.79      &   0.09      \\ 
10        &   1         &   6.25e+00  &   0.02      &   0.01      &   0.22      &   0.00      \\ 
200       &   1         &   8.59e-02  &   0.10      &   0.15      &   0.40      &   0.36      \\ 
20        &   1         &   4.82e+00  &   0.00      &   0.00      &   0.27      &   0.00      \\ 
300       &   1         &   2.25e-02  &   0.04      &   0.08      &   0.27      &   0.82      \\ 
500       &   1         &   1.51e-03  &   0.01      &   0.01      &   0.04      &   1.89      \\ 
50        &   1         &   1.95e+00  &   0.07      &   0.17      &   0.31      &   0.01      \\ 
10        &   500       &   2.12e-07  &   0.00      &   0.00      &   0.00      &   7.26      \\ 
500       &   500       &   3.10e-07  &   0.00      &   0.00      &   0.00      &   7.30      \\ 
995       &   500       &   6.70e-06  &   0.00      &   0.00      &   0.00      &   6.17      \\ 
10        &   50        &   3.19e-03  &   0.00      &   0.00      &   0.01      &   0.20      \\ 
200       &   50        &   8.56e-02  &   0.09      &   0.14      &   0.39      &   0.34      \\ 
50        &   50        &   4.19e-03  &   0.00      &   0.00      &   0.01      &   0.21      \\ 
95        &   50        &   2.21e-02  &   0.01      &   0.02      &   0.03      &   0.12      \\ 
\hline
\end{tabular}
\caption{Cross section, yields at 2~\ifb (split according to symmetric, asymmetric, and monojet categories), and total selection efficiency for the pseudo-scalar \DMbb samples.}
\label{summaryTableAN_DMbbP_xs10_2p1fb_exp}
\end{table}

%\begin{table}
\small
\label{summaryTableAN_DMbbS_xs10_2p1fb_exp}
\centering
\begin{tabular}{lllllll}
\hline
$m_\phi$ & $m_\chi$ & $\sigma$ [pb] & Yield (sym) & Yield (asy) & Yield (mon) & Efficiency [\%] \\ \hline
1000      &   1000      &   4.09e-10  &   0.00      &   0.00      &   0.00      &   13.48     \\ 
10        &   1000      &   2.88e-10  &   0.00      &   0.00      &   0.00      &   13.87     \\ 
100       &   10        &   1.24e+00  &   0.29      &   0.32      &   1.41      &   0.08      \\ 
10        &   10        &   2.68e-01  &   0.01      &   0.03      &   0.07      &   0.02      \\ 
15        &   10        &   3.52e-01  &   0.02      &   0.04      &   0.10      &   0.02      \\ 
50        &   10        &   7.66e+00  &   0.29      &   0.81      &   2.34      &   0.02      \\ 
200       &   150       &   5.91e-05  &   0.00      &   0.00      &   0.00      &   1.78      \\ 
295       &   150       &   2.58e-04  &   0.00      &   0.00      &   0.00      &   1.43      \\ 
500       &   150       &   1.50e-03  &   0.01      &   0.01      &   0.04      &   2.04      \\ 
1000      &   1         &   5.41e-05  &   0.00      &   0.00      &   0.00      &   4.44      \\ 
100       &   1         &   1.25e+00  &   0.37      &   0.49      &   1.75      &   0.10      \\ 
200       &   1         &   1.31e-01  &   0.08      &   0.23      &   0.63      &   0.34      \\ 
20        &   1         &   4.07e+01  &   0.00      &   1.41      &   0.56      &   0.00      \\ 
300       &   1         &   2.93e-02  &   0.06      &   0.11      &   0.32      &   0.79      \\ 
500       &   1         &   2.21e-03  &   0.01      &   0.02      &   0.06      &   2.00      \\ 
50        &   1         &   7.66e+00  &   0.87      &   0.44      &   1.10      &   0.02      \\ 
10        &   500       &   5.36e-08  &   0.00      &   0.00      &   0.00      &   8.61      \\ 
500       &   500       &   7.40e-08  &   0.00      &   0.00      &   0.00      &   8.27      \\ 
995       &   500       &   6.51e-07  &   0.00      &   0.00      &   0.00      &   6.72      \\ 
10        &   50        &   2.71e-03  &   0.00      &   0.00      &   0.01      &   0.32      \\ 
50        &   50        &   3.37e-03  &   0.00      &   0.01      &   0.01      &   0.29      \\ 
95        &   50        &   1.04e-02  &   0.01      &   0.01      &   0.03      &   0.19      \\ 
\hline
\end{tabular}
\caption{Cross section, yields at 2~\ifb (split according to symmetric, asymmetric, and monojet categories), and total selection efficiency for the pseudo-scalar \DMtt samples.}
\label{summaryTableAN_DMbbS_xs10_2p1fb_exp}
\end{table}
 
\clearpage


\subsubsection{Expected and observed sensitivities for DM+$t\bar{t}$}

The expected 95\% CL signal strength limits for simplified \DMtt models with scalar and
pseudo-scalar couplings are calculated for 2~\ifb. An uncertainty of 20\% is assumed for all 
heavy quark samples.



\clearpage
Expected limits obtained for DM+$t\bar{t}$ are given in Tables~\ref{limits_DMttP_xs10_2p1fb_exp}-\ref{limits_DMttS_xs10_2p1fb_exp}.
%\begin{table}
\begin{center}
\caption{DMttP 2.1\ifb exp 95\% CL upper limits}
\begin{tabular}{lcccccccc}
\label{limits_DMttP_xs10_2p1fb_exp}
\multirow{5}{*}{\rotatebox{90}{$m_{\rm{DM}}$ (GeV)}}
& \multicolumn{1}{c|}{500} &  &  &  &  &  &  & 2.83e+04\\ 
& \multicolumn{1}{c|}{150} &  &  &  &  & 509.12 &  & 37.06\\ 
& \multicolumn{1}{c|}{50} &  &  & 116.37 &  & 3.90 & 7.22 & \\ 
& \multicolumn{1}{c|}{10} & 42.02 &  & 2.27 & 2.80 &  &  & \\ 
& \multicolumn{1}{c|}{1} & 1.83 & 2.21 & 2.37 & 2.69 & 4.15 & 5.98 & 35.21\\ 
\cline{2-9}
& \multicolumn{1}{c|}{} & 10 & 20 & 50 & 100 & 200 & 300 & 500\\ 
& & \multicolumn{6}{c}{$M_{\rm{Med}}$ (GeV)}
\end{tabular}
\end{center}
\end{table}


%\begin{table}
\begin{center}
\caption{DMttS xs10 2p1fb exp 95\% CL upper limits}
\begin{tabular}{lccccccccc}
\label{limits_DMttS_xs10_2p1fb_exp}
\multirow{5}{*}{\rotatebox{90}{$m_{\rm{DM}}$ (GeV)}}
& \multicolumn{1}{c|}{500} &  &  &  &  &  &  & 9.38e+04 & \\ 
& \multicolumn{1}{c|}{150} &  &  &  &  & 1.24e+03 &  & 42.92 & \\ 
& \multicolumn{1}{c|}{50} &  &  & 207.56 &  & 4.29 & 8.25 &  & \\ 
& \multicolumn{1}{c|}{10} & 24.40 &  & 0.89 & 1.97 &  &  &  & \\ 
& \multicolumn{1}{c|}{1} & 0.38 & 0.49 & 0.77 & 2.06 & 4.97 & 8.77 & 34.88 & 304.60\\ 
\cline{2-10}
& \multicolumn{1}{c|}{} & 10 & 20 & 50 & 100 & 200 & 300 & 500 & 1000\\ 
& & \multicolumn{7}{c}{$M_{\rm{Med}}$ (GeV)}
\end{tabular}
\end{center}
\end{table}




Appendix~\ref{sec:dm_checklist} contains additional validation of the \DMj DM samples like signal and background yields for the most sensitive bins and sensitivities.

\subsubsection{Expected and observed sensitivities for DM+$b(\bar{b})$}

The expected 95\% CL signal strength limits for simplified DM+$t(\bar{t})$ models with scalar and
pseudo-scalar couplings are calculated for 2~\ifb. An uncertainty of 20\% is assumed for all 
heavy quark samples.


\clearpage
Expected limits obtained for DM+$t\bar{t}$ are given in Tables~\ref{limits_DMbbP_xs10_2p1fb_exp}-\ref{limits_DMbbS_xs10_2p1fb_exp}. 

%\begin{table}
\begin{center}
\tiny
\caption{DMbbP 2.1\ifb exp 95\% CL upper limits}
\begin{tabular}{lccccccccccccc}
\label{limits_DMbbP_xs10_2p1fb_exp}
\multirow{6}{*}{\rotatebox{90}{$m_{\rm{DM}}$ (GeV)}}
& \multicolumn{1}{c|}{1000} & 2.03e+08 &  &  &  &  &  &  &  &  &  &  & 1.44e+08\\ 
& \multicolumn{1}{c|}{500} & 3.76e+06 &  &  &  &  &  &  &  &  & 2.79e+06 & 1.67e+05 & \\ 
& \multicolumn{1}{c|}{150} &  &  &  &  &  &  & 3.14e+04 & 5.07e+03 &  & 3.16e+03 &  & \\ 
& \multicolumn{1}{c|}{50} & 6.87e+03 &  &  & 5.53e+03 & 1.50e+03 &  & 162.84 &  &  &  &  & \\ 
& \multicolumn{1}{c|}{10} & 1.60e+03 & 827.41 &  & 156.65 &  & 105.24 &  &  &  &  &  & \\ 
& \multicolumn{1}{c|}{1} & 177.94 &  & 78.47 & 70.34 &  & 93.38 & 180.58 &  & 415.85 & 2.64e+03 &  & 3.36e+04\\ 
\cline{2-14}
& \multicolumn{1}{c|}{} & 10 & 15 & 20 & 50 & 95 & 100 & 200 & 295 & 300 & 500 & 995 & 1000\\ 
& & \multicolumn{11}{c}{$M_{\rm{Med}}$ (GeV)}
\end{tabular}
\end{center}
\end{table}

%\begin{table}
\begin{center}
\tiny
\caption{DMbbS xs10 2p1fb exp 95\% CL upper limits}
\begin{tabular}{lccccccccccccc}
\label{limits_DMbbS_xs10_2p1fb_exp}
\multirow{6}{*}{\rotatebox{90}{$m_{\rm{DM}}$ (GeV)}}
& \multicolumn{1}{c|}{1000} & 1.03e+09 &  &  &  &  &  &  &  &  &  &  & 7.52e+08\\ 
& \multicolumn{1}{c|}{500} & 1.27e+07 &  &  &  &  &  &  &  &  & 1.08e+07 & 1.48e+06 & \\ 
& \multicolumn{1}{c|}{150} &  &  &  &  &  &  & 6.97e+04 & 2.12e+04 &  & 2.72e+03 &  & \\ 
& \multicolumn{1}{c|}{50} & 7.24e+03 &  &  & 5.96e+03 & 2.09e+03 &  &  &  &  &  &  & \\ 
& \multicolumn{1}{c|}{10} & 828.51 & 210.29 &  & 24.07 &  & 43.11 &  &  &  &  &  & \\ 
& \multicolumn{1}{c|}{1} & -1.00 &  & 21.19 & 11.30 &  & 38.25 & 161.04 &  & 290.34 & 1.85e+03 &  & 3.10e+04\\ 
\cline{2-14}
& \multicolumn{1}{c|}{} & 10 & 15 & 20 & 50 & 95 & 100 & 200 & 295 & 300 & 500 & 995 & 1000\\ 
& & \multicolumn{11}{c}{$M_{\rm{Med}}$ (GeV)}
\end{tabular}
\end{center}
\end{table}




Appendix~\ref{sec:dm_checklist} contains additional validation for the \DMtt and \DMtt DM samples like signal and background yields for the most sensitive bins and sensitivities.

