%%____________________________________________________________________________||
\section{Introduction}
\label{sec:intro}

In this note we summarise the search for WIMPs using the optimised RA1 analysis. The full analsyis documentation consists of two parts. Analysis approach, background estimation and systematic uncertainties are explained in great detail in the RA1 note~\cite{alphaTnote}. This document focuses on the WIMP interpretation.

While to date it is unkown if or in which way a DM candidate is produced at collider we can parameterise the various phyiscal possibilities. In many models the production of a hadronic particlel can be postules, either from initial- or final-state radiation or from associated production with a boson or quarks.
A large missing transverse momentum (\MET) is introduced to the event by the DM candidate escaping the detector undetected. Therefore one has to study $\etmiss$ plus jet final states emphasizing acceptance and low kinematics to account for a large range of possible kinmeatics. These range from low jets multiplicities and low energies to large masses, jet multiplicities and possibbly $b$-quark jets.

We consider all available topologies - mono-jet, multi-jet and heavy quarks - in one analysis. 

Using two special kinematic variables,  $\alpha_{\textrm T}$ and $\bdphi$ we maximise sensitivity while maintaining high purity for events with real missing transverse energy.
The \alphat variable allows to to efficiently select candidate signal events with genuine \MET while
providing robust rejection against background events from QCD multijet production. Further reduction of QCD multijet and instrumental background is achieved using the $\dphi$ variable.

This is an evolution of past searches for supersymmetry (SUSY) in proton-proton collisions data collected during LHC Run~1. With data at a centre-of-mass energy of 7 TeV collected in 2010 and 2011, the \alphat analysis has excluded a large parameter space of the constrained minimal supersymmetric extension of the standard model (CMSSM) \cite{Khachatryan:2011tk, Chatrchyan:2011zy, Chatrchyan:2012wa} and a parameter space of simplified models \cite{Chatrchyan:2012wa}. With data at a centre-of-mass energy of 8 TeV collected and promptly reconstructed in 2012, the \alphat analysis
further excluded a parameter space of simplified models \cite{Chatrchyan:2013lya}.  The search strategy in this note is an extension of that in Ref. \cite{CMS_AN_2013-366}.

With the start of Run 2 in June 2015 the LHC delivers proton-proton collisions at a higher centre-of-mass energy of 13 TeV. The higher energy collision considerably increases the production cross sections of possible dark matter processes. Members of this analysis team have been pivotal in establishing a set of 'Minimal Simplified Dark Matter' models and usage recommendations in the 'ATLAS-CMS-DM-Forum' for Run 2 of the LHC. This experience guided choice and optimisation in this particularly suited search for DM.
