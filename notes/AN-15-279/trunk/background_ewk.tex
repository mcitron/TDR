%%____________________________________________________________________________||

\section{Background estimation for processes with genuine \met}

The dominant backgrounds with significant \met in the final state
is $Z$+jets production with the $Z$ boson decaying to neutrinos, \znunu. 
Smaller contributions come from $Z\to \ell \ell$ decays with both leptons
mis-reconstructed and therefore not vetoed by the lepton veto. The next largest
background contribution is $W$+jets production where the $W$ decays into a tau lepton which
then itself decays hadronically (\wtaunu). These background processes are dominated
by light jet and low to intermediate jet multiplicities. For heavy quarks jets and larger jet 
multiplicities the single-top-quark, $\ttbar$V or $\ttbar$H, diboson become more important.

These SM processes are collectively referred to as the non-multijet backgrounds.

The background simulations are normalised using the most accurate cross
section calculations currently available, usually with next-to-next-to-leading-order (NNLO) accuracy. 
Effects of pileup, the simulated events are generated with a nominal distribution
of pp interactions per bunch crossing and are then corrected to reproduce the pileup distribution as measured in data. 

\subsection{The ``transfer factor'' method}
\label{sec:ewk-method}

The method used to estimate the aforementioned SM background contributions in the hadronic signal region relies on the use of a
transfer factor (TF) determined from MC samples to derive from the observed yield in a given \scalht, jet (\njet) and $b$-tag (\nb)
multiplicity bin of a control sample, $\nobs^{\rm control}(\njet,\nb,\scalht)$, into a predicted yield for the
corresponding bin of the hadronic signal region, $\npre^{\rm  signal}(\njet,\nb,\scalht)$. 

Each transfer factor is simply a ratio of the yields obtained from MC
simulation for the same bin of the signal region and a given control
sample:

\begin{equation}
  \label{equ:tf-ratio}
  {\rm TF} = \frac{N_{\rm MC}^{\rm signal}(\njet,\nb,\scalht)}{N_{\rm
      MC}^{\rm control}(\njet,\nb,\scalht)} 
\end{equation}

In this way, predictions of background counts from SM processes can be
made based on the various control samples:

\begin{equation}
  \label{equ:pred-method}
  \npre^{\rm signal}(\njet,\nb,\scalht) = \frac{N_{\rm MC}^{\rm
      signal}(\njet,\nb,\scalht)}{N_{\rm MC}^{\rm
      control}(\njet,\nb,\scalht)} \times \nobs^{\rm
    control}(\njet,\nb,\scalht)   
\end{equation}


The selection criteria for the data control samples closely resemble
those for the signal region, differing mainly through the use of a
lepton or photon object {\it tag} (that is ignored in the calculation
of jet-based kinematic variables such as \scalht, \mht, \alphat, \etc)
and minimal additional kinematic requirements (\eg invariant or
transerve mass windows) to obtain W, Z, and \ttbar-enriched event
samples. The same selection criteria are designed to suppress signal
contamination in the control samples so that unbiased data-driven
estimates for the SM backgrounds in the signal region can be
made. 

The transfer factors account for differences in cross sections and
branching ratios, acceptance and reconstruction efficiencies, and/or
kinematic requirements between the signal and control regions. Any
dependence on \njet, \nb, or \HT is largely attributable to
differences in acceptance due to the presence or otherwise of \alphat
or \mht requirements.

Many systematic effects are expected to cancel largely in the transfer
factor. However, a systematic uncertainty is assigned to each transfer
factor to account for theoretical uncertainties and effects such as
the mismodelling of kinematics (\eg acceptances) and instrumental
effects (\eg reconstruction efficiencies).

These methods are identical to e.g. the fit used in similar analyses to use $gjets$ to predict the \znunu background. One sees that Table \ref{tab:tf_mu_zinv_sym}. the corresponding scale factors behave physical with decreasing scale factor for larger \HT and consistent behaviour for all jet multiplicities and \HT bins. 
The choice of a TF for each signal region bin is conservative since we are using very similar kinematics in contrast to fits that extrapolate and the small statistics in many signal region bins lead to rather larger uncertainties. All remaining transfer factors are shown detail in Ref.~\cite{alphaTnote}. 

\begin{table}[h!]
\tiny
\centering
\caption{Transfer factors from the \mj control region to the \zInv~ background for symmetric categories.\label{tab:tf_mu_zinv_sym}}
\scalebox{0.85}{\begin{tabular}{ccccccccc}
	\hline\hline
	& \multicolumn{8}{c}{\scalht (\gev)} \\ 
	 (\njet,  \nb) & 200-250 & 250-300 & 300-350 & 350-400 & 400-500 & 500-600 & 600-800 & 800-$\infty$ \\ [0.8ex] 
\hline
	(2, 0) & $1.00\pm 0.02$ & $0.76\pm 0.01$ & $0.57\pm 0.01$ & $0.39\pm 0.01$ & $0.30\pm 0.01$ & $0.21\pm 0.01$ & $0.14\pm 0.00$ & $0.28\pm 0.01$ \\[0.5ex] 
	(2, 1) & $0.48\pm 0.03$ & $0.51\pm 0.02$ & $0.54\pm 0.03$ & $0.44\pm 0.03$ & $0.33\pm 0.02$ & $0.26\pm 0.02$ & $0.20\pm 0.01$ & $0.43\pm 0.03$ \\[0.5ex] 
	(2, 2) & $0.68\pm 0.13$ & $0.77\pm 0.14$ & $0.68\pm 0.14$ & $0.48\pm 0.12$ & $0.33\pm 0.07$ & $0.37\pm 0.12$ & $0.22\pm 0.06$ & -- \\[0.5ex] 
	(3, 0) & $0.19\pm 0.08$ & $0.45\pm 0.01$ & $0.53\pm 0.01$ & $0.53\pm 0.01$ & $0.42\pm 0.01$ & $0.27\pm 0.01$ & $0.19\pm 0.00$ & $0.26\pm 0.00$ \\[0.5ex] 
	(3, 1) & -- & $0.10\pm 0.01$ & $0.16\pm 0.01$ & $0.17\pm 0.01$ & $0.20\pm 0.01$ & $0.18\pm 0.01$ & $0.15\pm 0.01$ & $0.26\pm 0.01$ \\[0.5ex] 
	(3, 2) & -- & $0.08\pm 0.02$ & $0.10\pm 0.01$ & $0.08\pm 0.01$ & $0.08\pm 0.01$ & $0.07\pm 0.01$ & $0.07\pm 0.01$ & $0.14\pm 0.02$ \\[0.5ex] 
	(3, $\ge3$) & -- & -- & -- & -- & $0.03\pm 0.03$ & -- & -- & -- \\[0.5ex] 
	(4, 0) & -- & -- & $0.51\pm 0.03$ & $0.54\pm 0.02$ & $0.44\pm 0.01$ & $0.33\pm 0.01$ & $0.23\pm 0.00$ & $0.24\pm 0.00$ \\[0.5ex] 
	(4, 1) & -- & -- & $0.13\pm 0.01$ & $0.11\pm 0.01$ & $0.11\pm 0.00$ & $0.11\pm 0.01$ & $0.10\pm 0.00$ & $0.17\pm 0.01$ \\[0.5ex] 
	(4, 2) & -- & -- & $0.07\pm 0.02$ & $0.04\pm 0.01$ & $0.05\pm 0.00$ & $0.04\pm 0.00$ & $0.05\pm 0.00$ & $0.09\pm 0.01$ \\[0.5ex] 
	(4, $\ge3$) & -- & -- & -- & $0.06\pm 0.04$ & $0.07\pm 0.02$ & $0.03\pm 0.02$ & $0.03\pm 0.01$ & $0.12\pm 0.05$ \\[0.5ex] 
	($\ge5$, 0) & -- & -- & -- & $0.31\pm 0.04$ & $0.36\pm 0.01$ & $0.27\pm 0.01$ & $0.19\pm 0.00$ & $0.18\pm 0.00$ \\[0.5ex] 
	($\ge5$, 1) & -- & -- & -- & $0.07\pm 0.02$ & $0.07\pm 0.01$ & $0.05\pm 0.00$ & $0.05\pm 0.00$ & $0.06\pm 0.00$ \\[0.5ex] 
	($\ge5$, 2) & -- & -- & -- & $0.01\pm 0.01$ & $0.03\pm 0.00$ & $0.02\pm 0.00$ & $0.02\pm 0.00$ & $0.03\pm 0.00$ \\[0.5ex] 
	($\ge5$, $\ge3$) & -- & -- & -- & -- & $0.03\pm 0.02$ & $0.02\pm 0.01$ & $0.03\pm 0.01$ & $0.03\pm 0.01$ \\[0.5ex] 
	\hline
	\hline
\end{tabular}}
\end{table}



\subsection{Adding the \mht dimension}

The aforementioned TF define the overall background normalisation in each signal region bin (\njet,\nb,\HT) bin that is
integrated over \mht. The shape of the \mht distribution is taken from simulations and then propagated to the likelihood model. 
A separate \mht template is used per (\njet,\nb,\HT) bin


