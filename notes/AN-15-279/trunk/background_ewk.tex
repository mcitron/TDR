%%____________________________________________________________________________||

\section{Background estimation for processes with genuine \met}

The dominant backgrounds with real \met is $Z$+jets production with the $Z$ boson decaying to neutrinos.
Smaller contributions come from $Z\to \ell \ell$ decays with both leptons
mis-reconstructed. The next largest background contribution is \wtaunu production. All of these background processes are dominated
by light jet and low to intermediate jet multiplicities. For $b$-jets jets and larger jet 
multiplicities contributions from single-top-quark, $\ttbar$V or $\ttbar$H and diboson production become relevant. We will refer
to these backgrounds as non-multijet or electroweak background.

The background simulations are normalised using the best known cross
section calculations currently available, usually with next-to-next-to-leading-order (NNLO) accuracy. 

\subsection{Transfer Factors}
\label{sec:ewk-method}

We use the transfer factor (TF) method to derive corrections to the background normalisations by testing their normalisations
in control regions. These control regions are identical in their energy scale \scalht, jet multiplicity \njet and $b$-jet multiplicity \nb 
as the signal region to be predicted. 

Each transfer factor is simply a ratio of the yields obtained from MC
simulation for the same bin of the signal region and a given control
sample:

\begin{equation}
  \label{equ:tf-ratio}
  {\rm TF} = \frac{N_{\rm MC}^{\rm signal}(\njet,\nb,\scalht)}{N_{\rm
      MC}^{\rm control}(\njet,\nb,\scalht)} 
\end{equation}

In this way, predictions of background counts from SM processes can be
made based on the various control samples:

\begin{equation}
  \label{equ:pred-method}
  \npre^{\rm signal}(\njet,\nb,\scalht) = \frac{N_{\rm MC}^{\rm
      signal}(\njet,\nb,\scalht)}{N_{\rm MC}^{\rm
      control}(\njet,\nb,\scalht)} \times \nobs^{\rm
    control}(\njet,\nb,\scalht)   
\end{equation}


Here \nobs corresponds to the background yield measure in a control region bin in $(\njet,\nb,\scalht)$ used to predict the 
ield \npre in the corresponding hadronic signal region bin $(\njet,\nb,\scalht)$.

%The selection criteria for the data control samples closely resemble
%those for the signal region, differing mainly through the use of a
%lepton or photon object {\it tag} (that is ignored in the calculation
%of jet-based kinematic variables such as \scalht, \mht, \alphat, \etc)
%and minimal additional kinematic requirements (\eg invariant or
%transerve mass windows) to obtain W, Z, and \ttbar-enriched event
%samples. The same selection criteria are designed to suppress signal
%contamination in the control samples so that unbiased data-driven
%estimates for the SM backgrounds in the signal region can be
%made. 


The transfer factors are designed to account for differences in cross sections times branching ratio, acceptance and reconstruction efficiencies and 
kinematic requirements between control and signal regions. As discussed in Sec.~\ref{selection} the control regions closely resemble the kinematics of the signal selections with the exception mandated to enrich the desired processes. While by using transfer factors one can cancels many systematic effects one still has to assign systematic uncertainties due to mis-modelling of kinematics  (\eg acceptances) and instrumental effects (\eg reconstruction efficiencies.)


The transfer factors from all four control regions are tabularized in Sec. 11 of \cite{alphaTnote}.
We give here as example the results for the \gj to \znunu transfer factors because they can be compared to similar methods in related WIMP searches.
These are shown in Tables~\ref{tab:tf_mu_zinv_asym}, \ref{tab:tf_mumu_zinv_sym}. 

\begin{table}[h!]
\tiny
\centering
\caption{Transfer factors from the \gj control region to the \zInv~ background for symmetric categories.\label{tab:tf_gj_zinv_sym}}
\scalebox{0.85}{\begin{tabular}{ccccc}
	\hline\hline
	& \multicolumn{4}{c}{\scalht (\gev)} \\ 
	 (\njet,  \nb) & 400-500 & 500-600 & 600-800 & 800-$\infty$ \\ [0.8ex] 
\hline
	(2, 0) & $0.78\pm 0.04$ & $0.72\pm 0.06$ & $0.54\pm 0.04$ & $0.28\pm 0.01$ \\[0.5ex] 
	(2, 1) & $0.82\pm 0.16$ & $0.71\pm 0.20$ & $0.78\pm 0.18$ & $0.24\pm 0.03$ \\[0.5ex] 
	(2, 2) & $0.64\pm 0.42$ & $1.93\pm 1.63$ & $0.52\pm 0.55$ & -- \\[0.5ex] 
	(3, 0) & $0.72\pm 0.04$ & $0.64\pm 0.04$ & $0.67\pm 0.03$ & $0.26\pm 0.01$ \\[0.5ex] 
	(3, 1) & $0.78\pm 0.10$ & $0.68\pm 0.11$ & $1.02\pm 0.17$ & $0.26\pm 0.03$ \\[0.5ex] 
	(3, 2) & $0.71\pm 0.28$ & $0.89\pm 0.55$ & $0.63\pm 0.28$ & $0.21\pm 0.07$ \\[0.5ex] 
	(3, $\ge3$) & -- & -- & -- & -- \\[0.5ex] 
	(4, 0) & $0.83\pm 0.05$ & $0.73\pm 0.05$ & $0.58\pm 0.03$ & $0.27\pm 0.01$ \\[0.5ex] 
	(4, 1) & $0.72\pm 0.11$ & $0.85\pm 0.15$ & $0.69\pm 0.09$ & $0.29\pm 0.03$ \\[0.5ex] 
	(4, 2) & $0.75\pm 0.25$ & $0.70\pm 0.30$ & $0.31\pm 0.10$ & $0.27\pm 0.06$ \\[0.5ex] 
	(4, $\ge3$) & $577.40\pm 664.02$ & $0.38\pm 0.44$ & $0.26\pm 0.31$ & $0.04\pm 0.05$ \\[0.5ex] 
	($\ge5$, 0) & $0.89\pm 0.13$ & $0.71\pm 0.08$ & $0.57\pm 0.04$ & $0.30\pm 0.01$ \\[0.5ex] 
	($\ge5$, 1) & $0.79\pm 0.22$ & $0.57\pm 0.12$ & $0.48\pm 0.07$ & $0.26\pm 0.02$ \\[0.5ex] 
	($\ge5$, 2) & $0.85\pm 0.54$ & $0.40\pm 0.16$ & $0.43\pm 0.15$ & $0.22\pm 0.04$ \\[0.5ex] 
	($\ge5$, $\ge3$) & -- & -- & $1.40\pm 1.47$ & $0.52\pm 0.33$ \\[0.5ex] 
	\hline
	\hline
\end{tabular}}
\end{table}

\begin{table}[h!]
\tiny
\centering
\caption{Transfer factors from the \gj control region to the \zInv~ background for asymmetric categories.\label{tab:tf_gj_zinv_asym}}
\begin{tabular}
{ccccc}
	\hline\hline
	& \multicolumn{4}{c}{\scalht (\gev)} \\ 
	 (\njet,  \nb) & 400-500 & 500-600 & 600-800 & 800-$\infty$ \\ [0.8ex] 
\hline
	(2a, 0) & $0.65^{+ 0.05 }_{- 0.05 }$ & $0.56^{+ 0.08 }_{- 0.08 }$ & $0.54^{+ 0.07 }_{- 0.07 }$ & -- \\[0.5ex] 
	(2a, 1) & $1.04^{+ 0.35 }_{- 0.35 }$ & $0.54^{+ 0.24 }_{- 0.24 }$ & -- & -- \\[0.5ex] 
	(2a, 2) & -- & -- & -- & -- \\[0.5ex] 
	(3a, 0) & $0.71^{+ 0.08 }_{- 0.08 }$ & $0.61^{+ 0.12 }_{- 0.12 }$ & $0.76^{+ 0.17 }_{- 0.17 }$ & -- \\[0.5ex] 
	(3a, 1) & $0.68^{+ 0.22 }_{- 0.22 }$ & $0.26^{+ 0.13 }_{- 0.13 }$ & $0.67^{+ 0.45 }_{- 0.45 }$ & -- \\[0.5ex] 
	(3a, 2) & $0.68^{+ 0.43 }_{- 0.43 }$ & $1.08^{+ 1.22 }_{- 1.22 }$ & -- & -- \\[0.5ex] 
	(3a, $\ge3$) & -- & -- & -- & -- \\[0.5ex] 
	(4a, 0) & $0.75^{+ 0.08 }_{- 0.08 }$ & $0.69^{+ 0.15 }_{- 0.15 }$ & $0.42^{+ 0.13 }_{- 0.13 }$ & -- \\[0.5ex] 
	(4a, 1) & $0.87^{+ 0.21 }_{- 0.21 }$ & $0.36^{+ 0.18 }_{- 0.18 }$ & $0.50^{+ 0.36 }_{- 0.36 }$ & -- \\[0.5ex] 
	(4a, 2) & $0.95^{+ 0.63 }_{- 0.63 }$ & -- & -- & -- \\[0.5ex] 
	(4a, $\ge3$) & -- & -- & -- & -- \\[0.5ex] 
	($\ge5$a, 0) & $0.66^{+ 0.09 }_{- 0.09 }$ & $0.69^{+ 0.18 }_{- 0.18 }$ & $0.83^{+ 0.27 }_{- 0.27 }$ & -- \\[0.5ex] 
	($\ge5$a, 1) & $0.84^{+ 0.34 }_{- 0.34 }$ & $0.57^{+ 0.28 }_{- 0.28 }$ & $1.90^{+ 1.88 }_{- 1.88 }$ & -- \\[0.5ex] 
	($\ge5$a, 2) & $0.72^{+ 0.55 }_{- 0.55 }$ & $276.67^{+ 282.97 }_{- 282.97 }$ & -- & -- \\[0.5ex] 
	($\ge5$a, $\ge3$) & -- & -- & -- & -- \\[0.5ex] 
	\hline
	\hline
\end{tabular}
\end{table}



The uncertainty shown is from MC statistics only, the systematic uncertainty on the TFs is discussed in Sec.~\ref{sec:systematics}.
The scale factor decreases for larger \HT and behaves consistently across all jet multiplicities and \HT bins. 
The choice of a TF for each signal region bin is conservative as the prediction is made from control regions with similar kinematics
to the signal region bin (without extrapolation) which leads to larger statistical uncertainties.

The transfer factors define the overall background normalisation in each signal region bin (\njet,\nb,\HT) bin. These bins are
integrated over \mht. In each bin the shape of the \mht distribution is taken from simulations and then propagated to the likelihood model. 
A separate \mht template is used per (\njet,\nb,\HT) bin.


