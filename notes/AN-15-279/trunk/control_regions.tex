\subsection{Control Regions}

Several control regions are defined to study modeling of important distributions and provide
orthogonal selections to derive expected backgrounds in a data driven fway.


A hadronic control region that is enriched in multijet events and
disjoint with respect to the signal region is obtained by applying both the pre-selection criteria and lepton/photon vetoes, as defined
above, and inverting the (\HT-dependent) \alphat and/or \mhtmet requirements. 
The sample of events populating this control region are used primarily to estimate any residual background contamination from QCD multijet production.

Three more control regions are defined, with leptons or photons in the final state:: \mj, \mmj, and \gj. 

The full pre-selection is applied as part of the definition of each of these control regions. The cuts on event-level jet-based quantities are identical to
those applied in the hadronic search region and the same \njet, \nb, and \scalht binning is used. The lepton(s) or photon is not considered
in the calculation of the event-level variables.

The selection criteria of the various control regions are defined such that the background composition and event kinematics of the control
regions mirror as closely as possible those for the signal region. This is done in order to minimise the reliance on the simulation to model correctly the backgrounds and event kinematics in
the control and signal samples.

Two exceptions are made. First, no \bdphi requirement is imposed as part of the selection criteria defining the control regions. Second,
in the case of the four leptonic control regions, no requirement is made on \alphat. This is made possible by the remaining kinematic
selection criteria, which are sufficiently selective to ensure that the leptonic event samples remain rich in events from the \wj, \ttbar
and \zll processes with negligible contamination from QCD multijet events. Thus, the acceptance of the leptonic control regions can be
significantly increased, which simultaneously improves their predictive power and further reduces the effect of any potential
signal contamination.
The lepton event samples can be used to predict components of the SM background across all \scalht bins, while the \gj sample can only be
used for the region $\HT > 400\gev$ due to the photon trigger requirements. The following samples control regions are used:

\begin{itemize}
  \item{\mj} In order to select events containing W bosons, exactly one tight isolated muon within an acceptance of \PT $>$ 30 \gev and $|\eta| <$ 2.1 is required, and the transverse mass of the W candidate must satisfy $30 < \mt(\mu,\pfmet) < 125\gev$. 

  \item{\mmj} The selection criteria are identical to those for the \mj sample, while requiring two tightly isolated muons. The invariant mass of the two muons must satisfy $m_{Z} - 25 < M_{\mu_1\mu_2} < m_{Z} +25$ and they must have opposite charge.


  \item{\gj} The \gj sample is defined by requiring exactly one photon satisfying tight isolation criteria and within an acceptance of $\pt > 200\gev$ and $|\eta| < 1.45$. One important difference with respect to the leptonic control samples is the application of the \HT-dependent \alphat requirements imposed as part of the signal region definition. This is to ensure that the photon control sample and signal region are subject to identical kinematic requirements and the photon carries sufficient transverse energy so that the mass of the Z boson becomes a negligible effect when using the \gj sample to predict the kinematic distributions of the \znunu background. All cleaning requirements are applied. The \gj sample can only be used to predict background components the region $\HT > 400\gev$ due to trigger requirements.


\end{itemize}



\subsection{Cross section corrections}

The analysis strategy is such that there is little to no dependence to the normalization of simulated samples. Nevertheless if available NLO corrections (K-factors) are applied. These come either from theoretical calculations if available or enriched sidebands for the relevant process. 
The $V$+jets ($V=W,Z$) process is known in LO and therefore only small corrections are expected. $V$+jet enriched light jet samples with two or three jets have been studied. The $\ttbar$ sample is studied in a sample of two or more jets and requiring at least two $b$-tagged jets. Results of all studies are inconclusive a the early stages of data commissioning and correction has been derived. 

The \gjets process is known only in LO and no NLO cross section for 13 TeV has been calculated. Using a dedicated $350< \HT<400$~GeV sideband a K-factor of $1.30 \pm 0.08$ has been derived

Yield and distributions for all control regions can be found in Ref.~\cite{alphaTnote}.



