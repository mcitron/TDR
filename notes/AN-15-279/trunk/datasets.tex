\section{Data sets}
\label{sec:datasets}

\subsection{Data}


In this note, we use 2.1~\ifb proton-proton collision data at $\sqrt{s} =$ 13~TeV collected in 2015. The specific datasets are given in 

\begin{table}[!h]
\topcaption{Data sets}
\footnotesize \begin{center}
\begin{tabular}{l}
\hline\hline
\multicolumn{1}{c}{Data set}\tabularnewline
\hline
\verb!/HTMHT/Run2016B-23Sep2016-v3/MINIAOD!\tabularnewline
\verb!/HTMHT/Run2016C-23Sep2016-v1/MINIAOD!\tabularnewline
\verb!/HTMHT/Run2016D-23Sep2016-v1/MINIAOD!\tabularnewline
\verb!/HTMHT/Run2016E-23Sep2016-v1/MINIAOD!\tabularnewline
\verb!/HTMHT/Run2016F-23Sep2016-v1/MINIAOD!\tabularnewline
\verb!/HTMHT/Run2016G-23Sep2016-v2/MINIAOD!\tabularnewline
\verb!/HTMHT/Run2016H-PromptReco-v2/MINIAOD!\tabularnewline
\verb!/HTMHT/Run2016H-PromptReco-v3/MINIAOD!\tabularnewline
\verb!/JetHT/Run2016B-23Sep2016-v2/MINIAOD!\tabularnewline
\verb!/JetHT/Run2016C-23Sep2016-v2/MINIAOD!\tabularnewline
\verb!/JetHT/Run2016D-23Sep2016-v2/MINIAOD!\tabularnewline
\verb!/JetHT/Run2016E-23Sep2016-v1/MINIAOD!\tabularnewline
\verb!/JetHT/Run2016F-23Sep2016-v1/MINIAOD!\tabularnewline
\verb!/JetHT/Run2016G-23Sep2016-v2/MINIAOD!\tabularnewline
\verb!/JetHT/Run2016H-PromptReco-v2/MINIAOD!\tabularnewline
\verb!/JetHT/Run2016H-PromptReco-v3/MINIAOD!\tabularnewline
\verb!/MET/Run2016B-23Sep2016-v3/MINIAOD!\tabularnewline
\verb!/MET/Run2016C-23Sep2016-v1/MINIAOD!\tabularnewline
\verb!/MET/Run2016D-23Sep2016-v1/MINIAOD!\tabularnewline
\verb!/MET/Run2016E-23Sep2016-v1/MINIAOD!\tabularnewline
\verb!/MET/Run2016F-23Sep2016-v1/MINIAOD!\tabularnewline
\verb!/MET/Run2016G-23Sep2016-v1/MINIAOD!\tabularnewline
\verb!/MET/Run2016H-PromptReco-v2/MINIAOD!\tabularnewline
\verb!/SingleMuon/Run2016B-23Sep2016-v3/MINIAOD!\tabularnewline
\verb!/SingleMuon/Run2016C-23Sep2016-v1/MINIAOD!\tabularnewline
\verb!/SingleMuon/Run2016D-23Sep2016-v1/MINIAOD!\tabularnewline
\verb!/SingleMuon/Run2016E-23Sep2016-v1/MINIAOD!\tabularnewline
\verb!/SingleMuon/Run2016F-23Sep2016-v1/MINIAOD!\tabularnewline
\verb!/SingleMuon/Run2016G-23Sep2016-v1/MINIAOD!\tabularnewline
\verb!/SingleMuon/Run2016H-PromptReco-v2/MINIAOD!\tabularnewline
\verb!/SingleMuon/Run2016H-PromptReco-v3/MINIAOD!\tabularnewline
\verb!/SinglePhoton/Run2016B-23Sep2016-v3/MINIAOD!\tabularnewline
\verb!/SinglePhoton/Run2016C-23Sep2016-v1/MINIAOD!\tabularnewline
\verb!/SinglePhoton/Run2016D-23Sep2016-v1/MINIAOD!\tabularnewline
\verb!/SinglePhoton/Run2016E-23Sep2016-v1/MINIAOD!\tabularnewline
\verb!/SinglePhoton/Run2016F-23Sep2016-v1/MINIAOD!\tabularnewline
\verb!/SinglePhoton/Run2016G-23Sep2016-v1/MINIAOD!\tabularnewline
\verb!/SinglePhoton/Run2016H-PromptReco-v2/MINIAOD!\tabularnewline
\verb!/SinglePhoton/Run2016H-PromptReco-v3/MINIAOD!\tabularnewline
\verb!/DoubleEG/Run2016B-23Sep2016-v3/MINIAOD!\tabularnewline
\verb!/DoubleEG/Run2016C-23Sep2016-v1/MINIAOD!\tabularnewline
\verb!/DoubleEG/Run2016D-23Sep2016-v1/MINIAOD!\tabularnewline
\verb!/DoubleEG/Run2016E-23Sep2016-v1/MINIAOD!\tabularnewline
\verb!/DoubleEG/Run2016F-23Sep2016-v1/MINIAOD!\tabularnewline
\verb!/DoubleEG/Run2016G-23Sep2016-v1/MINIAOD!\tabularnewline
\verb!/DoubleEG/Run2016H-PromptReco-v2/MINIAOD!\tabularnewline
\verb!/DoubleEG/Run2016H-PromptReco-v3/MINIAOD!\tabularnewline
\hline
\end{tabular}\end{center}

\label{tab:datasets_data}
\end{table}

Latest JSON files are applied. Further details are given in Ref.~\cite{alphaTnote}. 



\subsection{Simulation}

\subsection{Background}

A large range of possible background sources has been simulated, these are:

\begin{center}
  \begin{itemize}
  \item[DY$\to \ell \ell$:] In bins of \HT from 100 GeV onwards
  \item[$\gamma+\textrm{jets}$:] In bins of \HT from 100 GeV onewards
  \item[$QCD$:] In bins of \HT from $100-2000$ GeV 
  \item[$\ttj$:] Hadronic, single lepton and dileptonic final states
  \item[$W+\textrm{jets}$:] In bins of \HT from 100 GeV onewards
  \item[$Z+\textrm{jets}$:] In bins of \HT from 100 GeV onewards
  \item[Diboson:] Inclusive $WW$, $WZ$ and $ZZ$ production
  \end{itemize}
\end{center}



In these datasets in additoin to the main interaction, each event contains on average 20 minimum bias interactions which simulate multiple interactions per bunch-crossing (in-time pileup). The expected detector signal from previous or following bunch crossings (out-of-time pileup) with $25$ns bunch spacing is overlapped.


The samples binned according to generator level quantities  
($W+$jets,D$Y+$jets,$QCD$,$\gamm+$jets, $Z\to \nu \nu$+jets) are provided with LO cross sections only. The LO to (N)NLO corrections (\kfactors) are usually determined determined using corresponding inclusive samples applied to each HT binned sample. Further studies can provide corrections to the cross sections, which can prove important to the closure test procedures described later. Residual cross section corrections are measured using data in sidebands designed to enriched specific processes.


In the $8$ TeV LHC results the top quark momentum spectrum was found to differ between
simulation and data. A reweighting is therefore applied to MC events that contain a generated top. The value of this correction is provided from the 8 TeV results, as described in~\cite{twiki-TopPtReweighting}.

\subsubsection{Signal}

A list of possible dark matter simplified models are produced following the ATLAS-CMS-Theory DM Forum recommendations. The samples have been centrally produced using the \textsc{POWHEG} generator for mediator mass range of $1 GeV \le \mphi\le 10000$ GeV and dark matter masses $\mchi$ between $1 \ge \mchi < 400$ GeV. We simulate four possible interaction, (vector-)axial (A, AV)and (pseudo-)scalar (PS, S).

Additionally we separately simulate light- ($u,~c,~d,~s$) and heavy quarks ($b,~t$) production. For all samples we assume for the couplings  $\gdm$ between mediator and dark maater $\gdm=1.0$. 

For the coupling $\gsm$ between mediator and SM two values are assumed, $\gsm=1.0$ for scalar and pseudo-scalar couplings and $\gsm=0.25$ for (A)V interactions. In all casesd we assume the minimal width, e.g. no decay to other particles is allowed.

Typically kinematics and expeced production cross section are comparable between V(A) couplings produced mostly via $q\bar{q}$-annihilation and (P)S production mainly from gluon-fusion.

A subset of these samples is given in Tables~\ref{datasets_dm_vector}-~\ref{tab:datasets_dm_ttbar_pseudoscalar}. 


\begin{table}[!p]
 \centering
\topcaption{Simulated signal samples: DM Vector}
 \scriptsize
 \scalebox{.7}[1.0]{%latex.default(d, title = NULL, booktabs = FALSE, width = 3, rowname = NULL,     helvetica = FALSE, caption.loc = "bottom", ...)%
\begin{center}
\begin{tabular}{l}
\hline\hline
\multicolumn{1}{c}{Data set}\tabularnewline
\hline
\verb!/DMV_NNPDF30_Vector_Mchi-10_Mchi-1_gSM-0p25_gDM-1p0_13TeV-powheg/RunIISpring15DR74-Asympt25ns_MCRUN2_74_V9-v1/MINIAODSIM! \tabularnewline
\verb!/DMV_NNPDF30_Vector_Mphi-10_Mchi-1_gSM-1p0_gDM-1p0_13TeV-powheg/RunIISpring15DR74-Asympt25ns_MCRUN2_74_V9-v1/MINIAODSIM! \tabularnewline
\verb!/DMV_NNPDF30_Vector_Mchi-10_Mchi-10_gSM-0p25_gDM-1p0_13TeV-powheg/RunIISpring15DR74-Asympt25ns_MCRUN2_74_V9-v1/MINIAODSIM! \tabularnewline
\verb!/DMV_NNPDF30_Vector_Mphi-10_Mchi-10_gSM-1p0_gDM-1p0_13TeV-powheg/RunIISpring15DR74-Asympt25ns_MCRUN2_74_V9-v1/MINIAODSIM! \tabularnewline
\verb!/DMV_NNPDF30_Vector_Mchi-10_Mchi-50_gSM-0p25_gDM-1p0_13TeV-powheg/RunIISpring15DR74-Asympt25ns_MCRUN2_74_V9-v1/MINIAODSIM! \tabularnewline
\verb!/DMV_NNPDF30_Vector_Mchi-10_Mchi-100_gSM-0p25_gDM-1p0_13TeV-powheg/RunIISpring15DR74-Asympt25ns_MCRUN2_74_V9-v1/MINIAODSIM! \tabularnewline
\verb!/DMV_NNPDF30_Vector_Mchi-10_Mchi-150_gSM-0p25_gDM-1p0_13TeV-powheg/RunIISpring15DR74-Asympt25ns_MCRUN2_74_V9-v1/MINIAODSIM! \tabularnewline
\verb!/DMV_NNPDF30_Vector_Mchi-10_Mchi-500_gSM-0p25_gDM-1p0_13TeV-powheg/RunIISpring15DR74-Asympt25ns_MCRUN2_74_V9-v1/MINIAODSIM! \tabularnewline
\verb!/DMV_NNPDF30_Vector_Mchi-20_Mchi-1_gSM-0p25_gDM-1p0_13TeV-powheg/RunIISpring15DR74-Asympt25ns_MCRUN2_74_V9-v1/MINIAODSIM! \tabularnewline
\verb!/DMV_NNPDF30_Vector_Mphi-20_Mchi-1_gSM-1p0_gDM-1p0_13TeV-powheg/RunIISpring15DR74-Asympt25ns_MCRUN2_74_V9-v1/MINIAODSIM! \tabularnewline
\verb!/DMV_NNPDF30_Vector_Mphi-50_Mchi-1_gSM-1p0_gDM-1p0_13TeV-powheg/RunIISpring15DR74-Asympt25ns_MCRUN2_74_V9-v1/MINIAODSIM! \tabularnewline
\verb!/DMV_NNPDF30_Vector_Mphi-50_Mchi-10_gSM-1p0_gDM-1p0_13TeV-powheg/RunIISpring15DR74-Asympt25ns_MCRUN2_74_V9-v1/MINIAODSIM! \tabularnewline
\verb!/DMV_NNPDF30_Vector_Mchi-50_Mchi-50_gSM-0p25_gDM-1p0_13TeV-powheg/RunIISpring15DR74-Asympt25ns_MCRUN2_74_V9-v1/MINIAODSIM! \tabularnewline
\verb!/DMV_NNPDF30_Vector_Mphi-100_Mchi-1_gSM-0p25_gDM-1p0_13TeV-powheg/RunIISpring15DR74-Asympt25ns_MCRUN2_74_V9-v1/MINIAODSIM! \tabularnewline
\verb!/DMV_NNPDF30_Vector_Mphi-100_Mchi-10_gSM-0p25_gDM-1p0_13TeV-powheg/RunIISpring15DR74-Asympt25ns_MCRUN2_74_V9-v1/MINIAODSIM! \tabularnewline
\verb!/DMV_NNPDF30_Vector_Mphi-100_Mchi-10_gSM-1p0_gDM-1p0_13TeV-powheg/RunIISpring15DR74-Asympt25ns_MCRUN2_74_V9-v1/MINIAODSIM! \tabularnewline
\verb!/DMV_NNPDF30_Vector_Mphi-100_Mchi-100_gSM-0p25_gDM-1p0_13TeV-powheg/RunIISpring15DR74-Asympt25ns_MCRUN2_74_V9-v1/MINIAODSIM! \tabularnewline
\verb!/DMV_NNPDF30_Vector_Mphi-100_Mchi-100_gSM-1p0_gDM-1p0_13TeV-powheg/RunIISpring15DR74-Asympt25ns_MCRUN2_74_V9-v1/MINIAODSIM! \tabularnewline
\verb!/DMV_NNPDF30_Vector_Mphi-200_Mchi-1_gSM-0p25_gDM-1p0_13TeV-powheg/RunIISpring15DR74-Asympt25ns_MCRUN2_74_V9-v1/MINIAODSIM! \tabularnewline
\verb!/DMV_NNPDF30_Vector_Mphi-200_Mchi-1_gSM-1p0_gDM-1p0_13TeV-powheg/RunIISpring15DR74-Asympt25ns_MCRUN2_74_V9-v1/MINIAODSIM! \tabularnewline
\verb!/DMV_NNPDF30_Vector_Mphi-200_Mchi-10_gSM-0p25_gDM-1p0_13TeV-powheg/RunIISpring15DR74-Asympt25ns_MCRUN2_74_V9-v1/MINIAODSIM! \tabularnewline
\verb!/DMV_NNPDF30_Vector_Mphi-200_Mchi-10_gSM-1p0_gDM-1p0_13TeV-powheg/RunIISpring15DR74-Asympt25ns_MCRUN2_74_V9-v1/MINIAODSIM! \tabularnewline
\verb!/DMV_NNPDF30_Vector_Mphi-200_Mchi-50_gSM-0p25_gDM-1p0_13TeV-powheg/RunIISpring15DR74-Asympt25ns_MCRUN2_74_V9-v1/MINIAODSIM! \tabularnewline
\verb!/DMV_NNPDF30_Vector_Mphi-200_Mchi-50_gSM-1p0_gDM-1p0_13TeV-powheg/RunIISpring15DR74-Asympt25ns_MCRUN2_74_V9-v1/MINIAODSIM! \tabularnewline
\verb!/DMV_NNPDF30_Vector_Mphi-200_Mchi-100_gSM-1p0_gDM-1p0_13TeV-powheg/RunIISpring15DR74-Asympt25ns_MCRUN2_74_V9-v1/MINIAODSIM! \tabularnewline
\verb!/DMV_NNPDF30_Vector_Mphi-200_Mchi-150_gSM-0p25_gDM-1p0_13TeV-powheg/RunIISpring15DR74-Asympt25ns_MCRUN2_74_V9-v1/MINIAODSIM! \tabularnewline
\verb!/DMV_NNPDF30_Vector_Mphi-200_Mchi-150_gSM-1p0_gDM-1p0_13TeV-powheg/RunIISpring15DR74-Asympt25ns_MCRUN2_74_V9-v1/MINIAODSIM! \tabularnewline
\verb!/DMV_NNPDF30_Vector_Mphi-300_Mchi-1_gSM-0p25_gDM-1p0_13TeV-powheg/RunIISpring15DR74-Asympt25ns_MCRUN2_74_V9-v1/MINIAODSIM! \tabularnewline
\verb!/DMV_NNPDF30_Vector_Mphi-300_Mchi-10_gSM-1p0_gDM-1p0_13TeV-powheg/RunIISpring15DR74-Asympt25ns_MCRUN2_74_V9-v1/MINIAODSIM! \tabularnewline
\verb!/DMV_NNPDF30_Vector_Mphi-300_Mchi-50_gSM-0p25_gDM-1p0_13TeV-powheg/RunIISpring15DR74-Asympt25ns_MCRUN2_74_V9-v1/MINIAODSIM! \tabularnewline
\verb!/DMV_NNPDF30_Vector_Mphi-300_Mchi-50_gSM-1p0_gDM-1p0_13TeV-powheg/RunIISpring15DR74-Asympt25ns_MCRUN2_74_V9-v1/MINIAODSIM! \tabularnewline
\verb!/DMV_NNPDF30_Vector_Mphi-300_Mchi-100_gSM-0p25_gDM-1p0_13TeV-powheg/RunIISpring15DR74-Asympt25ns_MCRUN2_74_V9-v1/MINIAODSIM! \tabularnewline
\verb!/DMV_NNPDF30_Vector_Mphi-300_Mchi-100_gSM-1p0_gDM-1p0_13TeV-powheg/RunIISpring15DR74-Asympt25ns_MCRUN2_74_V9-v1/MINIAODSIM! \tabularnewline
\verb!/DMV_NNPDF30_Vector_Mphi-300_Mchi-150_gSM-0p25_gDM-1p0_13TeV-powheg/RunIISpring15DR74-Asympt25ns_MCRUN2_74_V9-v1/MINIAODSIM! \tabularnewline
\verb!/DMV_NNPDF30_Vector_Mphi-300_Mchi-150_gSM-1p0_gDM-1p0_13TeV-powheg/RunIISpring15DR74-Asympt25ns_MCRUN2_74_V9-v1/MINIAODSIM! \tabularnewline
\verb!/DMV_NNPDF30_Vector_Mphi-500_Mchi-1_gSM-0p25_gDM-1p0_13TeV-powheg/RunIISpring15DR74-Asympt25ns_MCRUN2_74_V9-v1/MINIAODSIM! \tabularnewline
\verb!/DMV_NNPDF30_Vector_Mphi-500_Mchi-1_gSM-1p0_gDM-1p0_13TeV-powheg/RunIISpring15DR74-Asympt25ns_MCRUN2_74_V9-v1/MINIAODSIM! \tabularnewline
\verb!/DMV_NNPDF30_Vector_Mphi-500_Mchi-10_gSM-0p25_gDM-1p0_13TeV-powheg/RunIISpring15DR74-Asympt25ns_MCRUN2_74_V9-v1/MINIAODSIM! \tabularnewline
\verb!/DMV_NNPDF30_Vector_Mphi-500_Mchi-10_gSM-1p0_gDM-1p0_13TeV-powheg/RunIISpring15DR74-Asympt25ns_MCRUN2_74_V9-v1/MINIAODSIM! \tabularnewline
\verb!/DMV_NNPDF30_Vector_Mphi-500_Mchi-50_gSM-1p0_gDM-1p0_13TeV-powheg/RunIISpring15DR74-Asympt25ns_MCRUN2_74_V9-v1/MINIAODSIM! \tabularnewline
\verb!/DMV_NNPDF30_Vector_Mphi-500_Mchi-100_gSM-0p25_gDM-1p0_13TeV-powheg/RunIISpring15DR74-Asympt25ns_MCRUN2_74_V9-v1/MINIAODSIM! \tabularnewline
\verb!/DMV_NNPDF30_Vector_Mphi-500_Mchi-100_gSM-1p0_gDM-1p0_13TeV-powheg/RunIISpring15DR74-Asympt25ns_MCRUN2_74_V9-v1/MINIAODSIM! \tabularnewline
\verb!/DMV_NNPDF30_Vector_Mphi-500_Mchi-150_gSM-1p0_gDM-1p0_13TeV-powheg/RunIISpring15DR74-Asympt25ns_MCRUN2_74_V9-v1/MINIAODSIM! \tabularnewline
\verb!/DMV_NNPDF30_Vector_Mphi-500_Mchi-500_gSM-1p0_gDM-1p0_13TeV-powheg/RunIISpring15DR74-Asympt25ns_MCRUN2_74_V9-v1/MINIAODSIM! \tabularnewline
\verb!/DMV_NNPDF30_Vector_Mphi-1000_Mchi-50_gSM-0p25_gDM-1p0_13TeV-powheg/RunIISpring15DR74-Asympt25ns_MCRUN2_74_V9-v1/MINIAODSIM! \tabularnewline
\verb!/DMV_NNPDF30_Vector_Mphi-1000_Mchi-100_gSM-1p0_gDM-1p0_13TeV-powheg/RunIISpring15DR74-Asympt25ns_MCRUN2_74_V9-v1/MINIAODSIM! \tabularnewline
\verb!/DMV_NNPDF30_Vector_Mphi-1000_Mchi-150_gSM-1p0_gDM-1p0_13TeV-powheg/RunIISpring15DR74-Asympt25ns_MCRUN2_74_V9-v1/MINIAODSIM! \tabularnewline
\verb!/DMV_NNPDF30_Vector_Mphi-1000_Mchi-500_gSM-0p25_gDM-1p0_13TeV-powheg/RunIISpring15DR74-Asympt25ns_MCRUN2_74_V9-v1/MINIAODSIM! \tabularnewline
\verb!/DMV_NNPDF30_Vector_Mphi-1000_Mchi-1000_gSM-0p25_gDM-1p0_13TeV-powheg/RunIISpring15DR74-Asympt25ns_MCRUN2_74_V9-v1/MINIAODSIM! \tabularnewline
\verb!/DMV_NNPDF30_Vector_Mphi-1000_Mchi-1000_gSM-1p0_gDM-1p0_13TeV-powheg/RunIISpring15DR74-Asympt25ns_MCRUN2_74_V9-v1/MINIAODSIM! \tabularnewline
\verb!/DMV_NNPDF30_Vector_Mphi-2000_Mchi-1_gSM-0p25_gDM-1p0_13TeV-powheg/RunIISpring15DR74-Asympt25ns_MCRUN2_74_V9-v1/MINIAODSIM! \tabularnewline
\verb!/DMV_NNPDF30_Vector_Mphi-2000_Mchi-10_gSM-0p25_gDM-1p0_13TeV-powheg/RunIISpring15DR74-Asympt25ns_MCRUN2_74_V9-v1/MINIAODSIM! \tabularnewline
\verb!/DMV_NNPDF30_Vector_Mphi-2000_Mchi-100_gSM-1p0_gDM-1p0_13TeV-powheg/RunIISpring15DR74-Asympt25ns_MCRUN2_74_V9-v1/MINIAODSIM! \tabularnewline
\verb!/DMV_NNPDF30_Vector_Mphi-10000_Mchi-50_gSM-0p25_gDM-1p0_13TeV-powheg/RunIISpring15DR74-Asympt25ns_MCRUN2_74_V9-v1/MINIAODSIM! \tabularnewline
\verb!/DMV_NNPDF30_Vector_Mphi-10000_Mchi-50_gSM-1p0_gDM-1p0_13TeV-powheg/RunIISpring15DR74-Asympt25ns_MCRUN2_74_V9-v1/MINIAODSIM! \tabularnewline
\verb!/DMV_NNPDF30_Vector_Mphi-10000_Mchi-100_gSM-1p0_gDM-1p0_13TeV-powheg/RunIISpring15DR74-Asympt25ns_MCRUN2_74_V9-v1/MINIAODSIM! \tabularnewline
\verb!/DMV_NNPDF30_Vector_Mphi-10000_Mchi-150_gSM-1p0_gDM-1p0_13TeV-powheg/RunIISpring15DR74-Asympt25ns_MCRUN2_74_V9-v1/MINIAODSIM! \tabularnewline
\verb!/DMV_NNPDF30_Vector_Mphi-10000_Mchi-500_gSM-1p0_gDM-1p0_13TeV-powheg/RunIISpring15DR74-Asympt25ns_MCRUN2_74_V9-v1/MINIAODSIM! \tabularnewline
\hline
\end{tabular}\end{center}
}
\label{datasets_dm_vector}
\end{table}

\begin{table}[!p]
 \centering
\topcaption{Simulated signal samples: DM Axial}
 \tiny
 \scalebox{.7}[1.0]{%latex.default(d, title = NULL, booktabs = FALSE, width = 3, rowname = NULL,     helvetica = FALSE, caption.loc = "bottom", ...)%
\begin{center}
\begin{tabular}{l}
\hline\hline
\multicolumn{1}{c}{Data set}\tabularnewline
\hline
\verb!/DMV_NNPDF30_Axial_Mchi-10_Mchi-1_gSM-0p25_gDM-1p0_13TeV-powheg/RunIISpring15DR74-Asympt25ns_MCRUN2_74_V9-v1/MINIAODSIM! \tabularnewline
\verb!/DMV_NNPDF30_Axial_Mphi-10_Mchi-1_gSM-1p0_gDM-1p0_13TeV-powheg/RunIISpring15DR74-Asympt25ns_MCRUN2_74_V9-v1/MINIAODSIM! \tabularnewline
\verb!/DMV_NNPDF30_Axial_Mchi-10_Mchi-10_gSM-0p25_gDM-1p0_13TeV-powheg/RunIISpring15DR74-Asympt25ns_MCRUN2_74_V9-v1/MINIAODSIM! \tabularnewline
\verb!/DMV_NNPDF30_Axial_Mphi-10_Mchi-10_gSM-1p0_gDM-1p0_13TeV-powheg/RunIISpring15DR74-Asympt25ns_MCRUN2_74_V9-v1/MINIAODSIM! \tabularnewline
\verb!/DMV_NNPDF30_Axial_Mchi-10_Mchi-50_gSM-0p25_gDM-1p0_13TeV-powheg/RunIISpring15DR74-Asympt25ns_MCRUN2_74_V9-v1/MINIAODSIM! \tabularnewline
\verb!/DMV_NNPDF30_Axial_Mchi-10_Mchi-100_gSM-0p25_gDM-1p0_13TeV-powheg/RunIISpring15DR74-Asympt25ns_MCRUN2_74_V9-v1/MINIAODSIM! \tabularnewline
\verb!/DMV_NNPDF30_Axial_Mchi-10_Mchi-500_gSM-0p25_gDM-1p0_13TeV-powheg/RunIISpring15DR74-Asympt25ns_MCRUN2_74_V9-v1/MINIAODSIM! \tabularnewline
\verb!/DMV_NNPDF30_Axial_Mphi-20_Mchi-1_gSM-1p0_gDM-1p0_13TeV-powheg/RunIISpring15DR74-Asympt25ns_MCRUN2_74_V9-v1/MINIAODSIM! \tabularnewline
\verb!/DMV_NNPDF30_Axial_Mchi-20_Mchi-10_gSM-0p25_gDM-1p0_13TeV-powheg/RunIISpring15DR74-Asympt25ns_MCRUN2_74_V9-v1/MINIAODSIM! \tabularnewline
\verb!/DMV_NNPDF30_Axial_Mphi-20_Mchi-10_gSM-1p0_gDM-1p0_13TeV-powheg/RunIISpring15DR74-Asympt25ns_MCRUN2_74_V9-v1/MINIAODSIM! \tabularnewline
\verb!/DMV_NNPDF30_Axial_Mchi-50_Mchi-1_gSM-0p25_gDM-1p0_13TeV-powheg/RunIISpring15DR74-Asympt25ns_MCRUN2_74_V9-v1/MINIAODSIM! \tabularnewline
\verb!/DMV_NNPDF30_Axial_Mchi-50_Mchi-10_gSM-0p25_gDM-1p0_13TeV-powheg/RunIISpring15DR74-Asympt25ns_MCRUN2_74_V9-v1/MINIAODSIM! \tabularnewline
\verb!/DMV_NNPDF30_Axial_Mphi-50_Mchi-10_gSM-1p0_gDM-1p0_13TeV-powheg/RunIISpring15DR74-Asympt25ns_MCRUN2_74_V9-v1/MINIAODSIM! \tabularnewline
\verb!/DMV_NNPDF30_Axial_Mphi-50_Mchi-50_gSM-1p0_gDM-1p0_13TeV-powheg/RunIISpring15DR74-Asympt25ns_MCRUN2_74_V9-v1/MINIAODSIM! \tabularnewline
\verb!/DMV_NNPDF30_Axial_Mphi-100_Mchi-1_gSM-0p25_gDM-1p0_13TeV-powheg/RunIISpring15DR74-Asympt25ns_MCRUN2_74_V9-v1/MINIAODSIM! \tabularnewline
\verb!/DMV_NNPDF30_Axial_Mphi-100_Mchi-1_gSM-1p0_gDM-1p0_13TeV-powheg/RunIISpring15DR74-Asympt25ns_MCRUN2_74_V9-v1/MINIAODSIM! \tabularnewline
\verb!/DMV_NNPDF30_Axial_Mphi-100_Mchi-10_gSM-0p25_gDM-1p0_13TeV-powheg/RunIISpring15DR74-Asympt25ns_MCRUN2_74_V9-v1/MINIAODSIM! \tabularnewline
\verb!/DMV_NNPDF30_Axial_Mphi-100_Mchi-100_gSM-0p25_gDM-1p0_13TeV-powheg/RunIISpring15DR74-Asympt25ns_MCRUN2_74_V9-v1/MINIAODSIM! \tabularnewline
\verb!/DMV_NNPDF30_Axial_Mphi-100_Mchi-100_gSM-1p0_gDM-1p0_13TeV-powheg/RunIISpring15DR74-Asympt25ns_MCRUN2_74_V9-v1/MINIAODSIM! \tabularnewline
\verb!/DMV_NNPDF30_Axial_Mphi-200_Mchi-1_gSM-0p25_gDM-1p0_13TeV-powheg/RunIISpring15DR74-Asympt25ns_MCRUN2_74_V9-v1/MINIAODSIM! \tabularnewline
\verb!/DMV_NNPDF30_Axial_Mphi-200_Mchi-1_gSM-1p0_gDM-1p0_13TeV-powheg/RunIISpring15DR74-Asympt25ns_MCRUN2_74_V9-v1/MINIAODSIM! \tabularnewline
\verb!/DMV_NNPDF30_Axial_Mphi-200_Mchi-10_gSM-0p25_gDM-1p0_13TeV-powheg/RunIISpring15DR74-Asympt25ns_MCRUN2_74_V9-v1/MINIAODSIM! \tabularnewline
\verb!/DMV_NNPDF30_Axial_Mphi-200_Mchi-10_gSM-1p0_gDM-1p0_13TeV-powheg/RunIISpring15DR74-Asympt25ns_MCRUN2_74_V9-v1/MINIAODSIM! \tabularnewline
\verb!/DMV_NNPDF30_Axial_Mphi-200_Mchi-50_gSM-1p0_gDM-1p0_13TeV-powheg/RunIISpring15DR74-Asympt25ns_MCRUN2_74_V9-v1/MINIAODSIM! \tabularnewline
\verb!/DMV_NNPDF30_Axial_Mphi-200_Mchi-150_gSM-0p25_gDM-1p0_13TeV-powheg/RunIISpring15DR74-Asympt25ns_MCRUN2_74_V9-v1/MINIAODSIM! \tabularnewline
\verb!/DMV_NNPDF30_Axial_Mphi-300_Mchi-1_gSM-0p25_gDM-1p0_13TeV-powheg/RunIISpring15DR74-Asympt25ns_MCRUN2_74_V9-v1/MINIAODSIM! \tabularnewline
\verb!/DMV_NNPDF30_Axial_Mphi-300_Mchi-1_gSM-1p0_gDM-1p0_13TeV-powheg/RunIISpring15DR74-Asympt25ns_MCRUN2_74_V9-v1/MINIAODSIM! \tabularnewline
\verb!/DMV_NNPDF30_Axial_Mphi-300_Mchi-10_gSM-1p0_gDM-1p0_13TeV-powheg/RunIISpring15DR74-Asympt25ns_MCRUN2_74_V9-v1/MINIAODSIM! \tabularnewline
\verb!/DMV_NNPDF30_Axial_Mphi-300_Mchi-50_gSM-1p0_gDM-1p0_13TeV-powheg/RunIISpring15DR74-Asympt25ns_MCRUN2_74_V9-v1/MINIAODSIM! \tabularnewline
\verb!/DMV_NNPDF30_Axial_Mphi-300_Mchi-100_gSM-1p0_gDM-1p0_13TeV-powheg/RunIISpring15DR74-Asympt25ns_MCRUN2_74_V9-v1/MINIAODSIM! \tabularnewline
\verb!/DMV_NNPDF30_Axial_Mphi-300_Mchi-150_gSM-0p25_gDM-1p0_13TeV-powheg/RunIISpring15DR74-Asympt25ns_MCRUN2_74_V9-v1/MINIAODSIM! \tabularnewline
\verb!/DMV_NNPDF30_Axial_Mphi-300_Mchi-150_gSM-1p0_gDM-1p0_13TeV-powheg/RunIISpring15DR74-Asympt25ns_MCRUN2_74_V9-v1/MINIAODSIM! \tabularnewline
\verb!/DMV_NNPDF30_Axial_Mphi-500_Mchi-1_gSM-0p25_gDM-1p0_13TeV-powheg/RunIISpring15DR74-Asympt25ns_MCRUN2_74_V9-v1/MINIAODSIM! \tabularnewline
\verb!/DMV_NNPDF30_Axial_Mphi-500_Mchi-1_gSM-1p0_gDM-1p0_13TeV-powheg/RunIISpring15DR74-Asympt25ns_MCRUN2_74_V9-v1/MINIAODSIM! \tabularnewline
\verb!/DMV_NNPDF30_Axial_Mphi-500_Mchi-10_gSM-1p0_gDM-1p0_13TeV-powheg/RunIISpring15DR74-Asympt25ns_MCRUN2_74_V9-v1/MINIAODSIM! \tabularnewline
\verb!/DMV_NNPDF30_Axial_Mphi-500_Mchi-50_gSM-0p25_gDM-1p0_13TeV-powheg/RunIISpring15DR74-Asympt25ns_MCRUN2_74_V9-v1/MINIAODSIM! \tabularnewline
\verb!/DMV_NNPDF30_Axial_Mphi-500_Mchi-50_gSM-1p0_gDM-1p0_13TeV-powheg/RunIISpring15DR74-Asympt25ns_MCRUN2_74_V9-v1/MINIAODSIM! \tabularnewline
\verb!/DMV_NNPDF30_Axial_Mphi-500_Mchi-100_gSM-0p25_gDM-1p0_13TeV-powheg/RunIISpring15DR74-Asympt25ns_MCRUN2_74_V9-v1/MINIAODSIM! \tabularnewline
\verb!/DMV_NNPDF30_Axial_Mphi-500_Mchi-100_gSM-1p0_gDM-1p0_13TeV-powheg/RunIISpring15DR74-Asympt25ns_MCRUN2_74_V9-v1/MINIAODSIM! \tabularnewline
\verb!/DMV_NNPDF30_Axial_Mphi-500_Mchi-150_gSM-1p0_gDM-1p0_13TeV-powheg/RunIISpring15DR74-Asympt25ns_MCRUN2_74_V9-v1/MINIAODSIM! \tabularnewline
\verb!/DMV_NNPDF30_Axial_Mphi-500_Mchi-500_gSM-0p25_gDM-1p0_13TeV-powheg/RunIISpring15DR74-Asympt25ns_MCRUN2_74_V9-v1/MINIAODSIM! \tabularnewline
\verb!/DMV_NNPDF30_Axial_Mphi-500_Mchi-500_gSM-1p0_gDM-1p0_13TeV-powheg/RunIISpring15DR74-Asympt25ns_MCRUN2_74_V9-v1/MINIAODSIM! \tabularnewline
\verb!/DMV_NNPDF30_Axial_Mphi-1000_Mchi-1_gSM-1p0_gDM-1p0_13TeV-powheg/RunIISpring15DR74-Asympt25ns_MCRUN2_74_V9-v1/MINIAODSIM! \tabularnewline
\verb!/DMV_NNPDF30_Axial_Mphi-1000_Mchi-10_gSM-1p0_gDM-1p0_13TeV-powheg/RunIISpring15DR74-Asympt25ns_MCRUN2_74_V9-v1/MINIAODSIM! \tabularnewline
\verb!/DMV_NNPDF30_Axial_Mphi-1000_Mchi-50_gSM-0p25_gDM-1p0_13TeV-powheg/RunIISpring15DR74-Asympt25ns_MCRUN2_74_V9-v1/MINIAODSIM! \tabularnewline
\verb!/DMV_NNPDF30_Axial_Mphi-1000_Mchi-50_gSM-1p0_gDM-1p0_13TeV-powheg/RunIISpring15DR74-Asympt25ns_MCRUN2_74_V9-v1/MINIAODSIM! \tabularnewline
\verb!/DMV_NNPDF30_Axial_Mphi-1000_Mchi-150_gSM-0p25_gDM-1p0_13TeV-powheg/RunIISpring15DR74-Asympt25ns_MCRUN2_74_V9-v1/MINIAODSIM! \tabularnewline
\verb!/DMV_NNPDF30_Axial_Mphi-1000_Mchi-500_gSM-0p25_gDM-1p0_13TeV-powheg/RunIISpring15DR74-Asympt25ns_MCRUN2_74_V9-v1/MINIAODSIM! \tabularnewline
\verb!/DMV_NNPDF30_Axial_Mphi-2000_Mchi-1_gSM-0p25_gDM-1p0_13TeV-powheg/RunIISpring15DR74-Asympt25ns_MCRUN2_74_V9-v1/MINIAODSIM! \tabularnewline
\verb!/DMV_NNPDF30_Axial_Mphi-2000_Mchi-1_gSM-1p0_gDM-1p0_13TeV-powheg/RunIISpring15DR74-Asympt25ns_MCRUN2_74_V9-v1/MINIAODSIM! \tabularnewline
\verb!/DMV_NNPDF30_Axial_Mphi-2000_Mchi-50_gSM-0p25_gDM-1p0_13TeV-powheg/RunIISpring15DR74-Asympt25ns_MCRUN2_74_V9-v1/MINIAODSIM! \tabularnewline
\verb!/DMV_NNPDF30_Axial_Mphi-2000_Mchi-100_gSM-1p0_gDM-1p0_13TeV-powheg/RunIISpring15DR74-Asympt25ns_MCRUN2_74_V9-v1/MINIAODSIM! \tabularnewline
\verb!/DMV_NNPDF30_Axial_Mphi-2000_Mchi-150_gSM-1p0_gDM-1p0_13TeV-powheg/RunIISpring15DR74-Asympt25ns_MCRUN2_74_V9-v1/MINIAODSIM! \tabularnewline
\verb!/DMV_NNPDF30_Axial_Mphi-5000_Mchi-1_gSM-1p0_gDM-1p0_13TeV-powheg/RunIISpring15DR74-Asympt25ns_MCRUN2_74_V9-v1/MINIAODSIM! \tabularnewline
\verb!/DMV_NNPDF30_Axial_Mphi-5000_Mchi-10_gSM-1p0_gDM-1p0_13TeV-powheg/RunIISpring15DR74-Asympt25ns_MCRUN2_74_V9-v1/MINIAODSIM! \tabularnewline
\verb!/DMV_NNPDF30_Axial_Mphi-5000_Mchi-50_gSM-1p0_gDM-1p0_13TeV-powheg/RunIISpring15DR74-Asympt25ns_MCRUN2_74_V9-v1/MINIAODSIM! \tabularnewline
\verb!/DMV_NNPDF30_Axial_Mphi-5000_Mchi-100_gSM-1p0_gDM-1p0_13TeV-powheg/RunIISpring15DR74-Asympt25ns_MCRUN2_74_V9-v1/MINIAODSIM! \tabularnewline
\verb!/DMV_NNPDF30_Axial_Mphi-5000_Mchi-150_gSM-1p0_gDM-1p0_13TeV-powheg/RunIISpring15DR74-Asympt25ns_MCRUN2_74_V9-v1/MINIAODSIM! \tabularnewline
\verb!/DMV_NNPDF30_Axial_Mphi-5000_Mchi-500_gSM-1p0_gDM-1p0_13TeV-powheg/RunIISpring15DR74-Asympt25ns_MCRUN2_74_V9-v1/MINIAODSIM! \tabularnewline
\verb!/DMV_NNPDF30_Axial_Mphi-10000_Mchi-1_gSM-0p25_gDM-1p0_13TeV-powheg/RunIISpring15DR74-Asympt25ns_MCRUN2_74_V9-v1/MINIAODSIM! \tabularnewline
\verb!/DMV_NNPDF30_Axial_Mphi-10000_Mchi-1_gSM-1p0_gDM-1p0_13TeV-powheg/RunIISpring15DR74-Asympt25ns_MCRUN2_74_V9-v1/MINIAODSIM! \tabularnewline
\verb!/DMV_NNPDF30_Axial_Mphi-10000_Mchi-10_gSM-1p0_gDM-1p0_13TeV-powheg/RunIISpring15DR74-Asympt25ns_MCRUN2_74_V9-v1/MINIAODSIM! \tabularnewline
\verb!/DMV_NNPDF30_Axial_Mphi-10000_Mchi-100_gSM-0p25_gDM-1p0_13TeV-powheg/RunIISpring15DR74-Asympt25ns_MCRUN2_74_V9-v1/MINIAODSIM! \tabularnewline
\verb!/DMV_NNPDF30_Axial_Mphi-10000_Mchi-150_gSM-0p25_gDM-1p0_13TeV-powheg/RunIISpring15DR74-Asympt25ns_MCRUN2_74_V9-v1/MINIAODSIM! \tabularnewline
\hline
\end{tabular}\end{center}
}
\label{datasets_dm_axial}
\end{table}

\begin{table}[!p]
 \centering
\topcaption{Simulated signal samples: DM Scalar}
 \scriptsize
 \scalebox{.7}[1.0]{%latex.default(d, title = NULL, booktabs = FALSE, width = 3, rowname = NULL,     helvetica = FALSE, caption.loc = "bottom", ...)%
\begin{center}
\begin{tabular}{l}
\hline\hline
\multicolumn{1}{c}{Data set}\tabularnewline
\hline
\verb!/DMS_NNPDF30_Scalar_Mphi-10_Mchi-1_gSM-1p0_gDM-1p0_13TeV-powheg/RunIISpring15DR74-Asympt25ns_MCRUN2_74_V9-v1/MINIAODSIM! \tabularnewline
\verb!/DMS_NNPDF30_Scalar_Mphi-10_Mchi-10_gSM-1p0_gDM-1p0_13TeV-powheg/RunIISpring15DR74-Asympt25ns_MCRUN2_74_V9-v1/MINIAODSIM! \tabularnewline
\verb!/DMS_NNPDF30_Scalar_Mphi-20_Mchi-1_gSM-1p0_gDM-1p0_13TeV-powheg/RunIISpring15DR74-Asympt25ns_MCRUN2_74_V9-v1/MINIAODSIM! \tabularnewline
\verb!/DMS_NNPDF30_Scalar_Mphi-20_Mchi-10_gSM-1p0_gDM-1p0_13TeV-powheg/RunIISpring15DR74-Asympt25ns_MCRUN2_74_V9-v1/MINIAODSIM! \tabularnewline
\verb!/DMS_NNPDF30_Scalar_Mphi-50_Mchi-10_gSM-1p0_gDM-1p0_13TeV-powheg/RunIISpring15DR74-Asympt25ns_MCRUN2_74_V9-v1/MINIAODSIM! \tabularnewline
\verb!/DMS_NNPDF30_Scalar_Mphi-100_Mchi-1_gSM-1p0_gDM-1p0_13TeV-powheg/RunIISpring15DR74-Asympt25ns_MCRUN2_74_V9-v1/MINIAODSIM! \tabularnewline
\verb!/DMS_NNPDF30_Scalar_Mphi-100_Mchi-10_gSM-1p0_gDM-1p0_13TeV-powheg/RunIISpring15DR74-Asympt25ns_MCRUN2_74_V9-v1/MINIAODSIM! \tabularnewline
\verb!/DMS_NNPDF30_Scalar_Mphi-100_Mchi-50_gSM-1p0_gDM-1p0_13TeV-powheg/RunIISpring15DR74-Asympt25ns_MCRUN2_74_V9-v1/MINIAODSIM! \tabularnewline
\verb!/DMS_NNPDF30_Scalar_Mphi-100_Mchi-100_gSM-1p0_gDM-1p0_13TeV-powheg/RunIISpring15DR74-Asympt25ns_MCRUN2_74_V9-v1/MINIAODSIM! \tabularnewline
\verb!/DMS_NNPDF30_Scalar_Mphi-200_Mchi-1_gSM-1p0_gDM-1p0_13TeV-powheg/RunIISpring15DR74-Asympt25ns_MCRUN2_74_V9-v1/MINIAODSIM! \tabularnewline
\verb!/DMS_NNPDF30_Scalar_Mphi-200_Mchi-10_gSM-1p0_gDM-1p0_13TeV-powheg/RunIISpring15DR74-Asympt25ns_MCRUN2_74_V9-v1/MINIAODSIM! \tabularnewline
\verb!/DMS_NNPDF30_Scalar_Mphi-200_Mchi-50_gSM-1p0_gDM-1p0_13TeV-powheg/RunIISpring15DR74-Asympt25ns_MCRUN2_74_V9-v1/MINIAODSIM! \tabularnewline
\verb!/DMS_NNPDF30_Scalar_Mphi-200_Mchi-100_gSM-1p0_gDM-1p0_13TeV-powheg/RunIISpring15DR74-Asympt25ns_MCRUN2_74_V9-v1/MINIAODSIM! \tabularnewline
\verb!/DMS_NNPDF30_Scalar_Mphi-200_Mchi-150_gSM-1p0_gDM-1p0_13TeV-powheg/RunIISpring15DR74-Asympt25ns_MCRUN2_74_V9-v1/MINIAODSIM! \tabularnewline
\verb!/DMS_NNPDF30_Scalar_Mphi-300_Mchi-1_gSM-1p0_gDM-1p0_13TeV-powheg/RunIISpring15DR74-Asympt25ns_MCRUN2_74_V9-v1/MINIAODSIM! \tabularnewline
\verb!/DMS_NNPDF30_Scalar_Mphi-300_Mchi-10_gSM-1p0_gDM-1p0_13TeV-powheg/RunIISpring15DR74-Asympt25ns_MCRUN2_74_V9-v1/MINIAODSIM! \tabularnewline
\verb!/DMS_NNPDF30_Scalar_Mphi-300_Mchi-100_gSM-1p0_gDM-1p0_13TeV-powheg/RunIISpring15DR74-Asympt25ns_MCRUN2_74_V9-v1/MINIAODSIM! \tabularnewline
\verb!/DMS_NNPDF30_Scalar_Mphi-300_Mchi-150_gSM-1p0_gDM-1p0_13TeV-powheg/RunIISpring15DR74-Asympt25ns_MCRUN2_74_V9-v1/MINIAODSIM! \tabularnewline
\verb!/DMS_NNPDF30_Scalar_Mphi-500_Mchi-1_gSM-1p0_gDM-1p0_13TeV-powheg/RunIISpring15DR74-Asympt25ns_MCRUN2_74_V9-v1/MINIAODSIM! \tabularnewline
\verb!/DMS_NNPDF30_Scalar_Mphi-500_Mchi-10_gSM-1p0_gDM-1p0_13TeV-powheg/RunIISpring15DR74-Asympt25ns_MCRUN2_74_V9-v1/MINIAODSIM! \tabularnewline
\verb!/DMS_NNPDF30_Scalar_Mphi-500_Mchi-50_gSM-1p0_gDM-1p0_13TeV-powheg/RunIISpring15DR74-Asympt25ns_MCRUN2_74_V9-v1/MINIAODSIM! \tabularnewline
\verb!/DMS_NNPDF30_Scalar_Mphi-500_Mchi-150_gSM-1p0_gDM-1p0_13TeV-powheg/RunIISpring15DR74-Asympt25ns_MCRUN2_74_V9-v1/MINIAODSIM! \tabularnewline
\verb!/DMS_NNPDF30_Scalar_Mphi-500_Mchi-500_gSM-1p0_gDM-1p0_13TeV-powheg/RunIISpring15DR74-Asympt25ns_MCRUN2_74_V9-v1/MINIAODSIM! \tabularnewline
\verb!/DMS_NNPDF30_Scalar_Mphi-1000_Mchi-1_gSM-1p0_gDM-1p0_13TeV-powheg/RunIISpring15DR74-Asympt25ns_MCRUN2_74_V9-v1/MINIAODSIM! \tabularnewline
\verb!/DMS_NNPDF30_Scalar_Mphi-1000_Mchi-10_gSM-1p0_gDM-1p0_13TeV-powheg/RunIISpring15DR74-Asympt25ns_MCRUN2_74_V9-v1/MINIAODSIM! \tabularnewline
\verb!/DMS_NNPDF30_Scalar_Mphi-1000_Mchi-100_gSM-1p0_gDM-1p0_13TeV-powheg/RunIISpring15DR74-Asympt25ns_MCRUN2_74_V9-v1/MINIAODSIM! \tabularnewline
\verb!/DMS_NNPDF30_Scalar_Mphi-1000_Mchi-500_gSM-1p0_gDM-1p0_13TeV-powheg/RunIISpring15DR74-Asympt25ns_MCRUN2_74_V9-v1/MINIAODSIM! \tabularnewline
\verb!/DMS_NNPDF30_Scalar_Mphi-2000_Mchi-1_gSM-1p0_gDM-1p0_13TeV-powheg/RunIISpring15DR74-Asympt25ns_MCRUN2_74_V9-v1/MINIAODSIM! \tabularnewline
\verb!/DMS_NNPDF30_Scalar_Mphi-2000_Mchi-100_gSM-1p0_gDM-1p0_13TeV-powheg/RunIISpring15DR74-Asympt25ns_MCRUN2_74_V9-v1/MINIAODSIM! \tabularnewline
\verb!/DMS_NNPDF30_Scalar_Mphi-2000_Mchi-150_gSM-1p0_gDM-1p0_13TeV-powheg/RunIISpring15DR74-Asympt25ns_MCRUN2_74_V9-v1/MINIAODSIM! \tabularnewline
\verb!/DMS_NNPDF30_Scalar_Mphi-2000_Mchi-1000_gSM-1p0_gDM-1p0_13TeV-powheg/RunIISpring15DR74-Asympt25ns_MCRUN2_74_V9-v1/MINIAODSIM! \tabularnewline
\verb!/DMS_NNPDF30_Scalar_Mphi-5000_Mchi-10_gSM-1p0_gDM-1p0_13TeV-powheg/RunIISpring15DR74-Asympt25ns_MCRUN2_74_V9-v1/MINIAODSIM! \tabularnewline
\verb!/DMS_NNPDF30_Scalar_Mphi-5000_Mchi-50_gSM-1p0_gDM-1p0_13TeV-powheg/RunIISpring15DR74-Asympt25ns_MCRUN2_74_V9-v1/MINIAODSIM! \tabularnewline
\verb!/DMS_NNPDF30_Scalar_Mphi-5000_Mchi-100_gSM-1p0_gDM-1p0_13TeV-powheg/RunIISpring15DR74-Asympt25ns_MCRUN2_74_V9-v1/MINIAODSIM! \tabularnewline
\verb!/DMS_NNPDF30_Scalar_Mphi-5000_Mchi-150_gSM-1p0_gDM-1p0_13TeV-powheg/RunIISpring15DR74-Asympt25ns_MCRUN2_74_V9-v1/MINIAODSIM! \tabularnewline
\verb!/DMS_NNPDF30_Scalar_Mphi-5000_Mchi-500_gSM-1p0_gDM-1p0_13TeV-powheg/RunIISpring15DR74-Asympt25ns_MCRUN2_74_V9-v1/MINIAODSIM! \tabularnewline
\verb!/DMS_NNPDF30_Scalar_Mphi-5000_Mchi-1000_gSM-1p0_gDM-1p0_13TeV-powheg/RunIISpring15DR74-Asympt25ns_MCRUN2_74_V9-v1/MINIAODSIM! \tabularnewline
\verb!/DMS_NNPDF30_Scalar_Mphi-10000_Mchi-1_gSM-1p0_gDM-1p0_13TeV-powheg/RunIISpring15DR74-Asympt25ns_MCRUN2_74_V9-v1/MINIAODSIM! \tabularnewline
\verb!/DMS_NNPDF30_Scalar_Mphi-10000_Mchi-10_gSM-1p0_gDM-1p0_13TeV-powheg/RunIISpring15DR74-Asympt25ns_MCRUN2_74_V9-v1/MINIAODSIM! \tabularnewline
\verb!/DMS_NNPDF30_Scalar_Mphi-10000_Mchi-500_gSM-1p0_gDM-1p0_13TeV-powheg/RunIISpring15DR74-Asympt25ns_MCRUN2_74_V9-v1/MINIAODSIM! \tabularnewline
\verb!/DMS_NNPDF30_Scalar_Mphi-10000_Mchi-1000_gSM-1p0_gDM-1p0_13TeV-powheg/RunIISpring15DR74-Asympt25ns_MCRUN2_74_V9-v1/MINIAODSIM! \tabularnewline
\hline
\end{tabular}\end{center}
}
\label{datasets_dm_scalarw}
\end{table}

\begin{table}[!p]
 \centering
\topcaption{Simulated signal samples: DM Pseudoscalar}
 \scriptsize
 \scalebox{.7}[1.0]{%latex.default(d, title = NULL, booktabs = FALSE, width = 3, rowname = NULL,     helvetica = FALSE, caption.loc = "bottom", ...)%
\begin{center}
\begin{tabular}{l}
\hline\hline
\multicolumn{1}{c}{Data set}\tabularnewline
\hline
\verb!/DMS_NNPDF30_Pseudoscalar_Mphi-10_Mchi-1_gSM-1p0_gDM-1p0_13TeV-powheg/RunIISpring15DR74-Asympt25ns_MCRUN2_74_V9-v1/MINIAODSIM! \tabularnewline
\verb!/DMS_NNPDF30_Pseudoscalar_Mphi-10_Mchi-10_gSM-1p0_gDM-1p0_13TeV-powheg/RunIISpring15DR74-Asympt25ns_MCRUN2_74_V9-v1/MINIAODSIM! \tabularnewline
\verb!/DMS_NNPDF30_Pseudoscalar_Mphi-20_Mchi-1_gSM-1p0_gDM-1p0_13TeV-powheg/RunIISpring15DR74-Asympt25ns_MCRUN2_74_V9-v1/MINIAODSIM! \tabularnewline
\verb!/DMS_NNPDF30_Pseudoscalar_Mphi-20_Mchi-10_gSM-1p0_gDM-1p0_13TeV-powheg/RunIISpring15DR74-Asympt25ns_MCRUN2_74_V9-v1/MINIAODSIM! \tabularnewline
\verb!/DMS_NNPDF30_Pseudoscalar_Mphi-50_Mchi-10_gSM-1p0_gDM-1p0_13TeV-powheg/RunIISpring15DR74-Asympt25ns_MCRUN2_74_V9-v1/MINIAODSIM! \tabularnewline
\verb!/DMS_NNPDF30_Pseudoscalar_Mphi-50_Mchi-50_gSM-1p0_gDM-1p0_13TeV-powheg/RunIISpring15DR74-Asympt25ns_MCRUN2_74_V9-v1/MINIAODSIM! \tabularnewline
\verb!/DMS_NNPDF30_Pseudoscalar_Mphi-100_Mchi-10_gSM-1p0_gDM-1p0_13TeV-powheg/RunIISpring15DR74-Asympt25ns_MCRUN2_74_V9-v1/MINIAODSIM! \tabularnewline
\verb!/DMS_NNPDF30_Pseudoscalar_Mphi-100_Mchi-100_gSM-1p0_gDM-1p0_13TeV-powheg/RunIISpring15DR74-Asympt25ns_MCRUN2_74_V9-v1/MINIAODSIM! \tabularnewline
\verb!/DMS_NNPDF30_Pseudoscalar_Mphi-200_Mchi-1_gSM-1p0_gDM-1p0_13TeV-powheg/RunIISpring15DR74-Asympt25ns_MCRUN2_74_V9-v1/MINIAODSIM! \tabularnewline
\verb!/DMS_NNPDF30_Pseudoscalar_Mphi-200_Mchi-10_gSM-1p0_gDM-1p0_13TeV-powheg/RunIISpring15DR74-Asympt25ns_MCRUN2_74_V9-v1/MINIAODSIM! \tabularnewline
\verb!/DMS_NNPDF30_Pseudoscalar_Mphi-200_Mchi-100_gSM-1p0_gDM-1p0_13TeV-powheg/RunIISpring15DR74-Asympt25ns_MCRUN2_74_V9-v1/MINIAODSIM! \tabularnewline
\verb!/DMS_NNPDF30_Pseudoscalar_Mphi-200_Mchi-150_gSM-1p0_gDM-1p0_13TeV-powheg/RunIISpring15DR74-Asympt25ns_MCRUN2_74_V9-v1/MINIAODSIM! \tabularnewline
\verb!/DMS_NNPDF30_Pseudoscalar_Mphi-300_Mchi-10_gSM-1p0_gDM-1p0_13TeV-powheg/RunIISpring15DR74-Asympt25ns_MCRUN2_74_V9-v1/MINIAODSIM! \tabularnewline
\verb!/DMS_NNPDF30_Pseudoscalar_Mphi-300_Mchi-50_gSM-1p0_gDM-1p0_13TeV-powheg/RunIISpring15DR74-Asympt25ns_MCRUN2_74_V9-v1/MINIAODSIM! \tabularnewline
\verb!/DMS_NNPDF30_Pseudoscalar_Mphi-300_Mchi-100_gSM-1p0_gDM-1p0_13TeV-powheg/RunIISpring15DR74-Asympt25ns_MCRUN2_74_V9-v1/MINIAODSIM! \tabularnewline
\verb!/DMS_NNPDF30_Pseudoscalar_Mphi-300_Mchi-150_gSM-1p0_gDM-1p0_13TeV-powheg/RunIISpring15DR74-Asympt25ns_MCRUN2_74_V9-v1/MINIAODSIM! \tabularnewline
\verb!/DMS_NNPDF30_Pseudoscalar_Mphi-500_Mchi-1_gSM-1p0_gDM-1p0_13TeV-powheg/RunIISpring15DR74-Asympt25ns_MCRUN2_74_V9-v1/MINIAODSIM! \tabularnewline
\verb!/DMS_NNPDF30_Pseudoscalar_Mphi-500_Mchi-50_gSM-1p0_gDM-1p0_13TeV-powheg/RunIISpring15DR74-Asympt25ns_MCRUN2_74_V9-v1/MINIAODSIM! \tabularnewline
\verb!/DMS_NNPDF30_Pseudoscalar_Mphi-500_Mchi-100_gSM-1p0_gDM-1p0_13TeV-powheg/RunIISpring15DR74-Asympt25ns_MCRUN2_74_V9-v1/MINIAODSIM! \tabularnewline
\verb!/DMS_NNPDF30_Pseudoscalar_Mphi-500_Mchi-150_gSM-1p0_gDM-1p0_13TeV-powheg/RunIISpring15DR74-Asympt25ns_MCRUN2_74_V9-v1/MINIAODSIM! \tabularnewline
\verb!/DMS_NNPDF30_Pseudoscalar_Mphi-500_Mchi-500_gSM-1p0_gDM-1p0_13TeV-powheg/RunIISpring15DR74-Asympt25ns_MCRUN2_74_V9-v1/MINIAODSIM! \tabularnewline
\verb!/DMS_NNPDF30_Pseudoscalar_Mphi-1000_Mchi-1_gSM-1p0_gDM-1p0_13TeV-powheg/RunIISpring15DR74-Asympt25ns_MCRUN2_74_V9-v1/MINIAODSIM! \tabularnewline
\verb!/DMS_NNPDF30_Pseudoscalar_Mphi-1000_Mchi-10_gSM-1p0_gDM-1p0_13TeV-powheg/RunIISpring15DR74-Asympt25ns_MCRUN2_74_V9-v1/MINIAODSIM! \tabularnewline
\verb!/DMS_NNPDF30_Pseudoscalar_Mphi-1000_Mchi-100_gSM-1p0_gDM-1p0_13TeV-powheg/RunIISpring15DR74-Asympt25ns_MCRUN2_74_V9-v1/MINIAODSIM! \tabularnewline
\verb!/DMS_NNPDF30_Pseudoscalar_Mphi-1000_Mchi-150_gSM-1p0_gDM-1p0_13TeV-powheg/RunIISpring15DR74-Asympt25ns_MCRUN2_74_V9-v1/MINIAODSIM! \tabularnewline
\verb!/DMS_NNPDF30_Pseudoscalar_Mphi-1000_Mchi-1000_gSM-1p0_gDM-1p0_13TeV-powheg/RunIISpring15DR74-Asympt25ns_MCRUN2_74_V9-v1/MINIAODSIM! \tabularnewline
\verb!/DMS_NNPDF30_Pseudoscalar_Mphi-2000_Mchi-1_gSM-1p0_gDM-1p0_13TeV-powheg/RunIISpring15DR74-Asympt25ns_MCRUN2_74_V9-v1/MINIAODSIM! \tabularnewline
\verb!/DMS_NNPDF30_Pseudoscalar_Mphi-2000_Mchi-10_gSM-1p0_gDM-1p0_13TeV-powheg/RunIISpring15DR74-Asympt25ns_MCRUN2_74_V9-v1/MINIAODSIM! \tabularnewline
\verb!/DMS_NNPDF30_Pseudoscalar_Mphi-2000_Mchi-50_gSM-1p0_gDM-1p0_13TeV-powheg/RunIISpring15DR74-Asympt25ns_MCRUN2_74_V9-v1/MINIAODSIM! \tabularnewline
\verb!/DMS_NNPDF30_Pseudoscalar_Mphi-2000_Mchi-150_gSM-1p0_gDM-1p0_13TeV-powheg/RunIISpring15DR74-Asympt25ns_MCRUN2_74_V9-v1/MINIAODSIM! \tabularnewline
\verb!/DMS_NNPDF30_Pseudoscalar_Mphi-2000_Mchi-500_gSM-1p0_gDM-1p0_13TeV-powheg/RunIISpring15DR74-Asympt25ns_MCRUN2_74_V9-v1/MINIAODSIM! \tabularnewline
\verb!/DMS_NNPDF30_Pseudoscalar_Mphi-2000_Mchi-1000_gSM-1p0_gDM-1p0_13TeV-powheg/RunIISpring15DR74-Asympt25ns_MCRUN2_74_V9-v1/MINIAODSIM! \tabularnewline
\verb!/DMS_NNPDF30_Pseudoscalar_Mphi-5000_Mchi-1_gSM-1p0_gDM-1p0_13TeV-powheg/RunIISpring15DR74-Asympt25ns_MCRUN2_74_V9-v1/MINIAODSIM! \tabularnewline
\verb!/DMS_NNPDF30_Pseudoscalar_Mphi-5000_Mchi-10_gSM-1p0_gDM-1p0_13TeV-powheg/RunIISpring15DR74-Asympt25ns_MCRUN2_74_V9-v1/MINIAODSIM! \tabularnewline
\verb!/DMS_NNPDF30_Pseudoscalar_Mphi-5000_Mchi-50_gSM-1p0_gDM-1p0_13TeV-powheg/RunIISpring15DR74-Asympt25ns_MCRUN2_74_V9-v1/MINIAODSIM! \tabularnewline
\verb!/DMS_NNPDF30_Pseudoscalar_Mphi-5000_Mchi-100_gSM-1p0_gDM-1p0_13TeV-powheg/RunIISpring15DR74-Asympt25ns_MCRUN2_74_V9-v1/MINIAODSIM! \tabularnewline
\verb!/DMS_NNPDF30_Pseudoscalar_Mphi-5000_Mchi-150_gSM-1p0_gDM-1p0_13TeV-powheg/RunIISpring15DR74-Asympt25ns_MCRUN2_74_V9-v1/MINIAODSIM! \tabularnewline
\verb!/DMS_NNPDF30_Pseudoscalar_Mphi-5000_Mchi-500_gSM-1p0_gDM-1p0_13TeV-powheg/RunIISpring15DR74-Asympt25ns_MCRUN2_74_V9-v1/MINIAODSIM! \tabularnewline
\verb!/DMS_NNPDF30_Pseudoscalar_Mphi-5000_Mchi-1000_gSM-1p0_gDM-1p0_13TeV-powheg/RunIISpring15DR74-Asympt25ns_MCRUN2_74_V9-v1/MINIAODSIM! \tabularnewline
\verb!/DMS_NNPDF30_Pseudoscalar_Mphi-10000_Mchi-1_gSM-1p0_gDM-1p0_13TeV-powheg/RunIISpring15DR74-Asympt25ns_MCRUN2_74_V9-v1/MINIAODSIM! \tabularnewline
\verb!/DMS_NNPDF30_Pseudoscalar_Mphi-10000_Mchi-10_gSM-1p0_gDM-1p0_13TeV-powheg/RunIISpring15DR74-Asympt25ns_MCRUN2_74_V9-v1/MINIAODSIM! \tabularnewline
\verb!/DMS_NNPDF30_Pseudoscalar_Mphi-10000_Mchi-50_gSM-1p0_gDM-1p0_13TeV-powheg/RunIISpring15DR74-Asympt25ns_MCRUN2_74_V9-v1/MINIAODSIM! \tabularnewline
\verb!/DMS_NNPDF30_Pseudoscalar_Mphi-10000_Mchi-100_gSM-1p0_gDM-1p0_13TeV-powheg/RunIISpring15DR74-Asympt25ns_MCRUN2_74_V9-v1/MINIAODSIM! \tabularnewline
\verb!/DMS_NNPDF30_Pseudoscalar_Mphi-10000_Mchi-150_gSM-1p0_gDM-1p0_13TeV-powheg/RunIISpring15DR74-Asympt25ns_MCRUN2_74_V9-v1/MINIAODSIM! \tabularnewline
\verb!/DMS_NNPDF30_Pseudoscalar_Mphi-10000_Mchi-500_gSM-1p0_gDM-1p0_13TeV-powheg/RunIISpring15DR74-Asympt25ns_MCRUN2_74_V9-v1/MINIAODSIM! \tabularnewline
\verb!/DMS_NNPDF30_Pseudoscalar_Mphi-10000_Mchi-1000_gSM-1p0_gDM-1p0_13TeV-powheg/RunIISpring15DR74-Asympt25ns_MCRUN2_74_V9-v1/MINIAODSIM! \tabularnewline
\hline
\end{tabular}\end{center}
}
\label{datasets_dm_pseudoscalar}
\end{table}

\begin{table}[!p]
 \centering
\topcaption{Simulated signal samples: DM \bbbar Scalar}
 \scriptsize
 \scalebox{.7}[1.0]{%latex.default(d, title = NULL, booktabs = FALSE, width = 3, rowname = NULL,     helvetica = FALSE, caption.loc = "bottom", ...)%
\begin{center}
\begin{tabular}{l}
\hline\hline
\multicolumn{1}{c}{Data set}\tabularnewline
\hline
\verb!/BBbarDMJets_scalar_Mchi-1_Mphi-10_TuneCUETP8M1_13TeV-madgraphMLM-pythia8/RunIISpring15DR74-Asympt25ns_MCRUN2_74_V9-v1/MINIAODSIM! \tabularnewline
\verb!/BBbarDMJets_scalar_Mchi-1_Mphi-20_TuneCUETP8M1_13TeV-madgraphMLM-pythia8/RunIISpring15DR74-Asympt25ns_MCRUN2_74_V9-v1/MINIAODSIM! \tabularnewline
\verb!/BBbarDMJets_scalar_Mchi-1_Mphi-50_TuneCUETP8M1_13TeV-madgraphMLM-pythia8/RunIISpring15DR74-Asympt25ns_MCRUN2_74_V9-v1/MINIAODSIM! \tabularnewline
\verb!/BBbarDMJets_scalar_Mchi-1_Mphi-100_TuneCUETP8M1_13TeV-madgraphMLM-pythia8/RunIISpring15DR74-Asympt25ns_MCRUN2_74_V9-v1/MINIAODSIM! \tabularnewline
\verb!/BBbarDMJets_scalar_Mchi-1_Mphi-200_TuneCUETP8M1_13TeV-madgraphMLM-pythia8/RunIISpring15DR74-Asympt25ns_MCRUN2_74_V9-v1/MINIAODSIM! \tabularnewline
\verb!/BBbarDMJets_scalar_Mchi-1_Mphi-300_TuneCUETP8M1_13TeV-madgraphMLM-pythia8/RunIISpring15DR74-Asympt25ns_MCRUN2_74_V9-v1/MINIAODSIM! \tabularnewline
\verb!/BBbarDMJets_scalar_Mchi-1_Mphi-500_TuneCUETP8M1_13TeV-madgraphMLM-pythia8/RunIISpring15DR74-Asympt25ns_MCRUN2_74_V9-v1/MINIAODSIM! \tabularnewline
\verb!/BBbarDMJets_scalar_Mchi-1_Mphi-1000_TuneCUETP8M1_13TeV-madgraphMLM-pythia8/RunIISpring15DR74-Asympt25ns_MCRUN2_74_V9-v1/MINIAODSIM! \tabularnewline
\verb!/BBbarDMJets_scalar_Mchi-1_Mphi-10000_TuneCUETP8M1_13TeV-madgraphMLM-pythia8/RunIISpring15DR74-Asympt25ns_MCRUN2_74_V9-v1/MINIAODSIM! \tabularnewline
\verb!/BBbarDMJets_scalar_Mchi-10_Mphi-10_TuneCUETP8M1_13TeV-madgraphMLM-pythia8/RunIISpring15DR74-Asympt25ns_MCRUN2_74_V9-v1/MINIAODSIM! \tabularnewline
\verb!/BBbarDMJets_scalar_Mchi-10_Mphi-15_TuneCUETP8M1_13TeV-madgraphMLM-pythia8/RunIISpring15DR74-Asympt25ns_MCRUN2_74_V9-v1/MINIAODSIM! \tabularnewline
\verb!/BBbarDMJets_scalar_Mchi-10_Mphi-50_TuneCUETP8M1_13TeV-madgraphMLM-pythia8/RunIISpring15DR74-Asympt25ns_MCRUN2_74_V9-v1/MINIAODSIM! \tabularnewline
\verb!/BBbarDMJets_scalar_Mchi-10_Mphi-100_TuneCUETP8M1_13TeV-madgraphMLM-pythia8/RunIISpring15DR74-Asympt25ns_MCRUN2_74_V9-v1/MINIAODSIM! \tabularnewline
\verb!/BBbarDMJets_scalar_Mchi-10_Mphi-10000_TuneCUETP8M1_13TeV-madgraphMLM-pythia8/RunIISpring15DR74-Asympt25ns_MCRUN2_74_V9-v1/MINIAODSIM! \tabularnewline
\verb!/BBbarDMJets_scalar_Mchi-50_Mphi-10_TuneCUETP8M1_13TeV-madgraphMLM-pythia8/RunIISpring15DR74-Asympt25ns_MCRUN2_74_V9-v1/MINIAODSIM! \tabularnewline
\verb!/BBbarDMJets_scalar_Mchi-50_Mphi-50_TuneCUETP8M1_13TeV-madgraphMLM-pythia8/RunIISpring15DR74-Asympt25ns_MCRUN2_74_V9-v1/MINIAODSIM! \tabularnewline
\verb!/BBbarDMJets_scalar_Mchi-50_Mphi-95_TuneCUETP8M1_13TeV-madgraphMLM-pythia8/RunIISpring15DR74-Asympt25ns_MCRUN2_74_V9-v1/MINIAODSIM! \tabularnewline
\verb!/BBbarDMJets_scalar_Mchi-150_Mphi-200_TuneCUETP8M1_13TeV-madgraphMLM-pythia8/RunIISpring15DR74-Asympt25ns_MCRUN2_74_V9-v1/MINIAODSIM! \tabularnewline
\verb!/BBbarDMJets_scalar_Mchi-150_Mphi-295_TuneCUETP8M1_13TeV-madgraphMLM-pythia8/RunIISpring15DR74-Asympt25ns_MCRUN2_74_V9-v1/MINIAODSIM! \tabularnewline
\verb!/BBbarDMJets_scalar_Mchi-150_Mphi-500_TuneCUETP8M1_13TeV-madgraphMLM-pythia8/RunIISpring15DR74-Asympt25ns_MCRUN2_74_V9-v1/MINIAODSIM! \tabularnewline
\verb!/BBbarDMJets_scalar_Mchi-150_Mphi-1000_TuneCUETP8M1_13TeV-madgraphMLM-pythia8/RunIISpring15DR74-Asympt25ns_MCRUN2_74_V9-v1/MINIAODSIM! \tabularnewline
\verb!/BBbarDMJets_scalar_Mchi-150_Mphi-10000_TuneCUETP8M1_13TeV-madgraphMLM-pythia8/RunIISpring15DR74-Asympt25ns_MCRUN2_74_V9-v1/MINIAODSIM! \tabularnewline
\verb!/BBbarDMJets_scalar_Mchi-500_Mphi-10_TuneCUETP8M1_13TeV-madgraphMLM-pythia8/RunIISpring15DR74-Asympt25ns_MCRUN2_74_V9-v1/MINIAODSIM! \tabularnewline
\verb!/BBbarDMJets_scalar_Mchi-500_Mphi-500_TuneCUETP8M1_13TeV-madgraphMLM-pythia8/RunIISpring15DR74-Asympt25ns_MCRUN2_74_V9-v1/MINIAODSIM! \tabularnewline
\verb!/BBbarDMJets_scalar_Mchi-500_Mphi-995_TuneCUETP8M1_13TeV-madgraphMLM-pythia8/RunIISpring15DR74-Asympt25ns_MCRUN2_74_V9-v1/MINIAODSIM! \tabularnewline
\verb!/BBbarDMJets_scalar_Mchi-500_Mphi-10000_TuneCUETP8M1_13TeV-madgraphMLM-pythia8/RunIISpring15DR74-Asympt25ns_MCRUN2_74_V9-v1/MINIAODSIM! \tabularnewline
\verb!/BBbarDMJets_scalar_Mchi-1000_Mphi-10_TuneCUETP8M1_13TeV-madgraphMLM-pythia8/RunIISpring15DR74-Asympt25ns_MCRUN2_74_V9-v1/MINIAODSIM! \tabularnewline
\verb!/BBbarDMJets_scalar_Mchi-1000_Mphi-1000_TuneCUETP8M1_13TeV-madgraphMLM-pythia8/RunIISpring15DR74-Asympt25ns_MCRUN2_74_V9-v1/MINIAODSIM! \tabularnewline
\verb!/BBbarDMJets_scalar_Mchi-1000_Mphi-10000_TuneCUETP8M1_13TeV-madgraphMLM-pythia8/RunIISpring15DR74-Asympt25ns_MCRUN2_74_V9-v1/MINIAODSIM! \tabularnewline
\hline
\end{tabular}\end{center}
}
\label{datasets_dm_bbar_pseudoscalar}
\end{table}

\begin{table}[!p]
 \centering
\topcaption{Simulated signal samples: DM \bbbar Pseudoscalar}
 \scriptsize
 \scalebox{.7}[1.0]{%latex.default(d, title = NULL, booktabs = FALSE, width = 3, rowname = NULL,     helvetica = FALSE, caption.loc = "bottom", ...)%
\begin{center}
\begin{tabular}{l}
\hline\hline
\multicolumn{1}{c}{Data set}\tabularnewline
\hline
\verb!/BBbarDMJets_pseudoscalar_Mchi-1_Mphi-10_TuneCUETP8M1_13TeV-madgraphMLM-pythia8/RunIISpring15DR74-Asympt25ns_MCRUN2_74_V9-v1/MINIAODSIM! \tabularnewline
\verb!/BBbarDMJets_pseudoscalar_Mchi-1_Mphi-20_TuneCUETP8M1_13TeV-madgraphMLM-pythia8/RunIISpring15DR74-Asympt25ns_MCRUN2_74_V9-v1/MINIAODSIM! \tabularnewline
\verb!/BBbarDMJets_pseudoscalar_Mchi-1_Mphi-50_TuneCUETP8M1_13TeV-madgraphMLM-pythia8/RunIISpring15DR74-Asympt25ns_MCRUN2_74_V9-v1/MINIAODSIM! \tabularnewline
\verb!/BBbarDMJets_pseudoscalar_Mchi-1_Mphi-100_TuneCUETP8M1_13TeV-madgraphMLM-pythia8/RunIISpring15DR74-Asympt25ns_MCRUN2_74_V9-v1/MINIAODSIM! \tabularnewline
\verb!/BBbarDMJets_pseudoscalar_Mchi-1_Mphi-200_TuneCUETP8M1_13TeV-madgraphMLM-pythia8/RunIISpring15DR74-Asympt25ns_MCRUN2_74_V9-v1/MINIAODSIM! \tabularnewline
\verb!/BBbarDMJets_pseudoscalar_Mchi-1_Mphi-300_TuneCUETP8M1_13TeV-madgraphMLM-pythia8/RunIISpring15DR74-Asympt25ns_MCRUN2_74_V9-v1/MINIAODSIM! \tabularnewline
\verb!/BBbarDMJets_pseudoscalar_Mchi-1_Mphi-500_TuneCUETP8M1_13TeV-madgraphMLM-pythia8/RunIISpring15DR74-Asympt25ns_MCRUN2_74_V9-v1/MINIAODSIM! \tabularnewline
\verb!/BBbarDMJets_pseudoscalar_Mchi-1_Mphi-1000_TuneCUETP8M1_13TeV-madgraphMLM-pythia8/RunIISpring15DR74-Asympt25ns_MCRUN2_74_V9-v1/MINIAODSIM! \tabularnewline
\verb!/BBbarDMJets_pseudoscalar_Mchi-1_Mphi-10000_TuneCUETP8M1_13TeV-madgraphMLM-pythia8/RunIISpring15DR74-Asympt25ns_MCRUN2_74_V9-v1/MINIAODSIM! \tabularnewline
\verb!/BBbarDMJets_pseudoscalar_Mchi-10_Mphi-10_TuneCUETP8M1_13TeV-madgraphMLM-pythia8/RunIISpring15DR74-Asympt25ns_MCRUN2_74_V9-v1/MINIAODSIM! \tabularnewline
\verb!/BBbarDMJets_pseudoscalar_Mchi-10_Mphi-15_TuneCUETP8M1_13TeV-madgraphMLM-pythia8/RunIISpring15DR74-Asympt25ns_MCRUN2_74_V9-v1/MINIAODSIM! \tabularnewline
\verb!/BBbarDMJets_pseudoscalar_Mchi-10_Mphi-50_TuneCUETP8M1_13TeV-madgraphMLM-pythia8/RunIISpring15DR74-Asympt25ns_MCRUN2_74_V9-v1/MINIAODSIM! \tabularnewline
\verb!/BBbarDMJets_pseudoscalar_Mchi-10_Mphi-100_TuneCUETP8M1_13TeV-madgraphMLM-pythia8/RunIISpring15DR74-Asympt25ns_MCRUN2_74_V9-v1/MINIAODSIM! \tabularnewline
\verb!/BBbarDMJets_pseudoscalar_Mchi-10_Mphi-10000_TuneCUETP8M1_13TeV-madgraphMLM-pythia8/RunIISpring15DR74-Asympt25ns_MCRUN2_74_V9-v1/MINIAODSIM! \tabularnewline
\verb!/BBbarDMJets_pseudoscalar_Mchi-50_Mphi-10_TuneCUETP8M1_13TeV-madgraphMLM-pythia8/RunIISpring15DR74-Asympt25ns_MCRUN2_74_V9-v1/MINIAODSIM! \tabularnewline
\verb!/BBbarDMJets_pseudoscalar_Mchi-50_Mphi-50_TuneCUETP8M1_13TeV-madgraphMLM-pythia8/RunIISpring15DR74-Asympt25ns_MCRUN2_74_V9-v1/MINIAODSIM! \tabularnewline
\verb!/BBbarDMJets_pseudoscalar_Mchi-50_Mphi-95_TuneCUETP8M1_13TeV-madgraphMLM-pythia8/RunIISpring15DR74-Asympt25ns_MCRUN2_74_V9-v1/MINIAODSIM! \tabularnewline
\verb!/BBbarDMJets_pseudoscalar_Mchi-50_Mphi-200_TuneCUETP8M1_13TeV-madgraphMLM-pythia8/RunIISpring15DR74-Asympt25ns_MCRUN2_74_V9-v1/MINIAODSIM! \tabularnewline
\verb!/BBbarDMJets_pseudoscalar_Mchi-150_Mphi-200_TuneCUETP8M1_13TeV-madgraphMLM-pythia8/RunIISpring15DR74-Asympt25ns_MCRUN2_74_V9-v1/MINIAODSIM! \tabularnewline
\verb!/BBbarDMJets_pseudoscalar_Mchi-150_Mphi-295_TuneCUETP8M1_13TeV-madgraphMLM-pythia8/RunIISpring15DR74-Asympt25ns_MCRUN2_74_V9-v1/MINIAODSIM! \tabularnewline
\verb!/BBbarDMJets_pseudoscalar_Mchi-150_Mphi-500_TuneCUETP8M1_13TeV-madgraphMLM-pythia8/RunIISpring15DR74-Asympt25ns_MCRUN2_74_V9-v1/MINIAODSIM! \tabularnewline
\verb!/BBbarDMJets_pseudoscalar_Mchi-150_Mphi-1000_TuneCUETP8M1_13TeV-madgraphMLM-pythia8/RunIISpring15DR74-Asympt25ns_MCRUN2_74_V9-v1/MINIAODSIM! \tabularnewline
\verb!/BBbarDMJets_pseudoscalar_Mchi-150_Mphi-10000_TuneCUETP8M1_13TeV-madgraphMLM-pythia8/RunIISpring15DR74-Asympt25ns_MCRUN2_74_V9-v1/MINIAODSIM! \tabularnewline
\verb!/BBbarDMJets_pseudoscalar_Mchi-500_Mphi-10_TuneCUETP8M1_13TeV-madgraphMLM-pythia8/RunIISpring15DR74-Asympt25ns_MCRUN2_74_V9-v1/MINIAODSIM! \tabularnewline
\verb!/BBbarDMJets_pseudoscalar_Mchi-500_Mphi-500_TuneCUETP8M1_13TeV-madgraphMLM-pythia8/RunIISpring15DR74-Asympt25ns_MCRUN2_74_V9-v1/MINIAODSIM! \tabularnewline
\verb!/BBbarDMJets_pseudoscalar_Mchi-500_Mphi-995_TuneCUETP8M1_13TeV-madgraphMLM-pythia8/RunIISpring15DR74-Asympt25ns_MCRUN2_74_V9-v1/MINIAODSIM! \tabularnewline
\verb!/BBbarDMJets_pseudoscalar_Mchi-500_Mphi-10000_TuneCUETP8M1_13TeV-madgraphMLM-pythia8/RunIISpring15DR74-Asympt25ns_MCRUN2_74_V9-v1/MINIAODSIM! \tabularnewline
\verb!/BBbarDMJets_pseudoscalar_Mchi-1000_Mphi-10_TuneCUETP8M1_13TeV-madgraphMLM-pythia8/RunIISpring15DR74-Asympt25ns_MCRUN2_74_V9-v1/MINIAODSIM! \tabularnewline
\verb!/BBbarDMJets_pseudoscalar_Mchi-1000_Mphi-1000_TuneCUETP8M1_13TeV-madgraphMLM-pythia8/RunIISpring15DR74-Asympt25ns_MCRUN2_74_V9-v1/MINIAODSIM! \tabularnewline
\verb!/BBbarDMJets_pseudoscalar_Mchi-1000_Mphi-10000_TuneCUETP8M1_13TeV-madgraphMLM-pythia8/RunIISpring15DR74-Asympt25ns_MCRUN2_74_V9-v1/MINIAODSIM! \tabularnewline
\hline
\end{tabular}\end{center}
}
\label{datasets_dm_bbar_pseudoscalar}
\end{table}

\begin{table}[!p]
 \centering
 \topcaption{Simulated signal samples: DM \ttbar Scalar}
 \scriptsize
 \scalebox{.7}[1.0]{%latex.default(d, title = NULL, booktabs = FALSE, width = 3, rowname = NULL,     helvetica = FALSE, caption.loc = "bottom", ...)%
\begin{center}
\begin{tabular}{l}
\hline\hline
\multicolumn{1}{c}{Data set}\tabularnewline
\hline
\verb!/TTbarDMJets_scalar_Mchi-1_Mphi-10_TuneCUETP8M1_13TeV-madgraphMLM-pythia8/RunIISpring15DR74-Asympt25ns_MCRUN2_74_V9-v1/MINIAODSIM! \tabularnewline
\verb!/TTbarDMJets_scalar_Mchi-1_Mphi-20_TuneCUETP8M1_13TeV-madgraphMLM-pythia8/RunIISpring15DR74-Asympt25ns_MCRUN2_74_V9-v1/MINIAODSIM! \tabularnewline
\verb!/TTbarDMJets_scalar_Mchi-1_Mphi-50_TuneCUETP8M1_13TeV-madgraphMLM-pythia8/RunIISpring15DR74-Asympt25ns_MCRUN2_74_V9-v1/MINIAODSIM! \tabularnewline
\verb!/TTbarDMJets_scalar_Mchi-1_Mphi-100_TuneCUETP8M1_13TeV-madgraphMLM-pythia8/RunIISpring15DR74-Asympt25ns_MCRUN2_74_V9-v1/MINIAODSIM! \tabularnewline
\verb!/TTbarDMJets_scalar_Mchi-1_Mphi-200_TuneCUETP8M1_13TeV-madgraphMLM-pythia8/RunIISpring15DR74-Asympt25ns_MCRUN2_74_V9-v1/MINIAODSIM! \tabularnewline
\verb!/TTbarDMJets_scalar_Mchi-1_Mphi-300_TuneCUETP8M1_13TeV-madgraphMLM-pythia8/RunIISpring15DR74-Asympt25ns_MCRUN2_74_V9-v1/MINIAODSIM! \tabularnewline
\verb!/TTbarDMJets_scalar_Mchi-1_Mphi-500_TuneCUETP8M1_13TeV-madgraphMLM-pythia8/RunIISpring15DR74-Asympt25ns_MCRUN2_74_V9-v1/MINIAODSIM! \tabularnewline
\verb!/TTbarDMJets_scalar_Mchi-1_Mphi-1000_TuneCUETP8M1_13TeV-madgraphMLM-pythia8/RunIISpring15DR74-Asympt25ns_MCRUN2_74_V9-v2/MINIAODSIM! \tabularnewline
\verb!/TTbarDMJets_scalar_Mchi-10_Mphi-10_TuneCUETP8M1_13TeV-madgraphMLM-pythia8/RunIISpring15DR74-Asympt25ns_MCRUN2_74_V9-v1/MINIAODSIM! \tabularnewline
\verb!/TTbarDMJets_scalar_Mchi-10_Mphi-50_TuneCUETP8M1_13TeV-madgraphMLM-pythia8/RunIISpring15DR74-Asympt25ns_MCRUN2_74_V9-v1/MINIAODSIM! \tabularnewline
\verb!/TTbarDMJets_scalar_Mchi-10_Mphi-100_TuneCUETP8M1_13TeV-madgraphMLM-pythia8/RunIISpring15DR74-Asympt25ns_MCRUN2_74_V9-v1/MINIAODSIM! \tabularnewline
\verb!/TTbarDMJets_scalar_Mchi-50_Mphi-50_TuneCUETP8M1_13TeV-madgraphMLM-pythia8/RunIISpring15DR74-Asympt25ns_MCRUN2_74_V9-v1/MINIAODSIM! \tabularnewline
\verb!/TTbarDMJets_scalar_Mchi-50_Mphi-200_TuneCUETP8M1_13TeV-madgraphMLM-pythia8/RunIISpring15DR74-Asympt25ns_MCRUN2_74_V9-v1/MINIAODSIM! \tabularnewline
\verb!/TTbarDMJets_scalar_Mchi-50_Mphi-300_TuneCUETP8M1_13TeV-madgraphMLM-pythia8/RunIISpring15DR74-Asympt25ns_MCRUN2_74_V9-v1/MINIAODSIM! \tabularnewline
\verb!/TTbarDMJets_scalar_Mchi-150_Mphi-200_TuneCUETP8M1_13TeV-madgraphMLM-pythia8/RunIISpring15DR74-Asympt25ns_MCRUN2_74_V9-v1/MINIAODSIM! \tabularnewline
\verb!/TTbarDMJets_scalar_Mchi-150_Mphi-500_TuneCUETP8M1_13TeV-madgraphMLM-pythia8/RunIISpring15DR74-Asympt25ns_MCRUN2_74_V9-v3/MINIAODSIM! \tabularnewline
\verb!/TTbarDMJets_scalar_Mchi-150_Mphi-1000_TuneCUETP8M1_13TeV-madgraphMLM-pythia8/RunIISpring15DR74-Asympt25ns_MCRUN2_74_V9-v3/MINIAODSIM! \tabularnewline
\verb!/TTbarDMJets_scalar_Mchi-500_Mphi-500_TuneCUETP8M1_13TeV-madgraphMLM-pythia8/RunIISpring15DR74-Asympt25ns_MCRUN2_74_V9-v1/MINIAODSIM! \tabularnewline
\hline
\end{tabular}\end{center}
}
\label{datasets_dm_ttbar_scalar}
\end{table}

\begin{table}[!p]
 \centering
 \topcaption{Simulated signal samples: DM \ttbar Pseudoscalar}
 \scriptsize
 \scalebox{.7}[1.0]{%latex.default(d, title = NULL, booktabs = FALSE, width = 3, rowname = NULL,     helvetica = FALSE, caption.loc = "bottom", ...)%
\begin{center}
\begin{tabular}{l}
\hline\hline
\multicolumn{1}{c}{Data set}\tabularnewline
\hline
\verb!/TTbarDMJets_pseudoscalar_Mchi-1_Mphi-10_TuneCUETP8M1_13TeV-madgraphMLM-pythia8/RunIISpring15DR74-Asympt25ns_MCRUN2_74_V9-v1/MINIAODSIM! \tabularnewline
\verb!/TTbarDMJets_pseudoscalar_Mchi-1_Mphi-20_TuneCUETP8M1_13TeV-madgraphMLM-pythia8/RunIISpring15DR74-Asympt25ns_MCRUN2_74_V9-v1/MINIAODSIM! \tabularnewline
\verb!/TTbarDMJets_pseudoscalar_Mchi-1_Mphi-50_TuneCUETP8M1_13TeV-madgraphMLM-pythia8/RunIISpring15DR74-Asympt25ns_MCRUN2_74_V9-v1/MINIAODSIM! \tabularnewline
\verb!/TTbarDMJets_pseudoscalar_Mchi-1_Mphi-100_TuneCUETP8M1_13TeV-madgraphMLM-pythia8/RunIISpring15DR74-Asympt25ns_MCRUN2_74_V9-v1/MINIAODSIM! \tabularnewline
\verb!/TTbarDMJets_pseudoscalar_Mchi-1_Mphi-200_TuneCUETP8M1_13TeV-madgraphMLM-pythia8/RunIISpring15DR74-Asympt25ns_MCRUN2_74_V9-v1/MINIAODSIM! \tabularnewline
\verb!/TTbarDMJets_pseudoscalar_Mchi-1_Mphi-300_TuneCUETP8M1_13TeV-madgraphMLM-pythia8/RunIISpring15DR74-Asympt25ns_MCRUN2_74_V9-v1/MINIAODSIM! \tabularnewline
\verb!/TTbarDMJets_pseudoscalar_Mchi-1_Mphi-500_TuneCUETP8M1_13TeV-madgraphMLM-pythia8/RunIISpring15DR74-Asympt25ns_MCRUN2_74_V9-v1/MINIAODSIM! \tabularnewline
\verb!/TTbarDMJets_pseudoscalar_Mchi-10_Mphi-10_TuneCUETP8M1_13TeV-madgraphMLM-pythia8/RunIISpring15DR74-Asympt25ns_MCRUN2_74_V9-v1/MINIAODSIM! \tabularnewline
\verb!/TTbarDMJets_pseudoscalar_Mchi-10_Mphi-50_TuneCUETP8M1_13TeV-madgraphMLM-pythia8/RunIISpring15DR74-Asympt25ns_MCRUN2_74_V9-v1/MINIAODSIM! \tabularnewline
\verb!/TTbarDMJets_pseudoscalar_Mchi-10_Mphi-100_TuneCUETP8M1_13TeV-madgraphMLM-pythia8/RunIISpring15DR74-Asympt25ns_MCRUN2_74_V9-v1/MINIAODSIM! \tabularnewline
\verb!/TTbarDMJets_pseudoscalar_Mchi-50_Mphi-50_TuneCUETP8M1_13TeV-madgraphMLM-pythia8/RunIISpring15DR74-Asympt25ns_MCRUN2_74_V9-v1/MINIAODSIM! \tabularnewline
\verb!/TTbarDMJets_pseudoscalar_Mchi-50_Mphi-200_TuneCUETP8M1_13TeV-madgraphMLM-pythia8/RunIISpring15DR74-Asympt25ns_MCRUN2_74_V9-v1/MINIAODSIM! \tabularnewline
\verb!/TTbarDMJets_pseudoscalar_Mchi-50_Mphi-300_TuneCUETP8M1_13TeV-madgraphMLM-pythia8/RunIISpring15DR74-Asympt25ns_MCRUN2_74_V9-v1/MINIAODSIM! \tabularnewline
\verb!/TTbarDMJets_pseudoscalar_Mchi-150_Mphi-200_TuneCUETP8M1_13TeV-madgraphMLM-pythia8/RunIISpring15DR74-Asympt25ns_MCRUN2_74_V9-v1/MINIAODSIM! \tabularnewline
\verb!/TTbarDMJets_pseudoscalar_Mchi-150_Mphi-500_TuneCUETP8M1_13TeV-madgraphMLM-pythia8/RunIISpring15DR74-Asympt25ns_MCRUN2_74_V9-v1/MINIAODSIM! \tabularnewline
\verb!/TTbarDMJets_pseudoscalar_Mchi-150_Mphi-1000_TuneCUETP8M1_13TeV-madgraphMLM-pythia8/RunIISpring15DR74-Asympt25ns_MCRUN2_74_V9-v1/MINIAODSIM! \tabularnewline
\verb!/TTbarDMJets_pseudoscalar_Mchi-500_Mphi-500_TuneCUETP8M1_13TeV-madgraphMLM-pythia8/RunIISpring15DR74-Asympt25ns_MCRUN2_74_V9-v1/MINIAODSIM! \tabularnewline
\hline
\end{tabular}\end{center}
}
 \label{tab:datasets_dm_ttbar_pseudoscalar}
\end{table}





