\section{Data sets}
\label{sec:datasets}

\subsection{Data}


In this note, we use 2.1~\ifb proton-proton collision data at $\sqrt{s} =$ 13~TeV collected in 2015. The details of the data and
simulated samples for background processes are given in Ref.~\cite{alphaTnote}. In this note we focus on DM related items, including the simulated DM samples. 

\subsection{Simulation}

The generic signature of dark matter pair production in colliders is missing
transverse momentum from the dark matter along with recoiling energetic visible
particles that are used to trigger the event. First analyses using contact 
operators~\cite{Goodman:2010ku} in effective field theories (EFTs) calculated for a set of possible
couplings were used to interpret dark matter searches. Experimental limits using
monojet final states have been published using 7 and 8 TeV LHC 
data~\cite{Chatrchyan:2012me,ATLAS:2012ky}. 

The lack of predictive information and severe validity constraints of EFTs have
led to the development of improved minimal simplified dark matter (MSDM) models.
These models enable the comparison between different experimental
searches in a relatively model-independent way~\cite{Buchmueller:2014yoa}.

The \alphat analysis is sensitive to DM production in association with both
light ($g,~u,~d,~c,~s$) and heavy flavour ($b,~t$) jets. 

The simplified models we consider here correspond to the  ATLAS-CMS-Theory DM Forum 
for early LHC Run~2 searches. These samples populate key regions of the
{\mphi-\mchi} mass plane, as shown in Tab.~\ref{tab:DMgrid}

\begin{table}[h!] \centering \begin{tabular}{l|llllllllll}\hline \hline
$m_\textrm{DM}$  & \multicolumn{10}{c}{$m_\Phi$}
\\ \hline 1    & 10 & 20 & 50 & 100 & 200 & 300 & 500 & 1000 & 2000 & 10000 \\
10   & 10 & 15 & 50 & 100 &     &     &     &      &      & 10000 \\ 50   & 10 &
& 50 & 95  & 200 & 300 &     &      &      & 10000 \\ 150  & 10 &    &    &
& 200 & 295 & 500 & 1000 &      & 10000 \\ 500  & 10 &    &    &     &     &
& 500 & 995  &      & 10000 \\ 1000 & 10 &    &    &     &     &     &     &
1000 & 1995 & 10000\\ \hline \hline \end{tabular} \caption{Benchmark dark
matter and mediator masses. The parameter space follows the
DM Forum recommendations~\cite{Abercrombie:2015wmb}. Points are chosen roughly
equidistant on a logarithmic scale. Points on the on-shell diagonal are always 
chosen to be 5 GeV away from the threshold to avoid numerical instabilities in 
the event generation.} \label{tab:DMgrid} \end{table}


\dub

Centrally produced signal samples for (vector-)axial and (pseudo-)scalar with light and heavy quarks in the final states are used.
These samples are given in Tables~\ref{datasets_dm_vector}-~\ref{tab:datasets_dm_ttbar_pseudoscalar}. In these data sets, in addition to the main interaction, each
event contains on average 20 minimum bias interactions which simulate
multiple interactions per bunch-crossing (in-time pileup). The expected
detector signal from previous or following bunch crossings (out-of-time
pileup) with 25ns bunch spacing is overlapped.

\begin{table}[!p]
 \centering
\topcaption{Simulated signal samples: DM Vector}
 \scriptsize
 \scalebox{.7}[1.0]{%latex.default(d, title = NULL, booktabs = FALSE, width = 3, rowname = NULL,     helvetica = FALSE, caption.loc = "bottom", ...)%
\begin{center}
\begin{tabular}{l}
\hline\hline
\multicolumn{1}{c}{Data set}\tabularnewline
\hline
\verb!/DMV_NNPDF30_Vector_Mchi-10_Mchi-1_gSM-0p25_gDM-1p0_13TeV-powheg/RunIISpring15DR74-Asympt25ns_MCRUN2_74_V9-v1/MINIAODSIM! \tabularnewline
\verb!/DMV_NNPDF30_Vector_Mphi-10_Mchi-1_gSM-1p0_gDM-1p0_13TeV-powheg/RunIISpring15DR74-Asympt25ns_MCRUN2_74_V9-v1/MINIAODSIM! \tabularnewline
\verb!/DMV_NNPDF30_Vector_Mchi-10_Mchi-10_gSM-0p25_gDM-1p0_13TeV-powheg/RunIISpring15DR74-Asympt25ns_MCRUN2_74_V9-v1/MINIAODSIM! \tabularnewline
\verb!/DMV_NNPDF30_Vector_Mphi-10_Mchi-10_gSM-1p0_gDM-1p0_13TeV-powheg/RunIISpring15DR74-Asympt25ns_MCRUN2_74_V9-v1/MINIAODSIM! \tabularnewline
\verb!/DMV_NNPDF30_Vector_Mchi-10_Mchi-50_gSM-0p25_gDM-1p0_13TeV-powheg/RunIISpring15DR74-Asympt25ns_MCRUN2_74_V9-v1/MINIAODSIM! \tabularnewline
\verb!/DMV_NNPDF30_Vector_Mchi-10_Mchi-100_gSM-0p25_gDM-1p0_13TeV-powheg/RunIISpring15DR74-Asympt25ns_MCRUN2_74_V9-v1/MINIAODSIM! \tabularnewline
\verb!/DMV_NNPDF30_Vector_Mchi-10_Mchi-150_gSM-0p25_gDM-1p0_13TeV-powheg/RunIISpring15DR74-Asympt25ns_MCRUN2_74_V9-v1/MINIAODSIM! \tabularnewline
\verb!/DMV_NNPDF30_Vector_Mchi-10_Mchi-500_gSM-0p25_gDM-1p0_13TeV-powheg/RunIISpring15DR74-Asympt25ns_MCRUN2_74_V9-v1/MINIAODSIM! \tabularnewline
\verb!/DMV_NNPDF30_Vector_Mchi-20_Mchi-1_gSM-0p25_gDM-1p0_13TeV-powheg/RunIISpring15DR74-Asympt25ns_MCRUN2_74_V9-v1/MINIAODSIM! \tabularnewline
\verb!/DMV_NNPDF30_Vector_Mphi-20_Mchi-1_gSM-1p0_gDM-1p0_13TeV-powheg/RunIISpring15DR74-Asympt25ns_MCRUN2_74_V9-v1/MINIAODSIM! \tabularnewline
\verb!/DMV_NNPDF30_Vector_Mphi-50_Mchi-1_gSM-1p0_gDM-1p0_13TeV-powheg/RunIISpring15DR74-Asympt25ns_MCRUN2_74_V9-v1/MINIAODSIM! \tabularnewline
\verb!/DMV_NNPDF30_Vector_Mphi-50_Mchi-10_gSM-1p0_gDM-1p0_13TeV-powheg/RunIISpring15DR74-Asympt25ns_MCRUN2_74_V9-v1/MINIAODSIM! \tabularnewline
\verb!/DMV_NNPDF30_Vector_Mchi-50_Mchi-50_gSM-0p25_gDM-1p0_13TeV-powheg/RunIISpring15DR74-Asympt25ns_MCRUN2_74_V9-v1/MINIAODSIM! \tabularnewline
\verb!/DMV_NNPDF30_Vector_Mphi-100_Mchi-1_gSM-0p25_gDM-1p0_13TeV-powheg/RunIISpring15DR74-Asympt25ns_MCRUN2_74_V9-v1/MINIAODSIM! \tabularnewline
\verb!/DMV_NNPDF30_Vector_Mphi-100_Mchi-10_gSM-0p25_gDM-1p0_13TeV-powheg/RunIISpring15DR74-Asympt25ns_MCRUN2_74_V9-v1/MINIAODSIM! \tabularnewline
\verb!/DMV_NNPDF30_Vector_Mphi-100_Mchi-10_gSM-1p0_gDM-1p0_13TeV-powheg/RunIISpring15DR74-Asympt25ns_MCRUN2_74_V9-v1/MINIAODSIM! \tabularnewline
\verb!/DMV_NNPDF30_Vector_Mphi-100_Mchi-100_gSM-0p25_gDM-1p0_13TeV-powheg/RunIISpring15DR74-Asympt25ns_MCRUN2_74_V9-v1/MINIAODSIM! \tabularnewline
\verb!/DMV_NNPDF30_Vector_Mphi-100_Mchi-100_gSM-1p0_gDM-1p0_13TeV-powheg/RunIISpring15DR74-Asympt25ns_MCRUN2_74_V9-v1/MINIAODSIM! \tabularnewline
\verb!/DMV_NNPDF30_Vector_Mphi-200_Mchi-1_gSM-0p25_gDM-1p0_13TeV-powheg/RunIISpring15DR74-Asympt25ns_MCRUN2_74_V9-v1/MINIAODSIM! \tabularnewline
\verb!/DMV_NNPDF30_Vector_Mphi-200_Mchi-1_gSM-1p0_gDM-1p0_13TeV-powheg/RunIISpring15DR74-Asympt25ns_MCRUN2_74_V9-v1/MINIAODSIM! \tabularnewline
\verb!/DMV_NNPDF30_Vector_Mphi-200_Mchi-10_gSM-0p25_gDM-1p0_13TeV-powheg/RunIISpring15DR74-Asympt25ns_MCRUN2_74_V9-v1/MINIAODSIM! \tabularnewline
\verb!/DMV_NNPDF30_Vector_Mphi-200_Mchi-10_gSM-1p0_gDM-1p0_13TeV-powheg/RunIISpring15DR74-Asympt25ns_MCRUN2_74_V9-v1/MINIAODSIM! \tabularnewline
\verb!/DMV_NNPDF30_Vector_Mphi-200_Mchi-50_gSM-0p25_gDM-1p0_13TeV-powheg/RunIISpring15DR74-Asympt25ns_MCRUN2_74_V9-v1/MINIAODSIM! \tabularnewline
\verb!/DMV_NNPDF30_Vector_Mphi-200_Mchi-50_gSM-1p0_gDM-1p0_13TeV-powheg/RunIISpring15DR74-Asympt25ns_MCRUN2_74_V9-v1/MINIAODSIM! \tabularnewline
\verb!/DMV_NNPDF30_Vector_Mphi-200_Mchi-100_gSM-1p0_gDM-1p0_13TeV-powheg/RunIISpring15DR74-Asympt25ns_MCRUN2_74_V9-v1/MINIAODSIM! \tabularnewline
\verb!/DMV_NNPDF30_Vector_Mphi-200_Mchi-150_gSM-0p25_gDM-1p0_13TeV-powheg/RunIISpring15DR74-Asympt25ns_MCRUN2_74_V9-v1/MINIAODSIM! \tabularnewline
\verb!/DMV_NNPDF30_Vector_Mphi-200_Mchi-150_gSM-1p0_gDM-1p0_13TeV-powheg/RunIISpring15DR74-Asympt25ns_MCRUN2_74_V9-v1/MINIAODSIM! \tabularnewline
\verb!/DMV_NNPDF30_Vector_Mphi-300_Mchi-1_gSM-0p25_gDM-1p0_13TeV-powheg/RunIISpring15DR74-Asympt25ns_MCRUN2_74_V9-v1/MINIAODSIM! \tabularnewline
\verb!/DMV_NNPDF30_Vector_Mphi-300_Mchi-10_gSM-1p0_gDM-1p0_13TeV-powheg/RunIISpring15DR74-Asympt25ns_MCRUN2_74_V9-v1/MINIAODSIM! \tabularnewline
\verb!/DMV_NNPDF30_Vector_Mphi-300_Mchi-50_gSM-0p25_gDM-1p0_13TeV-powheg/RunIISpring15DR74-Asympt25ns_MCRUN2_74_V9-v1/MINIAODSIM! \tabularnewline
\verb!/DMV_NNPDF30_Vector_Mphi-300_Mchi-50_gSM-1p0_gDM-1p0_13TeV-powheg/RunIISpring15DR74-Asympt25ns_MCRUN2_74_V9-v1/MINIAODSIM! \tabularnewline
\verb!/DMV_NNPDF30_Vector_Mphi-300_Mchi-100_gSM-0p25_gDM-1p0_13TeV-powheg/RunIISpring15DR74-Asympt25ns_MCRUN2_74_V9-v1/MINIAODSIM! \tabularnewline
\verb!/DMV_NNPDF30_Vector_Mphi-300_Mchi-100_gSM-1p0_gDM-1p0_13TeV-powheg/RunIISpring15DR74-Asympt25ns_MCRUN2_74_V9-v1/MINIAODSIM! \tabularnewline
\verb!/DMV_NNPDF30_Vector_Mphi-300_Mchi-150_gSM-0p25_gDM-1p0_13TeV-powheg/RunIISpring15DR74-Asympt25ns_MCRUN2_74_V9-v1/MINIAODSIM! \tabularnewline
\verb!/DMV_NNPDF30_Vector_Mphi-300_Mchi-150_gSM-1p0_gDM-1p0_13TeV-powheg/RunIISpring15DR74-Asympt25ns_MCRUN2_74_V9-v1/MINIAODSIM! \tabularnewline
\verb!/DMV_NNPDF30_Vector_Mphi-500_Mchi-1_gSM-0p25_gDM-1p0_13TeV-powheg/RunIISpring15DR74-Asympt25ns_MCRUN2_74_V9-v1/MINIAODSIM! \tabularnewline
\verb!/DMV_NNPDF30_Vector_Mphi-500_Mchi-1_gSM-1p0_gDM-1p0_13TeV-powheg/RunIISpring15DR74-Asympt25ns_MCRUN2_74_V9-v1/MINIAODSIM! \tabularnewline
\verb!/DMV_NNPDF30_Vector_Mphi-500_Mchi-10_gSM-0p25_gDM-1p0_13TeV-powheg/RunIISpring15DR74-Asympt25ns_MCRUN2_74_V9-v1/MINIAODSIM! \tabularnewline
\verb!/DMV_NNPDF30_Vector_Mphi-500_Mchi-10_gSM-1p0_gDM-1p0_13TeV-powheg/RunIISpring15DR74-Asympt25ns_MCRUN2_74_V9-v1/MINIAODSIM! \tabularnewline
\verb!/DMV_NNPDF30_Vector_Mphi-500_Mchi-50_gSM-1p0_gDM-1p0_13TeV-powheg/RunIISpring15DR74-Asympt25ns_MCRUN2_74_V9-v1/MINIAODSIM! \tabularnewline
\verb!/DMV_NNPDF30_Vector_Mphi-500_Mchi-100_gSM-0p25_gDM-1p0_13TeV-powheg/RunIISpring15DR74-Asympt25ns_MCRUN2_74_V9-v1/MINIAODSIM! \tabularnewline
\verb!/DMV_NNPDF30_Vector_Mphi-500_Mchi-100_gSM-1p0_gDM-1p0_13TeV-powheg/RunIISpring15DR74-Asympt25ns_MCRUN2_74_V9-v1/MINIAODSIM! \tabularnewline
\verb!/DMV_NNPDF30_Vector_Mphi-500_Mchi-150_gSM-1p0_gDM-1p0_13TeV-powheg/RunIISpring15DR74-Asympt25ns_MCRUN2_74_V9-v1/MINIAODSIM! \tabularnewline
\verb!/DMV_NNPDF30_Vector_Mphi-500_Mchi-500_gSM-1p0_gDM-1p0_13TeV-powheg/RunIISpring15DR74-Asympt25ns_MCRUN2_74_V9-v1/MINIAODSIM! \tabularnewline
\verb!/DMV_NNPDF30_Vector_Mphi-1000_Mchi-50_gSM-0p25_gDM-1p0_13TeV-powheg/RunIISpring15DR74-Asympt25ns_MCRUN2_74_V9-v1/MINIAODSIM! \tabularnewline
\verb!/DMV_NNPDF30_Vector_Mphi-1000_Mchi-100_gSM-1p0_gDM-1p0_13TeV-powheg/RunIISpring15DR74-Asympt25ns_MCRUN2_74_V9-v1/MINIAODSIM! \tabularnewline
\verb!/DMV_NNPDF30_Vector_Mphi-1000_Mchi-150_gSM-1p0_gDM-1p0_13TeV-powheg/RunIISpring15DR74-Asympt25ns_MCRUN2_74_V9-v1/MINIAODSIM! \tabularnewline
\verb!/DMV_NNPDF30_Vector_Mphi-1000_Mchi-500_gSM-0p25_gDM-1p0_13TeV-powheg/RunIISpring15DR74-Asympt25ns_MCRUN2_74_V9-v1/MINIAODSIM! \tabularnewline
\verb!/DMV_NNPDF30_Vector_Mphi-1000_Mchi-1000_gSM-0p25_gDM-1p0_13TeV-powheg/RunIISpring15DR74-Asympt25ns_MCRUN2_74_V9-v1/MINIAODSIM! \tabularnewline
\verb!/DMV_NNPDF30_Vector_Mphi-1000_Mchi-1000_gSM-1p0_gDM-1p0_13TeV-powheg/RunIISpring15DR74-Asympt25ns_MCRUN2_74_V9-v1/MINIAODSIM! \tabularnewline
\verb!/DMV_NNPDF30_Vector_Mphi-2000_Mchi-1_gSM-0p25_gDM-1p0_13TeV-powheg/RunIISpring15DR74-Asympt25ns_MCRUN2_74_V9-v1/MINIAODSIM! \tabularnewline
\verb!/DMV_NNPDF30_Vector_Mphi-2000_Mchi-10_gSM-0p25_gDM-1p0_13TeV-powheg/RunIISpring15DR74-Asympt25ns_MCRUN2_74_V9-v1/MINIAODSIM! \tabularnewline
\verb!/DMV_NNPDF30_Vector_Mphi-2000_Mchi-100_gSM-1p0_gDM-1p0_13TeV-powheg/RunIISpring15DR74-Asympt25ns_MCRUN2_74_V9-v1/MINIAODSIM! \tabularnewline
\verb!/DMV_NNPDF30_Vector_Mphi-10000_Mchi-50_gSM-0p25_gDM-1p0_13TeV-powheg/RunIISpring15DR74-Asympt25ns_MCRUN2_74_V9-v1/MINIAODSIM! \tabularnewline
\verb!/DMV_NNPDF30_Vector_Mphi-10000_Mchi-50_gSM-1p0_gDM-1p0_13TeV-powheg/RunIISpring15DR74-Asympt25ns_MCRUN2_74_V9-v1/MINIAODSIM! \tabularnewline
\verb!/DMV_NNPDF30_Vector_Mphi-10000_Mchi-100_gSM-1p0_gDM-1p0_13TeV-powheg/RunIISpring15DR74-Asympt25ns_MCRUN2_74_V9-v1/MINIAODSIM! \tabularnewline
\verb!/DMV_NNPDF30_Vector_Mphi-10000_Mchi-150_gSM-1p0_gDM-1p0_13TeV-powheg/RunIISpring15DR74-Asympt25ns_MCRUN2_74_V9-v1/MINIAODSIM! \tabularnewline
\verb!/DMV_NNPDF30_Vector_Mphi-10000_Mchi-500_gSM-1p0_gDM-1p0_13TeV-powheg/RunIISpring15DR74-Asympt25ns_MCRUN2_74_V9-v1/MINIAODSIM! \tabularnewline
\hline
\end{tabular}\end{center}
}
\label{datasets_dm_vector}
\end{table}

\begin{table}[!p]
 \centering
\topcaption{Simulated signal samples: DM Axial}
 \tiny
 \scalebox{.7}[1.0]{%latex.default(d, title = NULL, booktabs = FALSE, width = 3, rowname = NULL,     helvetica = FALSE, caption.loc = "bottom", ...)%
\begin{center}
\begin{tabular}{l}
\hline\hline
\multicolumn{1}{c}{Data set}\tabularnewline
\hline
\verb!/DMV_NNPDF30_Axial_Mchi-10_Mchi-1_gSM-0p25_gDM-1p0_13TeV-powheg/RunIISpring15DR74-Asympt25ns_MCRUN2_74_V9-v1/MINIAODSIM! \tabularnewline
\verb!/DMV_NNPDF30_Axial_Mphi-10_Mchi-1_gSM-1p0_gDM-1p0_13TeV-powheg/RunIISpring15DR74-Asympt25ns_MCRUN2_74_V9-v1/MINIAODSIM! \tabularnewline
\verb!/DMV_NNPDF30_Axial_Mchi-10_Mchi-10_gSM-0p25_gDM-1p0_13TeV-powheg/RunIISpring15DR74-Asympt25ns_MCRUN2_74_V9-v1/MINIAODSIM! \tabularnewline
\verb!/DMV_NNPDF30_Axial_Mphi-10_Mchi-10_gSM-1p0_gDM-1p0_13TeV-powheg/RunIISpring15DR74-Asympt25ns_MCRUN2_74_V9-v1/MINIAODSIM! \tabularnewline
\verb!/DMV_NNPDF30_Axial_Mchi-10_Mchi-50_gSM-0p25_gDM-1p0_13TeV-powheg/RunIISpring15DR74-Asympt25ns_MCRUN2_74_V9-v1/MINIAODSIM! \tabularnewline
\verb!/DMV_NNPDF30_Axial_Mchi-10_Mchi-100_gSM-0p25_gDM-1p0_13TeV-powheg/RunIISpring15DR74-Asympt25ns_MCRUN2_74_V9-v1/MINIAODSIM! \tabularnewline
\verb!/DMV_NNPDF30_Axial_Mchi-10_Mchi-500_gSM-0p25_gDM-1p0_13TeV-powheg/RunIISpring15DR74-Asympt25ns_MCRUN2_74_V9-v1/MINIAODSIM! \tabularnewline
\verb!/DMV_NNPDF30_Axial_Mphi-20_Mchi-1_gSM-1p0_gDM-1p0_13TeV-powheg/RunIISpring15DR74-Asympt25ns_MCRUN2_74_V9-v1/MINIAODSIM! \tabularnewline
\verb!/DMV_NNPDF30_Axial_Mchi-20_Mchi-10_gSM-0p25_gDM-1p0_13TeV-powheg/RunIISpring15DR74-Asympt25ns_MCRUN2_74_V9-v1/MINIAODSIM! \tabularnewline
\verb!/DMV_NNPDF30_Axial_Mphi-20_Mchi-10_gSM-1p0_gDM-1p0_13TeV-powheg/RunIISpring15DR74-Asympt25ns_MCRUN2_74_V9-v1/MINIAODSIM! \tabularnewline
\verb!/DMV_NNPDF30_Axial_Mchi-50_Mchi-1_gSM-0p25_gDM-1p0_13TeV-powheg/RunIISpring15DR74-Asympt25ns_MCRUN2_74_V9-v1/MINIAODSIM! \tabularnewline
\verb!/DMV_NNPDF30_Axial_Mchi-50_Mchi-10_gSM-0p25_gDM-1p0_13TeV-powheg/RunIISpring15DR74-Asympt25ns_MCRUN2_74_V9-v1/MINIAODSIM! \tabularnewline
\verb!/DMV_NNPDF30_Axial_Mphi-50_Mchi-10_gSM-1p0_gDM-1p0_13TeV-powheg/RunIISpring15DR74-Asympt25ns_MCRUN2_74_V9-v1/MINIAODSIM! \tabularnewline
\verb!/DMV_NNPDF30_Axial_Mphi-50_Mchi-50_gSM-1p0_gDM-1p0_13TeV-powheg/RunIISpring15DR74-Asympt25ns_MCRUN2_74_V9-v1/MINIAODSIM! \tabularnewline
\verb!/DMV_NNPDF30_Axial_Mphi-100_Mchi-1_gSM-0p25_gDM-1p0_13TeV-powheg/RunIISpring15DR74-Asympt25ns_MCRUN2_74_V9-v1/MINIAODSIM! \tabularnewline
\verb!/DMV_NNPDF30_Axial_Mphi-100_Mchi-1_gSM-1p0_gDM-1p0_13TeV-powheg/RunIISpring15DR74-Asympt25ns_MCRUN2_74_V9-v1/MINIAODSIM! \tabularnewline
\verb!/DMV_NNPDF30_Axial_Mphi-100_Mchi-10_gSM-0p25_gDM-1p0_13TeV-powheg/RunIISpring15DR74-Asympt25ns_MCRUN2_74_V9-v1/MINIAODSIM! \tabularnewline
\verb!/DMV_NNPDF30_Axial_Mphi-100_Mchi-100_gSM-0p25_gDM-1p0_13TeV-powheg/RunIISpring15DR74-Asympt25ns_MCRUN2_74_V9-v1/MINIAODSIM! \tabularnewline
\verb!/DMV_NNPDF30_Axial_Mphi-100_Mchi-100_gSM-1p0_gDM-1p0_13TeV-powheg/RunIISpring15DR74-Asympt25ns_MCRUN2_74_V9-v1/MINIAODSIM! \tabularnewline
\verb!/DMV_NNPDF30_Axial_Mphi-200_Mchi-1_gSM-0p25_gDM-1p0_13TeV-powheg/RunIISpring15DR74-Asympt25ns_MCRUN2_74_V9-v1/MINIAODSIM! \tabularnewline
\verb!/DMV_NNPDF30_Axial_Mphi-200_Mchi-1_gSM-1p0_gDM-1p0_13TeV-powheg/RunIISpring15DR74-Asympt25ns_MCRUN2_74_V9-v1/MINIAODSIM! \tabularnewline
\verb!/DMV_NNPDF30_Axial_Mphi-200_Mchi-10_gSM-0p25_gDM-1p0_13TeV-powheg/RunIISpring15DR74-Asympt25ns_MCRUN2_74_V9-v1/MINIAODSIM! \tabularnewline
\verb!/DMV_NNPDF30_Axial_Mphi-200_Mchi-10_gSM-1p0_gDM-1p0_13TeV-powheg/RunIISpring15DR74-Asympt25ns_MCRUN2_74_V9-v1/MINIAODSIM! \tabularnewline
\verb!/DMV_NNPDF30_Axial_Mphi-200_Mchi-50_gSM-1p0_gDM-1p0_13TeV-powheg/RunIISpring15DR74-Asympt25ns_MCRUN2_74_V9-v1/MINIAODSIM! \tabularnewline
\verb!/DMV_NNPDF30_Axial_Mphi-200_Mchi-150_gSM-0p25_gDM-1p0_13TeV-powheg/RunIISpring15DR74-Asympt25ns_MCRUN2_74_V9-v1/MINIAODSIM! \tabularnewline
\verb!/DMV_NNPDF30_Axial_Mphi-300_Mchi-1_gSM-0p25_gDM-1p0_13TeV-powheg/RunIISpring15DR74-Asympt25ns_MCRUN2_74_V9-v1/MINIAODSIM! \tabularnewline
\verb!/DMV_NNPDF30_Axial_Mphi-300_Mchi-1_gSM-1p0_gDM-1p0_13TeV-powheg/RunIISpring15DR74-Asympt25ns_MCRUN2_74_V9-v1/MINIAODSIM! \tabularnewline
\verb!/DMV_NNPDF30_Axial_Mphi-300_Mchi-10_gSM-1p0_gDM-1p0_13TeV-powheg/RunIISpring15DR74-Asympt25ns_MCRUN2_74_V9-v1/MINIAODSIM! \tabularnewline
\verb!/DMV_NNPDF30_Axial_Mphi-300_Mchi-50_gSM-1p0_gDM-1p0_13TeV-powheg/RunIISpring15DR74-Asympt25ns_MCRUN2_74_V9-v1/MINIAODSIM! \tabularnewline
\verb!/DMV_NNPDF30_Axial_Mphi-300_Mchi-100_gSM-1p0_gDM-1p0_13TeV-powheg/RunIISpring15DR74-Asympt25ns_MCRUN2_74_V9-v1/MINIAODSIM! \tabularnewline
\verb!/DMV_NNPDF30_Axial_Mphi-300_Mchi-150_gSM-0p25_gDM-1p0_13TeV-powheg/RunIISpring15DR74-Asympt25ns_MCRUN2_74_V9-v1/MINIAODSIM! \tabularnewline
\verb!/DMV_NNPDF30_Axial_Mphi-300_Mchi-150_gSM-1p0_gDM-1p0_13TeV-powheg/RunIISpring15DR74-Asympt25ns_MCRUN2_74_V9-v1/MINIAODSIM! \tabularnewline
\verb!/DMV_NNPDF30_Axial_Mphi-500_Mchi-1_gSM-0p25_gDM-1p0_13TeV-powheg/RunIISpring15DR74-Asympt25ns_MCRUN2_74_V9-v1/MINIAODSIM! \tabularnewline
\verb!/DMV_NNPDF30_Axial_Mphi-500_Mchi-1_gSM-1p0_gDM-1p0_13TeV-powheg/RunIISpring15DR74-Asympt25ns_MCRUN2_74_V9-v1/MINIAODSIM! \tabularnewline
\verb!/DMV_NNPDF30_Axial_Mphi-500_Mchi-10_gSM-1p0_gDM-1p0_13TeV-powheg/RunIISpring15DR74-Asympt25ns_MCRUN2_74_V9-v1/MINIAODSIM! \tabularnewline
\verb!/DMV_NNPDF30_Axial_Mphi-500_Mchi-50_gSM-0p25_gDM-1p0_13TeV-powheg/RunIISpring15DR74-Asympt25ns_MCRUN2_74_V9-v1/MINIAODSIM! \tabularnewline
\verb!/DMV_NNPDF30_Axial_Mphi-500_Mchi-50_gSM-1p0_gDM-1p0_13TeV-powheg/RunIISpring15DR74-Asympt25ns_MCRUN2_74_V9-v1/MINIAODSIM! \tabularnewline
\verb!/DMV_NNPDF30_Axial_Mphi-500_Mchi-100_gSM-0p25_gDM-1p0_13TeV-powheg/RunIISpring15DR74-Asympt25ns_MCRUN2_74_V9-v1/MINIAODSIM! \tabularnewline
\verb!/DMV_NNPDF30_Axial_Mphi-500_Mchi-100_gSM-1p0_gDM-1p0_13TeV-powheg/RunIISpring15DR74-Asympt25ns_MCRUN2_74_V9-v1/MINIAODSIM! \tabularnewline
\verb!/DMV_NNPDF30_Axial_Mphi-500_Mchi-150_gSM-1p0_gDM-1p0_13TeV-powheg/RunIISpring15DR74-Asympt25ns_MCRUN2_74_V9-v1/MINIAODSIM! \tabularnewline
\verb!/DMV_NNPDF30_Axial_Mphi-500_Mchi-500_gSM-0p25_gDM-1p0_13TeV-powheg/RunIISpring15DR74-Asympt25ns_MCRUN2_74_V9-v1/MINIAODSIM! \tabularnewline
\verb!/DMV_NNPDF30_Axial_Mphi-500_Mchi-500_gSM-1p0_gDM-1p0_13TeV-powheg/RunIISpring15DR74-Asympt25ns_MCRUN2_74_V9-v1/MINIAODSIM! \tabularnewline
\verb!/DMV_NNPDF30_Axial_Mphi-1000_Mchi-1_gSM-1p0_gDM-1p0_13TeV-powheg/RunIISpring15DR74-Asympt25ns_MCRUN2_74_V9-v1/MINIAODSIM! \tabularnewline
\verb!/DMV_NNPDF30_Axial_Mphi-1000_Mchi-10_gSM-1p0_gDM-1p0_13TeV-powheg/RunIISpring15DR74-Asympt25ns_MCRUN2_74_V9-v1/MINIAODSIM! \tabularnewline
\verb!/DMV_NNPDF30_Axial_Mphi-1000_Mchi-50_gSM-0p25_gDM-1p0_13TeV-powheg/RunIISpring15DR74-Asympt25ns_MCRUN2_74_V9-v1/MINIAODSIM! \tabularnewline
\verb!/DMV_NNPDF30_Axial_Mphi-1000_Mchi-50_gSM-1p0_gDM-1p0_13TeV-powheg/RunIISpring15DR74-Asympt25ns_MCRUN2_74_V9-v1/MINIAODSIM! \tabularnewline
\verb!/DMV_NNPDF30_Axial_Mphi-1000_Mchi-150_gSM-0p25_gDM-1p0_13TeV-powheg/RunIISpring15DR74-Asympt25ns_MCRUN2_74_V9-v1/MINIAODSIM! \tabularnewline
\verb!/DMV_NNPDF30_Axial_Mphi-1000_Mchi-500_gSM-0p25_gDM-1p0_13TeV-powheg/RunIISpring15DR74-Asympt25ns_MCRUN2_74_V9-v1/MINIAODSIM! \tabularnewline
\verb!/DMV_NNPDF30_Axial_Mphi-2000_Mchi-1_gSM-0p25_gDM-1p0_13TeV-powheg/RunIISpring15DR74-Asympt25ns_MCRUN2_74_V9-v1/MINIAODSIM! \tabularnewline
\verb!/DMV_NNPDF30_Axial_Mphi-2000_Mchi-1_gSM-1p0_gDM-1p0_13TeV-powheg/RunIISpring15DR74-Asympt25ns_MCRUN2_74_V9-v1/MINIAODSIM! \tabularnewline
\verb!/DMV_NNPDF30_Axial_Mphi-2000_Mchi-50_gSM-0p25_gDM-1p0_13TeV-powheg/RunIISpring15DR74-Asympt25ns_MCRUN2_74_V9-v1/MINIAODSIM! \tabularnewline
\verb!/DMV_NNPDF30_Axial_Mphi-2000_Mchi-100_gSM-1p0_gDM-1p0_13TeV-powheg/RunIISpring15DR74-Asympt25ns_MCRUN2_74_V9-v1/MINIAODSIM! \tabularnewline
\verb!/DMV_NNPDF30_Axial_Mphi-2000_Mchi-150_gSM-1p0_gDM-1p0_13TeV-powheg/RunIISpring15DR74-Asympt25ns_MCRUN2_74_V9-v1/MINIAODSIM! \tabularnewline
\verb!/DMV_NNPDF30_Axial_Mphi-5000_Mchi-1_gSM-1p0_gDM-1p0_13TeV-powheg/RunIISpring15DR74-Asympt25ns_MCRUN2_74_V9-v1/MINIAODSIM! \tabularnewline
\verb!/DMV_NNPDF30_Axial_Mphi-5000_Mchi-10_gSM-1p0_gDM-1p0_13TeV-powheg/RunIISpring15DR74-Asympt25ns_MCRUN2_74_V9-v1/MINIAODSIM! \tabularnewline
\verb!/DMV_NNPDF30_Axial_Mphi-5000_Mchi-50_gSM-1p0_gDM-1p0_13TeV-powheg/RunIISpring15DR74-Asympt25ns_MCRUN2_74_V9-v1/MINIAODSIM! \tabularnewline
\verb!/DMV_NNPDF30_Axial_Mphi-5000_Mchi-100_gSM-1p0_gDM-1p0_13TeV-powheg/RunIISpring15DR74-Asympt25ns_MCRUN2_74_V9-v1/MINIAODSIM! \tabularnewline
\verb!/DMV_NNPDF30_Axial_Mphi-5000_Mchi-150_gSM-1p0_gDM-1p0_13TeV-powheg/RunIISpring15DR74-Asympt25ns_MCRUN2_74_V9-v1/MINIAODSIM! \tabularnewline
\verb!/DMV_NNPDF30_Axial_Mphi-5000_Mchi-500_gSM-1p0_gDM-1p0_13TeV-powheg/RunIISpring15DR74-Asympt25ns_MCRUN2_74_V9-v1/MINIAODSIM! \tabularnewline
\verb!/DMV_NNPDF30_Axial_Mphi-10000_Mchi-1_gSM-0p25_gDM-1p0_13TeV-powheg/RunIISpring15DR74-Asympt25ns_MCRUN2_74_V9-v1/MINIAODSIM! \tabularnewline
\verb!/DMV_NNPDF30_Axial_Mphi-10000_Mchi-1_gSM-1p0_gDM-1p0_13TeV-powheg/RunIISpring15DR74-Asympt25ns_MCRUN2_74_V9-v1/MINIAODSIM! \tabularnewline
\verb!/DMV_NNPDF30_Axial_Mphi-10000_Mchi-10_gSM-1p0_gDM-1p0_13TeV-powheg/RunIISpring15DR74-Asympt25ns_MCRUN2_74_V9-v1/MINIAODSIM! \tabularnewline
\verb!/DMV_NNPDF30_Axial_Mphi-10000_Mchi-100_gSM-0p25_gDM-1p0_13TeV-powheg/RunIISpring15DR74-Asympt25ns_MCRUN2_74_V9-v1/MINIAODSIM! \tabularnewline
\verb!/DMV_NNPDF30_Axial_Mphi-10000_Mchi-150_gSM-0p25_gDM-1p0_13TeV-powheg/RunIISpring15DR74-Asympt25ns_MCRUN2_74_V9-v1/MINIAODSIM! \tabularnewline
\hline
\end{tabular}\end{center}
}
\label{datasets_dm_axial}
\end{table}

\begin{table}[!p]
 \centering
\topcaption{Simulated signal samples: DM Scalar}
 \scriptsize
 \scalebox{.7}[1.0]{%latex.default(d, title = NULL, booktabs = FALSE, width = 3, rowname = NULL,     helvetica = FALSE, caption.loc = "bottom", ...)%
\begin{center}
\begin{tabular}{l}
\hline\hline
\multicolumn{1}{c}{Data set}\tabularnewline
\hline
\verb!/DMS_NNPDF30_Scalar_Mphi-10_Mchi-1_gSM-1p0_gDM-1p0_13TeV-powheg/RunIISpring15DR74-Asympt25ns_MCRUN2_74_V9-v1/MINIAODSIM! \tabularnewline
\verb!/DMS_NNPDF30_Scalar_Mphi-10_Mchi-10_gSM-1p0_gDM-1p0_13TeV-powheg/RunIISpring15DR74-Asympt25ns_MCRUN2_74_V9-v1/MINIAODSIM! \tabularnewline
\verb!/DMS_NNPDF30_Scalar_Mphi-20_Mchi-1_gSM-1p0_gDM-1p0_13TeV-powheg/RunIISpring15DR74-Asympt25ns_MCRUN2_74_V9-v1/MINIAODSIM! \tabularnewline
\verb!/DMS_NNPDF30_Scalar_Mphi-20_Mchi-10_gSM-1p0_gDM-1p0_13TeV-powheg/RunIISpring15DR74-Asympt25ns_MCRUN2_74_V9-v1/MINIAODSIM! \tabularnewline
\verb!/DMS_NNPDF30_Scalar_Mphi-50_Mchi-10_gSM-1p0_gDM-1p0_13TeV-powheg/RunIISpring15DR74-Asympt25ns_MCRUN2_74_V9-v1/MINIAODSIM! \tabularnewline
\verb!/DMS_NNPDF30_Scalar_Mphi-100_Mchi-1_gSM-1p0_gDM-1p0_13TeV-powheg/RunIISpring15DR74-Asympt25ns_MCRUN2_74_V9-v1/MINIAODSIM! \tabularnewline
\verb!/DMS_NNPDF30_Scalar_Mphi-100_Mchi-10_gSM-1p0_gDM-1p0_13TeV-powheg/RunIISpring15DR74-Asympt25ns_MCRUN2_74_V9-v1/MINIAODSIM! \tabularnewline
\verb!/DMS_NNPDF30_Scalar_Mphi-100_Mchi-50_gSM-1p0_gDM-1p0_13TeV-powheg/RunIISpring15DR74-Asympt25ns_MCRUN2_74_V9-v1/MINIAODSIM! \tabularnewline
\verb!/DMS_NNPDF30_Scalar_Mphi-100_Mchi-100_gSM-1p0_gDM-1p0_13TeV-powheg/RunIISpring15DR74-Asympt25ns_MCRUN2_74_V9-v1/MINIAODSIM! \tabularnewline
\verb!/DMS_NNPDF30_Scalar_Mphi-200_Mchi-1_gSM-1p0_gDM-1p0_13TeV-powheg/RunIISpring15DR74-Asympt25ns_MCRUN2_74_V9-v1/MINIAODSIM! \tabularnewline
\verb!/DMS_NNPDF30_Scalar_Mphi-200_Mchi-10_gSM-1p0_gDM-1p0_13TeV-powheg/RunIISpring15DR74-Asympt25ns_MCRUN2_74_V9-v1/MINIAODSIM! \tabularnewline
\verb!/DMS_NNPDF30_Scalar_Mphi-200_Mchi-50_gSM-1p0_gDM-1p0_13TeV-powheg/RunIISpring15DR74-Asympt25ns_MCRUN2_74_V9-v1/MINIAODSIM! \tabularnewline
\verb!/DMS_NNPDF30_Scalar_Mphi-200_Mchi-100_gSM-1p0_gDM-1p0_13TeV-powheg/RunIISpring15DR74-Asympt25ns_MCRUN2_74_V9-v1/MINIAODSIM! \tabularnewline
\verb!/DMS_NNPDF30_Scalar_Mphi-200_Mchi-150_gSM-1p0_gDM-1p0_13TeV-powheg/RunIISpring15DR74-Asympt25ns_MCRUN2_74_V9-v1/MINIAODSIM! \tabularnewline
\verb!/DMS_NNPDF30_Scalar_Mphi-300_Mchi-1_gSM-1p0_gDM-1p0_13TeV-powheg/RunIISpring15DR74-Asympt25ns_MCRUN2_74_V9-v1/MINIAODSIM! \tabularnewline
\verb!/DMS_NNPDF30_Scalar_Mphi-300_Mchi-10_gSM-1p0_gDM-1p0_13TeV-powheg/RunIISpring15DR74-Asympt25ns_MCRUN2_74_V9-v1/MINIAODSIM! \tabularnewline
\verb!/DMS_NNPDF30_Scalar_Mphi-300_Mchi-100_gSM-1p0_gDM-1p0_13TeV-powheg/RunIISpring15DR74-Asympt25ns_MCRUN2_74_V9-v1/MINIAODSIM! \tabularnewline
\verb!/DMS_NNPDF30_Scalar_Mphi-300_Mchi-150_gSM-1p0_gDM-1p0_13TeV-powheg/RunIISpring15DR74-Asympt25ns_MCRUN2_74_V9-v1/MINIAODSIM! \tabularnewline
\verb!/DMS_NNPDF30_Scalar_Mphi-500_Mchi-1_gSM-1p0_gDM-1p0_13TeV-powheg/RunIISpring15DR74-Asympt25ns_MCRUN2_74_V9-v1/MINIAODSIM! \tabularnewline
\verb!/DMS_NNPDF30_Scalar_Mphi-500_Mchi-10_gSM-1p0_gDM-1p0_13TeV-powheg/RunIISpring15DR74-Asympt25ns_MCRUN2_74_V9-v1/MINIAODSIM! \tabularnewline
\verb!/DMS_NNPDF30_Scalar_Mphi-500_Mchi-50_gSM-1p0_gDM-1p0_13TeV-powheg/RunIISpring15DR74-Asympt25ns_MCRUN2_74_V9-v1/MINIAODSIM! \tabularnewline
\verb!/DMS_NNPDF30_Scalar_Mphi-500_Mchi-150_gSM-1p0_gDM-1p0_13TeV-powheg/RunIISpring15DR74-Asympt25ns_MCRUN2_74_V9-v1/MINIAODSIM! \tabularnewline
\verb!/DMS_NNPDF30_Scalar_Mphi-500_Mchi-500_gSM-1p0_gDM-1p0_13TeV-powheg/RunIISpring15DR74-Asympt25ns_MCRUN2_74_V9-v1/MINIAODSIM! \tabularnewline
\verb!/DMS_NNPDF30_Scalar_Mphi-1000_Mchi-1_gSM-1p0_gDM-1p0_13TeV-powheg/RunIISpring15DR74-Asympt25ns_MCRUN2_74_V9-v1/MINIAODSIM! \tabularnewline
\verb!/DMS_NNPDF30_Scalar_Mphi-1000_Mchi-10_gSM-1p0_gDM-1p0_13TeV-powheg/RunIISpring15DR74-Asympt25ns_MCRUN2_74_V9-v1/MINIAODSIM! \tabularnewline
\verb!/DMS_NNPDF30_Scalar_Mphi-1000_Mchi-100_gSM-1p0_gDM-1p0_13TeV-powheg/RunIISpring15DR74-Asympt25ns_MCRUN2_74_V9-v1/MINIAODSIM! \tabularnewline
\verb!/DMS_NNPDF30_Scalar_Mphi-1000_Mchi-500_gSM-1p0_gDM-1p0_13TeV-powheg/RunIISpring15DR74-Asympt25ns_MCRUN2_74_V9-v1/MINIAODSIM! \tabularnewline
\verb!/DMS_NNPDF30_Scalar_Mphi-2000_Mchi-1_gSM-1p0_gDM-1p0_13TeV-powheg/RunIISpring15DR74-Asympt25ns_MCRUN2_74_V9-v1/MINIAODSIM! \tabularnewline
\verb!/DMS_NNPDF30_Scalar_Mphi-2000_Mchi-100_gSM-1p0_gDM-1p0_13TeV-powheg/RunIISpring15DR74-Asympt25ns_MCRUN2_74_V9-v1/MINIAODSIM! \tabularnewline
\verb!/DMS_NNPDF30_Scalar_Mphi-2000_Mchi-150_gSM-1p0_gDM-1p0_13TeV-powheg/RunIISpring15DR74-Asympt25ns_MCRUN2_74_V9-v1/MINIAODSIM! \tabularnewline
\verb!/DMS_NNPDF30_Scalar_Mphi-2000_Mchi-1000_gSM-1p0_gDM-1p0_13TeV-powheg/RunIISpring15DR74-Asympt25ns_MCRUN2_74_V9-v1/MINIAODSIM! \tabularnewline
\verb!/DMS_NNPDF30_Scalar_Mphi-5000_Mchi-10_gSM-1p0_gDM-1p0_13TeV-powheg/RunIISpring15DR74-Asympt25ns_MCRUN2_74_V9-v1/MINIAODSIM! \tabularnewline
\verb!/DMS_NNPDF30_Scalar_Mphi-5000_Mchi-50_gSM-1p0_gDM-1p0_13TeV-powheg/RunIISpring15DR74-Asympt25ns_MCRUN2_74_V9-v1/MINIAODSIM! \tabularnewline
\verb!/DMS_NNPDF30_Scalar_Mphi-5000_Mchi-100_gSM-1p0_gDM-1p0_13TeV-powheg/RunIISpring15DR74-Asympt25ns_MCRUN2_74_V9-v1/MINIAODSIM! \tabularnewline
\verb!/DMS_NNPDF30_Scalar_Mphi-5000_Mchi-150_gSM-1p0_gDM-1p0_13TeV-powheg/RunIISpring15DR74-Asympt25ns_MCRUN2_74_V9-v1/MINIAODSIM! \tabularnewline
\verb!/DMS_NNPDF30_Scalar_Mphi-5000_Mchi-500_gSM-1p0_gDM-1p0_13TeV-powheg/RunIISpring15DR74-Asympt25ns_MCRUN2_74_V9-v1/MINIAODSIM! \tabularnewline
\verb!/DMS_NNPDF30_Scalar_Mphi-5000_Mchi-1000_gSM-1p0_gDM-1p0_13TeV-powheg/RunIISpring15DR74-Asympt25ns_MCRUN2_74_V9-v1/MINIAODSIM! \tabularnewline
\verb!/DMS_NNPDF30_Scalar_Mphi-10000_Mchi-1_gSM-1p0_gDM-1p0_13TeV-powheg/RunIISpring15DR74-Asympt25ns_MCRUN2_74_V9-v1/MINIAODSIM! \tabularnewline
\verb!/DMS_NNPDF30_Scalar_Mphi-10000_Mchi-10_gSM-1p0_gDM-1p0_13TeV-powheg/RunIISpring15DR74-Asympt25ns_MCRUN2_74_V9-v1/MINIAODSIM! \tabularnewline
\verb!/DMS_NNPDF30_Scalar_Mphi-10000_Mchi-500_gSM-1p0_gDM-1p0_13TeV-powheg/RunIISpring15DR74-Asympt25ns_MCRUN2_74_V9-v1/MINIAODSIM! \tabularnewline
\verb!/DMS_NNPDF30_Scalar_Mphi-10000_Mchi-1000_gSM-1p0_gDM-1p0_13TeV-powheg/RunIISpring15DR74-Asympt25ns_MCRUN2_74_V9-v1/MINIAODSIM! \tabularnewline
\hline
\end{tabular}\end{center}
}
\label{datasets_dm_scalarw}
\end{table}

\begin{table}[!p]
 \centering
\topcaption{Simulated signal samples: DM Pseudoscalar}
 \scriptsize
 \scalebox{.7}[1.0]{%latex.default(d, title = NULL, booktabs = FALSE, width = 3, rowname = NULL,     helvetica = FALSE, caption.loc = "bottom", ...)%
\begin{center}
\begin{tabular}{l}
\hline\hline
\multicolumn{1}{c}{Data set}\tabularnewline
\hline
\verb!/DMS_NNPDF30_Pseudoscalar_Mphi-10_Mchi-1_gSM-1p0_gDM-1p0_13TeV-powheg/RunIISpring15DR74-Asympt25ns_MCRUN2_74_V9-v1/MINIAODSIM! \tabularnewline
\verb!/DMS_NNPDF30_Pseudoscalar_Mphi-10_Mchi-10_gSM-1p0_gDM-1p0_13TeV-powheg/RunIISpring15DR74-Asympt25ns_MCRUN2_74_V9-v1/MINIAODSIM! \tabularnewline
\verb!/DMS_NNPDF30_Pseudoscalar_Mphi-20_Mchi-1_gSM-1p0_gDM-1p0_13TeV-powheg/RunIISpring15DR74-Asympt25ns_MCRUN2_74_V9-v1/MINIAODSIM! \tabularnewline
\verb!/DMS_NNPDF30_Pseudoscalar_Mphi-20_Mchi-10_gSM-1p0_gDM-1p0_13TeV-powheg/RunIISpring15DR74-Asympt25ns_MCRUN2_74_V9-v1/MINIAODSIM! \tabularnewline
\verb!/DMS_NNPDF30_Pseudoscalar_Mphi-50_Mchi-10_gSM-1p0_gDM-1p0_13TeV-powheg/RunIISpring15DR74-Asympt25ns_MCRUN2_74_V9-v1/MINIAODSIM! \tabularnewline
\verb!/DMS_NNPDF30_Pseudoscalar_Mphi-50_Mchi-50_gSM-1p0_gDM-1p0_13TeV-powheg/RunIISpring15DR74-Asympt25ns_MCRUN2_74_V9-v1/MINIAODSIM! \tabularnewline
\verb!/DMS_NNPDF30_Pseudoscalar_Mphi-100_Mchi-10_gSM-1p0_gDM-1p0_13TeV-powheg/RunIISpring15DR74-Asympt25ns_MCRUN2_74_V9-v1/MINIAODSIM! \tabularnewline
\verb!/DMS_NNPDF30_Pseudoscalar_Mphi-100_Mchi-100_gSM-1p0_gDM-1p0_13TeV-powheg/RunIISpring15DR74-Asympt25ns_MCRUN2_74_V9-v1/MINIAODSIM! \tabularnewline
\verb!/DMS_NNPDF30_Pseudoscalar_Mphi-200_Mchi-1_gSM-1p0_gDM-1p0_13TeV-powheg/RunIISpring15DR74-Asympt25ns_MCRUN2_74_V9-v1/MINIAODSIM! \tabularnewline
\verb!/DMS_NNPDF30_Pseudoscalar_Mphi-200_Mchi-10_gSM-1p0_gDM-1p0_13TeV-powheg/RunIISpring15DR74-Asympt25ns_MCRUN2_74_V9-v1/MINIAODSIM! \tabularnewline
\verb!/DMS_NNPDF30_Pseudoscalar_Mphi-200_Mchi-100_gSM-1p0_gDM-1p0_13TeV-powheg/RunIISpring15DR74-Asympt25ns_MCRUN2_74_V9-v1/MINIAODSIM! \tabularnewline
\verb!/DMS_NNPDF30_Pseudoscalar_Mphi-200_Mchi-150_gSM-1p0_gDM-1p0_13TeV-powheg/RunIISpring15DR74-Asympt25ns_MCRUN2_74_V9-v1/MINIAODSIM! \tabularnewline
\verb!/DMS_NNPDF30_Pseudoscalar_Mphi-300_Mchi-10_gSM-1p0_gDM-1p0_13TeV-powheg/RunIISpring15DR74-Asympt25ns_MCRUN2_74_V9-v1/MINIAODSIM! \tabularnewline
\verb!/DMS_NNPDF30_Pseudoscalar_Mphi-300_Mchi-50_gSM-1p0_gDM-1p0_13TeV-powheg/RunIISpring15DR74-Asympt25ns_MCRUN2_74_V9-v1/MINIAODSIM! \tabularnewline
\verb!/DMS_NNPDF30_Pseudoscalar_Mphi-300_Mchi-100_gSM-1p0_gDM-1p0_13TeV-powheg/RunIISpring15DR74-Asympt25ns_MCRUN2_74_V9-v1/MINIAODSIM! \tabularnewline
\verb!/DMS_NNPDF30_Pseudoscalar_Mphi-300_Mchi-150_gSM-1p0_gDM-1p0_13TeV-powheg/RunIISpring15DR74-Asympt25ns_MCRUN2_74_V9-v1/MINIAODSIM! \tabularnewline
\verb!/DMS_NNPDF30_Pseudoscalar_Mphi-500_Mchi-1_gSM-1p0_gDM-1p0_13TeV-powheg/RunIISpring15DR74-Asympt25ns_MCRUN2_74_V9-v1/MINIAODSIM! \tabularnewline
\verb!/DMS_NNPDF30_Pseudoscalar_Mphi-500_Mchi-50_gSM-1p0_gDM-1p0_13TeV-powheg/RunIISpring15DR74-Asympt25ns_MCRUN2_74_V9-v1/MINIAODSIM! \tabularnewline
\verb!/DMS_NNPDF30_Pseudoscalar_Mphi-500_Mchi-100_gSM-1p0_gDM-1p0_13TeV-powheg/RunIISpring15DR74-Asympt25ns_MCRUN2_74_V9-v1/MINIAODSIM! \tabularnewline
\verb!/DMS_NNPDF30_Pseudoscalar_Mphi-500_Mchi-150_gSM-1p0_gDM-1p0_13TeV-powheg/RunIISpring15DR74-Asympt25ns_MCRUN2_74_V9-v1/MINIAODSIM! \tabularnewline
\verb!/DMS_NNPDF30_Pseudoscalar_Mphi-500_Mchi-500_gSM-1p0_gDM-1p0_13TeV-powheg/RunIISpring15DR74-Asympt25ns_MCRUN2_74_V9-v1/MINIAODSIM! \tabularnewline
\verb!/DMS_NNPDF30_Pseudoscalar_Mphi-1000_Mchi-1_gSM-1p0_gDM-1p0_13TeV-powheg/RunIISpring15DR74-Asympt25ns_MCRUN2_74_V9-v1/MINIAODSIM! \tabularnewline
\verb!/DMS_NNPDF30_Pseudoscalar_Mphi-1000_Mchi-10_gSM-1p0_gDM-1p0_13TeV-powheg/RunIISpring15DR74-Asympt25ns_MCRUN2_74_V9-v1/MINIAODSIM! \tabularnewline
\verb!/DMS_NNPDF30_Pseudoscalar_Mphi-1000_Mchi-100_gSM-1p0_gDM-1p0_13TeV-powheg/RunIISpring15DR74-Asympt25ns_MCRUN2_74_V9-v1/MINIAODSIM! \tabularnewline
\verb!/DMS_NNPDF30_Pseudoscalar_Mphi-1000_Mchi-150_gSM-1p0_gDM-1p0_13TeV-powheg/RunIISpring15DR74-Asympt25ns_MCRUN2_74_V9-v1/MINIAODSIM! \tabularnewline
\verb!/DMS_NNPDF30_Pseudoscalar_Mphi-1000_Mchi-1000_gSM-1p0_gDM-1p0_13TeV-powheg/RunIISpring15DR74-Asympt25ns_MCRUN2_74_V9-v1/MINIAODSIM! \tabularnewline
\verb!/DMS_NNPDF30_Pseudoscalar_Mphi-2000_Mchi-1_gSM-1p0_gDM-1p0_13TeV-powheg/RunIISpring15DR74-Asympt25ns_MCRUN2_74_V9-v1/MINIAODSIM! \tabularnewline
\verb!/DMS_NNPDF30_Pseudoscalar_Mphi-2000_Mchi-10_gSM-1p0_gDM-1p0_13TeV-powheg/RunIISpring15DR74-Asympt25ns_MCRUN2_74_V9-v1/MINIAODSIM! \tabularnewline
\verb!/DMS_NNPDF30_Pseudoscalar_Mphi-2000_Mchi-50_gSM-1p0_gDM-1p0_13TeV-powheg/RunIISpring15DR74-Asympt25ns_MCRUN2_74_V9-v1/MINIAODSIM! \tabularnewline
\verb!/DMS_NNPDF30_Pseudoscalar_Mphi-2000_Mchi-150_gSM-1p0_gDM-1p0_13TeV-powheg/RunIISpring15DR74-Asympt25ns_MCRUN2_74_V9-v1/MINIAODSIM! \tabularnewline
\verb!/DMS_NNPDF30_Pseudoscalar_Mphi-2000_Mchi-500_gSM-1p0_gDM-1p0_13TeV-powheg/RunIISpring15DR74-Asympt25ns_MCRUN2_74_V9-v1/MINIAODSIM! \tabularnewline
\verb!/DMS_NNPDF30_Pseudoscalar_Mphi-2000_Mchi-1000_gSM-1p0_gDM-1p0_13TeV-powheg/RunIISpring15DR74-Asympt25ns_MCRUN2_74_V9-v1/MINIAODSIM! \tabularnewline
\verb!/DMS_NNPDF30_Pseudoscalar_Mphi-5000_Mchi-1_gSM-1p0_gDM-1p0_13TeV-powheg/RunIISpring15DR74-Asympt25ns_MCRUN2_74_V9-v1/MINIAODSIM! \tabularnewline
\verb!/DMS_NNPDF30_Pseudoscalar_Mphi-5000_Mchi-10_gSM-1p0_gDM-1p0_13TeV-powheg/RunIISpring15DR74-Asympt25ns_MCRUN2_74_V9-v1/MINIAODSIM! \tabularnewline
\verb!/DMS_NNPDF30_Pseudoscalar_Mphi-5000_Mchi-50_gSM-1p0_gDM-1p0_13TeV-powheg/RunIISpring15DR74-Asympt25ns_MCRUN2_74_V9-v1/MINIAODSIM! \tabularnewline
\verb!/DMS_NNPDF30_Pseudoscalar_Mphi-5000_Mchi-100_gSM-1p0_gDM-1p0_13TeV-powheg/RunIISpring15DR74-Asympt25ns_MCRUN2_74_V9-v1/MINIAODSIM! \tabularnewline
\verb!/DMS_NNPDF30_Pseudoscalar_Mphi-5000_Mchi-150_gSM-1p0_gDM-1p0_13TeV-powheg/RunIISpring15DR74-Asympt25ns_MCRUN2_74_V9-v1/MINIAODSIM! \tabularnewline
\verb!/DMS_NNPDF30_Pseudoscalar_Mphi-5000_Mchi-500_gSM-1p0_gDM-1p0_13TeV-powheg/RunIISpring15DR74-Asympt25ns_MCRUN2_74_V9-v1/MINIAODSIM! \tabularnewline
\verb!/DMS_NNPDF30_Pseudoscalar_Mphi-5000_Mchi-1000_gSM-1p0_gDM-1p0_13TeV-powheg/RunIISpring15DR74-Asympt25ns_MCRUN2_74_V9-v1/MINIAODSIM! \tabularnewline
\verb!/DMS_NNPDF30_Pseudoscalar_Mphi-10000_Mchi-1_gSM-1p0_gDM-1p0_13TeV-powheg/RunIISpring15DR74-Asympt25ns_MCRUN2_74_V9-v1/MINIAODSIM! \tabularnewline
\verb!/DMS_NNPDF30_Pseudoscalar_Mphi-10000_Mchi-10_gSM-1p0_gDM-1p0_13TeV-powheg/RunIISpring15DR74-Asympt25ns_MCRUN2_74_V9-v1/MINIAODSIM! \tabularnewline
\verb!/DMS_NNPDF30_Pseudoscalar_Mphi-10000_Mchi-50_gSM-1p0_gDM-1p0_13TeV-powheg/RunIISpring15DR74-Asympt25ns_MCRUN2_74_V9-v1/MINIAODSIM! \tabularnewline
\verb!/DMS_NNPDF30_Pseudoscalar_Mphi-10000_Mchi-100_gSM-1p0_gDM-1p0_13TeV-powheg/RunIISpring15DR74-Asympt25ns_MCRUN2_74_V9-v1/MINIAODSIM! \tabularnewline
\verb!/DMS_NNPDF30_Pseudoscalar_Mphi-10000_Mchi-150_gSM-1p0_gDM-1p0_13TeV-powheg/RunIISpring15DR74-Asympt25ns_MCRUN2_74_V9-v1/MINIAODSIM! \tabularnewline
\verb!/DMS_NNPDF30_Pseudoscalar_Mphi-10000_Mchi-500_gSM-1p0_gDM-1p0_13TeV-powheg/RunIISpring15DR74-Asympt25ns_MCRUN2_74_V9-v1/MINIAODSIM! \tabularnewline
\verb!/DMS_NNPDF30_Pseudoscalar_Mphi-10000_Mchi-1000_gSM-1p0_gDM-1p0_13TeV-powheg/RunIISpring15DR74-Asympt25ns_MCRUN2_74_V9-v1/MINIAODSIM! \tabularnewline
\hline
\end{tabular}\end{center}
}
\label{datasets_dm_pseudoscalar}
\end{table}

\begin{table}[!p]
 \centering
\topcaption{Simulated signal samples: DM \bbbar Scalar}
 \scriptsize
 \scalebox{.7}[1.0]{%latex.default(d, title = NULL, booktabs = FALSE, width = 3, rowname = NULL,     helvetica = FALSE, caption.loc = "bottom", ...)%
\begin{center}
\begin{tabular}{l}
\hline\hline
\multicolumn{1}{c}{Data set}\tabularnewline
\hline
\verb!/BBbarDMJets_scalar_Mchi-1_Mphi-10_TuneCUETP8M1_13TeV-madgraphMLM-pythia8/RunIISpring15DR74-Asympt25ns_MCRUN2_74_V9-v1/MINIAODSIM! \tabularnewline
\verb!/BBbarDMJets_scalar_Mchi-1_Mphi-20_TuneCUETP8M1_13TeV-madgraphMLM-pythia8/RunIISpring15DR74-Asympt25ns_MCRUN2_74_V9-v1/MINIAODSIM! \tabularnewline
\verb!/BBbarDMJets_scalar_Mchi-1_Mphi-50_TuneCUETP8M1_13TeV-madgraphMLM-pythia8/RunIISpring15DR74-Asympt25ns_MCRUN2_74_V9-v1/MINIAODSIM! \tabularnewline
\verb!/BBbarDMJets_scalar_Mchi-1_Mphi-100_TuneCUETP8M1_13TeV-madgraphMLM-pythia8/RunIISpring15DR74-Asympt25ns_MCRUN2_74_V9-v1/MINIAODSIM! \tabularnewline
\verb!/BBbarDMJets_scalar_Mchi-1_Mphi-200_TuneCUETP8M1_13TeV-madgraphMLM-pythia8/RunIISpring15DR74-Asympt25ns_MCRUN2_74_V9-v1/MINIAODSIM! \tabularnewline
\verb!/BBbarDMJets_scalar_Mchi-1_Mphi-300_TuneCUETP8M1_13TeV-madgraphMLM-pythia8/RunIISpring15DR74-Asympt25ns_MCRUN2_74_V9-v1/MINIAODSIM! \tabularnewline
\verb!/BBbarDMJets_scalar_Mchi-1_Mphi-500_TuneCUETP8M1_13TeV-madgraphMLM-pythia8/RunIISpring15DR74-Asympt25ns_MCRUN2_74_V9-v1/MINIAODSIM! \tabularnewline
\verb!/BBbarDMJets_scalar_Mchi-1_Mphi-1000_TuneCUETP8M1_13TeV-madgraphMLM-pythia8/RunIISpring15DR74-Asympt25ns_MCRUN2_74_V9-v1/MINIAODSIM! \tabularnewline
\verb!/BBbarDMJets_scalar_Mchi-1_Mphi-10000_TuneCUETP8M1_13TeV-madgraphMLM-pythia8/RunIISpring15DR74-Asympt25ns_MCRUN2_74_V9-v1/MINIAODSIM! \tabularnewline
\verb!/BBbarDMJets_scalar_Mchi-10_Mphi-10_TuneCUETP8M1_13TeV-madgraphMLM-pythia8/RunIISpring15DR74-Asympt25ns_MCRUN2_74_V9-v1/MINIAODSIM! \tabularnewline
\verb!/BBbarDMJets_scalar_Mchi-10_Mphi-15_TuneCUETP8M1_13TeV-madgraphMLM-pythia8/RunIISpring15DR74-Asympt25ns_MCRUN2_74_V9-v1/MINIAODSIM! \tabularnewline
\verb!/BBbarDMJets_scalar_Mchi-10_Mphi-50_TuneCUETP8M1_13TeV-madgraphMLM-pythia8/RunIISpring15DR74-Asympt25ns_MCRUN2_74_V9-v1/MINIAODSIM! \tabularnewline
\verb!/BBbarDMJets_scalar_Mchi-10_Mphi-100_TuneCUETP8M1_13TeV-madgraphMLM-pythia8/RunIISpring15DR74-Asympt25ns_MCRUN2_74_V9-v1/MINIAODSIM! \tabularnewline
\verb!/BBbarDMJets_scalar_Mchi-10_Mphi-10000_TuneCUETP8M1_13TeV-madgraphMLM-pythia8/RunIISpring15DR74-Asympt25ns_MCRUN2_74_V9-v1/MINIAODSIM! \tabularnewline
\verb!/BBbarDMJets_scalar_Mchi-50_Mphi-10_TuneCUETP8M1_13TeV-madgraphMLM-pythia8/RunIISpring15DR74-Asympt25ns_MCRUN2_74_V9-v1/MINIAODSIM! \tabularnewline
\verb!/BBbarDMJets_scalar_Mchi-50_Mphi-50_TuneCUETP8M1_13TeV-madgraphMLM-pythia8/RunIISpring15DR74-Asympt25ns_MCRUN2_74_V9-v1/MINIAODSIM! \tabularnewline
\verb!/BBbarDMJets_scalar_Mchi-50_Mphi-95_TuneCUETP8M1_13TeV-madgraphMLM-pythia8/RunIISpring15DR74-Asympt25ns_MCRUN2_74_V9-v1/MINIAODSIM! \tabularnewline
\verb!/BBbarDMJets_scalar_Mchi-150_Mphi-200_TuneCUETP8M1_13TeV-madgraphMLM-pythia8/RunIISpring15DR74-Asympt25ns_MCRUN2_74_V9-v1/MINIAODSIM! \tabularnewline
\verb!/BBbarDMJets_scalar_Mchi-150_Mphi-295_TuneCUETP8M1_13TeV-madgraphMLM-pythia8/RunIISpring15DR74-Asympt25ns_MCRUN2_74_V9-v1/MINIAODSIM! \tabularnewline
\verb!/BBbarDMJets_scalar_Mchi-150_Mphi-500_TuneCUETP8M1_13TeV-madgraphMLM-pythia8/RunIISpring15DR74-Asympt25ns_MCRUN2_74_V9-v1/MINIAODSIM! \tabularnewline
\verb!/BBbarDMJets_scalar_Mchi-150_Mphi-1000_TuneCUETP8M1_13TeV-madgraphMLM-pythia8/RunIISpring15DR74-Asympt25ns_MCRUN2_74_V9-v1/MINIAODSIM! \tabularnewline
\verb!/BBbarDMJets_scalar_Mchi-150_Mphi-10000_TuneCUETP8M1_13TeV-madgraphMLM-pythia8/RunIISpring15DR74-Asympt25ns_MCRUN2_74_V9-v1/MINIAODSIM! \tabularnewline
\verb!/BBbarDMJets_scalar_Mchi-500_Mphi-10_TuneCUETP8M1_13TeV-madgraphMLM-pythia8/RunIISpring15DR74-Asympt25ns_MCRUN2_74_V9-v1/MINIAODSIM! \tabularnewline
\verb!/BBbarDMJets_scalar_Mchi-500_Mphi-500_TuneCUETP8M1_13TeV-madgraphMLM-pythia8/RunIISpring15DR74-Asympt25ns_MCRUN2_74_V9-v1/MINIAODSIM! \tabularnewline
\verb!/BBbarDMJets_scalar_Mchi-500_Mphi-995_TuneCUETP8M1_13TeV-madgraphMLM-pythia8/RunIISpring15DR74-Asympt25ns_MCRUN2_74_V9-v1/MINIAODSIM! \tabularnewline
\verb!/BBbarDMJets_scalar_Mchi-500_Mphi-10000_TuneCUETP8M1_13TeV-madgraphMLM-pythia8/RunIISpring15DR74-Asympt25ns_MCRUN2_74_V9-v1/MINIAODSIM! \tabularnewline
\verb!/BBbarDMJets_scalar_Mchi-1000_Mphi-10_TuneCUETP8M1_13TeV-madgraphMLM-pythia8/RunIISpring15DR74-Asympt25ns_MCRUN2_74_V9-v1/MINIAODSIM! \tabularnewline
\verb!/BBbarDMJets_scalar_Mchi-1000_Mphi-1000_TuneCUETP8M1_13TeV-madgraphMLM-pythia8/RunIISpring15DR74-Asympt25ns_MCRUN2_74_V9-v1/MINIAODSIM! \tabularnewline
\verb!/BBbarDMJets_scalar_Mchi-1000_Mphi-10000_TuneCUETP8M1_13TeV-madgraphMLM-pythia8/RunIISpring15DR74-Asympt25ns_MCRUN2_74_V9-v1/MINIAODSIM! \tabularnewline
\hline
\end{tabular}\end{center}
}
\label{datasets_dm_bbar_pseudoscalar}
\end{table}

\begin{table}[!p]
 \centering
\topcaption{Simulated signal samples: DM \bbbar Pseudoscalar}
 \scriptsize
 \scalebox{.7}[1.0]{%latex.default(d, title = NULL, booktabs = FALSE, width = 3, rowname = NULL,     helvetica = FALSE, caption.loc = "bottom", ...)%
\begin{center}
\begin{tabular}{l}
\hline\hline
\multicolumn{1}{c}{Data set}\tabularnewline
\hline
\verb!/BBbarDMJets_pseudoscalar_Mchi-1_Mphi-10_TuneCUETP8M1_13TeV-madgraphMLM-pythia8/RunIISpring15DR74-Asympt25ns_MCRUN2_74_V9-v1/MINIAODSIM! \tabularnewline
\verb!/BBbarDMJets_pseudoscalar_Mchi-1_Mphi-20_TuneCUETP8M1_13TeV-madgraphMLM-pythia8/RunIISpring15DR74-Asympt25ns_MCRUN2_74_V9-v1/MINIAODSIM! \tabularnewline
\verb!/BBbarDMJets_pseudoscalar_Mchi-1_Mphi-50_TuneCUETP8M1_13TeV-madgraphMLM-pythia8/RunIISpring15DR74-Asympt25ns_MCRUN2_74_V9-v1/MINIAODSIM! \tabularnewline
\verb!/BBbarDMJets_pseudoscalar_Mchi-1_Mphi-100_TuneCUETP8M1_13TeV-madgraphMLM-pythia8/RunIISpring15DR74-Asympt25ns_MCRUN2_74_V9-v1/MINIAODSIM! \tabularnewline
\verb!/BBbarDMJets_pseudoscalar_Mchi-1_Mphi-200_TuneCUETP8M1_13TeV-madgraphMLM-pythia8/RunIISpring15DR74-Asympt25ns_MCRUN2_74_V9-v1/MINIAODSIM! \tabularnewline
\verb!/BBbarDMJets_pseudoscalar_Mchi-1_Mphi-300_TuneCUETP8M1_13TeV-madgraphMLM-pythia8/RunIISpring15DR74-Asympt25ns_MCRUN2_74_V9-v1/MINIAODSIM! \tabularnewline
\verb!/BBbarDMJets_pseudoscalar_Mchi-1_Mphi-500_TuneCUETP8M1_13TeV-madgraphMLM-pythia8/RunIISpring15DR74-Asympt25ns_MCRUN2_74_V9-v1/MINIAODSIM! \tabularnewline
\verb!/BBbarDMJets_pseudoscalar_Mchi-1_Mphi-1000_TuneCUETP8M1_13TeV-madgraphMLM-pythia8/RunIISpring15DR74-Asympt25ns_MCRUN2_74_V9-v1/MINIAODSIM! \tabularnewline
\verb!/BBbarDMJets_pseudoscalar_Mchi-1_Mphi-10000_TuneCUETP8M1_13TeV-madgraphMLM-pythia8/RunIISpring15DR74-Asympt25ns_MCRUN2_74_V9-v1/MINIAODSIM! \tabularnewline
\verb!/BBbarDMJets_pseudoscalar_Mchi-10_Mphi-10_TuneCUETP8M1_13TeV-madgraphMLM-pythia8/RunIISpring15DR74-Asympt25ns_MCRUN2_74_V9-v1/MINIAODSIM! \tabularnewline
\verb!/BBbarDMJets_pseudoscalar_Mchi-10_Mphi-15_TuneCUETP8M1_13TeV-madgraphMLM-pythia8/RunIISpring15DR74-Asympt25ns_MCRUN2_74_V9-v1/MINIAODSIM! \tabularnewline
\verb!/BBbarDMJets_pseudoscalar_Mchi-10_Mphi-50_TuneCUETP8M1_13TeV-madgraphMLM-pythia8/RunIISpring15DR74-Asympt25ns_MCRUN2_74_V9-v1/MINIAODSIM! \tabularnewline
\verb!/BBbarDMJets_pseudoscalar_Mchi-10_Mphi-100_TuneCUETP8M1_13TeV-madgraphMLM-pythia8/RunIISpring15DR74-Asympt25ns_MCRUN2_74_V9-v1/MINIAODSIM! \tabularnewline
\verb!/BBbarDMJets_pseudoscalar_Mchi-10_Mphi-10000_TuneCUETP8M1_13TeV-madgraphMLM-pythia8/RunIISpring15DR74-Asympt25ns_MCRUN2_74_V9-v1/MINIAODSIM! \tabularnewline
\verb!/BBbarDMJets_pseudoscalar_Mchi-50_Mphi-10_TuneCUETP8M1_13TeV-madgraphMLM-pythia8/RunIISpring15DR74-Asympt25ns_MCRUN2_74_V9-v1/MINIAODSIM! \tabularnewline
\verb!/BBbarDMJets_pseudoscalar_Mchi-50_Mphi-50_TuneCUETP8M1_13TeV-madgraphMLM-pythia8/RunIISpring15DR74-Asympt25ns_MCRUN2_74_V9-v1/MINIAODSIM! \tabularnewline
\verb!/BBbarDMJets_pseudoscalar_Mchi-50_Mphi-95_TuneCUETP8M1_13TeV-madgraphMLM-pythia8/RunIISpring15DR74-Asympt25ns_MCRUN2_74_V9-v1/MINIAODSIM! \tabularnewline
\verb!/BBbarDMJets_pseudoscalar_Mchi-50_Mphi-200_TuneCUETP8M1_13TeV-madgraphMLM-pythia8/RunIISpring15DR74-Asympt25ns_MCRUN2_74_V9-v1/MINIAODSIM! \tabularnewline
\verb!/BBbarDMJets_pseudoscalar_Mchi-150_Mphi-200_TuneCUETP8M1_13TeV-madgraphMLM-pythia8/RunIISpring15DR74-Asympt25ns_MCRUN2_74_V9-v1/MINIAODSIM! \tabularnewline
\verb!/BBbarDMJets_pseudoscalar_Mchi-150_Mphi-295_TuneCUETP8M1_13TeV-madgraphMLM-pythia8/RunIISpring15DR74-Asympt25ns_MCRUN2_74_V9-v1/MINIAODSIM! \tabularnewline
\verb!/BBbarDMJets_pseudoscalar_Mchi-150_Mphi-500_TuneCUETP8M1_13TeV-madgraphMLM-pythia8/RunIISpring15DR74-Asympt25ns_MCRUN2_74_V9-v1/MINIAODSIM! \tabularnewline
\verb!/BBbarDMJets_pseudoscalar_Mchi-150_Mphi-1000_TuneCUETP8M1_13TeV-madgraphMLM-pythia8/RunIISpring15DR74-Asympt25ns_MCRUN2_74_V9-v1/MINIAODSIM! \tabularnewline
\verb!/BBbarDMJets_pseudoscalar_Mchi-150_Mphi-10000_TuneCUETP8M1_13TeV-madgraphMLM-pythia8/RunIISpring15DR74-Asympt25ns_MCRUN2_74_V9-v1/MINIAODSIM! \tabularnewline
\verb!/BBbarDMJets_pseudoscalar_Mchi-500_Mphi-10_TuneCUETP8M1_13TeV-madgraphMLM-pythia8/RunIISpring15DR74-Asympt25ns_MCRUN2_74_V9-v1/MINIAODSIM! \tabularnewline
\verb!/BBbarDMJets_pseudoscalar_Mchi-500_Mphi-500_TuneCUETP8M1_13TeV-madgraphMLM-pythia8/RunIISpring15DR74-Asympt25ns_MCRUN2_74_V9-v1/MINIAODSIM! \tabularnewline
\verb!/BBbarDMJets_pseudoscalar_Mchi-500_Mphi-995_TuneCUETP8M1_13TeV-madgraphMLM-pythia8/RunIISpring15DR74-Asympt25ns_MCRUN2_74_V9-v1/MINIAODSIM! \tabularnewline
\verb!/BBbarDMJets_pseudoscalar_Mchi-500_Mphi-10000_TuneCUETP8M1_13TeV-madgraphMLM-pythia8/RunIISpring15DR74-Asympt25ns_MCRUN2_74_V9-v1/MINIAODSIM! \tabularnewline
\verb!/BBbarDMJets_pseudoscalar_Mchi-1000_Mphi-10_TuneCUETP8M1_13TeV-madgraphMLM-pythia8/RunIISpring15DR74-Asympt25ns_MCRUN2_74_V9-v1/MINIAODSIM! \tabularnewline
\verb!/BBbarDMJets_pseudoscalar_Mchi-1000_Mphi-1000_TuneCUETP8M1_13TeV-madgraphMLM-pythia8/RunIISpring15DR74-Asympt25ns_MCRUN2_74_V9-v1/MINIAODSIM! \tabularnewline
\verb!/BBbarDMJets_pseudoscalar_Mchi-1000_Mphi-10000_TuneCUETP8M1_13TeV-madgraphMLM-pythia8/RunIISpring15DR74-Asympt25ns_MCRUN2_74_V9-v1/MINIAODSIM! \tabularnewline
\hline
\end{tabular}\end{center}
}
\label{datasets_dm_bbar_pseudoscalar}
\end{table}

\begin{table}[!p]
 \centering
 \topcaption{Simulated signal samples: DM \ttbar Scalar}
 \scriptsize
 \scalebox{.7}[1.0]{%latex.default(d, title = NULL, booktabs = FALSE, width = 3, rowname = NULL,     helvetica = FALSE, caption.loc = "bottom", ...)%
\begin{center}
\begin{tabular}{l}
\hline\hline
\multicolumn{1}{c}{Data set}\tabularnewline
\hline
\verb!/TTbarDMJets_scalar_Mchi-1_Mphi-10_TuneCUETP8M1_13TeV-madgraphMLM-pythia8/RunIISpring15DR74-Asympt25ns_MCRUN2_74_V9-v1/MINIAODSIM! \tabularnewline
\verb!/TTbarDMJets_scalar_Mchi-1_Mphi-20_TuneCUETP8M1_13TeV-madgraphMLM-pythia8/RunIISpring15DR74-Asympt25ns_MCRUN2_74_V9-v1/MINIAODSIM! \tabularnewline
\verb!/TTbarDMJets_scalar_Mchi-1_Mphi-50_TuneCUETP8M1_13TeV-madgraphMLM-pythia8/RunIISpring15DR74-Asympt25ns_MCRUN2_74_V9-v1/MINIAODSIM! \tabularnewline
\verb!/TTbarDMJets_scalar_Mchi-1_Mphi-100_TuneCUETP8M1_13TeV-madgraphMLM-pythia8/RunIISpring15DR74-Asympt25ns_MCRUN2_74_V9-v1/MINIAODSIM! \tabularnewline
\verb!/TTbarDMJets_scalar_Mchi-1_Mphi-200_TuneCUETP8M1_13TeV-madgraphMLM-pythia8/RunIISpring15DR74-Asympt25ns_MCRUN2_74_V9-v1/MINIAODSIM! \tabularnewline
\verb!/TTbarDMJets_scalar_Mchi-1_Mphi-300_TuneCUETP8M1_13TeV-madgraphMLM-pythia8/RunIISpring15DR74-Asympt25ns_MCRUN2_74_V9-v1/MINIAODSIM! \tabularnewline
\verb!/TTbarDMJets_scalar_Mchi-1_Mphi-500_TuneCUETP8M1_13TeV-madgraphMLM-pythia8/RunIISpring15DR74-Asympt25ns_MCRUN2_74_V9-v1/MINIAODSIM! \tabularnewline
\verb!/TTbarDMJets_scalar_Mchi-1_Mphi-1000_TuneCUETP8M1_13TeV-madgraphMLM-pythia8/RunIISpring15DR74-Asympt25ns_MCRUN2_74_V9-v2/MINIAODSIM! \tabularnewline
\verb!/TTbarDMJets_scalar_Mchi-10_Mphi-10_TuneCUETP8M1_13TeV-madgraphMLM-pythia8/RunIISpring15DR74-Asympt25ns_MCRUN2_74_V9-v1/MINIAODSIM! \tabularnewline
\verb!/TTbarDMJets_scalar_Mchi-10_Mphi-50_TuneCUETP8M1_13TeV-madgraphMLM-pythia8/RunIISpring15DR74-Asympt25ns_MCRUN2_74_V9-v1/MINIAODSIM! \tabularnewline
\verb!/TTbarDMJets_scalar_Mchi-10_Mphi-100_TuneCUETP8M1_13TeV-madgraphMLM-pythia8/RunIISpring15DR74-Asympt25ns_MCRUN2_74_V9-v1/MINIAODSIM! \tabularnewline
\verb!/TTbarDMJets_scalar_Mchi-50_Mphi-50_TuneCUETP8M1_13TeV-madgraphMLM-pythia8/RunIISpring15DR74-Asympt25ns_MCRUN2_74_V9-v1/MINIAODSIM! \tabularnewline
\verb!/TTbarDMJets_scalar_Mchi-50_Mphi-200_TuneCUETP8M1_13TeV-madgraphMLM-pythia8/RunIISpring15DR74-Asympt25ns_MCRUN2_74_V9-v1/MINIAODSIM! \tabularnewline
\verb!/TTbarDMJets_scalar_Mchi-50_Mphi-300_TuneCUETP8M1_13TeV-madgraphMLM-pythia8/RunIISpring15DR74-Asympt25ns_MCRUN2_74_V9-v1/MINIAODSIM! \tabularnewline
\verb!/TTbarDMJets_scalar_Mchi-150_Mphi-200_TuneCUETP8M1_13TeV-madgraphMLM-pythia8/RunIISpring15DR74-Asympt25ns_MCRUN2_74_V9-v1/MINIAODSIM! \tabularnewline
\verb!/TTbarDMJets_scalar_Mchi-150_Mphi-500_TuneCUETP8M1_13TeV-madgraphMLM-pythia8/RunIISpring15DR74-Asympt25ns_MCRUN2_74_V9-v3/MINIAODSIM! \tabularnewline
\verb!/TTbarDMJets_scalar_Mchi-150_Mphi-1000_TuneCUETP8M1_13TeV-madgraphMLM-pythia8/RunIISpring15DR74-Asympt25ns_MCRUN2_74_V9-v3/MINIAODSIM! \tabularnewline
\verb!/TTbarDMJets_scalar_Mchi-500_Mphi-500_TuneCUETP8M1_13TeV-madgraphMLM-pythia8/RunIISpring15DR74-Asympt25ns_MCRUN2_74_V9-v1/MINIAODSIM! \tabularnewline
\hline
\end{tabular}\end{center}
}
\label{datasets_dm_ttbar_scalar}
\end{table}

\begin{table}[!p]
 \centering
 \topcaption{Simulated signal samples: DM \ttbar Pseudoscalar}
 \scriptsize
 \scalebox{.7}[1.0]{%latex.default(d, title = NULL, booktabs = FALSE, width = 3, rowname = NULL,     helvetica = FALSE, caption.loc = "bottom", ...)%
\begin{center}
\begin{tabular}{l}
\hline\hline
\multicolumn{1}{c}{Data set}\tabularnewline
\hline
\verb!/TTbarDMJets_pseudoscalar_Mchi-1_Mphi-10_TuneCUETP8M1_13TeV-madgraphMLM-pythia8/RunIISpring15DR74-Asympt25ns_MCRUN2_74_V9-v1/MINIAODSIM! \tabularnewline
\verb!/TTbarDMJets_pseudoscalar_Mchi-1_Mphi-20_TuneCUETP8M1_13TeV-madgraphMLM-pythia8/RunIISpring15DR74-Asympt25ns_MCRUN2_74_V9-v1/MINIAODSIM! \tabularnewline
\verb!/TTbarDMJets_pseudoscalar_Mchi-1_Mphi-50_TuneCUETP8M1_13TeV-madgraphMLM-pythia8/RunIISpring15DR74-Asympt25ns_MCRUN2_74_V9-v1/MINIAODSIM! \tabularnewline
\verb!/TTbarDMJets_pseudoscalar_Mchi-1_Mphi-100_TuneCUETP8M1_13TeV-madgraphMLM-pythia8/RunIISpring15DR74-Asympt25ns_MCRUN2_74_V9-v1/MINIAODSIM! \tabularnewline
\verb!/TTbarDMJets_pseudoscalar_Mchi-1_Mphi-200_TuneCUETP8M1_13TeV-madgraphMLM-pythia8/RunIISpring15DR74-Asympt25ns_MCRUN2_74_V9-v1/MINIAODSIM! \tabularnewline
\verb!/TTbarDMJets_pseudoscalar_Mchi-1_Mphi-300_TuneCUETP8M1_13TeV-madgraphMLM-pythia8/RunIISpring15DR74-Asympt25ns_MCRUN2_74_V9-v1/MINIAODSIM! \tabularnewline
\verb!/TTbarDMJets_pseudoscalar_Mchi-1_Mphi-500_TuneCUETP8M1_13TeV-madgraphMLM-pythia8/RunIISpring15DR74-Asympt25ns_MCRUN2_74_V9-v1/MINIAODSIM! \tabularnewline
\verb!/TTbarDMJets_pseudoscalar_Mchi-10_Mphi-10_TuneCUETP8M1_13TeV-madgraphMLM-pythia8/RunIISpring15DR74-Asympt25ns_MCRUN2_74_V9-v1/MINIAODSIM! \tabularnewline
\verb!/TTbarDMJets_pseudoscalar_Mchi-10_Mphi-50_TuneCUETP8M1_13TeV-madgraphMLM-pythia8/RunIISpring15DR74-Asympt25ns_MCRUN2_74_V9-v1/MINIAODSIM! \tabularnewline
\verb!/TTbarDMJets_pseudoscalar_Mchi-10_Mphi-100_TuneCUETP8M1_13TeV-madgraphMLM-pythia8/RunIISpring15DR74-Asympt25ns_MCRUN2_74_V9-v1/MINIAODSIM! \tabularnewline
\verb!/TTbarDMJets_pseudoscalar_Mchi-50_Mphi-50_TuneCUETP8M1_13TeV-madgraphMLM-pythia8/RunIISpring15DR74-Asympt25ns_MCRUN2_74_V9-v1/MINIAODSIM! \tabularnewline
\verb!/TTbarDMJets_pseudoscalar_Mchi-50_Mphi-200_TuneCUETP8M1_13TeV-madgraphMLM-pythia8/RunIISpring15DR74-Asympt25ns_MCRUN2_74_V9-v1/MINIAODSIM! \tabularnewline
\verb!/TTbarDMJets_pseudoscalar_Mchi-50_Mphi-300_TuneCUETP8M1_13TeV-madgraphMLM-pythia8/RunIISpring15DR74-Asympt25ns_MCRUN2_74_V9-v1/MINIAODSIM! \tabularnewline
\verb!/TTbarDMJets_pseudoscalar_Mchi-150_Mphi-200_TuneCUETP8M1_13TeV-madgraphMLM-pythia8/RunIISpring15DR74-Asympt25ns_MCRUN2_74_V9-v1/MINIAODSIM! \tabularnewline
\verb!/TTbarDMJets_pseudoscalar_Mchi-150_Mphi-500_TuneCUETP8M1_13TeV-madgraphMLM-pythia8/RunIISpring15DR74-Asympt25ns_MCRUN2_74_V9-v1/MINIAODSIM! \tabularnewline
\verb!/TTbarDMJets_pseudoscalar_Mchi-150_Mphi-1000_TuneCUETP8M1_13TeV-madgraphMLM-pythia8/RunIISpring15DR74-Asympt25ns_MCRUN2_74_V9-v1/MINIAODSIM! \tabularnewline
\verb!/TTbarDMJets_pseudoscalar_Mchi-500_Mphi-500_TuneCUETP8M1_13TeV-madgraphMLM-pythia8/RunIISpring15DR74-Asympt25ns_MCRUN2_74_V9-v1/MINIAODSIM! \tabularnewline
\hline
\end{tabular}\end{center}
}
 \label{tab:datasets_dm_ttbar_pseudoscalar}
\end{table}



All samples are corrected to reproduce the pileup interactions per bunch crossing observed in data

