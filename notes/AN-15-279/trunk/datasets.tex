\section{Data sets}
\label{sec:datasets}

\subsection{Data}


In this note, we use 2.1~\ifb proton-proton collision data at $\sqrt{s} =$ 13~TeV collected in 2015. The specific datasets are given in 

\begin{table}[!h]
\topcaption{Data sets}
\footnotesize %latex.default(d, title = NULL, booktabs = FALSE, width = 3, rowname = NULL,     helvetica = FALSE, caption.loc = "bottom", ...)%
\begin{center}
\begin{tabular}{lr}
\hline\hline
\multicolumn{1}{c}{Data set}&\multicolumn{1}{c}{$\int\mathcal{L}\textrm{d}t [\textrm{pb}^{-1}]$}\tabularnewline
\hline
\verb!/HTMHT/Run2015D-05Oct2015-v1/MINIAOD! &$ 551.60$\tabularnewline
\verb!/HTMHT/Run2015D-PromptReco-v4/MINIAOD! &$1599.66$\tabularnewline
\verb!/JetHT/Run2015D-05Oct2015-v1/MINIAOD! &$ 552.67$\tabularnewline
\verb!/JetHT/Run2015D-PromptReco-v4/MINIAOD! &$1599.66$\tabularnewline
\verb!/MET/Run2015D-05Oct2015-v1/MINIAOD! &$ 552.67$\tabularnewline
\verb!/MET/Run2015D-PromptReco-v4/MINIAOD! &$1599.66$\tabularnewline
\verb!/SingleElectron/Run2015D-05Oct2015-v1/MINIAOD! &$ 552.63$\tabularnewline
\verb!/SingleElectron/Run2015D-PromptReco-v4/MINIAOD! &$1599.11$\tabularnewline
\verb!/SingleMuon/Run2015D-05Oct2015-v1/MINIAOD! &$ 552.67$\tabularnewline
\verb!/SingleMuon/Run2015D-PromptReco-v4/MINIAOD! &$1599.53$\tabularnewline
\verb!/SinglePhoton/Run2015D-05Oct2015-v1/MINIAOD! &$ 552.67$\tabularnewline
\verb!/SinglePhoton/Run2015D-PromptReco-v4/MINIAOD! &$1598.83$\tabularnewline
\hline
\end{tabular}\end{center}

\label{tab:datasets_data}
\end{table}

Latest JSON files are applied. Further details are given in Ref.~\cite{alphaTnote}. 



\subsection{Simulation}

\subsection{Background}

A large range of possible background sources has been simulated, these are:

\begin{center}
  \begin{itemize}
  \item[DY$\to \ell \ell$:] In bins of \HT from 100 GeV onwards
  \item[$\gamma+\textrm{jets}$:] In bins of \HT from 100 GeV onewards
  \item[$QCD$:] In bins of \HT from $100-2000$ GeV 
  \item[$\ttj$:] Hadronic, single lepton and dileptonic final states
  \item[$W+\textrm{jets}$:] In bins of \HT from 100 GeV onewards
  \item[$Z+\textrm{jets}$:] In bins of \HT from 100 GeV onewards
  \item[Diboson:] Inclusive $WW$, $WZ$ and $ZZ$ production
  \end{itemize}
\end{center}



In these datasets in additoin to the main interaction, each event contains on average 20 minimum bias interactions which simulate multiple interactions per bunch-crossing (in-time pileup). The expected detector signal from previous or following bunch crossings (out-of-time pileup) with $25$ns bunch spacing is overlapped.


The samples binned according to generator level quantities  
($W+$jets,D$Y+$jets,$QCD$,$\gamm+$jets, $Z\to \nu \nu$+jets) are provided with LO cross sections only. The LO to (N)NLO corrections (\kfactors) are usually determined determined using corresponding inclusive samples applied to each HT binned sample. Further studies can provide corrections to the cross sections, which can prove important to the closure test procedures described later. Residual cross section corrections are measured using data in sidebands designed to enriched specific processes.


In the $8$ TeV LHC results the top quark momentum spectrum was found to differ between
simulation and data. A reweighting is therefore applied to MC events that contain a generated top. The value of this correction is provided from the 8 TeV results, as described in~\cite{twiki-TopPtReweighting}.

\subsubsection{Signal}

A list of possible dark matter simplified models are produced following the ATLAS-CMS-Theory DM Forum recommendations. The samples have been centrally produced using the \textsc{POWHEG} generator for mediator mass range of $1 GeV \le \mphi\le 10000$ GeV and dark matter masses $\mchi$ between $1 \ge \mchi < 400$ GeV. We simulate four possible interaction, (vector-)axial (A, AV)and (pseudo-)scalar (PS, S).

Additionally we separately simulate light- ($u,~c,~d,~s$) and heavy quarks ($b,~t$) production. For all samples we assume for the couplings  $\gdm$ between mediator and dark maater $\gdm=1.0$. 

For the coupling $\gsm$ between mediator and SM two values are assumed, $\gsm=1.0$ for scalar and pseudo-scalar couplings and $\gsm=0.25$ for (A)V interactions. In all casesd we assume the minimal width, e.g. no decay to other particles is allowed.

Typically kinematics and expeced production cross section are comparable between V(A) couplings produced mostly via $q\bar{q}$-annihilation and (P)S production mainly from gluon-fusion.

A subset of these samples is given in Tables~\ref{datasets_dm_vector}-~\ref{tab:datasets_dm_ttbar_pseudoscalar}. 


\begin{table}[!p]
 \centering
\topcaption{Simulated signal samples: DM Vector}
 \scriptsize
 \scalebox{.7}[1.0]{\input{tables/datasets/c140607_c151022_l004_DMV_NNPDF30_Vector.tex}}
\label{datasets_dm_vector}
\end{table}

\begin{table}[!p]
 \centering
\topcaption{Simulated signal samples: DM Axial}
 \tiny
 \scalebox{.7}[1.0]{\input{tables/datasets/c140607_c151022_l004_DMV_NNPDF30_Axial.tex}}
\label{datasets_dm_axial}
\end{table}

\begin{table}[!p]
 \centering
\topcaption{Simulated signal samples: DM Scalar}
 \scriptsize
 \scalebox{.7}[1.0]{%latex.default(d, title = NULL, booktabs = FALSE, width = 3, rowname = NULL,     helvetica = FALSE, caption.loc = "bottom", ...)%
\begin{center}
\begin{tabular}{l}
\hline\hline
\multicolumn{1}{c}{Data set}\tabularnewline
\hline
\verb!/DMS_NNPDF30_Scalar_Mphi-10_Mchi-1_gSM-1p0_gDM-1p0_13TeV-powheg/RunIISpring15DR74-Asympt25ns_MCRUN2_74_V9-v1/MINIAODSIM! \tabularnewline
\verb!/DMS_NNPDF30_Scalar_Mphi-10_Mchi-10_gSM-1p0_gDM-1p0_13TeV-powheg/RunIISpring15DR74-Asympt25ns_MCRUN2_74_V9-v1/MINIAODSIM! \tabularnewline
\verb!/DMS_NNPDF30_Scalar_Mphi-20_Mchi-1_gSM-1p0_gDM-1p0_13TeV-powheg/RunIISpring15DR74-Asympt25ns_MCRUN2_74_V9-v1/MINIAODSIM! \tabularnewline
\verb!/DMS_NNPDF30_Scalar_Mphi-20_Mchi-10_gSM-1p0_gDM-1p0_13TeV-powheg/RunIISpring15DR74-Asympt25ns_MCRUN2_74_V9-v1/MINIAODSIM! \tabularnewline
\verb!/DMS_NNPDF30_Scalar_Mphi-50_Mchi-10_gSM-1p0_gDM-1p0_13TeV-powheg/RunIISpring15DR74-Asympt25ns_MCRUN2_74_V9-v1/MINIAODSIM! \tabularnewline
\verb!/DMS_NNPDF30_Scalar_Mphi-100_Mchi-1_gSM-1p0_gDM-1p0_13TeV-powheg/RunIISpring15DR74-Asympt25ns_MCRUN2_74_V9-v1/MINIAODSIM! \tabularnewline
\verb!/DMS_NNPDF30_Scalar_Mphi-100_Mchi-10_gSM-1p0_gDM-1p0_13TeV-powheg/RunIISpring15DR74-Asympt25ns_MCRUN2_74_V9-v1/MINIAODSIM! \tabularnewline
\verb!/DMS_NNPDF30_Scalar_Mphi-100_Mchi-50_gSM-1p0_gDM-1p0_13TeV-powheg/RunIISpring15DR74-Asympt25ns_MCRUN2_74_V9-v1/MINIAODSIM! \tabularnewline
\verb!/DMS_NNPDF30_Scalar_Mphi-100_Mchi-100_gSM-1p0_gDM-1p0_13TeV-powheg/RunIISpring15DR74-Asympt25ns_MCRUN2_74_V9-v1/MINIAODSIM! \tabularnewline
\verb!/DMS_NNPDF30_Scalar_Mphi-200_Mchi-1_gSM-1p0_gDM-1p0_13TeV-powheg/RunIISpring15DR74-Asympt25ns_MCRUN2_74_V9-v1/MINIAODSIM! \tabularnewline
\verb!/DMS_NNPDF30_Scalar_Mphi-200_Mchi-10_gSM-1p0_gDM-1p0_13TeV-powheg/RunIISpring15DR74-Asympt25ns_MCRUN2_74_V9-v1/MINIAODSIM! \tabularnewline
\verb!/DMS_NNPDF30_Scalar_Mphi-200_Mchi-50_gSM-1p0_gDM-1p0_13TeV-powheg/RunIISpring15DR74-Asympt25ns_MCRUN2_74_V9-v1/MINIAODSIM! \tabularnewline
\verb!/DMS_NNPDF30_Scalar_Mphi-200_Mchi-100_gSM-1p0_gDM-1p0_13TeV-powheg/RunIISpring15DR74-Asympt25ns_MCRUN2_74_V9-v1/MINIAODSIM! \tabularnewline
\verb!/DMS_NNPDF30_Scalar_Mphi-200_Mchi-150_gSM-1p0_gDM-1p0_13TeV-powheg/RunIISpring15DR74-Asympt25ns_MCRUN2_74_V9-v1/MINIAODSIM! \tabularnewline
\verb!/DMS_NNPDF30_Scalar_Mphi-300_Mchi-1_gSM-1p0_gDM-1p0_13TeV-powheg/RunIISpring15DR74-Asympt25ns_MCRUN2_74_V9-v1/MINIAODSIM! \tabularnewline
\verb!/DMS_NNPDF30_Scalar_Mphi-300_Mchi-10_gSM-1p0_gDM-1p0_13TeV-powheg/RunIISpring15DR74-Asympt25ns_MCRUN2_74_V9-v1/MINIAODSIM! \tabularnewline
\verb!/DMS_NNPDF30_Scalar_Mphi-300_Mchi-100_gSM-1p0_gDM-1p0_13TeV-powheg/RunIISpring15DR74-Asympt25ns_MCRUN2_74_V9-v1/MINIAODSIM! \tabularnewline
\verb!/DMS_NNPDF30_Scalar_Mphi-300_Mchi-150_gSM-1p0_gDM-1p0_13TeV-powheg/RunIISpring15DR74-Asympt25ns_MCRUN2_74_V9-v1/MINIAODSIM! \tabularnewline
\verb!/DMS_NNPDF30_Scalar_Mphi-500_Mchi-1_gSM-1p0_gDM-1p0_13TeV-powheg/RunIISpring15DR74-Asympt25ns_MCRUN2_74_V9-v1/MINIAODSIM! \tabularnewline
\verb!/DMS_NNPDF30_Scalar_Mphi-500_Mchi-10_gSM-1p0_gDM-1p0_13TeV-powheg/RunIISpring15DR74-Asympt25ns_MCRUN2_74_V9-v1/MINIAODSIM! \tabularnewline
\verb!/DMS_NNPDF30_Scalar_Mphi-500_Mchi-50_gSM-1p0_gDM-1p0_13TeV-powheg/RunIISpring15DR74-Asympt25ns_MCRUN2_74_V9-v1/MINIAODSIM! \tabularnewline
\verb!/DMS_NNPDF30_Scalar_Mphi-500_Mchi-150_gSM-1p0_gDM-1p0_13TeV-powheg/RunIISpring15DR74-Asympt25ns_MCRUN2_74_V9-v1/MINIAODSIM! \tabularnewline
\verb!/DMS_NNPDF30_Scalar_Mphi-500_Mchi-500_gSM-1p0_gDM-1p0_13TeV-powheg/RunIISpring15DR74-Asympt25ns_MCRUN2_74_V9-v1/MINIAODSIM! \tabularnewline
\verb!/DMS_NNPDF30_Scalar_Mphi-1000_Mchi-1_gSM-1p0_gDM-1p0_13TeV-powheg/RunIISpring15DR74-Asympt25ns_MCRUN2_74_V9-v1/MINIAODSIM! \tabularnewline
\verb!/DMS_NNPDF30_Scalar_Mphi-1000_Mchi-10_gSM-1p0_gDM-1p0_13TeV-powheg/RunIISpring15DR74-Asympt25ns_MCRUN2_74_V9-v1/MINIAODSIM! \tabularnewline
\verb!/DMS_NNPDF30_Scalar_Mphi-1000_Mchi-100_gSM-1p0_gDM-1p0_13TeV-powheg/RunIISpring15DR74-Asympt25ns_MCRUN2_74_V9-v1/MINIAODSIM! \tabularnewline
\verb!/DMS_NNPDF30_Scalar_Mphi-1000_Mchi-500_gSM-1p0_gDM-1p0_13TeV-powheg/RunIISpring15DR74-Asympt25ns_MCRUN2_74_V9-v1/MINIAODSIM! \tabularnewline
\verb!/DMS_NNPDF30_Scalar_Mphi-2000_Mchi-1_gSM-1p0_gDM-1p0_13TeV-powheg/RunIISpring15DR74-Asympt25ns_MCRUN2_74_V9-v1/MINIAODSIM! \tabularnewline
\verb!/DMS_NNPDF30_Scalar_Mphi-2000_Mchi-100_gSM-1p0_gDM-1p0_13TeV-powheg/RunIISpring15DR74-Asympt25ns_MCRUN2_74_V9-v1/MINIAODSIM! \tabularnewline
\verb!/DMS_NNPDF30_Scalar_Mphi-2000_Mchi-150_gSM-1p0_gDM-1p0_13TeV-powheg/RunIISpring15DR74-Asympt25ns_MCRUN2_74_V9-v1/MINIAODSIM! \tabularnewline
\verb!/DMS_NNPDF30_Scalar_Mphi-2000_Mchi-1000_gSM-1p0_gDM-1p0_13TeV-powheg/RunIISpring15DR74-Asympt25ns_MCRUN2_74_V9-v1/MINIAODSIM! \tabularnewline
\verb!/DMS_NNPDF30_Scalar_Mphi-5000_Mchi-10_gSM-1p0_gDM-1p0_13TeV-powheg/RunIISpring15DR74-Asympt25ns_MCRUN2_74_V9-v1/MINIAODSIM! \tabularnewline
\verb!/DMS_NNPDF30_Scalar_Mphi-5000_Mchi-50_gSM-1p0_gDM-1p0_13TeV-powheg/RunIISpring15DR74-Asympt25ns_MCRUN2_74_V9-v1/MINIAODSIM! \tabularnewline
\verb!/DMS_NNPDF30_Scalar_Mphi-5000_Mchi-100_gSM-1p0_gDM-1p0_13TeV-powheg/RunIISpring15DR74-Asympt25ns_MCRUN2_74_V9-v1/MINIAODSIM! \tabularnewline
\verb!/DMS_NNPDF30_Scalar_Mphi-5000_Mchi-150_gSM-1p0_gDM-1p0_13TeV-powheg/RunIISpring15DR74-Asympt25ns_MCRUN2_74_V9-v1/MINIAODSIM! \tabularnewline
\verb!/DMS_NNPDF30_Scalar_Mphi-5000_Mchi-500_gSM-1p0_gDM-1p0_13TeV-powheg/RunIISpring15DR74-Asympt25ns_MCRUN2_74_V9-v1/MINIAODSIM! \tabularnewline
\verb!/DMS_NNPDF30_Scalar_Mphi-5000_Mchi-1000_gSM-1p0_gDM-1p0_13TeV-powheg/RunIISpring15DR74-Asympt25ns_MCRUN2_74_V9-v1/MINIAODSIM! \tabularnewline
\verb!/DMS_NNPDF30_Scalar_Mphi-10000_Mchi-1_gSM-1p0_gDM-1p0_13TeV-powheg/RunIISpring15DR74-Asympt25ns_MCRUN2_74_V9-v1/MINIAODSIM! \tabularnewline
\verb!/DMS_NNPDF30_Scalar_Mphi-10000_Mchi-10_gSM-1p0_gDM-1p0_13TeV-powheg/RunIISpring15DR74-Asympt25ns_MCRUN2_74_V9-v1/MINIAODSIM! \tabularnewline
\verb!/DMS_NNPDF30_Scalar_Mphi-10000_Mchi-500_gSM-1p0_gDM-1p0_13TeV-powheg/RunIISpring15DR74-Asympt25ns_MCRUN2_74_V9-v1/MINIAODSIM! \tabularnewline
\verb!/DMS_NNPDF30_Scalar_Mphi-10000_Mchi-1000_gSM-1p0_gDM-1p0_13TeV-powheg/RunIISpring15DR74-Asympt25ns_MCRUN2_74_V9-v1/MINIAODSIM! \tabularnewline
\hline
\end{tabular}\end{center}
}
\label{datasets_dm_scalarw}
\end{table}

\begin{table}[!p]
 \centering
\topcaption{Simulated signal samples: DM Pseudoscalar}
 \scriptsize
 \scalebox{.7}[1.0]{\input{tables/datasets/c140607_c151022_l004_DMS_NNPDF30_Pseudoscalar.tex}}
\label{datasets_dm_pseudoscalar}
\end{table}

\begin{table}[!p]
 \centering
\topcaption{Simulated signal samples: DM \bbbar Scalar}
 \scriptsize
 \scalebox{.7}[1.0]{\input{tables/datasets/c140607_c151022_l004_BBbarDMJets_scalar.tex}}
\label{datasets_dm_bbar_pseudoscalar}
\end{table}

\begin{table}[!p]
 \centering
\topcaption{Simulated signal samples: DM \bbbar Pseudoscalar}
 \scriptsize
 \scalebox{.7}[1.0]{%latex.default(d, title = NULL, booktabs = FALSE, width = 3, rowname = NULL,     helvetica = FALSE, caption.loc = "bottom", ...)%
\begin{center}
\begin{tabular}{l}
\hline\hline
\multicolumn{1}{c}{Data set}\tabularnewline
\hline
\verb!/BBbarDMJets_pseudoscalar_Mchi-1_Mphi-10_TuneCUETP8M1_13TeV-madgraphMLM-pythia8/RunIISpring15DR74-Asympt25ns_MCRUN2_74_V9-v1/MINIAODSIM! \tabularnewline
\verb!/BBbarDMJets_pseudoscalar_Mchi-1_Mphi-20_TuneCUETP8M1_13TeV-madgraphMLM-pythia8/RunIISpring15DR74-Asympt25ns_MCRUN2_74_V9-v1/MINIAODSIM! \tabularnewline
\verb!/BBbarDMJets_pseudoscalar_Mchi-1_Mphi-50_TuneCUETP8M1_13TeV-madgraphMLM-pythia8/RunIISpring15DR74-Asympt25ns_MCRUN2_74_V9-v1/MINIAODSIM! \tabularnewline
\verb!/BBbarDMJets_pseudoscalar_Mchi-1_Mphi-100_TuneCUETP8M1_13TeV-madgraphMLM-pythia8/RunIISpring15DR74-Asympt25ns_MCRUN2_74_V9-v1/MINIAODSIM! \tabularnewline
\verb!/BBbarDMJets_pseudoscalar_Mchi-1_Mphi-200_TuneCUETP8M1_13TeV-madgraphMLM-pythia8/RunIISpring15DR74-Asympt25ns_MCRUN2_74_V9-v1/MINIAODSIM! \tabularnewline
\verb!/BBbarDMJets_pseudoscalar_Mchi-1_Mphi-300_TuneCUETP8M1_13TeV-madgraphMLM-pythia8/RunIISpring15DR74-Asympt25ns_MCRUN2_74_V9-v1/MINIAODSIM! \tabularnewline
\verb!/BBbarDMJets_pseudoscalar_Mchi-1_Mphi-500_TuneCUETP8M1_13TeV-madgraphMLM-pythia8/RunIISpring15DR74-Asympt25ns_MCRUN2_74_V9-v1/MINIAODSIM! \tabularnewline
\verb!/BBbarDMJets_pseudoscalar_Mchi-1_Mphi-1000_TuneCUETP8M1_13TeV-madgraphMLM-pythia8/RunIISpring15DR74-Asympt25ns_MCRUN2_74_V9-v1/MINIAODSIM! \tabularnewline
\verb!/BBbarDMJets_pseudoscalar_Mchi-1_Mphi-10000_TuneCUETP8M1_13TeV-madgraphMLM-pythia8/RunIISpring15DR74-Asympt25ns_MCRUN2_74_V9-v1/MINIAODSIM! \tabularnewline
\verb!/BBbarDMJets_pseudoscalar_Mchi-10_Mphi-10_TuneCUETP8M1_13TeV-madgraphMLM-pythia8/RunIISpring15DR74-Asympt25ns_MCRUN2_74_V9-v1/MINIAODSIM! \tabularnewline
\verb!/BBbarDMJets_pseudoscalar_Mchi-10_Mphi-15_TuneCUETP8M1_13TeV-madgraphMLM-pythia8/RunIISpring15DR74-Asympt25ns_MCRUN2_74_V9-v1/MINIAODSIM! \tabularnewline
\verb!/BBbarDMJets_pseudoscalar_Mchi-10_Mphi-50_TuneCUETP8M1_13TeV-madgraphMLM-pythia8/RunIISpring15DR74-Asympt25ns_MCRUN2_74_V9-v1/MINIAODSIM! \tabularnewline
\verb!/BBbarDMJets_pseudoscalar_Mchi-10_Mphi-100_TuneCUETP8M1_13TeV-madgraphMLM-pythia8/RunIISpring15DR74-Asympt25ns_MCRUN2_74_V9-v1/MINIAODSIM! \tabularnewline
\verb!/BBbarDMJets_pseudoscalar_Mchi-10_Mphi-10000_TuneCUETP8M1_13TeV-madgraphMLM-pythia8/RunIISpring15DR74-Asympt25ns_MCRUN2_74_V9-v1/MINIAODSIM! \tabularnewline
\verb!/BBbarDMJets_pseudoscalar_Mchi-50_Mphi-10_TuneCUETP8M1_13TeV-madgraphMLM-pythia8/RunIISpring15DR74-Asympt25ns_MCRUN2_74_V9-v1/MINIAODSIM! \tabularnewline
\verb!/BBbarDMJets_pseudoscalar_Mchi-50_Mphi-50_TuneCUETP8M1_13TeV-madgraphMLM-pythia8/RunIISpring15DR74-Asympt25ns_MCRUN2_74_V9-v1/MINIAODSIM! \tabularnewline
\verb!/BBbarDMJets_pseudoscalar_Mchi-50_Mphi-95_TuneCUETP8M1_13TeV-madgraphMLM-pythia8/RunIISpring15DR74-Asympt25ns_MCRUN2_74_V9-v1/MINIAODSIM! \tabularnewline
\verb!/BBbarDMJets_pseudoscalar_Mchi-50_Mphi-200_TuneCUETP8M1_13TeV-madgraphMLM-pythia8/RunIISpring15DR74-Asympt25ns_MCRUN2_74_V9-v1/MINIAODSIM! \tabularnewline
\verb!/BBbarDMJets_pseudoscalar_Mchi-150_Mphi-200_TuneCUETP8M1_13TeV-madgraphMLM-pythia8/RunIISpring15DR74-Asympt25ns_MCRUN2_74_V9-v1/MINIAODSIM! \tabularnewline
\verb!/BBbarDMJets_pseudoscalar_Mchi-150_Mphi-295_TuneCUETP8M1_13TeV-madgraphMLM-pythia8/RunIISpring15DR74-Asympt25ns_MCRUN2_74_V9-v1/MINIAODSIM! \tabularnewline
\verb!/BBbarDMJets_pseudoscalar_Mchi-150_Mphi-500_TuneCUETP8M1_13TeV-madgraphMLM-pythia8/RunIISpring15DR74-Asympt25ns_MCRUN2_74_V9-v1/MINIAODSIM! \tabularnewline
\verb!/BBbarDMJets_pseudoscalar_Mchi-150_Mphi-1000_TuneCUETP8M1_13TeV-madgraphMLM-pythia8/RunIISpring15DR74-Asympt25ns_MCRUN2_74_V9-v1/MINIAODSIM! \tabularnewline
\verb!/BBbarDMJets_pseudoscalar_Mchi-150_Mphi-10000_TuneCUETP8M1_13TeV-madgraphMLM-pythia8/RunIISpring15DR74-Asympt25ns_MCRUN2_74_V9-v1/MINIAODSIM! \tabularnewline
\verb!/BBbarDMJets_pseudoscalar_Mchi-500_Mphi-10_TuneCUETP8M1_13TeV-madgraphMLM-pythia8/RunIISpring15DR74-Asympt25ns_MCRUN2_74_V9-v1/MINIAODSIM! \tabularnewline
\verb!/BBbarDMJets_pseudoscalar_Mchi-500_Mphi-500_TuneCUETP8M1_13TeV-madgraphMLM-pythia8/RunIISpring15DR74-Asympt25ns_MCRUN2_74_V9-v1/MINIAODSIM! \tabularnewline
\verb!/BBbarDMJets_pseudoscalar_Mchi-500_Mphi-995_TuneCUETP8M1_13TeV-madgraphMLM-pythia8/RunIISpring15DR74-Asympt25ns_MCRUN2_74_V9-v1/MINIAODSIM! \tabularnewline
\verb!/BBbarDMJets_pseudoscalar_Mchi-500_Mphi-10000_TuneCUETP8M1_13TeV-madgraphMLM-pythia8/RunIISpring15DR74-Asympt25ns_MCRUN2_74_V9-v1/MINIAODSIM! \tabularnewline
\verb!/BBbarDMJets_pseudoscalar_Mchi-1000_Mphi-10_TuneCUETP8M1_13TeV-madgraphMLM-pythia8/RunIISpring15DR74-Asympt25ns_MCRUN2_74_V9-v1/MINIAODSIM! \tabularnewline
\verb!/BBbarDMJets_pseudoscalar_Mchi-1000_Mphi-1000_TuneCUETP8M1_13TeV-madgraphMLM-pythia8/RunIISpring15DR74-Asympt25ns_MCRUN2_74_V9-v1/MINIAODSIM! \tabularnewline
\verb!/BBbarDMJets_pseudoscalar_Mchi-1000_Mphi-10000_TuneCUETP8M1_13TeV-madgraphMLM-pythia8/RunIISpring15DR74-Asympt25ns_MCRUN2_74_V9-v1/MINIAODSIM! \tabularnewline
\hline
\end{tabular}\end{center}
}
\label{datasets_dm_bbar_pseudoscalar}
\end{table}

\begin{table}[!p]
 \centering
 \topcaption{Simulated signal samples: DM \ttbar Scalar}
 \scriptsize
 \scalebox{.7}[1.0]{\input{tables/datasets/c140607_c151022_l004_TTbarDMJets_scalar.tex}}
\label{datasets_dm_ttbar_scalar}
\end{table}

\begin{table}[!p]
 \centering
 \topcaption{Simulated signal samples: DM \ttbar Pseudoscalar}
 \scriptsize
 \scalebox{.7}[1.0]{\input{tables/datasets/c140607_c151022_l004_TTbarDMJets_pseudoscalar.tex}}
 \label{tab:datasets_dm_ttbar_pseudoscalar}
\end{table}





