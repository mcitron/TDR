%%____________________________________________________________________________||
\section{Definition of alphaT}
\label{sec:alphatdef}

The \alphat~\cite{Randall:2008rw, CMS:2008vya, CMS-PAS-SUS-09-001} variable is used to
efficiently reject multijet events without significant \met or with
transverse energy mismeasurements, while retaining a large sensitivity
to new physics with final-state signatures containing significant
\met.

The measurement of \met typically relies on independent sources of
information from each of the calorimeter, tracking, and muon
subdetectors. Relative to other physics objects, this
measurement is particularly sensitive to the beam conditions and
detector performance. This difficulty is compounded by the
high-energy, high-luminosity hadron collider environment at the LHC
and the lack of precise theoretical predictions for the kinematic
properties and cross sections of multijet events. Hence it is very
challenging to control and model accurately the multijet background in
the high-\met phase space that is typical of SUSY searches.

Given these difficulties, the variable \alphat was developed to avoid
direct reliance on a measurement of \met, instead depending solely on
the measurements of the transverse energies and (relative) azimuthal
angles of jets. The variable is
intrinsically robust against the presence of jet energy
mismeasurements in multijet systems.\\
%and is observed to be robust against the effects of pileup. 
For dijet events, the \alphat variable is defined
as:

\begin{equation}
\label{eq:alphat}
\alphat\, =\, \frac{\Et^{{\rm j}_2}}{M_\text{T}} \, ,
\end{equation}

where $\Et^{\rm j_2}$ is the transverse energy of the less energetic
jet and $M_\text{T}$ is the transverse mass of the dijet system,
defined as

\begin{equation}
  \label{eq:mt}
  M_\text{T}\, = \,\sqrt{ \left( \sum_{i=1}^2 \Et^{{\rm j}_i}
    \right)^2 - \left( \sum_{i=1}^2 p_x^{{\rm j}_i} \right)^2 - \left(
      \sum_{i=1}^2 p_y^{{\rm j}_i} \right)^2} \, .
\end{equation}

where $\Et^{{\rm j}_i}$ is the transverse energy of jet ${\rm j}_i$ (
$\Et^{{\rm j}_i} = E^{{\rm j}_i}\sin\theta^{{\rm j}_i}$), and
$p_x^{{\rm j}_i}$ and $p_y^{{\rm j}_i}$ are the $x$ and $y$ components
of the transverse momentum of the jet.

For a perfectly measured dijet event with $\Et^{\rm j_1} = \Et^{\rm
  j_2}$ and jets back-to-back in $\phi$, and in the limit in which the
momentum of each jet is large compared with its mass, the value of
\alphat is 0.5. For the case of an imbalance in the measured
transverse energies of back-to-back jets, \alphat is reduced to a
value smaller than 0.5, which gives the variable its intrinsic
robustness with respect to jet energy mismeasurements. Values
significantly greater than 0.5 are observed when the two jets are not
back-to-back and are recoiling against significant, genuine \met.

The definition of the \alphat variable can be generalised for events
with two or more jets as follows. The mass scale of the physics
processes being probed is characterised by the scalar sum of the
transverse energy $\Et$ of jets considered in the analysis,
$\sum_{i=1}^{\njet} \Et^{{\rm j}_i}$, where \njet is the
number of jets with \Et above a predefined threshold.\footnote{We note
here that the variable \scalht is reserved for the scalar sum of the
transverse momentum $\Pt$ of jets considered in the analysis.} 
The estimator
for \met is given by the magnitude of the transverse momenta
$\vec{\pt}$ vectorial sum over these jets, defined as $\mht =
|\sum_{i=1}^{n_{\rm jet}} \vec{\pt}^{{\rm j}_i}|$.

For events with three or more jets, a pseudo-dijet system is formed by
combining the jets in the event into two pseudo-jets. The total \Et
for each of the two pseudo-jets is calculated as the scalar sum of the
measured \Et of the contributing jets. The combination chosen is the
one that minimises the absolute \Et difference between the two
pseudo-jets, \dEt. This simple clustering criterion provides the best
separation between multijet events and events with genuine
\met. Eq.~\ref{eq:alphat} can therefore be generalised as:

\begin{equation}
  \label{eq:alphat2}
   \alpha_\textrm{T} = \frac{\sum_{i} E_\textrm{T}^{j_i} - \Delta E_\textrm{T}}{2\sqrt{\left(\sum_{i} E_\textrm{T}^{j_i}\right)^2 - {H_\textrm{T}^{\textrm{miss}}}^2}}
\end{equation}

% with the explicit bounds of the summations
%\begin{equation}
%  \label{eq:alphat2}
%   \alpha_\textrm{T} = \frac{\displaystyle\sum_{i=1}^{\njet} E_\textrm{T}^{j_i} -
%   \Delta E_\textrm{T}}
%   {\displaystyle 2\sqrt{\left(\sum_{i=1}^{\njet} E_\textrm{T}^{j_i}\right)^2 - {H_\textrm{T}^{\textrm{miss}}}^2}}
%\end{equation}

In the presence of jet energy mismeasurements, the direction and
magnitude of the apparent missing transverse energy, \mht, and energy
imbalance of the pseudo-dijet system, \dEt, are highly correlated.
This correlation is much weaker for R-parity-conserving SUSY with each
of the two decay chains producing the LSP.

It is useful to consider the limiting case $\dEt \rightarrow 0$ to
understand the relationship between \alphat and the ratio
$\mht/\sum_{i} E_\textrm{T}^{j_i}$:

\begin{equation}
  \label{eq:alphat3}
  \frac{\mht}{\sum_{i} E_\textrm{T}^{j_i}} \, = \, \sqrt{ 1 - \frac{1}{4 \cdot \alphat^2} }
\end{equation}

For reference, under the assumption of $\dEt = 0$, the values of
$\alphat = 0.65$, 0.60, 0.55, 0.53, and 0.52 map onto values of the ratio
$\mht/\sum_{i} E_\textrm{T}^{j_i} = 0.64$, 0.55, 0.42, 0.33, and 0.27, 
respectively.


%%____________________________________________________________________________||
