%%____________________________________________________________________________||

\section{History}
\label{sec:history}

This Analysis Note (AN-17-122) supersedes that with the identifier
AN-16-400, which is now deprecated. The changes to this AN for each
version are summarised below.

\subsection{December 2016 (v1)}

This version supports the Full Status report talk given on
2\superscript{nd} December 2016~\cite{fullstatus}. It is identical to
AN-16-400 v2 (now deprecated).

This version contains studies in the control regions based on
36\fbinv. Following the release of the preliminary public result for
ICHEP 2016~\cite{CMS-PAS-SUS-16-016}, the signal region was
reblinded. In time for the Full Status report, a partial unblinding of
the signal region data corresponding to 5\fbinv (from Run Era G) was
performed, which revealed an excess of events in the core of the
discriminating variables employed by this search.\footnote{The fit
  result shown in the Full Status report~\cite{fullstatus} was
  understood to be correctly executed and healthy. However, a mistake
  in the data cards for this fit was realised soon afterwards. Only
  then was the excess observed, which prompted the reblinding of the
  signal region and further detailed control-region-based studies.}
The data sets at the time were based on the \verb!Prompt!
reconstruction and the calibrations from the various POGs were still
considered very preliminary. Known issues with the data at the time,
such as the ``HIP effect'', were still under study by the
Collaboration.

\subsection{April 2017 (v2)}

This version contains several updates w.r.t. v1, as detailed
below. Some of the updates were planned, and stated at the time of, or
prior to, the Full Status talk (\ie they were still outstanding). Some
updates concern the addition of new control-region-based (cross check)
studies. Some updates represent modifications to the analysis methods,
as indicated below.
  
\begin{itemize}
  
\item (Planned) We have updated to now use the \verb!23Sep2016!
  \verb!ReReco! campaign, the \verb!re-miniAOD!, the
  \verb!Summer16!-campaign MC samples, and all the latest relevant
  recommended scale factors from the POGs (Secs~\ref{sec:},
  \ref{sec:}, and \ref{sec:}.)
    
\item (Planned) Several new studies have been performed. The most
  important one concerns the use of (missing) higher order corrections
  that modify the boson \Pt distributions obtained from simulation.

\item (New cross checks) New closure tests have been added, the
  correlation model for the systematic uncertainties derived from
  closure tests has been revisited, and a likelihood description of
  these closure tests has also been implemented to understand if any
  observed non-closure can be explained away by the known experimental
  or theoretical uncertainties included in our likelihood model.

\item (New cross checks) Fits to data across multiple control regions
  using a likelihood model as close as possible to that used to
  determine the final result have been studied, in order to
  identify/demonstrate a good understanding of the data in the control
  regions.

\item (Planned) The event selection criteria for the \gj sample has
  been relaxed (removed \alphat requirement, $\Delta R > 0.4$) and the
  treatment for estimating the sample contamination from fakes and
  fragmentation has been improved.

\item (New) We now rely solely on the \mmj sample to predict the
  \znunu\ + jets background. The \gj sample is now reserved for
  testing assumptions in our likelihood model (Sec.~\ref{sec:}).
  
\item (New) We have raised the \HTmiss threshold used on the signal
  and all control regions to $\HTmiss > 200\GeV$
  (Sec.~\ref{sec:trigger}).

\end{itemize}

%%____________________________________________________________________________||
