%%____________________________________________________________________________||

\section{History}
\label{sec:history}

This Analysis Note (AN-17-122) supersedes that with the identifier
AN-16-400, which is now deprecated. The changes to this AN for each
version are summarised below.

\subsection{December 2016 (v1)}

This version (v1) supports the Full Status report talk given on
2\superscript{nd} December 2016~\cite{fullstatus}. It is identical to
AN-16-400 v2 (now deprecated).

This version contains studies in the control regions based on
36\fbinv. Following the release of the preliminary public result for
ICHEP 2016~\cite{CMS-PAS-SUS-16-016}, the signal region was
reblinded. In time for the Full Status report, a partial unblinding of
the signal region data corresponding to 5\fbinv (from Run Era G) was
performed, which revealed an excess of events in the core of
distributions of the discriminating variables employed by this
search.\footnote{The fit result shown in the Full Status
  report~\cite{fullstatus} was understood to be correctly executed and
  healthy. However, a mistake in the data cards for this fit was
  realised soon afterwards. Only then was the excess observed, which
  prompted the reblinding of the signal region and further detailed
  studies of the control regions.}  The data sets at the time were
based on the \verb!Prompt!  reconstruction and the calibrations from
the various POGs were still considered very preliminary. Known issues
with the data at the time, such as the ``HIP effect'', were still
under study by the Collaboration.

\subsection{April 2017 (v2)}

This analysis note has received a substantial overhaul since v1. Both
the structure and content have significantly changed.  We have
reorganised the sections on the background estimation methods, and
tried to minimise the length of the body of the document, by providing
summary figures / tables, while much of the detailed figures / tables
have been moved to appendices. The updated content reflect changes to
the analysis, which are detailed below.

Some of the updates were planned and stated at the time of (or prior
to) the Full Status talk. Some updates concern the addition of new
cross check studies based on the control regions. Some updates
represent modifications to the analysis selections and/or
methods. These updates, labelled as ``Planned'', ``Checks'', or
``Mod'', are summarised below.
  
\begin{itemize}
  
\item (Planned) We have updated to now use the \verb!23Sep2016!
  \verb!ReReco! campaign, the \verb!re-miniAOD!, the
  \verb!Summer16!-campaign MC samples, and all the latest relevant
  recommended scale factors from the POGs. \eg see
  Sec.~\ref{sec:sim-corrs}.
  
\item (Mod) We have raised the \HTmiss threshold used for the signal
  and all control regions to $\HTmiss > 200\GeV$. See
  Sec.~\ref{sec:energysums}.
    
\item (Planned) Several new studies of systematics have been
  performed. The most important one concerns the use of (missing)
  higher order corrections that modify the boson \Pt distributions
  obtained from simulation. \eg see Secs.~\ref{sec:nlo-intro},
  \ref{sec:nlo}, and \ref{sec:nlo-zinv}.

\item (Checks) New closure tests have been added, and a likelihood
  model of these closure tests has also been implemented to understand
  if any observed non-closure can be explained away by known
  experimental or theoretical effects (such as jet energy scale
  corrections). In particular, the modelling of the \alphat and \bdphi
  variables has been studied in detail. \eg see
  Secs.~\ref{sec:tfSyst_alphaT} and \ref{sec:tfSyst_alphaT-zinv}.

\item (Mod) The correlation model for the systematic uncertainties
  derived from closure tests has been revisited. See
  Secs.~\ref{sec:closure-tests} and \ref{sec:closure-tests}.

%\item (Checks) Fits to data {\it across} multiple control regions
%  using a likelihood model have been studied, in order to demonstrate
%  a good modelling of the data in and across control regions. See
%  Sec.~\ref{sec:crfits}. \fixme{SEC NEEDS TO BE ADDED.}

\item (Mod) We now rely solely on the \mmj sample to predict the
  \znunu\ + jets background. Further, an extrapolation in \nb is
  performed as part of the \znunuj background estimate. See
  Sec.~\ref{sec:zinv}. The \gj sample is now reserved for testing
  assumptions in our likelihood model. See Sec.~\ref{app:gjets}.

\item (Planned) The event selection criteria for the \gj (cross check)
  sample has been relaxed (removed \alphat requirement, $\Delta R >
  0.4$) and the treatment for estimating the sample contamination from
  fakes and fragmentation has been improved. See App.~\ref{app:gjets}.

\end{itemize}

\subsection{June  2017 (v3)}

In addition to several minor tweaks, the following major changes have
been made:
\begin{itemize}
\item The use of the \mmj sample to predict the \znunuj background has
  been modified slightly. Secs~\ref{sec:event-categorisation} and
  \ref{sec:zinv} have been updated accordingly.
\item A brief description of corrections and uncertainties to the
  signal samples has been provided in Sec.~\ref{sec:signalcorrs}.
\item The multijet background estimation method is unchanged, but some
  figures have been updated, and others added to further support the
  method, in Sec.~\ref{sec:qcd}. 
\item The likelihood description has been added in
  Sec.~\ref{sec:likelihood}. 
\item The signal region has been fully unblinded using the 2016 data
  set and documented in Sec.~\ref{sec:results}. 
\item Interpretations with SUSY simplified models have been provided
  in Sec.~\ref{sec:susy}. 
\end{itemize}

%%____________________________________________________________________________||
