%%____________________________________________________________________________||
\clearpage
\section{``Lost lepton'' background estimation}
\label{app:ttw}

\subsection{The ``transfer factor'' method}

\begin{figure}[!h]
  \centering
  \subfigure[Transfer factors as a function of (\njet,\nb) and \scalht.]{
    \includegraphics[width=0.5\textwidth]{figures/mcSystematics36p4fb/plots/tf_mu_Ttw_2d_nominalUp.pdf}
  } \\
  \subfigure[Transfer factor as a function of \njet.]{
    \includegraphics[width=0.5\textwidth]{figures/mcSystematics36p4fb/plots/tf_mu_Ttw_njet_nominalUp.pdf}
  } ~
  \subfigure[Transfer factor as a function of \scalht.]{
    \includegraphics[width=0.5\textwidth]{figures/mcSystematics36p4fb/plots/tf_mu_Ttw_ht_nominalUp.pdf}
  } \\
  \subfigure[Transfer factor as a function of \nb.]{
    \includegraphics[width=0.5\textwidth]{figures/mcSystematics36p4fb/plots/tf_mu_Ttw_bjet_nominalUp.pdf}
  } ~
  \subfigure[Transfer factor as a function of \mht.]{
    \includegraphics[width=0.5\textwidth]{figures/mcSystematics36p4fb/plots/tf_mu_Ttw_mht_nominalUp.pdf}
  } \\
  \caption{\label{fig:tf_muToTtw} Transfer factors as a function of
    (\njet, \nb) event category and \scalht. Also shown are
    ``inclusive'' transfer factors as a function of \njet, \scalht,
    \nb, and \HTmiss. (Each ``inclusive'' dependence is shown when
    integrating over all other variables.) }
\end{figure}

\clearpage
\subsection{Minimum bias cross section / pileup}

\begin{figure}[!h]
  \centering
  \subfigure[Up variation versus (\njet,\nb) category and \scalht.]{
    \includegraphics[width=0.5\textwidth]{figures/mcSystematics36p4fb/plots/tfratio_mu_Ttw_2d_puWeightUp.pdf}
  } ~
  \subfigure[Down variation versus (\njet,\nb) category and \scalht.]{
    \includegraphics[width=0.5\textwidth]{figures/mcSystematics36p4fb/plots/tfratio_mu_Ttw_2d_puWeightDown.pdf}
  }\\
  \subfigure[Up/down variations versus \njet.]{
    \includegraphics[width=0.5\textwidth]{figures/mcSystematics36p4fb/plots/tfratio_mu_Ttw_njet_puWeightUp.pdf}
  } ~
  \subfigure[Up/down variations versus \scalht.]{
    \includegraphics[width=0.5\textwidth]{figures/mcSystematics36p4fb/plots/tfratio_mu_Ttw_ht_puWeightUp.pdf}
  } \\
  \subfigure[Up/down variations versus \nb.]{
    \includegraphics[width=0.5\textwidth]{figures/mcSystematics36p4fb/plots/tfratio_mu_Ttw_bjet_puWeightUp.pdf}
  } ~
  \subfigure[Up/down variations versus \mht.]{
    \includegraphics[width=0.5\textwidth]{figures/mcSystematics36p4fb/plots/tfratio_mu_Ttw_mht_puWeightUp.pdf}
  } \\
  \caption{\label{fig:tfSyst_pu_muToTtw} The relative change in the
    ``$\mj \rightarrow \mathrm{tt+W}$'' transfer factors from
    simulation due to $\pm1\sigma$ uncertainties in pileup.  }
\end{figure}

\clearpage
\subsection{Effect of scale and PDF on lepton acceptance}

\begin{figure}[!h]
  \centering
  \subfigure[Up variation versus (\njet,\nb) category and \scalht.]{
    \includegraphics[width=0.5\textwidth]{figures/mcSystematics36p4fb/plots/tfratio_mu_Ttw_2d_scaleWeightUp.pdf}
  } ~
  \subfigure[Down variation versus (\njet,\nb) category and \scalht.]{
    \includegraphics[width=0.5\textwidth]{figures/mcSystematics36p4fb/plots/tfratio_mu_Ttw_2d_scaleWeightDown.pdf}
  }\\
  \subfigure[Up/down variations versus \njet.]{
    \includegraphics[width=0.5\textwidth]{figures/mcSystematics36p4fb/plots/tfratio_mu_Ttw_njet_scaleWeightUp.pdf}
  } ~
  \subfigure[Up/down variations versus \scalht.]{
    \includegraphics[width=0.5\textwidth]{figures/mcSystematics36p4fb/plots/tfratio_mu_Ttw_ht_scaleWeightUp.pdf}
  } \\
  \subfigure[Up/down variations versus \nb.]{
    \includegraphics[width=0.5\textwidth]{figures/mcSystematics36p4fb/plots/tfratio_mu_Ttw_bjet_scaleWeightUp.pdf}
  } ~
  \subfigure[Up/down variations versus \mht.]{
    \includegraphics[width=0.5\textwidth]{figures/mcSystematics36p4fb/plots/tfratio_mu_Ttw_mht_scaleWeightUp.pdf}
  } \\
  \caption{\label{fig:tfSyst_scale_muToTtw} The relative change in the
    ``$\mj \rightarrow \mathrm{tt+W}$'' transfer factors from
    simulation due to $\pm1\sigma$ uncertainties in
    renormalisation/factorisation scales.  }
\end{figure}

\clearpage
\begin{figure}[!h]
  \centering
  \subfigure[Up variation versus (\njet,\nb) category and \scalht.]{
    \includegraphics[width=0.5\textwidth]{figures/mcSystematics36p4fb/plots/tfratio_mu_Ttw_2d_pdfWeightUp.pdf}
  } ~
  \subfigure[Down variation versus (\njet,\nb) category and \scalht.]{
    \includegraphics[width=0.5\textwidth]{figures/mcSystematics36p4fb/plots/tfratio_mu_Ttw_2d_pdfWeightDown.pdf}
  }\\
  \subfigure[Up/down variations versus \njet.]{
    \includegraphics[width=0.5\textwidth]{figures/mcSystematics36p4fb/plots/tfratio_mu_Ttw_njet_pdfWeightUp.pdf}
  } ~
  \subfigure[Up/down variations versus \scalht.]{
    \includegraphics[width=0.5\textwidth]{figures/mcSystematics36p4fb/plots/tfratio_mu_Ttw_ht_pdfWeightUp.pdf}
  } \\
  \subfigure[Up/down variations versus \nb.]{
    \includegraphics[width=0.5\textwidth]{figures/mcSystematics36p4fb/plots/tfratio_mu_Ttw_bjet_pdfWeightUp.pdf}
  } ~
  \subfigure[Up/down variations versus \mht.]{
    \includegraphics[width=0.5\textwidth]{figures/mcSystematics36p4fb/plots/tfratio_mu_Ttw_mht_pdfWeightUp.pdf}
  } \\
  \caption{\label{fig:tfSyst_pdf_muToTtw} The relative change in the
    ``$\mj \rightarrow \mathrm{tt+W}$'' transfer factors from
    simulation due to $\pm1\sigma$ uncertainties in parton density
    functions.  } 
\end{figure}

\clearpage
\subsection{\texorpdfstring{\wj}{W+jets} and \texorpdfstring{\ttbar}{TTbar} composition}

\begin{figure}[!h]
  \centering
  \subfigure[Up variation versus (\njet,\nb) category and \scalht.]{
    \includegraphics[width=0.5\textwidth]{figures/mcSystematics36p4fb/plots/tfratio_mu_Ttw_2d_xsWeightWUp.pdf}
  } ~
  \subfigure[Down variation versus (\njet,\nb) category and \scalht.]{
    \includegraphics[width=0.5\textwidth]{figures/mcSystematics36p4fb/plots/tfratio_mu_Ttw_2d_xsWeightWDown.pdf}
  }\\
  \subfigure[Up/down variations versus \njet.]{
    \includegraphics[width=0.5\textwidth]{figures/mcSystematics36p4fb/plots/tfratio_mu_Ttw_njet_xsWeightWUp.pdf}
  } ~
  \subfigure[Up/down variations versus \scalht.]{
    \includegraphics[width=0.5\textwidth]{figures/mcSystematics36p4fb/plots/tfratio_mu_Ttw_ht_xsWeightWUp.pdf}
  } \\
  \subfigure[Up/down variations versus \nb.]{
    \includegraphics[width=0.5\textwidth]{figures/mcSystematics36p4fb/plots/tfratio_mu_Ttw_bjet_xsWeightWUp.pdf}
  } ~
  \subfigure[Up/down variations versus \mht.]{
    \includegraphics[width=0.5\textwidth]{figures/mcSystematics36p4fb/plots/tfratio_mu_Ttw_mht_xsWeightWUp.pdf}
  } \\
  \caption{\label{fig:tfSyst_wjetxs_muToTtw} The relative change in the
    ``$\mj \rightarrow \mathrm{tt+W}$'' transfer factors from
    simulation due to $\pm1\sigma$ uncertainties in the \wj inclusive
    cross section.  } 
\end{figure}

\clearpage
\begin{figure}[!h]
  \centering
  \subfigure[Up variation versus (\njet,\nb) category and \scalht.]{
    \includegraphics[width=0.5\textwidth]{figures/mcSystematics36p4fb/plots/tfratio_mu_Ttw_2d_xsWeightTtUp.pdf}
  } ~
  \subfigure[Down variation versus (\njet,\nb) category and \scalht.]{
    \includegraphics[width=0.5\textwidth]{figures/mcSystematics36p4fb/plots/tfratio_mu_Ttw_2d_xsWeightTtDown.pdf}
  }\\
  \subfigure[Up/down variations versus \njet.]{
    \includegraphics[width=0.5\textwidth]{figures/mcSystematics36p4fb/plots/tfratio_mu_Ttw_njet_xsWeightTtUp.pdf}
  } ~
  \subfigure[Up/down variations versus \scalht.]{
    \includegraphics[width=0.5\textwidth]{figures/mcSystematics36p4fb/plots/tfratio_mu_Ttw_ht_xsWeightTtUp.pdf}
  } \\
  \subfigure[Up/down variations versus \nb.]{
    \includegraphics[width=0.5\textwidth]{figures/mcSystematics36p4fb/plots/tfratio_mu_Ttw_bjet_xsWeightTtUp.pdf}
  } ~
  \subfigure[Up/down variations versus \mht.]{
    \includegraphics[width=0.5\textwidth]{figures/mcSystematics36p4fb/plots/tfratio_mu_Ttw_mht_xsWeightTtUp.pdf}
  } \\
  \caption{\label{fig:tfSyst_ttbarxs_muToTtw} The relative change in the
    ``$\mj \rightarrow \mathrm{tt+W}$'' transfer factors from
    simulation due to $\pm1\sigma$ uncertainties in the \ttbar
    inclusive cross section.  }
\end{figure}

\clearpage
\subsection{Missing higher-order corrections in LO \texorpdfstring{\MADGRAPH}{MadGraph}
  samples}

\begin{figure}[!h]
  \centering
  \subfigure[Correction versus W boson \Pt.]{
    \includegraphics[width=0.4\textwidth,trim=0 150 0 150]{figures/NLO/W.pdf}
  } ~
  \subfigure[Variation versus (\njet,\nb) category and \scalht.]{
    \includegraphics[width=0.5\textwidth]{figures/mcSystematics36p4fb/plots/tfratio_mu_Ttw_2d_bosonPtWeightDown.pdf}
  }\\
  \subfigure[Variation versus \njet.]{
    \includegraphics[width=0.5\textwidth]{figures/mcSystematics36p4fb/plots/tfratio_mu_Ttw_njet_bosonPtWeightUp.pdf}
  } ~
  \subfigure[Variation versus \scalht.]{
    \includegraphics[width=0.5\textwidth]{figures/mcSystematics36p4fb/plots/tfratio_mu_Ttw_ht_bosonPtWeightUp.pdf}
  } \\
  \subfigure[Variation versus \nb.]{
    \includegraphics[width=0.5\textwidth]{figures/mcSystematics36p4fb/plots/tfratio_mu_Ttw_bjet_bosonPtWeightUp.pdf}
  } ~
  \subfigure[Variation versus \mht.]{
    \includegraphics[width=0.5\textwidth]{figures/mcSystematics36p4fb/plots/tfratio_mu_Ttw_mht_bosonPtWeightUp.pdf}
  } \\
  \caption{\label{fig:tfSyst_nlo_muToTtw} The relative change in the
    ``$\mj \rightarrow \mathrm{tt+W}$'' transfer factors from
    simulation when assuming uncertainties equal in magnitude to QCD +
    EWK NLO corrections versus W boson \Pt.  }
\end{figure}

\clearpage
\begin{figure}[!h]
  \centering
  \subfigure[\label{fig:NLO:mono-asymm}Monojet and asymmetric topologies]{
    \includegraphics[width=0.48\textwidth]{figures/NLO/monojet_no-fit_NLO.pdf}
  }~ 
  \subfigure[\label{fig:NLO:dijet}Dijet topologies]{
    \includegraphics[width=0.48\textwidth]{figures/NLO/di-jet_no-fit_NLO.pdf}
  }\\
  \subfigure[\label{fig:NLO:three-jet}Topologies with 3 jets]{
    \includegraphics[width=0.48\textwidth]{figures/NLO/3jet_no-fit_NLO.pdf}
  }~ 
  \subfigure[\label{fig:NLO:four-jet}Topologies with 4 jets]{
    \includegraphics[width=0.48\textwidth]{figures/NLO/4jet_no-fit_NLO.pdf}
  }\\
  \subfigure[\label{fig:NLO:five-jet}Topologies with 5 jets]{
    \includegraphics[width=0.48\textwidth]{figures/NLO/5jet_no-fit_NLO.pdf}
  }~ 
  \subfigure[\label{fig:NLO:six-jet}Topologies with $\ge$6 jets]{
    \includegraphics[width=0.48\textwidth]{figures/NLO/6+_jets_no-fit_NLO.pdf}
  }\\
  \caption{\label{fig:NLO_app} Size of the NLO corrections applied to data per MHT bin, in the signal region.
	}
\end{figure}

\clearpage
\subsection{\texorpdfstring{\njet}{Njet}-dependent (``\texorpdfstring{\nisr}{Nisr}'') corrections for \texorpdfstring{\ttbar}{TTbar}}

\begin{figure}[!h]
  \centering
  \subfigure[Up variation versus (\njet,\nb) category and \scalht.]{
    \includegraphics[width=0.5\textwidth]{figures/mcSystematics36p4fb/plots/tfratio_mu_Ttw_2d_nIsrWeightUp.pdf}
  } ~
  \subfigure[Down variation versus (\njet,\nb) category and \scalht.]{
    \includegraphics[width=0.5\textwidth]{figures/mcSystematics36p4fb/plots/tfratio_mu_Ttw_2d_nIsrWeightDown.pdf}
  }\\
  \subfigure[Up/down variations versus \njet.]{
    \includegraphics[width=0.5\textwidth]{figures/mcSystematics36p4fb/plots/tfratio_mu_Ttw_njet_nIsrWeightUp.pdf}
  } ~
  \subfigure[Up/down variations versus \scalht.]{
    \includegraphics[width=0.5\textwidth]{figures/mcSystematics36p4fb/plots/tfratio_mu_Ttw_ht_nIsrWeightUp.pdf}
  } \\
  \subfigure[Up/down variations versus \nb.]{
    \includegraphics[width=0.5\textwidth]{figures/mcSystematics36p4fb/plots/tfratio_mu_Ttw_bjet_nIsrWeightUp.pdf}
  } ~
  \subfigure[Up/down variations versus \mht.]{
    \includegraphics[width=0.5\textwidth]{figures/mcSystematics36p4fb/plots/tfratio_mu_Ttw_mht_nIsrWeightUp.pdf}
  } \\
  \caption{\label{fig:tfSyst_nisr_muToTtw} The relative change in the
    ``$\mj \rightarrow \mathrm{tt+W}$'' transfer factors from
    simulation due to $\pm0.5\sigma$ uncertainties in \nisr
    corrections for \ttbar. }
\end{figure}

\clearpage
\subsection{Signal trigger efficiency}

\begin{figure}[!h]
  \centering
  \subfigure[Up variation versus (\njet,\nb) category and \scalht.]{
    \includegraphics[width=0.5\textwidth]{figures/mcSystematics36p4fb/plots/tfratio_mu_Ttw_2d_triggerWeightUp.pdf}
  } ~
  \subfigure[Down variation versus (\njet,\nb) category and \scalht.]{
    \includegraphics[width=0.5\textwidth]{figures/mcSystematics36p4fb/plots/tfratio_mu_Ttw_2d_triggerWeightDown.pdf}
  }\\
  \subfigure[Up/down variations versus \njet.]{
    \includegraphics[width=0.5\textwidth]{figures/mcSystematics36p4fb/plots/tfratio_mu_Ttw_njet_triggerWeightUp.pdf}
  } ~
  \subfigure[Up/down variations versus \scalht.]{
    \includegraphics[width=0.5\textwidth]{figures/mcSystematics36p4fb/plots/tfratio_mu_Ttw_ht_triggerWeightUp.pdf}
  } \\
  \subfigure[Up/down variations versus \nb.]{
    \includegraphics[width=0.5\textwidth]{figures/mcSystematics36p4fb/plots/tfratio_mu_Ttw_bjet_triggerWeightUp.pdf}
  } ~
  \subfigure[Up/down variations versus \mht.]{
    \includegraphics[width=0.5\textwidth]{figures/mcSystematics36p4fb/plots/tfratio_mu_Ttw_mht_triggerWeightUp.pdf}
  } \\
  \caption{\label{fig:tfSyst_trigger_muToTtw} The relative change in the
    ``$\mj \rightarrow \mathrm{tt+W}$'' transfer factors from
    simulation due to $\pm1\sigma$ uncertainties in the signal trigger
    efficiencies.  }
\end{figure}

\clearpage
\subsection{Lepton trigger / identification / isolation efficiencies}

\begin{figure}[!h]
  \centering
  \subfigure[Up variation versus (\njet,\nb) category and \scalht.]{
    \includegraphics[width=0.5\textwidth]{figures/mcSystematics36p4fb/plots/tfratio_mu_Ttw_2d_muonSfWeightUp.pdf}
  } ~
  \subfigure[Down variation versus (\njet,\nb) category and \scalht.]{
    \includegraphics[width=0.5\textwidth]{figures/mcSystematics36p4fb/plots/tfratio_mu_Ttw_2d_muonSfWeightDown.pdf}
  }\\
  \subfigure[Up/down variations versus \njet.]{
    \includegraphics[width=0.5\textwidth]{figures/mcSystematics36p4fb/plots/tfratio_mu_Ttw_njet_muonSfWeightUp.pdf}
  } ~
  \subfigure[Up/down variations versus \scalht.]{
    \includegraphics[width=0.5\textwidth]{figures/mcSystematics36p4fb/plots/tfratio_mu_Ttw_ht_muonSfWeightUp.pdf}
  } \\
  \subfigure[Up/down variations versus \nb.]{
    \includegraphics[width=0.5\textwidth]{figures/mcSystematics36p4fb/plots/tfratio_mu_Ttw_bjet_muonSfWeightUp.pdf}
  } ~
  \subfigure[Up/down variations versus \mht.]{
    \includegraphics[width=0.5\textwidth]{figures/mcSystematics36p4fb/plots/tfratio_mu_Ttw_mht_muonSfWeightUp.pdf}
  } \\
  \caption{\label{fig:tfSyst_muonsf_muToTtw} The relative change in the
    ``$\mj \rightarrow \mathrm{tt+W}$'' transfer factors from
    simulation due to $\pm1\sigma$ uncertainties in efficiencies
    related to muon trigger, identification, and isolation.  }
\end{figure}

\clearpage
\begin{figure}[!h]
  \centering
  \subfigure[Up variation versus (\njet,\nb) category and \scalht.]{
    \includegraphics[width=0.5\textwidth]{figures/mcSystematics36p4fb/MuonSF_Numer_up.pdf}
  } \\
%  \subfigure[Up variation versus (\njet,\nb) category and \scalht.]{
%    \includegraphics[width=0.5\textwidth]{figures/mcSystematics36p4fb/plots/tfratio_mu_Ttw_2d_leptonVetoSystFullSimWeightUp.pdf}
%  } ~
%  \subfigure[Down variation versus (\njet,\nb) category and \scalht.]{
%    \includegraphics[width=0.5\textwidth]{figures/mcSystematics36p4fb/plots/tfratio_mu_Ttw_2d_leptonVetoSystFullSimWeightDown.pdf}
%  }\\
%  \subfigure[Up/down variations versus \njet.]{
%    \includegraphics[width=0.5\textwidth]{figures/mcSystematics36p4fb/plots/tfratio_mu_Ttw_njet_leptonVetoSystFullSimWeightUp.pdf}
%  } ~
%  \subfigure[Up/down variations versus \scalht.]{
%    \includegraphics[width=0.5\textwidth]{figures/mcSystematics36p4fb/plots/tfratio_mu_Ttw_ht_leptonVetoSystFullSimWeightUp.pdf}
%  } \\
%  \subfigure[Up/down variations versus \nb.]{
%    \includegraphics[width=0.5\textwidth]{figures/mcSystematics36p4fb/plots/tfratio_mu_Ttw_bjet_leptonVetoSystFullSimWeightUp.pdf}
%  } ~
%  \subfigure[Up/down variations versus \mht.]{
%    \includegraphics[width=0.5\textwidth]{figures/mcSystematics36p4fb/plots/tfratio_mu_Ttw_mht_leptonVetoSystFullSimWeightUp.pdf}
%  } \\
  \caption{\label{fig:tfSyst_leptonveto_muToTtw} The relative change in the
    ``$\mj \rightarrow \mathrm{tt+W}$'' transfer factors from
    simulation due to $\pm1\sigma$ uncertainties in the efficiency
    with which leptons are vetoed for the signal region.  }
\end{figure}

\clearpage
\subsection{Jet energy scale}

\begin{figure}[!h]
  \centering
  \subfigure[Up variation versus (\njet,\nb) category and \scalht.]{
    \includegraphics[width=0.5\textwidth]{figures/mcSystematics36p4fb/plots/tfratio_mu_Ttw_2d_jecUp.pdf}
  } ~
  \subfigure[Down variation versus (\njet,\nb) category and \scalht.]{
    \includegraphics[width=0.5\textwidth]{figures/mcSystematics36p4fb/plots/tfratio_mu_Ttw_2d_jecDown.pdf}
  }\\
  \subfigure[Up/down variations versus \njet.]{
    \includegraphics[width=0.5\textwidth]{figures/mcSystematics36p4fb/plots/tfratio_mu_Ttw_njet_jecUp.pdf}
  } ~
  \subfigure[Up/down variations versus \scalht.]{
    \includegraphics[width=0.5\textwidth]{figures/mcSystematics36p4fb/plots/tfratio_mu_Ttw_ht_jecUp.pdf}
  } \\
  \subfigure[Up/down variations versus \nb.]{
    \includegraphics[width=0.5\textwidth]{figures/mcSystematics36p4fb/plots/tfratio_mu_Ttw_bjet_jecUp.pdf}
  } ~
  \subfigure[Up/down variations versus \mht.]{
    \includegraphics[width=0.5\textwidth]{figures/mcSystematics36p4fb/plots/tfratio_mu_Ttw_mht_jecUp.pdf}
  } \\
  \caption{\label{fig:tfSyst_jec_muToTtw} The relative change in the
    ``$\mj \rightarrow \mathrm{tt+W}$'' transfer factors from
    simulation due to $\pm1\sigma$ uncertainties in corrections to the
    jet energy scale.  }
\end{figure}

\clearpage
\subsection{B-tagging efficiency and mistag probability}

\begin{figure}[!h]
  \centering
  \subfigure[Up variation versus (\njet,\nb) category and \scalht.]{
    \includegraphics[width=0.5\textwidth]{figures/mcSystematics36p4fb/plots/tfratio_mu_Ttw_2d_bsfWeightUp.pdf}
  } ~
  \subfigure[Down variation versus (\njet,\nb) category and \scalht.]{
    \includegraphics[width=0.5\textwidth]{figures/mcSystematics36p4fb/plots/tfratio_mu_Ttw_2d_bsfWeightDown.pdf}
  }\\
  \subfigure[Up/down variations versus \njet.]{
    \includegraphics[width=0.5\textwidth]{figures/mcSystematics36p4fb/plots/tfratio_mu_Ttw_njet_bsfWeightUp.pdf}
  } ~
  \subfigure[Up/down variations versus \scalht.]{
    \includegraphics[width=0.5\textwidth]{figures/mcSystematics36p4fb/plots/tfratio_mu_Ttw_ht_bsfWeightUp.pdf}
  } \\
  \subfigure[Up/down variations versus \nb.]{
    \includegraphics[width=0.5\textwidth]{figures/mcSystematics36p4fb/plots/tfratio_mu_Ttw_bjet_bsfWeightUp.pdf}
  } ~
  \subfigure[Up/down variations versus \mht.]{
    \includegraphics[width=0.5\textwidth]{figures/mcSystematics36p4fb/plots/tfratio_mu_Ttw_mht_bsfWeightUp.pdf}
  } \\
  \caption{\label{fig:tfSyst_bsf_muToTtw} The relative change in the
    ``$\mj \rightarrow \mathrm{tt+W}$'' transfer factors from
    simulation due to $\pm1\sigma$ uncertainties in the data/MC scale
    factors associated with b-quark tag efficiency.  }
\end{figure}

\clearpage
\begin{figure}[!h]
  \centering
  \subfigure[Up variation versus (\njet,\nb) category and \scalht.]{
    \includegraphics[width=0.5\textwidth]{figures/mcSystematics36p4fb/plots/tfratio_mu_Ttw_2d_bsfLightWeightUp.pdf}
  } ~
  \subfigure[Down variation versus (\njet,\nb) category and \scalht.]{
    \includegraphics[width=0.5\textwidth]{figures/mcSystematics36p4fb/plots/tfratio_mu_Ttw_2d_bsfLightWeightDown.pdf}
  }\\
  \subfigure[Up/down variations versus \njet.]{
    \includegraphics[width=0.5\textwidth]{figures/mcSystematics36p4fb/plots/tfratio_mu_Ttw_njet_bsfLightWeightUp.pdf}
  } ~
  \subfigure[Up/down variations versus \scalht.]{
    \includegraphics[width=0.5\textwidth]{figures/mcSystematics36p4fb/plots/tfratio_mu_Ttw_ht_bsfLightWeightUp.pdf}
  } \\
  \subfigure[Up/down variations versus \nb.]{
    \includegraphics[width=0.5\textwidth]{figures/mcSystematics36p4fb/plots/tfratio_mu_Ttw_bjet_bsfLightWeightUp.pdf}
  } ~
  \subfigure[Up/down variations versus \mht.]{
    \includegraphics[width=0.5\textwidth]{figures/mcSystematics36p4fb/plots/tfratio_mu_Ttw_mht_bsfLightWeightUp.pdf}
  } \\
  \caption{\label{fig:tfSyst_bsfl_muToTtw} The relative change in the
    ``$\mj \rightarrow \mathrm{tt+W}$'' transfer factors from
    simulation due to $\pm1\sigma$ uncertainties in the data/MC scale
    factors associated with b-quark mistag probabilities.  }
\end{figure}

\clearpage
\subsection{Extrapolation in \texorpdfstring{\alphat}{AlphaT}}

\begin{figure}[h!]
  \begin{center}
    \includegraphics[width=0.45\textwidth]{figures/closureTests/AlphaT/SingleMu_alphaTExtrapolation_ht.pdf}~
    \includegraphics[width=0.45\textwidth]{figures/closureTests/AlphaT/SingleMu_alphaTExtrapolation_nJet.pdf}\\
    \caption{Data-driven closure tests probing the modelling of an
      extrapolation in the \alphat variable with the \mj sample. The
      level of closure (solid markers) is indicated as a function of
      \scalht (left) and \njet (right). The blue histogram indicates
      the quadrature sum of the magnitude of non-closure and its
      statistical uncertainty. The post-fit closure (open markers) is
      also shown (see text for details).  }
    \label{fig:closure_AlphaT_mu}
  \end{center} 
\end{figure}

\begin{figure}[h!]
  \begin{center}
    \includegraphics[width=0.45\textwidth]{figures/closureTests/AlphaT/AlphaT_Correlated_nuisances.pdf}
    \caption{The post-fit nuisance parameter values (relative to
      pre-fit) for the \alphat closure test when implemented as a
      binned likelihood fit.} 
    \label{fig:closure_AlphaT_LH_mu}
  \end{center} 
\end{figure}

\clearpage
\subsection{Extrapolation in \texorpdfstring{\bdphi}{biased dPhi}}

\begin{figure}[h!]
  \begin{center}
    \includegraphics[width=0.45\textwidth]{figures/closureTests/bDPhi/SingleMu_bdphiExtrapolation_ht.pdf}~
    \includegraphics[width=0.45\textwidth]{figures/closureTests/bDPhi/SingleMu_bdphiExtrapolation_nJet.pdf}\\
    \caption{Data-driven closure tests probing the modelling of an
      extrapolation in the \bdphi variable with the \mj sample. The
      level of closure (solid markers) is indicated as a function of
      \scalht (left) and \njet (right). The blue histogram indicates
      the quadrature sum of the magnitude of non-closure and its
      statistical uncertainty. The post-fit closure (open markers) is
      also shown (see text for details).  }
    \label{fig:closure_bDPhi_mu}
  \end{center} 
\end{figure}

\clearpage
\subsection{Extrapolation in \texorpdfstring{\alphat}{AlphaT} and
  \texorpdfstring{\bdphi}{biased dPhi}}

\begin{figure}[h!]
  \begin{center}
    \includegraphics[width=0.45\textwidth]{figures/closureTests/AlphaT_bDPhi/SingleMu_alphaTbdphi_ht.pdf}~
    \includegraphics[width=0.45\textwidth]{figures/closureTests/AlphaT_bDPhi/SingleMu_alphaTbdphi_nJet.pdf}\\
    \caption{Data-driven closure tests probing the modelling of an
      extrapolation in both \alphat and \bdphi variables, as done in
      the analysis, with the \mj sample. The level of closure (solid
      markers) is indicated as a function of \scalht (left) and \njet
      (right). The blue histogram indicates the quadrature sum of the
      magnitude of non-closure and its statistical uncertainty. The
      post-fit closure (open markers) is also shown (see text for
      details). \fixme{UPDATE PLOTS WITH CLOSURE FIT!} }
    \label{fig:closure_AlphaT_bDPhi_mu}
  \end{center} 
\end{figure}

\begin{figure}[h!]
  \begin{center}
    \includegraphics[width=0.45\textwidth]{figures/closureTests/AlphaT/AlphaT_Correlated_nuisances.pdf}
    \caption{The post-fit nuisance parameter values (relative to
      pre-fit) for the combined \alphat and \bdphi closure test when
      implemented as a binned likelihood fit. \fixme{UPDATE PLOT!!!}} 
    \label{fig:closure_AlphaT_bDPhi_LH_mu}
  \end{center} 
\end{figure}

\clearpage
\subsection{The effect of W polarisation on lepton acceptance}

\begin{figure}[h!]
  \begin{center}
    \includegraphics[width=0.45\textwidth]{figures/closureTests/WPol/SingleMuPlus_to_SingleMuMinus_ht.pdf}~
    \includegraphics[width=0.45\textwidth]{figures/closureTests/WPol/SingleMuPlus_to_SingleMuMinus_nJet.pdf}\\
    \caption{Data-driven closure tests that probe the modelling of W
      polarisation in the \mj sample. The level of closure (solid
      markers) is indicated as a function of \scalht (left) and \njet
      (right). The blue histogram indicates the quadrature sum of the
      magnitude of non-closure and its statistical uncertainty. }
    \label{fig:closure_WPol_mu}
  \end{center} 
\end{figure}

\clearpage
\subsection{The single isolated track veto}

\begin{figure}[h!]
  \begin{center}
    \includegraphics[width=0.45\textwidth]{figures/closureTests/SITV/SingleMu_Sit_ht.pdf}~
    \includegraphics[width=0.45\textwidth]{figures/closureTests/SITV/SingleMu_Sit_nJet.pdf}\\
    \caption{Data-driven closure tests that probe the modelling of the
      single isolated track veto. The level of closure (solid markers)
      is indicated as a function of \scalht (left) and \njet
      (right). The blue histogram indicates the quadrature sum of the
      magnitude of non-closure and its statistical uncertainty. }
    \label{fig:closure_SITV_mu}
  \end{center} 
\end{figure}

\clearpage
\begin{figure}[h!]
  \begin{center}
    \includegraphics[width=0.3\textwidth,page=10,trim=0 100 50 100,clip]{figures/SITV/SIT/SIT.pdf}~
    \includegraphics[width=0.3\textwidth,page=6,trim=0 100 50 100,clip]{figures/SITV/SIT/SIT.pdf}~
    \includegraphics[width=0.3\textwidth,page=5,trim=0 100 50 100,clip]{figures/SITV/SIT/SIT.pdf}\\
    \includegraphics[width=0.3\textwidth,page=8,trim=0 100 50 100,clip]{figures/SITV/SIT/SIT.pdf}~
    \includegraphics[width=0.3\textwidth,page=4,trim=0 100 50 100,clip]{figures/SITV/SIT/SIT.pdf}~
    \includegraphics[width=0.3\textwidth,page=7,trim=0 100 50 100,clip]{figures/SITV/SIT/SIT.pdf}\\
    \includegraphics[width=0.3\textwidth,page=2,trim=0 100 50 100,clip]{figures/SITV/SIT/SIT.pdf}~
    \includegraphics[width=0.3\textwidth,page=1,trim=0 100 50 100,clip]{figures/SITV/SIT/SIT.pdf}~
    \includegraphics[width=0.3\textwidth,page=9,trim=0 100 50 100,clip]{figures/SITV/SIT/SIT.pdf}\\
    \includegraphics[width=0.3\textwidth,page=3,trim=0 100 50 100,clip]{figures/SITV/SIT/SIT.pdf} 
    \caption{Distributions of various quantities related to SITs
      within a sample of \mj events containing at least SIT.}
    \label{fig:dataMC_SIT_mu}
  \end{center} 
\end{figure}

\clearpage
\begin{figure}[h!]
  \begin{center}
    \includegraphics[width=0.3\textwidth,page=18,trim=0 100 50 100,clip]{figures/SITV/Event/Event.pdf}~
    \includegraphics[width=0.3\textwidth,page=17,trim=0 100 50 100,clip]{figures/SITV/Event/Event.pdf}~
    \includegraphics[width=0.3\textwidth,page=13,trim=0 100 50 100,clip]{figures/SITV/Event/Event.pdf}\\
    \includegraphics[width=0.3\textwidth,page=3,trim=0 100 50 100,clip]{figures/SITV/Event/Event.pdf}~
    \includegraphics[width=0.3\textwidth,page=1,trim=0 100 50 100,clip]{figures/SITV/Event/Event.pdf}~
    \includegraphics[width=0.3\textwidth,page=2,trim=0 100 50 100,clip]{figures/SITV/Event/Event.pdf}\\
    \includegraphics[width=0.3\textwidth,page=11,trim=0 100 50 100,clip]{figures/SITV/Event/Event.pdf}~
    \includegraphics[width=0.3\textwidth,page=9,trim=0 100 50 100,clip]{figures/SITV/Event/Event.pdf}~
    \includegraphics[width=0.3\textwidth,page=12,trim=0 100 50 100,clip]{figures/SITV/Event/Event.pdf}\\
    \caption{Distributions of various event-level quantities within a
      sample of \mj events containing at least SIT.}
    \label{fig:dataMC_SITEvent_mu}
  \end{center} 
\end{figure}

\clearpage
\subsection{Systematics uncertainties in the \texorpdfstring{\HTmiss}{MHT} templates}
\label{app:mht}


\begin{figure}[h!]
  \begin{center}
    \subfigure[$600 < \scalht < 750\GeV$]{\includegraphics[width=0.32\textwidth]{figures/mhtTemplate/exclusive/fits/Mu/ht600_750_control_eq0b_eq1j_Mu.pdf}}
    \subfigure[$750 < \scalht < 900\GeV$]{\includegraphics[width=0.32\textwidth]{figures/mhtTemplate/exclusive/fits/Mu/ht750_900_control_eq0b_eq1j_Mu.pdf}}
    \subfigure[$\scalht > 900\GeV$]{\includegraphics[width=0.32\textwidth]{figures/mhtTemplate/exclusive/fits/Mu/ht900_Inf_control_eq0b_eq1j_Mu.pdf}}\\
    \caption{The ratio of event yields obtained from data and simulation as a function of \mht [GeV] based on a sample of \mj events that satisfy $\njet = 1$ and $\nb = 0$, as well as the requirements on \scalht indicated by each sub-figure caption. Also shown are fits to the ratios using first-order orthogonal polynomials.}
    \label{fig:mhtval_Mu_eq1j_eq0b}
  \end{center}
\end{figure}

\begin{figure}[h!]
  \begin{center}
    \subfigure[$400 < \scalht < 500\GeV$]{\includegraphics[width=0.32\textwidth]{figures/mhtTemplate/exclusive/fits/Mu/ht400_500_control_eq0b_eq2j_Mu.pdf}}
    \subfigure[$500 < \scalht < 600\GeV$]{\includegraphics[width=0.32\textwidth]{figures/mhtTemplate/exclusive/fits/Mu/ht500_600_control_eq0b_eq2j_Mu.pdf}}
    \subfigure[$600 < \scalht < 750\GeV$]{\includegraphics[width=0.32\textwidth]{figures/mhtTemplate/exclusive/fits/Mu/ht600_750_control_eq0b_eq2j_Mu.pdf}}\\
    \subfigure[$750 < \scalht < 900\GeV$]{\includegraphics[width=0.32\textwidth]{figures/mhtTemplate/exclusive/fits/Mu/ht750_900_control_eq0b_eq2j_Mu.pdf}}
    \subfigure[$900 < \scalht < 1050\GeV$]{\includegraphics[width=0.32\textwidth]{figures/mhtTemplate/exclusive/fits/Mu/ht900_1050_control_eq0b_eq2j_Mu.pdf}}
    \subfigure[$1050 < \scalht < 1200\GeV$]{\includegraphics[width=0.32\textwidth]{figures/mhtTemplate/exclusive/fits/Mu/ht1050_1200_control_eq0b_eq2j_Mu.pdf}}\\
    \subfigure[$\scalht > 1200\GeV$]{\includegraphics[width=0.32\textwidth]{figures/mhtTemplate/exclusive/fits/Mu/ht1200_Inf_control_eq0b_eq2j_Mu.pdf}}
    \caption{The ratio of event yields obtained from data and simulation as a function of \mht [GeV] based on a sample of \mj events that satisfy $\njet = 2$ and $\nb = 0$, as well as the requirements on \scalht indicated by each sub-figure caption. Also shown are fits to the ratios using first-order orthogonal polynomials.}
    \label{fig:mhtval_Mu_eq2j_eq0b}
  \end{center}
\end{figure}

\begin{figure}[h!]
  \begin{center}
    \subfigure[$400 < \scalht < 500\GeV$]{\includegraphics[width=0.32\textwidth]{figures/mhtTemplate/exclusive/fits/Mu/ht400_500_control_eq1b_eq2j_Mu.pdf}}
    \subfigure[$500 < \scalht < 600\GeV$]{\includegraphics[width=0.32\textwidth]{figures/mhtTemplate/exclusive/fits/Mu/ht500_600_control_eq1b_eq2j_Mu.pdf}}
    \subfigure[$600 < \scalht < 750\GeV$]{\includegraphics[width=0.32\textwidth]{figures/mhtTemplate/exclusive/fits/Mu/ht600_750_control_eq1b_eq2j_Mu.pdf}}\\
    \subfigure[$750 < \scalht < 900\GeV$]{\includegraphics[width=0.32\textwidth]{figures/mhtTemplate/exclusive/fits/Mu/ht750_900_control_eq1b_eq2j_Mu.pdf}}
    \subfigure[$900 < \scalht < 1050\GeV$]{\includegraphics[width=0.32\textwidth]{figures/mhtTemplate/exclusive/fits/Mu/ht900_1050_control_eq1b_eq2j_Mu.pdf}}
    \subfigure[$1050 < \scalht < 1200\GeV$]{\includegraphics[width=0.32\textwidth]{figures/mhtTemplate/exclusive/fits/Mu/ht1050_1200_control_eq1b_eq2j_Mu.pdf}}\\
    \subfigure[$\scalht > 1200\GeV$]{\includegraphics[width=0.32\textwidth]{figures/mhtTemplate/exclusive/fits/Mu/ht1200_Inf_control_eq1b_eq2j_Mu.pdf}}
    \caption{The ratio of event yields obtained from data and simulation as a function of \mht [GeV] based on a sample of \mj events that satisfy $\njet = 2$ and $\nb = 1$, as well as the requirements on \scalht indicated by each sub-figure caption. Also shown are fits to the ratios using first-order orthogonal polynomials.}
    \label{fig:mhtval_Mu_eq2j_eq1b}
  \end{center}
\end{figure}

\begin{figure}[h!]
  \begin{center}
    \subfigure[$400 < \scalht < 500\GeV$]{\includegraphics[width=0.32\textwidth]{figures/mhtTemplate/exclusive/fits/Mu/ht400_500_control_eq2b_eq2j_Mu.pdf}}
    \subfigure[$600 < \scalht < 750\GeV$]{\includegraphics[width=0.32\textwidth]{figures/mhtTemplate/exclusive/fits/Mu/ht600_750_control_eq2b_eq2j_Mu.pdf}}
    \caption{The ratio of event yields obtained from data and simulation as a function of \mht [GeV] based on a sample of \mj events that satisfy $\njet = 2$ and $\nb = 2$, as well as the requirements on \scalht indicated by each sub-figure caption. Also shown are fits to the ratios using first-order orthogonal polynomials.}
    \label{fig:mhtval_Mu_eq2j_eq2b}
  \end{center}
\end{figure}

\begin{figure}[h!]
  \begin{center}
    \subfigure[$400 < \scalht < 500\GeV$]{\includegraphics[width=0.32\textwidth]{figures/mhtTemplate/exclusive/fits/Mu/ht400_500_control_eq0b_eq3j_Mu.pdf}}
    \subfigure[$500 < \scalht < 600\GeV$]{\includegraphics[width=0.32\textwidth]{figures/mhtTemplate/exclusive/fits/Mu/ht500_600_control_eq0b_eq3j_Mu.pdf}}
    \subfigure[$600 < \scalht < 750\GeV$]{\includegraphics[width=0.32\textwidth]{figures/mhtTemplate/exclusive/fits/Mu/ht600_750_control_eq0b_eq3j_Mu.pdf}}\\
    \subfigure[$750 < \scalht < 900\GeV$]{\includegraphics[width=0.32\textwidth]{figures/mhtTemplate/exclusive/fits/Mu/ht750_900_control_eq0b_eq3j_Mu.pdf}}
    \subfigure[$900 < \scalht < 1050\GeV$]{\includegraphics[width=0.32\textwidth]{figures/mhtTemplate/exclusive/fits/Mu/ht900_1050_control_eq0b_eq3j_Mu.pdf}}
    \subfigure[$1050 < \scalht < 1200\GeV$]{\includegraphics[width=0.32\textwidth]{figures/mhtTemplate/exclusive/fits/Mu/ht1050_1200_control_eq0b_eq3j_Mu.pdf}}\\
    \subfigure[$\scalht > 1200\GeV$]{\includegraphics[width=0.32\textwidth]{figures/mhtTemplate/exclusive/fits/Mu/ht1200_Inf_control_eq0b_eq3j_Mu.pdf}}
    \caption{The ratio of event yields obtained from data and simulation as a function of \mht [GeV] based on a sample of \mj events that satisfy $\njet = 3$ and $\nb = 0$, as well as the requirements on \scalht indicated by each sub-figure caption. Also shown are fits to the ratios using first-order orthogonal polynomials.}
    \label{fig:mhtval_Mu_eq3j_eq0b}
  \end{center}
\end{figure}

\begin{figure}[h!]
  \begin{center}
    \subfigure[$400 < \scalht < 500\GeV$]{\includegraphics[width=0.32\textwidth]{figures/mhtTemplate/exclusive/fits/Mu/ht400_500_control_eq1b_eq3j_Mu.pdf}}
    \subfigure[$500 < \scalht < 600\GeV$]{\includegraphics[width=0.32\textwidth]{figures/mhtTemplate/exclusive/fits/Mu/ht500_600_control_eq1b_eq3j_Mu.pdf}}
    \subfigure[$600 < \scalht < 750\GeV$]{\includegraphics[width=0.32\textwidth]{figures/mhtTemplate/exclusive/fits/Mu/ht600_750_control_eq1b_eq3j_Mu.pdf}}\\
    \subfigure[$750 < \scalht < 900\GeV$]{\includegraphics[width=0.32\textwidth]{figures/mhtTemplate/exclusive/fits/Mu/ht750_900_control_eq1b_eq3j_Mu.pdf}}
    \subfigure[$900 < \scalht < 1050\GeV$]{\includegraphics[width=0.32\textwidth]{figures/mhtTemplate/exclusive/fits/Mu/ht900_1050_control_eq1b_eq3j_Mu.pdf}}
    \subfigure[$1050 < \scalht < 1200\GeV$]{\includegraphics[width=0.32\textwidth]{figures/mhtTemplate/exclusive/fits/Mu/ht1050_1200_control_eq1b_eq3j_Mu.pdf}}\\
    \subfigure[$\scalht > 1200\GeV$]{\includegraphics[width=0.32\textwidth]{figures/mhtTemplate/exclusive/fits/Mu/ht1200_Inf_control_eq1b_eq3j_Mu.pdf}}
    \caption{The ratio of event yields obtained from data and simulation as a function of \mht [GeV] based on a sample of \mj events that satisfy $\njet = 3$ and $\nb = 1$, as well as the requirements on \scalht indicated by each sub-figure caption. Also shown are fits to the ratios using first-order orthogonal polynomials.}
    \label{fig:mhtval_Mu_eq3j_eq1b}
  \end{center}
\end{figure}

\begin{figure}[h!]
  \begin{center}
    \subfigure[$400 < \scalht < 500\GeV$]{\includegraphics[width=0.32\textwidth]{figures/mhtTemplate/exclusive/fits/Mu/ht400_500_control_eq2b_eq3j_Mu.pdf}}
    \subfigure[$500 < \scalht < 600\GeV$]{\includegraphics[width=0.32\textwidth]{figures/mhtTemplate/exclusive/fits/Mu/ht500_600_control_eq2b_eq3j_Mu.pdf}}
    \subfigure[$600 < \scalht < 750\GeV$]{\includegraphics[width=0.32\textwidth]{figures/mhtTemplate/exclusive/fits/Mu/ht600_750_control_eq2b_eq3j_Mu.pdf}}\\
    \subfigure[$750 < \scalht < 900\GeV$]{\includegraphics[width=0.32\textwidth]{figures/mhtTemplate/exclusive/fits/Mu/ht750_900_control_eq2b_eq3j_Mu.pdf}}
    \caption{The ratio of event yields obtained from data and simulation as a function of \mht [GeV] based on a sample of \mj events that satisfy $\njet = 3$ and $\nb = 2$, as well as the requirements on \scalht indicated by each sub-figure caption. Also shown are fits to the ratios using first-order orthogonal polynomials.}
    \label{fig:mhtval_Mu_eq3j_eq2b}
  \end{center}
\end{figure}

\begin{figure}[h!]
  \begin{center}
    \subfigure[$400 < \scalht < 500\GeV$]{\includegraphics[width=0.32\textwidth]{figures/mhtTemplate/exclusive/fits/Mu/ht400_500_control_eq0b_eq4j_Mu.pdf}}
    \subfigure[$500 < \scalht < 600\GeV$]{\includegraphics[width=0.32\textwidth]{figures/mhtTemplate/exclusive/fits/Mu/ht500_600_control_eq0b_eq4j_Mu.pdf}}
    \subfigure[$600 < \scalht < 750\GeV$]{\includegraphics[width=0.32\textwidth]{figures/mhtTemplate/exclusive/fits/Mu/ht600_750_control_eq0b_eq4j_Mu.pdf}}\\
    \subfigure[$750 < \scalht < 900\GeV$]{\includegraphics[width=0.32\textwidth]{figures/mhtTemplate/exclusive/fits/Mu/ht750_900_control_eq0b_eq4j_Mu.pdf}}
    \subfigure[$900 < \scalht < 1050\GeV$]{\includegraphics[width=0.32\textwidth]{figures/mhtTemplate/exclusive/fits/Mu/ht900_1050_control_eq0b_eq4j_Mu.pdf}}
    \subfigure[$1050 < \scalht < 1200\GeV$]{\includegraphics[width=0.32\textwidth]{figures/mhtTemplate/exclusive/fits/Mu/ht1050_1200_control_eq0b_eq4j_Mu.pdf}}\\
    \subfigure[$\scalht > 1200\GeV$]{\includegraphics[width=0.32\textwidth]{figures/mhtTemplate/exclusive/fits/Mu/ht1200_Inf_control_eq0b_eq4j_Mu.pdf}}
    \caption{The ratio of event yields obtained from data and simulation as a function of \mht [GeV] based on a sample of \mj events that satisfy $\njet = 4$ and $\nb = 0$, as well as the requirements on \scalht indicated by each sub-figure caption. Also shown are fits to the ratios using first-order orthogonal polynomials.}
    \label{fig:mhtval_Mu_eq4j_eq0b}
  \end{center}
\end{figure}

\begin{figure}[h!]
  \begin{center}
    \subfigure[$400 < \scalht < 500\GeV$]{\includegraphics[width=0.32\textwidth]{figures/mhtTemplate/exclusive/fits/Mu/ht400_500_control_eq1b_eq4j_Mu.pdf}}
    \subfigure[$500 < \scalht < 600\GeV$]{\includegraphics[width=0.32\textwidth]{figures/mhtTemplate/exclusive/fits/Mu/ht500_600_control_eq1b_eq4j_Mu.pdf}}
    \subfigure[$600 < \scalht < 750\GeV$]{\includegraphics[width=0.32\textwidth]{figures/mhtTemplate/exclusive/fits/Mu/ht600_750_control_eq1b_eq4j_Mu.pdf}}\\
    \subfigure[$750 < \scalht < 900\GeV$]{\includegraphics[width=0.32\textwidth]{figures/mhtTemplate/exclusive/fits/Mu/ht750_900_control_eq1b_eq4j_Mu.pdf}}
    \subfigure[$900 < \scalht < 1050\GeV$]{\includegraphics[width=0.32\textwidth]{figures/mhtTemplate/exclusive/fits/Mu/ht900_1050_control_eq1b_eq4j_Mu.pdf}}
    \subfigure[$\scalht > 1200\GeV$]{\includegraphics[width=0.32\textwidth]{figures/mhtTemplate/exclusive/fits/Mu/ht1200_Inf_control_eq1b_eq4j_Mu.pdf}}\\
    \caption{The ratio of event yields obtained from data and simulation as a function of \mht [GeV] based on a sample of \mj events that satisfy $\njet = 4$ and $\nb = 1$, as well as the requirements on \scalht indicated by each sub-figure caption. Also shown are fits to the ratios using first-order orthogonal polynomials.}
    \label{fig:mhtval_Mu_eq4j_eq1b}
  \end{center}
\end{figure}

\begin{figure}[h!]
  \begin{center}
    \subfigure[$400 < \scalht < 500\GeV$]{\includegraphics[width=0.32\textwidth]{figures/mhtTemplate/exclusive/fits/Mu/ht400_500_control_eq2b_eq4j_Mu.pdf}}
    \subfigure[$500 < \scalht < 600\GeV$]{\includegraphics[width=0.32\textwidth]{figures/mhtTemplate/exclusive/fits/Mu/ht500_600_control_eq2b_eq4j_Mu.pdf}}
    \subfigure[$600 < \scalht < 750\GeV$]{\includegraphics[width=0.32\textwidth]{figures/mhtTemplate/exclusive/fits/Mu/ht600_750_control_eq2b_eq4j_Mu.pdf}}\\
    \subfigure[$750 < \scalht < 900\GeV$]{\includegraphics[width=0.32\textwidth]{figures/mhtTemplate/exclusive/fits/Mu/ht750_900_control_eq2b_eq4j_Mu.pdf}}
    \subfigure[$900 < \scalht < 1050\GeV$]{\includegraphics[width=0.32\textwidth]{figures/mhtTemplate/exclusive/fits/Mu/ht900_1050_control_eq2b_eq4j_Mu.pdf}}
    \subfigure[$1050 < \scalht < 1200\GeV$]{\includegraphics[width=0.32\textwidth]{figures/mhtTemplate/exclusive/fits/Mu/ht1050_1200_control_eq2b_eq4j_Mu.pdf}}\\
    \subfigure[$\scalht > 1200\GeV$]{\includegraphics[width=0.32\textwidth]{figures/mhtTemplate/exclusive/fits/Mu/ht1200_Inf_control_eq2b_eq4j_Mu.pdf}}
    \caption{The ratio of event yields obtained from data and simulation as a function of \mht [GeV] based on a sample of \mj events that satisfy $\njet = 4$ and $\nb = 2$, as well as the requirements on \scalht indicated by each sub-figure caption. Also shown are fits to the ratios using first-order orthogonal polynomials.}
    \label{fig:mhtval_Mu_eq4j_eq2b}
  \end{center}
\end{figure}

\begin{figure}[h!]
  \begin{center}
    \subfigure[$400 < \scalht < 500\GeV$]{\includegraphics[width=0.32\textwidth]{figures/mhtTemplate/exclusive/fits/Mu/ht400_500_control_eq0b_eq5j_Mu.pdf}}
    \subfigure[$500 < \scalht < 600\GeV$]{\includegraphics[width=0.32\textwidth]{figures/mhtTemplate/exclusive/fits/Mu/ht500_600_control_eq0b_eq5j_Mu.pdf}}
    \subfigure[$600 < \scalht < 750\GeV$]{\includegraphics[width=0.32\textwidth]{figures/mhtTemplate/exclusive/fits/Mu/ht600_750_control_eq0b_eq5j_Mu.pdf}}\\
    \subfigure[$750 < \scalht < 900\GeV$]{\includegraphics[width=0.32\textwidth]{figures/mhtTemplate/exclusive/fits/Mu/ht750_900_control_eq0b_eq5j_Mu.pdf}}
    \subfigure[$900 < \scalht < 1050\GeV$]{\includegraphics[width=0.32\textwidth]{figures/mhtTemplate/exclusive/fits/Mu/ht900_1050_control_eq0b_eq5j_Mu.pdf}}
    \subfigure[$1050 < \scalht < 1200\GeV$]{\includegraphics[width=0.32\textwidth]{figures/mhtTemplate/exclusive/fits/Mu/ht1050_1200_control_eq0b_eq5j_Mu.pdf}}\\
    \subfigure[$\scalht > 1200\GeV$]{\includegraphics[width=0.32\textwidth]{figures/mhtTemplate/exclusive/fits/Mu/ht1200_Inf_control_eq0b_eq5j_Mu.pdf}}
    \caption{The ratio of event yields obtained from data and simulation as a function of \mht [GeV] based on a sample of \mj events that satisfy $\njet = 5$ and $\nb = 0$, as well as the requirements on \scalht indicated by each sub-figure caption. Also shown are fits to the ratios using first-order orthogonal polynomials.}
    \label{fig:mhtval_Mu_eq5j_eq0b}
  \end{center}
\end{figure}

\begin{figure}[h!]
  \begin{center}
    \subfigure[$400 < \scalht < 500\GeV$]{\includegraphics[width=0.32\textwidth]{figures/mhtTemplate/exclusive/fits/Mu/ht400_500_control_eq1b_eq5j_Mu.pdf}}
    \subfigure[$500 < \scalht < 600\GeV$]{\includegraphics[width=0.32\textwidth]{figures/mhtTemplate/exclusive/fits/Mu/ht500_600_control_eq1b_eq5j_Mu.pdf}}
    \subfigure[$600 < \scalht < 750\GeV$]{\includegraphics[width=0.32\textwidth]{figures/mhtTemplate/exclusive/fits/Mu/ht600_750_control_eq1b_eq5j_Mu.pdf}}\\
    \subfigure[$750 < \scalht < 900\GeV$]{\includegraphics[width=0.32\textwidth]{figures/mhtTemplate/exclusive/fits/Mu/ht750_900_control_eq1b_eq5j_Mu.pdf}}
    \subfigure[$900 < \scalht < 1050\GeV$]{\includegraphics[width=0.32\textwidth]{figures/mhtTemplate/exclusive/fits/Mu/ht900_1050_control_eq1b_eq5j_Mu.pdf}}
    \subfigure[$1050 < \scalht < 1200\GeV$]{\includegraphics[width=0.32\textwidth]{figures/mhtTemplate/exclusive/fits/Mu/ht1050_1200_control_eq1b_eq5j_Mu.pdf}}\\
    \subfigure[$\scalht > 1200\GeV$]{\includegraphics[width=0.32\textwidth]{figures/mhtTemplate/exclusive/fits/Mu/ht1200_Inf_control_eq1b_eq5j_Mu.pdf}}
    \caption{The ratio of event yields obtained from data and simulation as a function of \mht [GeV] based on a sample of \mj events that satisfy $\njet = 5$ and $\nb = 1$, as well as the requirements on \scalht indicated by each sub-figure caption. Also shown are fits to the ratios using first-order orthogonal polynomials.}
    \label{fig:mhtval_Mu_eq5j_eq1b}
  \end{center}
\end{figure}

\begin{figure}[h!]
  \begin{center}
    \subfigure[$500 < \scalht < 600\GeV$]{\includegraphics[width=0.32\textwidth]{figures/mhtTemplate/exclusive/fits/Mu/ht500_600_control_eq2b_eq5j_Mu.pdf}}
    \subfigure[$600 < \scalht < 750\GeV$]{\includegraphics[width=0.32\textwidth]{figures/mhtTemplate/exclusive/fits/Mu/ht600_750_control_eq2b_eq5j_Mu.pdf}}
    \subfigure[$750 < \scalht < 900\GeV$]{\includegraphics[width=0.32\textwidth]{figures/mhtTemplate/exclusive/fits/Mu/ht750_900_control_eq2b_eq5j_Mu.pdf}}\\
    \subfigure[$900 < \scalht < 1050\GeV$]{\includegraphics[width=0.32\textwidth]{figures/mhtTemplate/exclusive/fits/Mu/ht900_1050_control_eq2b_eq5j_Mu.pdf}}
    \subfigure[$1050 < \scalht < 1200\GeV$]{\includegraphics[width=0.32\textwidth]{figures/mhtTemplate/exclusive/fits/Mu/ht1050_1200_control_eq2b_eq5j_Mu.pdf}}
    \subfigure[$\scalht > 1200\GeV$]{\includegraphics[width=0.32\textwidth]{figures/mhtTemplate/exclusive/fits/Mu/ht1200_Inf_control_eq2b_eq5j_Mu.pdf}}\\
    \caption{The ratio of event yields obtained from data and simulation as a function of \mht [GeV] based on a sample of \mj events that satisfy $\njet = 5$ and $\nb = 2$, as well as the requirements on \scalht indicated by each sub-figure caption. Also shown are fits to the ratios using first-order orthogonal polynomials.}
    \label{fig:mhtval_Mu_eq5j_eq2b}
  \end{center}
\end{figure}

\begin{figure}[h!]
  \begin{center}
    \subfigure[$500 < \scalht < 600\GeV$]{\includegraphics[width=0.32\textwidth]{figures/mhtTemplate/exclusive/fits/Mu/ht500_600_control_eq0b_ge6j_Mu.pdf}}
    \subfigure[$600 < \scalht < 750\GeV$]{\includegraphics[width=0.32\textwidth]{figures/mhtTemplate/exclusive/fits/Mu/ht600_750_control_eq0b_ge6j_Mu.pdf}}
    \subfigure[$750 < \scalht < 900\GeV$]{\includegraphics[width=0.32\textwidth]{figures/mhtTemplate/exclusive/fits/Mu/ht750_900_control_eq0b_ge6j_Mu.pdf}}\\
    \subfigure[$900 < \scalht < 1050\GeV$]{\includegraphics[width=0.32\textwidth]{figures/mhtTemplate/exclusive/fits/Mu/ht900_1050_control_eq0b_ge6j_Mu.pdf}}
    \subfigure[$1050 < \scalht < 1200\GeV$]{\includegraphics[width=0.32\textwidth]{figures/mhtTemplate/exclusive/fits/Mu/ht1050_1200_control_eq0b_ge6j_Mu.pdf}}
    \subfigure[$\scalht > 1200\GeV$]{\includegraphics[width=0.32\textwidth]{figures/mhtTemplate/exclusive/fits/Mu/ht1200_Inf_control_eq0b_ge6j_Mu.pdf}}\\
    \caption{The ratio of event yields obtained from data and simulation as a function of \mht [GeV] based on a sample of \mj events that satisfy $\njet \geq 6$ and $\nb = 0$, as well as the requirements on \scalht indicated by each sub-figure caption. Also shown are fits to the ratios using first-order orthogonal polynomials.}
    \label{fig:mhtval_Mu_ge6j_eq0b}
  \end{center}
\end{figure}

\begin{figure}[h!]
  \begin{center}
    \subfigure[$500 < \scalht < 600\GeV$]{\includegraphics[width=0.32\textwidth]{figures/mhtTemplate/exclusive/fits/Mu/ht500_600_control_eq1b_ge6j_Mu.pdf}}
    \subfigure[$600 < \scalht < 750\GeV$]{\includegraphics[width=0.32\textwidth]{figures/mhtTemplate/exclusive/fits/Mu/ht600_750_control_eq1b_ge6j_Mu.pdf}}
    \subfigure[$750 < \scalht < 900\GeV$]{\includegraphics[width=0.32\textwidth]{figures/mhtTemplate/exclusive/fits/Mu/ht750_900_control_eq1b_ge6j_Mu.pdf}}\\
    \subfigure[$900 < \scalht < 1050\GeV$]{\includegraphics[width=0.32\textwidth]{figures/mhtTemplate/exclusive/fits/Mu/ht900_1050_control_eq1b_ge6j_Mu.pdf}}
    \subfigure[$1050 < \scalht < 1200\GeV$]{\includegraphics[width=0.32\textwidth]{figures/mhtTemplate/exclusive/fits/Mu/ht1050_1200_control_eq1b_ge6j_Mu.pdf}}
    \subfigure[$\scalht > 1200\GeV$]{\includegraphics[width=0.32\textwidth]{figures/mhtTemplate/exclusive/fits/Mu/ht1200_Inf_control_eq1b_ge6j_Mu.pdf}}\\
    \caption{The ratio of event yields obtained from data and simulation as a function of \mht [GeV] based on a sample of \mj events that satisfy $\njet \geq 6$ and $\nb = 1$, as well as the requirements on \scalht indicated by each sub-figure caption. Also shown are fits to the ratios using first-order orthogonal polynomials.}
    \label{fig:mhtval_Mu_ge6j_eq1b}
  \end{center}
\end{figure}

\begin{figure}[h!]
  \begin{center}
    \subfigure[$600 < \scalht < 750\GeV$]{\includegraphics[width=0.32\textwidth]{figures/mhtTemplate/exclusive/fits/Mu/ht600_750_control_eq2b_ge6j_Mu.pdf}}
    \subfigure[$750 < \scalht < 900\GeV$]{\includegraphics[width=0.32\textwidth]{figures/mhtTemplate/exclusive/fits/Mu/ht750_900_control_eq2b_ge6j_Mu.pdf}}
    \subfigure[$900 < \scalht < 1050\GeV$]{\includegraphics[width=0.32\textwidth]{figures/mhtTemplate/exclusive/fits/Mu/ht900_1050_control_eq2b_ge6j_Mu.pdf}}\\
    \subfigure[$\scalht > 1200\GeV$]{\includegraphics[width=0.32\textwidth]{figures/mhtTemplate/exclusive/fits/Mu/ht1200_Inf_control_eq2b_ge6j_Mu.pdf}}
    \caption{The ratio of event yields obtained from data and simulation as a function of \mht [GeV] based on a sample of \mj events that satisfy $\njet \geq 6$ and $\nb = 2$, as well as the requirements on \scalht indicated by each sub-figure caption. Also shown are fits to the ratios using first-order orthogonal polynomials.}
    \label{fig:mhtval_Mu_ge6j_eq2b}
  \end{center}
\end{figure}


