%%____________________________________________________________________________||
\section{Interpretations with simplified ``split SUSY'' models}
\label{app:LLP}

\subsection{Distributions}
\label{app:LLP-distributions}

Figure \ref{fig:T1qqqqLLvsT1qqqqLL} shows a comparison of the distributions of 
the main kinematic variables between various \ctau models. In general, the
distributions become softer with increasing lifetime as the jet response and
reconstruction efficiency fall with increasing displacement from the primary
vertex. In addition, the gluinos become more likely to decay outside the detector.
There is also an enhancement in the number of b-tagged jets when \ctau
is of order 1~mm. This enhancement is less noticeable for compressed models
as the displaced jets are softer and therefore more likely to be below the jet 
\pt threshold.

Figure \ref{fig:T1qqqqLLvsT1qqqq} shows a comparison of the kinematic
distributions between the prompt (FastSim) T1qqqq model and the
T1qqqqLL model with a relatively prompt lifetime of \ctau$=0.001$~mm. The two
samples are in good agreement within statistical uncertainties.

Figure \ref{fig:T1qqqqLLvsT1bbbb} shows a comparison of the kinematic
distributions between the \ctau$=1$~mm model and the T1bbbb model. The two
models have similar kinematics. There is some difference in the \nb distribution
that arises from the difference in b-tagging efficiency for b jets and
displaced jets. This difference becomes larger with decreasing jet \pt, as
seen in Sec.~\ref{app:LLP-btagging}, and is thus more noticeable
for the compressed samples.

\begin{figure}[h!]
  \begin{center}
    \subfigure{\includegraphics[width=0.28\textwidth]{figures/LLPResults/T1qqqqLL_1000_200/ht40_all_all}} ~
    \subfigure{\includegraphics[width=0.28\textwidth]{figures/LLPResults/T1qqqqLL_1000_900/ht40_all_all}} \\
    \subfigure{\includegraphics[width=0.28\textwidth]{figures/LLPResults/T1qqqqLL_1000_200/mht40_pt_all_all}} ~
    \subfigure{\includegraphics[width=0.28\textwidth]{figures/LLPResults/T1qqqqLL_1000_900/mht40_pt_all_all}} \\
    \subfigure{\includegraphics[width=0.28\textwidth]{figures/LLPResults/T1qqqqLL_1000_200/nJet40_all_all}} ~
    \subfigure{\includegraphics[width=0.28\textwidth]{figures/LLPResults/T1qqqqLL_1000_900/nJet40_all_all}} \\
    \subfigure{\includegraphics[width=0.28\textwidth]{figures/LLPResults/T1qqqqLL_1000_200/nBJet40_all_all}} ~
    \subfigure{\includegraphics[width=0.28\textwidth]{figures/LLPResults/T1qqqqLL_1000_900/nBJet40_all_all}}
    \caption{Kinematic distributions comparing various lifetimes, for an uncompressed (1000,200) (Left)
        and compressed (1000,900) (Right) mass point.}
    \label{fig:T1qqqqLLvsT1qqqqLL}
  \end{center}
\end{figure}

\begin{figure}[h!]
  \begin{center}
    \subfigure{\includegraphics[width=0.28\textwidth]{figures/LLPResults/T1qqqqLL_vs_T1qqqq_1800_200/ht40_all_all}} ~
    \subfigure{\includegraphics[width=0.28\textwidth]{figures/LLPResults/T1qqqqLL_vs_T1qqqq_1000_900/ht40_all_all}} \\
    \subfigure{\includegraphics[width=0.28\textwidth]{figures/LLPResults/T1qqqqLL_vs_T1qqqq_1800_200/mht40_pt_all_all}} ~
    \subfigure{\includegraphics[width=0.28\textwidth]{figures/LLPResults/T1qqqqLL_vs_T1qqqq_1000_900/mht40_pt_all_all}} \\
    \subfigure{\includegraphics[width=0.28\textwidth]{figures/LLPResults/T1qqqqLL_vs_T1qqqq_1800_200/nJet40_all_all}} ~
    \subfigure{\includegraphics[width=0.28\textwidth]{figures/LLPResults/T1qqqqLL_vs_T1qqqq_1000_900/nJet40_all_all}} \\
    \subfigure{\includegraphics[width=0.28\textwidth]{figures/LLPResults/T1qqqqLL_vs_T1qqqq_1800_200/nBJet40_all_all}} ~
    \subfigure{\includegraphics[width=0.28\textwidth]{figures/LLPResults/T1qqqqLL_vs_T1qqqq_1000_900/nBJet40_all_all}}
    \caption{Kinematic distributions comparing prompt T1qqqq and T1qqqqLL with \ctau$=0.001$~mm, for an 
        uncompressed (1800,200) (Left) and compressed (1000,900) (Right) mass point.}
    \label{fig:T1qqqqLLvsT1qqqq}
  \end{center}
\end{figure}

\begin{figure}[h!]
  \begin{center}
    \subfigure{\includegraphics[width=0.28\textwidth]{figures/LLPResults/T1qqqqLL_vs_T1bbbb_1800_200/ht40_all_all}} ~
    \subfigure{\includegraphics[width=0.28\textwidth]{figures/LLPResults/T1qqqqLL_vs_T1bbbb_1000_900/ht40_all_all}} \\
    \subfigure{\includegraphics[width=0.28\textwidth]{figures/LLPResults/T1qqqqLL_vs_T1bbbb_1800_200/mht40_pt_all_all}} ~
    \subfigure{\includegraphics[width=0.28\textwidth]{figures/LLPResults/T1qqqqLL_vs_T1bbbb_1000_900/mht40_pt_all_all}} \\
    \subfigure{\includegraphics[width=0.28\textwidth]{figures/LLPResults/T1qqqqLL_vs_T1bbbb_1800_200/nJet40_all_all}} ~
    \subfigure{\includegraphics[width=0.28\textwidth]{figures/LLPResults/T1qqqqLL_vs_T1bbbb_1000_900/nJet40_all_all}} \\
    \subfigure{\includegraphics[width=0.28\textwidth]{figures/LLPResults/T1qqqqLL_vs_T1bbbb_1800_200/nBJet40_all_all}} ~
    \subfigure{\includegraphics[width=0.28\textwidth]{figures/LLPResults/T1qqqqLL_vs_T1bbbb_1000_900/nBJet40_all_all}}
    \caption{Kinematic distributions comparing prompt T1bbbb and T1qqqqLL with \ctau$=1$~mm, for an
        uncompressed (1800,200) (Left) and compressed (1000,900) (Right) mass point.}
    \label{fig:T1qqqqLLvsT1bbbb}
  \end{center}
\end{figure}

%\subsection{Trigger efficiency}
%\label{app:LLP-trigger}
%
%Table \ref{tab:LLP-triggereff} shows the efficiency of the signal region 
%triggers (listed in Sec.~\ref{sec:triggers}) for some representative mass points
%and various lifetimes. Some comment about magnitude of inefficiency, related to
%jet id requirement in trigger.
%
%Table goes here.
%
%A trigger requirement is imposed on these signal models according to the
%trigger emulation provided in the simulation samples. A systematic uncertainty
%is assigned that is dependent on \mht and \njet. These uncertainties are taken 
%from the efficiencies measured in data, as described in Sec.~\ref{sec:triggers}.

\subsection{Jet response}
\label{app:LLP-jetresponse}

Figure \ref{fig:T1qqqq:response} shows the jet response (defined as the ratio
of reconstructed \pt to generator-level \pt) for generator jets with $\pt>40$ and 
$|\eta|<2.4$ for the different lifetimes considered.

\begin{figure}
    \begin{center}
    \includegraphics[width=0.6\textwidth]{figures/LLPResults/T1qqqqLL_response}
    \caption{Distribution of response for generator-level jets with $\pt>40$ and $|\eta|<2.4$,
        for various \ctau models. Only jets originating from one of the long-lived gluinos
        and that are matched to a reconstructed jet are considered.}
    \label{fig:T1qqqq:response}
    \end{center}
\end{figure}

\subsection{b-tagging}
\label{app:LLP-btagging}

Documentation in progress. See slides and email exchange with BTV on the
hypernews for the agreement that was reached in terms of b-tag scale factors 
and associated uncertainties.

