%%____________________________________________________________________________||
\section{Interpretation in Simplified SUSY models}
\label{sec:susy}
To interpret the results of this search, simplified
%models~\cite{Alwall:2008ag,Alwall:2008va,Alves:2011wf} for the production of
supersymmetric particles are considered.  They use only a limited set of
sparticles in the production and decay and enable comprehensive studies of
individual SUSY event topologies. These studies can be performed in terms of
fundamental properties such as decay modes, production cross sections and
sparticle masses.

\subsection{Signal models}
\label{sec:susy_models}
Interpretations are presented for pair production of gluino, stop, sbottom and
scharm particles with several different possibilities for the decay chain. The
simplified models used in the analysis are summarised in
%Tab.~\ref{tab:simplified-models}. %FIXME: Benchmark points are also listed, which will
%be used throughout the section. Two benchmarks are provided, where relevant,
%to cover the phenomenology of compressed and uncompressed spectra. All the
%benchmarks are chosen to be right within the expected exclusion. \\
%A sensitivity study has been performed on these benchmark models, which is
%described in App.~\ref{app:sensitivity-benchmarks}. \\
Sketches of the production and decay in these models are shown in
Fig.~\ref{fig:simplified-models-feyn}. Systematic uncertainties on the signal
%acceptance are detailed in Sec.~\ref{sec:sig-syst}.

%All the models are generated using the FastSim package \cite{Abdullin:2011zz}.
FastSim Monte Carlo samples are corrected using FastSim-to-FullSim scale factor,
accounting for difference in b-tag efficiency. These scale factors are applied
on top of the FullSim scale factors applied for the other Monte Carlo samples.

\begin{table}[h!]
    \scriptsize
    \caption{A summary of the simplified models used in this analysis.}
    \label{tab:simplified-models}
    \centering
    \begin{tabular}{ lllllll }
        \hline \hline
        Model & Production & Decay & Run 1 (CMS) & Figure & Benchmarks $(m_{\mathrm{Susy}},m_{\mathrm{LSP}})$ \\ 
        \hline \hline
        T1bbbb & \ppToGluGlu & \gluToBBNo & $m_{\mathrm{Gluino}}>\sim 1350 \,\mathrm{GeV}$ & \ref{fig:T1bbbb_feyn} & \parbox[t]{5cm}{Compressed: (XXX,YYY)\\Uncompressed: (XXX,YYY)} \\
        \hline
        T2tt   & \ppToStopStop & \stopToTNo & $m_{\mathrm{\mathrm{Stop}}}>\sim 760 \,\mathrm{GeV}$ & \ref{fig:T2tt_feyn} & \parbox[t]{5cm}{Compressed: (XXX,YYY)\\Uncompressed: (XXX,YYY)} \\
        \hline
        T2bb   & \ppToSbotSbot & \sbottomToB & $m_{\mathrm{\mathrm{Sbottom}}}>\sim 750 \,\mathrm{GeV}$ & \ref{fig:T2bb_feyn} & \parbox[t]{5cm}{Compressed: (XXX,YYY)\\Uncompressed: (XXX,YYY)} \\
        \hline
        T2cc   & \ppToStopStop & \stopToCNo & $m_{\mathrm{\mathrm{Stop}}}>\sim 250 \,\mathrm{GeV}$ & \ref{fig:T2cc_feyn} & Compressed: (XXX,YYY) \\
        \hline \hline
    \end{tabular}
\end{table}

\begin{figure}[h!]
    \begin{center}
        \subfigure[T1bbbb]{
            \includegraphics[width=0.35\textwidth]{figures/susyResults/T1bbbb_feyn}
            \label{fig:T1bbbb_feyn}
            } ~~
        \subfigure[T2tt]{
            \includegraphics[width=0.35\textwidth]{figures/susyResults/T2tt_feyn}
            \label{fig:T2tt_feyn}
            } \\
        \subfigure[T2bb]{
            \includegraphics[width=0.35\textwidth]{figures/susyResults/T2bb_feyn}
            \label{fig:T2bb_feyn}
            }
        \subfigure[T2cc]{
            \includegraphics[width=0.35\textwidth]{figures/susyResults/T2cc_feyn}
            \label{fig:T2cc_feyn}
            } ~~
        \caption{Graphical representation of the production and decay of
            supersymmetric particles in models with the production of: (a)
            gluinos (with decoupled third generation squarks), (b) stops, (c)
            sbottoms and (d) scharms.}
        \label{fig:simplified-models-feyn}
    \end{center}
\end{figure}

\subsection{Signal acceptance and contamination}
\label{sec:sig-accept-contam}
In Tab. \ref{tab:sig-eff} the signal efficiency for benchmark mass points for
all the models is summarised. The signal efficiency across the whole
($m_{\mathrm{Susy}},m_{\mathrm{LSP}}$) for the simplified models used in the
analysis is shown in Fig.~\ref{fig:T1bbbb_eff}-\ref{fig:T2cc_eff}.

\begin{table}[h!]
    \caption{Signal efficiency for compressed and uncompressed models used in
        the analysis.}
    \label{tab:sig-eff}
    \centering
    \begin{tabular}{ lllll }
        \hline \hline
        Model & $(m_{\mathrm{Susy}},m_{\mathrm{LSP}})$ & Efficiency (total) \\ 
        \hline
        \multirow{2}{*}{T1bbbb}
            & (XXX,YYY) & XX\% \\
            & (XXX,YYY) & XX\% \\
        \hline
        \multirow{2}{*}{T2tt}
            & (XXX,YYY) & XX\% \\
            & (XXX,YYY) & XX\% \\
        \hline
        T2cc
            & (XXX,YYY) & XX\% \\
        \hline
        \multirow{2}{*}{T2bb}
            & (XXX,YYY) & XX\% \\
            & (XXX,YYY) & XX\% \\
        \hline \hline
    \end{tabular}
\end{table}

The level of signal contamination in the control regions is expected to be
negligible for most of the models that are targeted by this search. The
requirement of one muon or two muons in the \mj and \mmj respectively ensures
that the control regions are depleted from signal events, in the case where the
final state is all-hadronic. The only partial exception is the stop pair
production followed by decay into top quarks, called T2tt. In this case, when
top decays leptonically, a residual signal contamination may be found in the
muon control regions. However, the kinematic selection applied to the control
regions, like the absence of any \alt cut and the $m_{T}$ cut, helps to reduce
the signal contamination. \\
The metric that is chosen to study the signal contamination in the following is
the double-ratio $(S_{SR}/B_{SR})/(S_{CR}/B_{CR})$, as the sensitivity of the
control region ($S_{CR}/B_{CR}$) is expected to be small compared to the one in
the signal region ($S_{SR}/B_{SR}$) by definition.

Fig.~\ref{fig:contamination_T2tt} characterises the level of signal
contamination for the T2tt ($m_{\mathrm{Stop}}=XXX$ GeV, $m_{\mathrm{LSP}}=YYY$ GeV)
model, as a function of event category and \scalht bin. This benchmark point
has $m_{\mathrm{Susy}}-m_{\mathrm{LSP}} \sim m_{\mathrm{top}}$, which is the
scenario where the largest signal contamination is expected, since the
kinematics is similar to the top quark production, which is more likely to
satisfy the control region selection with respect to the signal region. \\
Fig. ~\ref{fig:contamination_T2tt_yields_had} and
~\ref{fig:contamination_T2tt_yields_had} show the expected signal yield counts
in the signal region and \mj control region respectively, for the T2tt benchmark
model. Fig.~\ref{fig:contamination_T2tt_relEff} shows the ratio of signal
contamination to signal efficiency of the signal region, for the T2tt benchmark
model. Fig.~\ref{fig:contamination_T2tt_doubleRatio} shows the ratio of
sensitivity in the control region to the sensitivity in the signal region, for
T2tt benchmark model. The sensitivity is defined as the ratio of signal to
background expected counts.

%\begin{figure}[h!]
%    \begin{center}
%        \subfigure[Expected counts in the signal region]{
%            \includegraphics[width=0.5\textwidth]{figures/susyResults/sigYields_had_SMS-T2tt_mStop-250_mLSP-50_25ns}
%            \label{fig:contamination_T2tt_yields_had}
%        }
%        \subfigure[Expected counts in the \mj control region]{
%            \includegraphics[width=0.5\textwidth]{figures/susyResults/sigYields_SingleMu_SMS-T2tt_mStop-250_mLSP-50_25ns}
%            \label{fig:contamination_T2tt_yields_mj}
%        } \\
%        \subfigure[Ratio of signal acceptance (\mj to signal region)]{
%            \includegraphics[width=0.5\textwidth]{figures/susyResults/relEff_SingleMu_SMS-T2tt_mStop-250_mLSP-50_25ns}
%            \label{fig:contamination_T2tt_relEff}
%        }
%        \subfigure[Ratio of S/B values (\mj to signal region)]{
%            \includegraphics[width=0.5\textwidth]{figures/susyResults/doubleRatio_SingleMu_SMS-T2tt_mStop-250_mLSP-50_25ns}
%            \label{fig:contamination_T2tt_doubleRatio}
%        }
%        \caption{Characterisation of signal acceptance and contamination in the
%            signal and \mj control regions, respectively, for the benchmark
%            model T2tt (250,50).}
%        \label{fig:contamination_T2tt}
%    \end{center}
%\end{figure}

The effect of signal contamination can be sizeable in these particular
scenario for T2tt, as shown in
Fig.~\ref{fig:contamination_T2tt_doubleRatio}, where in the most
sensitive bins (high \njet, high \nb) the contribution of the control
region sensitivity is comparable to the one of the signal
region. However this is not an issue in the analysis, as the potential
for signal contamination in all control samples is fully accounted for
in the likelihood model used to extract the statistical interpretation
(see Sec.~\ref{sec:likelihood}).

The level of contamination for the \mmj sample is smaller still due to the
requirement of a second muon.

\subsection{Systematic uncertainties on signal efficiency times acceptance}
\label{sec:sig-syst}
The following sources of systematic uncertainty are propagated to the signal
acceptance and shape, according to the recommendations agreed on within the
collaboration. Relative effect on the yields are presented in
Tab.~\ref{tab:sig-systematics} for some benchmark models.

\begin{itemize}
    \item Luminosity: XX\%, taken as correlated across all bins.
    \item Trigger: conservatively, the size of the inefficiency is taken as
        systematic variation where not in the plateau (see Sec.~\ref{sec:triggers}).
    \item MC statistics:  uncorrelated bin-by-bin uncertainty, affecting the
        shape of the signal.
    \item Pileup reweighting: XX\% uncertainty on the minimum bias cross section
        (see Sec.~\ref{sec:pileup-reweighting}).
    \item b-tag efficiency: uncertainty on the FullSim and FastSim b-tag scale
        factor is propagated and taken as correlated across the bins.
    \item Lepton efficiency: uncertainty on the lepton scale factors is
        propagated and taken as correlated across the bins.
    \item Jet energy scale: uncertainty on the jet energy corrections is
        propagated and taken as correlated across the bins.
    \item Initial State Radiation (ISR): 15\% (30\%) uncertainty for the \Pt of
        the gluino-gluino system between 400-600 GeV (above 600 GeV).
\end{itemize}

\begin{table}[h!]
    \caption{Representative range of uncertainty across the analysis bins for
        each source of signal systematic. Two benchmark point are chosen for
        each model, corresponding to ``compressed'' and ``uncompressed''
        scenarios, i.e. with small and large mass splitting between the mother
        particle and the LSP.}
    \label{tab:sig-systematics}
    \centering
    \begin{tabular}{ ccccccccc }
        \hline \hline
        Model & ($m_{\mathrm{Susy}},m_{\mathrm{LSP}}$) & Luminosity & ISR & JEC & PU & b-tag & Trigger & MC stat. \\ \hline
        \multirow{2}{*}{T1bbbb}
            & (XXX,YYY) & XX\% & XX\% & XX\% & XX\% & XX\% & XX\% & XX\% \\ 
            & (XXX,YYY) & XX\% & XX\% & XX\% & XX\% & XX\% & XX\% & XX\% \\
        \hline
        \multirow{2}{*}{T2tt}
            & (XXX,YYY) & XX\% & XX\% & XX\% & XX\% & XX\% & XX\% & XX\% \\ 
            & (XXX,YYY) & XX\% & XX\% & XX\% & XX\% & XX\% & XX\% & XX\% \\
        \hline
        \multirow{1}{*}{T2cc}
            & (XXX,YYY) & XX\% & XX\% & XX\% & XX\% & XX\% & XX\% & XX\% \\
        \hline
        \multirow{2}{*}{T2bb}
            & (XXX,YYY) & XX\% & XX\% & XX\% & XX\% & XX\% & XX\% & XX\% \\ 
            & (XXX,YYY) & XX\% & XX\% & XX\% & XX\% & XX\% & XX\% & XX\% \\
        \hline \hline
    \end{tabular}
\end{table}

\subsection{Exclusion limits}
\label{sec:susy_results}
In order to extract the signal contribution in the fit, the distribution of
events according to the \mht variable, encoded as template histograms, is used
as described in Sec.~\ref{sec:had-shape} and \ref{sec:likelihood}. \\
Upper limits on the cross section are computed using the $\text{CL}_{s}$
criterion \cite{CLsTechnique}. Asymptotic formulae \cite{AsymptoticFormulae} are
utilised to approximate the distribution of the test statistics. \\
All the statistical results are produced using the \textit{combine} tool,
provided within the HiggsAnalysis-CombinedLimit package \cite{Combine}.

In Fig. ~\ref{fig:T1bbbb}-\ref{fig:T2bb} (top) the 95\% C.L. upper limits on the
cross section are shown in the $(m_{\mathrm{Susy}},m_{\mathrm{LSP}})$ plane for
the models considered in this interpretation. These results correspond to
35.9~\ifb of integrated luminosity. The exclusion contour is also shown together
with $\pm1\sigma$ (and $\pm2\sigma$ for the expected exclusion) uncertainty.
The band around the expected exclusion reflects the experimental uncertainty,
while the band around the observed exclusion correspond to the theoretical
uncertainty on the signal cross section.\\
In Fig. ~\ref{fig:T1bbbb}-\ref{fig:T2bb} (bottom left) the signal acceptance
including the 4 most excluding jet categories is shown. \\
In Fig. ~\ref{fig:T1bbbb}-\ref{fig:T2bb} (bottom right) the signal acceptance
including the whole signal region is shown.

The models are grouped according to the following categorisation:
\begin{itemize}
    \item \textbf{Gluino-mediated production of off-shell (decoupled) 3rd generation squarks}:
        gluino pair production followed by 3-body decay to $t\bar{t}\chiz$,$b\bar{b}\chiz$.
        It includes T1bbbb only.
    \item \textbf{Direct production of 3rd generation squarks}: stop/sbottom
        pair production, with several possibility for the decay
        (see Tab.~\ref{tab:simplified-models}). It includes T2tt, T2cc and T2bb.
\end{itemize}

Summary exclusion plots according to this categorisation are presented in
Fig.~\ref{fig:summary-excl-plots}.

\newpage
\begin{figure}[h!]
    \begin{center}
        \subfigure[T1bbbb: Upper limit on the cross section in the $(m_{\mathrm{Gluino}},m_{\mathrm{Susy}})$ plane]{
            \includegraphics[width=0.6\textwidth]{figures/susyResults/T1bbbbXSEC}
            \label{fig:T1bbbb_excl}
        } \\
        \subfigure[T1bbbb: $\epsilon_{sig}^{\mathrm{4\,cat}}$]{
            %\includegraphics[width=0.45\textwidth]{figures/jetRanking/T1bbbb/eff/T1bbbb_merging_4_cats}
            \label{fig:T1bbbb_eff}
        } ~~
        \caption{Top: the 95\% C.L. observed upper limit on the cross section
            (histogram), with the expected (solid black line) observed
            (solid red line) exclusion contours. Bottom left: signal acceptance
            including all jet categories.}
        \label{fig:T1bbbb}
    \end{center}
\end{figure}

\newpage
\begin{figure}[h!]
    \begin{center}
        \subfigure[T2tt: Upper limit on the cross section in the $(m_{\mathrm{Gluino}},m_{\mathrm{Susy}})$ plane]{
            \includegraphics[width=0.6\textwidth]{figures/susyResults/T2ttXSEC}
            \label{fig:T2tt_excl}
        } \\
        \subfigure[T2tt: $\epsilon_{sig}^{\mathrm{4\,cat}}$]{
            %\includegraphics[width=0.45\textwidth]{figures/jetRanking/T2tt/eff/T2tt_merging_4_cats}
            \label{fig:T2tt_eff}
        } ~~
        \caption{Top: the 95\% C.L. observed upper limit on the cross section
            (histogram), with the expected (solid black line) observed
            (solid red line) exclusion contours. Bottom left: signal acceptance
            including all jet categories.}
        \label{fig:T2tt}
    \end{center}
\end{figure}

\newpage
\begin{figure}[h!]
    \begin{center}
        \subfigure[T2cc: Upper limit on the cross section in the $(m_{\mathrm{Gluino}},m_{\mathrm{Susy}})$ plane]{
            \includegraphics[width=0.6\textwidth]{figures/susyResults/T2ccXSEC}
            \label{fig:T2cc_excl}
        } \\
        \subfigure[T2cc: $\epsilon_{sig}^{\mathrm{4\,cat}}$]{
            %\includegraphics[width=0.45\textwidth]{figures/jetRanking/T2cc/eff/T2cc_merging_4_cats}
            \label{fig:T2cc_eff}
        } ~~
        \caption{Top: the 95\% C.L. observed upper limit on the cross section
            (histogram), with the expected (solid black line) observed
            (solid red line) exclusion contours. Bottom left: signal acceptance
            including all jet categories.}
        \label{fig:T2cc}
    \end{center}
\end{figure}

\newpage
\begin{figure}[h!]
    \begin{center}
        \subfigure[T2bb: Upper limit on the cross section in the $(m_{\mathrm{Gluino}},m_{\mathrm{Susy}})$ plane]{
            \includegraphics[width=0.6\textwidth]{figures/susyResults/T2bbXSEC}
            \label{fig:T2bb_excl}
        } \\
        \subfigure[T2bb: $\epsilon_{sig}^{\mathrm{4\,cat}}$]{
            %\includegraphics[width=0.45\textwidth]{figures/jetRanking/T2bb/eff/T2bb_merging_4_cats}
            \label{fig:T2bb_eff}
        } ~~
        \caption{Top: the 95\% C.L. observed upper limit on the cross section
            (histogram), with the expected (solid black line) observed
            (solid red line) exclusion contours. Bottom left: signal acceptance
            including all jet categories.}
        \label{fig:T2bb}
    \end{center}
\end{figure}

\begin{figure*}[thp!]
    \begin{center}
        %\includegraphics[width=0.49\textwidth]{figures/susyResults/gluinoSUMMARY.pdf} \\
        %\includegraphics[width=0.49\textwidth]{figures/susyResults/allThirdGenSUMMARY.pdf} \\
        \caption{Summary for the observed (solid lines) and expected
            (dashed lines) exclusions in the $m_{\mathrm{Susy}},m_{\mathrm{LSP}}$
            plane for the models considered in the analysis.Exclusion contours
            are grouped into 4 summary plots according to the categorisation
            presented at the begin of Sec.~\ref{sec:susy_results}:
            ``Gluino-mediated production of off-shell (decoupled) 3rd generation
                squarks'' (top right),
            ``Direct production of 3rd generation squarks'' (bottom right).
        \label{fig:summary-excl-plots} }
    \end{center}
\end{figure*}

\newpage
Gluino masses up to $\sim$2050 GeV are excluded (T1bbbb). Stop production is
excluded up to $\sim$1050 GeV in the 2-body decay to top quarks (T2tt), and up
to $\sim$550 GeV in the compressed region, in the decay to charm quarks (T2cc).
Sbottom (squark) masses smaller than $\sim$1100 GeV are excluded for small LSP
masses (T2bb,T2qq). %For the models considered, the exclusion exceeds the one of
%the 8 TeV data.

%%____________________________________________________________________________||
