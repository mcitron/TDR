%%____________________________________________________________________________||
\section{Trigger strategy}
\label{sec:triggers}

\subsection{Signal regions\label{sec:hadronic_signal_region}}

In Run~2, the RA1 analysis retains low thresholds comparable to those
used during Run~1 with developments to the trigger selection,
maintaining sensitivity to signatures of new physics with hadronic
energies as low as $\scalht = 200$ GeV. This in part is achieved by a
migration to PF-based jet reconstruction within the HLT which, in
conjunction with a reduction of clustering radius parameter $\Delta R
= 0.4$, provides improvements in jet energy resolution in high-pileup
conditions and mitigates the effects of pileup contamination within
the jet cone.

The RA1 analysis utilises a range of triggers for the selection of
events in the hadronic signal region to provide coverage over a wide
range of event topologies. A guiding principle of the analysis is to
be as inclusive as possible by maintaining low online and offline
thresholds. 

A suite of cross-triggers of the form
\verb!HLT_PFHTXXX_PFDijetAveYYY_AlphaT0pZZ!, comprising requirements
on \scalht, \alphat, and the average \pt of the leading two jets, are
used to record candidate signal events. The jets considered by these
cross-triggers satisfy $\Pt > 40\GeV$ and $|\eta| < 3.0$. A further
cross-trigger, \verb!HLT_PFMETNoMuXX_PFMHTNoMuXX_IDTight!  comprising
requirements on \MET, \HTmiss, and the presence of a jet with $|\eta|
< 3.0$, as well as a single-object \scalht trigger, are also employed
to record candidate signal events. The trigger requirements and
thresholds are summarised in
Table~\ref{tab:2015_Hadronic_Signal_Triggers}.

%The use of a \pt-averaged threshold of the two leading jets suppresses
%the QCD multijet background while providing acceptance to events
%exhibiting asymmetric jet topologies, such as monojet-like signatures
%of compressed spectrum and DM models. It was found that a dijet
%average threshold of 90 \GeV ensured the optimum performance when
%balancing efficiency and rate, with both the $\scalht$ and $\alphat$
%thresholds across all jet topologies. The dijet average requirement
%does however lead to a loss in efficiency for asymmetric jet events
%where the sub-leading jet is soft.
%This loss in efficiency is mitigated by taking the disjunction of the
%\alt and monojet triggers which provides a recovery of efficiency in
%the turn-on and close to the plateau for the low-\scalht asymmetric
%categories.  

%The Level-1 seeds for the $\scalht$--$\alphat$ HLT paths are given by
%the disjunction of all available hadronic scalar energy and missing
%energy sum seeds for the given run scenario. A loose calorimeter
%trigger prefilter is utilised to reduce the pass-through rate prior to
%track-based reconstruction, ensuring the PF-based filters meet timing
%requirements. The calorimeter prefilter utilises loose \scalht and
%dijet average \pt requirements in addition to a new variable \alphat',
%defined as \alphat in the limit $\Delta\scalht \rightarrow 0$, which
%better correlates \alphat between calorimeter and PF-based
%reconstruction.

%The choice of threshold for the \scalht--\alphat triggers were tuned
%to maintain acceptance for a range of signal topologies whilst
%effectively suppressing QCD multijet events to maintain acceptable
%trigger rates.

\begin{table}[h!]
  \topcaption{Summary of the L1 seeds and HLT trigger paths used by
    the analysis. 
    The lowest and highest (in parentheses) thresholds used for each
    HLT trigger path during the 2016 data taking period are given. 
  } 
  \footnotesize
  \centering
\begin{tabular}{ll} 
  \hline
  L1 seed & HLT path \\
  \hline
  {\scriptsize\verb!HTT or ETM!} & {\scriptsize\verb!HLT_PFHT200_PFDijetAve90_AlphaT0p57(65)!} \\
  {\scriptsize\verb!HTT or ETM!} & {\scriptsize\verb!HLT_PFHT250_PFDijetAve90_AlphaT0p55(57)!} \\
  {\scriptsize\verb!HTT or ETM!} & {\scriptsize\verb!HLT_PFHT300_PFDijetAve90_AlphaT0p53(55)!} \\
  {\scriptsize\verb!HTT or ETM!} & {\scriptsize\verb!HLT_PFHT350_PFDijetAve90_AlphaT0p52(53)!} \\
  {\scriptsize\verb!HTT or ETM!} & {\scriptsize\verb!HLT_PFHT400_PFDijetAve90_AlphaT0p51(52)!} \\
  {\scriptsize\verb!HTT!}        & {\scriptsize\verb!HLT_PFHT800(900)!} \\
  {\scriptsize\verb!ETM!}        & {\scriptsize\verb!HLT_PFMETNoMu90(110)_PFMHTNoMu90(110)_IDTight!} \\
  \hline
\end{tabular}
\label{tab:2015_Hadronic_Signal_Triggers}
\end{table}

The efficiency of the signal triggers are measured in data using event
samples containing an electron and muon, recorded with unprescaled
electron and muon ``reference'' triggers and defined by a
signal-region-like selection criteria (when ignoring the lepton in the
computation of event-level variables, such as \scalht or \MET). Biases
in these measurements can be introduced due to the contamination in
the computation of event-level variables and different treatments
between trigger and offline reconstructions, the degree of which
varies with \scalht and \njet. In the case of efficiency measurements
using the electron reference trigger, no cross-cleaning of electrons
from jets is performed offline, with the electron being included in
the computation of event-level jet energy sums, such as \scalht, \MHT
and \alt. When using the muon reference triggers, offline
cross-cleaning of the muons is performed. 

The trigger efficiency as a function of \mht for the full suite of
triggers, as determined from the data samples described above, is
shown per \scalht bin (following the \scalht binning scheme of the
signal region) in Fig.~\ref{fig:alphat_turnons} (in
Appendix~\ref{app:triggers}). The figure contains efficiency curves
determined from both the muon and electron event samples. Significant
differences are visible in the turn-on region, due to the
cross-cleaning issues through the various steps of the
trigger--analysis chain (L1, HLT Calo-based pre-filter, HLT PF-based
decision, offline) as described above. However, the efficiencies are
close to 100\% for the requirement $\HTmiss > 200\GeV$, as illustrated
by Table~\ref{tab:trigger-eff}, which is now the default \HTmiss
requirement used in this analysis. The measured efficiencies and their
uncertainties are applied as corrections to the MC samples. The
central value of the correction is taken from the efficiency measured
with the electron reference trigger. The statistical uncertainties in
the central values, as well as the difference in efficiencies between
those measured with the electron and muon triggers, are propagated.

\begin{table}[h!]
  \topcaption{Efficiency of the full suite of signal triggers as a
    function of \scalht after applying the signal region selection
    criteria, determined from $e$ + jets and \mj event samples.
  } 
  \footnotesize
  \centering
  \begin{tabular}{lcccc} 
    \hline
    Event sample & \multicolumn{4}{c}{\scalht bin [GeV]}                                                       \\
    \cline{2-5}
                 & 200--400             & 400--600             & 600--900              & $>$900                \\
    \hline
    $e$ + jets\T & $97.4^{+0.5}_{-0.6}$ & $97.9^{+0.8}_{-1.2}$ & $100.0^{+0.0}_{-1.8}$ & $100.0^{+0.0}_{-3.6}$ \\
    \mj\B        & $98.0^{+0.0}_{-0.0}$ & $98.9^{+0.0}_{-0.0}$ & $99.4^{+0.1}_{-0.1}$  & $99.9^{+0.0}_{-0.1}$  \\
    \hline
  \end{tabular}
  \label{tab:trigger-eff}
\end{table}

\begin{table}[H]
\caption{The luminosity collected by each HLT, or combination of Triggers, from the entire 2016 run. Over the course of the run, the threshold for the lowest unprescaled trigger in each category rose. The fraction of luminosity collected by each Trigger whilst it was lowest threshold is included.}
\resizebox{\textwidth}{!}{
\begin{tabular}{lp{0.25\linewidth}p{0.33\linewidth}}
  \hline
  Trigger & Luminosity collected (fb$^{-1}$) & Fraction collected as lowest unprescaled trigger \\
  \hline
  \texttt{HLT\_IsoMu22 or HLT\_IsoTkMu22} & 28.6 & 0.80 \\
  \texttt{HLT\_IsoMu24 or HLT\_IsoTkMu24} & 35.9 & 0.20 \\
  \hline
  \texttt{HLT\_PFHT800} & 27.3 & 0.76 \\
  \texttt{HLT\_PFHT900} & 35.9 & 0.24 \\
  \hline
  \texttt{HLT\_PFMETNoMu90\_PFMHTNoMu90\_IDTight} & 13.9 & 0.39 \\
  \texttt{HLT\_PFMETNoMu100\_PFMHTNoMu100\_IDTight} & 17.6 & 0.10 \\
  \texttt{HLT\_PFMETNoMu110\_PFMHTNoMu110\_IDTight} & 35.3 & 0.49 \\
  \texttt{HLT\_PFMETNoMu120\_PFMHTNoMu120\_IDTight} & 35.9 & 0.02 \\
  \hline
\end{tabular}
}
\label{tab:triggerbylumi}
\end{table}

\subsection{Control regions\label{sec:control_samples}}

Samples of \mj and \mmj events are selected with the logical OR of the
\verb!HLT_IsoMu22!, \verb!HLT_IsoTkMu22!, \verb!HLT_IsoMu24! and
\verb!HLT_IsoTkMu24! triggers. The trigger efficiencies are provided
by the muon POG and are determined by the muon for both data and
simulated samples (containing a trigger emulation) using a tag and
probe method. The POG also provide data-to-simulation scale factors,
which we apply to the trigger-efficiency-corrected event yields from
simulation. The muon trigger efficiencies per (\njet, \nb, \scalht)
bin are determined for both the \mj and \mmj samples. For each bin,
the trigger efficiency values provided by the POG are weighted
according to \Pt and $\eta$ distributions of the muon(s) to give a
bin-averaged value. For the \mj sample, the efficiencies are typically
around 90\% and rather independent of \njet, \nb, and \scalht. For the
\mmj sample, the average efficiencies are $\sim99\%$ (as either muon
can provided the positive trigger decision). Statistical and
systematic uncertainties are propagated through via the scale factor
corrections, following the prescription from the muon POG.

A sample of \gj events is recorded with an \verb!OR! of the
\verb!HLT_Photon175! and \verb!HLT_ECALHT800! triggers. The efficiency
is measured as a function of both photon \Pt and \HTmiss for events in
the JetHT data set satisfying the \gj control region selection, as
shown in Fig.~\ref{fig:photon_turnons_photonPt}. The single photon
trigger alone exhibits a decreasing efficiency with increasing photon
\Pt, which is attributed to an H/E cut at Level 1. The inefficiency
can be largely recovered by also employing the \verb!ECALHT800!
trigger. In this way, the inefficiency can be recovered for photon
$\Pt > 600\GeV$. The photon efficiency is $\sim 98\%$. The
efficiencies are used to correct the simulated event counts in the \gj
sample, and a systematic uncertainty is assumed to be the magnitude of
the inefficiency ($\sim 2\%$).

%A sample of QCD multijet events is recorded with the same trigger
%requirements used for the signal region, summarised in
%Table~\ref{tab:2015_Hadronic_Signal_Triggers}. The trigger efficiency
%is determined from data using a sample of \gj events in which the
%photon is treated as a jet. 
%This sample is recorded with the \verb!HLT_Photon175! and
%\verb!HLT_ECALHT800! triggers.
%\fixme{QCD SIDEBAND TRIGGER EFF PLOTS?}

%%____________________________________________________________________________||
