%%____________________________________________________________________________||
\section{Physics objects}
\label{sec:objects}

The definitions of the physics objects used in this analysis follow
the recommendations of the various Physics Object Groups (POGs).
During data taking these recommendations are subject to change and
will be be updated if necessary.

\subsection{Jets}
\label{sec:jetreco}

Jets are defined as sets of particle-flow (PF) candidates clustered by
the anti-$k_{T}$ jet clustering algorithm \cite{Cacciari:2008gp} with
a distance parameter of 0.4 (PFJets). Charged Hadron Subtraction (CHS)
is applied, i.e., charged hadrons that can be traced back to pileup
vertices are not clustered.  The four-momenta of jets are initially
defined as the four-vector sum of the four-momenta of the constituent
particle-flow candidates and then scaled by the jet energy correction
factors designated as L1FastJet, L2Relative, and L3Absolute
\cite{Chatrchyan:2011ds}.

The ``loose'' working point Jet-Id selection criteria is chosen.  The
cuts are listed in Tab.~\ref{tab:loose-jet-id}.  In addition, a
dedicated selection is applied to reject ``beam halo'' candidate
events, as described in Section~\ref{sec:had-signal}.

\begin{table}[ht!]
  \caption{The ``loose'' jet ID requirements. \label{tab:loose-jet-id}}
  \centering
  \begin{tabular}{ ccc }
    \hline
    \hline
    Variable & cut & notes \\ \hline
    \multicolumn{3}{c}{$-3.0 < \eta_{\mathrm{jet}} < 3.0$} \\ \hline    
    Neutral Hadron Fraction & $<0.99$ & - \\
    Neutral EM Fraction & $<0.99$ & - \\
    Number of constituents & $>1$ & - \\
    Charged Hadron Fraction & $>0$ & only for $|\eta_{\mathrm{jet}}| < 2.4$ \\
    Charged Multiplicity & $>0$ & only for $|\eta_{\mathrm{jet}}| < 2.4$ \\
    Charged EM Fraction & $<0.99$ & only for $|\eta_{\mathrm{jet}}| < 2.4$ \\ \hline
    \multicolumn{3}{c}{$|\eta_{\mathrm{jet}}| > 3.0$} \\ \hline        
    Neutral EM Fraction & $<0.90$ & - \\
    Number of Neutral Particles & $>10$ & - \\
    \hline
    \hline
  \end{tabular}
\end{table}

\subsection{b-tagged jets}
\label{sec:btags}

Jets originating from bottom quarks are identified through vertices
that are displaced with respect to the primary interaction
\cite{Chatrchyan:2012jua}.  The algorithm used to tag b-jets is the
Combined Secondary Vertex tagger V2, with the ``medium'' working
point, which is achieved by requiring a cut of $>$ 0.800 on the
algorithm discriminator variable.  This results in a gluon/light-quark
mis-tag rate of $\sim$1 \% (where ``light'' means $u$, $d$ and $s$
quarks), a charm-quark mis-tag rate of $\sim$10 \% and a b-quark b-tag
efficiency of about 60 \%.

Some validation studies used by the analysis make use of the ``loose''
and ``tight'' working points, which are defined by the thresholds
requirements of 0.460 and 0.935 on the discriminator variable. These
requirements yield light-flavour mistag probabilities of $\sim$10 and
$\sim$0.1\%, respectively.

\subsection{Muons}
\label{sec:muon-id}

\subsubsection{Selection}

Muons are selected in the \mj and \mmj control regions using the
``tight'' working point definition of the recommended identification
algorithm from the Muon POG.  Muons are also required to be well
isolated, i.e. with a low activity in the vicinity of their track.
The transverse momenta of PF neutral and charged candidates, as well
as photons, lying within a cone around the lepton are summed.  The
relative combined isolation $I^{rel}_{comb}$ is then defined as the
ratio of this scalar sum to the transverse momentum of the lepton
candidate. Additionally, $\rho\times A_{eff}$ corrections are applied
to remove the effects of pileup.  Muons are defined to be isolated if
they fulfill the criterium $I^{rel}_{comb} < 0.15$.

\subsubsection{Veto}

For the purpose of vetoing muons in the signal region, the ``loose''
working point is used, which provides $\sim$ 98 $\%$ efficiency.  In
the hadronic signal region a variable cone size for the isolation is
used, which is referred to as ``mini-isolation''.  This isolation
algorithm helps in recovering some efficiency in the lepton selection
for boosted topology of top quark decays, in which the muon's track
may be found close to the jet activity due to the boost of the parent
top.  Therefore, the cone size used for the calculation of the lepton
isolation is reduced as a function of the lepton \Pt, as follows:
$R=0.2$ for $\Pt_{\ell}\leq50\gev$, $R=10\gev/\Pt_{\ell}$ for $50 \gev
< \Pt_{\ell} < 200\gev$ and $R=0.05$ for $\Pt_{\ell} > 200 \gev$.  In
the signal region, identified muons with mini-isolation satisfying
$I^{rel}_{comb} < 0.2$ are vetoed.

\subsection{Photons}
\label{sec:photon-id}

\begin{table}[ht!]
  \caption{Photon identification.\label{tab:photon-id-gamma}}
  \centering
  \footnotesize
  \begin{tabular}{ ccc }
    \hline
    \hline
    Categories & \multicolumn{2}{c}{Barrel}   \\
    Working point  & Tight & Loose \\
    \hline
    Conversion safe electron veto & Yes & Yes  \\
    Single Tower H/E              & 0.05 & 0.05  \\
    $\sigma_{i\eta i\eta}$        & 0.0100 & 0.0102 \\
    PF charged hadron isolation   & 0.76 & 3.32  \\
    PF neutral hadron isolation   & 0.97 + 0.014 $\times$ $p_{\mathrm{T},\gamma}$ + 0.000019 $\times$ $p_{\mathrm{T},\gamma}^{2}$ & 1.92 + 0.014 $\times$ $p_{\mathrm{T},\gamma}$ + 0.000019 $\times$ $p_{\mathrm{T},\gamma}^{2}$  \\
    PF photon isolation           & 0.08 + 0.0053 $\times$ $p_{\mathrm{T},\gamma}$ & 0.81 + 0.0053 $\times$ $p_{\mathrm{T},\gamma}$ \\
    \hline
    \hline
    Categories & \multicolumn{2}{c}{Endcap}   \\
    Working point  & Tight & Loose \\
    \hline
    Conversion safe electron veto & Yes & Yes  \\
    Single Tower H/E              & 0.05 & 0.05  \\
    $\sigma_{i\eta i\eta}$        & 0.0268 & 0.0274 \\
    PF charged hadron isolation   & 0.56 & 1.97  \\
    PF neutral hadron isolation   & 2.09 + 0.014 $\times$ $p_{\mathrm{T},\gamma}$ + 0.000025 $\times$ $p_{\mathrm{T},\gamma}^{2}$ & 11.86 + 0.014 $\times$ $p_{\mathrm{T},\gamma}$ + 0.000025 $\times$ $p_{\mathrm{T},\gamma}^{2}$ \\
    PF photon isolation           &  0.16 + 0.0034 $\times$ $p_{\mathrm{T},\gamma}$ & 0.83 + 0.0034 $\times$ $p_{\mathrm{T},\gamma}$ \\
    \hline
    \hline
  \end{tabular}
  \end{table}

\subsubsection{Selection}

Photons are identified according to the ``tight'' working point
definition ($\sim$ 71 $\%$ efficiency) of the simple cut-based photon
identification algorithm \cite{photon-id} and required to be well
isolated.  A PF-based isolation is used with a cone size $\Delta R$
$<$ 0.3 and $\rho\times A_{eff}$ corrections are applied to remove the
effects of pileup \cite{pf-photon}.  Table \ref{tab:photon-id-gamma}
summarises the identification and isolation selection used.

\subsubsection{Veto}

Photons are vetoed in the definition of the hadronic signal region and
muon control regions, as described in Sec.~\ref{sec:preSelection},
while a control region with one photon (``\gj'') is defined for the
purpose of the background estimation, as described in
Sec.~\ref{sec:photoncontrolSelection}.


\subsection{Electrons}
\label{sec:electron-id}

\begin{table}[h!]
  \caption{Electron identification (``tight'' working point).\label{tab:ele-id}}
  \centering
  \footnotesize
  \begin{tabular}{ lcc }
    \hline
    \hline
    Categories                                               & Barrel    & EndCap    \\
    \hline
    $\Delta \eta_{In}$                                       & 0.0105   & 0.00814  \\
    $\Delta \phi_{In}$                                       & 0.115    & 0.182  \\
    $\sigma_{i\eta i\eta}$                                   & 0.0103    & 0.0301  \\
    H/E                                                      & 0.104    & 0.0897   \\
    d0 (vtx)                                                 & 0.0261    & 0.118  \\
    dZ (vtx)                                                 & 0.041    & 0.822  \\
    $\lvert(1/E_{\textrm{ECAL}} - 1/p_{\textrm{trk}})\rvert$ & 0.102     & 0.126  \\
    Missing hits (inner tracker)                             & 2         & 1         \\
    Conversion veto                                          & yes       & yes   \\
    \hline
    \hline
  \end{tabular}
\end{table}

\subsubsection{Selection}

Electron-based control regions are not used by this analysis. 

\subsubsection{Veto}

In order to veto electrons in the hadronic signal region and the muon
and photon control regions, the ``loose'' working point definition
($\sim$ 90 $\%$ efficiency) of the cut-based electron identification
\cite{electron-id} is used.  Electrons are also require required to be
isolated.  Similar to muons, in the hadronic signal regions a PF-based
isolation \cite{pf-photon} is used with a cone size determined by the
mini isolation algorithm (see Sec.~\ref{sec:muon-id}) and $\rho\times
A_{eff}$ corrections are applied to remove the effects of pileup.
Isolated electrons are defined by $I^{rel}_{comb} < 0.1$.

Table \ref{tab:ele-id} summarises the identification requirements. 

\subsection{Single isolated tracks}
\label{sec:SIT}

\subsubsection{Selection}

Event samples containing only a single isolated track (and no leptons)
are not used by this analysis, other than for ``closure test''
studies, see Sec.~\ref{sec:}. In this case, the veto definition below
is simply inverted.

\subsubsection{Veto}

A single isolated track (SIT) can be used to identify W bosons through
their leptonic decays: W $\rightarrow$ $\mu \nu$, W $\rightarrow$
$e\nu$, and W $\rightarrow$ $\tau$($\rightarrow l$) $\nu$.  Single
prong decays of the tau lepton can also be identified: W $\rightarrow$
$\tau$ ($\rightarrow$ h$^{\pm}$ + n$\pi^{0}$) $\nu$.  A single
isolated track comprises a charged PF candidate with $\Pt > 10 \gev$,
$\Delta z(\mathrm{track}, \mathrm{PV}) < 0.05 \, \mathrm{cm}$ and with
a relative isolation smaller than 0.1, where the isolation is
determined from the sum of the \Pt of the charged PF candidates within
$\Delta R < 0.3$.

\subsection{Missing transverse momentum}

As indicated in Sec.~\ref{sec:}, the transverse momentum imbalance of
an event (\met) is defined as the magnitude of the vector sum of the
transverse momentum of all particle-flow candidates in the event.

The Type-I \met correction \cite{Khachatryan:2014gga} is applied, \ie
the transverse momentum of the particle-flow candidates clustered as
jets are replaced with the transverse momentum of the jets that are
scaled by the jet energy correction factors.

The \met is used in the definition of the transverse mass, $M_{T}$,
which is in turn used as part of the selection criteria that define
the single muon control sample (Sec.~\ref{sec:mucontrolSelection}),
and for the $\mhtmet$ cleaning filter, as described in
Sec.~\ref{sec:selection}.

%%____________________________________________________________________________||
