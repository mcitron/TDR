%%____________________________________________________________________________||

\section{``Lost lepton'' background estimation}
\label{sec:ttw}

Once all the signal region selection requirements have been imposed,
the contribution from QCD multijet events is expected to be
negligible, as demonstrated in Sec.~\ref{sec:qcd}. In the absence of
multijet events, the background counts in the signal region arise from
SM processes with significant \met in the final state. In events with
low counts of jets and b-quark jets, the largest backgrounds with
genuine \met are from the associated production of W or Z bosons with
jets, followed by either the weak decays \znunu or \wtaunu, where the
$\tau$ decays hadronically and is identified as a jet; or by leptonic
decays that are not rejected by the dedicated electron or muon
vetoes. The veto of events containing isolated tracks is efficient at
further suppressing these backgrounds as well as the single-prong
hadronic decay of the tau lepton. At higher jet and b-quark jet
multiplicities, top quark production followed by semileptonic weak top
quark decay becomes important.  Residual contributions from processes
such as single-top-quark, $\ttbar$V or $\ttbar$H, diboson, and
Drell-Yan production are also expected. These SM processes are
collectively referred to as the non-multijet backgrounds. 

The ``lost lepton'' background consists primarily of processes
yielding at least one W boson, which decays leptonically and the
result lepton is either outside the experimental acceptance or does
not satisfy requirements related to reconstruction ``quality'' or
isolation. Predominantly, the lost lepton background arises from the
\wj and \ttbar processes, which smaller contributions from processes
such as single top, {\ttbar}V, {\ttbar}H, \etc All these processes,
plus any others that yield non-negligible contributions are estimated
using the \mj control region. (The \mmj sample is reserved to estimate
solely the \znunu\ background.)

\subsection{The ``transfer factor'' method}
\label{sec:ewk-method}

The method used to estimate the lost lepton background relies on the
use of a transfer factor (TF) determined from simulated event samples
to transform the observed event yields in the \mj sample, categorised
according to \njet, \scalht, and \nb, $\nobs^{\mj}(\njet, \scalht,
\nb)$, into an estimate of the lost lepton background for the
corresponding (\njet, \scalht, \nb) category of the signal region,
$\npre^{\ttw}(\njet, \scalht, \nb)$. The categorisation of events
according to \njet, \scalht, and \nb in the \mj control region is
identical to that for the signal region, as defined in
Sec.~\ref{sec:selection}.\footnote{One detail here is that predictions
  are made according to the \scalht categorisation used in the control
  regions (up to 11 bins), but the predictions are then aggregated
  over a range in \scalht to map onto a coarser binning scheme is used
  in the signal region (up to 5 bins). Further details can be found in
  Sec.~\ref{sec:ht-categorisation}.}  Each transfer factor is simply a
ratio of the yields obtained from Monte Carlo (MC) simulation for the
same (\njet, \scalht, \nb) category of the \mj control and signal
regions:
\begin{equation}
  \label{equ:tf-ratio}
  {\rm TF} = \frac{N_{\rm MC}^{\ttw}(\njet, \scalht, \nb)}{N_{\rm MC}^{\mj}(\njet, \scalht, \nb)} 
\end{equation}

In this way, predictions for the lost lepton background can be made,
based on the \mj control region, as follows:
\begin{equation}
  \label{equ:pred-method}
  \npre^{\ttw}(\njet, \scalht, \nb) = 
  \frac{N_{\rm MC}^{\ttw}(\njet, \scalht, \nb)}{N_{\rm MC}^{\mj}(\njet, \scalht, \nb)} 
  \times 
  \nobs^{\mj}(\njet, \scalht, \nb)   
\end{equation}

The selection criteria for the \mj control region
(Sec.~\ref{sec:selection}) closely resemble those for the signal
region, differing mainly through the use of a muon {\it tag} (that is
ignored in the calculation of jet-based kinematic variables such as
\scalht, \mht, \alphat, \etc) and minimal additional kinematic
requirements (\eg a transverse mass window) to obtain a sample
enriched in \wj and \ttbar events. The same selection criteria are
designed to suppress signal contamination so that unbiased data-driven
estimates for the SM backgrounds in the signal region can be
made. Any signal contamination is accounted for in the likelihood
model (Sec.~\ref{sec:likelihood}). 

Figure~\ref{fig:tf_muToTtw} (App.~\ref{app:ttw}) shows the magnitude
of the transfer factors, typically $\ll 1$, The transfer factors
account for differences between the \mj control and signal due to the
acceptance times efficiency for the reconstructed muon and
extrapolations in kinematic quantities. For example, the dependence on
\njet, \scalht, and \nb is largely attributable to the \alphat and
\bdphi requirements for the signal region (that are not used in the
\mj control region).

Many systematic effects are expected to cancel largely in the transfer
factor. However, a systematic uncertainty is assigned to each transfer
factor to account for theoretical uncertainties and effects such as
the mismodelling of kinematics (\eg acceptances) and instrumental
effects (\eg reconstruction efficiencies). These uncertainties are
discussed below. 

%In the end, a fitting procedure that provides the final result is
%defined formally by the likelihood model described in 
%Sec.~\ref{sec:likelihood}.
%In summary, the observation in each bin (defined in terms of the
%variables \njet, \scalht, and \nb) of the signal sample is modelled as
%Poisson-distributed about the sum of a SM expectation (and a potential
%signal contribution). The components of this SM expectation are
%related to the expected yields in the control samples via transfer
%factors derived from simulation. The observations in each bin (again
%defined by \njet, \nb, and \scalht) of the control samples are
%similarly modelled as Poisson-distributed about the expected yields
%for each control sample. In this way, for a given bin, the observed
%yields in the signal and control samples are connected via the
%transfer factors derived from simulation.
%The procedure to determine the systematic uncertainties associated
%with these transfer factors is described in
%Sec.~\ref{sec:systematics}.


%% The transfer factors are shown in tables below.

%% \begin{table}[h!]
\tiny
\centering
\caption{Transfer factors from the \mj control region to the \zInv~ background for symmetric categories.\label{tab:tf_mu_zinv_sym}}
\scalebox{0.85}{\begin{tabular}{ccccccccc}
	\hline\hline
	& \multicolumn{8}{c}{\scalht (\gev)} \\ 
	 (\njet,  \nb) & 200-250 & 250-300 & 300-350 & 350-400 & 400-500 & 500-600 & 600-800 & 800-$\infty$ \\ [0.8ex] 
\hline
	(2, 0) & $1.00\pm 0.02$ & $0.76\pm 0.01$ & $0.57\pm 0.01$ & $0.39\pm 0.01$ & $0.30\pm 0.01$ & $0.21\pm 0.01$ & $0.14\pm 0.00$ & $0.28\pm 0.01$ \\[0.5ex] 
	(2, 1) & $0.48\pm 0.03$ & $0.51\pm 0.02$ & $0.54\pm 0.03$ & $0.44\pm 0.03$ & $0.33\pm 0.02$ & $0.26\pm 0.02$ & $0.20\pm 0.01$ & $0.43\pm 0.03$ \\[0.5ex] 
	(2, 2) & $0.68\pm 0.13$ & $0.77\pm 0.14$ & $0.68\pm 0.14$ & $0.48\pm 0.12$ & $0.33\pm 0.07$ & $0.37\pm 0.12$ & $0.22\pm 0.06$ & -- \\[0.5ex] 
	(3, 0) & $0.19\pm 0.08$ & $0.45\pm 0.01$ & $0.53\pm 0.01$ & $0.53\pm 0.01$ & $0.42\pm 0.01$ & $0.27\pm 0.01$ & $0.19\pm 0.00$ & $0.26\pm 0.00$ \\[0.5ex] 
	(3, 1) & -- & $0.10\pm 0.01$ & $0.16\pm 0.01$ & $0.17\pm 0.01$ & $0.20\pm 0.01$ & $0.18\pm 0.01$ & $0.15\pm 0.01$ & $0.26\pm 0.01$ \\[0.5ex] 
	(3, 2) & -- & $0.08\pm 0.02$ & $0.10\pm 0.01$ & $0.08\pm 0.01$ & $0.08\pm 0.01$ & $0.07\pm 0.01$ & $0.07\pm 0.01$ & $0.14\pm 0.02$ \\[0.5ex] 
	(3, $\ge3$) & -- & -- & -- & -- & $0.03\pm 0.03$ & -- & -- & -- \\[0.5ex] 
	(4, 0) & -- & -- & $0.51\pm 0.03$ & $0.54\pm 0.02$ & $0.44\pm 0.01$ & $0.33\pm 0.01$ & $0.23\pm 0.00$ & $0.24\pm 0.00$ \\[0.5ex] 
	(4, 1) & -- & -- & $0.13\pm 0.01$ & $0.11\pm 0.01$ & $0.11\pm 0.00$ & $0.11\pm 0.01$ & $0.10\pm 0.00$ & $0.17\pm 0.01$ \\[0.5ex] 
	(4, 2) & -- & -- & $0.07\pm 0.02$ & $0.04\pm 0.01$ & $0.05\pm 0.00$ & $0.04\pm 0.00$ & $0.05\pm 0.00$ & $0.09\pm 0.01$ \\[0.5ex] 
	(4, $\ge3$) & -- & -- & -- & $0.06\pm 0.04$ & $0.07\pm 0.02$ & $0.03\pm 0.02$ & $0.03\pm 0.01$ & $0.12\pm 0.05$ \\[0.5ex] 
	($\ge5$, 0) & -- & -- & -- & $0.31\pm 0.04$ & $0.36\pm 0.01$ & $0.27\pm 0.01$ & $0.19\pm 0.00$ & $0.18\pm 0.00$ \\[0.5ex] 
	($\ge5$, 1) & -- & -- & -- & $0.07\pm 0.02$ & $0.07\pm 0.01$ & $0.05\pm 0.00$ & $0.05\pm 0.00$ & $0.06\pm 0.00$ \\[0.5ex] 
	($\ge5$, 2) & -- & -- & -- & $0.01\pm 0.01$ & $0.03\pm 0.00$ & $0.02\pm 0.00$ & $0.02\pm 0.00$ & $0.03\pm 0.00$ \\[0.5ex] 
	($\ge5$, $\ge3$) & -- & -- & -- & -- & $0.03\pm 0.02$ & $0.02\pm 0.01$ & $0.03\pm 0.01$ & $0.03\pm 0.01$ \\[0.5ex] 
	\hline
	\hline
\end{tabular}}
\end{table}

%% \begin{table}[h!]
\tiny
\centering
\caption{Transfer factors from the \mj control region to the \zInv~ background for asymmetric categories. The letter ``a'' in jet \eg ``2a''  indicates the asymmetric jet bins. All entries are non-zero but are truncated to one decimal place.\label{tab:tf_mu_zinv_asym}}
\begin{tabular}
{ccccccccc}
	\hline\hline
&	& \multicolumn{8}{c}{\scalht (\gev)} \\ 
	 (\njet,  \nb) & 200-250 & 250-300 & 300-350 & 350-400 & 400-500 & 500-600 & 600-800 & 800-$\infty$ \\ [0.8ex] 
\hline
	(2a, 0) & $0.55^{+ 0.01 }_{- 0.01 }$ & $0.34^{+ 0.01 }_{- 0.01 }$ & $0.30^{+ 0.01 }_{- 0.01 }$ & $0.30^{+ 0.01 }_{- 0.01 }$ & $0.30^{+ 0.01 }_{- 0.01 }$ & $0.28^{+ 0.01 }_{- 0.01 }$ & $0.25^{+ 0.01 }_{- 0.01 }$ & -- \\[0.5ex] 
	(2a, 1) & $0.22^{+ 0.01 }_{- 0.01 }$ & $0.18^{+ 0.01 }_{- 0.01 }$ & $0.16^{+ 0.01 }_{- 0.01 }$ & $0.21^{+ 0.02 }_{- 0.02 }$ & $0.18^{+ 0.02 }_{- 0.02 }$ & $0.20^{+ 0.03 }_{- 0.03 }$ & $0.19^{+ 0.03 }_{- 0.03 }$ & -- \\[0.5ex] 
	(2a, 2) & $0.21^{+ 0.02 }_{- 0.02 }$ & $0.12^{+ 0.02 }_{- 0.02 }$ & $0.12^{+ 0.03 }_{- 0.03 }$ & $0.07^{+ 0.02 }_{- 0.02 }$ & $0.09^{+ 0.04 }_{- 0.04 }$ & $0.07^{+ 0.05 }_{- 0.05 }$ & $0.07^{+ 0.06 }_{- 0.06 }$ & -- \\[0.5ex] 
	(3a, 0) & $0.57^{+ 0.01 }_{- 0.01 }$ & $0.44^{+ 0.01 }_{- 0.01 }$ & $0.42^{+ 0.01 }_{- 0.01 }$ & $0.32^{+ 0.01 }_{- 0.01 }$ & $0.23^{+ 0.01 }_{- 0.01 }$ & $0.19^{+ 0.01 }_{- 0.01 }$ & $0.18^{+ 0.01 }_{- 0.01 }$ & -- \\[0.5ex] 
	(3a, 1) & $0.10^{+ 0.00 }_{- 0.00 }$ & $0.09^{+ 0.00 }_{- 0.00 }$ & $0.10^{+ 0.00 }_{- 0.00 }$ & $0.10^{+ 0.01 }_{- 0.01 }$ & $0.08^{+ 0.01 }_{- 0.01 }$ & $0.04^{+ 0.01 }_{- 0.01 }$ & $0.06^{+ 0.01 }_{- 0.01 }$ & -- \\[0.5ex] 
	(3a, 2) & $0.04^{+ 0.00 }_{- 0.00 }$ & $0.04^{+ 0.00 }_{- 0.00 }$ & $0.05^{+ 0.01 }_{- 0.01 }$ & $0.03^{+ 0.01 }_{- 0.01 }$ & $0.05^{+ 0.01 }_{- 0.01 }$ & $0.04^{+ 0.02 }_{- 0.02 }$ & $0.03^{+ 0.01 }_{- 0.01 }$ & -- \\[0.5ex] 
	(3a, $\ge3$) & $0.02^{+ 0.01 }_{- 0.01 }$ & $0.04^{+ 0.02 }_{- 0.02 }$ & -- & -- & -- & -- & -- & -- \\[0.5ex] 
	(4a, 0) & $0.16^{+ 0.04 }_{- 0.04 }$ & $0.28^{+ 0.01 }_{- 0.01 }$ & $0.43^{+ 0.01 }_{- 0.01 }$ & $0.42^{+ 0.02 }_{- 0.02 }$ & $0.39^{+ 0.01 }_{- 0.01 }$ & $0.24^{+ 0.02 }_{- 0.02 }$ & $0.15^{+ 0.02 }_{- 0.02 }$ & -- \\[0.5ex] 
	(4a, 1) & $0.02^{+ 0.01 }_{- 0.01 }$ & $0.05^{+ 0.00 }_{- 0.00 }$ & $0.06^{+ 0.00 }_{- 0.00 }$ & $0.07^{+ 0.01 }_{- 0.01 }$ & $0.09^{+ 0.01 }_{- 0.01 }$ & $0.05^{+ 0.01 }_{- 0.01 }$ & $0.04^{+ 0.01 }_{- 0.01 }$ & -- \\[0.5ex] 
	(4a, 2) & -- & $0.03^{+ 0.01 }_{- 0.01 }$ & $0.02^{+ 0.00 }_{- 0.00 }$ & $0.02^{+ 0.00 }_{- 0.00 }$ & $0.03^{+ 0.00 }_{- 0.00 }$ & $0.02^{+ 0.01 }_{- 0.01 }$ & $0.00^{+ 0.00 }_{- 0.00 }$ & -- \\[0.5ex] 
	(4a, $\ge3$) & -- & $0.02^{+ 0.01 }_{- 0.01 }$ & $0.02^{+ 0.01 }_{- 0.01 }$ & $0.02^{+ 0.01 }_{- 0.01 }$ & -- & -- & -- & -- \\[0.5ex] 
	($\ge5$a, 0) & -- & -- & $0.40^{+ 0.04 }_{- 0.04 }$ & $0.38^{+ 0.02 }_{- 0.02 }$ & $0.32^{+ 0.02 }_{- 0.02 }$ & $0.22^{+ 0.02 }_{- 0.02 }$ & $0.21^{+ 0.03 }_{- 0.03 }$ & -- \\[0.5ex] 
	($\ge5$a, 1) & -- & -- & $0.05^{+ 0.01 }_{- 0.01 }$ & $0.05^{+ 0.01 }_{- 0.01 }$ & $0.04^{+ 0.00 }_{- 0.00 }$ & $0.03^{+ 0.01 }_{- 0.01 }$ & $0.02^{+ 0.01 }_{- 0.01 }$ & -- \\[0.5ex] 
	($\ge5$a, 2) & -- & -- & $0.02^{+ 0.01 }_{- 0.01 }$ & $0.02^{+ 0.00 }_{- 0.00 }$ & $0.02^{+ 0.00 }_{- 0.00 }$ & $0.02^{+ 0.00 }_{- 0.00 }$ & $0.01^{+ 0.00 }_{- 0.00 }$ & -- \\[0.5ex] 
	($\ge5$a, $\ge3$) & -- & -- & -- & $0.00^{+ 0.00 }_{- 0.00 }$ & $0.02^{+ 0.01 }_{- 0.01 }$ & $0.03^{+ 0.02 }_{- 0.02 }$ & -- & -- \\[0.5ex] 
	\hline
	\hline
\end{tabular}
\end{table}

%% \begin{table}[h!]
\tiny
\centering
\caption{Transfer factors from the \mj control region to the \zInv~ background for monojet categories.\label{tab:tf_mu_zinv_mono}}
\scalebox{0.85}{\begin{tabular}{ccccccccc}
	\hline\hline
	& \multicolumn{8}{c}{\scalht (\gev)} \\ 
	 (\njet,  \nb) & 200-250 & 250-300 & 300-350 & 350-400 & 400-500 & 500-600 & 600-800 & 800-$\infty$ \\ [0.8ex] 
\hline
	(1, 0) & $1.36\pm 0.01$ & $1.33\pm 0.01$ & $1.30\pm 0.02$ & $1.24\pm 0.02$ & $1.09\pm 0.02$ & $1.17\pm 0.06$ & $1.45\pm 0.17$ & -- \\[0.5ex] 
	(1, 1) & $1.38\pm 0.03$ & $1.45\pm 0.06$ & $1.33\pm 0.08$ & $1.29\pm 0.11$ & $1.11\pm 0.11$ & $0.68\pm 0.15$ & -- & -- \\[0.5ex] 
	\hline
	\hline
\end{tabular}}
\end{table}

%% \begin{table}[h!]
\tiny
\centering
\caption{Transfer factors from the \gj control region to the \zInv~ background for symmetric categories.\label{tab:tf_gj_zinv_sym}}
\scalebox{0.85}{\begin{tabular}{ccccc}
	\hline\hline
	& \multicolumn{4}{c}{\scalht (\gev)} \\ 
	 (\njet,  \nb) & 400-500 & 500-600 & 600-800 & 800-$\infty$ \\ [0.8ex] 
\hline
	(2, 0) & $0.78\pm 0.04$ & $0.72\pm 0.06$ & $0.54\pm 0.04$ & $0.28\pm 0.01$ \\[0.5ex] 
	(2, 1) & $0.82\pm 0.16$ & $0.71\pm 0.20$ & $0.78\pm 0.18$ & $0.24\pm 0.03$ \\[0.5ex] 
	(2, 2) & $0.64\pm 0.42$ & $1.93\pm 1.63$ & $0.52\pm 0.55$ & -- \\[0.5ex] 
	(3, 0) & $0.72\pm 0.04$ & $0.64\pm 0.04$ & $0.67\pm 0.03$ & $0.26\pm 0.01$ \\[0.5ex] 
	(3, 1) & $0.78\pm 0.10$ & $0.68\pm 0.11$ & $1.02\pm 0.17$ & $0.26\pm 0.03$ \\[0.5ex] 
	(3, 2) & $0.71\pm 0.28$ & $0.89\pm 0.55$ & $0.63\pm 0.28$ & $0.21\pm 0.07$ \\[0.5ex] 
	(3, $\ge3$) & -- & -- & -- & -- \\[0.5ex] 
	(4, 0) & $0.83\pm 0.05$ & $0.73\pm 0.05$ & $0.58\pm 0.03$ & $0.27\pm 0.01$ \\[0.5ex] 
	(4, 1) & $0.72\pm 0.11$ & $0.85\pm 0.15$ & $0.69\pm 0.09$ & $0.29\pm 0.03$ \\[0.5ex] 
	(4, 2) & $0.75\pm 0.25$ & $0.70\pm 0.30$ & $0.31\pm 0.10$ & $0.27\pm 0.06$ \\[0.5ex] 
	(4, $\ge3$) & $577.40\pm 664.02$ & $0.38\pm 0.44$ & $0.26\pm 0.31$ & $0.04\pm 0.05$ \\[0.5ex] 
	($\ge5$, 0) & $0.89\pm 0.13$ & $0.71\pm 0.08$ & $0.57\pm 0.04$ & $0.30\pm 0.01$ \\[0.5ex] 
	($\ge5$, 1) & $0.79\pm 0.22$ & $0.57\pm 0.12$ & $0.48\pm 0.07$ & $0.26\pm 0.02$ \\[0.5ex] 
	($\ge5$, 2) & $0.85\pm 0.54$ & $0.40\pm 0.16$ & $0.43\pm 0.15$ & $0.22\pm 0.04$ \\[0.5ex] 
	($\ge5$, $\ge3$) & -- & -- & $1.40\pm 1.47$ & $0.52\pm 0.33$ \\[0.5ex] 
	\hline
	\hline
\end{tabular}}
\end{table}

%% \begin{table}[h!]
\tiny
\centering
\caption{Transfer factors from the \gj control region to the \zInv~ background for asymmetric categories.\label{tab:tf_gj_zinv_asym}}
\begin{tabular}
{ccccc}
	\hline\hline
	& \multicolumn{4}{c}{\scalht (\gev)} \\ 
	 (\njet,  \nb) & 400-500 & 500-600 & 600-800 & 800-$\infty$ \\ [0.8ex] 
\hline
	(2a, 0) & $0.65^{+ 0.05 }_{- 0.05 }$ & $0.56^{+ 0.08 }_{- 0.08 }$ & $0.54^{+ 0.07 }_{- 0.07 }$ & -- \\[0.5ex] 
	(2a, 1) & $1.04^{+ 0.35 }_{- 0.35 }$ & $0.54^{+ 0.24 }_{- 0.24 }$ & -- & -- \\[0.5ex] 
	(2a, 2) & -- & -- & -- & -- \\[0.5ex] 
	(3a, 0) & $0.71^{+ 0.08 }_{- 0.08 }$ & $0.61^{+ 0.12 }_{- 0.12 }$ & $0.76^{+ 0.17 }_{- 0.17 }$ & -- \\[0.5ex] 
	(3a, 1) & $0.68^{+ 0.22 }_{- 0.22 }$ & $0.26^{+ 0.13 }_{- 0.13 }$ & $0.67^{+ 0.45 }_{- 0.45 }$ & -- \\[0.5ex] 
	(3a, 2) & $0.68^{+ 0.43 }_{- 0.43 }$ & $1.08^{+ 1.22 }_{- 1.22 }$ & -- & -- \\[0.5ex] 
	(3a, $\ge3$) & -- & -- & -- & -- \\[0.5ex] 
	(4a, 0) & $0.75^{+ 0.08 }_{- 0.08 }$ & $0.69^{+ 0.15 }_{- 0.15 }$ & $0.42^{+ 0.13 }_{- 0.13 }$ & -- \\[0.5ex] 
	(4a, 1) & $0.87^{+ 0.21 }_{- 0.21 }$ & $0.36^{+ 0.18 }_{- 0.18 }$ & $0.50^{+ 0.36 }_{- 0.36 }$ & -- \\[0.5ex] 
	(4a, 2) & $0.95^{+ 0.63 }_{- 0.63 }$ & -- & -- & -- \\[0.5ex] 
	(4a, $\ge3$) & -- & -- & -- & -- \\[0.5ex] 
	($\ge5$a, 0) & $0.66^{+ 0.09 }_{- 0.09 }$ & $0.69^{+ 0.18 }_{- 0.18 }$ & $0.83^{+ 0.27 }_{- 0.27 }$ & -- \\[0.5ex] 
	($\ge5$a, 1) & $0.84^{+ 0.34 }_{- 0.34 }$ & $0.57^{+ 0.28 }_{- 0.28 }$ & $1.90^{+ 1.88 }_{- 1.88 }$ & -- \\[0.5ex] 
	($\ge5$a, 2) & $0.72^{+ 0.55 }_{- 0.55 }$ & $276.67^{+ 282.97 }_{- 282.97 }$ & -- & -- \\[0.5ex] 
	($\ge5$a, $\ge3$) & -- & -- & -- & -- \\[0.5ex] 
	\hline
	\hline
\end{tabular}
\end{table}

%% \begin{table}[h!]
\tiny
\centering
\caption{Transfer factors from the \gj control region to the \zInv~ background for monojet categories.\label{tab:tf_gj_zinv_mono}}
\begin{tabular}
{ccccc}
	\hline\hline
	& \multicolumn{4}{c}{\scalht (\gev)} \\ 
	 (\njet,  \nb) & 400-500 & 500-600 & 600-800 & 800-$\infty$ \\ [0.8ex] 
\hline
	(1, 0) & $0.57^{+ 0.02 }_{- 0.02 }$ & $0.55^{+ 0.03 }_{- 0.03 }$ & $0.52^{+ 0.03 }_{- 0.03 }$ & -- \\[0.5ex] 
	(1, 1) & $0.59^{+ 0.11 }_{- 0.11 }$ & $0.56^{+ 0.17 }_{- 0.17 }$ & -- & -- \\[0.5ex] 
	\hline
	\hline
\end{tabular}
\end{table}

%% \begin{table}[h!]
\tiny
\centering
\caption{Transfer factors from the \mj control region to the \ttbar/W background for symmetric categories. The letter ``a'' in jet \eg ``2a''  indicates the asymmetric jet bins. All entries are non-zero but are truncated to one decimal place.\label{tab:tf_mu_ttw_sym}}
\begin{tabular}
{ccccccccc}
	\hline\hline
&	& \multicolumn{8}{c}{\scalht (\gev)} \\ 
	 (\njet,  \nb) & 200-250 & 250-300 & 300-350 & 350-400 & 400-500 & 500-600 & 600-800 & 800-$\infty$ \\ [0.8ex] 
\hline
	(2, 0) & $0.9^{+ 0.03 }_{- 0.03 }$ & $0.6^{+ 0.02 }_{- 0.02 }$ & $0.4^{+ 0.02 }_{- 0.02 }$ & $0.3^{+ 0.01 }_{- 0.01 }$ & $0.2^{+ 0.01 }_{- 0.01 }$ & $0.1^{+ 0.01 }_{- 0.01 }$ & $0.1^{+ 0.00 }_{- 0.00 }$ & $0.1^{+ 0.01 }_{- 0.01 }$ \\[0.5ex] 
	(2, 1) & $0.7^{+ 0.04 }_{- 0.04 }$ & $0.5^{+ 0.03 }_{- 0.03 }$ & $0.4^{+ 0.03 }_{- 0.03 }$ & $0.2^{+ 0.03 }_{- 0.03 }$ & $0.1^{+ 0.01 }_{- 0.01 }$ & $0.1^{+ 0.01 }_{- 0.01 }$ & $0.0^{+ 0.01 }_{- 0.01 }$ & $0.1^{+ 0.02 }_{- 0.02 }$ \\[0.5ex] 
	(2, 2) & $0.5^{+ 0.12 }_{- 0.12 }$ & $0.4^{+ 0.09 }_{- 0.09 }$ & $0.4^{+ 0.10 }_{- 0.10 }$ & -- & $0.1^{+ 0.03 }_{- 0.03 }$ & $0.2^{+ 0.06 }_{- 0.06 }$ & $0.0^{+ 0.01 }_{- 0.01 }$ & $0.0^{+ 0.02 }_{- 0.02 }$ \\[0.5ex] 
	(3, 0) & $0.4^{+ 0.22 }_{- 0.22 }$ & $0.4^{+ 0.03 }_{- 0.03 }$ & $0.5^{+ 0.02 }_{- 0.02 }$ & $0.5^{+ 0.02 }_{- 0.02 }$ & $0.3^{+ 0.01 }_{- 0.01 }$ & $0.2^{+ 0.01 }_{- 0.01 }$ & $0.1^{+ 0.00 }_{- 0.00 }$ & $0.1^{+ 0.00 }_{- 0.00 }$ \\[0.5ex] 
	(3, 1) & $0.2^{+ 0.12 }_{- 0.12 }$ & $0.3^{+ 0.02 }_{- 0.02 }$ & $0.3^{+ 0.01 }_{- 0.01 }$ & $0.3^{+ 0.01 }_{- 0.01 }$ & $0.2^{+ 0.01 }_{- 0.01 }$ & $0.1^{+ 0.01 }_{- 0.01 }$ & $0.1^{+ 0.00 }_{- 0.00 }$ & $0.1^{+ 0.01 }_{- 0.01 }$ \\[0.5ex] 
	(3, 2) & -- & $0.2^{+ 0.03 }_{- 0.03 }$ & $0.2^{+ 0.02 }_{- 0.02 }$ & $0.2^{+ 0.02 }_{- 0.02 }$ & $0.1^{+ 0.01 }_{- 0.01 }$ & $0.1^{+ 0.01 }_{- 0.01 }$ & $0.0^{+ 0.01 }_{- 0.01 }$ & $0.0^{+ 0.01 }_{- 0.01 }$ \\[0.5ex] 
	(3, $\ge3$) & -- & -- & -- & $0.4^{+ 0.17 }_{- 0.17 }$ & $0.1^{+ 0.03 }_{- 0.03 }$ & $0.1^{+ 0.04 }_{- 0.04 }$ & -- & -- \\[0.5ex] 
	(4, 0) & -- & -- & $0.7^{+ 0.07 }_{- 0.07 }$ & $0.6^{+ 0.03 }_{- 0.03 }$ & $0.5^{+ 0.01 }_{- 0.01 }$ & $0.3^{+ 0.01 }_{- 0.01 }$ & $0.1^{+ 0.00 }_{- 0.00 }$ & $0.1^{+ 0.00 }_{- 0.00 }$ \\[0.5ex] 
	(4, 1) & -- & -- & $0.4^{+ 0.03 }_{- 0.03 }$ & $0.4^{+ 0.02 }_{- 0.02 }$ & $0.3^{+ 0.01 }_{- 0.01 }$ & $0.1^{+ 0.01 }_{- 0.01 }$ & $0.1^{+ 0.01 }_{- 0.01 }$ & $0.1^{+ 0.01 }_{- 0.01 }$ \\[0.5ex] 
	(4, 2) & -- & -- & $0.4^{+ 0.04 }_{- 0.04 }$ & $0.3^{+ 0.02 }_{- 0.02 }$ & $0.2^{+ 0.01 }_{- 0.01 }$ & $0.1^{+ 0.01 }_{- 0.01 }$ & $0.0^{+ 0.01 }_{- 0.01 }$ & $0.0^{+ 0.01 }_{- 0.01 }$ \\[0.5ex] 
	(4, $\ge3$) & -- & -- & -- & $0.3^{+ 0.06 }_{- 0.06 }$ & $0.3^{+ 0.04 }_{- 0.04 }$ & $0.1^{+ 0.03 }_{- 0.03 }$ & $0.0^{+ 0.01 }_{- 0.01 }$ & $0.0^{+ 0.01 }_{- 0.01 }$ \\[0.5ex] 
	($\ge5$, 0) & -- & -- & -- & $0.5^{+ 0.10 }_{- 0.10 }$ & $0.5^{+ 0.03 }_{- 0.03 }$ & $0.3^{+ 0.01 }_{- 0.01 }$ & $0.2^{+ 0.01 }_{- 0.01 }$ & $0.1^{+ 0.00 }_{- 0.00 }$ \\[0.5ex] 
	($\ge5$, 1) & -- & -- & -- & $0.3^{+ 0.04 }_{- 0.04 }$ & $0.4^{+ 0.01 }_{- 0.01 }$ & $0.2^{+ 0.01 }_{- 0.01 }$ & $0.1^{+ 0.00 }_{- 0.00 }$ & $0.1^{+ 0.00 }_{- 0.00 }$ \\[0.5ex] 
	($\ge5$, 2) & -- & -- & -- & $0.3^{+ 0.05 }_{- 0.05 }$ & $0.3^{+ 0.02 }_{- 0.02 }$ & $0.2^{+ 0.01 }_{- 0.01 }$ & $0.1^{+ 0.00 }_{- 0.00 }$ & $0.1^{+ 0.00 }_{- 0.00 }$ \\[0.5ex] 
	($\ge5$, $\ge3$) & -- & -- & -- & -- & $0.2^{+ 0.04 }_{- 0.04 }$ & $0.2^{+ 0.03 }_{- 0.03 }$ & $0.1^{+ 0.01 }_{- 0.01 }$ & $0.0^{+ 0.01 }_{- 0.01 }$ \\[0.5ex] 
	\hline
	\hline
\end{tabular}
\end{table}

%% \begin{table}[h!]
\tiny
\centering
\caption{Transfer factors from the \mj control region to the \ttbar/W background for asymmetric categories.\label{tab:tf_mu_ttw_asym}}
\scalebox{0.85}{\begin{tabular}{ccccccccc}
	\hline\hline
	& \multicolumn{8}{c}{\scalht (\gev)} \\ 
	 (\njet,  \nb) & 200-250 & 250-300 & 300-350 & 350-400 & 400-500 & 500-600 & 600-800 & 800-$\infty$ \\ [0.8ex] 
\hline
	(2a, 0) & $0.52\pm 0.01$ & $0.27\pm 0.01$ & $0.20\pm 0.01$ & $0.16\pm 0.01$ & $0.12\pm 0.01$ & $0.14\pm 0.02$ & $0.09\pm 0.02$ & -- \\[0.5ex] 
	(2a, 1) & $0.39\pm 0.01$ & $0.22\pm 0.02$ & $0.15\pm 0.02$ & $0.10\pm 0.02$ & $0.12\pm 0.03$ & $0.11\pm 0.05$ & -- & -- \\[0.5ex] 
	(2a, 2) & $0.28\pm 0.04$ & $0.15\pm 0.04$ & $0.16\pm 0.08$ & $0.13\pm 0.09$ & $0.08\pm 0.06$ & -- & -- & -- \\[0.5ex] 
	(3a, 0) & $0.68\pm 0.02$ & $0.50\pm 0.01$ & $0.42\pm 0.02$ & $0.24\pm 0.02$ & $0.13\pm 0.01$ & $0.08\pm 0.01$ & $0.05\pm 0.01$ & -- \\[0.5ex] 
	(3a, 1) & $0.47\pm 0.02$ & $0.36\pm 0.01$ & $0.31\pm 0.02$ & $0.21\pm 0.02$ & $0.09\pm 0.02$ & $0.07\pm 0.02$ & $0.03\pm 0.01$ & -- \\[0.5ex] 
	(3a, 2) & $0.31\pm 0.02$ & $0.24\pm 0.02$ & $0.27\pm 0.02$ & $0.20\pm 0.03$ & $0.04\pm 0.02$ & $0.07\pm 0.06$ & -- & -- \\[0.5ex] 
	(3a, $\ge3$) & $0.19\pm 0.06$ & $0.25\pm 0.06$ & $0.17\pm 0.07$ & -- & -- & -- & -- & -- \\[0.5ex] 
	(4a, 0) & $0.13\pm 0.06$ & $0.46\pm 0.03$ & $0.58\pm 0.03$ & $0.59\pm 0.03$ & $0.35\pm 0.02$ & $0.14\pm 0.02$ & $0.05\pm 0.01$ & -- \\[0.5ex] 
	(4a, 1) & $0.09\pm 0.03$ & $0.23\pm 0.01$ & $0.42\pm 0.02$ & $0.40\pm 0.02$ & $0.29\pm 0.02$ & $0.08\pm 0.02$ & $0.01\pm 0.01$ & -- \\[0.5ex] 
	(4a, 2) & $0.09\pm 0.05$ & $0.16\pm 0.02$ & $0.35\pm 0.02$ & $0.34\pm 0.02$ & $0.24\pm 0.02$ & $0.06\pm 0.02$ & $0.01\pm 0.01$ & -- \\[0.5ex] 
	(4a, $\ge3$) & -- & $0.27\pm 0.09$ & $0.31\pm 0.06$ & $0.27\pm 0.07$ & $0.24\pm 0.08$ & -- & -- & -- \\[0.5ex] 
	($\ge5$a, 0) & -- & $1.81\pm 1.14$ & $0.79\pm 0.11$ & $0.69\pm 0.06$ & $0.64\pm 0.04$ & $0.32\pm 0.04$ & $0.18\pm 0.04$ & -- \\[0.5ex] 
	($\ge5$a, 1) & -- & $0.32\pm 0.13$ & $0.56\pm 0.05$ & $0.51\pm 0.03$ & $0.43\pm 0.02$ & $0.25\pm 0.03$ & $0.08\pm 0.02$ & -- \\[0.5ex] 
	($\ge5$a, 2) & -- & $0.03\pm 0.03$ & $0.49\pm 0.06$ & $0.52\pm 0.04$ & $0.42\pm 0.02$ & $0.18\pm 0.02$ & $0.10\pm 0.03$ & -- \\[0.5ex] 
	($\ge5$a, $\ge3$) & -- & -- & $0.27\pm 0.09$ & $0.39\pm 0.08$ & $0.45\pm 0.07$ & $0.19\pm 0.06$ & -- & -- \\[0.5ex] 
	\hline
	\hline
\end{tabular}}
\end{table}

%% \begin{table}[h!]
\tiny
\centering
\caption{Transfer factors from the \mj control region to the \ttbar/W background for monojet categories.\label{tab:tf_mu_ttw_mono}}
\scalebox{0.85}{\begin{tabular}{ccccccccc}
	\hline\hline
	& \multicolumn{8}{c}{\scalht (\gev)} \\ 
	 (\njet,  \nb) & 200-250 & 250-300 & 300-350 & 350-400 & 400-500 & 500-600 & 600-800 & 800-$\infty$ \\ [0.8ex] 
\hline
	(1, 0) & $1.09\pm 0.01$ & $0.88\pm 0.01$ & $0.74\pm 0.02$ & $0.62\pm 0.02$ & $0.48\pm 0.02$ & $0.40\pm 0.04$ & $0.44\pm 0.09$ & -- \\[0.5ex] 
	(1, 1) & $0.75\pm 0.03$ & $0.77\pm 0.05$ & $0.63\pm 0.06$ & $0.43\pm 0.07$ & $0.66\pm 0.10$ & $0.28\pm 0.10$ & -- & -- \\[0.5ex] 
	\hline
	\hline
\end{tabular}}
\end{table}

%% \begin{table}[h!]
\tiny
\centering
\caption{Transfer factors from the \mmj control region to the \zInv~ background for symmetric categories. The letter ``a'' in jet \eg ``2a''  indicates the asymmetric jet bins. All entries are non-zero but are truncated to one decimal place.\label{tab:tf_mumu_zinv_sym}}
\begin{tabular}
{ccccccccc}
	\hline\hline
&	& \multicolumn{8}{c}{\scalht (\gev)} \\ 
	 (\njet,  \nb) & 200-250 & 250-300 & 300-350 & 350-400 & 400-500 & 500-600 & 600-800 & 800-$\infty$ \\ [0.8ex] 
\hline
	(2, 0) & $9.90^{+ 0.68 }_{- 0.68 }$ & $7.80^{+ 0.44 }_{- 0.44 }$ & $5.05^{+ 0.27 }_{- 0.27 }$ & $4.15^{+ 0.21 }_{- 0.21 }$ & $3.06^{+ 0.09 }_{- 0.09 }$ & $2.16^{+ 0.07 }_{- 0.07 }$ & $1.31^{+ 0.04 }_{- 0.04 }$ & $2.71^{+ 0.09 }_{- 0.09 }$ \\[0.5ex] 
	(2, 1) & $7.16^{+ 1.53 }_{- 1.53 }$ & $4.77^{+ 0.83 }_{- 0.83 }$ & $3.58^{+ 0.60 }_{- 0.60 }$ & $2.98^{+ 0.47 }_{- 0.47 }$ & $2.37^{+ 0.19 }_{- 0.19 }$ & $2.15^{+ 0.24 }_{- 0.24 }$ & $1.53^{+ 0.15 }_{- 0.15 }$ & $2.69^{+ 0.26 }_{- 0.26 }$ \\[0.5ex] 
	(2, 2) & $10.32^{+ 5.34 }_{- 5.34 }$ & $3.33^{+ 1.49 }_{- 1.49 }$ & $7.27^{+ 3.60 }_{- 3.60 }$ & -- & $2.34^{+ 0.87 }_{- 0.87 }$ & $3.46^{+ 1.51 }_{- 1.51 }$ & $1.41^{+ 0.54 }_{- 0.54 }$ & $3.10^{+ 1.89 }_{- 1.89 }$ \\[0.5ex] 
	(3, 0) & $4.98^{+ 4.53 }_{- 4.53 }$ & $6.05^{+ 0.77 }_{- 0.77 }$ & $6.76^{+ 0.50 }_{- 0.50 }$ & $5.94^{+ 0.33 }_{- 0.33 }$ & $4.76^{+ 0.14 }_{- 0.14 }$ & $3.07^{+ 0.09 }_{- 0.09 }$ & $2.00^{+ 0.05 }_{- 0.05 }$ & $2.52^{+ 0.07 }_{- 0.07 }$ \\[0.5ex] 
	(3, 1) & -- & $4.45^{+ 1.07 }_{- 1.07 }$ & $3.74^{+ 0.60 }_{- 0.60 }$ & $4.66^{+ 0.61 }_{- 0.61 }$ & $3.30^{+ 0.27 }_{- 0.27 }$ & $2.22^{+ 0.18 }_{- 0.18 }$ & $1.58^{+ 0.11 }_{- 0.11 }$ & $2.24^{+ 0.17 }_{- 0.17 }$ \\[0.5ex] 
	(3, 2) & -- & $1.17^{+ 0.49 }_{- 0.49 }$ & $2.46^{+ 0.72 }_{- 0.72 }$ & $1.71^{+ 0.43 }_{- 0.43 }$ & $1.71^{+ 0.32 }_{- 0.32 }$ & $1.42^{+ 0.33 }_{- 0.33 }$ & $0.94^{+ 0.20 }_{- 0.20 }$ & $1.69^{+ 0.49 }_{- 0.49 }$ \\[0.5ex] 
	(3, $\ge3$) & -- & -- & -- & $0.76^{+ 0.74 }_{- 0.74 }$ & $0.23^{+ 0.22 }_{- 0.22 }$ & $4.24^{+ 5.21 }_{- 5.21 }$ & -- & -- \\[0.5ex] 
	(4, 0) & -- & -- & $5.24^{+ 1.03 }_{- 1.03 }$ & $7.61^{+ 0.91 }_{- 0.91 }$ & $6.13^{+ 0.26 }_{- 0.26 }$ & $4.15^{+ 0.16 }_{- 0.16 }$ & $2.69^{+ 0.08 }_{- 0.08 }$ & $2.52^{+ 0.09 }_{- 0.09 }$ \\[0.5ex] 
	(4, 1) & -- & -- & $9.05^{+ 4.45 }_{- 4.45 }$ & $6.41^{+ 1.09 }_{- 1.09 }$ & $3.46^{+ 0.35 }_{- 0.35 }$ & $2.75^{+ 0.25 }_{- 0.25 }$ & $1.96^{+ 0.15 }_{- 0.15 }$ & $2.19^{+ 0.16 }_{- 0.16 }$ \\[0.5ex] 
	(4, 2) & -- & -- & $5.02^{+ 2.97 }_{- 2.97 }$ & $1.49^{+ 0.57 }_{- 0.57 }$ & $2.66^{+ 0.50 }_{- 0.50 }$ & $2.11^{+ 0.48 }_{- 0.48 }$ & $1.12^{+ 0.25 }_{- 0.25 }$ & $1.55^{+ 0.28 }_{- 0.28 }$ \\[0.5ex] 
	(4, $\ge3$) & -- & -- & -- & $3.59^{+ 3.91 }_{- 3.91 }$ & $2.82^{+ 1.99 }_{- 1.99 }$ & $2.49^{+ 1.86 }_{- 1.86 }$ & $1.33^{+ 0.94 }_{- 0.94 }$ & $0.36^{+ 0.23 }_{- 0.23 }$ \\[0.5ex] 
	($\ge5$, 0) & -- & -- & -- & $17.39^{+ 6.00 }_{- 6.00 }$ & $6.71^{+ 0.84 }_{- 0.84 }$ & $5.37^{+ 0.35 }_{- 0.35 }$ & $3.23^{+ 0.14 }_{- 0.14 }$ & $2.54^{+ 0.08 }_{- 0.08 }$ \\[0.5ex] 
	($\ge5$, 1) & -- & -- & -- & $4.12^{+ 2.82 }_{- 2.82 }$ & $3.58^{+ 0.70 }_{- 0.70 }$ & $2.65^{+ 0.49 }_{- 0.49 }$ & $2.34^{+ 0.22 }_{- 0.22 }$ & $1.68^{+ 0.11 }_{- 0.11 }$ \\[0.5ex] 
	($\ge5$, 2) & -- & -- & -- & $1.73^{+ 2.20 }_{- 2.20 }$ & $1.53^{+ 0.45 }_{- 0.45 }$ & $1.43^{+ 0.34 }_{- 0.34 }$ & $1.28^{+ 0.26 }_{- 0.26 }$ & $1.02^{+ 0.14 }_{- 0.14 }$ \\[0.5ex] 
	($\ge5$, $\ge3$) & -- & -- & -- & -- & $0.43^{+ 0.37 }_{- 0.37 }$ & $0.88^{+ 0.67 }_{- 0.67 }$ & $2.78^{+ 1.34 }_{- 1.34 }$ & $1.04^{+ 0.45 }_{- 0.45 }$ \\[0.5ex] 
	\hline
	\hline
\end{tabular}
\end{table}

%% \begin{table}[h!]
\tiny
\centering
\caption{Transfer factors from the \mmj control region to the \zInv~ background for asymmetric categories.\label{tab:tf_mumu_zinv_asym}}
\scalebox{0.85}{\begin{tabular}{ccccccccc}
	\hline\hline
	& \multicolumn{8}{c}{\scalht (\gev)} \\ 
	 (\njet,  \nb) & 200-250 & 250-300 & 300-350 & 350-400 & 400-500 & 500-600 & 600-800 & 800-$\infty$ \\ [0.8ex] 
\hline
	(2a, 0) & $5.36^{+ 0.15 }_{- 0.15 }$ & $3.08^{+ 0.12 }_{- 0.12 }$ & $2.45^{+ 0.13 }_{- 0.13 }$ & $2.05^{+ 0.13 }_{- 0.13 }$ & $2.46^{+ 0.13 }_{- 0.13 }$ & $1.78^{+ 0.14 }_{- 0.14 }$ & $1.76^{+ 0.16 }_{- 0.16 }$ & -- \\[0.5ex] 
	(2a, 1) & $3.46^{+ 0.29 }_{- 0.29 }$ & $2.25^{+ 0.26 }_{- 0.26 }$ & $1.72^{+ 0.27 }_{- 0.27 }$ & $2.26^{+ 0.48 }_{- 0.48 }$ & $1.89^{+ 0.30 }_{- 0.30 }$ & $1.73^{+ 0.53 }_{- 0.53 }$ & -- & -- \\[0.5ex] 
	(2a, 2) & $2.78^{+ 0.57 }_{- 0.57 }$ & $1.30^{+ 0.39 }_{- 0.39 }$ & $6.57^{+ 3.34 }_{- 3.34 }$ & $5.82^{+ 6.16 }_{- 6.16 }$ & $1.03^{+ 0.74 }_{- 0.74 }$ & -- & -- & -- \\[0.5ex] 
	(3a, 0) & $7.12^{+ 0.47 }_{- 0.47 }$ & $4.96^{+ 0.26 }_{- 0.26 }$ & $5.25^{+ 0.38 }_{- 0.38 }$ & $3.30^{+ 0.28 }_{- 0.28 }$ & $2.07^{+ 0.13 }_{- 0.13 }$ & $1.42^{+ 0.16 }_{- 0.16 }$ & $1.27^{+ 0.14 }_{- 0.14 }$ & -- \\[0.5ex] 
	(3a, 1) & $4.65^{+ 0.69 }_{- 0.69 }$ & $3.11^{+ 0.37 }_{- 0.37 }$ & $2.74^{+ 0.40 }_{- 0.40 }$ & $2.08^{+ 0.38 }_{- 0.38 }$ & $1.32^{+ 0.22 }_{- 0.22 }$ & $0.43^{+ 0.16 }_{- 0.16 }$ & $1.01^{+ 0.29 }_{- 0.29 }$ & -- \\[0.5ex] 
	(3a, 2) & $2.26^{+ 0.62 }_{- 0.62 }$ & $1.73^{+ 0.43 }_{- 0.43 }$ & $2.34^{+ 0.61 }_{- 0.61 }$ & $2.98^{+ 1.27 }_{- 1.27 }$ & $1.35^{+ 0.47 }_{- 0.47 }$ & $0.99^{+ 0.63 }_{- 0.63 }$ & -- & -- \\[0.5ex] 
	(3a, $\ge3$) & $3.85^{+ 5.06 }_{- 5.06 }$ & $0.85^{+ 0.84 }_{- 0.84 }$ & -- & -- & -- & -- & -- & -- \\[0.5ex] 
	(4a, 0) & $1.39^{+ 0.96 }_{- 0.96 }$ & $4.13^{+ 0.65 }_{- 0.65 }$ & $6.66^{+ 0.77 }_{- 0.77 }$ & $6.56^{+ 0.78 }_{- 0.78 }$ & $5.00^{+ 0.46 }_{- 0.46 }$ & $2.29^{+ 0.38 }_{- 0.38 }$ & $1.19^{+ 0.22 }_{- 0.22 }$ & -- \\[0.5ex] 
	(4a, 1) & $1.10^{+ 1.24 }_{- 1.24 }$ & $3.82^{+ 1.30 }_{- 1.30 }$ & $3.23^{+ 0.68 }_{- 0.68 }$ & $3.84^{+ 0.95 }_{- 0.95 }$ & $3.63^{+ 0.58 }_{- 0.58 }$ & $1.86^{+ 0.61 }_{- 0.61 }$ & $0.67^{+ 0.31 }_{- 0.31 }$ & -- \\[0.5ex] 
	(4a, 2) & -- & $7.92^{+ 4.39 }_{- 4.39 }$ & $3.58^{+ 1.20 }_{- 1.20 }$ & $4.29^{+ 1.67 }_{- 1.67 }$ & $3.07^{+ 1.14 }_{- 1.14 }$ & $0.32^{+ 0.24 }_{- 0.24 }$ & $0.00^{+ 0.01 }_{- 0.01 }$ & -- \\[0.5ex] 
	(4a, $\ge3$) & -- & $10318.30^{+ 14528.12 }_{- 14528.12 }$ & $3.56^{+ 3.40 }_{- 3.40 }$ & -- & -- & -- & -- & -- \\[0.5ex] 
	($\ge5$a, 0) & -- & -- & $7.38^{+ 3.65 }_{- 3.65 }$ & $8.54^{+ 2.29 }_{- 2.29 }$ & $6.86^{+ 1.21 }_{- 1.21 }$ & $3.67^{+ 0.60 }_{- 0.60 }$ & $2.75^{+ 0.58 }_{- 0.58 }$ & -- \\[0.5ex] 
	($\ge5$a, 1) & -- & -- & $5.59^{+ 3.26 }_{- 3.26 }$ & $2.76^{+ 1.02 }_{- 1.02 }$ & $2.89^{+ 0.68 }_{- 0.68 }$ & $3.67^{+ 1.46 }_{- 1.46 }$ & $1.21^{+ 0.58 }_{- 0.58 }$ & -- \\[0.5ex] 
	($\ge5$a, 2) & -- & -- & $9.13^{+ 10.19 }_{- 10.19 }$ & $7.66^{+ 5.88 }_{- 5.88 }$ & $3.05^{+ 1.26 }_{- 1.26 }$ & $2.85^{+ 1.66 }_{- 1.66 }$ & $0.25^{+ 0.27 }_{- 0.27 }$ & -- \\[0.5ex] 
	($\ge5$a, $\ge3$) & -- & -- & -- & -- & $1.40^{+ 1.34 }_{- 1.34 }$ & $81.43^{+ 112.36 }_{- 112.36 }$ & -- & -- \\[0.5ex] 
	\hline
	\hline
\end{tabular}}
\end{table}

%% \begin{table}[h!]
\tiny
\centering
\caption{Transfer factors from the \mmj control region to the \zInv~ background for monojet categories.\label{tab:tf_mumu_zinv_mono}}
\begin{tabular}
{ccccccccc}
	\hline\hline
	& \multicolumn{8}{c}{\scalht (\gev)} \\ 
	 (\njet,  \nb) & 200-250 & 250-300 & 300-350 & 350-400 & 400-500 & 500-600 & 600-800 & 800-$\infty$ \\ [0.8ex] 
\hline
	(1, 0) & $11.96^{+ 0.31 }_{- 0.31 }$ & $10.23^{+ 0.40 }_{- 0.40 }$ & $9.84^{+ 0.55 }_{- 0.55 }$ & $9.47^{+ 0.66 }_{- 0.66 }$ & $8.75^{+ 0.45 }_{- 0.45 }$ & $8.71^{+ 0.52 }_{- 0.52 }$ & $8.19^{+ 0.52 }_{- 0.52 }$ & -- \\[0.5ex] 
	(1, 1) & $9.75^{+ 1.12 }_{- 1.12 }$ & $8.00^{+ 1.31 }_{- 1.31 }$ & $6.45^{+ 1.39 }_{- 1.39 }$ & $6.84^{+ 1.86 }_{- 1.86 }$ & $9.69^{+ 2.68 }_{- 2.68 }$ & $9.37^{+ 2.61 }_{- 2.61 }$ & -- & -- \\[0.5ex] 
	\hline
	\hline
\end{tabular}
\end{table}


\subsection{\texorpdfstring{\mht}{MHT} templates}

The transfer factors described above provide an estimate of the lost
lepton background as a function of the (\njet, \scalht, \nb) category,
but inclusive with respect to \mht. However, the analysis utilises the
\mht variable to discriminate effectively between SM background and
\eg a potential contribution from SUSY processes. The \HTmiss
information is incorporated into the likelihood model via an \mht
template per (\njet, \scalht, \nb) category, $N_{\rm
  MC}^{\ttw}(\HTmiss; \njet, \scalht, \nb)$, which is equivalent to
dicing the numerator of the TF according to \mht, \ie
\begin{equation}
  \label{equ:pred-method}
  \npre^{\ttw}(\njet, \scalht, \nb, \HTmiss) = 
  \frac{N_{\rm MC}^{\ttw}(\HTmiss; \njet, \scalht, \nb)}{N_{\rm MC}^{\mj}(\njet, \scalht, \nb)} 
  \times 
  \nobs^{\mj}(\njet, \scalht, \nb)
\end{equation}
In this regard, the TFs described in Sec.~\ref{sec:ewk-method} provide
an estimate of the normalisation for each \mht template. The \HTmiss
templates are taken from simulation and the validity of the simulation
modelling is tested extensively in the control regions, as described
in Sec.~\ref{}.

\subsection{Correcting the normalisation of the \texorpdfstring{\wj}{W+jets} and \texorpdfstring{\ttbar}{TTbar} simulated samples} 
\label{sec:sideband-corrections}

In this section, a procedure is described to derive process-dependent
``sideband corrections'' by means of a binned likelihood fit using
data sidebands to the \mj and \mmj control regions. Various different
data sidebands have been considered. However, the one used in this
procedure is the \HTmiss sideband, defined by $100 < \HTmiss <
200\GeV$.

Corrections to the inclusive cross sections of the dominant processes
in the signal and control regions, namely \wj, \ttbar, and \zmumu\ +
jets (and correspondingly \znunu\ + jets) are determined. These
corrections are derived after all relevant corrections are applied to
the simulated samples. Hence, these corrections are expected to be
near unity for each process given that, for example, the simulated
samples are normalised to the integrated luminosity of the data sets
using the most accurate cross section calculations available,
typically NNLO or better. However, there are various potential sources
of bias that may lead to a non-negligible correction. For example,
non-unity corrections may be due to the limited applicability of
inclusive cross section calculations in the high-\scalht, high-\ETmiss
corner of phase space used in this analysis. 

It is important to note that this iteration of the analysis is {\em
  not} sensitive to whether these corrections are applied or not, due
to the way the transfer factors are constructed. Only the \zmmj
sample is used to predict the \znunu\ background.\footnote{In the
  previous iteration of the analysis, all three control regions (\mj,
  \mmj, and \gj) were used to predict the \znunu\ background.}  The
\mj sample, rich in \wj and \ttbar events, is used to predict the lost
lepton background. Both control regions are constructed and binned to
cover a kinematically similar phase space. Hence, the extrapolations
are minimal and so the corrections have little impact on the transfer
factors.

However, the corrections are still determined and propagated to all
steps of the analysis, as they are relevant for some of the
data-driven derivation of systematic uncertainties, such as the
closure tests (Sec.~\ref{sec:closure-tests}) and cross checks of the
\HTmiss modelling (Sec.~\ref{sec:}). Without these corrections, the
derived systematic uncertainties could be (unnecessarily)
inflated. Further, the uncertainty in these corrections are not
propagated, as any inaccuracy is -- by construction -- covered by the
aforementioned data-driven procedures to determine systematic
uncertainties.

The \HTmiss sidebands to the control regions are defined with an
identical acceptance to the control regions (except for the \HTmiss
requirement itself) and the sidebands are binned identically in terms
of \njet, \scalht, and \nb.  

A floating parameter per dominant process, namely \wj, \ttbar, and
\zllj, encodes the correction for that process (fully correlated
across all bins).  The \wj and \ttbar processes are mainly constrained
by the \mj sideband while the \zllj process is constrained by the \mmj
sideband. The values of the corrections and uncertainties
given by the fit are shown in Table~\ref{tab:sbCorrsFromFit}.\\
The correction derived for \zllj is also applied to the \znunu + jets
simulated event sample.

\begin{table}[!h]
  \scriptsize
  \centering
  \topcaption{Corrections to the inclusive cross sections 
    for SM processes determined from a binned likelihood fit to data
    in the $100 < \HTmiss < 200\GeV$ sideband to the control regions.}  
  \label{tab:sbCorrsFromFit}
  \begin{tabular}
    {clc}
    \hline
    \textbf{Process} & \textbf{Sample} & \textbf{Corrrection} \\
    \hline
    \wj              & \mj             & $1.21 \pm 0.01$      \\
    \ttbar + jets    & \mj, \mmj       & $0.93 \pm 0.01$      \\
    \zj              & \mmj            & $1.07 \pm 0.01$      \\
    \hline
  \end{tabular}
\end{table}

\subsection{Overview of systematic uncertainties}
\label{sec:systematics}

The following sections address the estimation of systematic
uncertainties related to the lost lepton background estimation. All
relevant uncertainties are listed in Table~\ref{tab:systs}, along with
their representative magnitudes and assumptions on inter-bin
correlations. How the uncertainties are incorporated into the
likelihood model is described in Sec.~\ref{sec:likelihood}. 

Several sources of uncertainty are evaluated.  The most relevant
effects are discussed below, and generally fall into one of two
categories. The first category concerns known theoretical and
experimental effects that are propagated through to the transfer
factors and \HTmiss templates. These are decribed in
Sec.~\ref{sec:mc-variations}. The second set can be considered as
``known unknowns'' that are derived from dedicated ``closure test''
and \HTmiss-modelling studies that involve confronting transfer
factors determined in the phase space of this search against data in
the control regions. This latter category of uncertainties are
summarised in Secs.~\ref{sec:closure-tests} and
\ref{sec:syst-on-shape}.

%These uncertainties will be often referred to as
%\textit{``normalisation uncertainties''}, as opposed to the
%\textit{``\HTmiss template uncertainties''} described in
%Sec.~\ref{sec:syst-on-shape}. The former affect the total number of
%events in each (\njet,~\nb,~\scalht) bin (integrating over \mht),
%while the latter encode the limited knowledge on how these events
%distribute in the \mht dimension. The two sets of systematic
%uncertainties have a separate treatment.

\subsection{Known theoretical and experimental uncertainties}
\label{sec:mc-variations}

A set of corrections are applied to the simulated samples in order to
account for known theoretical and experimental biases, such as jet
energy response, b-tagging efficiency, \etc These corrections are
described in Sec.~\ref{sec:sim-corrs}. These corrections are provided
with uncertainties to cover assumptions in the procedures used in
their derivation. Further sources of uncertainties are also
considered, such as the use of LO generators. The effect of various
uncertainties on the transfer factors and \HTmiss templates is
discussed below.

These sources of uncertainty are each assumed to originate from a
unique underlying source and so the effect of each source is varied
assuming a fully correlated behaviour across the full phase space of
the signal and control regions.

\subsubsection{PU reweighting}
\label{sec:tfSyst_pu}

Events in simulation are reweighted in order to match the distribution
of the primary vertex multiplicity observed in data
(Sec.~\ref{sec:pileup-reweighting}).  A systematic uncertainty is
derived by propagating the 5\% uncertainty on the minimum bias cross
section used in the reweighting procedure.  The relative change in the
transfer factors under this variation is small, at the few percent
level, as shown in Fig.~\ref{fig:tfSyst_pu_muToTtw}
(App~\ref{app:ttw}).

\subsubsection{Effect of scale and PDF on lepton acceptance}
\label{sec:tfSyst_pdf}

Theoretical uncertainties from choices in the renormalisation and
factorisation scales, as well as the parton distribution function, can
introduce systematic uncertaintes on lepton acceptance. The scales
$\mu_R$ and $\mu_F$ are varied by 0.5 and 2.0. Theoretical
uncertainties from parton distribution function are varied according
to the recommended prescription from PDF4LHC~\cite{PDF4LHC:2015}. The
magnitude of the variations in the transfer factors are small,
typically at the $\sim X\%$
level. Figures~\ref{fig:tfSyst_scale_muToTtw} and
\ref{fig:tfSyst_pdf_muToTtw} (App.~\ref{app:ttw}) show the behaviour
in the transfer factors for variations in scale and PDF, respectively.

\subsubsection{\texorpdfstring{\wj}{W+jets} and \texorpdfstring{\ttbar}{TTbar} composition}
\label{sec:tfSyst_ttW_composition}

The relative composition of \wj with respect to \ttbar in the \mj
control region differs relative to that in the signal region, due to
the application of different kinematic requirements (\eg \mt in the
control region, \alphat and \bdphi in the signal region). Hence,
uncertainties in the inclusive cross sections for \wj and \ttbar may
result in variations in the transfer factors as a function of \njet,
\scalht, and \nb, as well as the \HTmiss templates.

A correction to the inclusive cross sections is applied via the
procedure described in Sec.~\ref{sec:sideband-corrections}, which
adjusts the normalisation of the simulated \wj and \ttbar event
samples to match the data counts in a \HTmiss sideband that
(otherwise) covers an identical region of phase space to the \mj
control region.

Following this procedure, the normalisation for the \wj process is
varied by twice the relative uncertainty in the latest CMS measurement
of the inclusive cross section, while preserving the {\it sum} of the
\wj and \ttbar (and other residual) contributions integrated across
the full phase space of the signal region and, separately, the \mj
control region, in order to see the effect on the ``shape'' of the
transfer factors and the \HTmiss templates. Similarly, the same same
procedure is repeated for variations in the \ttbar normalisation. The
resulting variations are summarised in
Fig.~\ref{fig:tfSyst_ttW_composition_muToTtw} (App.~\ref{app:ttw}).

\subsubsection{Missing higher-order corrections in LO \texorpdfstring{\MADGRAPH}{MadGraph}
  samples}
\label{sec:nlo}

The \wj process are generated at leading order with the \MADGRAPH
code. The effect of missing higher-order QCD and EWK corrections are
studied by considering the effect of LO$\rightarrow$NLO corrections,
determined as a function of W boson \Pt and shown in in
Fig.~\ref{fig:tfSyst_nlo_muToTtw} (App.~\ref{app:ttw}), on the
transfer factors and \HTmiss templates. 

All events in the control and signal regions are categorised
identically according to \njet and \scalht. The NLO corrections are
determined based on matrix element calculations that consider up to 3
(1) radiated partons for QCD (EWK) processes. Hence the calculations
are not necessarily correct for the high-\njet phase space covered by
this analysis. Hence, these NLO corrections are not applied directly,
but instead are propagated as an uncertainty through the
analysis. Fig.~\ref{fig:tfSyst_nlo_muToTtw} (App.~\ref{app:ttw}) show
the effect on the transfer factors as a function of the various
discriminating variables. Uncertainties are typically at the percent
level and as large as $\sim 10\%$.

\subsubsection{\texorpdfstring{\njet}{Njet}-dependent (``\texorpdfstring{$N_\textrm{isr}$}{Nisr}'') corrections for \ttbar}
\label{sec:nisr}

The recommendations from the SUSY group to reweight \MADGRAPH \ttbar
Monte Carlo events according to the number of ISR jets ($N_J^{ISR}$)
identified in the event is described in
Sec.~\ref{sec:nisr}. 

For consistency, we follow a similar treatment to the NLO corrections
described above. All events in the control and signal regions are
categorised identically according to \njet, hence the analysis is not
sensitive to the $N_\textrm{isr}$ corrections. Hence, these
corrections are not applied directly, but instead are propagated as an
uncertainty through the analysis. Fig.~\ref{fig:tfSyst_nisr_muToTtw}
(App.~\ref{app:ttw}) show the effect on the transfer factors as a
function of the various discriminating variables. Uncertainties are in
typically $\ll 5\%$.

\subsubsection{Signal trigger uncertainty}
\label{sec:tfSyst_trigger}

The effect of uncertainties in the signal trigger efficiency for the
signal region are studied (Sec.~\ref{sec:triggers}). A systematic is
taken as the difference in the efficiency measured using muon and
electron reference triggers.  The relative change in the transfer
factors is typically at the few percent level, as presented in
Fig.~\ref{fig:tfSyst_trigger_muToTtw} (App~\ref{app:ttw}).

\subsubsection{Lepton trigger / identification / isolation efficiencies}
\label{sec:leptonSyst}

The muon trigger efficiency is emulated in the simulation and data/MC
scale factors provided by the muon POG are applied to account for
differences in the determination of the muon trigger efficiency from
data and simulation.

Leptons out of $p_{T}$ and $\eta$ acceptance, or within detector
acceptance but not identified properly by lepton identification or
isolation requirements contribute to the so-called ``lost-lepton
background'', which mainly stem from W and \ttbar events.  The
fraction of events with leptons out of acceptance (\flepAccep) is
calculated from generator truth level information for each MC
sample. Differences in efficiencies between data and simulation are
accounted for with data/MC scale factors for trigger, lepton
identification and isolation~\cite{twiki-leptonSF}, and change the
lost lepton background in the signal region and yields in muon control
regions simultaneously. 

The effect of lepton reconstruction on the lost lepton background can
be summarised in the following expressions respectively:
\begin{equation}
    \label{eq:lostLepSR}
    \sum_{sample} [ R_{sample} \times \flepAccep \times N^{GEN}_{sample} \times ( 1 - \epsilon_{Loose} ) + ( 1 - \flepAccep ) \times N^{GEN}_{sample} ]
\end{equation}
\begin{equation}
    \label{eq:lostLepCR}
    \sum_{sample} N^{GEN}_{sample} \times \epsilon_{Tight} \times R_{sample}
\end{equation}
where $R_{sample}$ is the cross section reweighting factor for each
sample, $N^{GEN}_{sample}$ is the total MC events for the category,
$\epsilon_{Tight}$ and $\epsilon_{Loose}$ are the lepton efficiency
for Tight and Loose working point. For the numerator, full signal
selection except the lepton veto to mimic signal region phase space as
closely as possible. For the denominator, the full selection for the
\mj control sample is applied.

The systematic variations on the signal region and control region are
computed by varying the lepton scale factor up and down according to
each source of uncertainty.  Data/MC lepton scale factors have
negligible statistical uncertainties. The procedure is repeated
separately for muons and electrons. The relative change in the
transfer factors is summarised in
Fig.~\ref{fig:tfSyst_muon_scale_factor_muToTtw} (App~\ref{app:ttw}).

A log-normal systematic uncertainty of 5\% is assigned to
$\mathrm{tt+W}$ in the signal region, and one with 2\% to the control
region. The two systematics are taken to be anti-correlated with each
other.

Finally, the $\eta$-dependent muon tracking scale factors provided by
the muon POG to cover the effect of HIP inefficiencies have been
applied and their uncertainties taken into consideration. Their effect
on the transfer factors is found to be less than 1\%.

\subsubsection{Jet energy scale}
\label{sec:tfSyst_jec}

The effect of varying the jet energy scale in the \mj and \mmj control
regions is investigated.  The energies of jets used in the analysis
are corrected as a function of their \pt and $\eta$ via the procedure
recommended by the JetMET POG. These corrections have an associated
uncertainty, which is propagated through the analysis.  As the \scalht
and jet multiplicity binning is mirrored in signal and control
regions, the effect of jet energy scale on the transfer factor is
expected to be small.  However, the jet energy scale can still have an
effect due to jets moving in and out acceptance (above and below
$40\gev$). The relative change in the transfer factors is presented as
a function of \scalht and jet category in
Fig.~\ref{fig:tfSyst_jec_muToTtw} (App.~\ref{app:ttw}). The changes
are typically in the range of $1-5\%$.

\subsubsection{B-tagging efficiency and mistag probability}
\label{sec:tfSyst_btag}

Scale factors provided by the BTV POG are applied to the MC samples to
correct for differences in the b-tagging efficiencies and
misidentifications between simulation and data.  The method employed
is based on simple event reweighting as described in
Ref.~\cite{btagSFMethods}.  Events are reweighted according to the
probability of obtaining a particular jet configuration in data and
simulation, as determined by the b-tagging efficiencies computed in
the MC samples and the scale factors measured in data.  Since no
extrapolation is performed in the background prediction across
different \nb multiplicities, the analysis is expected to be robust
against variations in the b-tagging efficiency.  To test this effect
the change in the transfer factors is measured by varying the scale
factors within their uncertainties. The scale factors associated with
b and c jets are varied together (since their measurements are
correlated), while those associated with light jets are varied
separately.  The relative change in the transfer factors is presented
as a function of \scalht and jet category in
Figs.~\ref{fig:tfSyst_bsf_muToTtw} and \ref{fig:tfSyst_bsfl_muToTtw}
(App.~\ref{app:ttw}).  They are typically in the range of $1-3\%$.

\subsection{Systematics uncertainties from data-driven tests}
\label{sec:closure-tests}

The second category of uncertainty is determined from sets of closure
tests based on data control samples~\cite{RA1Paper2012}.  Each set of
tests targets a specific (potential) source of bias in the simulation
modelling that may introduce an \njet- or \scalht-dependent source of
systematic bias in the transfer factors~\cite{RA1Paper2012}.

The level of closure with respect to data is inspected:
\begin{itemize}
\item as a function of \njet while integrating over \scalht and \nb; 
\item as a function of \scalht while integrating over \njet and \nb;
\item as a function of \nb, while integrating over \njet and \scalht.
\end{itemize}
Any non-closure is covered by a systematic uncertainty based on
summing in quadrature the level of closure and its statistical
uncertainty. The magnitude of the uncertainties are summarised in
Table~\ref{tab:systs} at the end of this section. In the likelihood
model (Sec.~\ref{sec:likelihood}), one log-normal nuisance parameter
is used per \scalht category (\ie correlated over \njet and \nb) and
per \njet category (\ie correlated over \scalht and \nb) for all
sources of uncertainty listed in this section. 

\subsubsection{Extrapolation in \texorpdfstring{\alphat}{AlphaT} and
  \texorpdfstring{\bdphi}{biased dPhi}}
\label{sec:tfSyst_alphaT}

Unlike the signal region, events in the \mj control region are not
required to satisfy any requirement on \alphat or \bdphi. Hence, the
accuracy of the modelling of the the \alphat and \bdphi variables are
estimated using the \mj sample. From these estimates, systematic
uncertainties in the extrapolation from the control to the signal
region, due to the differing \alphat and \bdphi requirements, is
derived.

Dedicated closure tests confront data yields in a subsample of \mj
events, satisfying the \alphat and/or \bdphi requirements, against
predictions that are determined using another subsample of \mj events
that satisfy an inverted requirement on \alphat and/or \bdphi. The
subsamples are connected via transfer factors that account for the
extrapolation in the variables. Both subsamples comprise predominantly
\wj and \ttbar events. The signal region requirements on \alphat and
\bdphi are summarised in Table~\ref{tab:alphat-thresholds} in
Sec.~\ref{sec:had-signal}.

The level of closure for an extrapolation in \alphat (only) with the
\mj sample is shown in Fig.~\ref{fig:closure_AlphaT_mu}
(App.~\ref{app:ttw}) as function of \scalht and \njet. Similarly,
Fig.~\ref{fig:closure_bDPhi_mu} (App.~\ref{app:ttw}) shows the same
information for the extrapolation in \bdphi (only). Finally, the level
of closure observed when both \alphat and \bdphi requirements are
simultaneously imposed is summarised in
Fig.~\ref{fig:closure_AlphaT_bDPhi_mu} (App.~\ref{app:ttw}).

Any non-closure in the extrapolation of both \alphat and \bdphi is
covered by a systematic uncertainty, which is defined as the
quadrature sum of the observed non-closure and its statistical
uncertainty (indicated by the blue histogram in the figures). 

\subsubsection{The effect of W polarisation on lepton acceptance}
\label{sec:tfSyst_Wpol}

A data-driven test is introduced to check the modelling of the W
polarisation in simulation.  In this study, based on events in the \mj
control region, the number of $\mu^{+}$ events are used to predict the
number of $\mu^{-}$ events, using transfer factors constructed from
simulation. 
%The polarisation of the W boson has an impact on the prediction of the
%\znunu background using the \mj control region, as explained in the
%following.  The production mechanism of $W$ from pp collisions means
%high $p_T$ $W$ bosons are predominantly left handed \cite{WPol}.  For
%high $p_T$ bosons, this implies that $W^+$ decays to the left handed
%neutrino along its direction of motion while the lepton is pointing
%backward. The opposite behaviour is expected for the $W^-$. The lepton
%is therefore more boosted (and the neutrino less boosted) in $W^+$
%decays than $W^-$ decays.  This leads to a larger number of $W^+$
%decays in the single lepton control regions (which relies on the
%lepton $p_T$ for acceptance) than in the signal region (which relies
%on the neutrino $p_T$ for acceptance).
The closure tests are shown in Fig.~\ref{fig:closure_WPol_mu} as a
function of \scalht and \njet. The systematic uncertainties are
typically in the range $\lesssim 5\%$.

\subsubsection{The single isolated track veto}
\label{sec:tfSyst_SITV}

A data-driven test is introduced to check the modelling of the single
isolated track veto, which is applied in the signal and all control
regions of the analysis.

The study is performed in the \mj sample. The single isolated tracks
(SITs) are identified ignoring the reconstructed muon. A subsample of
events containing no SITs is used to predict the event count for a
subsample containing at least one SIT. 

The closure tests are shown in Fig.~\ref{fig:closure_SITV_mu} as a
function of \scalht and \njet. A significant bias is observed, in
which the number of events containing SITs are underestimated by $\sim
40\%$, largely independent of \scalht and \njet. 

A study of kinematic distributions related to SITs and event-level
variables has been performed on a subsample of \mj events containing
at least one SIT. Various distributions can be found in
Figs.~\ref{fig:dataMC_SIT_mu} and \ref{fig:dataMC_SITEvent_mu}
(App.~\ref{app:ttw}). No significant discrepancies are observed in
shapes, only normalisation. The data/MC scale factors (1.42) shown in
each plot are consistent with the bias observed in the aforementioned
closure tests.








%\subsubsection{Top $p_T$ reweighting}
%\label{sec:tfSyst_topPt}
%
%Variations in the reweighting of top $p_{T}$ distribution, as first outlined in 
%Sec.~\ref{sec:SMxs}, are studied. A conservative systematic
%uncertainty on this correction is taken as the size of the correction itself. 
%The relative change in transfer factors is presented in Fig.
%~\ref{fig:tfSyst_topPt_muToZinv}-\ref{fig:tfSyst_topPt_muToTtw}. The
%variation is typically in the range $0-15\%$.

%\subsubsection{QCD contamination}
%\label{sec:tfSyst_qcdCont}
%
%A check has also been performed on the systematic effect on the
%background prediction due to QCD contamination in the control samples,
%which has been found to be at the ~5\% level for the \gj
%control region. Applying an arbitrarily large variation of $\pm
%100\%$ on the number of Monte Carlo QCD events leads to a systematic
%variation on the transfer factors of at most 5\% in the majority of bins.
%This preliminary study suggests that effect from QCD
%contamination in the \gj control region is small compared 
%to the total uncertainty assigned to transfer factors. 
%This systematic source is covered in the data-driven study  
%using the photon control region, described in Sec. ~\ref{sec:tfSyst_ZGratio}.























%\subsection{Data control samples used in the method}
%
%To estimate the contributions from backgrounds with genuine missing
%transverse momentum, three data control regions are used, which are
%binned identically to the signal region: \mj, \mmj and \gj.  Their
%definitions are provided in Sec.~\ref{sec:selection}. The selection
%criteria for these control regions are defined such that any potential
%contamination from new physics processes or QCD multijets is
%negligible.

%% In previous versions of this analysis, the \mmj and \gj control
%% samples are used to predict the \znunu +jets background. We plan to
%% extend this approach by relying on all (and not just a sub-set of)
%% relevant control samples to predict the two dominant components of the
%% total SM background (\wj and \ttbar, or \znunu + jets). Specifically,
%% we are using the \mj sample to predict the \wj and \ttbar backgrounds
%% (across all \nb bins) and up to three samples comprising \zmmj,
%% \gj and \wmj to predict the \znunu + jets background for events
%% containing exactly zero or one b-tagged jets. Any correlations are
%% appropriately handled by the likelihood model (via the
%% \texttt{Combine} tool).

%% The predictions of the \znunu + jets background based on the \zmmj
%% control samples exhibits significantly larger statistical
%% uncertainties at high \njet, \nb, \scalht, or \mht due to lower event
%% counts arising from the lower Z cross section (w.r.t. \gj and
%% \wj). Regardless, these samples are included in the likelihood fit
%% to provide additional confidence in the control of the \znunu + jets
%% background.

%% Concerning the use of $W$-enriched samples to predict the \znunu +jets
%% background, we have studied this approach
%% in detail with the 8\TeV dataset. Based on the outcome of these studies,
%% we have decided to proceed with this approach, as part
%% of the baseline likelihood description. Studies were
%% based on data-driven tests with 8\TeV data
%% (as described in Sec.~\ref{sec:closure-tests}). Closure tests are a critical
%% tool to determine which samples can be used to predict the SM
%% background components. In particular, we have studied the effect of
%% new closure tests designed to test the $W$-enriched to $Z$-enriched
%% extrapolation, specifically \mj to \gj, \mj to \mmj
%% and $\mu^{+}$ to $\mu^{-}$ closure tests. 
%% If further studies of the closure tests with $13\tev$
%% data suggest that using the \mj control region to predict the \znunu
%% background is not feasible, we will revert back to the approach
%% used in Run~I analysis (\ie relying solely on the \zll and \gj
%% samples). Early investigations suggest there are not any major problems.
%% %The closure tests provide important event samples for probing the
%% %accuracy of the simulation modelling implicit in the transfer factors.
%% %Specific examples include ``\mj to predict \mmj'' and ``zeej to
%% %predict \gj'' with events containing exactly zero or one b-tagged
%% %jets. The former test relies on a \wmj-enriched sample to predict
%% %yields in the \zmmj sample (in the presence of some ``\ttbar
%% %contamination'' for events with $\nb = 1$). The latter test is a
%% %consistency check between a dilepton and a \gj sample (as done in
%% %previous iterations of the analysis).
%% Most importantly, we retain full flexibility in our approach and the
%% control samples used to predict the various background components.

%% Currently, the projected sensitivity, as described in the current
%% version of this note, is based on predictions of SM background
%% components made from control regions as follows. For events containing
%% exactly zero or one b-tagged jets, the \mj (enriched in \wej), \gj and
%% \mmj control samples are used to estimate the irreducible \znunu + jets
%% background, while the \mj control sample is used to estimate all
%% remaining SM processes (predominately \wj and \ttbar). For events
%% containing two or more b-tagged jets, the \mj sample is
%% used to predict the total SM background (dominated by \ttbar).
