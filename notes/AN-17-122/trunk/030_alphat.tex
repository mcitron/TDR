%%____________________________________________________________________________||
\section{Key kinematic variables}

This search considers all-jet (or all-hadronic) final states. The
details on the reconstruction algorithms used to identify jets are
provided in Sec.~\ref{sec:}. The experimental acceptance for jets used
in this analysis is defined in Sec.~\ref{sec:}. The event selection
criteria that define the signal and control regions are detailed in
Sec.~\ref{sec:}. The following section introduces the key variables
used in this analysis, without providing details on the recontruction
algorithms nor the final event selection criteria.

\subsection{\texorpdfstring{\njet}{Njet}}

The jets considered in this search are required to satisfy criteria
related to the ``quality'' of the physics object (to reject anomalous
from \eg reconstruction failures or spurious detector noise, see
Sec.~\ref{sec:}), as well as thresholds related to the transverse
momentum (\Pt) and pseudorapidity ($\eta$) of each jet. In this
analysis, higher thresholds for transverse momentum are required for
the highest-\Pt jets. The number of jets that satisfy these selection
criteria is defined as \njet.

\subsection{\texorpdfstring{\nb}{Nb}}

Jets identified as orignating from b-quarks are ``tagged'' with a
dedicated algorithm (see Sec.~\ref{sec:}). The number of ``b-jets''
per event is identified by the variable \nb.

\subsection{\texorpdfstring{\scalht}{HT}}

The mass scale of the physics processes being probed is characterised
by the scalar \pt sum of the jets, defined as $\scalht =
\sum_{i=1}^{\njet} \pt^{\,\mathrm{j}_i}$.

\subsection{\texorpdfstring{\HTmiss}{HTmiss}}

The magnitude of the vector \ptvec sum of these jets, defined by
$\HTmiss = \abs{\sum_{{i}=1}^{\njet} \ptvec^{\,\mathrm{j}_i}}$, is
used to identify events with significant \ptvecmiss.

%\subsection{\texorpdfstring{\ETmiss}{ETmiss}}
%
%The most accurate estimator of \ptvecmiss is given by the projection
%on the plane perpendicular to the beams of the negative vector sum of
%the momenta of all candidate particles in an event~\cite{cms-met}, as
%determined by the PF algorithm (Sec.~\ref{sec:}). Its magnitude is
%referred to as \ETmiss.
%
%The measurement of \met typically relies on independent sources of
%information from each of the calorimeter, tracking, and muon
%subdetectors. Relative to other physics objects, this measurement is
%particularly sensitive to the beam conditions and detector
%performance. This difficulty is compounded by the high-energy,
%high-luminosity hadron collider environment at the LHC and the lack of
%precise theoretical predictions for the kinematic properties and cross
%sections of multijet events. Hence it is very challenging to control
%and model accurately the multijet background in the high-\met phase
%space that is typical of SUSY searches. Given these difficulties, this
%search relies primarily on \HTmiss as an estimator for \ptvecmiss.

\subsection{\texorpdfstring{\alphat}{AlphaT}}
\label{sec:alphatdef}

For an event sample of all-jet final states, the multijet background
dominates over all other SM backgrounds. Several variables are
employed to reduce the multijet contribution to a low level with
respect to other SM backgrounds, the first of which is described
below.

The dimensionless kinematic variable \alphat~\cite{Randall:2008rw,
  CMS:2008vya, CMS-PAS-SUS-09-001}, defined in Eq.~(\ref{eq:alphat})
below, is used to provide discrimination against multijet events that
do not contain significant \ptvecmiss or that contain large \ptvecmiss
only because of \pt mismeasurements, while retaining sensitivity to
new-physics events with significant \ptvecmiss. The \alphat variable
depends solely on the transverse component of jet four-momenta and is
intrinsically robust against the presence of jet energy
mismeasurements in multijet systems. 
%The variable does not rely on the \ETmiss variable.

For events containing only two jets, \alphat is defined as $\alphat =
\ET^{\mathrm{j}_2}/M_\mathrm{T}$, where $\ET= E\sin\theta$, where $E$
is the energy of the jet and $\theta$ is its polar angle with respect
to the beam axis, $\ET^{\mathrm{j}_2}$ is the transverse energy of the
jet with smaller \ET, and $M_\mathrm{T}$ is the transverse mass of the
dijet system, defined as:
\begin{equation}
  \label{eq:mt}
  M_\mathrm{T} = \sqrt{ \Bigl( \sum_{i=1,2} \ET^{\mathrm{j}_i}
    \Bigr)^2 - \Bigl( \sum_{i=1,2} p_x^{\mathrm{j}_i} \Bigr)^2 - \Bigl(
      \sum_{i=1,2} p_y^{\mathrm{j}_i} \Bigr)^2}\, ,
\end{equation}
where $\ET^{\mathrm{j}_i}$, $p_x^{\mathrm{j}_i}$, and
$p_y^{\mathrm{j}_i}$ are, respectively, the transverse energy, and the
$x$ and $y$ components of the transverse momentum of jet
$\mathrm{j}_i$.

For a perfectly measured dijet event with $\ET^{\mathrm{j_1}} =
\ET^{\mathrm{j_2}}$ and back-to-back jets ($\Delta\phi = \pi$), and in
the limit in which the momentum of each jet is large compared with its
mass, the value of \alphat is 0.5. For an imbalance in the \ET of
back-to-back jets, \alphat is reduced to a value $<$0.5, which gives
the variable its intrinsic robustness. Values significantly greater
than 0.5 are observed when the two jets are not back-to-back and
recoil against \ptvecmiss from weakly interacting particles that
escape the detector.

The definition of the \alphat variable can be generalised for events
with more than two jets~\cite{RA1Paper} as follows. For events with
three or more jets, a pseudo-dijet system is formed by combining the
jets in the event into two pseudo-jets. The mass scale is
characterised by the scalar sum of the jet transverse energies,
$\scalst = \sum_{i=1}^{N_\text{jet}} \ET^{\mathrm{j}_i}$, where
$N_\text{jet}$ is the number of jets with \ET above a predefined
threshold.\footnote{{\it Nota bene} the use of \Et and \scalst versus
  \Pt and \scalht. This difference arises due to the implementation in
  the trigger logic, described in Sec.~\ref{sec:}, which relies on \Et
  rather than \Pt. Ultimately, the effects due nonnegligible jet
  masses is small, at the percent level. Only in the definition of
  \alphat are scalar energy sums based on jet \Et; otherwise,
  throughout the analysis, the scalar sum \scalht uses jet \Pt.} The
\scalst for each of the two pseudo-jets is given by the scalar \ET sum
of its contributing jets. The combination chosen is the one that
minimises \dst, defined as the difference between these sums for the
two pseudo-jets.  This clustering criterion assumes a balanced-event
hypothesis, which provides strong separation between SM multijet
events and events with genuine \ptvecmiss. The \alphat definition can
be generalised to:
\begin{equation}
  \label{eq:alphat}
  \alphat = \frac{1}{2} \frac{\scalst -
    \dst}{\sqrt{(\scalst)^2 - (\HTmiss)^2}}.
\end{equation}

When jet energies are mismeasured, or there are neutrinos from
heavy-flavour quark decays, the magnitudes of \HTmiss and \dst are
highly correlated. This correlation is much weaker for
$R$-parity-conserving SUSY events, where each of the two decay chains
produces an undetected LSP.

Multijet events populate the region $\alphat< 0.5$ and the $\alphat$
distribution is characterised by a sharp edge at 0.5, beyond which the
multijet event yield falls by several orders of magnitude. Multijet
events with extremely rare but large stochastic fluctuations in the
calorimetric measurements of jet energies can lead to values of
\alphat slightly above 0.5. The edge at 0.5 sharpens with increasing
\scalht for events containing at least three jets, primarily due to a
corresponding increase in the average jet energy and consequently a
(relative) improvement in the jet energy resolution.

As a result, a \scalht-dependent threshold is imposed on the \alphat
variable (see Sec.~\ref{sec:}). Hence, it is useful to consider the
limiting case $\dst \rightarrow 0$ to understand the relationship
between \alphat and the ratio $\mht/\sum_{i} E_\textrm{T}^{j_i}$:
\begin{equation}
  \label{eq:alphat3}
  \frac{\mht}{\scalst} \, = \, \sqrt{ 1 - \frac{1}{4 \cdot \alphat^2} }.
\end{equation}

For reference, under the assumption of $\dst = 0$, the values of
$\alphat = 0.65$, 0.60, 0.55, 0.53, and 0.52 map onto values of the
ratio $\mht/\scalst = 0.64$, 0.55, 0.42, 0.33, and 0.27, respectively.

\subsection{\texorpdfstring{\bdphi}{biased dPhi}}
\label{sec:bdphi-def}

An additional powerful variable \bdphi is used to suppress multijet
contamination due to both instrumental effects and semi-leptonic
heavy-flavour decays with genuine \met in the final state. The
variable is defined as follows.

The jet-based estimate of the missing transverse energy, ${\mhtvec}$,
is determined from all jets except for one of the reconstructed jets,
the ``test'' jet $\mathrm{j}_k$, \ie $\mathrm{j}_i \,\in\, [1,\njet],
\mathrm{j}_i \ne \mathrm{j}_k$. The difference in the azimuthal angle
between the recomputed $\mhtvec$ and the ``test'' jet $\mathrm{j}_k$
is then determined. This process is repeated for each jet
$\mathrm{j}_i$ in the event. The \bdphi variables is defined as the
minimum value of the differences in azimuthal angles:
\begin{equation}
  \bdphi = \min_{\,\forall\, \mathrm{j}_k\,\in\, [1,\njet]}
  \Delta\phi \Bigl( \ptvec^{\,\mathrm{j}_k}, \,
    -\hspace{-0.5em}\sum_{\substack{\mathrm{j}_i= 1 \\ \mathrm{j}_i \ne \mathrm{j}_k}}^{\njet}
    \ptvec^{\,\mathrm{j}_i} \Bigr).
  \label{eq:bdphi}
\end{equation}

For monojet events, the calculation is performed using a lower \Pt
threshold on all jets, $\Pt > 25\gev$, with the variable identified as
$\bdphimod$.

The ``test'' jet corresponding to \bdphi is identified as the jet that
is most likely to have given rise to the missing transverse energy in
the event. This variable discriminates between final states with
genuine \ptvecmiss, \eg from the leptonic decay of the W boson, and
energetic multijet events that have significant \ptvecmiss through jet
energy mismeasurements or through the production of neutrinos,
collinear with the axis of a jet, from semileptonic heavy-flavour
decays. Multijet events exhibit a population at $\bdphi \approx
0\,\mathrm{radians}$. Events with a genuine source of \ptvecmiss
exhibit a long tail in \bdphi with values as large as
$\pi\,\mathrm{radians}$.

The use of both the \bdphi and \alphat variables provide an extremely
powerful rejection factor against contamination from multijet events
and allow to maintain low jet-\PT, \HT, and \mht thresholds, which in
turn maximises signal acceptance for a large range of DM and SUSY
models with final states characterised by the presence of significant
\met.

%}%____________________________________________________________________________||
