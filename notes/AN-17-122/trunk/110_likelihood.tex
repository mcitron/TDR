%%____________________________________________________________________________||
\section{Likelihood model}
\label{sec:likelihood}

The \alphat~analysis relies on control regions to predict the normalisation of each category of \njet, \nb~and \scalht 
(which are in the following identified with \htcat) and on simulation to predict the shape 
of \mht~within each category. The likelihood model is split into hadronic and control components
linked by floating parameters for the prediction. In one signal \htcat~category, J, 
the hadronic component may be written as a multiple of Poisson likelihoods
for the observation in each \mht bin 
($\mathrm{Pois}(n|\lambda) \equiv e^{-n}\frac{\lambda^{n}}{n!}$):

\begin{multline}
\label{eq:hadronicLikelihood}
\mathcal{L}^{J}_{\mathrm{had}} = \prod_i \mathcal{L}^{J,i}_{\mathrm{had}} = \prod_i \mathrm{Pois}(n^{J,i}_{\mathrm{had}} |\sum_{j \in J}\, b^{j,i}_{\zInv~,\mathrm{had}}\times\phi^{j}(\mu\mu\rightarrow\zInv~)\times a_{\zInv~}^{j}\times\rho^{j,i}_{\zInv~,\mathrm{had}}\, + \\ 
b^{j,i}_{\ttw,\mathrm{had}}\times\phi^{j}(\mu\rightarrow\ttw)\times a_{\ttw}^{j}\times\rho^{j,i}_{\ttw,\mathrm{had}}\, + b^{j,i}_{\text{QCD},\mathrm{had}}\times\omega^{j,i}_{\text{QCD},\mathrm{had}} + \,r\times s^{j,i}_{\mathrm{had}}\times\rho^{j,i}_{s,\mathrm{had}}) 
\end{multline}

where 
the product is over each \mht bin, i, in the \htcat~category;
the sum over j accounts for the fine control region predictions in \scalht; 
$b^{j,i}_{\zInv~/\ttw,\mathrm{had}}$ are the predicted number 
of events from simulation for the electroweak backgrounds (categorised according to the control region binning); 
$b^{j,i}_{\text{QCD},\mathrm{had}}$ are the predicted 
number of events for the QCD multijet component (from the method described in Section~\ref{sec:qcd});
the $a_{\zInv~/\ttw}^{j}$ parameters are unconstrained and connect the prediction of the signal region
to the control regions (see below); $\phi^{j}$ contains the systematic uncertainties on the 
transfer factors from the data-driven tests; $\rho^{j,i}$ contains the systematics from 
variations in simulation, the systematics derived from the control regions on the~\mht shape 
and the uncertainty from the limited number of simulated events; 
$r$ is the unconstrained `signal strength' parameter and $\omega_{\text{QCD},\mathrm{had}}^{j,i}$ 
contains the uncertainties on the QCD multijet component. 

The \ttw component is predicted using the \mj control region. The relevant 
components of the likelihood, which is not categorised in \mht, may be written as:

\begin{equation}
\label{eq:muLikelihood}
\mathcal{L}^{j}_{\mu} = \mathrm{Pois}(n^{j}_{\mu} |\, b^{j}_{\mu}\times a_{\ttw}^{j}\times\rho^{j}_{\mu} +\, b^{j}_{\text{QCD},\,\mu} + \,r \times s^{j}_{\mu}\times\rho^{j}_{s,\mu})
\end{equation}

similarly to Equation~\ref{eq:hadronicLikelihood}, $\rho^{j}_{\mu}$ contains the uncertainty in the control \htcat from variations in simulation. 
The signal contamination in the control region is encoded by $s^{j}_{\mu}$, which is small by design. 
The connection between the control and signal region
is included by the unconstrained $a_{\ttw}^{j}$ parameter. The 
subdominant QCD component, $b^{j}_{\text{QCD},\,\mu}$, is taken from simulation. 

The \zInv~component in the signal region is predicted using the \mmj control regions. 
Within each category of \njet and \scalht the prediction is correlated 
for  $\nb > 0$. The relevant components of the likelihood, which is not categorised in \mht, may be written as:

\begin{equation}
\label{eq:mumuLikelihood}
\mathcal{L}^{j}_{\mu\mu} = \mathrm{Pois}(n^{j}_{\mu\mu} |\, b^{j}_{\mu\mu}\times a_{\zInv~}^{j}\times\rho^{j}_{\mu\mu} +\, b^{j}_{\text{QCD},\,\mu\mu} + \,r \times s^{j}_{\mu\mu}\times\rho^{j}_{s,\mu\mu})
\end{equation}

where parameters are defined as in Equations~\ref{eq:hadronicLikelihood} and~\ref{eq:muLikelihood}. 
The $a_{\zInv~}^{j}$ parameter is correlated for $\nb > 0$ within each 
category of \njet and \scalht.

The modifier and constraint terms of the parameters representing the systematic uncertainties and 
connections between control and signal regions are summarised below:

\begin{itemize}
\item The transfer factor systematics and uncertainties on the QCD multijet contribution 
are taken to be `log normal' uncertainties such that the logarithm of the variable has 
a Gaussian (normal) constraint~\cite{templateMorphing}. These uncertainties are correlated per topology and \scalht bin 
(pair of \scalht bins for uncertainties derived using \mmj)
\item The systematic uncertainties from variations in simulation and those derived on the \mht~shape from the control regions 
are included using `vertical template morphing'~\cite{templateMorphing}. For vertical template morphing, the yield
in each bin is interpolated quadratically between the $\pm 1\sigma$ variations for each source of
uncertainty and extrapolated linearly beyond this range. The constraint term is Gaussian
with mean 0 and width 1. The uncertainties from simulation and those derived from the control regions
are fully correlated and uncorrelated across all categories respectively.
\item The Poisson uncertainty due to the limited number of simulated events is approximated using
two parameters per bin that multiply the total background and signal contributions and 
are Gaussian constrained. The relevant correlations induced from the 'formula method` (described in Sec.~\ref{sec:formula}) 
are included. 
\end{itemize}

The total likelihood (including the constraint terms in $\mathcal{L}_\mathrm{constraint}$) 
can be written as a product over all (signal) \htcat~bins:

\begin{equation}
\label{eq:totalLikelihood}
\mathcal{L} =  \mathcal{L}_\mathrm{constraint} \times \prod_{J\in\htcat} \mathcal{L}^{J}_{\mathrm{had}} \times  \prod_{j\in J} \mathcal{L}^{j}_{\mu\mu} 
\times \mathcal{L}^{j}_{\mu}
\end{equation}


