%%____________________________________________________________________________||

\section{\texorpdfstring{\znunu\ + jets}{Zinv} background estimation}
\label{sec:zinv}

The irreducible background contribution from the \znunu\ + jets
process can be a sizeable, and often the dominating, contribution in
many event categories of the signal region, particularly at high
\HTmiss. Table~\ref{} (App.~\ref{}) shows

\fixme{Diff w.r.t. old analysis. Heavy flavour.}

\subsection{The ``transfer factor'' method}
\label{sec:tf-method-zinv}

The method used to estimate the \znunuj background relies on the use
of a transfer factor (TF) determined from simulated event samples to
transform the observed event yields in the \mmj sample, categorised
according to \njet and \scalht, $\nobs^{\mmj}(\njet, \scalht)$, into
an estimate of the lost lepton background for the corresponding
(\njet, \scalht) category of the signal region, $\npre^{\znunu}(\njet,
\scalht)$. The categorisation of events according to \njet and \scalht
in the \mmj control region is identical to that for the signal region,
as defined in Sec.~\ref{sec:selection}.\footnote{One detail here is
  that predictions are made according to the \scalht categorisation
  used in the control regions (up to 11 bins), but the predictions are
  then aggregated over a range in \scalht to map onto a coarser
  binning scheme is used in the signal region (up to 5 bins). Further
  details can be found in Sec.~\ref{sec:ht-categorisation}.}  Each
transfer factor is simply a ratio of the yields obtained from Monte
Carlo (MC) simulation for the same (\njet, \scalht) category of the
\mmj control and signal regions:
\begin{equation}
  \label{equ:tf-ratio}
  {\rm TF} = \frac{N_{\rm MC}^{\znunu}(\njet, \scalht)}{N_{\rm MC}^{\mmj}(\njet, \scalht)} 
\end{equation}

In this way, predictions for the \znunuj background can be made, based
on the \mmj control region, as follows:
\begin{equation}
  \label{equ:pred-method}
  \npre^{\znunu}(\njet, \scalht) = 
  \frac{N_{\rm MC}^{\znunu}(\njet, \scalht)}{N_{\rm MC}^{\mmj}(\njet, \scalht)} 
  \times 
  \nobs^{\mmj}(\njet, \scalht)   
\end{equation}

The selection criteria for the \m,j control region
(Sec.~\ref{sec:selection}) closely resemble those for the signal
region, differing mainly through the use of a dimuon {\it tag} (that
is ignored in the calculation of jet-based kinematic variables such as
\scalht, \mht, \alphat, \etc) and minimal additional kinematic
requirements (\eg a dimuon invariant mas window) to obtain a sample
enriched in \zmmj events. The same selection criteria are designed to
suppress signal contamination so that unbiased data-driven estimates
for the SM backgrounds in the signal region can be made. Any signal
contamination is accounted for in the likelihood model
(Sec.~\ref{sec:likelihood}).

Figure~\ref{fig:tf_mumuToZinv} (App.~\ref{app:zinv}) shows the
magnitude of the transfer factors, typically $\ll 1$, The transfer
factors account for differences between the \mmj control and signal
due to the different branching fractions for the decays to neutrinos
or muons, as well as differences in acceptance times efficiency for
the reconstructed muons and extrapolations in kinematic
quantities. For example, the dependence on \njet and \scalht is
largely attributable to the \alphat and \bdphi requirements for the
signal region (that are not used in the \mmj control region).

Many systematic effects are expected to cancel largely in the transfer
factor. However, a systematic uncertainty is assigned to each transfer
factor to account for theoretical uncertainties and effects such as
the mismodelling of kinematics (\eg acceptances) and instrumental
effects (\eg reconstruction efficiencies). These uncertainties are
discussed below. 

\subsection{\texorpdfstring{\nb}{Nb} and \texorpdfstring{\mht}{MHT} templates}

The transfer factors described above provide an estimate of the lost
lepton background as a function of the (\njet, \scalht) category, but
inclusive with respect to \nb and \mht. However, the analysis utilises
the \nb and \mht variables to discriminate effectively between SM
background and \eg a potential contribution from SUSY processes. The
\nb and \HTmiss information is incorporated into the likelihood model
via \nb and \mht templates per (\njet, \scalht) category, $N_{\rm
  MC}^{\ttw}(\nb, \HTmiss;\; \njet, \scalht)$, which is equivalent to
dicing the numerator of the TF according to \nb and \mht, \ie
\begin{equation}
  \label{equ:pred-method}
  \npre^{\ttw}(\njet, \scalht, \nb, \HTmiss) = 
  \frac{N_{\rm MC}^{\ttw}(\nb, \HTmiss;\; \njet, \scalht)}{N_{\rm MC}^{\mj}(\njet, \scalht)} 
  \times 
  \nobs^{\mj}(\njet, \scalht)
\end{equation}
In this regard, the TFs described in Sec.~\ref{sec:tf-method-zinv}
provide an estimate of the normalisation for each \nb and \mht
template. The \nb and \HTmiss templates are taken from simulation and
the validity of the simulation modelling is tested extensively in the
control regions, as described in Sec.~\ref{}.

\subsection{Correcting the normalisation of the \texorpdfstring{\zj}{Z+jets} simulated samples} 
\label{sec:sideband-corrections-zinv}

Corrections to the inclusive cross section used for the \zmmj
simulated samples are determined following a procedure that uses a
binned likelihood fit to data in \HTmiss sidebands of the \mj and \mmj
control regions. Correspondingly, the same corrections are applied to
the \znunuj simulated samples. The procedure is defined in
Sec.~\ref{sec:sideband-corrections}.

As noted previously, this iteration of the analysis is {\em not}
sensitive to whether these corrections are applied or not, due to the
way the transfer factors are constructed. Contrary to the previous
iteration of this analysis, only the \zmmj sample is now used to
predict the \znunuj background.\footnote{In the previous iteration of
  the analysis, all three control regions (\mj, \mmj, and \gj) were
  used to predict the \znunu\ background.} The corrections and
uncertainties for the \zmmj and \znunuj processes, as determined by
the fit, are shown in Table~\ref{tab:sbCorrsFromFit-zinv}.

\begin{table}[!h]
  \scriptsize
  \centering
  \topcaption{Corrections to the inclusive cross sections 
    for the \zmmj and \znunuj processes determined from a binned
    likelihood fit to data in the $100 < \HTmiss < 200\GeV$ sideband
    to the control regions.}
  \label{tab:sbCorrsFromFit-zinv}
  \begin{tabular}
    {clc}
    \hline
    \textbf{Process} & \textbf{Sample} & \textbf{Corrrection} \\
    \hline
    \zmmj            & \mmj            & $1.07 \pm 0.01$      \\
    \znunuj          & (\mmj as proxy) & $1.07 \pm 0.01$      \\
    \hline
  \end{tabular}
\end{table}


