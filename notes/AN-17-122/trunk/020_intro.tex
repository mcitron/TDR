%%____________________________________________________________________________||
\section{Introduction}
\label{sec:intro}

%\fixme{NEED POLISHING!}

The nature of dark matter (DM) is one of the outstanding major
problems in particle physics and a massive weakly interactive particle
(WIMP) is highly motivated. A WIMP is not a specific elementary
particle, but rather a broad class of possible particles. The most
highly scrutinized thermal relic DM candidate is the lightest
neutralino particle of supersymmetric (SUSY) theories. The neutralino
is particularly well-motivated since, in addition to solving the DM
problem, SUSY extensions of the SM contain a number of other
attractive features both for particle physics and in early Universe
cosmology. Most prominently known, is that SUSY does not only provide
an excellent DM candidate but also solves the ``hierarchy problem''.

In this note we summarise the \alphat analysis, which searches for
signatures of physics beyond the standard model in events with at
least one jet and missing transverse momentum. The analysis is based
around the kinematic variable \alphat (defined in
Sec.~\ref{sec:alphatdef}), which efficiently selects candidate signal
events with genuine missing transverse momentum while providing robust
rejection against QCD multijet background events. Additional
protection against the multijet background is provided by the \bdphi
variable (Sec.~\ref{sec:bdphi-def}).

With the \alphat variable, CMS has been searching for supersymmetry
(SUSY) in proton-proton collisions data collected during LHC
Run~1. With data at a centre-of-mass energy of 7 TeV collected in 2010
and 2011, the \alphat analysis has excluded a large parameter space of
the constrained minimal supersymmetric extension of the standard model
(CMSSM)~\cite{Khachatryan:2011tk, Chatrchyan:2011zy,
  Chatrchyan:2012wa} and a parameter space of simplified models
\cite{Chatrchyan:2012wa}. With 11.7\fbinv of data at a centre-of-mass
energy of 8 TeV collected and promptly reconstructed in 2012, the
\alphat analysis further excluded a parameter space of simplified
models \cite{Chatrchyan:2013lya}. Additional sets of data which were
collected in 2012 but were reconstructed later during the LHC Long
Shutdown 1 (LS1) are called the ``parked data'', which contain data
for events triggered with lower energy thresholds. The result, based
on 18.5\fbinv, was interpreted using simplified models involving
third-generation squarks and compressed mass
spectra~\cite{Khachatryan:2016pxa}. 

After two years of the LS1, in June 2015, the LHC started its Run 2
and is delivering proton-proton collisions to CMS at a higher
centre-of-mass energy of 13 TeV. The higher energy collision
considerably increases the production cross sections of heavy
particles, providing a strong motivation to continue the search for
supersymmetry. The analysis was also extended to provide sensitivity
to generic dark matter production at the LHC by providing acceptance
to final states with low jet multiplicities, including ``monojet''
final states. A large part of the parameter space of several
simplified models has been excluded with the first 2.3 \ifb at a
center-of-mass energy of 13 TeV~\cite{Khachatryan:2016dvc}. Analysis
of a further 12.9\fbinv of data, collected in 2016, was also performed
in time for ICHEP 2016~\cite{CMS-PAS-SUS-16-016}. This iteration of
the search is based on the full data set collected in 2016,
corresponding to 35.9\fbinv. 

Section \ref{sec:history} summarises the recent history of the
analysis and documentation. Section~\ref{sec:kine} defines important
kinematic variables used in the analysis, including the \alphat
variable. Section~\ref{sec:datasets} lists the data sets and
simulation event samples used in this note. Section~\ref{sec:triggers}
describes the trigger logic used during
Run~2. Section~\ref{sec:objects} defines the physics objects used in
this analysis. Section~\ref{sec:selection} describes how the signal
and control regions are defined in terms of the event selection
criteria. Sections~\ref{sec:qcd}, \ref{sec:ttw}, and \ref{sec:zinv}
describe the methods used to estimate the multijet, ``lost lepton'',
and ``Z invisible'' backgrounds, respectively. Finally, the likelihood
model is summarised in Sec.~\ref{sec:likelihood}, the analysis result
is presented in Sec.~\ref{sec:results}, and the SUSY interpretations
are provided in Sec.~\ref{sec:susy}. We summarise in
Sec.~\ref{sec:summary}.

%%____________________________________________________________________________||
