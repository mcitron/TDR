%%____________________________________________________________________________||
\section{Introduction}
\label{sec:intro}

\subsection{Analysis strategy}
\label{sec:analysis-strategy}

In this note we summarise the \alphat analysis, which searches for
signatures of physics beyond the standard model in events with at least one jet 
and missing transverse momentum (\met). This analysis uses the
kinematic variable \alphat (defined in Section~\ref{sec:alphatdef}) to efficiently select candidate signal
events with genuine missing transverse momentum while providing robust
rejection against QCD multijet background events. With the \alphat
variable, CMS has been searching for supersymmetry (SUSY) in
proton-proton collisions data collected during LHC Run~1. With data at
a centre-of-mass energy of 7 TeV collected in 2010 and 2011, the
\alphat analysis has excluded a large parameter space of the
constrained minimal supersymmetric extension of the standard model
(CMSSM) \cite{Khachatryan:2011tk, Chatrchyan:2011zy,
Chatrchyan:2012wa} and a parameter space of simplified models
\cite{Chatrchyan:2012wa}. With data at a centre-of-mass energy of 8
TeV collected and promptly reconstructed in 2012, the \alphat analysis
further excluded a parameter space of simplified models
\cite{Chatrchyan:2013lya}. Additional sets of data which were
collected in 2012 but were reconstructed later during the LHC Long
Shutdown 1 (LS1) are called the ``parked data'', which contain data
for events triggered with lower energy
thresholds~\cite{Khachatryan:2016pxa}. 
After two years of the LS1, in June 2015, the LHC started its Run 2 and is
delivering proton-proton collisions to CMS at a higher centre-of-mass
energy of 13 TeV. The higher energy collision considerably increases
the production cross sections of heavy particles, providing a strong
motivation to continue the search for supersymmetry.
A large part of the parameter space of several simplified models has been excluded 
with the first 2.3 \ifb at a center-of-mass energy of 13 TeV \cite{Khachatryan:2016dvc}. 
With further data collected by CMS in 2016, gluino and squark models have 
been further excluded using the \alphat variable \cite{CMS-PAS-SUS-16-016}.

% we should probably paraphrase/reword this next section ... 
The nature of dark matter (DM) is one of the outstanding problems in
particle physics and a massive weakly interactive particle (WIMP) is
highly motivated. A WIMP is not a specific elementary particle, but
rather a broad class of possible particles. The most highly
scrutinized thermal relic DM candidate is the lightest neutralino
particle of supersymmetric (SUSY) theories. The neutralino is
particularly well-motivated since, in addition to solving the DM
problem, SUSY extensions of the SM contain a number of other
attractive features both for particle physics and in early Universe
cosmology. Most prominently known, is that SUSY does not only provide
an excellent DM candidate but also solves the ``hierarchy problem''.
However, not only SUSY but also non-supersymmetric models of physics
beyond the standard model predict the production of DM at LHC Run~2.
The \alphat variable efficiently selects events that potentially
contain DM candidates produced in the collisions. In Run~2, we have extended
the \alphat analysis to significantly improve the acceptance to dark
matter production at the LHC. 
% 1407.0017

% FIXME: have to have all the sections before referencing
%% Section \ref{sec:strategy} discusses changes in the search methods
%% made since Run~1. Sec.~\ref{sec:alphatdef} defines the \alphat
%% variable. Section \ref{sec:datasets} lists the data sets used in this
%% note. Section \ref{sec:triggers} describes the preparation of the
%% triggers for Run~2. Section \ref{sec:objects} defines the physics
%% objects used in this note. Section \ref{sec:selection} describes how
%% events are selected. Section \ref{sec:yields} shows the expected event
%% yields in the signal and control regions. While Section \ref{sec:qcd}
%% discusses how multijet background events are controlled, Section
%% \ref{sec:backgroundmet} estimates background from other processes.
%% Then, Sec.~\ref{sec:systematics} discusses systematic uncertainties in
%% the estimates. With likelihood models introduced in Section
%% \ref{sec:likelihood}, we interpret the results in simplified SUSY
%% models in Sec.~\ref{sec:susy}. Finally Sec.~\ref{sec:summary} provides
%% a summarise.


\subsection{Missing inputs and plans for update of the analysis}
\label{sec:inputs-and-updates}
Some inputs, like corrections to simulation and uncertainties are still missing or not up-to-date. 
In the following the status on this is summarised along with estimated time for updates.

\begin{itemize}
\item \textbf{Simulated samples}. The new Monte Carlo (MC) samples are being injected in the central production.
  They will include updated reconstruction and extension in statistics. We plan to use them as soon as a complete set is available.
\item \textbf{Jet Energy Corrections (JEC)}. New JECs have been recently released for data and MC (v8). 
  We plan to include them in the next iteration of the analysis. 
\item \textbf{B-tag efficiency scale factors}. The new b-tag SF have not been released yet. 
  After some checks we have decided not to apply the old (ICHEP16) SF and set them to unity, while still propagating their uncertainties to the fit.
  We will update to the new ones as soon as they become available.
\item \textbf{Lepton scale factors (trigger,ID,isolation)}. The new lepton SF have not been released yet. 
  For the moment we stick to the ICHEP16 SFs and we plan to update to the new ones as soon as they become available. 
  We also plan to derive the SF with our own framework and cross-check against the numbers provided by the Muon POG.
\end{itemize}


\subsection{Analysis note history}
\label{sec:an-history}
The changes/additions to the Analysis Note for each version are summarised in the following, starting from the most recent one.

\begin{itemize}
\item \textbf{v1}: this version is written in support of the Full Status report talk on 2nd of December 2016.
\end{itemize}

%%____________________________________________________________________________||
