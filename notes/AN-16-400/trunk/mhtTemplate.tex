%%____________________________________________________________________________||
\section{Systematic uncertainties in the \mht dimension}
\label{sec:syst-on-shape}

The estimate of the number of events per (\njet,~\nb,~\scalht) bin,
integrated over \mht, is derived from data control samples, with
the associated systematic uncertainty determined as 
described in Sec.~\ref{sec:closure-tests}. This section
describes the method used to assess the systematic uncertainties in
the distribution of events according to \mht. A data driven approach is
utilised under the hypothesis of zero bias in the data and MC agreement.
The level of closure in the control regions is used
to derive alternative templates accounting for systematic uncertainties.
It is important to note that the data-driven systematics on the 
\mht distribution are included in addition to known uncertainties.

When looking at the \mht dimension inclusively in \scalht there are
large theoretical uncertainties that originate from mixing events
at different scales. These uncertainties can be mitigated if the events 
are binned according to a variable, such as \scalht, 
which is strongly correlated with the scale of the event. 
After this categorisation is applied, the uncertainty in 
the distribution of the \mht variable
(as well as any other MET-like variable) is expected to be 
mainly affected by the MC modelling of the particle 
decays and, to a lesser extent, by jet reconstruction effects, 
such as jet energy scale and resolution. 
This approach, which will be often referred to as \textit{scale anchoring}
in the following, is used in this analysis. The distribution in \mht
is measured in data using the control regions and compared to MC
to determine the validity of the zero bias hypothesis after scale anchoring.

In Sec.~\ref{sec:valid13} the 13 \TeV data is used used 
to validate the scale anchoring approach. 
In Sec.~\ref{sec:systMhtDimension} 
the procedure used to extract the systematic uncertainties in the 
\mht dimension is described and results shown with 13 \TeV data. 
% In Appendix~\ref{sec:bias_injection} the results of a bias injection test described. 
% This shows that the fit is sensitive to a bias affecting only the last \mht bin.
%% Finally, in Sec.~\ref{app:closureTests3fb} the expected systematics for 3\ifb in Run 2 are be shown. 


% extent to which the control regions can constrain this bias 
% to the flat hypothesis
%Motivated by the inclusive distribution, a linear bias is assumed.

\subsection{Validation with 13 TeV Data}
\label{sec:valid13}
In previous versions of this analysis three control regions
were used: \mj, \mmj and \gj. To parameterise the data/MC agreement
orthogonal polynomials are used such that odd and even powers 
are decorrelated \cite{cohen2013applied}. 
The $n^{th}$ order orthogonal polynomial which is fitted to the data/MC 
distribution is defined in Eq.~\ref{equ:orthog-polynomial}.

\begin{equation}
  \label{equ:orthog-polynomial}
  f_n(x) = \sum_{k=0}^{k=n}{(p_k)\times(\bar{x}-x)^k}
\end{equation}

where $\bar{x}$ is the weighted mean of the distribution and $p_k = 0$ 
implies the $k^{th}$ order term is negligible.
In Fig.~\ref{fig:linearMotiv} the data/MC 
distribution against \mht for the control region selection 
(detailed in \cite{Khachatryan:2016pxa}) is shown inclusive 
in \scalht and categories. By fitting a first order
orthogonal polynomial it can be seen that there is a large bias. 
This bias in the data/MC agreement is expected as events 
at different scales are mixed.
The first order orthogonal polynomial
is used to measure the level of bias remaining 
when the \mht dimension is binned in \scalht.
This allows the normalisation parameter to be
decorrelated from the parameter which controls
the distribution in \mht.
The normalisation in each \scalht bin and its systematics 
is then defined following the data-driven method used in Run 1, see Sec,~\ref{sec:systMhtDimension} for details.
\begin{figure}[h!]
  \centering
  \subfigure[\mj]{
    \includegraphics[width=0.5\textwidth]{figures/mhtTemplate/inclusive/mht_Inc_Inc_ht_Inc_SingleMu_Graph.pdf}
  }~~
  \subfigure[\gj]{
    \includegraphics[width=0.5\textwidth]{figures/mhtTemplate/inclusive/mht_Inc_Inc_ht_Inc_SinglePhoton_Graph.pdf}
  }\\
  \subfigure[\mmj]{
    \includegraphics[width=0.5\textwidth]{figures/mhtTemplate/inclusive/mht_Inc_Inc_ht_Inc_DoubleMu_Graph.pdf}
  }\\
  \caption{\label{fig:linearMotiv} 
  The data/MC distribution against \mht for an inclusive selection on category and \scalht
  showing the results of a linear fit. A large bias is observed as well as a low pValue for the constant fits. 
 }
\end{figure}

By anchoring the scale using binning in the \scalht dimension the remaining
bias is assumed to be negligible. In Fig.~\ref{fig:linearFitExamples} 
example fits of an orthogonal linear function to the data/MC ratio 
are shown for the three control regions. Comparing to the inclusive distribution 
the linear component can be seen to be compatible with the null hypothesis, 
i.e. no bias. In order to formalise this assertion 
the pull of the linear component from zero is calculated.
This pull distribution is shown for each of the three control regions in
in Fig.~\ref{fig:pulls}. For the \mmj and \gj samples the mean is consistent 
with zero implying no overall bias. The \mj sample shows some evidence for a small
bias of around 0.3 sigma. Section~\ref{sec:systMhtDimension} details how this bias
is covered in the signal region. All three samples have a sigma consistent with one.
Fig.~\ref{fig:frenchFlagPulls} shows the distribution of the pulls 
in category and \scalht bins showing the pulls are randomly spread across
the categories.


% The linear fits to the data/MC ratio additionally show a p-value following 
% a uniform distribution between 0 and 1 as shown in Fig.~\ref{fig:pValues}.
% This confirms that the linear component is reasonably compatible with zero ($p_1 = 0$), %i.e. no significant bias is observed, 

%Finally, an additional validation 
%can be seen from the p-value distribution of the constant fits
% -- should add p value of constant fit but currently not flat due to
% non guassian behaviour.

\begin{figure}[h!]
  \centering
  \subfigure[\gj, 0b, 5j category and \scalht 750-900\GeV bin]{
    \includegraphics[width=0.5\textwidth]{figures/mhtTemplate/examples/Pho/ht750_900_control_eq0b_eq5j_Pho.pdf}
  }~~
  \subfigure[\mj, 0b, 4j category and \scalht 500-600\GeV bin]{
    \includegraphics[width=0.5\textwidth]{figures/mhtTemplate/examples/Mu/ht500_600_control_eq0b_eq4j_Mu.pdf}
  }\\
  \subfigure[\mmj,2b, 2a category and \scalht 300-350\GeV bin]{
    \includegraphics[width=0.5\textwidth]{figures/mhtTemplate/examples/MuMu/ht300_350_control_eq2b_ge2a_MuMu.pdf}
  }~~
  \subfigure[\gj, 1b, 5j category and \scalht 900-1050\GeV bin]{
    \includegraphics[width=0.5\textwidth]{figures/mhtTemplate/examples/Pho/ht900_1050_control_eq1b_eq5j_Pho.pdf}
  }\\
  \subfigure[\mj, 2b, 5j category and \scalht 600-750\GeV bin]{
    \includegraphics[width=0.5\textwidth]{figures/mhtTemplate/examples/Mu/ht600_750_control_eq2b_eq5j_Mu.pdf}
  }~~
  \subfigure[\mmj, 1b, 2j category and \scalht 400-500\GeV bin]{
    \includegraphics[width=0.5\textwidth]{figures/mhtTemplate/examples/MuMu/ht400_500_control_eq1b_eq2j_MuMu.pdf}
  }\\
  \caption{\label{fig:linearFitExamples} 
  The data/MC distribution against \mht for example categories and control regions.
  The large bias in the linear component seen in Fig.~\ref{fig:linearMotiv} is mitigated.}
\end{figure}

\begin{figure}[h!]
  \centering
  \subfigure[\mj]{
    \includegraphics[width=0.5\textwidth]{figures/mhtTemplate/pulls/Mu}
  }~~
  \subfigure[\gj]{
    \includegraphics[width=0.5\textwidth]{figures/mhtTemplate/pulls/Pho}
  }\\
  \subfigure[\mmj]{
    \includegraphics[width=0.5\textwidth]{figures/mhtTemplate/pulls/MuMu}
  }~~
  \\
  \caption{\label{fig:pulls} 
  The pull distribution of the linear parameter from the flat hypothesis showing no large significant bias.}
\end{figure}
\begin{figure}[h!]
  \centering
  \subfigure[\mj]{
    \includegraphics[width=0.5\textwidth]{figures/mhtTemplate/pulls/Mu_2D}
  }~~
  \subfigure[\gj]{
    \includegraphics[width=0.5\textwidth]{figures/mhtTemplate/pulls/Pho_2D}
  }\\
  \subfigure[\mmj]{
    \includegraphics[width=0.5\textwidth]{figures/mhtTemplate/pulls/MuMu_2D}
  }~~
  \\
  \caption{\label{fig:frenchFlagPulls} The pull distribution of the linear parameter from the flat hypothesis across all
  \scalht bins and categories. There are no pulls for the \scalht binned are approximately consistent with gaussian scatter}
\end{figure}

% \begin{figure}[h!]
%   \centering
%   \subfigure[\mj]{
%     \includegraphics[width=0.5\textwidth]{figures/template2016Data/shapeOutput12Fb/scale_ht_variable_mht/SingleMu/fitOut/Linear2DShiftMean/pValue_Linear2DShiftMean_SingleMu.pdf}
%   }~~
%   \subfigure[\gj]{
%     \includegraphics[width=0.5\textwidth]{figures/template2016Data/shapeOutput12Fb/scale_ht_variable_mht/SinglePhoton/fitOut/Linear2DShiftMean/pValue_Linear2DShiftMean_SinglePhoton.pdf}
%   }\\
%   \subfigure[\mmj]{
%     \includegraphics[width=0.5\textwidth]{figures/template2016Data/shapeOutput12Fb/scale_ht_variable_mht/DoubleMu/fitOut/Linear2DShiftMean/pValue_Linear2DShiftMean_DoubleMu.pdf}
%   }~~
%   \\
%   \caption{\label{fig:pValues} The distributions of the p-value for the linear fit.} 
% \end{figure}
\subsection{Deriving systematics on the \texorpdfstring{\mht}~dimension}
\label{sec:systMhtDimension}

Systematics in the \mht dimension are extracted using the data in the control 
regions to determine the statistical precision to which the hypothesis of zero bias can
be confirmed. In this analysis the main discriminating variables
are \njet,\nb and \scalht. Two sets of systematic uncertainties on the \mht distributions,
correlated in \scalht and \njet, are derived using data from the control regions.
The set of systematics decorrelated in \scalht allows any disagreement which 
is localised at a particular scale to be covered while the set of systematics 
decorrelated in \njet covers systematic effects localised in the
number of partons or ISR jets. The \nb dimension is not decorrelated as this discriminating
variable is not directly related to the scale of the event and will be strongly correlated to \njet.
The sets of 7 and 5 systematics for decorrelation in \njet and \scalht respectively
are derived in data as described in the following.

Each background in the signal region (\ttbar/W  and \zInv~) is predicted 
using several control regions. In order to determine the uncertainty in
the \mht dimension a combined linear fit is made over all relevant control regions
In each case the normalisation term is decorrelated across all categories while the linear 
term is correlated across the correlated categories for the systematic. The 
post fit values and uncertainties of the linear parameters are shown in
Figure~\ref{fig:postFitPerNJet} and Figure~\ref{fig:postFitPerHt} for the systematics decorrelated in \njet and \scalht respectively.
The uncertainty on the linear parameters from the fit are then
used to define the up and down one sigma variations of the nominal templates in the signal region
for each systematic. As a conservative estimate, the best fit value of the parameter is 
added in quadrature to its uncertainty in order to derive the overall variation for each template.
The templates decorrelated in \njet and \scalht are added as independant nuisances on the \mht 
distribution in the signal region in the fit.

Example templates with this uncertainty are shown in Fig.~\ref{fig:mht-templates-sym}, \ref{fig:mht-templates-asym} 
in Sec.~\ref{sec:results}. Here the template variations for both relevant 
backgrounds are combined to show the overall uncertainty on the \mht dimension. 
The scatter of the points around one is compatible with statistical fluctuations.

The relative uncertainties per 100 \GeV from the weighted mean of \mht
for \ttbar/W and \zInv~ are shown in Fig.~\ref{fig:uncPerNJet}
and Fig.~\ref{fig:uncPerHt} for the systematics decorrelated in
\njet and \scalht respectively. As an additional gauge of the 
effect of the template variations the uncertainty parameter 
can be translate to an uncertainty on the last bin. This is typically
of the order of 10\% depending on the category and is shown
in Fig.~\ref{fig:frenchFlagLastBin} for the \ttbar/W prediction.

\begin{figure}[h!]
  \centering
  \subfigure[\ttbar/W]{
    \includegraphics[width=0.5\textwidth]{figures/mhtTemplate/postFitValues/Mu_perNJet_graph.pdf}
  }~~
  \subfigure[\zInv~]{
    \includegraphics[width=0.5\textwidth]{figures/mhtTemplate/postFitValues/All_perNJet_graph.pdf}
  }\\
  \caption{\label{fig:postFitPerNJet} 
  Post fit values and uncertainties of the linear parameters used to determine the systematics decorrelated in \njet for 
  the \ttbar/W and \zInv~ predictions}
\end{figure}
\begin{figure}[h!]
  \centering
  \subfigure[\ttbar/W]{
    \includegraphics[width=0.5\textwidth]{figures/mhtTemplate/postFitValues/Mu_perHt_graph.pdf}
  }~~
  \subfigure[\zInv~]{
    \includegraphics[width=0.5\textwidth]{figures/mhtTemplate/postFitValues/All_perHt_graph.pdf}
  }\\
  \caption{\label{fig:postFitPerHt} 
  Post fit values and uncertainties of the linear parameters used to determine the systematics decorrelated in \scalht for 
  the \ttbar/W and \zInv~ predictions}
\end{figure}
\begin{figure}[h!]
  \centering
  \subfigure[\ttbar/W]{
    \includegraphics[width=0.5\textwidth]{figures/mhtTemplate/postFitValues/Mu_perNJet_unc.pdf}
  }~~
  \subfigure[\zInv~]{
    \includegraphics[width=0.5\textwidth]{figures/mhtTemplate/postFitValues/All_perNJet_unc.pdf}
  }\\
  \caption{\label{fig:uncPerNJet} 
     for the \ttbar/W and \zInv~ predictionsPercentage uncertainties per 100 GeV on the \mht distribution in the signal region for the systematics decorrelated in \njet
    for the \ttbar/W and \zInv~ predictions}
\end{figure}
\begin{figure}[h!]
  \centering
  \subfigure[\ttbar/W]{
    \includegraphics[width=0.5\textwidth]{figures/mhtTemplate/postFitValues/Mu_perHt_unc.pdf}
  }~~
  \subfigure[\zInv~]{
    \includegraphics[width=0.5\textwidth]{figures/mhtTemplate/postFitValues/All_perHt_unc.pdf}
  }\\
  \caption{\label{fig:uncPerHt} 
    Percentage uncertainties per 100 GeV on the \mht distribution in the signal region for the systematics decorrelated in \scalht
    for the \ttbar/W and \zInv~ predictions}
\end{figure}

\begin{figure}[h!]
  \centering
  \subfigure[\ttbar/W]{
    \includegraphics[width=0.5\textwidth]{figures/mhtTemplate/postFitValues/quadAddMu.pdf}
  }~~
  \subfigure[\zInv~]{
    \includegraphics[width=0.5\textwidth]{figures/mhtTemplate/postFitValues/quadAddAll.pdf}
  }\\
  \caption{\label{fig:frenchFlagLastBin} Observed percentage uncertainties in the 
  last bin of \mht over all categories and \scalht bins for the \ttbar/W and \zInv~ predictions.}
\end{figure}

\newpage
% \subsection{Comparison to known systematic sources}
% \label{sec:mcSystStudiesShape}
% To further validate the data driven procedure a range of known systematic sources are studied.
% The size of the variation of the \mht distribution under $\pm1\sigma$ shifts of these sources is compared 
% to that of the orthogonal polynomial systematic described in \ref{sec:valid13} to ensure they are
% covered. The sources considered are the b-tag scale factor uncertainty, jet energy correction uncertainty, 
% pile-up reweighting uncertainty, and top \Pt rewieghting uncertainty. For each source of systematic the prediction
% is varied by $\pm1\sigma$. To study the effect of this systematic variation on the \mht dimension, and not normalisation, 
% the resulting template is normalised to the nominal template in each jet category and \scalht bin. 
% These systematic effects can then be compared to the data-driven orthogonal polynomial systematic. 
%
% Figures~\ref{fig:mcCompLow} and \ref{fig:mcCompHigh} show the representative examples for two different \scalht 
% bins and jet-categories. As can be seen in each case the \mht dimension change under the variation of all 
% systematics considered is easily contained within the orthogonal polynomial variation. 
% To show the effect across all jet categories, except the monojet categories for which the \mht dimension is not used,
% and \scalht bins Fig~\ref{fig:lastBinVar} shows the maximum upwards/downwards variation in the last 
% \mht bin as a proportion of the orthogonal polynomial variation. In almost all bins this is shown to be 
% significantly less than unity confirming that the orthogonal polynomial can cover the known systematic
% deviations.
%
% \begin{figure}[h!]
%   \centering
%   \includegraphics[width=0.8\textwidth]{figures/template13TeV/mcComparison6fb/totalSMS-T1tttt_mGluino-1000_mLSP-100_25ns_mht_ge5j_ge3b_800.pdf}
%   \caption{\label{fig:mcCompLow} MC based systematic variations shown to be considerably smaller 
%   than the orthogonal polynomial data-driven systematic for an example bin \scalht $800-\infty$, \njet $\geq 5$, \nb $\geq 2$.}
% \end{figure}
% \begin{figure}[h!]
%   \centering
%   \includegraphics[width=0.8\textwidth]{figures/template13TeV/mcComparison6fb/totalSMS-T1tttt_mGluino-1000_mLSP-100_25ns_mht_eq2j_eq0b_400.pdf}
%   \caption{\label{fig:mcCompHigh} MC based systematic variations shown to be considerably smaller 
%   than the orthogonal polynomial data-driven systematic for an example bin \scalht 400-500, \njet $= 2$, \nb $= 0$.}
% \end{figure}
%
% \begin{figure}[h!]
% \centering
% \subfigure[Upwards deviation]{
% \includegraphics[width=0.5\textwidth]{figures/template13TeV/mcComparison6fb/lastBinRatioMax.pdf}
% }
% \subfigure[Downwards deviation]{
% \includegraphics[width=0.5\textwidth]{figures/template13TeV/mcComparison6fb/lastBinRatioMin.pdf}
% }\\
% \caption{\label{fig:lastBinVar} Maximum upwards and downwards variations in the last \mht bin as a proportion of the orthogonal polynomial deviation}
% \end{figure}
