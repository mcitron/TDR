%%____________________________________________________________________________||
\section{Data sets}
\label{sec:datasets}

\subsection{Data}


In this note, we use 36.4~\ifb of proton-proton collision data at
$\sqrt{s} =$ 13~TeV collected in 2016. The JSON file listed in
Table~\ref{tab:cert_json} specifies the periods of the time in which
these certified data are collected. Table~\ref{tab:datasets_data} lists
the names of the data sets.

We perform a blind analysis. The data in the signal region is
currently blinded with the exception of 5.2~\ifb of the certified data
in Run2016G, as recommended by the SUS PAG. We have intentionally not
unblinded the ICHEP data set. The control regions are populated with
the full 2016 data set.

\begin{table}[!h]
  \topcaption{The JSON file used to specify the certified data set of
    36.4~\ifb} 
  \footnotesize
  %latex.default(d, title = NULL, booktabs = FALSE, width = 3, rowname = NULL,     helvetica = FALSE, caption.loc = "bottom", ...)%
\begin{center}
\begin{tabular}{c}
\hline\hline
\verb!Cert_246908-258750_13TeV_PromptReco_Collisions15_25ns_JSON.txt!\tabularnewline
\hline
\end{tabular}\end{center}
 
  \label{tab:cert_json}
\end{table}

\begin{table}[!h]
  \topcaption{Data sets}
  \footnotesize %latex.default(d, title = NULL, booktabs = FALSE, width = 3, rowname = NULL,     helvetica = FALSE, caption.loc = "bottom", ...)%
\begin{center}
\begin{tabular}{lr}
\hline\hline
\multicolumn{1}{c}{Data set}&\multicolumn{1}{c}{$\int\mathcal{L}\textrm{d}t [\textrm{pb}^{-1}]$}\tabularnewline
\hline
\verb!/HTMHT/Run2015D-05Oct2015-v1/MINIAOD! &$ 551.60$\tabularnewline
\verb!/HTMHT/Run2015D-PromptReco-v4/MINIAOD! &$1599.66$\tabularnewline
\verb!/JetHT/Run2015D-05Oct2015-v1/MINIAOD! &$ 552.67$\tabularnewline
\verb!/JetHT/Run2015D-PromptReco-v4/MINIAOD! &$1599.66$\tabularnewline
\verb!/MET/Run2015D-05Oct2015-v1/MINIAOD! &$ 552.67$\tabularnewline
\verb!/MET/Run2015D-PromptReco-v4/MINIAOD! &$1599.66$\tabularnewline
\verb!/SingleElectron/Run2015D-05Oct2015-v1/MINIAOD! &$ 552.63$\tabularnewline
\verb!/SingleElectron/Run2015D-PromptReco-v4/MINIAOD! &$1599.11$\tabularnewline
\verb!/SingleMuon/Run2015D-05Oct2015-v1/MINIAOD! &$ 552.67$\tabularnewline
\verb!/SingleMuon/Run2015D-PromptReco-v4/MINIAOD! &$1599.53$\tabularnewline
\verb!/SinglePhoton/Run2015D-05Oct2015-v1/MINIAOD! &$ 552.67$\tabularnewline
\verb!/SinglePhoton/Run2015D-PromptReco-v4/MINIAOD! &$1598.83$\tabularnewline
\hline
\end{tabular}\end{center}

  \label{tab:datasets_data}
\end{table}

%% \begin{table}[!h]
%% \topcaption{The JSON file specifying the 0.8~\ifb of the certified
%% data that we never blind} \footnotesize
%% \input{tables/datasets/cert_unblind_json.tex}
%% \label{tab:cert_unblind_json}
%% \end{table}

\subsection{Simulation}

Table~\ref{tab:datasets_bkg} lists the data sets of simulated events
of the standard model background processes used in this note. In these
data sets, in addition to the main interaction, each event contains on
average 20 minimum bias interactions which simulate multiple
interactions per bunch-crossing (in-time pileup). The expected
detector signal from previous or following bunch crossings
(out-of-time pileup) with 25ns bunch spacing is also overlaid.

\begin{table}[!h]
  \centering
  \topcaption{Simulated background samples}
  \tiny
  \scalebox{.7}[1.0]{%latex.default(d, title = NULL, booktabs = FALSE, width = 3, rowname = NULL,     helvetica = FALSE, caption.loc = "bottom", ...)%
\begin{center}
\begin{tabular}{ll}
\hline\hline
\multicolumn{1}{c}{Data set}&\multicolumn{1}{c}{Cross section [pb]}\tabularnewline
\hline
\verb!/TT_TuneCUETP8M1_13TeV-powheg-pythia8/RunIISpring16MiniAODv2-PUSpring16_80X_mcRun2_asymptotic_2016_miniAODv2_v0_ext4-v1/MINIAODSIM! &$8.318\times 10^{+02}$\tabularnewline
\verb!/TTJets_HT-600to800_TuneCUETP8M1_13TeV-madgraphMLM-pythia8/RunIISpring16MiniAODv2-PUSpring16_80X_mcRun2_asymptotic_2016_miniAODv2_v0_ext1-v1/MINIAODSIM! &$2.667\times 10^{+00}$\tabularnewline
\verb!/TTJets_HT-800to1200_TuneCUETP8M1_13TeV-madgraphMLM-pythia8/RunIISpring16MiniAODv2-PUSpring16_80X_mcRun2_asymptotic_2016_miniAODv2_v0_ext1-v1/MINIAODSIM! &$1.098\times 10^{+00}$\tabularnewline
\verb!/TTJets_HT-1200to2500_TuneCUETP8M1_13TeV-madgraphMLM-pythia8/RunIISpring16MiniAODv2-PUSpring16_80X_mcRun2_asymptotic_2016_miniAODv2_v0_ext1-v1/MINIAODSIM! &$1.987\times 10^{-01}$\tabularnewline
\verb!/TTJets_HT-2500toInf_TuneCUETP8M1_13TeV-madgraphMLM-pythia8/RunIISpring16MiniAODv2-PUSpring16_80X_mcRun2_asymptotic_2016_miniAODv2_v0-v1/MINIAODSIM! &$2.368\times 10^{-03}$\tabularnewline
\verb!/TTJets_SingleLeptFromT_TuneCUETP8M1_13TeV-madgraphMLM-pythia8/RunIISpring16MiniAODv2-PUSpring16_80X_mcRun2_asymptotic_2016_miniAODv2_v0-v1/MINIAODSIM! &$1.827\times 10^{+02}$\tabularnewline
\verb!/TTJets_SingleLeptFromTbar_TuneCUETP8M1_13TeV-madgraphMLM-pythia8/RunIISpring16MiniAODv2-PUSpring16_80X_mcRun2_asymptotic_2016_miniAODv2_v0-v1/MINIAODSIM! &$1.827\times 10^{+02}$\tabularnewline
\verb!/TTJets_SingleLeptFromTbar_TuneCUETP8M1_13TeV-madgraphMLM-pythia8/RunIISpring16MiniAODv2-PUSpring16_80X_mcRun2_asymptotic_2016_miniAODv2_v0_ext1-v1/MINIAODSIM! &$1.827\times 10^{+02}$\tabularnewline
\verb!/TTJets_DiLept_TuneCUETP8M1_13TeV-madgraphMLM-pythia8/RunIISpring16MiniAODv2-PUSpring16_80X_mcRun2_asymptotic_2016_miniAODv2_v0_ext1-v1/MINIAODSIM! &$8.829\times 10^{+01}$\tabularnewline
\verb!/WJetsToLNu_TuneCUETP8M1_13TeV-madgraphMLM-pythia8/RunIISpring16MiniAODv1-PUSpring16_80X_mcRun2_asymptotic_2016_v3-v2/MINIAODSIM! &$6.153\times 10^{+04}$\tabularnewline
\verb!/WJetsToLNu_HT-100To200_TuneCUETP8M1_13TeV-madgraphMLM-pythia8/RunIISpring16MiniAODv2-PUSpring16_80X_mcRun2_asymptotic_2016_miniAODv2_v0_ext1-v1/MINIAODSIM! &$1.627\times 10^{+03}$\tabularnewline
\verb!/WJetsToLNu_HT-200To400_TuneCUETP8M1_13TeV-madgraphMLM-pythia8/RunIISpring16MiniAODv2-PUSpring16_80X_mcRun2_asymptotic_2016_miniAODv2_v0_ext1-v1/MINIAODSIM! &$4.352\times 10^{+02}$\tabularnewline
\verb!/WJetsToLNu_HT-400To600_TuneCUETP8M1_13TeV-madgraphMLM-pythia8/RunIISpring16MiniAODv2-PUSpring16_80X_mcRun2_asymptotic_2016_miniAODv2_v0-v1/MINIAODSIM! &$5.918\times 10^{+01}$\tabularnewline
\verb!/WJetsToLNu_HT-600To800_TuneCUETP8M1_13TeV-madgraphMLM-pythia8/RunIISpring16MiniAODv2-PUSpring16_80X_mcRun2_asymptotic_2016_miniAODv2_v0-v1/MINIAODSIM! &$1.458\times 10^{+01}$\tabularnewline
\verb!/WJetsToLNu_HT-800To1200_TuneCUETP8M1_13TeV-madgraphMLM-pythia8/RunIISpring16MiniAODv2-PUSpring16_80X_mcRun2_asymptotic_2016_miniAODv2_v0_ext1-v1/MINIAODSIM! &$6.656\times 10^{+00}$\tabularnewline
\verb!/WJetsToLNu_HT-1200To2500_TuneCUETP8M1_13TeV-madgraphMLM-pythia8/RunIISpring16MiniAODv2-PUSpring16_80X_mcRun2_asymptotic_2016_miniAODv2_v0-v1/MINIAODSIM! &$1.608\times 10^{+00}$\tabularnewline
\verb!/WJetsToLNu_HT-2500ToInf_TuneCUETP8M1_13TeV-madgraphMLM-pythia8/RunIISpring16MiniAODv2-PUSpring16_80X_mcRun2_asymptotic_2016_miniAODv2_v0-v1/MINIAODSIM! &$3.891\times 10^{-02}$\tabularnewline
\verb!/ZJetsToNuNu_HT-100To200_13TeV-madgraph/RunIISpring15DR74-Asympt25ns_MCRUN2_74_V9-v1/MINIAODSIM! &$3.450\times 10^{+02}$\tabularnewline
\verb!/ZJetsToNuNu_HT-200To400_13TeV-madgraph/RunIISpring15DR74-Asympt25ns_MCRUN2_74_V9-v1/MINIAODSIM! &$9.638\times 10^{+01}$\tabularnewline
\verb!/ZJetsToNuNu_HT-400To600_13TeV-madgraph/RunIISpring15DR74-Asympt25ns_MCRUN2_74_V9-v1/MINIAODSIM! &$1.346\times 10^{+01}$\tabularnewline
\verb!/ZJetsToNuNu_HT-600ToInf_13TeV-madgraph/RunIISpring15DR74-Asympt25ns_MCRUN2_74_V9-v1/MINIAODSIM! &$5.170\times 10^{+00}$\tabularnewline
\verb!/QCD_HT300to500_TuneCUETP8M1_13TeV-madgraphMLM-pythia8/RunIISpring16MiniAODv2-PUSpring16_80X_mcRun2_asymptotic_2016_miniAODv2_v0_ext1-v1/MINIAODSIM! &$3.477\times 10^{+05}$\tabularnewline
\verb!/QCD_HT700to1000_TuneCUETP8M1_13TeV-madgraphMLM-pythia8/RunIISpring16MiniAODv2-PUSpring16_80X_mcRun2_asymptotic_2016_miniAODv2_v0-v1/MINIAODSIM! &$6.831\times 10^{+03}$\tabularnewline
\verb!/QCD_HT700to1000_TuneCUETP8M1_13TeV-madgraphMLM-pythia8/RunIISpring16MiniAODv2-PUSpring16_80X_mcRun2_asymptotic_2016_miniAODv2_v0_ext1-v1/MINIAODSIM! &$6.831\times 10^{+03}$\tabularnewline
\verb!/QCD_HT1000to1500_TuneCUETP8M1_13TeV-madgraphMLM-pythia8/RunIISpring16MiniAODv2-PUSpring16_80X_mcRun2_asymptotic_2016_miniAODv2_v0-v2/MINIAODSIM! &$1.207\times 10^{+03}$\tabularnewline
\verb!/QCD_HT1000to1500_TuneCUETP8M1_13TeV-madgraphMLM-pythia8/RunIISpring16MiniAODv2-PUSpring16_80X_mcRun2_asymptotic_2016_miniAODv2_v0_ext1-v1/MINIAODSIM! &$1.207\times 10^{+03}$\tabularnewline
\verb!/QCD_HT1500to2000_TuneCUETP8M1_13TeV-madgraphMLM-pythia8/RunIISpring16MiniAODv2-PUSpring16_80X_mcRun2_asymptotic_2016_miniAODv2_v0-v3/MINIAODSIM! &$1.199\times 10^{+02}$\tabularnewline
\verb!/QCD_HT1500to2000_TuneCUETP8M1_13TeV-madgraphMLM-pythia8/RunIISpring16MiniAODv2-PUSpring16_80X_mcRun2_asymptotic_2016_miniAODv2_v0_ext1-v1/MINIAODSIM! &$1.199\times 10^{+02}$\tabularnewline
\verb!/QCD_HT2000toInf_TuneCUETP8M1_13TeV-madgraphMLM-pythia8/RunIISpring16MiniAODv2-PUSpring16_80X_mcRun2_asymptotic_2016_miniAODv2_v0-v1/MINIAODSIM! &$2.524\times 10^{+01}$\tabularnewline
\verb!/QCD_HT2000toInf_TuneCUETP8M1_13TeV-madgraphMLM-pythia8/RunIISpring16MiniAODv2-PUSpring16_80X_mcRun2_asymptotic_2016_miniAODv2_v0_ext1-v1/MINIAODSIM! &$2.524\times 10^{+01}$\tabularnewline
\verb!/QCD_HT100to200_TuneCUETP8M1_13TeV-madgraphMLM-pythia8/RunIISpring15DR74-Asympt25ns_MCRUN2_74_V9-v2/MINIAODSIM! &$2.785\times 10^{+07}$\tabularnewline
\verb!/QCD_HT200to300_TuneCUETP8M1_13TeV-madgraphMLM-pythia8/RunIISpring15DR74-Asympt25ns_MCRUN2_74_V9-v2/MINIAODSIM! &$1.717\times 10^{+06}$\tabularnewline
\verb!/QCD_HT300to500_TuneCUETP8M1_13TeV-madgraphMLM-pythia8/RunIISpring15DR74-Asympt25ns_MCRUN2_74_V9-v2/MINIAODSIM! &$3.513\times 10^{+05}$\tabularnewline
\verb!/QCD_HT500to700_TuneCUETP8M1_13TeV-madgraphMLM-pythia8/RunIISpring15DR74-Asympt25ns_MCRUN2_74_V9-v1/MINIAODSIM! &$3.163\times 10^{+04}$\tabularnewline
\verb!/QCD_HT700to1000_TuneCUETP8M1_13TeV-madgraphMLM-pythia8/RunIISpring15DR74-Asympt25ns_MCRUN2_74_V9-v1/MINIAODSIM! &$6.802\times 10^{+03}$\tabularnewline
\verb!/QCD_HT1000to1500_TuneCUETP8M1_13TeV-madgraphMLM-pythia8/RunIISpring15DR74-Asympt25ns_MCRUN2_74_V9-v2/MINIAODSIM! &$1.206\times 10^{+03}$\tabularnewline
\verb!/QCD_HT1500to2000_TuneCUETP8M1_13TeV-madgraphMLM-pythia8/RunIISpring15DR74-Asympt25ns_MCRUN2_74_V9-v1/MINIAODSIM! &$1.204\times 10^{+02}$\tabularnewline
\verb!/QCD_HT2000toInf_TuneCUETP8M1_13TeV-madgraphMLM-pythia8/RunIISpring15DR74-Asympt25ns_MCRUN2_74_V9-v1/MINIAODSIM! &$2.525\times 10^{+01}$\tabularnewline
\verb!/DYJetsToLL_M-50_TuneCUETP8M1_13TeV-amcatnloFXFX-pythia8/RunIISpring16MiniAODv2-PUSpring16_80X_mcRun2_asymptotic_2016_miniAODv2_v0-v1/MINIAODSIM! &$6.025\times 10^{+03}$\tabularnewline
\verb!/DYJetsToLL_M-50_TuneCUETP8M1_13TeV-madgraphMLM-pythia8/RunIISpring16MiniAODv2-PUSpring16_80X_mcRun2_asymptotic_2016_miniAODv2_v0_ext1-v1/MINIAODSIM! &$6.025\times 10^{+03}$\tabularnewline
\verb!/DYJetsToLL_M-50_HT-100to200_TuneCUETP8M1_13TeV-madgraphMLM-pythia8/RunIISpring16MiniAODv2-PUSpring16_80X_mcRun2_asymptotic_2016_miniAODv2_v0_ext1-v1/MINIAODSIM! &$1.813\times 10^{+02}$\tabularnewline
\verb!/DYJetsToLL_M-50_HT-200to400_TuneCUETP8M1_13TeV-madgraphMLM-pythia8/RunIISpring16MiniAODv2-PUSpring16_80X_mcRun2_asymptotic_2016_miniAODv2_v0_ext1-v1/MINIAODSIM! &$5.042\times 10^{+01}$\tabularnewline
\verb!/DYJetsToLL_M-50_HT-400to600_TuneCUETP8M1_13TeV-madgraphMLM-pythia8/RunIISpring16MiniAODv2-PUSpring16_80X_mcRun2_asymptotic_2016_miniAODv2_v0_ext1-v1/MINIAODSIM! &$6.984\times 10^{+00}$\tabularnewline
\verb!/DYJetsToLL_M-50_HT-600toInf_TuneCUETP8M1_13TeV-madgraphMLM-pythia8/RunIISpring16MiniAODv2-PUSpring16_80X_mcRun2_asymptotic_2016_miniAODv2_v0-v1/MINIAODSIM! &$2.704\times 10^{+00}$\tabularnewline
\verb!/DYJetsToLL_M-50_HT-600toInf_TuneCUETP8M1_13TeV-madgraphMLM-pythia8/RunIISpring16MiniAODv2-PUSpring16_80X_mcRun2_asymptotic_2016_miniAODv2_v0_ext1-v1/MINIAODSIM! &$2.704\times 10^{+00}$\tabularnewline
\verb!/GJets_HT-100To200_TuneCUETP8M1_13TeV-madgraphMLM-pythia8/RunIISpring16MiniAODv2-PUSpring16_80X_mcRun2_asymptotic_2016_miniAODv2_v0-v4/MINIAODSIM! &$9.238\times 10^{+03}$\tabularnewline
\verb!/GJets_HT-200To400_TuneCUETP8M1_13TeV-madgraphMLM-pythia8/RunIISpring16MiniAODv2-PUSpring16_80X_mcRun2_asymptotic_2016_miniAODv2_v0-v1/MINIAODSIM! &$2.305\times 10^{+03}$\tabularnewline
\verb!/GJets_HT-400To600_TuneCUETP8M1_13TeV-madgraphMLM-pythia8/RunIISpring16MiniAODv2-PUSpring16_80X_mcRun2_asymptotic_2016_miniAODv2_v0-v1/MINIAODSIM! &$2.744\times 10^{+02}$\tabularnewline
\verb!/GJets_HT-600ToInf_TuneCUETP8M1_13TeV-madgraphMLM-pythia8/RunIISpring16MiniAODv2-PUSpring16_80X_mcRun2_asymptotic_2016_miniAODv2_v0-v1/MINIAODSIM! &$9.346\times 10^{+01}$\tabularnewline
\verb!/ttHJetToNonbb_M125_13TeV_amcatnloFXFX_madspin_pythia8_mWCutfix/RunIISpring16MiniAODv1-PUSpring16RAWAODSIM_80X_mcRun2_asymptotic_2016_v3_ext1-v1/MINIAODSIM! &$2.151\times 10^{-01}$\tabularnewline
\verb!/ttHJetTobb_M125_13TeV_amcatnloFXFX_madspin_pythia8/RunIISpring16MiniAODv1-PUSpring16RAWAODSIM_80X_mcRun2_asymptotic_2016_v3_ext3-v1/MINIAODSIM! &$2.934\times 10^{-01}$\tabularnewline
\verb!/TTGJets_TuneCUETP8M1_13TeV-amcatnloFXFX-madspin-pythia8/RunIISpring16MiniAODv2-PUSpring16_80X_mcRun2_asymptotic_2016_miniAODv2_v0-v1/MINIAODSIM! &$3.697\times 10^{+00}$\tabularnewline
\verb!/TTWJetsToLNu_TuneCUETP8M1_13TeV-amcatnloFXFX-madspin-pythia8/RunIISpring16MiniAODv2-PUSpring16_80X_mcRun2_asymptotic_2016_miniAODv2_v0-v1/MINIAODSIM! &$2.043\times 10^{-01}$\tabularnewline
\verb!/TTWJetsToQQ_TuneCUETP8M1_13TeV-amcatnloFXFX-madspin-pythia8/RunIISpring16MiniAODv2-PUSpring16_80X_mcRun2_asymptotic_2016_miniAODv2_v0-v1/MINIAODSIM! &$4.062\times 10^{-01}$\tabularnewline
\verb!/TTZToLLNuNu_M-10_TuneCUETP8M1_13TeV-amcatnlo-pythia8/RunIISpring16MiniAODv2-PUSpring16_80X_mcRun2_asymptotic_2016_miniAODv2_v0-v1/MINIAODSIM! &$2.529\times 10^{-01}$\tabularnewline
\verb!/TTZToQQ_TuneCUETP8M1_13TeV-amcatnlo-pythia8/RunIISpring16MiniAODv2-PUSpring16_80X_mcRun2_asymptotic_2016_miniAODv2_v0-v1/MINIAODSIM! &$5.297\times 10^{-01}$\tabularnewline
\verb!/WW_TuneCUETP8M1_13TeV-pythia8/RunIISpring16MiniAODv2-PUSpring16_80X_mcRun2_asymptotic_2016_miniAODv2_v0-v1/MINIAODSIM! &$1.139\times 10^{+02}$\tabularnewline
\verb!/WZ_TuneCUETP8M1_13TeV-pythia8/RunIISpring16MiniAODv2-PUSpring16_80X_mcRun2_asymptotic_2016_miniAODv2_v0-v1/MINIAODSIM! &$4.713\times 10^{+01}$\tabularnewline
\verb!/ZZ_TuneCUETP8M1_13TeV-pythia8/RunIISpring16MiniAODv2-PUSpring16_80X_mcRun2_asymptotic_2016_miniAODv2_v0-v1/MINIAODSIM! &$1.652\times 10^{+01}$\tabularnewline
\verb!/ST_s-channel_4f_leptonDecays_13TeV-amcatnlo-pythia8_TuneCUETP8M1/RunIISpring16MiniAODv2-PUSpring16_80X_mcRun2_asymptotic_2016_miniAODv2_v0-v1/MINIAODSIM! &$3.681\times 10^{+00}$\tabularnewline
\verb!/ST_tW_antitop_5f_inclusiveDecays_13TeV-powheg-pythia8_TuneCUETP8M1/RunIISpring16MiniAODv2-PUSpring16_80X_mcRun2_asymptotic_2016_miniAODv2_v0-v1/MINIAODSIM! &$3.560\times 10^{+01}$\tabularnewline
\verb!/ST_tW_top_5f_inclusiveDecays_13TeV-powheg-pythia8_TuneCUETP8M1/RunIISpring16MiniAODv2-PUSpring16_80X_mcRun2_asymptotic_2016_miniAODv2_v0-v2/MINIAODSIM! &$3.560\times 10^{+01}$\tabularnewline
\hline
\end{tabular}\end{center}
}
  \label{tab:datasets_bkg}
\end{table}


\clearpage

\subsection{Pileup reweighting}
\label{sec:pileup-reweighting}

The distribution of the numbers of the pileup interactions in the
simulated events is different from that in the data; the simulated
events contain, on average, a larger number of pileup interactions than
the data. We reweight simulated events to correct for this difference.
This procedure is called \textit{pileup reweighting}.

In deriving the pileup reweighting factors, we follow the
recommendation by the physics validation group
\cite{twiki-PdmVPileUpDescription, twiki-PileupJSONFileforData}. In
the recommendation, the reweighting factors are a function of the
variable called \verb!nTrueInt!.

The variable \verb!nTrueInt! is the parameter of the Poisson
distribution from which the numbers of pileup interactions are drown
as random numbers. In each simulated event, the number of the in-time
pileup interactions and the number of the interactions in each
neighbouring bunch crossing to simulate the out-of-time pileup are
random numbers from the Poisson distribution with the same parameter,
\verb!nTrueInt!. The value of \verb!nTrueInt! is not a constant of the
data set. It is a random number from the distribution specified in
Ref. \cite{github-mix_2016_25ns_SpringMC_PUScenarioV1_PoissonOOTPU_cfi}.

The \verb!nTrueInt! in the data is the average pileup interactions for
a colliding bunch pair in a lumi section. The distribution of
\verb!nTrueInt! in the data is derived from the measured instantaneous
luminosity for each colliding bunch pair in each lumi section and the
cross section of the total inelastic pp interaction. We use the method
in Ref. \cite{twiki-PileupJSONFileforData} in deriving the
distribution with the recommended value of 63.00~mb as the minimum
bias cross section. In addition, we derive distributions with $\pm
5\%$ of the variations of the minimum bias cross section, i.e,
66.15~mb and 59.850~mb.

The pileup reweighting factors are the ratios of the distributions of
\verb!nTrueInt! in the data and in the simulated events and are
normalised so as to preserve the number of the simulated events.

\begin{figure}[!b]
  \centering
  \includegraphics[scale=1.00]{figures/pileup_reweighting/f044_corr_nTrueInt_data_mc_norm}
  \caption{The distribution of the average numbers of the inelastic
    interactions per colliding bunch pair per lumi section in the data,
    corresponding distribution in the simulated events, and that of the
    reweighted simulated events.} \label{f044_corr_nTrueInt_data_mc_norm}
\end{figure}


Figure~\ref{f044_corr_nTrueInt_data_mc_norm} shows the distributions
of \verb!nTrueInt! in the data, simulated events and reweighted
simulated events. The figure demonstrates that the reweighted
simulated events have the distribution of \verb!nTrueInt! nearly
identical to that in the data. 
We note that the simulation does not contain events with 37 or more
overlapping pp interactions, while the data do, causing the MC
weighted distribution not to match perfectly the data. This small
issue will be recovered once the new MC samples with updated PU
profile will be available.



\subsection{Cross sections for SM samples}
\label{sec:SMxs}
This analysis chooses to use \MADGRAPH samples binned in partonic \HT 
for the set of MC samples (W+jets, DY+jets, QCD, $\gamma$+jets, $Z\rightarrow \nu\nu$+jets).
These binned samples are provided with LO cross sections. 
The \kfactors required to go from LO to NNLO cross section are typically determined using corresponding
inclusive samples applied to each \HT binned sample.
The inclusive distributions of the MC samples
with respect to the binning variable
$H_{T}^{parton}$ are shown in Fig.~\ref{fig:Lhe_Ht}, with
the bin by bin derivative drawn below. 
As can be seen in the distributions, the stitched samples
exhibit a good level of smoothness,
further demonstrated by the derivative which shows a flat trend for each 
cross section.

As described in Section~\ref{sec:mc-corrections}, residual cross
section corrections are measured using data in sidebands designed to
enrich specific SM processes. These corrections can prove to be
important for the closure test procedures described in Section
\ref{sec:closure-tests}.

\begin{figure}[!h]
  \begin{center}
    \subfigure[$Z\rightarrow \nu\nu$ +jets] {\includegraphics[width=0.40\textwidth]{figures/binnedMCsamples/2016/6p3/Zinv.pdf}} ~~
    \subfigure[$W\rightarrow l \nu$ + jets]{\includegraphics[width=0.40\textwidth]{figures/binnedMCsamples/2016/6p3/WJetsToLNu_HT.pdf}} \\
    \subfigure[$DY\rightarrow ll$ + jets]{\includegraphics[width=0.40\textwidth]{figures/binnedMCsamples/2016/6p3/DYJetsToLL_M50_HT.pdf}} ~~
    \subfigure[QCD]{\includegraphics[width=0.40\textwidth]{figures/binnedMCsamples/2016/6p3/QCD_HT.pdf}} \\
    \subfigure[$\gamma$+jets]{\includegraphics[width=0.40\textwidth]{figures/binnedMCsamples/2016/6p3/GJets_HT.pdf}} ~~
    \caption{Generator-level $H_{T}^{parton}$ distributions for SM process, $Z\rightarrow \nu\nu$ + jets, W+jets, DY+jets, QCD and $\gamma$+jets}
    \label{fig:Lhe_Ht}
  \end{center}
\end{figure}

%%____________________________________________________________________________||
