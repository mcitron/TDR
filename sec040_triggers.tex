%%____________________________________________________________________________||
\section{Triggers}
\label{sec:triggers}


In 2012 the RA1 analysis used four prompt and one parked $\hT-\alphaT$ cross-triggers, using calojet-based reconstruction with 40 GeV requirement. The Run 2 trigger strategy aims to maintain the accepatance of Run 1 with several developments in triggering of events.



In Run 2 the RA1 analysis will aim to retain the low-thresholds to keep sensitivity to signatures of new physics, where possible maintaining the original offline selections in Run 1 including the lowest $\hT = 200$ GeV bin. Several improvements to the analysis will be implemented to further improve of the analysis, jet reconstruction will be performed by the particle flow algorithm which exhibits better energy r however to mitigate the effects of pileup the jet reconstruction algorithm utlised in the final trigger decision and the offline analysis will be migrated to particle flow which exhibits better performance in high pileup.\\


% Pileup mitigation
Jet reconstruction will be performed with the anti-kt algorithm with radius parameter $\R = 0.4$ using PF-based reconstruction algorithm. The reduction in cone size reduces pileup contamination. PFJet energy resolution is stable with increasing pileup.


Central jets restricted to $|\eta| < 3$ with 40 GeV threshold.
Maintain the same acceptance as 2012.
Currently have menu approved of triggers

Able to maintain 2012 thresholds with the introduction of a minimum second jet 90 GeV requirement which controls rate with little loss of signal efficiency.
PF HT900 seed high HT bins

Table of triggers, rates and eff for (compressed model/backgrounds)

Aim to extend acceptance to compressed spectrum and DM models with inclusion of asymmetric dijet analysis bin offline, seeded by dijet average trigger.


% TABLE : 2012 analysis and triggers
%----------------------------------------------------------------------
\begin{table}[h!]
\tiny
\centering
\begin{tabular}{|rc||c|c||cc|} 
\hline
\multicolumn{2}{|c||}{Offline selection}         & L1 seed                  & HLT trigger                                               &   \multicolumn{2}{c|}{Efficiency (\%)}            \\[0.7 ex] 
    $\hT$ range (GeV)  & $\alphaT$ threshold &                               &                                                                  &  $2 \le n_{\rm jet} \le 3$ &   $n_{\rm jet} \ge 4$\\[0.7 ex] 
\hline
200 $\le \hT < 275$    & 0.65                         & DoubleJetC64                    &  ${\rm HLT\_HT200\_AlphaT0p57\_v*}$ & $81.8^{+0.4}_{-0.4}$    & $78.9^{+0.3}_{-0.4}$ \\
275 $\le \hT < 325$    & 0.60                         & DoubleJetC64                    &  ${\rm HLT\_HT250\_AlphaT0p55\_v*}$ & $95.2^{+0.3}_{-0.4}$    & $90.0^{+1.2}_{-1.3}$ \\
325 $\le \hT < 375$    & 0.55                         & DoubleJetC64 OR HTT175 &  ${\rm HLT\_HT300\_AlphaT0p53\_v*}$ & $97.9^{+0.3}_{-0.3}$    & $95.6^{+0.9}_{-1.0}$ \\
375 $\le \hT < 475$    & 0.55                         & DoubleJetC64 OR HTT175 &  ${\rm HLT\_HT350\_AlphaT0p52\_v*}$ & $99.2^{+0.2}_{-0.2}$    & $98.7^{+0.5}_{-0.7}$ \\
       $     \hT > 475$    & 0.55                         & DoubleJetC64 OR HTT175 &  ${\rm HLT\_HT400\_AlphaT0p51\_v*}$ & $99.8^{+0.1}_{-0.3}$    & $99.6^{+0.3}_{-0.7}$ \\
\hline


\end{tabular}
\caption{HLT paths for Run 1. {\color{red} Typo in AN2013\_366\_v3 for ${\rm HLT\_HT250\_AlphaT0p55\_v*}$? }  }
\end{table}




Compressed models can be observed under boost from ISR, yeilding a monojet or asymmetric dijet signature, the RA1 analysis will seek to improve the coverage with the extension to events with asymmetric dijet and monojet signatures, which require new trigger paths.\\

Trigger studies were performed by emulating the 2015 HLT with phase 1 L1 seeds on 13 TeV MC samples with simulation of the three Run 2 scenarios, a full list of the samples used in the study are listed in Appendix {\color{blue} A1}. To ensure the trigger selections were as inclusive as possible a range of signal models and SM backgrounds which exhibit SUSY event topologies were studied in the optimisation of trigger working points.\\

Offline selection: 50 GeV gen jets, 100 GeV second jet threshold, forward jet veto, lepton veto, MHT/MET cleaning. \\

\subsection{L1 seeds}

To estimate the rate for Level-1 studies a pure-pileup neutrino gun sample is utilised.

New jet algorithm, applying PUM0 subtraction to regions prior to clustering. HTT determined from pileup corrected regions exceeding 7 GeV threshold.
Due to the influence of pileup the HTT turnon is much slower [Include HTT turn on].  Compressed supersymmetry models, where there is a small mass splitting between the parent and daughter sparticle yields little visible energy in the final state necessitates low $\hT$ thresholds to maintain acceptance to such models. Require new triggers that effectively suppress QCD to enable a lowering of these thresholds.\\

{\color{blue} [ROC curves for 4 different options: HTT, MHT/HTT, DPhi trigger, DPhi x MHT/HTT}\\


\subsection{High level trigger}


The conditions expected in 2015 require an optimisation of trigger thresholds.\\



%%____________________________________________________________________________||
