%%____________________________________________________________________________||
\section{Summary}
\label{sec:summary}

We report on the prospects for missing energy plus jet searches that are very sensitive to 
 models of Supersymmetry and Dark Matter. The analysis follows an inclusive approach designed to 
capture a wide scope of possible final states, use of robust methods to be insensitive to multijet production,
instrumental effects and MC mis-modelling. Thus making it ideal for early data discoveries.

The proven Run I analysis concept has been extended to larger values of $H_\textrm{T}$ while also adding a new event category,
the asymetric jet selection. This selection uses an orthogonal requirement on the sub-leading jet to improve 
acceptance for ISR and monojet-type final states. This increases the acceptance of compressed SUSY models by about a factor of three
and DM models as much as a factor of five.

All signal selections region are binned according to the number of reconstructed jets, the scalar sum of the
transverse energy of jets, and the number of jets identified to originate from bottom quarks. The
sum of standard model backgrounds per bin has been estimated from a simultaneous binned
likelihood fit to event yields in the signal region and $\mu$ + jets, $\mu\mu$ + jets, and $\gamma$ + jets control sam-
ples. 

The addition of simplified and heavy quark flavored DM models yields to strong results on DM models in the spin-dependent and spin-independent nucleon cross section scattering plane in a largely model-independent way. These results may provide the strongest constraints with early 13~TeV data for low mass Dark Matter and the strongest collider constraints across a wide range of masses. In particular probe excesses observed in direct detection experiments at energies of about 10 GeV but also by the Fermi-LAT (2013) satellite indicating a DM particle of about 50 GeV mass. We expect to conclusively probe this region with the full Run II dataset.


The analysis also has been adapted to latest standards of the CMS collaboration. All analysis tools have been ported and verified using the \textsc{PHYS14} prescription in {\tt cmgtools}, the use of calojet off- and online has been replaced by particle flow (PF) jet and the trigger requirements have been adapted to maintain 2012 thresholds.
Further plans for this analysis will be to improve/increase the analysis binning for large $H_\textrm{T}$, improve and maintain trigger thresholds using PF trigger paths and evolve the selection according to the running conditions with increasing luminosity.

%\begin{itemize}
%  \item summarize the PHYS14 exercise described in the other sections
%  \item mention briefly further preparation plans for Run 2
%  \item conclude with the outlook for Run 2
%\end{itemize}

%%____________________________________________________________________________||
