%____________________________________________________________________________||
\section{Event selection}
\label{sec:selection}

\subsection{Event vetoes for leptons, photons, and single isolated tracks\label{sec:vetoes}}

To suppress SM processes with genuine \met from neutrinos, events
containing an isolated electron with $\pt > 20\GeV$ and $|\eta| < 2.5$ or an isolated muon
with $\pt > 10\GeV$ and $|\eta| < 2.5$ are vetoed. To select a pure
multijet topology, events are vetoed in which an isolated
photon with $\pt > 25\GeV$ and $|\eta| < 2.5$ is
found.  Further, to reduce the ``lost leptons'' backgrounds from W~+~jets 
and \ttbar, events containing single isolated tracks with $\pt >
10\GeV$ and $|\eta| < 2.5$, as defined in
Section~\ref{sec:reconstruction}, are vetoed as part of the signal
region selection criteria. In the case of the \mj and \mmj
samples, a further requirement is made such that events are not vetoed
due to the presence of a track from the well identified muons, by
requiring $\Delta R(\textrm{track},\mu) < 0.02$.


%%____________________________________________________________________________||
\subsection{Hadronic signal region selection}

%aim to main 2012 acceptance for compressed susy
Events are required to have significant hadronic activity by requiring
$\scalht > 200\GeV$. The aim is to keep this threshold at the level of the 2012
analysis~\cite{parkedDataAN} to maintain acceptance for SUSY models with
compressed spectra.  

%Jet selection

%alphaT cuts

% put in new alphaT cuts!! FIXME
\begin{table}[h!]
  \caption{\alphat and (effective) \mht/\scalht and \mht thresholds per \scalht bin.\label{tab:alphat-thresholds}}
  \centering
  \footnotesize
  \begin{tabular}{ lcccc }
    \hline
    \hline
    \scalht bin  & 200--275   & 275--325   & 325--375   & $>$375       \\
    \hline
    \alphat      & 0.65       & 0.60       & 0.55       & 0.55         \\
    \mht/\scalht & $\sim$0.64 & $\sim$0.55 & $\sim$0.42 & $\sim$0.42   \\
    \mht         & $\sim$130  & $\sim$150  & $\sim$135  & $\gtrsim$155 \\
    \hline
    \hline
  \end{tabular}
\end{table}

To protect against multiple jets failing the $\Et$ threshold, the
jet-based estimate of the missing transverse energy, \mht, is compared
to the Particle Flow estimate of missing transverse energy, $\pfmet$,
and events with $R_{\rm miss}=\mht/\pfmet > 1.25$ are rejected.

To protect against severe energy losses, events with significant jet
mismeasurements caused by masked regions in the ECAL (which amount to
about 1\% of the ECAL channel count), or by missing instrumentation in
the barrel-endcap gap, are removed with the following procedure. The
jet-based estimate of the missing transverse energy, \mht, is used to
identify jets most likely to have given rise to the \mht as those
whose momentum is closest in $\phi$ to the total $\vec{\mht}$ which
results after removing them from the event.  The azimuthal distance
between this jet and the recomputed \mht is referred to as
$\Delta\phi^*$ in what follows. Events with $\Delta\phi^* < 0.5$ are
rejected if the distance in the ($\eta,\phi$) plane between the
selected jet and the closest masked ECAL region, $\Delta R_{\rm
  ECAL}$, is smaller than 0.3. Similarly, events are rejected if the
jet points within 0.3 in $\eta$ of the ECAL barrel-endcap gap at
$|\eta| = 1.5$. These final selections complete the definition of the
acceptance of the hadronic signal sample.


\subsubsection{Reduction of \alphat thresholds for high \HT}

%HT>900 seeded by flat trigger so can choose low alphaT
%this maps flat onto MHT - show table from Tai talk

\subsubsection{Key distributions for the hadronic signal
  region\label{sec:mc-data-comp}}

%%____________________________________________________________________________||
\subsection{Breakdown of SM backgrounds in the hadronic signal
  region\label{sec:bkgd-comp}}

% put yield tables here

%%____________________________________________________________________________||
\subsection{Analysis bins}

Events in the hadronic signal region (and the
three control regions described in Sec.~\ref{sec:controlSelection}) are
categorised according to the number of jets (\njet) reconstructed in
each event and the number of jets identified as originating from
bottom quarks (\nb) in each event. By construction, $\nb \leq \njet$.

Additionally these categories are split up into \HT bins, detailed in 
Table~\ref{tab:htBins}.
%HT bins table

% Introduce 2012 bins?

\subsubsection{Introduction of asymmetric jet bin}

% introduction of asymmetric dijet bin, describe the bin and small motivation

\subsubsection{Extension of \HT bins}

\subsubsection{Fine \njet binning}

%%____________________________________________________________________________||
\subsection{Control region selection}
\label{sec:controlSelection}

\subsubsection{Hadronic control sample}

A disjoint hadronic control sample consisting predominantly of
multijet events is defined by applying the hadronic pre-selection
criteria and inverting the \alphat and/or \mhtmet requirements for a
given \scalht region, which is used primarily in the estimation of any
residual background from QCD multijet events, described in
Sec.~\ref{sec:qcd}.

\subsubsection{The \texorpdfstring{\mj}{muon plus jets} control sample}

\subsubsection{The \texorpdfstring{\mmj}{di-muon plus jets} control sample}

\subsubsection{The \texorpdfstring{\gj}{photon plus jets} control sample}

%%____________________________________________________________________________||
\subsection{Increasing the acceptance of the control samples\label{sec:larger}}

%Add electron control sample
%Gamma+jets - remove alphaT cut (due to artificial increase of photon pT) 

%%____________________________________________________________________________||
