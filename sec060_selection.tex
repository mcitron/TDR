%____________________________________________________________________________||
\section{Event selection}
\label{sec:selection}

\subsection{Event vetoes for leptons, photons, and single isolated tracks\label{sec:vetoes}}

To suppress SM processes with genuine \met from neutrinos, events
containing an isolated electron with $\pt > 20\GeV$ and $|\eta| < 2.5$ or an isolated muon
with $\pt > 10\GeV$ and $|\eta| < 2.5$ are vetoed. To select a pure
multijet topology, events are vetoed in which an isolated
photon with $\pt > 25\GeV$ and $|\eta| < 2.5$ is
found.  Further, to reduce the ``lost leptons'' backgrounds from W~+~jets 
and \ttbar, events containing single isolated tracks with $\pt >
10\GeV$ and $|\eta| < 2.5$, as defined in
Section~\ref{sec:reconstruction}, are vetoed as part of the signal
region selection criteria. In the case of the \mj and \mmj
samples, a further requirement is made such that events are not vetoed
due to the presence of a track from the well identified muons, by
requiring $\Delta R(\textrm{track},\mu) < 0.02$.


%%____________________________________________________________________________||
\subsection{Hadronic region selection}

\subsubsection{Hadronic pre-selection}
Events are required to have significant hadronic activity by requiring
$\scalht > 200\GeV$. Despite the increase in centre of mass energy and pile-up
in Run~2, this threshold is kept at the level of the Run~1 analyses~\cite{run1Analyses}  
to maintain acceptance for SUSY models with compressed spectra. This should be
made possible by the advances made in jet reconstruction, particularly pileup
subtraction~\cite{puppi}.

%Jet selection %fjv
Jets considered in the analysis are required to have an $\ET>40\gev$ and be
within the central tracker acceptance ($|\eta|<3.0$). This threshold is chosen
to be flat across \HT values, unlike the SUS-13-366 analysis. The \ET value is
chosen to be as low as possible to reduce the number of jets falling below
threshold and introducing artificial \mht, while remaining in the realm of
reliable jet energy corrections. In the nominal analysis the lead two jets 
are required to satisfy $\ET>100\gev$ and the lead jet must
have $|\eta|<2.5$. The jets that are accepted are used for the calculation of
the variables \HT and \mht.

Events in which jets with $\ET>40\gev$ are reconstructed with $|\eta|>3.0$ are
vetoed. This reduces the number of events with non genuine \mht caused by jets
just out of acceptance.

\subsubsection{The hadronic signal region\label{sec:had-signal}}
%alphaT cuts
Following the hadronic pre-selection, the multijet background from QCD
is still several orders of magnitude larger than the typical signal
expected from SUSY. Most multijet background sits at $\alphat<0.5$ so can be 
rejected with very high efficiency by requiring an appropriate cut on this
variable (plus the application of two dedicated cleaning filters, described below in
Sec.~\ref{sec:had-signal}). As the mulitjet background is more prevalent at low
\HT values, an \alphat cut that scales with \HT is required, detailed in
Table~\ref{tab:alphat-thresholds}. The \alphat threshold is to keep the
effective \mht cut approximately constant. As high \alphat cuts can have a
significant effect on signal acceptance these thresholds can be tuned to
maintain a high signal to background ratio while remaining QCD free.

% put in new alphaT cuts!! FIXME
\begin{table}[h!]
  \caption{\alphat and (effective) \mht thresholds per \scalht bin.\label{tab:alphat-thresholds}}
  \centering
  \footnotesize
  \begin{tabular}{ lcccccc }
    \hline
    \hline
    \scalht bin  & 200--250   & 250--300   & 300--350  & 350--400  & 400--900  & $>$900       \\
    \hline                                                                     
    \alphat      & 0.65       & 0.60       & 0.55      & 0.53      & 0.52      & 0.505         \\
    "Min \mht"   & $\sim$128  & $\sim$138  & $\sim$125 & $\sim$133 & $\sim$137 & $\gtrsim$126 \\
    \hline
    \hline
  \end{tabular}
\end{table}

In the Run~1 analyses there was a minimum cut of $\alphat>0.55$. This was chosen
as a conservative way to ensure the analysis was QCD free. It was also partially required 
by the \alphat requirement in the HLT paths used to seed the analysis bins. For
Run~2 it is proposed to seed the the bins with $\HT>900\gev$ by a flat
$PF\HT>900\gev$ HLT path, removing the \alphat requirement.

To protect against multiple jets failing the $\Et$ threshold, the
jet-based estimate of the missing transverse energy, \mht, is compared
to the Particle Flow estimate of missing transverse energy, $\pfmet$,
and events with $R_{\rm miss}=\mht/\pfmet > 1.25$ are rejected.

To protect against severe energy losses, events with significant jet
mismeasurements caused by masked regions in the ECAL (which amount to
about 1\% of the ECAL channel count), or by missing instrumentation in
the barrel-endcap gap, are removed with the following procedure. The
jet-based estimate of the missing transverse energy, \mht, is used to
identify jets most likely to have given rise to the \mht as those
whose momentum is closest in $\phi$ to the total $\vec{\mht}$ which
results after removing them from the event.  The azimuthal distance
between this jet and the recomputed \mht is referred to as
$\Delta\phi^*$ in what follows. Events with $\Delta\phi^* < 0.5$ are
rejected if the distance in the ($\eta,\phi$) plane between the
selected jet and the closest masked ECAL region, $\Delta R_{\rm
  ECAL}$, is smaller than 0.3. Similarly, events are rejected if the
jet points within 0.3 in $\eta$ of the ECAL barrel-endcap gap at
$|\eta| = 1.5$. These final selections complete the definition of the
acceptance of the hadronic signal sample.

%\subsubsection{Reduction of \alphat thresholds for high \HT}
% Did this above instead

\subsubsection{Key distributions for the hadronic signal
  region\label{sec:mc-data-comp}}

%%____________________________________________________________________________||
\subsection{Breakdown of SM backgrounds in the hadronic signal
  region\label{sec:bkgd-comp}}

% put yield tables here

%%____________________________________________________________________________||
\subsection{Analysis bins}

Events in the hadronic signal region (and the
three control regions described in Sec.~\ref{sec:controlSelection}) are
categorised according to the number of jets (\njet) reconstructed in
each event and the number of jets identified as originating from
bottom quarks (\nb) in each event. By construction, $\nb \leq \njet$.

Additionally these categories are split up into \HT bins, detailed in 
Table~\ref{tab:htBins}.
%HT bins table

% Introduce 2012 bins?

\subsubsection{Introduction of asymmetric jet bin}

% introduction of asymmetric dijet bin, describe the bin and small motivation

\subsubsection{Extension of \HT bins}

\subsubsection{Fine \njet binning}

%%____________________________________________________________________________||
\subsection{Control region selection}
\label{sec:controlSelection}

\subsubsection{Hadronic control sample}

A disjoint hadronic control sample consisting predominantly of
multijet events is defined by applying the hadronic pre-selection
criteria and inverting the \alphat and/or \mhtmet requirements for a
given \scalht region, which is used primarily in the estimation of any
residual background from QCD multijet events, described in
Sec.~\ref{sec:qcd}.

\subsubsection{The \texorpdfstring{\mj}{muon plus jets} control sample}

\subsubsection{The \texorpdfstring{\mmj}{di-muon plus jets} control sample}

\subsubsection{The \texorpdfstring{\gj}{photon plus jets} control sample}

%%____________________________________________________________________________||
\subsection{Increasing the acceptance of the control samples\label{sec:larger}}

%Add electron control sample
%Gamma+jets - remove alphaT cut (due to artificial increase of photon pT) 

%%____________________________________________________________________________||
