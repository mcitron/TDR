%%____________________________________________________________________________||

%%____________________________________________________________________________||
\RCS$Revision: 273188 $
\RCS$HeadURL: svn+ssh://svn.cern.ch/reps/tdr2/notes/AN-15-004/trunk/AN-15-004.tex $
\RCS$Id: AN-15-004.tex 273188 2015-01-08 15:07:24Z alverson $

%%____________________________________________________________________________||
\newlength\cmsFigWidth
\ifthenelse{\boolean{cms@external}}{\setlength\cmsFigWidth{0.85\columnwidth}}{\setlength\cmsFigWidth{0.4\textwidth}}
\ifthenelse{\boolean{cms@external}}{\providecommand{\cmsLeft}{top\xspace}}{\providecommand{\cmsLeft}{left\xspace}}
\ifthenelse{\boolean{cms@external}}{\providecommand{\cmsRight}{bottom\xspace}}{\providecommand{\cmsRight}{right\xspace}}

%%____________________________________________________________________________||
\cmsNoteHeader{AN-15-004}

%%____________________________________________________________________________||
\title{PHYS14 exercise for supersymmetry searches with the $\alpha_\rm T$ variable}

%%____________________________________________________________________________||
\address[neu]{Northeastern University}
\address[fnal]{Fermilab}
\address[cern]{CERN}
\author[cern]{The CMS Collaboration}

%%____________________________________________________________________________||
\date{\today}

%%____________________________________________________________________________||
\abstract{
   This is an example of a \textit{CMS Note} written in \LaTeX
    using the \emph{cms-tdr} document class and processed using the
    same \texttt{tdr} perl script used in generating the CMS Physics TDRs.
    Instructions for producing CMS Notes and Internal Notes are given.
}

%%____________________________________________________________________________||
\hypersetup{%
pdfauthor={George Alverson, Lucas Taylor, A. Cern Person},%
pdftitle={PHYS14 exercise for supersymmetry searches with the alpha_rm T variable},%
pdfsubject={CMS},%
pdfkeywords={CMS, physics, software, computing}}

%%____________________________________________________________________________||
\maketitle
