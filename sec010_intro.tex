%%____________________________________________________________________________||
\section{Introduction}
\label{sec:intro}


In this note, we summarize the PHYS14 exercise [REF] for the \alphat
analysis, which searches for signatures of physics beyond the standard
model in events with jets and missing transverse momentum (MET). This
analysis uses the kinematic variable \alphat in selecting events with
missing transverse momentum.

With this variable \alphat, CMS has been searching for supersymmetry
(SUSY) in proton-proton collisions data collected during LHC Run 1. With
data at a centre-of-mass energy of 7 TeV collected in 2010 and 2011, the
\alphat analysis has excluded a large parameter space of the constrained
minimal supersymmetric extension of the standard model (CMSSM)
\cite{Khachatryan:2011tk, Chatrchyan:2011zy, Chatrchyan:2012wa} and a
parameter space of simplified models \cite{Chatrchyan:2012wa}. With data
at a centre-of-mass energy of 8 TeV collected and promptly reconstructed
in 2012, the \alphat analysis further excluded a parameter space of
simplified models \cite{Chatrchyan:2013lya}. Additional sets of data
which were collected in 2012 but were reconstructed later during the LHC
Long Shutdown 1 (LS1) are called the ``parked data'', which contain data
for events triggered with lower energy thresholds. The review of the
\alphat analysis of the parked data \cite{CMS_AN_2013-366} is ongoing.
The search strategy in this note is an extension of that in Ref.
\cite{CMS_AN_2013-366}.

After two years of the LS1, in the middle of 2015, LHC will start its
Run 2 and deliver proton-proton collisions to CMS at a higher
centre-of-mass energy of 13 TeV. It is highly motivated to continue the
search for supersymmetry at this higher energy. The \alphat analysis is
particularly suited for this search and has potential for discovery in
early data. The higher energy collision considerably increases the
production cross sections of heavy particles. The reconstruction of
missing transverse momentum in different collider environment or at new
collider energy is typically very challenging and often requires an
extended period of the development. However, the variable \alphat can
effectively select events with missing transverse momentum without
directly using the reconstructed missing transverse momentum.

The production of dark matter at LHC Run 2 has been predicted by many
models of physics beyond the standard model, by both supersymmetric and
non-supersymmetric models. The variable \alphat can be also effective to
select events that potentially contain dark matter produced in the
collisions. In Run 2, we will extend the \alphat analysis to include the
collider dark matter search.

The PHYS14 exercise, which started in October 2014 and will conclude in
February 2015, is a preparation for LHC Run 2 carried out in the CMS
collaboration. This exercise is particularly focused on the preparation
of physics analyses. In this exercise, we analyzed events which were
simulated under conditions anticipated in early LHC Run 2, e.g, the
centre-of-mass energy, the bunch-spacing, the number of the interactions
per bunch crossing. We have derived expected exclusion limits on
parameter spaces of simplified models.

The section 2 is ...
The section 3 is ...
The section 4 is ...
The section 5 is ...
The section 6 is ...
The section 7 is ...
The section 8 is ...
The section 9 is ...
The section 10 is ...
The section 11 is ...



%%____________________________________________________________________________||
