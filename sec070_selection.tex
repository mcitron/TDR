%____________________________________________________________________________||
\section{Event selection}
\label{sec:selection}

\subsection{Event vetoes for leptons, photons, and single isolated tracks\label{sec:vetoes}}

To suppress SM processes with genuine \met from neutrinos, events
containing an isolated electron with $\pt > 20\GeV$ and $|\eta| < 2.5$ or an isolated muon
with $\pt > 10\GeV$ and $|\eta| < 2.5$ are vetoed. To select a pure
multijet topology, events are vetoed in which an isolated
photon with $\pt > 25\GeV$ and $|\eta| < 2.5$ is
found.  Further, to reduce the ``lost leptons'' backgrounds from W~+~jets 
and \ttbar, events containing single isolated tracks with $\pt >
10\GeV$ and $|\eta| < 2.5$, as defined in
Section~\ref{sec:objects}, are vetoed as part of the signal
region selection criteria. In the case of the \mj and \mmj
samples, a further requirement is made such that events are not vetoed
due to the presence of a track from the well identified muons, by
requiring $\Delta R(\textrm{track},\mu) > 0.02$.


%%____________________________________________________________________________||
\subsection{Hadronic region selection}
\label{sec:hadSelection}

\subsubsection{Hadronic pre-selection}
Events are required to have significant hadronic activity by requiring
$\scalht > 200\GeV$. Despite the increase in centre of mass energy and pileup
in Run~2, this threshold is kept at the level of the Run~1 analyses~\cite{Chatrchyan:2013lya}  
to maintain acceptance for SUSY models with compressed spectra. This is possible due to advances 
made in offline object reconstruction, particularly pileup
subtraction%~\cite{puppi}.

Jets considered in the analysis are required to have transverse momenta of $\PT>40\gev$ and be
within the acceptance of the central tracker ($|\eta|<3.0$). This threshold is chosen
to be flat across \HT values, unlike the SUS-14-006 analysis~\cite{CMS_AN_2013-366}. The \PT value is
chosen to be as low as possible to reduce the number of jets falling below
threshold and introducing artificial \mht, while remaining in the realm of
reliable jet energy corrections. In the nominal analysis the two highest energy jets
are required to satisfy $\PT>100\gev$ and the leading jet must be within $|\eta|<2.5$. 
The jets that are selected are used for the calculation of the variables \HT and \mht.

Events in which jets with $\PT>40\gev$ are reconstructed within $|\eta|>3.0$ are
rejected. This reduces the number of events with non genuine \mht caused by jets
just out of acceptance.

\subsubsection{The hadronic signal region\label{sec:had-signal}}

Following the hadronic pre-selection, the multijet background from QCD
is still several orders of magnitude larger than the typical signal
expected from SUSY. Most multijet background is located at values less than $\alphat<0.5$ and therefore can
be rejected with very high efficiency by requiring an appropriate cut on \alphat (plus the application of two dedicated cleaning filters, described below in
Sec.~\ref{sec:had-signal}). As the mulitjet background is more prevalent at low
\HT values, the \alphat requirement scales with \HT as detailed in
Table~\ref{tab:alphat-thresholds}. The minimum \mht value that the \alphat cut
corresponds to is calculated by assuming $\dht=0$ and inverting the formula for
\alphat. 

\begin{table}[h!]
  \caption{\alphat and (effective) \mht thresholds per \scalht bin.\label{tab:alphat-thresholds}}
  \centering
  \footnotesize
  \begin{tabular}{ lcccccc }
    \hline
    \hline
    \scalht bin  & 200--250   & 250--300   & 300--350  & 350--400  & 400--900  & $>$900       \\
    \hline                                                                     
    \alphat      & 0.65       & 0.60       & 0.55      & 0.53      & 0.52      & 0.505         \\
    "Min \mht"   & $\sim$128  & $\sim$138  & $\sim$125 & $\sim$133 & $\sim$137 & $\sim$126 \\
    \hline
    \hline
  \end{tabular}
\end{table}

In the Run~1 analyses a general minimum cut of $\alphat>0.55$ was used. This was chosen
as a conservative cut to remove QCD and also part of the \alphat requirement in the HLT paths used to seed the analysis bins. For
Run~2 it is proposed to seed the the bins with $\HT>900\gev$ by a flat
$PF\HT>900\gev$ HLT path with no \alphat requirement, see Sec.~\ref{triggers}.

Subsequently, additional cleaning cuts are applied. To protect against 
multiple jets failing the $\Et$ threshold, the
jet-based estimate of the missing transverse energy, \mht, is compared
to the Particle Flow estimate of missing transverse energy, $\pfmet$,
and events with $R_{\rm miss}=\mht/\pfmet > 1.25$ are rejected.

To protect against severe energy losses, events with significant jet
mismeasurements caused by masked regions in the ECAL (which amount to
about 1\% of the ECAL channel count), or by missing instrumentation in
the barrel-endcap gap, are removed with the following procedure. The
jet-based estimate of the missing transverse energy, \mht, is used to
identify jets most likely to have given rise to the \mht as those
whose momentum is closest in $\phi$ to the total $\vec{\mht}$ which
results after removing them from the event.  The azimuthal distance
between this jet and the recomputed \mht is referred to as
$\Delta\phi^*$ in what follows. Events with $\Delta\phi^* < 0.5$ are
rejected if the distance in the ($\eta,\phi$) plane between the
selected jet and the closest masked ECAL region, $\Delta R_{\rm
  ECAL}$, is smaller than 0.3. Similarly, events are rejected if the
jet points within 0.3 in $\eta$ of the ECAL barrel-endcap gap at
$|\eta| = 1.5$. These final selections complete the definition of the
acceptance of the hadronic signal sample.

%%____________________________________________________________________________||
\subsection{Analysis bins}

Events in the hadronic signal region (and the
three control regions described in Sec.~\ref{sec:controlSelection}) are
categorised according to the number of jets (\njet) reconstructed in
each event and the number of jets identified as originating from
bottom quarks (\nb) in each event. By construction, $\nb \leq \njet$. The \njet
bins are likely to be at the level of one \njet per category.

Additionally these categories are binned in \HT. These bins begin at
$\HT=200\gev$ and extend in $50\gev$ steps up to $\HT=400\gev$. Above this, the
bins are $100\gev$ wide extending up to the highest \HT reachable. 
These bin widths have been rationalised with respect to the SUS-14-006 analysis. 
As there are sufficient events in the control
samples (detailed in Sec.~\ref{sec:controlSelection}) at low \HT, $50\gev$ bins
are an appropriate size. In SUS-14-006 the low \HT bins were seeded by different
trigger paths, and their lower edge was chosen to sit in the trigger turn-on. For
Run~2 this strategy has been inverted, with the selection of the offline \HT
bins first. The trigger threshold is then chosen to be sufficiently below that
of it's relevant offline bin so as to be efficient. More details in
Sec~\ref{sec:triggers}. An overview of the preliminary binning is given in Tab.~\ref{tab:alphat-binning}.
%detailed in Table~\ref{tab:htBins}.

\begin{table}[h!]
  \centering
  \footnotesize
  \begin{tabular}{ lcccccc }
    \hline
    \hline
    \scalht bin  & \multicolumn{5}{c}{$p_\textrm{T}(j^{1,2})>100$~GeV, $\HT > 200$~GeV, $\alpha_{\rm T}$}\\
    \scalht bin  & 200--250      & 250--300      & 300--350      & 350--400     & 400--900       & $>$900       \\
    $n_\textrm{jet}$        & $2,3,4,\ge 5$ & $2,3,4,\ge 5$ & $2,3,4,\ge 5$ & $2,3,4,\ge 5$ & $2,3,4,\ge 5$ & $2,3,4,\ge 5$\\
    $n_b$        & 0--min($n_\textrm{jet},\ge 4$) & 0--min($n_\textrm{jet},\ge 4$) & 0--min($n_\textrm{jet},\ge 4$) & 0--min($n_\textrm{jet},\ge 4$) & 0--min($n_\textrm{jet},\ge 4$) & 0--min($n_\textrm{jet},\ge 4$)\\
    \hline
    \hline
  \end{tabular}
\caption{Preliminary signal region binning \label{tab:alphat-binning}}
\end{table}

Also a new orthogonal signal region has been added. In order to increase acceptance and sensitivity for ISR and mono-jet type events is lowered to 40~GeV with an upper bound of 100~GeV, see Sec~\ref{sec:physics}.


\begin{table}[h!]
  \centering
  \footnotesize
  \begin{tabular}{ lcccccc }
    \hline
    \hline
    \scalht bin  & \multicolumn{5}{c}{$p_\textrm{T}(j^{1})>100$~GeV, $40\ge p_\textrm{T}(j^{2})\le100$~GeV, $\HT > 200$~GeV, $\alpha_{\rm T}$}\\
    \scalht bin  & 200--250      & 250--300      & 300--350      & 350--400     & 400--900       & $>$900       \\
    $n_\textrm{jet}$        & $2,3,4,\ge 5$ & $2,3,4,\ge 5$ & $2,3,4,\ge 5$ & $2,3,4,\ge 5$ & $2,3,4,\ge 5$ & $2,3,4,\ge 5$\\
    $n_b$        & 0--min($n_\textrm{jet},\ge 4$) & 0--min($n_\textrm{jet},\ge 4$) & 0--min($n_\textrm{jet},\ge 4$) & 0--min($n_\textrm{jet},\ge 4$) & 0--min($n_\textrm{jet},\ge 4$) & 0--min($n_\textrm{jet},\ge 4$)\\
    \hline
    \hline
  \end{tabular}
\caption{Preliminary asymmetric signal region binning \label{tab:alphat-asym-binning}}
\end{table}

%%____________________________________________________________________________||

\subsection{Definition of the control samples}
\label{sec:controlSelection}

\subsubsection{Hadronic control sample}

A disjoint hadronic control sample consisting predominantly of
multijet events is defined by applying the hadronic pre-selection
criteria and inverting the \alphat and/or \mhtmet requirements for a
given \scalht region, which is used primarily in the estimation of any
residual background from QCD multijet events, described in
Sec.~\ref{sec:qcd}.

\subsubsection{The \texorpdfstring{\mj}{muon plus jets} control sample}
\label{subsec:mucontrolSelection}

Events from the \wj and \ttbar processes are found in the hadronic
signal sample due to unidentified leptons (either out of acceptance or
not reconstructed) and hadronic tau decays originating from
high-p$_{T}$ W bosons. An estimate of these background processes is
obtained through the use of a \mj sample. The selection criteria for
this sample are chosen to identify W bosons decaying to a muon and a
neutrino in the phase-space of the signal. The muon is not considered
in the calculation of event-level variables such as \scalht, \mht and
\alphat. All cuts on such jet-based quantities are consistent with
those applied in the hadronic search region and the same \njet, \nb,
and \scalht binning is used. The only exception is that no \alphat
requirement is made, as motivated by the discussion in
Sec.~\ref{sec:larger}. In order to select events containing W bosons,
exactly one tight isolated muon within an acceptance of \PT $>$ 30
\gev and $|\eta| <$ 2.1 is required (due to the trigger), and the
transverse mass of the W candidate must satisfy $30 < \mt(\mu,\pfmet)
< 125\gev$ (to suppress QCD multijet and potential signal
events). Events are vetoed if $\Delta R(\mu,\textrm{jet}_i) < 0.5$
running over all jets $i$. The single isolated track veto, described
in Sections~\ref{sec:objects} and~\ref{sec:vetoes}, is also
applied, which considers all single isolated tracks in the event
except that associated with the identified, isolated muon. Finally,
the cleaning cut $\mht/\pfmet$ is also applied, as done in the signal
region, where the \pfmet is adjusted to account for the transverse
momentum of the identified, isolated muon.

\subsubsection{The \texorpdfstring{\mmj}{di-muon plus jets} control sample}

The \znunu\ + jets process forms an irreducible background and can be
estimated using the \zmumu + jets process, which has similar kinematic
properties but a different acceptance and a smaller branching ratio. A
background estimate is obtained through the use of a \mmj sample. Most
of the selection criteria are identical to those for the \mj sample,
but the few that differ are tuned to identify Z bosons decaying to two
muons in the kinematic phase space of the signal region. The muons are
not considered in the calculation of event-level variables such as
\scalht, \mht and \alphat. All cuts on such jet-based quantities are
consistent with those applied in the hadronic search region and the
same \njet, \nb, and \scalht binning is used. The only exception is
that no \alphat requirement is made, as motivated by the discussion in
Sec.~\ref{sec:larger}. In order to select an event sample containing Z
bosons, exactly two tight isolated muons within an acceptance of $\Pt
> 30\gev$ and $|\eta| < 2.1$ are required (due to the trigger). The
invariant mass of the two muons must satisfy $m_{Z} - 25 <
M_{\mu_1\mu_2} < m_{Z} + 25$. Events are vetoed if $\Delta
R(\mu_{i},\textrm{jet}_j) < 0.5$ is satisfied, running over all muons
$i$ and all jets $j$. The single isolated track veto, described in
Sections~\ref{sec:objects} and~\ref{sec:vetoes}, is also
applied, considering all single isolated tracks in the event except
those associated with the two identified, isolated muons. Finally, the
cleaning cut $\mht/\pfmet$ is also applied, as done in the signal
region, where the \pfmet is adjusted to account for the transverse
momenta of the two identified, isolated muons. The \mmj sample can be
used to make predictions in all the \scalht bins, providing coverage
at low \scalht where the \gj sample cannot.

\subsubsection{The \texorpdfstring{\gj}{photon plus jets} control sample}
\label{subsec:photoncontrolSelection}

The \znunu\ + jets process can also be estimated using the \gj
process, which has a larger cross section and kinematic properties
similar to those of \znunu\ events when the photon is
ignored~\cite{PAS-SUS-08-002,Bern:2011pa}. The \gj sample is defined
by requiring exactly one photon satisfying tight isolation criteria
and within an acceptance of $\pt > 175\gev$ and $|\eta| <
1.45$ (the anticipated limitation from the trigger). Furthermore, events are vetoed if $\Delta
R(\gamma,\textrm{jet}_j) < 1.0$ is satisfied, running over all jets
$j$. As for the muon-based samples, the photon is not considered in
the calculation of event-level variables such as \scalht, \mht, \met and 
\alphat. All cuts on jet-based quantities are consistent with those
applied in the hadronic search region, and the same \HT binning is
used. 
% Given that the photon is ignored, the \gj sample can only be
% used for the region $\scalht > 375\gev$ due to the photon acceptance
% of $\pt > 165\gev$ (enforced by the trigger) and the requirement
% $\alphat > 0.55$.

%%____________________________________________________________________________||
\subsection{Increasing the acceptance of the control samples\label{sec:larger}}

As described in Sec.~\ref{sec:controlSelection} above, the
selection criteria of the three control samples are defined such that
the background composition and event kinematics of the three control
samples mirror as closely as possible those for the signal
region. This is done in order to minimise the reliance on the
simulation to model correctly the backgrounds and event kinematics in
the control and signal samples.

However, in the case of the \mj and \mmj samples, no requirement is
made on \alphat in the selection criteria of the samples. This is made
possible by the remaining kinematic selection criteria, which are
sufficiently selective to ensure that the muon samples remain rich in
events from the \wj, \ttbar and \zmumu processes with negligible
contamination from QCD multijet events. Thus, the acceptance of the two 
muon control samples can be significantly increased, which simultaneously 
improves their predictive power and further reduces the effect of any potential
signal contamination. 

In the Run~1 analyses the \gj control sample had a tight requirement on the
photon. After recent developments presented in EGM-14-001~\cite{CMS-EGM-14-001}
, it
should be possible to reduce the photon requirement from tight to medium, while
still having a significant background rejection. This will be investigated and
should help to increase the number of events available, giving a better control
of the analysis background prediction detailed in Sec.~\ref{sec:background}.

To further help with the number of events available in control samples, it is
proposed to introduce electron~+~jets control samples. These would work
analagously to the \mj and \mmj samples, with electrons in place of muons. With
a tight enough requirement on the electron ID it should be possible to have a
similar control over backgrounds in these samples to those in the muon control
samples. As electrons in leptonic decay processes behave kinematically similarly
to muons, it should be possible to effectively increase the statistics available
from the control samples. 
%%____________________________________________________________________________||
